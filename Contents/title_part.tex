\begin{SCn}
	
\scseparatedfragment{Оглавление Стандарта OSTIS}
\begin{SCn}
    \scnsectionheader{Оглавление Стандарта OSTIS}
    \scnidtf{Оглавление текущей версии Стандарта OSTIS}
    \scntext{пояснение}{Иерархический перечень разделов, входящих в состав
        \textit{Стандарта OSTIS}, с дополнительной спецификацией некоторых разделов,
        указывающей альтернативные названия разделов, а также их авторов и редакторов}
    \begin{scnindent}
   	\scntext{примечание}{Существенно подчеркнуть, что иерархия разделов \textit{Стандарта
   			OSTIS} как и \textit{разделов} любой другой \textit{базы знаний} не означает
   		то, что \textit{разделы} более низкого уровня иерархии входят в состав
   		(являются частями) соответствующих разделов более высокого уровня. Связь между
   		\textit{разделами} разных уровней иерархии означает то, что \textit{раздел}
   		более низкого уровня иерархии является \textit{дочерним} разделом по отношению
   		к соответствующему \textit{разделу} более высокого уровня, т.е.
   		\textit{разделом}, который наследует свойства указанного \textit{раздела} более
   		высокого уровня.}
   	\end{scnindent}
    \scntext{примечание}{Описание логико-семантических связей каждого раздела
        \textit{Стандарта OSTIS} с другими разделами \textit{Стандарта OSTIS}
        приводится в рамках \textit{титульной спецификации} каждого \textit{раздела}.}
    \scntext{примечание}{Названия тех \textit{разделов}, которые планируется написать в
        последующих изданиях \textit{Стандарта OSTIS} или \textit{разделов}, которые
        \uline{в данном} издании \textit{Стандарта OSTIS} не печатаются, поскольку
        содержание их не изменилось по сравнению с предыдущим \uline{явно указываемым}
        изданием \textit{Стандарта OSTIS}, выделяются также \uline{жирным курсивом}, но
        для них страницы в рамках данного издания \textit{Стандарта OSTIS} не
        указываются}
    \begin{scnhierstruct}
        \normalsize
        \begingroup
        \let\clearpage\relax
        \tableofcontents
        \endgroup
    \end{scnhierstruct} \scnsourcecommentinline{Завершили \textit{Оглавление
            Стандарта OSTIS}}
\end{SCn}

\newpage

\scseparatedfragment{Титульная спецификация Стандарта OSTIS}
\begin{SCn}
    \scnsectionheader{Титульная спецификация Стандарта OSTIS}

    \begin{scnsubstruct}
        \scnrelto{титульная спецификация}{Стандарт OSTIS}
        \scnrelfrom{оглавление}{Оглавление Стандарта OSTIS}
        \scnrelfrom{общая структура}{Общая Структура Стандарта OSTIS}
        \scnrelfrom{система ключевых знаков}{Система ключевых знаков Стандарта OSTIS}
        \scnrelfrom{редакционная коллегия}{Редакционная коллегия Стандарта OSTIS}
        \scnrelfrom{авторский коллектив}{Авторский коллектив Стандарта OSTIS}
        \scnrelfrom{направления развития}{Направления развития Стандарта OSTIS}
        \scnrelfrom{правила построения}{Правила построения Стандарта OSTIS}
        \begin{scnindent}
            \scnidtf{правила построения*(Стандарт OSTIS)}
        \end{scnindent}
        \scniselement{sc-выражение}
        \scnrelfrom{правила организации развития}{Правила организации развития
            Стандарта OSTIS}
        \begin{scnrelfromset}{декомпозиция}
            \scnitem{Правила организации развития исходного текста Стандарта OSTIS}
            \scnitem{Правила организации развития Стандарта OSTIS на уровне его
                внутреннего представления в памяти Метасистемы IMS.ostis}
        \end{scnrelfromset}
        \scnsourcecommentinline{Вычитать то, что дальше}
        \scnnote{В титульную спецификацию \textit{Стандарта OSTIS} должны быть
            включены ссылки на все разделы и фрагменты этих разделов,
            где описываются правила построения и оформления всех видов информационных
            конструкций, входящих в состав \textit{Стандарта OSTIS}
            (внешних идентификаторов знаков, входящих в состав \textit{Стандарта OSTIS},
            спецификаций различного вида cущностей, описываемых в \textit{Стандарте OSTIS})

            В \textit{базе знаний ostis-систем} задаются правила унифицированного построения
            (представления, оформления) следующих видов \textit{информационных
                конструкций}:
            \begin{scnitemize}
                \item \textit{sc-идентификаторов} --- внешних идентификаторов
                    \textit{sc-элементов} следующих классов:
                    \begin{scnitemizeii}
                        \item \textit{sc-элементов} (имеются в виду общие правила идентификации
                                    любых sc-элементов) --- смотрите в разделе\scnqqi{\nameref{intro_idtf}}
                        \item \textit{sc-переменных, sc-констант}
                        \item знаков материальных сущностей
                            \begin{scnitemizeiii}
                                \item знаков персон
                                \item знаков библиографических источников
                            \end{scnitemizeiii}
                        \item знаков множеств
                            \begin{scnitemizeiii}
                                \item классов, понятий
                                \begin{itemize}
                                    \item отношений
                                    \item параметров
                                    \item структур
                                    \item знаний
                                \end{itemize}
                            \end{scnitemizeiii}
                        \item знаков файлов ostis-систем
                        \item знаков sc-знаний баз знаний
                    \end{scnitemizeii}
                \item \textit{sc-конструкций}
                \item \textit{sc.g-конструкций}
                \item \textit{sc.s-конструкций}
                \item \textit{sc.n-конструкций}
                \item базовых правил \textit{sc-спецификаций:}
                    \begin{scnitemizeii}
                        \item понятий
                        \item разделов баз знаний (титульные спецификации разделов)
                        \item файлов ostis-систем
                        \item библ. источников
                        \item предметных областей
                    \end{scnitemizeii}
                \item специализированная \textit{sc-спецификация}
                    \begin{scnitemizeii}
                        \item информационная конструкция
                            \begin{scnitemizeiii}
                                \item оглавление
                                \item система ключевых знаков
                            \end{scnitemizeiii}
                        \item понятий
                            \begin{scnitemizeiii}
                                \item пояснение
                                \item определения
                                \item теоретико-множественная окрестность
                                \item семейство утверждений
                            \end{scnitemizeiii}
                        \item сегментов баз знаний (титульная спецификация)
                        \item семейство разделов баз знаний
                    \end{scnitemizeii}
            \end{scnitemize}}

        \scnheader{Стандарт OSTIS-2021}
        \scnidtf{Издание Документации Технологии OSTIS-2021}
        \scnidtf{Первое издание (публикация) Внешнего представления Документации
            Технологии OSTIS в виде книги}
        \scniselement{публикация}
        \begin{scnindent}
            \scnidtf{библиографический источник}
        \end{scnindent}
        \scniselement{официальная версия Стандарта OSTIS}
        \scniselement{бумажное издание}
        \scniselement{научное издание}
        \scnrelfrom{рекомендация издания}{Совет БГУИР}
        \begin{scnrelfromset}{рецензенты}
            \scnitem{Курбацкий А.Н.}
            \scnitem{Дудкин А.А.}
        \end{scnrelfromset}
        \scnrelfrom{издательство}{Бестпринт}
        \scniselement{\scnnonamednode}
        \scniselement{УДК}
        \scniselement{параметр}
        \scnrelfrom{Индекс УДК}{004.8}
        \scnidtftext{ISBN}{978-985-7267-13-2}

        \scnheader{Стандарт OSTIS-2022}
        \scnidtf{Издание Документации Технологии OSTIS-2022}
        \scnidtf{Второе издание (публикация) Внешнего представления Документации Технологии OSTIS в виде книги}
        \scniselement{публикация}
        \scniselement{официальная версия Стандарта OSTIS}
        \scniselement{бумажное издание}
        \scniselement{научное издание}
        \bigskip
    \end{scnsubstruct}
    \scnsourcecommentinline{Завершили Титульную спецификацию Стандарта OSTIS}
\end{SCn}

\newpage

\scseparatedfragment{Титульная спецификация второго издания Стандарта OSTIS}
\begin{SCn}
    \scnsectionheader{Титульная спецификация второго издания Стандарта OSTIS}

    \begin{scnsubstruct}
        \scnidtf{Титульная спецификация Стандарта OSTIS-2022}
        \scnrelto{титульная спецификация}{Стандарт OSTIS-2022}
        \begin{scnindent}
            \scnidtf{Второе издание стандарта OSTIS}
        \end{scnindent}
        \scnrelfrom{предисловие}{Предисловие ко второму изданию Стандарта OSTIS}
    \end{scnsubstruct}
\end{SCn}


\newpage

\begin{SCn}
\scnsectionheader{Общая Структура Стандарта OSTIS}
\begin{scnstruct}
	\scnheader{Стандарт OSTIS}
	\scntext{общая структура}{\textit{Основной текст Стандарта OSTIS} состоит из следующих частей:
		\begin{scnitemize}
			\item Анализ текущего состояния \textit{технологий Искусственного интеллекта} и постановка задачи на создание комплекса совместимых \textit{технологий Искусственного интеллекта}, ориентированного на создание и эксплуатацию \textit{интеллектуальных компьютерных систем нового поколения}.
			\\Данная часть \textit{Стандарта OSTIS} начинается с \textit{раздела} \scnqqi{\textbf{\textit{Предметная область и онтология кибернетических систем}}} и заканчивается \textit{разделом} \scnqqi{\textit{\textbf{Предметная область и онтология логико-семантических моделей компьютерных систем, основанных на смысловом представлении информации}}}.
			\item Документация предлагаемой комплексной технологии создания и эксплуатации \textit{интеллектуальных компьютерных систем нового поколения}, которая названа нами \textit{Технологией OSTIS}. Эта часть \textit{Стандарта OSTIS} начинается с \textit{раздела} \scnqqi{\textit{\textbf{Предметная область и онтология предлагаемой комплексной технологии создания и эксплуатации интеллектуальных компьютерных систем нового поколения}}}, заканчивается
			\textit{разделом} \scnqqi{\textit{\textbf{Предметная область и онтология встроенных ostis-систем поддержки эксплуатации соответствующих ostis-систем конечными пользователями}}} и включает в себя:
			\begin{scnitemizeii}
				\item Описание формальных структурно-функциональных логико-семантических моделей предполагаемых \textit{интеллектуальных компьютерных систем нового поколения}(такие системы названы нами \textit{ostis-системами}). Сюда входит:
				\begin{scnitemizeiii}
					\item описание моделей \textit{знаний} и \textit{баз знаний}, а также методов и средств их проектирования;
					\item описание логических и продукционных моделей обработки \textit{знаний}, а также методов и средств их проектирования;
					\item описание \scnqq{нейросетевых} моделей обработки \textit{знаний}, а также методов и средств их проектирования;
					\item описание моделей \textit{решателей задач};
					\item описание моделей \textit{интерфейсов ostis-систем};
					\item описание онтологических моделей \textit{интерфейсов интеллектуальных компьютерных систем}, а также методов и средств их проектирования, включая описание онтологических моделей естественно-языковых \textit{интерфейсов интеллектуальных компьютерных систем}, а также методов и средств их проектирования.
				\end{scnitemizeiii}
				\item Описание методов:
				\begin{scnitemizeiii}
					\item методов проектирования \textit{баз знаний ostis-систем};
					\item методов проектирования \textit{решателей задач ostis-систем};
					\item методов проектирования \textit{интерфейсов ostis-систем};
					\item методов производства (реализации) \textit{ostis-систем};
					\item методов реинжиниринга \textit{ostis-систем};
					\item методов использования \textit{ostis-систем} конечными пользователями.
				\end{scnitemizeiii}
				\item Описание средств:
				\begin{scnitemizeiii}
					\item средств поддержки проектирования \textit{баз знаний ostis-систем};
					\item средств поддержки проектирования \textit{решателей задач ostis-систем};
					\item средств поддержки проектирования \textit{интерфейсов ostis-систем};
					\item средств производства \textit{ostis-систем} --- программных средств интеллектуализации логико-семантических моделей \textit{ostis-систем} и специально предназначенных для этого ассоциативных семантических компьютеров;
					\item средств поддержки реинжиниринга \textit{ostis-систем} в ходе их эксплуатации;
					\item средств поддержки использования \textit{ostis-систем} конечными пользователями.
				\end{scnitemizeiii}
				\item Описание реализации системы управления \textit{базами знаний ostis-систем} на основе системы управления графовыми базами данных.
				\item Описание аппаратной реализации графодинамической памяти, а также средств обработки знаний в этой памяти.
			\end{scnitemizeii}
			\item Описание продуктов, создаваемых с помощью \textit{Технологии OSTIS}, основным из которых является глобальная \textit{Экосистема OSTIS} --- Экосистема семантически совместимых и активно взаимодействующих \textit{ostis-систем}.
			\\Эта часть \textit{Стандарта OSTIS} представлена \textit{разделом} \scnqqi{\textit{\textbf{Предметная область и онтология Экосистемы OSTIS}}}
			\item \textit{\textbf{Библиография Стандарта OSTIS}}
		\end{scnitemize}
	}
\end{scnstruct}
\end{SCn}

\newpage

\begin{SCn}
    \scnsectionheader{Система ключевых знаков Cтандарта OSTIS}
    \begin{scnsubstruct}
        \scntext{пояснение}{\textit{Система ключевых знаков Стандарта OSTIS} должна
            стать целостным дополнением  к Оглавлению Стандарта OSTIS
            \begin{scnitemize}
                \item иерархия и последовательность ключевых знаков  должны четко
                соответствовать иерархии и последовательности разделов стандарта;
                \item система ключевых знаков Стандарта OSTIS, как и его Оглавление, должна
                восприниматься (читаться) как целостный понятный текст
            \end{scnitemize}
        }
        \scnheader{Стандарт OSTIS}
        \begin{scnrelfromvector}{ключевые знаки}
            \scnitem{база знаний ostis-системы}
                \begin{scnindent}
                    \scnidtf{база знаний, представленная в SC-коде}
                    \scnidtf{sc-модель базы знаний}
                \end{scnindent}
            \scnitem{технология проектирования баз знаний ostis-систем}
            \scnitem{логическая sc-модель обработки знаний}
            \scnitem{технология проектирования логических sc-моделей обработки знаний}
            \scnitem{продукционная sc-модель обработки знаний}
            \scnitem{технология проектирования продукционных sc-моделей обработки знаний}
            \scnitem{sc-модель искусственной нейронной сети}
            \scnitem{технология проектирования sc-моделей искусственных нейронных сетей}
            \scnitem{sc-модель интерфейса ostis-системы}
            \scnitem{технология проектирования  sc-моделей интерфейсов ostis-систем}
            \scnitem{sc-модель интерфейса ostis-системы}
                \begin{scnindent}
                    \scnidtf{онтологическая модель интерфейса, построенная на основе SC-кода}
                \end{scnindent}
            \scnitem{технология проектирования sc-моделей интерфейсов ostis-систем}
            \scnitem{программная платформа реализации ostis-систем, построенная на основе
                системы управления графовыми базами данных}
                \begin{scnindent}
                    \scnidtf{программная система интерпретации логико-семантических моделей
                        ostis-систем, построенная на основе графовой СУБД}
                    \scnidtf{система управления базами знаний (СУБЗ) ostis-систем, построенная на
                        основе графовых СУБД}
                \end{scnindent}
            \scnitem{ассоциативный семантический компьютер для ostis-систем}
                \begin{scnindent}
                    \scnidtf{компьютер с ассоциативной графодинамической (структурно
                        реконфигурируемой) памятью, ориентированный на реализацию ostis-систем}
                    \scnidtf{компьютер с ассоциативной графодинамической памятью, обеспечивающий
                        интерпретацию логико-семантических моделей ostis-систем}
                    \scnidtf{аппаратная платформа реализации ostis-систем}
                \end{scnindent}
            \scnitem{Проект OSTIS}
            \scnitem{Стандарт OSTIS}
            \scnitem{Экосистема OSTIS}
                \begin{scnindent}
                    \scnidtf{Экосистема ostis-систем и их пользователей}
                    \scnidtf{Вариант построения smart-общества (общества 5.0) на основе
                        ostis-систем}
                \end{scnindent}
            \scnitem{агентно-ориентированная модель обработки информации}
                \begin{scnindent}
                    \scnidtf{многоагентная модель обработки информации}
                    \scnidtf{\textit{модель обработки информации}, рассматривающая \textit{процесс
                        обработки информации} как \textit{деятельность}, выполняемую некоторым
                        \textit{коллективом} самостоятельных \textit{информационных агентов} (агентов
                        обработки информации)}
                \end{scnindent}
        \end{scnrelfromvector}
        \scnrelfrom{ключевой объект спецификации}{Технология OSTIS}
        \scnrelfrom{основные создаваемые продукты}{ostis-система}
            \begin{scnindent}
                \scnidtf{множество всевозможных ostis-систем}
                \scnidtf{компьютерная система, построенная по Технологии OSTIS}
            \end{scnindent}
        \begin{scnrelfromvector}{ключевые понятия, соответствующие принципам, лежащим в
                основе}
            \scnitem{cмысловое представление информации}
            \scnitem{агентно-ориентированная обработка информации}
            \scnitem{интерфейс компьютерной системы}
                \begin{scnindent}
                    \scnidtf{интерфейс компьютерной системы, построенный на основе онтологий}
                    \scnidtf{ontology based interface}
                \end{scnindent}
            \scnitem{мультимодальность}
            \scnitem{конвергенция}
            \scnitem{семантическая совместимость}
            \scnitem{унификация}
            \scnitem{мультимодальная база знаний}
            \scnitem{универсальный язык смыслового представления знаний}
            \scnitem{мультимодальный решатель задач}
            \scnitem{мультимодальный интерфейс компьютерной системы}
            \scnitem{гибридная интеллектуальная компьютерная система}
                \begin{scnindent}
                    \scnidtf{мультимодальная интеллектуальная компьютерная система}
                \end{scnindent}    
            \scnitem{мультимодальный (гибридный) характер и.к.с. в целом}
            \scnitem{мультимодальный характер баз знаний и.к.с.}
            \scnitem{мультимодальный решатель задач и.к.с.}
            \scnitem{мультимодальный (мультиязычный) интерфейс и.к.с.}
            \scnitem{конвергенция и.к.с.(знаний, моделей, решателей задач, моделей
                взаимодействий с внешней средой, моделей общения с внешним субъектом)}
            \scnitem{семантическая совместимость и.к.с.(знаний, моделей, решателей задач,
                моделей взаимодействий с внешней средой, моделей общения с внешним субъектом)}
            \scnitem{онтологическая модель}
            \scnitem{онтологическая логико-семантическая модель и.к.с.}
            \scnitem{онтологическая модель базы знаний и.к.с.}
            \scnitem{онтологическая модель решатель задач и.к.с.}
            \scnitem{онтологическая модель интерфейса  и.к.с.}
            \scnitem{неатомарный раздел}
            \scnitem{атомарный раздел}
            \scnitem{ключевой знак*}
                \begin{scnindent}
                    \scnidtf{ключевая сущность*}
                \end{scnindent}
            \scnitem{SC-код}
            \scnitem{SCg-код}
            \scnitem{SCs-код}
            \scnitem{SCn-код}
            \scnitem{декомпозиция*}
            \scnitem{конкатенация*}
            \scnitem{предметная область}
                \begin{scnindent}
                    \scnidtf{sc-модель предметной области}
                    \scnidtf{sc-текст, являющийся представлением предметной области}
                \end{scnindent}
            \scnitem{максимальный класс объекта исследования\scnrolesign}
            \scnitem{немаксимальный класс объекта исследования\scnrolesign}
                \begin{scnindent}
                    \scnidtf{подкласс максимального объекта исследования\scnrolesign}
                \end{scnindent}
            \scnitem{исследуемое отношение\scnrolesign}
            \scnitem{исследуемый параметр\scnrolesign}
            \scnitem{онтология}
            \scnitem{алфавит*(языка)}
            \scnitem{сформированное множество}
                \begin{scnindent}
                    \scnidtf{конечное множество, все элементы которого представлены
                    соответствующими sc-элементами}
                \end{scnindent}
            \scnitem{бинарное отношение}
            \scnitem{ориентированное отношение}
            \scnitem{первый домен*}
            \scnitem{второй домен*}
            \scnitem{пояснение*}
            \scnitem{семантическая эквивалентность*}
            \scnitem{следствие*}
            \scnitem{примечание*}
            \scnitem{определение*}
        \end{scnrelfromvector}
    \end{scnsubstruct}
\end{SCn}

\newpage

\begin{SCn}
\scnsectionheader{Редакционная коллегия Cтандарта OSTIS}
\begin{scnstruct}
    \scnheader{Состав редакционной коллегии Стандарта OSTIS}
    \begin{scneqtoset}
        \scnitem{Голенков В.В.}
        \scnitem{Головко В.А.}
        \scnitem{Гулякина Н.А.}
        \scnitem{Краснопрошин В.В.}
        \scnitem{Курбацкий А.Н.}
        \scnitem{Гордей А.Н.}
        \scnitem{Шункевич Д.В.}
        \scnitem{Азаров И.С.}
        \scnitem{Захарьев В.А.}
        \scnitem{Родченко В.Г.}
        \scnitem{Голубева О.В.}
        \scnitem{Кобринский Б.А.}
        \scnitem{Борисов В.В.}
        \scnitem{Аверкин А.Н.}
        \scnitem{Кузнецов О.П.}
        \scnitem{Козлова Е.И.}
        \scnitem{Гернявский А.Ф.}
        \scnitem{Таранчук В.Б.}
        \scnitem{Ростовцев В.Н.}
        \scnitem{Витязь С.П.}
    \end{scneqtoset}

    \scnidtf{Редколлегия Стандарта OSTIS}

    \begin{scnrelfromset}{направления и принципы организации деятельности}
        \scnfileitem{Обеспечение целостности и повышения качества постоянно развиваемой
            (совершенствуемой) Технологии OSTIS, а также достаточно точное описание
            (документирование) каждой текущей версии этой технологии.}
        \scnfileitem{Обеспечение чёткого контроля совместимости версий Технологии OSTIS
            в целом, а также версий различных компонентов этой технологии.}
        \scnfileitem{Постоянное уточнение степени важности различных направлений
            развития Технологии OSTIS для каждого текущего момента.}
        \scnfileitem{Формирование и постоянное уточнение плана тактического и
            стратегического развития самой \textit{Технологии OSTIS}, а также полной
            документации этой Технологии в виде \textit{Стандарта OSTIS}. Подчеркнем при
            этом, что указанная документация является неотъемлемой частью
            \textit{Технологии OSTIS}.}
    \end{scnrelfromset}

    \begin{scnrelfromset}{направление деятельности}

        \scnfileitem{Принципы организации деятельности \textit{Редколлегии Стандарта
                OSTIS}
            \begin{scnitemize}
                \item Продуктом является база знаний, а также ежегодно издаваемые
                            коллективные монографии
                \item Все рецензируют всё
                \item Но по каждому разделу есть ответственный редактор
                \item Согласование (рецензирование) дополнений/изменений осуществляется по
                            следующим принципам:
                            \begin{scnitemizeii}
                                \item Либо формируется рецензия с замечаниями по обязательной доработке
                                            и пожеланиями
                                \item Либо фиксируется согласие с предлагаемыми изменениями
                                \item При этом каждый рецензент должен подтвердить устранение своих
                                            обязательных замечаний
                                \item Формируется и учитывается рейтинг рецензентов-экспертов
                                \item Предложение принимается автоматически по формуле, учитывающей
                                            рейтинг и количество согласий/рецензий.
                            \end{scnitemizeii}
            \end{scnitemize}}
    \end{scnrelfromset}
\end{scnstruct}
\end{SCn}

\newpage

\begin{SCn}
\scnsectionheader{Авторский коллектив Стандарта OSTIS}
\begin{scnstruct}
    \scnheader{соавтор Стандарта OSTIS}
    \scnidtf{\uline{член Авторского Коллектива Стандарта OSTIS}, направленного
        на
        развитие Технологии OSTIS и описание каждой текущей версии этой
        Технологии в
        виде соответствующего Стандарта}
    \begin{scnrelfromset}{направления и принципы организации деятельности}
        \scnfileitem{Отслеживать и читать новые публикации по тематике,
            рассматриваемой в Стандарте OSTIS.Близкими источниками для
            этого
            являются:\begin{scnitemize}
                \item выпуски журналов:
                \begin{scnitemizeii}
                    \item Онтология проектирования
                \end{scnitemizeii}
                \item материалы конференций:
                \begin{scnitemizeii}
                    \item КИИ, Консорциум W3C
                \end{scnitemizeii}
                \item публикации, ключевыми терминами которых
                являются:\begin{scnitemizeii}
                    \item формальная онтология\item онтология верхнего
                    уровня\item семантическая
                    сеть\item граф знаний\item графовая база данных\item
                    смысловое представление
                    знаний\item конвергенция в ИИ\end{scnitemizeii}
                \item стандарты\end{scnitemize}
        }

        \scnfileitem{Фиксировать результаты знакомства с новыми публикациями
            по тематике, близкой Стандарту OSTIS, в Библиографии OSTIS, а
            также в основном
            тексте Стандарта OSTIS в виде соответствующих ссылок, цитат,
            сравнительного
            анализа}

        \scnfileitem{Отслеживать текущее состояние всего текста Стандарта
            OSTIS, формировать предложения, направленные на развитие
            Стандарта OSTIS и на
            повышение темпов этого развития. Активно участвовать в
            обсуждении проблем
            развития Технологии OSTIS}

        \scnfileitem{Максимально возможным образом увязывать персональную
            работу над Стандартом OSTIS с другими формами деятельности -
            научной, учебной,
            прикладной}

        \scnfileitem{Указывать авторство своих предложений по дополнению и/или
            корректировке текущего текста Стандарта OSTIS}

        \scnfileitem{Участвовать в рецензировании и согласовании предложений,
            представленных другими авторами Стандарта OSTIS}

    \end{scnrelfromset}
    \scnheader{Организация работ по развитию Стандарта OSTIS}
    \begin{scneqtoset}
        \scnfileitem{Всем прочитать текст текущей версии Стандарта OSTIS и
            написать конструктивные замечания к доработке}
        \scnfileitem{Каждому уточнить свой персональный план участия в работе
            над второй версией (с учетом своих интересов и диссертационной
            тематики)}
    \end{scneqtoset}

    \scnheader{План работ по подготовке \textit{Стандарта OSTIS-2022}}
    \begin{scneqtoset}
        \scnfileitem{Распределить разделы по авторам (для всех сотрудников)}
        \scnfileitem{Составить четкий план доработки каждого раздела}
        \scnfileitem{Организовать регулярные семинары по обсуждению развития
            каждого раздела}
    \end{scneqtoset}

    \scnheader{Орг-План каждого члена авторского коллектива}
    \begin{scneqtoset}
        \scnfileitem{Прочитать весь текст текущей версии Стандарта OSTIS}
        \scnfileitem{Сформировать свое мнение о текущих недостатках и
            направлениях развития Стандарта (осознать свою ответственность
            как соавтора)}
        \scnfileitem{Из оглавления последующей версии Стандарта определить те
            разделы, в развитии которых вы готовы активно участвовать}
        \scnfileitem{По согласованию утвердить распределение авторов по
            разделам и определить \uline{иерархическую} структуру локальных
            рабочих
            коллективов}
        \begin{scnindent}
            \scntext{примечание}{По каждому рабочему коллективу определить перечень
                вопросов,
                требующих обсуждения и согласования}
        \end{scnindent}
    \end{scneqtoset}
\end{scnstruct}
\end{SCn}

\newpage

\begin{SCn}
\scnsectionheader{Правила оформления Стандарта OSTIS}
\begin{scnstruct}
	\scnheader{Стандарт OSTIS}
	\scnrelfrom{общие правила построения}{\scnkeyword{Общие правила
			построения Стандарта OSTIS}}
		\begin{scnindent}
			\scnidtf{принципы, лежащие в основе структуризации и оформления
				Стандарта OSTIS}
		\end{scnindent}
	\begin{scneqtovector}
        \scnfileitem{Основной формой представления \textit{Стандарта OSTIS} как
                    полной документации текущего состояния \textit{Технологии OSTIS} является
                    \textit{текущее состояние} основной части \textit{базы знаний} специальной
                    интеллектуальной компьютерной \textit{Метасистемы IMS.ostis}, обеспечивающей
                    использование и эволюцию (перманентное совершенствование) \textit{Технологии
                        OSTIS}. Такое представление \textit{Стандарта OSTIS} обеспечивает эффективную
                    семантическую навигацию по содержанию \textit{Стандарта OSTIS} и возможность
                    задавать \textit{Метасистеме IMS.ostis} широкий спектр нетривиальных вопросов о
                    самых различных деталях и тонкостях \textit{Технологии OSTIS}}
        \scnfileitem{Непосредственно сам \textit{Стандарт OSTIS} представляет
                    собой внутреннее \textit{смысловое представление} основной части базы знаний
                    \textit{Метасистемы IMS.ostis} на внутреннем смысловом языке
                    \textit{ostis-систем} (этот язык назван нами \textit{SC-кодом} - Semantic
                    Computer Code)}
        \scnfileitem{С семантической точки зрения \textit{Стандарт OSTIS}
                    представляет собой иерархическую систему формальных моделей \textit{предметных
                        областей} и соответствующих им \textit{формальных онтологий}}
        \scnfileitem{С семантической точки зрения \textit{Стандарт OSTIS}
                    представляет собой большую \textit{рафинированную семантическую сеть}, которая,
                    соответственно, имеет нелинейный характер и которая включает в себя знаки любых
                    видов описываемых сущностей(материальных сущностей, абстрактных сущностей,
                    понятий, связей, структур) и, соответственно этому, содержит связи между всеми
                    этими видами сущностей(в частности, связи между связями, связи между
                    структурами)}
       \scnfileitem{В состав \textit{Стандарта OSTIS} входят также файлы
                    информационных конструкций, не являющихся конструкциями \textit{SC-кода} (в том
                    числе и sc-текстов, принадлежащих различным естественным языкам). Такие файлы
                    позволяют формально описывать в базе знаний синтаксис и семантику различных
                    внешних языков, а также позволяют включать в состав базы знаний различного рода
                    пояснения, примечания, адресуемые непосредственно пользователям и помогающие им
                    в понимании формального текста базы знаний}
        \scnfileitem{Кроме представления \textit{Стандарта OSTIS} на внутреннем
        \textit{языке представления знаний} используется также внешняя форма
        представления \textit{Стандарта OSTIS} на \textit{внешнем языке представления
            знаний}. При этом указанное внешнее представление \textit{Стандарта OSTIS}
        должно быть структурировано и оформлено так,чтобы читатель мог достаточно легко
        \scnqq{вручную} найти в этом тексте практически любую интересующую его
        \textit{информацию}. В качестве \textit{формального языка} внешнего
        представления \textit{Стандарта OSTIS} используется \textit{SCn-код}, описание
        которого приведено в \textit{Стандарте OSTIS} в разделе \nameref{intro_scn}}
        \scnfileitem{Предлагаемое Вам издание \textit{Стандарта OSTIS}
                    представляется на формальном языке \textit{SCn-код}, который является языком
                    внешнего представления больших текстов \textit{SC-кода}, в которых большое
                    значение имеет наглядная структуризация таких текстов.}
        \scnfileitem{\textit{Стандарт OSTIS} имеет онтологическую
                    структуризацию, т.е. представляет собой иерархическую систему связанных между
                    собой \textit{формальных предметных областей} и соответствующих им
                    \textit{формальных онтологий}.Благодаря этому обеспечивается высокий уровень
                    стратифицированности \textit{Стандарта OSTIS}}
        \scnfileitem{Каждому \textit{понятию}, используемому в
                    \textit{Стандарте OSTIS}, соответствует свое место в рамках этого Стандарта,
                    своя \textit{предметная область} и соответствующая ей \textit{онтология}, где
                    это \textit{понятие} подробно рассматривается (исследуется), где
                    концентрируется вся основная информация об этом \textit{понятии}, различные его
                    свойства}
        \scnfileitem{Кроме \textit{Общих правил построения Стандарта OSTIS} в
                    \textit{Стандарте OSTIS} приводятся описания различных частных
                    (специализированных) правил построения (оформления) различных видов фрагментов
                    \textit{Стандарта OSTIS}.К таким видам фрагментов относятся следующие:
                    \begin{scnitemize}
                        \item \textit{sc-идентификатор}
                            \textit{внешний идентификатор внутреннего знака (\textit{sc-элемента}) входящего в состав \textit{базы знаний
                                    ostis-системы}}
                            \textit{информационная конструкция
                                (чаще всего это строка символов), обеспечивающая однозначную идентификацию
                                соответствующей сущности, описываемой в \textit{базах знаний ostis-систем}, и
                                являющаяся, чаще всего, именем (термином), соответствующим описываемой
                                сущности, именем, обозначающим эту сущность во внешних текстах
                                \textit{ostis-систем}}
                        \item \textit{sc-спецификация}
                                \textit{семантическая окрестность}
                                \textit{семантическая окрестность
                                    соответствующего \textit{sc-элемента} (внутреннего знака, хранимого в памяти
                                    \textit{ostis-системы} в составе её \textit{базы знаний}, представленной на
                                    внутреннем языке \textit{ostis-систем.})}
                                \textit{семантическая окрестность некоторого
                                    \textit{sc-элемента}, хранимого в \textit{sc-памяти}, в рамках текущего
                                    состояния этой \textit{sc-памяти}}
                        \item (\textit{sc-конструкция} $\setminus$
                              \textit{sc-спецификация})
                        \textit{sc-конструкция (конструкция
                            \textit{SC-кода} --- внутреннего языка \textit{ostis-систем}), не являющаяся
                            \textit{sc-спецификацией}}
                        \item (\textit{файл ostis-системы $\setminus$
                                    sc-идентификатор})
                            \textit{\textit{файл ostis-системы}, не
                                    являющийся sc-идентификатором}
                    \end{scnitemize}
        }
        \scnfileitem{Правила построения \textit{sc-идентификаторов} будем также
                    называть \textit{Правилами внешней идентификации sc-элементов}.\\
                    \textit{Общие правила построения sc-идентификаторов} смотрите в
                    разделе \scnqqi{\nameref{intro_idtf}}. В состав этих правил входят правила внешней
                    идентификации \textit{sc-констант, sc-переменных}, \textit{отношений},
                        \textit{параметров},\textit{sc-конструкций}, \textit{файлов ostis-систем}.}
        \scnfileitem{К числу частных правил построения sc-идентификаторов
                    относятся:
                    \begin{scnitemize}
                        \item \textit{Правила построения sc-идентификаторов
                            персон} --- смотрите раздел \scnqqi{\nameref{sd_person}}.
                        \item \textit{Правила построения sc-идентификаторов
                            библиографических источников} --- смотрите раздел \scnqqi{\nameref{sd_bibliography}}.
                    \end{scnitemize}
        }
        \scnfileitem{Общие правила построения \textit{sc-конструкций}
                    (конструкций \textit{SC-кода} --- внутреннего языка ostis-систем) смотрите в
                    разделе \scnqqi{\nameref{intro_sc_code}}, а также в разделе
                    \scnqqi{\nameref{sd_sc_code_syntax}} и в разделе \scnqqi{\nameref{sd_sc_code_semantic}}.
        }
        \scnfileitem{К числу частных правил построения \textit{sc-конструкций}
                    относятся:
                    \begin{scnitemize}
                        \item \textit{Правила построения баз знаний
                            ostis-систем} --- смотрите раздел \scnqqi{\nameref{sd_knowledge}}. Эти правила направлены
                        на обеспечение целостности баз знаний ostis-систем, на обеспечение (1)
                        востребованности (нужности) знаний, входящих в состав каждой базы знаний, и (2)
                        целостности самой базы знаний, т.е. достаточности знаний, входящих в состав
                        каждой базы знаний для эффективного функционирования соответствующей
                        ostis-системы
                        \item \textit{Правила построения разделов и сегментов
                            баз знаний ostis-систем}
                        \item \textit{Правила представления логических формул и высказываний в базах знаний ostis-систем}
                        \item \textit{Правила представления формальных предметных областей в базах знаний ostis-систем}
                        \item \textit{Правила представления формальных логических онтологий в базах знаний ostis-систем}
                    \end{scnitemize}
        }
        \scnfileitem{Формальное описание синтаксиса и семантики \textit{SCn-кода} приведено в разделе \scnqqi{\nameref{intro_scn}}.}
	\end{scneqtovector}
	\scnnote{\textit{сослаться на:}
		\begin{scnitemize}
			\item правила построения sc-идентификаторов для различных классов
			sc-элементов
			\item правила построения различного вида sc-текстов
			\begin{itemize}
				\item разделов базы знаний
				\item баз знаний
			\end{itemize}
		\end{scnitemize}
		\begin{scnitemize}
			\item правила построения sc-спецификаций сущностей
			\begin{itemize}
				\item разделов базы знаний
				\item предметная область
				\item сегментов базы знаний
				\item библ.- источников
			\end{itemize}
		\end{scnitemize}
		\begin{scnitemize}
			\item типичные опции
			\begin{itemize}
				\item баз знаний
			\end{itemize}
		\end{scnitemize}
		\begin{scnitemize}
			\item общие направления развития
			\begin{itemize}
				\item раздел базы знаний
			\end{itemize}
		\end{scnitemize}
		\begin{scnitemize}
			\item направления развития*:
			\begin{itemize}
				\item Стандарта OSTIS
				\item Введение в язык OSTIS-Cn
				\item Предметная область и онтология библиографии
				\item Библиография OSTIS
			\end{itemize}
		\end{scnitemize}
		\begin{scnitemize}
			\item направления и правила
			\begin{itemize}
				\item деятельность
				\item Редколлегия Стандарта OSTIS
				\item соавтор Стандарта OSTIS
			\end{itemize}
		\end{scnitemize}
	}
	\scnheader{sc-идентификатор файла ostis-системы}
	\scnrelfrom{правила построения}{Правила идентификации файлов
		ostis-систем}
	\scnheader{sc-файл ostis-системы}
	\scnrelfrom{правила построения}{Правила построения sc-файлов
		ostis-систем}
	\scnheader{спецификация файла ostis-системы}
	\scnrelfrom{правила построения}{Правила спецификации файлов
		ostis-систем}
\end{scnstruct}
\end{SCn}

\newpage

\begin{SCn}
\scnsectionheader{Общие направления развития Cтандарта OSTIS}
\begin{scnstruct}
    \scnheader{Направления развития Cтандарта OSTIS}

    \begin{scneqtoset}
        \scnfileitem{Все sc-идентификаторы, входящие в состав scg-текстов,
            scn-текстов и различных иллюстраций должны иметь одинаковый
            шрифт и размер. За
            исключением, возможно, размера sc-идентификаторов в
            scg-текстах. (См.,
            например, страницы 443-448 стандарта-2021)}
        \scnfileitem{Уточнить Алфавит SCg-кода}
    \end{scneqtoset}

    \scnheader{Cтандарт OSTIS}
    \begin{scnrelfromset}{направления перманентного развития}
        \scnfileitem{В рамках титульных спецификаций разделов Стандарта OSTIS
            постоянно уточнять и детализировать семантические связи каждого
            раздела с
            другими разделами}
        \scnfileitem{Каждая новая именуемая (идентифицируемая) сущность,
            описываемая в Стандарте OSTIS (в первую очередь, каждое
            понятие), должна быть
            подробно специфицирована в соответствующем разделе}
    \end{scnrelfromset}

    \begin{scnrelfromset}{направления текущего этапа развития}
        \scnfileitem{Доработать в текущую версию Стандарта OSTIS и ввести:
            \begin{scnitemize}
                \item Оглавление
                \item Титульную спецификацию Стандарта OSTIS
            \end{scnitemize}}
        \scnfileitem{Привести текущую версию Стандарта OSTIS в соответствие с
            новой версией оглавления Стандарта OSTIS}
        \scnfileitem{Включить расширенный материал статьи В.В. Голенкова и
            соавторов в Springer-2021 в текущую версию Стандарта OSTIS}
        \scnfileitem{Включить в Стандарт OSTIS все методические рекомендации
            по развитию Стандарта OSTIS (правила построения различных
            фрагментов, правила
            идентификации sc-элементов, правила спецификации sc-элементов
            --- точнее
            обозначение сущностей, направления развития)}
    \end{scnrelfromset}

    \scnheader{Cтандарт OSTIS --- Общий орг-план}
    \begin{scnrelfromset}{направления развития}
        \scnfileitem{Доработать правила оформления Стандарта OSTIS:
            \begin{scnitemize}
                \item Правила структуризации, типология компонентов,
                семантические связи
                между компонентами
                \item Правила идентификации sc-элементов (именования
                различных
                классифицированных сущностей)
                \item Правила спецификации различных классов sc-элементов
            \end{scnitemize}}
        \scnfileitem{Существенно расширить библиографию Стандарта OSTIS и
            библиографии всех разделов. \scnqqi{Преобразовать простой список
            библиографических
            источников в библиографическую спецификацию соответствующего
            текста}}
        \scnfileitem{Чётко сформировать общий план доработки всего текста
            Стандарта OSTIS, а также конкретные планы доработки каждого
            раздела}
        \scnfileitem{Содержание всех работ, опубликованных авторами Стандарта
            OSTIS по Технологии OSTIS должны быть формализованы и включены
            в
            соответствующие разделы Стандарта OSTIS (имеются в виду отчёты
            по лабораторным
            работам студентов, расчётные работы, курсовые проекты,дипломные
            проекты,
            диссертации, статьи в материалах конференций OSTIS, в сборниках
            Springer, в
            журнале Онтология проектирования и других изданиях). Более
            того, если первичной
            публикацией новых материалов будет их включение в состав
            Стандарта OSTIS, то
            это существенно повысит результативность работ по развитию
            Стандарта OSTIS}
        \scnfileitem{Включить в текст монографии все статьи Голенкова В.В. на
            конференциях OSTIS и другие публикации (в том числе книги)
            \begin{scnitemize}
                \item Это актуально для работы над изданием Стандарта
                OSTIS(в эту версию
                Стандарта OSTIS  надо собрать абсолютно всё, что нами
                сделано)
            \end{scnitemize}}
        \scnfileitem{Увеличить число ссылок на библиографические источники из
            текста Стандарта OSTIS}
    \end{scnrelfromset}

    \scnheader{Cтандарт OSTIS}
    \begin{scnrelfromset}{направления развития}
        \scnfileitem{Некоторые материалы текущего состояния ряда разделов
            целесообразно перенести в дочерние разделы (если имеющаяся
            детализация этих
            материалов более уместна для дочерних разделов).Это, например,
            касается
            некоторых сегментов раздела \scnqqi{\nameref{intro_ostis}} (в
            частности сегмента об
            Экосистеме OSTIS)}
        \scnfileitem{Дополнить этот список разделов}
        \scnfileitem{Совершенствовать стратификацию}
    \end{scnrelfromset}

    \scnheader{Правила организации развития исходного текста Стандарта OSTIS}
    \scnidtf{Правила организации коллективной деятельности по развитию
        исходного
        текста Стандарта OSTIS}
    \scnheader{Cтандарт OSTIS}
    \begin{scnrelfromset}{правила построения}
        \scnfileitem{Каждому разделу приписать одного ответственного редактора
            и возможно несколько соавторов (ответственный редактор является
            единственным
            автором)}
        \scnfileitem{LaTeX + макросы}
        \scnfileitem{GitHub --- структуризация файлов ostis-ai}
        \scnfileitem{Предложения/рецензирования/включение}
        \scnfileitem{Рецензирование}
        \scnfileitem{Извлечение и конвертирование в pdf-файл любого раздела
            или группы разделов}
        \scnfileitem{Просмотр pdf-файла}
        \scnfileitem{Конвертирование в scs}
        \scnfileitem{Загрузка в sc-память}
        \scnfileitem{Просмотр базы знаний}
    \end{scnrelfromset}
\end{scnstruct}
\end{SCn}

\newpage

\bigskip
%\scnfragmentcaption

\scnheader{Пояснения к оглавлению Стандарта OSTIS и к некоторым разделам этого Стандарта}

\scnstartsubstruct

\scnheader{Спецификация второго издания Стандарта OSTIS}
\scnidtf{Спецификация второй официальной версии Стандарта OSTIS}
\scnidtf{Спецификация Стандарта OSTIS-2022}

\scnheader{Анализ методологических проблем современного состояния работ в области Искусственного интеллекта}
\scnidtf{Актуальность Технологии OSTIS}
\scnidtf{Современные требования, предъявляемые к деятельности в области Искусственного интеллекта  к интеллектуальным компьютерным системам следующего поколения --- конвергенция, глубокая (\scnqq{бесшовная}) интеграция, высокий уровень обучаемости (гибкости, стратифицированности, рефлексивности), высокий уровень социализации (взаимопонимания, договороспособности, способности координировать свои действия с другими субъектами), стандартизация}

\scnheader{Введение в описание внутреннего языка ostis-систем}
\scnidtf{Введение в SC-code (Semantic Computer Code)}

\scnheader{Предметная область и онтология внешних идентификаторов знаков, входящих в информационные конструкции внутреннего языка ostis-систем}
\scnidtf{Предметная область и онтология sc-идентификаторов}

\scnheader{Введение в описание языка графического представления информационных конструкций, хранимых в памяти ostis-систем}
\scnidtf{Введение в SCg-code (Semantic Code graphical)}

\scnheader{Введение в описание языка линейного представления информационных конструкций, хранимых в памяти ostis-систем}
\scnidtf{Введение в SCs-code (Semantic Code string)}

\scnheader{Введение в описание языка форматирования линейного представления информационных конструкций, хранимых в памяти ostis-систем}
\scnidtf{Введение в SCn-code (Semantic Code natural)}

\scnheader{Предметная область и онтология кибернетических систем}
\scnidtf{Предпосылки создания компьютерных систем нового поколения}

\scnheader{Предметная область и онтология компьютерных систем}
\scnidtf{Этапы эволюции (повышения качества) компьютерных систем --- эволюции памяти, информации, хранимой в памяти, решателей задач, интерфейсов}

\scnheader{Предметная область и онтология интеллектуальных компьютерных систем}
\scnidtf{Этапы эволюции (повышения качества) интеллектуальных компьютерных систем и проблемы дальнейшей их эволюции}

\scnheader{Предметная область и онтология технологий автоматизации различных видов человеческой деятельности}
\scnidtf{Эволюция технологий проектирования, производства и эксплуатации компьютерных систем и предпосылки создания компьютерных технологий нового поколения}

\scnheader{Предметная область и онтология логико-семантических моделей компьютерных систем, основанных на смысловом представлении информации}
\scnidtf{Предлагаемый подход к построению интеллектуальных компьютерных систем следующего поколения}

\scnheader{Предметная область и онтология внутреннего языка ostis-систем}
\scnidtf{Предметная область и онтология SC-кода (Semantic Computer Code)}
\scnrelfrom{введение}{\textit{\nameref{intro_sc_code}}}

\scnheader{Предметная область и онтология  базовой денотационной семантики SC-кода}
\scniselement{\textit{предметная область и онтология верхнего уровня}}


\scnheader{Предметная область и онтология языка графического представления информационных конструкций, хранимых в памяти ostis-систем}
\scnidtf{Предметная область и онтология SCg-кода (Semantic Code graphical)}
\scnrelfrom{введение}{\textit{\nameref{intro_scg}}}

\scnheader{Предметная область и онтология языка линейного представления информационных конструкций, хранимых в памяти ostis-систем}
\scnidtf{Предметная область и онтология SCs-кода (Semantic Code string)}
\scnrelfrom{введение}{\textit{\nameref{intro_scs}}}

\scnheader{Предметная область и онтология языка форматирования линейного представления информационных конструкций, хранимых в памяти ostis-систем}
\scnidtf{Предметная область и онтология SCn-кода (Semantic Code natural)}
\scnrelfrom{введение}{\textit{\nameref{intro_scn}}}

\scnheader{Предметная область и онтология файлов, внешних информационных конструкций и внешних языков ostis-систем}
\scnrelto{дочерний раздел}{\nameref{intro_lang}}

\scnheader{Предметная область и онтология операционной семантики sc-языка вопросов}
\scnidtf{Предметная область информационно-поисковых действий и агентов, а также соответствующая онтология методов}

\scnheader{Предметная область и онтология операционной семантики логических sc-языков}
\scnidtf{Предметная область и онтология логических исчислений}
\scnidtf{Предметная область и онтология действий и агентов логического вывода, а также соответствующая онтология методов (правил) логического вывода}

\scnheader{Предметная область и онтология sc-языков программирования высокого уровня}
\scnidtf{Предметная область и онтология sc-языков программирования высокого и сверхвысокого уровня, ориентированных на обработку баз знаний ostis-систем}

\scnheader{Предметная область и онтология операционной семантики sc-моделей искусственных нейронных сетей}
\scnidtf{Предметная область и онтология процессов функционирования sc-моделей искусственных нейронных сетей при обработке баз знаний ostis-систем}

\scnheader{Логико-семантическая модель средств автоматизации управления взаимодействием разработчиков различных категорий в процессе проектирования базы знаний ostis-системы}
\scnidtf{Логико-семантическая модель средств автоматизации управления взаимодействием менеджеров, авторов, рецензентов, экспертов и редакторов в процессе проектирования базы знаний ostis-системы}

\scnheader{Предметная область и онтология встроенных ostis-систем поддержки эксплуатации соответствующих ostis-систем конечными пользователями}
\scnidtf{Интеллектуальные \textit{встроенные ostis-системы}, обучающие \textit{конечных пользователей} эффективной эксплуатаии тех \textit{ostis-систем}, в состав которых они входят}
\scnidtf{Предметная область и онтология методов и средств реализации целенаправленного и персонифицированного обучения пользователей каждой ostis-системы}

\scnheader{Предметная область и онтология Экосистемы OSTIS}
\scnidtf{Проект smart-общества}

\scnheader{Логико-семантическая модель Метасистемы IMS.ostis}
\scnrelfrom{примечание}{\scnnonamednode}
\begin{scnindent}
	\begin{scneqtoset}
		\scnfilelong{IMS.ostis}
		\scnrelto{сокращение}{\scnfilelong{Метасистема IMS.ostis}}
		\begin{scnindent}
			\scnrelto{сокращение}{\scnfilelong{Intelligent MetaSystem of Open Semantic Technology for Intelligent Systems}}
		\end{scnindent}
	\end{scneqtoset}
\end{scnindent}
\scnidtf{Логико-семантическая модель интеллектуального ostis-портала научно-технических знаний по Технологии OSTIS}

\scnendstruct

\end{SCn}

\newpage