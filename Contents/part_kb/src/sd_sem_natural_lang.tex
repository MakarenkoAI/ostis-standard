\begin{SCn}
    \scnsectionheader{Предметная область и онтология денотационной семантики естественных языков}
    \begin{scnsubstruct}

        \scnsectionheader{Предметная область денотационной семантики естественных языков}
        \begin{scnhaselementrolelist}{максимальный класс объектов исследования}
            \scnitem{денотационная семантика естественного языка}
        \end{scnhaselementrolelist}
        \begin{scnrelfromlist}{библиографическая ссылка}
            \scnitem{\scncite{Heim1998}}
            \scnitem{\scncite{Winter2016}}
            \scnitem{\scncite{Portner2008}}
        \end{scnrelfromlist}

        \scnheader{формализация денотационной семантики естественного языка}
        \scntext{определение}{Формализацией \textit{денотационной семантики} \textit{естественного языка} мы будем считать множество общих правил перехода от \textit{синтаксиса информационных конструкций} \textit{естественных языков} к \textit{смысловому представлению информации}, содержащейся в исходной \textit{информационной конструкции}.}
        \scntext{пояснение}{Предлагается вариант формализации \textit{денотационной семантики} \textit{естественных языков} в рамках \textit{Технологии OSTIS}, для составления которой использовались стандартные положения формальной семантики.}
        \begin{scnindent}
            \begin{scnrelfromset}{источник}
                \scnitem{\scncite{Heim1998}}
                \scnitem{\scncite{Winter2016}}
                \scnitem{\scncite{Portner2008}}
            \end{scnrelfromset}
        \end{scnindent}
        
        \scnheader{денотационная семантика естественного языка}
        \scntext{примечание}{\textit{Денотационная семантика} \textit{языка} специфицирует интерпретацию элементов \textit{синтаксиса} данного \textit{языка} и представляет собой множество формул, описывающих то, каким образом \textit{информационным конструкциям} \textit{языка} ставятся в соответствие обозначаемые ими сущности и конфигурации отношений между этими сущностями.}
        \scntext{примечание}{\textit{Денотационная семантика} \textit{естественных языков} должна обладать свойством композициональности --- то есть интерпретация всего высказывания должна выводиться из интерпретации отдельных его частей.
            \\Таким образом, необходимо предоставить формальное описание интерпретации элементов \textit{синтаксиса} \textit{естественного языка}, представленных в предыдущем разделе, а также описание правил совмещения интерпретации отдельных элементов для получения \textit{смысла} всего высказывания.}

        \scnheader{правила, реализующие \textit{денотационную семантику} \textit{языка}}
        \scntext{примечание}{Данные правила должны применяться последовательно и позволяют получить \textit{смысл} текста \textit{естественного языка} по его синтаксической структуре, \scnqq{поднимаясь} по дереву \textit{составляющих} от \textit{вершин} к \textit{максимальным проекциям}.}
        \scnhaselement{Правило интерпретации вершины группы прилагательного и вершины именной группы}
        \begin{scnindent}
            \scnrelfrom{пример}{SCg-текст. Правило интерпретации вершины группы прилагательного и вершины именной группы}
        \end{scnindent}
        \scnhaselement{Правило интерпретации максимальной проекции вершины именной группы}
        \begin{scnindent}
            \scnrelfrom{пример}{SCg-текст. Правило интерпретации максимальной проекции вершины именной группы}
        \end{scnindent}
        \scnhaselement{Правило интерпретации максимальной проекции вершины глагольной группы, содержащей непереходный глагол}
        \begin{scnindent}
            \scnrelfrom{пример}{SCg-текст. Правило интерпретации максимальной проекции вершины глагольной группы, содержащей непереходный глагол}
        \end{scnindent}
        \scnhaselement{Правило интерпретации максимальной проекции вершины группы детерминанта}
        \begin{scnindent}
            \scnrelfrom{пример}{SCg-текст. Правило интерпретации максимальной проекции вершины группы детерминанта}
        \end{scnindent}
        \scnhaselement{Правило интерпретации промежуточной проекции вершины временной группы}
        \begin{scnindent}
            \scnrelfrom{пример}{SCg-текст. Правило интерпретации промежуточной проекции вершины временной группы}
        \end{scnindent}
        \scnhaselement{Правило интерпретации максимальной проекции вершины временной группы}
        \begin{scnindent}
            \scnrelfrom{пример}{SCg-текст. Правило интерпретации максимальной проекции вершины временной группы}
        \end{scnindent}
        \scnhaselement{Правило интерпретации предложения с переходным глаголом}
        \begin{scnindent}
            \scnrelfrom{пример}{SCg-текст. Правило интерпретации предложения с переходным глаголом}
        \end{scnindent}

        \scnheader{SCg-текст. Правило интерпретации вершины группы прилагательного и вершины именной группы}
        \scneq{{\scnfileimage[30em]{Contents/part_kb/src/images/sd_natural_languages/d_sem_1.png}}}
        \scntext{примечание}{На \textit{SCg-текст. Правило интерпретации вершины группы прилагательного и вершины именной группы} приведено правило, по которому происходит интерпретация вершин \textit{именной группы} и \textit{группы прилагательного}.
            \\Смыслом таких вершин является класс, например: \textit{прилагательному} \scnqq{черный} соответствует множество черных объектов, а \textit{существительному} \scnqq{кот} --- множество котов.}

        \scnheader{SCg-текст. Правило интерпретации максимальной проекции вершины именной группы}
        \scneq{{\scnfileimage[30em]{Contents/part_kb/src/images/sd_natural_languages/d_sem_2.png}}}
        \scntext{примечание}{На \textit{SCg-текст. Правило интерпретации максимальной проекции вершины именной группы} приведено правило, по которому происходит интерпретация \textit{именной группы}, максимальная проекция которой включается в себя также группу прилагательного.
            \\Как говорилось выше, для применения данного правила необходимо предварительное применение правила, представленного на \textit{SCg-текст. Правило интерпретации вершины группы прилагательного и вершины именной группы}.
            \\Смыслом таких конструкций является класс, являющийся результатом \textit{пересечения} классов, полученных в результате интерпретации \textit{вершин} \textit{групп прилагательного} и \textit{именной группы} по отдельности.
            \\Например: \textit{черный кот} --- множество черных котов, пересечение множества котов и черных объектов.}
        %тут мы комбинируем смыслы прилагательного и существительного, которые входят в одну именную группу

        \scnheader{SCg-текст. Правило интерпретации максимальной проекции вершины глагольной группы, содержащей непереходный глагол}
        \scneq{{\scnfileimage[30em]{Contents/part_kb/src/images/sd_natural_languages/d_sem_3.png}}}
        \scntext{примечание}{На \textit{SCg-текст. Правило интерпретации максимальной проекции вершины глагольной группы, содержащей непереходный глагол} приведено правило, по которому происходит интерпретация \textit{глагольной группы}.
            \\Необходимость включения в посылку правила всей ветки глагольной группы объясняется ее необходимостью для определения типа \textit{глагола} --- данное правило предназначено для интерпретации непереходных \textit{глаголов}.
            \\\textit{смыслом} такой конструкции является класс \textit{действий}.}
        %тут мы задаем интерпретацию всей глагольной группы (макс проекции) только для непереходных глаголов. написать, что смотрим по всей структуре группы целиком, потому что для того, чтобы отличить непереходный от переходного нам нужна вся ветка глагольной группы в дереве целиком

        \scnheader{SCg-текст. Правило интерпретации максимальной проекции вершины группы детерминанта}
        \scneq{{\scnfileimage[30em]{Contents/part_kb/src/images/sd_natural_languages/d_sem_4.png}}}
        \scntext{примечание}{На \textit{SCg-текст. Правило интерпретации максимальной проекции вершины группы детерминанта} приведено правило, по которому происходит интерпретация \textit{группы детерминанта} с неопределенным артиклем.
            \\Смыслом такой конструкции является существование элемента класса, являющегося \textit{смыслом} входящей в состав данной \textit{группы детерминанта} \textit{именной группы}.}
        %тут задается интерпретация сочетания именной группы с артиклем (в данном случае неопределенным)

        \scnheader{SCg-текст. Правило интерпретации промежуточной проекции вершины временной группы}
        \scneq{{\scnfileimage[30em]{Contents/part_kb/src/images/sd_natural_languages/d_sem_5.png}}}
        \scntext{примечание}{На \textit{SCg-текст. Правило интерпретации промежуточной проекции вершины временной группы} приведено правило, по которому происходит интерпретация \textit{промежуточной проекции вершины временной группы}, состоящей из вспомогательного \textit{глагола} и полнозначного глагола.
            \\Вспомогательный \textit{глагол} в данном случае задет класс действий по времени (является ли оно запланированным, выполняемым, уже выполненным и так далее).}
        %тут задаем интерпретацию сочетания вспомогательного глагола и основного глагола. вспомогательный у нас соответствует классу действий по времени

        \scnheader{SCg-текст. Правило интерпретации максимальной проекции вершины временной группы}
        \scneq{{\scnfileimage[30em]{Contents/part_kb/src/images/sd_natural_languages/d_sem_6.png}}}
        %тут задаем интерпретацию для аргументной структуры непереходного глагола (сочетания подлежащего с непереходным глаголом)
        \scntext{примечание}{На \textit{SCg-текст. Правило интерпретации максимальной проекции вершины временной группы} приведено правило, по которому происходит интерпретация \textit{максимальной проекции вершины временной группы} на основе полученного на предыдущем шаге \textit{смысла} \textit{промежуточной проекции вершины временной группы} и \textit{смысла} \textit{максимальной проекции группы детерминанта}.}

        \scnheader{SCg-текст. Правило интерпретации предложения с переходным глаголом}
        \scneq{{\scnfileimage[30em]{Contents/part_kb/src/images/sd_natural_languages/d_sem_7.png}}}
        \scntext{примечание}{На \textit{SCg-текст. Правило интерпретации предложения с переходным глаголом} приведено правило, по которому происходит интерпретация \textit{максимальной проекции вершины группы комплементатора} на основе полученных на предыдущих шагах \textit{смыслов} более частных конструкций.
            \\Данным правилом задается интерпретация предложения с переходным \textit{глаголом}.}
        %тут задаем интерпретацию аргументной структуры переходного глагола (сочетания переходного глагола с его аргументами --- подлажещим и дополнением)

    \end{scnsubstruct}
    \bigskip
    \scnendcurrentsectioncomment
\end{SCn}
