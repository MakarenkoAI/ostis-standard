\begin{SCn}
	\scnsectionheader{Предметная область и онтология пространственных сущностей различных форм}
	\begin{scnsubstruct}
	\begin{scnrelfromlist}{максимальный класс исследования}
		\scnitem{пространственная сущность}
		\scnitem{материальная сущность}
		\scnitem{вещество}
		\scnitem{физическое поле}
		\scnitem{персона}
		\scnitem{юридическое лицо}
		\scnitem{предприятие}
		\scnitem{географический объект}
		\scnitem{и др.}
	\end{scnrelfromlist}
	
	
	
	\scnheader{пространственная сущность}
	\scnidtf{класс материальных объектов, временно находящихся в определенном положении в пространстве}
	\begin{scnrelfromlist}{связь}
		\scnitem{предметная область материальных сущностей}
		\scnitem{предметная область ситуаций и событий}
		\scnitem{предметная область временных сущностей}
	\end{scnrelfromlist}
	
	\scnheader{пространственная сущность}
	\scnsuperset{геометрический объект}
	\begin{scnrelfromlist}{разбиение}
		\scnitem{точка}
		\begin{scnindent}
			\scnidtf{это наименьшая пространственная сущность, характеризуюзщаяся только положением}
		\end{scnindent}
		\scnitem{линия}
		\begin{scnindent}
			\scnidtf{это одномерная пространственная сущность, которая имеет длину, но не имеет ширины или высоты. Линия состоит из бесконечного числа точек}
		\end{scnindent}
		\scnitem{плоскость}
		\begin{scnindent}
			\scnidtf{это двухмерная пространственная сущность, которая имеет длину и ширину, но не имеет высоты. Плоскость состоит из бесконечного числа линий}
		\end{scnindent}
		\scnitem{объём}
		\begin{scnindent}
			\scnidtf{это трехмерная пространственная сущность, которая имеет длину, ширину и высоту. Объем состоит из бесконечного числа плоскостей}
		\end{scnindent}
		\scnitem{гиперпространство}
		\begin{scnindent}
			\scnidtf{это пространственная сущность, которая имеет более трех измерений}
		\end{scnindent}
	\end{scnrelfromlist}
	
	\scnheader{основные пространственные отношения}
	\begin{scnrelfromlist}{разбиение}
		\scnitem{на*}
		\begin{scnindent}
			\scnidtf{ориентированное бинарное отношение, первой и второй компонентами связок которого являются знаки материальных сущностей, где первая из которых находится на второй}
		\end{scnindent}
		\scnitem{рядом*}
		\begin{scnindent}
			\scnidtf{неориентированное бинарное отношение, первой и второй компонентами связок которого являются знаки материальных сущностей, которые располагаются с какой-либо стороны относительно друг друга, находятся в непосредственной близости}
		\end{scnindent}
		\scnitem{над*}
		\begin{scnindent}
			\scnidtf{ориентированное бинарное отношение, первой и второй компонентами связок которого являются знаки материальных сущностей, первая из которых находится выше второй}
		\end{scnindent}
		\scnitem{под*}
		\begin{scnindent}
			\scnidtf{ориентированное бинарное отношение, первой и второй компонентами связок которого являются знаки материальных сущностей, первая из которых находится ниже второй}
		\end{scnindent}
		\scnitem{внутри*}
		\begin{scnindent}
			\scnidtf{ориентированное бинарное отношение, означающая расположение одной материальной сущности в другой, где знаки данных сущностей являются соответственно первой и второй компонентами связок отношения}
		\end{scnindent}
	\end{scnrelfromlist}
	\scntext{примечание}{Благодаря данным подклассам отношений при построении семантических описаний, допустим, изображений можно конкретизировать их содержание и пространственную взаимосвязь объектов на изображении}
	
	\scnrelfrom{описание примера}{\scnfileimage[20em]{Contents/part_kb/src/images/car.png}}
	\scnrelfrom{результат примера}{\scnfileimage[24em]{Contents/part_kb/src/images/orm.png}}
	\scnendcurrentsectioncomment
	 \end{scnsubstruct}
\end{SCn}