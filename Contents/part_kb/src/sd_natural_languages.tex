\begin{SCn}
    \scnsectionheader{Предметная область и онтология естественных языков}
    \begin{scnsubstruct}
        \begin{scnrelfromlist}{соавтор}
            \scnitem{Гордей А.Н.}
            \scnitem{Никифоров С.А.}
            \scnitem{Бобёр Е.С.}
            \scnitem{Святощик М.И.}
        \end{scnrelfromlist}
        
        \scnheader{Предметная область естественных языков}
        \scniselement{предметная область}
        \begin{scnhaselementrolelist}{максимальный класс объектов исследования}
            \scnitem{язык}
        \end{scnhaselementrolelist}
        \begin{scnhaselementrolelist}{класс объектов исследования}
            \scnitem{плановый язык}
            \scnitem{язык общения}
            \scnitem{лексема}
            \scnitem{номинативная единица}
            \scnitem{комбинаторный вариант лексемы}
            \scnitem{естественный язык}
            \scnitem{тайген}
            \scnitem{ёген}
            \scnitem{грамматическая категория}
            \scnitem{часть речи}
            \scnitem{словоформа}
            \scnitem{составляющая}
            \scnitem{синтаксическая группа}
            \scnitem{вершина}
            \scnitem{комплемент}
            \scnitem{адъюнкт}
            \scnitem{спецификатор}
        \end{scnhaselementrolelist}
        \begin{scnrelfromset}{ключевой знак}
            \scnitem{Общие правила синтаксической структуры конструкций естественных языков}
            \scnitem{Правила построения синтаксических групп}
            \scnitem{Базовые правила денотационной семантики естественных языков}
        \end{scnrelfromset}
        \begin{scnhaselementrolelist}{исследуемое отношение}
            \scnitem{морфологическая парадигма*}
            \scnitem{член предложения\scnrolesign}
        \end{scnhaselementrolelist}
        \begin{scnrelfromlist}{библиографическая ссылка}
            \scnitem{\scncite{Golenkov2021}}
            \scnitem{\scncite{Pileggi2018}}
            \scnitem{\scncite{Lando2007}}
            \scnitem{\scncite{Farrar2002}}
            \scnitem{\scncite{Chiarcos2012}}
            \scnitem{\scncite{Text2022}}
            \scnitem{\scncite{EAGLES2022}}
            \scnitem{\scncite{Ide2010}}
            \scnitem{\scncite{Schalley2019}}
            \scnitem{\scncite{Mccrae2015}}
            \scnitem{\scncite{GOLD2022}}
            \scnitem{\scncite{Pease2002}}
            \scnitem{\scncite{Farrar2003}}
            \scnitem{\scncite{Chiarcos2012a}}
            \scnitem{\scncite{Bateman1997}}
            \scnitem{\scncite{Bateman2002}}
            \scnitem{\scncite{Buitelaar2009}}
            \scnitem{\scncite{Kostareva2016}}
            \scnitem{\scncite{Nevzorova2019}}
            \scnitem{\scncite{Cimiano2013}}
            \scnitem{\scncite{Bouayad2014}}
            \scnitem{\scncite{Saha2016}}
            \scnitem{\scncite{Shamsfard2004}}
            \scnitem{\scncite{Bateman2010}}
            \scnitem{\scncite{Moens1987}}
            \scnitem{\scncite{Dobrov2018}}
            \scnitem{\scncite{WordNet}}
            \scnitem{\scncite{VerbNet}}
            \scnitem{\scncite{FrameNet}}
            \scnitem{\scncite{Pease2010}}
            \scnitem{\scncite{Matsukawa1991}}
            \scnitem{\scncite{Calzolari1991}}
            \scnitem{\scncite{Buitelaar2006a}}
            \scnitem{\scncite{Cimiano2007}}
            \scnitem{\scncite{Buitelaar2006b}}
            \scnitem{\scncite{McCrae2012}}
            \scnitem{\scncite{SemanticWeb}}
            \scnitem{\scncite{NLTK}}
            \scnitem{\scncite{Spacy}}
            \scnitem{\scncite{Erekhinskaya2020}}
            \scnitem{\scncite{Standart2021}}
            \scnitem{\scncite{SILGlossary}}
            \scnitem{\scncite{Adger2003}}
            \scnitem{\scncite{Jackendoff1977}}
            \scnitem{\scncite{Haegeman1994}}
            \scnitem{\scncite{Carnie2012}}
            \scnitem{\scncite{Heim1998}}
            \scnitem{\scncite{Winter2016}}
            \scnitem{\scncite{Portner2008}}
        \end{scnrelfromlist}

        \begin{scnrelfromvector}{введение}
            \scnfileitem{В настоящее время научные исследования в области искусственного интеллекта развиваются по большому спектру различных направлений, однако между ними отсутствует согласованность систем понятий и, как следствие этого, совместимость разрабатываемых систем.}
            \begin{scnindent}
                \begin{scnrelfromset}{источник}
                    \scnitem{\scncite{Golenkov2021}}
                \end{scnrelfromset}
            \end{scnindent}
            \scnfileitem{Так в области создания программного обеспечения в силу его значительной сложности остро стоит проблема обеспечения интероперабельности различных программных сущностей, а также переиспользования результатов предыдущих аналогичных разработок.}
            \scnfileitem{Одним из путей решения данных проблем является создание \textit{онтологий} программного обеспечения, к которым предъявляются следующие требования:
            \begin{itemize}
                \item путем спецификация формальной семантики для избежания двусмысленных определений, а также нежелательных интерпретаций с целью обеспечения интероперабельности;
                \item созданные \textit{онтологии} должны быть применены в иной или же более широкой предметной области, что позволило бы избежать дорогостоящей специальной разработки и может повысить качество конечного продукта;
                \item возможность применения на их базе механизмов логического вывода.
            \end{itemize}}
            \begin{scnindent}
                \begin{scnrelfromset}{источник}
                    \scnitem{\scncite{Pileggi2018}}
                \end{scnrelfromset}
                \begin{scnrelfromset}{смотрите}
                    \scnitem{Предметная область и онтология онтологий}
                \end{scnrelfromset}
            \end{scnindent}
            \scnfileitem{В качестве примера \textit{онтологии} в данной области можно привести COPS. Целью данной \textit{онтологии} являлась формализация общих понятий из области программного обеспечения с целью упрощения его разработки и использования.}
            \begin{scnindent}
                \begin{scnrelfromset}{источник}
                    \scnitem{\scncite{Lando2007}}
                \end{scnrelfromset}
            \end{scnindent}
            \scnfileitem{Проблема совместимости результатов исследований также остро стоит и в лингвистике --- науке, в которой существует множество различных теорий, часто несовместимых друг с другом.
                В лингвистических исследованиях используются разные варианты разметки данных, нет одного подхода к структуризации корпусов текстов и различаются способы представления данных в них.}
            \begin{scnindent}
                \begin{scnrelfromset}{источник}
                    \scnitem{\scncite{Farrar2002}}
                    \scnitem{\scncite{Chiarcos2012}}
                \end{scnrelfromset}
            \end{scnindent}
            \scnfileitem{В качестве решения проблемы несовместимости различных способов описания данных в лингвистике предлагались варианты стандартизации форматов такого описания.
                Примером могут служить \textit{Text Encoding Initiative} --- консорциум по стандартизации представления текстов в цифровом виде и гайдлайны экспертной группы по стандартизации представления языковых данных \textit{EAGLES} (например, рекомендации по разметке корпусов текстов).
                Однако ни один из таких стандартов не получил распространения и не стал использоваться лингвистами повсеместно.}
            \begin{scnindent}
                \begin{scnrelfromset}{источник}
                    \scnitem{\scncite{EAGLES2022}}
                    \scnitem{\scncite{Text2022}}
                    \scnitem{\cite{Ide2010}/с.4}
                    \begin{scnindent}
                        \scnrelto{часть}{\scncite{Ide2010}}
                    \end{scnindent}
                \end{scnrelfromset}
            \end{scnindent}
            \scnfileitem{Вместо создания рекомендаций по разметке языкового материала в качестве более эффективного средства решения указанных выше проблем предлагается создание \textit{онтологий}.
                Помимо того, что \textit{онтология верхнего уровня} для предметной области языкознания может служить связующим звеном между различными лингвистическими теориями, она также представляет собой формализованное описание лингвистических концептов, представленное в удобном для компьютеров формате, что обусловливает ее применимость в системах, способных понимать аннотированные языковые данные, совершать интеллектуальный поиск по корпусам текстов, а также потенциально выполнять анализ существующих лингвистических исследований.}
            \begin{scnindent}
                \begin{scnrelfromset}{источник}
                    \scnitem{\scncite{Mccrae2015}}
                    \scnitem{\scncite{Schalley2019}}
                    \scnitem{\scncite{Farrar2002}}
                \end{scnrelfromset}
            \end{scnindent}
            \scnfileitem{В качестве такой \textit{онтологии} в предметной области лингвистики выступает \textit{The General Ontology of Linguistic Description} (\textit{GOLD}).
                В этой \textit{онтологии} формализованы наиболее базовые категории и отношения, используемые в лингвистике, а сама онтология интегрирована с онтологией верхнего уровня \textit{Suggested Upper Merged Ontology} (\textit{SUMO}).
                Авторы \textit{GOLD} пишут, что создавали онтологию в первую очередь для того, чтобы решить проблему интероперабельности данных лингвистической типологии и для того, чтобы с ее помощью экспертные системы могли обрабатывать научные данные по естественным языкам --- то есть целью создателей \textit{онтологии} не являлось непосредственно решение задач из области обработки текстов на естественном языке.}
            \begin{scnindent}
                \begin{scnrelfromset}{источник}
                    \scnitem{\scncite{GOLD2022}}
                    \scnitem{\scncite{Pease2002}}
                    \scnitem{\cite{Farrar2003}/с.4}
                    \begin{scnindent}
                        \scnrelto{часть}{\scncite{Farrar2003}}
                    \end{scnindent}
                \end{scnrelfromset}
            \end{scnindent}
            \scnfileitem{\textit{онтологией} естественных языков, нацеленной непосредственно на использование при решении задач по обработке естественного языка, является \textit{Ontologies of Linguistic Annotation} (\textit{OLiA}).
                Основной идеей \textit{онтологии} является обеспечение совместимости разметки языковых данных, полученных в результате выполненного компьютерными системами анализа текстов на \textit{естественном языке} с соответствующими им лингвистическими концептами из \textit{онтологии} --- в отличие от других лингвистических \textit{онтологий}, \textit{OLiA} предоставляет не только инвентарь концептов и отношений, но и необходимую спецификацию интеграции этих элементов с разметкой языковых данных (например, в корпусах)}
            \begin{scnindent}
                \begin{scnrelfromset}{источник}
                    \scnitem{\scncite{Chiarcos2012a}}
                \end{scnrelfromset}
            \end{scnindent}
            \scnfileitem{При создании \textit{онтологий} \textit{естественного языка}, встает вопрос о статусе спецификации лингвистической информации в таких онтологиях.}
            \scnfileitem{Дж. Бейтман выделяет три типа онтологий в зависимости от интегрированности в них естественно-языковой информации:
            \begin{itemize}
                \item \textit{онтологии}, представляющие собой абстрактную семантико-концептуальную репрезентацию знаний о мире, которая используется непосредственно в качестве \textit{денотационной семантики} для \textit{синтаксиса} и \textit{лексики} естественного языка;
                \item \textit{онтологии}, в которых есть отдельная спецификация \textit{денотационной семантики} \textit{естественного языка}, которая служит интерфейсом между \textit{синтаксисом} естественных \textit{языков} и концептуальной онтологией;
                \item \textit{онтологии}, представляющие собой абстрактную спецификацию \textit{знаний} о реальном мире вне зависимости от ограничений \textit{естественного языка}.
            \end{itemize}}
            \begin{scnindent}
                \begin{scnrelfromset}{источник}
                    \scnitem{\scncite{Bateman1997}}
                \end{scnrelfromset}
            \end{scnindent}
            \scnfileitem{Популярность в сфере обработки естественного языка приобрел второй тип \textit{онтологии}, так как он, в отличие от третьего подхода, который совсем не формализует лингвистическую информацию, позволяет специфицировать больше информации о естественных языках.
                Так, одна из самых популярных онтологий, используемых в системах для обработки естественного языка, the \textit{Generalized Upper Model}, является онтологией второго типа.
                П. Буителар и другие подчеркивают, что всем \textit{формальным онтологиям} необходима связь с языковой информацией для решения таких задач как выделение информации из текстов \textit{естественного языка}, автоматизированное заполнение \textit{онтологий} и генерации текста на \textit{естественном языке}.}
            \begin{scnindent}
                \begin{scnrelfromset}{источник}
                    \scnitem{\scncite{Bateman1997}}
                    \scnitem{\scncite{Bateman2002}}
                    \scnitem{\scncite{Buitelaar2009}}
                \end{scnrelfromset}
            \end{scnindent}
            \scnfileitem{Так как использование онтологий в обработке \textit{естественного языка} позволяет задать семантику получаемым в результате обработки \textit{естественного языка} данным и потенциально повысить качество анализа, начинается переход к созданию движимых \textit{онтологиями} систем обработки \textit{естественных языков}. \textit{онтологии} \textit{естественного языка} активно применяются для генерации текстов \textit{естественного языка} на основе некоторой \textit{онтологии} предметной области.}
            \begin{scnindent}
                \begin{scnrelfromset}{источник}
                    \scnitem{\scncite{Kostareva2016}}
                    \scnitem{\scncite{Nevzorova2019}}
                    \scnitem{\scncite{Bouayad2014}}
                    \scnitem{\scncite{Cimiano2013}}
                \end{scnrelfromset}
            \end{scnindent}
            \scnfileitem{Онтологический подход также используется в системах естественно-языковых запросов для баз данных, в которых запрос на \textit{естественном языке} транслируется в \textit{язык} запросов по \textit{онтологиям} конкретных \textit{предметных областей}, конструкции которого затем транслируются в SQL для обеспечения взаимодействия с реляционными базами данных.}
            \begin{scnindent}
                \begin{scnrelfromset}{источник}
                    \scnitem{\scncite{Saha2016}}
                \end{scnrelfromset}
            \end{scnindent}
            \scnfileitem{Кроме того, спецификация лингвистической информации в виде \textit{онтологий} помогает решать задачу автоматизированного создания \textit{онтологий} на основе естественно-языковых текстов.}
            \begin{scnindent}
                \begin{scnrelfromset}{источник}
                    \scnitem{\scncite{Shamsfard2004}}
                \end{scnrelfromset}
            \end{scnindent}
            \scnfileitem{Создаются \textit{онтологии} частных областей лингвистики: например, онтология пространственных выражений в естественных языках, онтология темпоральных сущностей на основе естественного языка, \textit{онтологии} конкретных естественных языков.}
            \scnfileitem{При использовании \textit{онтологий} для обработки естественного языка необходимо \scnqq{связать} концепты из \textit{онтологии} с лексикой конкретного \textit{естественного языка}.}
            \scnfileitem{Для этого создаются различные расширения существующих языковых баз данных, таких как \textit{WordNet}, \textit{VerbNet} и \textit{FrameNet}, направленные на их использование совместно с \textit{онтологиями верхнего уровня}.}
            \scnfileitem{Активные разработки идут в сфере создания онтологий словарного состава \textit{естественных языков}, в результате которых появилось множество формализованных описаний лексики.}
            \scnfileitem{Так как распространенные базы данных лексики естественного языка не являются \textit{онтологиями} и не имеют достаточной степени формализации (например, \textit{WordNet}), создаются \textit{онтологии}, являющиеся своего рода \scnqq{надстройкой} над такими базами данных, самой известной из которых является \textit{lemon}.}
            \begin{scnindent}
                \begin{scnrelfromset}{источник}
                    \scnitem{\scncite{Dobrov2018}}
                    \scnitem{\scncite{Moens1987}}
                    \scnitem{\scncite{Bateman2010}}
                    \scnitem{\scncite{FrameNet}}
                    \scnitem{\scncite{VerbNet}}
                    \scnitem{\scncite{WordNet}}
                    \scnitem{\scncite{Pease2010}}
                    \scnitem{\scncite{Matsukawa1991}}
                    \scnitem{\scncite{Calzolari1991}}
                    \scnitem{\scncite{Buitelaar2006a}}
                    \scnitem{\scncite{Cimiano2007}}
                    \scnitem{\scncite{Buitelaar2006b}}
                    \scnitem{\scncite{McCrae2012}}
                \end{scnrelfromset}
            \end{scnindent}
            \scnfileitem{Многие из приведенных выше онтологий созданы с использованием технологии \textit{Semantic Web}, который является внешней технологией по отношению к существующим решениям для обработки \textit{естественных языков}, поэтому последним приходится обращаться к ней с помощью API и стандартизированных \textit{языков запросов} (в частности, \textit{SPARQL}).}
            \begin{scnindent}
                \begin{scnrelfromset}{источник}
                    \scnitem{\scncite{Bouayad2014}}
                    \scnitem{\scncite{SemanticWeb}}
                \end{scnrelfromset}
            \end{scnindent}
            \scnfileitem{Стоит отметить, что несмотря на активное развитие в направлении применения \textit{онтологий} для обработки \textit{естественного языка}, многие популярные библиотеки по обработке естественного языка (например, \textit{NLTK} и \textit{spaCy} в принципе не поддерживают использование \textit{онтологий}, а большинство инструментов для разметки естественно-языковых текстов используют обычно свой формат, что требует использования специфичных для таких инструментов парсеров и конвертеров, чтобы данные можно было применить при решении каких-либо задач.}
            \begin{scnindent}
                \begin{scnrelfromset}{источник}
                    \scnitem{\scncite{Spacy}}
                    \scnitem{\scncite{NLTK}}
                    \scnitem{\cite{Erekhinskaya2020}/с.3}
                    \begin{scnindent}
                        \scnrelto{часть}{\scncite{Erekhinskaya2020}}
                    \end{scnindent}
                \end{scnrelfromset}
            \end{scnindent}
            \scnfileitem{Таким образом, в настоящее время в данной области можно выделить следующие проблемы:
            \begin{itemize}
                \item Отсутствие унификации (стандартизации) приведенных выше решений приводит к существенным накладным расходам на их интеграцию и значительно усложняет построение различных систем с их использованием в силу большой трудоемкости их интеграции.
                \item Несмотря на то, что \textit{онтологии} потенциально способствуют решению широкого круга задач в сфере обработки естественного языка, большинство движимых \textit{онтологиями} систем по обработке естественного языка сконцентрированы на решении специализированных задач (например, только генерации текста, только заполнения \textit{онтологии} или только обеспечения поиска с помощью естественного языка).
                \item Создано довольно большое количество частных лингвистических \textit{онтологий}, формализующих, однако, лишь некий подраздел предметной области лингвистики (в особенности лексики), что отчасти вытекает из предыдущего пункта.
                \\В то же время, существующие лингвистические \textit{онтологии} верхнего уровня (например, \textit{OLiA}) все равно не до конца решают проблему унификации, так как им требуется вводить промежуточный уровень для интеграции полученных в результате анализа текста естественного языка данных с фрагментами \textit{онтологии}.
            \end{itemize}}
            \begin{scnindent}
                \begin{scnrelfromset}{источник}
                    \scnitem{\scncite{Standart2021}}
                    \scnitem{\scncite{Golenkov2021}}
                \end{scnrelfromset}
            \end{scnindent}
            \scnfileitem{Так как используемый в \textit{Технологии OSTIS} \textit{язык} --- \textit{SC-код} --- обладает достаточной экспрессивностью для описания знаний любого вида, а сама технология нацелена на создание интероперабельных интеллектуальных систем нового поколения, \textit{естественно-языковые интерфейсы} \textit{ostis-систем} смогут справляться с широким кругом задач по обработке текстов на естественных языках --- будь то синтез естественно-языковых текстов в целом, ведение диалога в диалоговых системах, поиск с использованием \textit{естественного языка}, выделение информации из текстов и тому подобное.
                При этом в то время как в текущем состоянии сферы обработки естественных языков данные классы задач выполняются зачастую специализированными средствами и требуют дополнительных затрат на обеспечение потенциальной совместимости с конкретными компьютерными системами, в рамках \textit{Технологии OSTIS} для их решения будет использоваться один универсальный \textit{язык смыслового представления знаний}, на котором будут написаны как компоненты решателя задач, так и онтология языков и конкретных предметных областей, что позволит решить проблему интероперабельности.}
            \begin{scnindent}
                \begin{scnrelfromset}{смотрите}
                    \scnitem{Предметная область и онтология внутреннего языка ostis-систем}
                \end{scnrelfromset}
            \end{scnindent}
            \scnfileitem{Более того, \textit{онтология} \textit{естественных языков}, разработанная в рамках такой технологии, могла бы быть использована не только для решения прикладных задач по обработке \textit{естественного языка}, но и для обеспечения интероперабельности данных, полученных в ходе лингвистических исследований, что было бы ценным вкладом в область теоретической лингвистики.}
            \scnfileitem{Наконец, \textit{онтологию} \textit{естественных языков} можно рассматривать в качестве подмножества \textit{онтологии} языков вообще (как естественных, так и искусственных и формальных), чего не делают рассмотренные выше существующие \textit{онтологии}.
                Это позволит концептуализировать \textit{естественный язык} в одной системе с языками программирования и более тесно связать используемые в соответствующих предметных областях понятия для более эффективного решения задач по обработке \textit{естественного языка} в интеллектуальных компьютерных системах.}
            \scnfileitem{Цель данной работы --- предложить базовые средства формального описания \textit{синтаксиса} и \textit{денотационной семантики} различных \textit{языков} в виде фрагмента \textit{онтологии} \textit{языков} и \textit{информационных конструкций}, который можно будет использовать при проектировании интеллектуальных компьютерных систем нового поколения.}
            \scnfileitem{Как уже говорилось выше, для использования достижений лингвистики при проектировании интеллектуальных компьютерных систем требуется представить полученные результаты в формальном виде.}
            \scnfileitem{Далее мы предложим формализацию основных лингвистических концептов, выполненную на формальном языке представления знаний --- \textit{SC-коде}.}
        \end{scnrelfromvector}
    
        \scnheader{язык}
        \begin{scnsubdividing}
            \scnitem{естественный язык}
            \begin{scnindent}
                \scntext{пояснение}{Естественный язык представляет собой язык, который не был создан целенаправленно}
            \end{scnindent}
            \scnitem{искусственный язык}
            \begin{scnindent}
                \scntext{пояснение}{Искусственный язык представляет собой язык, специально разработанный для достижения определённых целей}
                \scnhaselement{Эсперанто}
                \scnhaselement{Python}
                \scnsuperset{сконструированный язык}
                \begin{scnindent}
	                \scntext{пояснение}{Сконструированный язык представляет собой искусственный язык, предназначенный для общения людей}
	                \scnhaselement{Эсперанто}
	            \end{scnindent}
            \end{scnindent}
        \end{scnsubdividing}
        \scnsuperset{международный язык}
        \begin{scnindent}
	        \scntext{пояснение}{Международный язык представляет собой естественный или искусственный язык, использующийся для общения людей разных из стран}
	        \scnhaselement{Английский язык}
	        \scnhaselement{Русский язык}
	    \end{scnindent}
        
        \scnheader{плановый язык}
        \begin{scnreltoset}{пересечение}
            \scnitem{сконструированный язык}
            \scnitem{международный язык}
        \end{scnreltoset}
        
        \scnheader{язык общения}
        \begin{scnreltoset}{объединение}
            \scnitem{естественный язык}
            \scnitem{сконструированный язык}
        \end{scnreltoset}
        \scnhaselement{Английский язык}
        \scnhaselement{Русский язык}
        \scnhaselement{Эсперанто}
        \begin{scnreltoset}{объединение}
            \scnitem{корневой язык}
            \begin{scnindent}
                \scntext{пояснение}{Корневой язык представляет собой язык, для которого характерно полное отсутствие словоизменения и наличие грамматической значимости порядка слов, состоящих только из корня.}
                \scnhaselement{Английский язык}
            \end{scnindent}
            \scnitem{агглютинативный язык}
            \begin{scnindent}
                \scntext{пояснение}{Агглютинативный язык характеризуется развитой системой употребления суффиксов, приставок, добавляемых к неизменяемой основе слова, которые используются для выражения категорий числа, падежа, рода и др.}
                \scnhaselement{Английский язык}
            \end{scnindent}
            \scnitem{флективный язык}
            \begin{scnindent}
                \scntext{пояснение}{Для флективного языка характерно развитое употребление окончаний для выражения категорий рода, числа, падежа, сложная система склонения глаголов, чередование гласных в корне, а также строгое различение частей речи.}
                \scnhaselement{Русский язык}
            \end{scnindent}
            \scnitem{профлективный язык}
            \begin{scnindent}
                \scntext{пояснение}{Для профлективного языка характерны агглютинация (в случае именного словоизменения), флексия и чередование гласных (аблаут)(в случае глагольного словоизменения).}
            \end{scnindent}
        \end{scnreltoset}
        
        \scnheader{лексема}
        \scnsubset{файл}
        \scntext{определение}{\textbf{\textit{лексема}} --- минимальная единица \textit{языка}, имеющая семантическую интерпретацию и обозначающая концепт, отражающий взгляд на мир некоторого языкового сообщества}
        \begin{scnindent}
            \scnrelfrom{источник}{\cite{SILGlossary}}
        \end{scnindent}
        \scntext{пояснение}{\textit{Лексема} --- тайген или ёген конкретного естественного языка.}
        \begin{scnindent}
        	\scnrelfrom{источник}{\cite{Hardzei2005}}
        \end{scnindent}
        \scnrelfrom{пример}{{\scnfileimage[20em]{Contents/part_kb/src/images/sd_natural_languages/lexeme_example_a.png}}}
        \begin{scnindent}
            \scnidtf{SCg-текст. Иллюстрация к спецификации лексемы в базе знаний}
            \scntext{примечание}{Пример формализации \textit{отношений} в \textit{SCg-коде}.}
        \end{scnindent}

        \scnheader{номинативная единица}
        \scnsubset{файл}
        \scntext{определение}{\textit{номинативная единица} --- это устойчивая последовательность комбинторных вариантов лексем, в которой один вариант лексемы (модификатор) определяет другой (актуализатор).}
        \begin{scnindent}
        	\scnrelfrom{источник}{\cite{Hardzei2005}}
        \end{scnindent}
        \begin{scnrelfromset}{пример}
            \scnfileitem{записная книжка}
            \scnfileitem{бежать галопом}
        \end{scnrelfromset}
        
        \scnheader{комбинаторный вариант лексемы}
        \scnsubset{файл}
        \scntext{пояснение}{\textit{Комбинторный вариант лексемы} --- вариант лексемы в упорядоченном наборе её вариантов (парадигме).}
        \begin{scnindent}
        	\scnrelfrom{источник}{\cite{Hardzei2007}}
        \end{scnindent}
        
        \scnheader{морфологическая парадигма*}
        \scniselement{квазибинарное отношение}
        \scntext{определение}{\textbf{\textit{морфологическая парадигма*}} --- \textit{бинарное ориентированное отношение}, связывающее \textit{лексему} и множество ее \textit{словоформ}.}
        \scntext{пояснение}{\textit{Морфологическая парадигма*} --- квазибинарное отношение, связывающее лексему с её комбинторными вариантами.}
        \scnrelfrom{первый домен}{словоформа}
        \scnrelfrom{второй домен}{лексема}
        
        \scnheader{естественный язык}
        \begin{scnsubdividing}
            \scnitem{часть языка}
            \begin{scnindent}
                \begin{scnsubdividing}
                    \scnitem{тайген}
                    \scnitem{ёген}
                \end{scnsubdividing}
            \end{scnindent}
            \scnitem{знак алфавита синтаксиса}
            \begin{scnindent}
                \scntext{пояснение}{\textit{Знаки алфавита синтаксиса} --- вспомогательные средства синтаксиса (на макроуровне --- предлоги, послелоги, союзы, частицы и др., на микроуровне --- флексии, префиксы, постфиксы, инфиксы и др.), служащие для соединения составных частей языковых структур и образования морфологических парадигм.}
                \begin{scnindent}
                	\scnrelfrom{источник}{\cite{Hardzei2005}}
                \end{scnindent}
            \end{scnindent}
        \end{scnsubdividing}
        
        \scnheader{тайген}
        \scntext{определение}{\textit{тайген} --- часть языка, обозначающая индивида.}
        \begin{scnindent}
	        \scnrelfrom{источник}{\cite{Hardzei2006}}
	        \scnrelfrom{источник}{\cite{Hardzei2015}}
	    \end{scnindent}
        \begin{scnsubdividing}
            \scnitem{развёрнутый тайген}
            \begin{scnindent}
                \begin{scnsubdividing}
                    \scnitem{составной тайген}
                    \scnitem{сложный тайген}
                \end{scnsubdividing}
            \end{scnindent}
            \scnitem{свёрнутый тайген}
            \begin{scnindent}
                \begin{scnsubdividing}
                    \scnitem{сокращённый тайген}
                    \scnitem{сжатый тайген}
                    \begin{scnindent}
                        \begin{scnsubdividing}
                            \scnitem{информационный тайген}
                            \begin{scnindent}
                                \scntext{определение}{\textit{информационный тайген} --- тайген, обозначающий индивида в информационном фрагменте модели мира.}
                                \begin{scnindent}
	                                \scnrelfrom{источник}{\cite{Hardzei2006}}
	                                \scnrelfrom{источник}{\cite{Hardzei2015}}
	                            \end{scnindent}
                            \end{scnindent}
                            \scnitem{физический тайген}
                            \begin{scnindent}
                                \scntext{определение}{\textit{физический тайген} --- тайген, обозначающий индивида в физическом фрагменте модели мира.}
                                \begin{scnindent}
	                                \scnrelfrom{источник}{\cite{Hardzei2006}}
	                                \scnrelfrom{источник}{\cite{Hardzei2015}}
	                            \end{scnindent}
                            \begin{scnsubdividing}
                                 \scnitem{постоянный тайген}
                                 \begin{scnindent}
                                     \scntext{определение}{\textit{постоянный тайген} --- физический тайген, обозначающий постоянного индивида.}
                                     \begin{scnindent}
                                      \scnrelfrom{источник}{\cite{Hardzei2006}}
                                      \scnrelfrom{источник}{\cite{Hardzei2015}}
                                  \end{scnindent}
                                 \end{scnindent}
                                 \scnitem{переменный тайген}
                                 \begin{scnindent}
                                     \scntext{определение}{\textit{переменный тайген} --- физический тайген, обозначающий переменного индивида.}
                                     \begin{scnindent}
                                      \scnrelfrom{источник}{\cite{Hardzei2006}}
                                      \scnrelfrom{источник}{\cite{Hardzei2015}}
                                  \end{scnindent}
                                 \end{scnindent}
                            \end{scnsubdividing}
                            \begin{scnsubdividing}
                                 \scnitem{качественный тайген}
                                 \scnitem{количественный тайген}
                            \end{scnsubdividing}
                            \begin{scnsubdividing}
                                 \scnitem{одноместный тайген}
                                 \scnitem{многоместный тайген}
                                 \begin{scnindent}
                                     \scnsuperset{интенсивный тайген}
                                     \scnsuperset{экстенсивный тайген}
                                  \end{scnindent}
                            \end{scnsubdividing}
                        \end{scnindent}
                        \end{scnsubdividing}
                        
                    \end{scnindent}
                \end{scnsubdividing}
            \end{scnindent}
        \end{scnsubdividing}
        
        \scnheader{ёген}
        \scntext{определение}{\textit{ёген} --- часть языка, обозначающая признак индивида.}
        \begin{scnindent}
	        \scnrelfrom{источник}{\cite{Hardzei2006}}
	        \scnrelfrom{источник}{\cite{Hardzei2015}}
	    \end{scnindent}
        \begin{scnsubdividing}
            \scnitem{развёрнутый ёген}
            \begin{scnindent}
                \begin{scnsubdividing}
                    \scnitem{составной ёген}
                    \scnitem{сложный ёген}
                \end{scnsubdividing}
            \end{scnindent}
            \scnitem{свёрнутый ёген}
            \begin{scnindent}
                \begin{scnsubdividing}
                    \scnitem{сокращённый ёген}
                    \begin{scnindent}
                        \begin{scnsubdividing}
                            \scnitem{информационный ёген}
                            \begin{scnindent}
                                \scntext{определение}{\textit{информационный ёген} --- еген, обозначающий признак индивида в информационном фрагменте модели мира.}
                                \begin{scnindent}
	                                \scnrelfrom{источник}{\cite{Hardzei2006}}
	                                \scnrelfrom{источник}{\cite{Hardzei2007a}}
	                            \end{scnindent}
                            \end{scnindent}
                            \scnitem{физический ёген}
                            \begin{scnindent}
                                \scntext{определение}{\textit{физический ёген} --- еген, обозначающий признак индивида в физическом фрагменте модели мира.}
                                \begin{scnindent}
	                                \scnrelfrom{источник}{\cite{Hardzei2006}}
	                                \scnrelfrom{источник}{\cite{Hardzei2007a}}
	                            \end{scnindent}
                                \begin{scnsubdividing}
                                    \scnitem{постоянный ёген}
                                    \begin{scnindent}
                                        \scntext{определение}{\textit{постоянный ёген} --- физический ёген, обозначающий постоянный признак индивида.}
                                        \begin{scnindent}
	                                        \scnrelfrom{источник}{\cite{Hardzei2006}}
	                                        \scnrelfrom{источник}{\cite{Hardzei2007a}}
	                                    \end{scnindent}
                                    \end{scnindent}
                                    \scnitem{переменный ёген}
                                    \begin{scnindent}
                                        \scntext{определение}{\textit{переменный ёген} --- физический ёген, обозначающий переменный признак индивида.}
                                        \begin{scnindent}
	                                        \scnrelfrom{источник}{\cite{Hardzei2006}}
	                                        \scnrelfrom{источник}{\cite{Hardzei2007a}}
	                                    \end{scnindent}
                                    \end{scnindent}
                                \end{scnsubdividing}
                                \begin{scnsubdividing}
                                    \scnitem{качественный ёген}
                                    \scnitem{количественный ёген}
                                \end{scnsubdividing}
                                \begin{scnsubdividing}
                                    \scnitem{одноместный ёген}
                                    \scnitem{многоместный ёген}
                                    \begin{scnindent}
                                        \begin{scnsubdividing}
                                            \scnitem{интенсивный ёген}
                                            \scnitem{экстенсивный ёген}
                                        \end{scnsubdividing}
                                    \end{scnindent}
                                \end{scnsubdividing}
                            \end{scnindent}
                        \end{scnsubdividing}
                    \end{scnindent}
                    \scnitem{сжатый ёген}
                \end{scnsubdividing}
            \end{scnindent}
        \end{scnsubdividing}
        
        \scnheader{член предложения\scnrolesign}
        \scniselement{ролевое отношение}
        \scntext{определение}{\textit{член предложения\scnrolesign} --- это отношение, связывающее декомпозицию текста с файлом, содержимое которого (часть языка) играет в декомпозируемом тексте определенную синтаксическую роль.}
        \begin{scnindent}
        	\scnrelfrom{источник}{\cite{Hardzei2005}}
        \end{scnindent}
        \begin{scnsubdividing}
            \scnitem{главный член предложения\scnrolesign}
            \begin{scnindent}
                \begin{scnsubdividing}
                    \scnitem{подлежащее\scnrolesign}
                    \begin{scnindent}
                        \scntext{определение}{\textit{подлежащее\scnrolesign} --- это одно из главных ролевых отношений, связывающее декомпозицию текста с файлом, содержимое которого обозначает исходный пункт описания события, выбранный наблюдателем.}
                        \begin{scnindent}
                        	\scnrelfrom{источник}{\cite{Hardzei2020}}
                        \end{scnindent}
                    \end{scnindent}
                    \scnitem{сказуемое\scnrolesign}
                    \begin{scnindent}
                        \scntext{определение}{\textit{сказуемое\scnrolesign} --- это одно из главных ролевых отношений, связывающее декомпозицию текста с файлом, содержимое которого обозначает отображение наблюдателем исходного пункта описания события в конечный.}
                        \begin{scnindent}
                        	\scnrelfrom{источник}{\cite{Hardzei2020}}
                        \end{scnindent}
                    \end{scnindent}
                    \scnitem{прямое дополнение\scnrolesign}
                    \begin{scnindent}
                        \scntext{определение}{\textit{прямое дополнение\scnrolesign} --- это одно из главных ролевых отношений, связывающее декомпозицию текста с файлом, содержимое которого обозначает конечный пункт описания события, выбранный наблюдателем.}
                        \begin{scnindent}
                        	\scnrelfrom{источник}{\cite{Hardzei2020}}
                        \end{scnindent}
                    \end{scnindent}
                \end{scnsubdividing}
            \end{scnindent}
            \scnitem{второстепенный член предложения\scnrolesign}
            \begin{scnindent}
                \begin{scnsubdividing}
                    \scnitem{косвенное дополнение\scnrolesign}
                    \scnitem{определение\scnrolesign}
                    \begin{scnindent}
                        \scntext{определение}{\textit{определение\scnrolesign} --- это одно из второстепенных ролевых отношений, связывающее декомпозицию текста с файлом, содержимое которого обозначает модификацию подлежащего, дополнения, обстоятельства места и времени.}
                        \begin{scnindent}
	                        \scnrelfrom{источник}{\cite{Hardzei2007a}}
	                        \scnrelfrom{источник}{\cite{Hardzei2017a}}
	                        \scnrelfrom{источник}{\cite{Hardzei2007b}}
	                    \end{scnindent}
                    \end{scnindent}
                    \scnitem{обстоятельство\scnrolesign}
                    \begin{scnindent}
                        \scntext{определение}{\textit{обстоятельство\scnrolesign} --- это одно из второстепенных ролевых отношений, связывающее декомпозицию текста с файлом, содержимое которого обозначает либо модификацию, либо локализацию сказуемого.}
                        \begin{scnindent}
	                        \scnrelfrom{источник}{\cite{Hardzei2007a}}
	                        \scnrelfrom{источник}{\cite{Hardzei2017a}}
	                        \scnrelfrom{источник}{\cite{Hardzei2007b}}
	                        \begin{scnsubdividing}
	                            \scnitem{обстоятельство степени\scnrolesign}
	                            \begin{scnindent}
	                                \scntext{определение}{обстоятельство степени --- обстоятельство, обозначающее модификацию сказуемого.}
	                            \end{scnindent}
	                            \scnitem{обстоятельство образа действия\scnrolesign}
	                            \begin{scnindent}
	                                \scntext{определение}{обстоятельство образа действия --- обстоятельство, обозначающее модификацию сказуемого.}
	                            \end{scnindent}
	                            \scnitem{обстоятельство места\scnrolesign}
	                            \begin{scnindent}
	                                \scntext{определение}{обстоятельства места --- обстоятельство, обозначающее пространственную локализацию сказуемого.}
	                                \begin{scnindent}
		                                \begin{scnsubdividing}
		                                    \scnitem{динамическое обстоятельство места\scnrolesign}
		                                    \scnitem{статическое обстоятельство места\scnrolesign}
		                                \end{scnsubdividing}
	                                \end{scnindent}
	                            \end{scnindent}
	                            \scnitem{обстоятельство времени\scnrolesign}
	                            \begin{scnindent}
	                                \scntext{определение}{обстоятельство времени --- обстоятельство, обозначающее временную локализацию сказуемого.}
		                            \begin{scnsubdividing}
		                                \scnitem{динамическое обстоятельство времени\scnrolesign}
		                                \scnitem{статическое обстоятельство времени\scnrolesign}
		                            \end{scnsubdividing}
	                            \end{scnindent}
	                        \end{scnsubdividing}
	                    \end{scnindent}
                    \end{scnindent}
                \end{scnsubdividing}
            \end{scnindent}
        \end{scnsubdividing}
        
        \scnheader{Пример sc.g-текста, описывающего лексему}
        \scniselement{sc.g-текст}
        \scntext{пояснение}{Здесь представлено описание лексемы с указанием ее принадлежности определённой части речи. Также описание содержит морфологическую парадигму данной лексемы, связывающую ее с ее словоформами.}
        \scneq{\scnfileimage[20em]{Contents/part_kb/src/images/sd_natural_languages/lexeme_example.png}}
        
        \scnheader{Пример этапов разбора текста естественного языка}
        \begin{scneqtoset}
            \scnfileitem{\includegraphics{Contents/part_kb/src/images/sd_natural_languages/nl_text.png}}
            \begin{scnindent}
            	\scntext{пояснение}{с точки зрения ostis-системы, любой естественно-языкой текст является \textit{файлом.}}
	            \scnrelfrom{лексическая структура}{\scnfileimage[20em]{Contents/part_kb/src/images/sd_natural_languages/nl_lexical.png}}
			    \begin{scnindent}
		            \scntext{пояснение}{Данная конструкция описывает декомпозицию исходного текста на фрагменты с указанием их принадлежности определённой \textit{номинативной единице} или \textit{знаку алфавита синтаксиса}.}
		            \scnrelfrom{синтаксическая структура}{\scnfileimage[20em]{Contents/part_kb/src/images/sd_natural_languages/nl_synactical.png}}
		            \scntext{пояснение}{Здесь приведена только частью синтаксической структуры. Оставшаяся часть записывается аналогично.}
		         \end{scnindent}
        	\end{scnindent}
        \end{scneqtoset}
    \end{scnsubstruct}
    \bigskip
    \scnendcurrentsectioncomment
\end{SCn}
