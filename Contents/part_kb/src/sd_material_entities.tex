\begin{SCn}
	\scnsectionheader{Предметная область и онтология материальных сущностей}
	\begin{scnsubstruct}
	\scntext{аннотация}{В данной предметной области рассмотрена структура онтологии предметной области материальных сущностей.}
	\begin{scnrelfromlist}{максимальный класс исследования}
		\scnitem{пространственная сущность}
		\scnitem{материальная сущность}
		\scnitem{вещество}
		\scnitem{физическое поле}
		\scnitem{персона}
		\scnitem{юридическое лицо}
		\scnitem{предприятие}
		\scnitem{географический объект}
	\end{scnrelfromlist}
	
	\scnheader{материальная сущность}
	\begin{scnrelfromlist}{связь}
		\scnitem{предметная область пространственных сущностей}
		\scnitem{предметная область ситуаций и событий}
		\scnitem{предметная область временных сущностей}
	\end{scnrelfromlist}
	
	\scnheader{материальная сущность}
	\scnidtf{философский термин, используемый для обозначения физического или конкретного объекта в реальном мире}
	\scnidtf{сущность, относящаяся к представлению или моделированию реального объекта в цифровой форме}
	\scnidtf{сущность, относящаяся к представлению или моделированию реального объекта в sc-памяти}
	\begin{scnrelfromlist}{включает}
		\scnitem{материал объекта}
		\scnitem{свойство объекта}
		\scnitem{структура объекта}
		\scnitem{процесс}
	\end{scnrelfromlist}	
	
	\scnheader{материальная сущность}
	\begin{scnrelfromlist}{пояснение}
		\scnidtf{Каждой материальной сущности можно поставить в соответствие различные \textbf{процессы}, описывающие ее преобразование из одного состояния в другое.}
	\end{scnrelfromlist}

	\scnheader{основные отношения между материальными сущностями}
	\begin{scnrelfromlist}{разбиение}
		\scnitem{является частью*}
		\begin{scnindent}
			\scnidtf{отношение указывает на то, что один объект является частью другого объекта}
		\end{scnindent}
		\scnitem{состоит из*}
		\begin{scnindent}
			\scnidtf{отношение указывает на то, из каких материалов или компонентов состоит объект}
		\end{scnindent}
		\scnitem{находится в*}
		\begin{scnindent}
			\scnidtf{ отношение указывает на местоположение объекта относительно другого объекта}
		\end{scnindent}
		\scnitem{связано с*}
		\begin{scnindent}
			\scnidtf{отношение указывает на любую связь между двумя объектами, которая не покрывается другими отношениями}
		\end{scnindent}
		\scnitem{имеет свойство*}
		\begin{scnindent}
			\scnidtf{отношение указывает на то, что объект обладает определенным свойством}
		\end{scnindent}
		\scnitem{используется в*}
		\begin{scnindent}
			\scnidtf{отношение указывает на то, в каких процессах или приложениях используется объект}
		\end{scnindent}
	\end{scnrelfromlist}
	\end{scnsubstruct}
	\scnendcurrentsectioncomment
\end{SCn}
