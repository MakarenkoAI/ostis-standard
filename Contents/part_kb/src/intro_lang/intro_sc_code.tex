\begin{SCn}
    \scnsectionheader{Предметная область и онтология внутреннего языка ostis-систем}
    \begin{scnsubstruct}
       	\begin{scnrelfromset}{автор}
       		\scnitem{Голенков В.В.}
       		\scnitem{Ивашенко В.П.}
       	\end{scnrelfromset}
        \begin{scnreltovector}{конкатенация сегментов}
            \scnitem{Основные положения внутреннего языка ostis-систем}
            \scnitem{Описание Ядра SC-кода}
            \scnitem{SC-код как синтаксическое расширение Ядра SC-кода}
            \scnitem{Использование SC-кода для формального описания собственного синтаксиса}
            \scnitem{Уточнение смысла выделенных классов sc-элементов и связей между этими классами}
            \scnitem{Структура базовой семантической спецификации sc-элемента}
            \scnitem{Онтологическая формализация Базовой денотационной семантики SC-кода}
			\scnitem{Смысловое пространство ostis-систем}
        \end{scnreltovector}
        \begin{scnrelfromlist}{ключевое понятие}
           	\scnitem{sc-элемент}
           	\scnitem{обозначение sc-множества}
           	\scnitem{обозначение sc-связки}
           	\scnitem{обозначение sc-класса}		
           	\scnitem{обозначение sc-структуры}		
           	\scnitem{обозначение внешней сущности}			
           	\scnitem{sc-константа}
           	\scnitem{sc-переменная}				
           	\scnitem{sc-множество}
           	\scnitem{sc-связка}
           	\scnitem{sc-класс}						
           	\scnitem{sc-структура}						
           	\scnitem{внешняя сущность}			
        \end{scnrelfromlist}
        \begin{scnrelfromlist}{ключевое знание}
           	\scnitem{Базовая денотационная семантика SC-кода}
        \end{scnrelfromlist}
        \begin{scnrelfromlist}{библиографическая ссылка}
           	\scnitem{\scncite{Narinjani2000}}
           	\scnitem{\scncite{Ivashenko2022}}
           	\scnitem{\scncite{Ivashenko2020String}}
           	\scnitem{\scncite{Collatz1966}}
           	\scnitem{\scncite{Ivashenko2017}}
           	\scnitem{\scncite{Ivashenko2014}}
           	\scnitem{\scncite{Bohm1993}}
           	\scnitem{\scncite{Bohm2002}}
           	\scnitem{\scncite{Nalimov1995}}
           	\scnitem{\scncite{Nalimov1989}}
           	\scnitem{\scncite{Nalimov1979}}
           	\scnitem{\scncite{Martynov2004}}
           	\scnitem{\scncite{Manin2016}}
           	\scnitem{\scncite{Melchuk2016}}
           	\scnitem{\scncite{Harris1992}}
           	\scnitem{\scncite{Alt1995}}
           	\scnitem{\scncite{Study1905}}
           	\scnitem{\scncite{Kostrikin1997}}
           	\scnitem{\scncite{Lowe2001}}
           	\scnitem{\scncite{Manin2014}}
           	\scnitem{\scncite{Martynov2009}}
           	\scnitem{\scncite{Gordey2014}}
        \end{scnrelfromlist}
        \begin{scnrelfromvector}{введение}
           	\scnfileitem{Поскольку все элементы \textit{информационных конструкций} являются обозначениями описываемых сущностей и, в том числе, обозначениями различных выделяемых классов \textit{sc-элементов}, можно явно ввести различные семантически значимые и синтаксически выделяемые классы \textit{sc-элементов} и на основе этого явно описать средствами \textit{SC-кода} \textit{Базовую денотационную семантику} и \textit{Синтаксис SC-кода}.}
           	\begin{scnindent}
           		\begin{scnrelfromset}{смотрите}
           			\scnitem{Денотационная семантика Ядра SC-кода}
           			\scnitem{Синтаксис Ядра SC-кода}
           		\end{scnrelfromset}
           	\end{scnindent}
           	\scnfileitem{\textit{Синтаксис SC-кода} задается семейством классов синтаксический выделяемых \textit{sc-элементов}. Элементы, принадлежащие каждому синтаксически выделяемому классу \textit{sc-элементов} должны иметь одинаковые синтаксические признаки (синтаксические метки). При этом очевидно, что \textit{Синтаксис SC-кода} существенно упростится, если синтаксически выделяемые классы sс-элементов будут одновременно иметь и четкую семантическую интерпретацию.}
           	\scnfileitem{Таким образом, формализацию \textit{Синтаксиса SC-кода} целесообразно осуществлять после формализации \textit{Базовой денотационной семантики SC-кода}. Путем синтаксического выделения тех семантически выделенных классов \textit{sc-элементов}, которые:
           		\begin{scnitemize} 
           			\item Во-первых, необходимы для кодирования sc-конструкций в памяти ostis-систем (в sc-памяти).
           			\item Во-вторых, обеспечивают максимально возможное упрощение обработки sc-конструкций (например, упрощение анализа часто проверяемых семантических характеристик обрабатываемых \textit{sc-элементов}).
           		\end{scnitemize}}
        \end{scnrelfromvector}
        
        \scnsegmentheader{Основные положения внутреннего языка ostis-систем}
\begin{scnsubstruct}
    \scnheader{SC-код}
    \scnidtf{Язык унифицированного смыслового представления знаний в памяти \textit{интеллектуальных компьютерных систем}}
    \scnidtf{Внутренний язык \textit{ostis-систем}}
    \scnrelto{внутренний язык}{ostis-система}
    \scntext{эпиграф}{Информация содержится не в самих знаках, а в конфигурации связей между ними.}
    \scntext{эпиграф}{Он вскочил на коня и поскакал во все стороны.}
    \scntext{основной внешний идентификатор sc-элемента}{\textbf{SC-код}}
    \scniselement{имя собственное}
    \scntext{часто используемый неосновной внешний идентификатор sc-элемента}{sc-текст}
    \scniselement{имя нарицательное}
    \scniselement{абстрактный язык}
    \scniselement{графовый язык}
    \scnidtf{Универсальный язык, обеспечивающий внутреннее представление и структуризацию \uline{всех}(!), используемых ostis-системой в процессе своего функционирования.}
    \scnidtf{Универсальный язык, являющийся результатом унификации (уточнения) синтаксиса и денотационной семантики семантических сетей.}
    \scntext{пояснение}{Универсальность SC-кода обеспечивается и тем, что элементами текстов SC-кода могут быть знаки описываемых сущностей \uline{любого} вида, в том числе, и  знаки связей между описываемыми сущностями и/или их знаками.}
    \scntext{следствие}{Тексты SC-кода являются графовыми структурами расширенного вида, в которых знаки описываемых связей могут соединять не только вершины (узлы) графовой структуры, но и знаки других связей.}
    \scnidtf{Базовый универсальный язык внутреннего представления знаний в памяти ostis-систем.}
    \scnidtf{Базовый внутренний язык ostis-систем.}
    \scnidtf{Максимальный внутренний язык ostis-систем, по отношению к которому все остальные (специализированные) внутренние языки являются его подъязыками (подмножествами)}
    \scnidtf{Множество всевозможных текстов SC-кода}
    \scniselement{имя собственное}
    \scnidtf{текст SC-кода}
    \scniselement{имя нарицательное}
    \begin{scnrelfromvector}{принципы, лежащие в основе}
        \scnfileitem{\textit{Знаки} (обозначения) всех \textit{сущностей}, описываемых в \textit{sc-текстах} (текстах \textit{\textbf{SC-кода}}) представляются в виде синтаксически элементарных (атомарных) фрагментов \textit{sc-текстов} и, следовательно, не имеющих внутренней структуры, не состоящих из более простых фрагментов \textit{текста}, как, например, имена (термины), которые представляют \textit{знаки} описываемых \textit{сущностей} в привычных \textit{языках} и состоят из \textit{букв}.}
        \scnfileitem{\textit{Имена} (термины), \textit{естественно-языковые тексты} и другие информационные конструкции, не являющиеся \textit{sc-текстами}, могут входить в состав \textit{sc-текста}, но только как \textit{файлы}, описываемые (специфицируемые) \textit{sc-текстами}. Таким образом, в состав базы знаний \textit{интеллектуальной компьютерной системы}, построенной на основе \textit{\textbf{SC-кода}}, могут входить \textit{имена} (термины), обозначающие некоторые описываемые \textit{сущности} и представленные соответствующими \textit{файлами}. Каждый \mbox{\textit{sc-элемент}} будем называть внутренним обозначением некоторой \textit{сущности}, а \textit{имя} этой \textit{сущности}, представленное соответствующим файлом, будем называть \textit{внешним идентификатором} (внешним обозначением) этой \textit{сущности}. При этом каждый именуемый (идентифицируемый) \textbf{\textit{sc-элемент}} связывается дугой, принадлежащей отношению \scnqqi{быть \textit{\textbf{внешним идентификатором*}}}, с \textit{узлом}, содержимым которого является \textit{файл} идентификатора (в частности, \textit{имени}), обозначающего ту же \textit{сущность}, что и указанный выше \textit{sc-элемент}. \textit{Внешним идентификатором} может быть не только \textit{имя} (термин), но и иероглиф, пиктограмма, озвученное имя, жест.
        	\\Особо отметим, что \textit{внешние идентификаторы} описываемых \textit{сущностей} в \textit{интеллектуальной компьютерной системе}, построенной на основе \textit{\textbf{SC-кода}}, используются только (1) для анализа информации, поступающей в эту систему из вне из различных источников, и ввода (понимания и погружения) этой информации в \textit{базу знаний}, а также (2) для синтеза различных \textit{сообщений}, адресуемых различным субъектам (в т.ч. пользователям).}
        \scnfileitem{Тексты \textit{\textbf{SC-кода}} (\textit{sc-тексты}) имеют в общем случае нелинейную (графовую) структуру, поскольку (1) \textit{знак} каждой описываемой сущности входит в состав \textit{sc-текста} однократно и (2) каждый такой \textit{знак} может быть инцидентен неограниченному числу других \textit{знаков}, поскольку каждая описываемая \textit{сущность} может быть связана неограниченным числом связей с другими описываемыми \textit{сущностями}.}
        \scnfileitem{\textit{База знаний}, представленная текстом \textit{\textbf{SC-кода}}, является \textit{графовой структурой} специального вида, алфавит элементов которой включает в себя множество \textit{узлов}, множество \textit{ребер}, множество \textit{дуг}, множество \textit{базовых дуг} --- дуг специально выделенного типа, обеспечивающих структуризацию \textit{баз знаний}, а также множество специальных \textit{узлов}, каждый из которых имеет содержимое, являющееся \textit{файлом}, хранящимся в памяти \textit{интеллектуальной компьютерной системы}. Структурная особенность данной \textit{графовой структуры} заключается в том, что ее \textit{дуги} и \textit{ребра} могут связывать не только \textit{узел} с \textit{узлом}, но и \textit{узел} с \textit{ребром} или \textit{дугой}, \textit{ребро} или \textit{дугу} с другим \textit{ребром} или \textit{дугой}.}
        \scnfileitem{\uline{Все элементы} (\textit{sc-элементы}) указанной выше \textit{графовой структуры} (текста \textit{\textbf{SC-кода}}), т.е. все ее узлы (\textit{sc-узлы}), ребра (\textit{sc-ребра}) и дуги (\textit{sc-дуги}) являются обозначениями различных сущностей. При этом ребро является обозначением бинарной неориентированной связки между двумя сущностями, каждая из которых либо представлена в рассматриваемой графовой структуре соответствующим знаком, либо является самим этим знаком. Дуга является обозначением бинарной ориентированной связки между двумя сущностями. Дуга специального вида (\textit{\textbf{базовая дуга}}) является знаком связи между узлом, обозначающим некоторое множество элементов рассматриваемой графовой структуры, и одним из элементов этой графовой структуры, который принадлежит указанному множеству. Узел, имеющий содержимое (узел, для которого содержимое существует, но может в текущий момент быть неизвестным) является знаком файла, который является содержимым этого узла. Узел, не являющийся знаком файла, может обозначать какой-либо материальный объект, первичный абстрактный объект(например, число, точку в некотором абстрактном пространстве), какую-либо бинарную связь, какое-либо множество (в частности, понятие, структуру, ситуацию, событие, процесс). При этом сущности, обозначаемые элементами рассматриваемой графовой структуры, могут быть постоянными (существующими всегда) и временными (сущностями, которым соответствует отрезок времени их существования).
        	\\Кроме того, сущности, обозначаемые элементами рассматриваемой графовой структуры, могут быть константными (конкретными) сущностями и переменными (произвольными) сущностями. Каждому элементу рассматриваемой графовой структуры, являющемуся обозначением переменной сущности, ставится в соответствие область возможных значений этого обозначения. Область возможных значений каждого переменного ребра является подмножеством множества всевозможных константных ребер, область возможных значений каждой переменной дуги является подмножеством множества всевозможных константных дуг, область возможных значений каждого переменного узла является подмножеством множества всевозможных константных узлов.}
        \scnfileitem{В рассматриваемой графовой структуре, являющейся представлением базы знаний в \textit{\textbf{SC-коде}}, могут, но не должны существовать разные элементы графовой структуры, обозначающие одну и ту же сущность. Если пара таких элементов обнаруживается, то эти элементы склеиваются (отождествляются). Таким образом, синонимия внутренних обозначений в базе знаний интеллектуальной компьютерной системы, построенной на основе \textit{\textbf{SC-кода}}, запрещена. При этом синонимия внешних обозначений считается нормальным явлением. Формально это означает, что из некоторых элементов рассматриваемой графовой структуры выходит несколько дуг, принадлежащих отношению \scnqqi{быть \textit{\textbf{внешним идентификатором*}}}.
        	\\Из всех указанных дуг, принадлежащих отношению \scnqqi{быть \textit{\textbf{внешним идентификатором*}}} и выходящих из одного элемента рассматриваемой графовой структуры, обязательно выделяется одна (очень редко две) путем включения их в число дуг, принадлежащих отношению \scnqqi{быть \textit{\textbf{основным внешним идентификатором*}}}. Это означает, что указываемый таким образом внешний идентификатор не является омонимичным, т.е. не может быть использован как внешний идентификатор, соответствующий другому элементу рассматриваемой графовой структуры.}
        \scnfileitem{Кроме файлов, представляющих различные внешние обозначения (имена, иероглифы, пиктограммы), в памяти интеллектуальной компьютерной системе, построенной на основе \textit{\textbf{SC-кода}}, могут хранится файлы различных текстов (книг, статей, документов, примечаний, комментариев, пояснений, чертежей, рисунков, схем, фотографий, видео-материалов, аудио-материалов).}
        \scnfileitem{\uline{Любую сущность}, требующую описания, в тексте \textit{\textbf{SC-кода}} можно обозначить в виде \textit{sc-элемента}. Это являетс яодним из факторов, обеспечивающих универсальность \textit{\textbf{SC-кода}}. Особо подчеркнем, что sc-элементы являются не просто обозначениями различных описываемых сущностей, а обозначениями, которые являются элементарными (атомарными) фрагментами знаковой конструкции, т.е. фрагментами, детализация структуры которых не требуется для \scnqq{прочтения} и понимания этой знаковой конструкции.}
        \scnfileitem{Текст \textit{\textbf{SC-кода}}, как и любая другая графовой структура, является абстрактным математическим объектом, не требующим детализации (уточнения) его кодирования в памяти компьютерной системы (например, в виде матрицы смежности, матрицы инцидентности, списковой структуры). Но такая детализация потребуется для технической реализации памяти, в которой хранятся и обрабатываются \textit{sc-тексты}.}
        \scnfileitem{Важнейшим дополнительным свойством \textit{\textbf{SC-кода}} является то,что он удобен не просто для внутреннего представления знаний в памяти интеллектуальной компьютерной системы, но также и для внутреннего представления информации в памяти компьютеров, специально предназначенных для интерпретации семантических моделей интеллектуальных компьютерных систем. Т.е., \textit{\textbf{SC-код}} определяет синтаксические, семантические и функциональные принципы организации памяти компьютеров нового поколения, ориентированных на реализацию интеллектуальных компьютерных систем, --- принципы организации графодинамической ассоциативной семантической памяти.}
        \scnfileitem{\textit{\textbf{SC-код}} рассматривается нами как объединение трех его подъязыков, в число которых входит \textit{\textbf{Ядро SC-кода}}, подъязык \textit{\textbf{SC-кода}}, обеспечивающий представление текстов \textit{\textbf{SC-кода}} (\textit{sc-текстов}) в форме орграфов классического вида, являющихся подразбиениями текстов \textit{\textbf{Ядра SC-кода}} и, соответственно, использующих \uline{явное} представление пар инцидентности элементов sc-текстов (sc-элементов), синтаксическое \textit{\textbf{Расширение Ядра SC-кода}}, обеспечивающее представление в памяти ostis-системы информационных конструкций инородного для \textit{\textbf{SC-кода}} вида.}
    \end{scnrelfromvector}

    \scnheader{абстрактный язык}
    \scnidtf{Язык, для которого не уточняется способ представления символов (синтаксически элементарных фрагментов), входящих в состав текстов этого языка, а задается только \uline{алфавит*} этих символов, то есть семейство классов символов, считающихся синтаксически эквивалентными друг другу.}
    \begin{scnindent}
        \scntext{примечание}{Каждому абстрактному языку можно поставить в соответствие целое семейство \textit{реальных языков}, обеспечивающих \uline{изоморфное} реальное представление текстов указанного абстрактного языка путем уточнения способов представления (изображения, кодирования) символов, входящих в состав этих текстов, а также путем уточнения правил установления синтаксической эквивалентности этих символов. Очевидно, что во всём остальном синтаксис и денотационная семантика указанных реальных языков полностью совпадает с синтаксисом и денотационной семантикой соответствующего абстрактного языка.}
        \begin{scnindent}
            \scntext{примечание}{Для \textit{SC-кода} как абстрактного языка необходима разработка как минимум трех синтаксически и семантически эквивалентных ему реальных языков: (1) язык кодирования текстов \textit{SC-кода} в памяти традиционных компьютеров; (2) язык кодирования текстов \textit{SC-кода} в семантической ассоциативной памяти; (3) \textit{Ядро SCg-кода} --- язык, синтаксически и семантически эквивалентный \textit{SC-коду} и обеспечивающий графическое представление текстов \textit{SC-кода}.}
        \end{scnindent}
    \end{scnindent}
    
    \scnheader{графовый язык}
    \begin{scnrelfromvector}{быть заданным}
        \scnfileitem{множество входящих в язык элементарных фрагментов (символов), которое, в свою очередь, состоит:
            \begin{scnitemize}
                \item из множества узлов (вершин), возможно, синтаксически разного вида;
                \item из множества связок, которые также могут принадлежать разным синтаксически выделяемым классам.
            \end{scnitemize}}
        \scnfileitem{в общем случае несколько отношений инцидентности связок с компонентами этих связок (при этом указанными компонентами в общем случае могут быть не только вершины, но и другие связки).}
    \end{scnrelfromvector}
    
    \scnheader{SC-код}
    \scntext{примечание}{Следует особо подчеркнуть, что  унификация и максимально возможное упрощение  \textbf{\textit{синтаксиса}} и \textbf{\textit{денотационной семантики}} внутреннего языка интеллектуальных компьютерных систем прежде всего необходимы потому, что подавляющий объем \textbf{\textit{знаний}}, хранимых в составе  базы знаний интеллектуальной компьютерной системы, представляют собой \textbf{\textit{метазнания}}, описывающие свойства других знаний.
    	\\К \textit{метазнаниям}, в частности, следует отнести и различного вида логические высказывания и всевозможного вида программы, описания методов (навыков), обеспечивающих решение различных классов задач. Необходимо исключить зависимость формы представляемого знания от вида этого знания.
    	\\Форма (структура) внутреннего представления знания любого вида должна зависеть \uline{только}(!) от смысла этого знания. Более того, конструктивное (формальное) развитие теории интеллектуальных компьютерных систем невозможно без уточнения (унификации, стандартизации) и обеспечения семантической совместимости различных видов знаний, хранимых в базе знаний интеллектуальной компьютерной  системы.  Очевидно, что многообразие форм представления семантически эквивалентных знаний делает разработку общей теории  интеллектуальных компьютерных систем практически невозможной.}
    \scntext{примечание}{\textit{SC-код} является одним из возможных вариантов \textit{смыслового представления знаний}.}
    \begin{scnindent}
        \scnrelfrom{смотрите}{}
    \end{scnindent}
            
    \scnheader{SC-пространство}
    \scntext{примечание}{Понятие \textit{SC-пространства} наряду с понятием \textit{SC-кода} играет важнейшую роль для уточнения и формализации понятия смысла информационных конструкций, для унификации смыслового представления информации и для максимально возможного исключения субъективизма в трактовке понятия смысла. Смысл информационной конструкции в конечном счете определяется (1) конфигурацией смыслового представления этой конструкции и (2) и местоположением (контекстом) смыслового представления указанной информационной конструкции в рамках смыслового пространства, т.е. в рамках объединенного смыслового представления \uline{всевозможных} информационных конструкций, либо в рамках объединенного смыслового представления информации, накопленной к заданному моменту времени некоторым индивидуальным субъектом или коллективом субъектов. Таким объединенным смысловым представлением информации, в частности, является смысловое представление глобальной базы всех знаний, накопленных человечеством к текущему моменту.}\scntext{пояснение}{Объединение (вместилище) \uline{всевозможных} унифицированных семантических сетей (текстов SC-кода)}\scntext{примечание}{При теоретико-множественном объединении текстов \textit{SC-кода} семантически эквивалентные (синонимичные) элементы (синтаксически элементарные фрагменты) этих текстов считаются совпадающими элементами и при объединении указанных текстов склеиваются.}\scnrelto{объединение}{SC-код}
    \scnidtf{Унифицированное смысловое пространство}
    \scntext{достоинство}{Важнейшим достоинством \textit{SC-пространства} является возможность уточнения таких понятий, как понятие аналогичности (сходства и отличия) различных описываемых внешних сущностей, аналогичности унифицированных семантических сетей (текстов \textit{SC-кода}), понятие семантической близости описываемых сущностей (в том числе, и текстов \textit{SC-кода}).}
    
    \bigskip
\end{scnsubstruct}
\scnsourcecommentinline{Завершили Сегмент \scnqqi{Основные положения внутреннего языка ostis-систем}}

        
\scnsegmentheader{Описание Ядра SC-кода}
\begin{scnsubstruct}

    \scnstructheader{Синтаксис Ядра SC-кода}
    \begin{scnsubstruct}
        \scnheader{Синтаксис Ядра SC-кода}
        \begin{scnrelfromvector}{быть заданным}
            \scnitem{Алфавит Ядра SC-кода}
            \scnitem{Отношение \textit{инцидентности sc-коннекторов*}}
            \scnitem{Отношение \textit{инцидентности входящих sc-дуг*}}
        \end{scnrelfromvector}
        \scnrelto{синтаксис}{Ядро SC-кода}
                
        \scnheader{Синтаксическая структура линейных информационных конструкций}
        \begin{scnrelfromvector}{быть заданным}
            \scnfileitem{алфавит используемых символов (элементарных, атомарных фрагментов информационных конструкций, каковыми, в частности, являются буквы), то есть семейство таких попарно непересекающихся классов синтаксически эквивалентных символов, для которых существует простая процедура, позволяющая для любого символа по его синтаксическим особенностям установить факт его принадлежности одному из указанных классов}
            \scnfileitem{бинарное ориентированное отношением, определяющее непосредственный порядок (последовательность) символов в строках символов}
        \end{scnrelfromvector}

        \scnheader{Cинтаксическая структура \textit{sc-конструкций}}
        \begin{scnrelfromvector}{быть заданным}
            \scnfileitem{семейство классов \textit{синтаксически} эквивалентных \textit{sc-элементов}, в каждый из которых входят \textit{sc-элементы} с одинаковыми \textit{синтаксическими} характеристиками или, условно говоря, с одинаковыми наборами \textit{синтаксических} меток}
            \scnfileitem{двое \textit{бинарных ориентированных} \textit{отношений инцидентности sc-элементов}, заданных на множестве всех \textit{sc-элементов}:
            \begin{scnitemize}
                \item \textit{Отношением инцидентности обозначений sc-пар с их компонентами}
                \item \textit{Отношением инцидентности обозначений \textit{ориентированных} sc-пар с их вторыми компонентами},которое является подмножеством \textit{Отношения инцидентности обозначений sc-пар с их компонентами}
            \end{scnitemize}}
        \end{scnrelfromvector}

        \scnheader{Ядро SC-кода}
        \scnrelfrom{множество всех элементов конструкций данного языка}{sc-элемент}
            \begin{scnindent}
                \scnidtf{элемент конструкции \textit{Ядра SC-кода}}
                \scnidtf{синтаксически элементарный (атомарный) фрагмент дискретной информационной конструкции, принадлежащей \textit{Ядру SC-кода}}
                \scnidtf{Класс элементов конструкций \textit{Ядра SC-кода}}
                \scnidtf{Множество всех элементов всевозможных конструкций \textit{Ядра SC-кода}}
            \end{scnindent}
        \scnrelfrom{алфавит}{Алфавит Ядра SC-кода\scnsupergroupsign}
                
        \scnheader{Алфавит Ядра SC-кода\scnsupergroupsign}    
        \scnidtf{Множество (Семейство) всех классов синтаксически эквивалентных sc-элементов Ядра SC-кода}
        \scnidtf{класс синтаксически эквивалентных sc-элементов Ядра SC-кода}
        \scnidtf{класс синтаксически эквивалентных элементов конструкций Ядра SC-кода}
        \scnidtf{элемент Алфавита Ядра SC-кода}
        \scnidtf{синтаксический тип sc-элемента Ядра SC-кода}
        \begin{scneqtoset}
            \scnitem{sc-ребро}
            \scnitem{sc-дуга общего вида}
            \scnitem{базовая sc-дуга}
            \scnitem{sc-узел, являющийся знаком файла}
            \scnitem{sc-узел, не являющийся знаком файла}
        \end{scneqtoset}
                    
        \scnheader{Минимальный алфавит SC-кода\scnsupergroupsign}
        \scnidtf{\textit{Класс константных постоянных позитивных sc-пар принадлежности} и Класс всех остальных \textit{sc-элементов} (задаваемых по умолчанию)}
        \scntext{примечание}{Тем не менее, если учитывать особенности обработки в \textit{sc-памяти} разных классов \textit{sc-элементов}, целесообразно сделать расширение \textit{Минимального алфавита SC-кода} и, соответственно, ввести понятие \textbf{\textit{Синтаксического Ядра SC-кода}}, а также целого семейства \textit{синтаксических расширений Ядра SC-кода}}
        \scntext{пояснение}{Если известен смысл выделяемых классов sc-элементов (\textit{sc-классов}), каждый из которых в \textit{sc-памяти} представлен константным \textit{sc-элементом}, обозначающим этот \textit{sc-класс}, то для \scnqqi{анализа} и понимания \textit{sc-конструкций}, хранимых в \textit{sc-памяти}, достаточно синтаксически выделить только Класс \textit{константных постоянных позитивных sc-пар принадлежности}, с помощью которых каждый \textit{sc-элемент} будет \textit{явно} соединяться с \textit{sc-элементами}, обозначающими те \textit{sc-классы}, которым этот \textit{sc-элемент} принадлежит. Очевидно, что таким явным способом выделить указанные \textit{константные постоянные позитивные sc-пар принадлежности} с помощью самих этих sc-пар невозможно.}
            
        \scnheader{класс sc-элементов}
        \scntext{пояснение}{\textbf{класс sc-элементов} можно выделить \textit{явно} путем:
        \begin{scnitemize}
            \item включения в состав базы знаний \textit{sc-элемента}, являющегося знаком этого класса sc-элементов (\textit{sc-класса});
            \item проведения \textit{постоянных позитивных sc-пар принадлежности} во все \textit{sc-элементы}, являющиеся элементами выделяемого \textit{sc-класса} и хранимые (присутствующие) в текущем состоянии \textit{sc-памяти}.
        \end{scnitemize}}

        \scnheader{SC-код}
        \begin{scnrelfromset}{понятия, лежащие в основе}
            \scnitem{sc-элемент}
            \scnitem{sc-множество}
            \scnitem{sc-структура}
            \scnitem{sc-текст}
            \scnitem{sc-знание}
            \scnitem{файл}
            \scnitem{sc-идентификатор}
            \scnitem{основной sc-идентификатор}
        \end{scnrelfromset}
        \scntext{примечание}{\textit{SC-коду} соответствует несколько синтаксических модификаций, каждая из которых задается:
            \begin{scnitemize}
                \item своим алфавитом, то есть семейством \textit{синтаксически выделяемых классов sc-элементов};
                \item своим способом представления (кодирования) \textit{пар инцидентности sc-элементов}, связывающих \textit{sc-элементы} между собой.
            \end{scnitemize}}
        
        \scnheader{алфавит синтаксической модификации SC-кода}
	    \scnidtf{семейство синтаксических меток, приписываемых \textit{sc-элементам} в рамках соответствующей синтаксической модификации SC-кода и указывающих факт принадлежности \textit{sc-элемента} соответствующему классу \textit{sc-элементов} (\textit{sc-классу})}

        \scnheader{sc-элемент}
        \scnidtf{элементарный (атомарный) фрагмент информационной конструкции, принадлежащей SC-коду}
        \scnidtf{обозначение одной из описываемых сущностей}
        \scnrelfrom{разбиение}{Алфавит Ядра SC-кода\scnsupergroupsign}
        \begin{scnindent}
            \scntext{примечание}{\textit{Алфавит Ядра SC-кода} является одним из признаков классификации sc-элементов.}
            \scntext{примечание}{В процессе обработки текстов \textit{Ядра SC-кода} синтаксический тип \textit{sc-элементов} может меняться --- \textit{sc-узел} может трансформироваться в \textit{sc-ребро}, \textit{sc-ребро} --- в \textit{sc-дугу}, \textit{sc-дуга} общего вида --- в \textit{базовую sc-дугу}.}
        \end{scnindent}
        
        \scnheader{sc-узел}
        \scneq{\textup{(} sc-узел, являющийся знаком файла $\bigcup$ sc-узел, не являющийся знаком файла \textup{)}}

        \scnheader{sc-множество}
        \scnidtf{sc-конструкция}
        \scnidtf{множество sc-элементов}
        \scnidtf{информационная конструкция SC-кода}
        
        \scnheader{sc-структура}
        \scnidtf{sc-множество, содержащее sc-связки (знаки связей) между элементами этого множества}
                
        \scnheader{sc-текст}
        \scnidtf{связная sc-структура, являющаяся семантически корректной в рамках Базовой денотационной семантики SC-кода, а также синтаксически корректной в рамках соответствующей синтаксической модификации SC-кода}
        
        \scnheader{sc-знание}
        \scnidtf{sc-текст, обладающий дополнительным свойством иметь истинное значение по отношению к соответствующей предметной области}
        
        \scnheader{файл}
        \scnidtf{информационная конструкция, которая не является sc-конструкцией и которая может храниться в файловой памяти ostis-системы}
        
        \scnheader{sc-идентификатор}
        \scnsubset{файл}
        \scnidtf{файл, являющийся внешним идентификатором (в частности, именем) соответствующего sc-элемента, хранимого в sc-памяти ostis-системы}

        \scnheader{основной sc-идентификатор}
        \scnidtf{sc-идентификатор, который взаимно однозначно соответствует идентифицируемому sc-элементу}
           
        \scnheader{синтаксически выделяемый sc-класс sc-элементов в рамках Ядра SC-кода\scnsupergroupsign}
        \scnidtf{класс \textit{sc-элементов}, определяемый на основе \textit{Алфавита Ядра SC-кода\scnsupergroupsign}}
        \scnidtf{синтаксически выделяемый в рамках \textit{Ядра SC-кода} класс sc-элементов}
        \scnidtf{синтаксическая метка, приписываемая sc-элементам в рамках \textit{Ядра SC-кода}}
        \scnidtf{синтаксическая метка sc-элементов, выделяющая в рамках \textit{Ядра SC-кода} соответствующий класс синтаксически эквивалентных sc-элементов}
        \scnidtf{класс синтаксически эквивалентных sc-элементов в рамках \textit{Ядра SC-кода}}
        \scnidtf{синтаксический тип sc-элементов, выделяемый в рамках \textit{Ядра SC-кода}}
        \scntext{примечание}{В различных синтаксических \textit{расширениях Ядра SC-кода} синтаксически выделяемые sc-классы могут пересекаться. То есть sc-элемент может принадлежать сразу несколькими синтаксически выделяемым \textit{sc-классам}.}
        \scnhaselement{sc-коннектор}
        \scnhaselement{sc-дуга}
        \scnsuperset{Алфавит Ядра SC-кода\scnsupergroupsign}
                
        \scnheader{sc-дуга}
        \scneq{\textup{(} базовая sc-дуга $\bigcup$ sc-дуга общего вида \textup{)}}
        \begin{scnsubdividing}
            \scnitem{sc-дуга общего вида}
            \scnitem{базовая sc-дуга}
        \end{scnsubdividing}
        
        \scnheader{sc-коннектор}
        \scneq{\textup{(} sc-дуга $\bigcup$ sc-ребро \textup{)}}

        \begin{scnsubdividing}
            \scnitem{sc-ребро общего вида}
            \scnitem{sc-дуга общего вида}
        \end{scnsubdividing}
    \end{scnsubstruct}



    \scnstructheader{Синтаксическая классификация sc-элементов в рамках Ядра SC-кода}
    \begin{scnsubstruct}
        \scnheader{sc-элемент}
        \begin{scnsubdividing}
            \scnitem{sc-коннектор}
            \begin{scnindent}
                \begin{scnsubdividing}
                    \scnitem{sc-ребро}
                    \scnitem{sc-дуга}
                    \begin{scnindent}
                        \begin{scnsubdividing}
                            \scnitem{базовая sc-дуга}
                            \scnitem{sc-дуга общего вида}
                        \end{scnsubdividing}
                    \end{scnindent}	
                \end{scnsubdividing}
            \end{scnindent}	
            \scnitem{sc-узел}
                \begin{scnsubdividing}
                    \scnitem{sc-узел, являющийся знаком файла}
                    \scnitem{sc-узел, не являющийся знаком файла}
                \end{scnsubdividing}
        \end{scnsubdividing}
        \scntext{примечание}{Все классы \textit{sc-элементов}, входящие в состав синтаксической классификации sc-элементов являются синтаксически выделяемыми классами \textit{sc-элементов}.}
        \scntext{примечание}{Формирование, семейства \textit{синтаксически выделяемых sc-классов} (то есть семейства синтаксических меток, приписываемых sc-элементам) может осуществляться на основе \textit{синтаксической классификации} \textit{sc-элементов} по \textit{различным} признакам. Желательно при этом, чтобы такая синтаксическая классификация \textit{sc-элементов} была согласована с семантической классификацией sc-элементов.
        	\\Другими словами, каждый \textit{синтаксически выделяемый sc-класс} (каждая синтаксическая метка) должен иметь четкую семантическую интерпретацию, то есть должен быть одновременно и \textit{семантически выделяемым sc-классом}.}
        \begin{scnindent}
        	\scnrelfrom{смотрите}{Семантическая классификация sc-элементов}
        \end{scnindent}
        
        \scnheader{следует отличать*}
        \begin{scnhaselementset}
            \scnitem{синтаксически выделяемый sc-класс в рамках Ядра SC-кода}
            \scnitem{синтаксически выделяемый sc-класс в рамках расширения Ядра SC-кода}
            \scnitem{sc-класс}
            \begin{scnindent}
                \scnidtf{Класс sc-элементов, выделяемый (задаваемый) явно с помощью sc-конструкции, состоящей (1) из sc-элемента, являющего \textit{знаком} этого класса и (2) из константных постоянных позитивных sc-пар принадлежности, соединяющих указанный знак выделяемого класса sc-элементов со всеми sc-элементами, принадлежащими этому классу и хранимыми в текущем состоянии sc-памяти}
            \end{scnindent} 
        \end{scnhaselementset}
        \begin{scnhaselementset}
            \scnitem{денотационную семантику каждого sc-элемента}
            \begin{scnindent}
            \scntext{пояснение}{Соотношение между sc-элементом и тем, что он обозначает (его денотатом) и соответствующую \textit{семантическую} классификацию всего множества sc-элементов}
            \end{scnindent}
            \scnitem{синтаксический тип каждого sc-элемента}
            \begin{scnindent}
            	\scntext{пояснение}{Синтаксическую метку (значение синтаксического признака-параметра), приписываемую каждому sc-элементу и соответствующую \textit{синтаксическую} классификацию всего множества sc-элементов (Алфавит SC-кода\scnsupergroupsign)}
            \end{scnindent}
        \end{scnhaselementset}
        
        \scnheader{инцидентность sc-коннекторов*}
        \scnidtftext{определение}{Бинарное ориентированное отношение, первым компонентом каждой ориентированной пары которого является некоторый sc-коннектор, а вторым компонентом является один из sc-элементов, соединяемых указанным sc-коннектором с некоторым другим sc-элементом, который указывается в другой паре инцидентности для этого же sc-коннектора}
            
        \scnheader{инцидентность входящих sc-дуг*}
        \scnidtftext{определение}{Бинарное ориентированное отношение, первым компонентом каждой ориентированной пары которого является некоторая sc-дуга, а вторым компонентом --- sc-элемент, в который указанная sc-дуга входит, т.е. sc-элемент, который является вторым компонентом, соединяемым (связываемым) указанной sc-дугой}
            
        \scnheader{Ядро SC-кода}
        \scnrelfrom{синтаксические правила}{\scnstructidtf{Синтаксические правила Ядра SC-кода}}
        \begin{scnindent}
            \begin{scnhassubset}
                \scnitem{
                    \begin{scnset}
                        \scnitem{инцидентность sc-коннекторов*}
                        \begin{scnindent}
                            \scnsuperset{инцидентность входящих sc-дуг*}
                            \scniselement{бинарное ориентированное отношение}
                        \end{scnindent}
                    \end{scnset}
                }
                \scnfileitem{Для каждого sc-коннектора существует две и только две пары \textit{инцидентности sc-коннекторов*}, указанный sc-коннектор является первым связующим компонентом. При этом для каждой sc-дуги из двух указанных пар инцидентности \uline{одна} должна принадлежать отношению инцидентности \textit{входящей sc-дуги*}.}
                \scnfileitem{Пары инцидентности sc-коннекторов могут быть \uline{кратными}. То есть sc-коннектор может соединять (связывать) sc-элемент с самим собой. Такие sc-коннекторы будем называть петлевыми sc-коннекторами (петлевыми sc-ребрами и петлевыми sc-дугами).}
                \scnfileitem{Само \textit{Отношение инцидентности sc-коннекторов*} и, следовательно, \textit{Отношение инцидентности входящих sc-дуг*} не имеет кратных пар инцидентности. То есть sc-коннектор не может быть инцидентен самому себе.}
                \scnfileitem{В область определения \textit{Отношения инцидентности sc-коннекторов*} и \textit{Отношения инцидентности входящих sc-дуг*} входят не только sc-узлы общего вида, но и sc-коннекторы. Это значит, что sc-коннектор может соединять (связывать) не только sc-узел с sc-узлом, но также sc-узел с sc-коннектором и даже sc-коннектор с sc-коннектором.}
            \end{scnhassubset}
        \end{scnindent}
    \end{scnsubstruct}



    \scnstructheader{Денотационная семантика Ядра SC-кода}
    \begin{scnsubstruct}
        \scnheader{Ядро SC-кода}
        \scnrelfrom{денотационная семантика}{Денотационная семантика Ядра SC-кода}
        \begin{scnindent}
            \scnidtf{Описание соответствия информационных конструкций, принадлежащих \textit{Ядру SC-кода}, и сущностей, описываемых этими конструкциями}
        \end{scnindent}
        
        \scnheader{параметр, заданный на множестве sc-элементов}
        \scnhaselement{Алфавит Ядра SC-кода\scnsupergroupsign}
        \scnhaselement{Алфавит SC-кода\scnsupergroupsign}
        \scnhaselement{Структурная типология sc-элементов\scnsupergroupsign}
        \scnhaselement{Типология sc-элементов по признаку константности\scnsupergroupsign}
        \scnhaselement{Типология sc-элементов по признаку постоянства обозначаемой сущности\scnsupergroupsign}
        \scnhaselement{Типология sc-элементов по признаку доступности sc-элемента в процессе эксплуатации и эволюции базы знаний\scnsupergroupsign}
    \end{scnsubstruct}



    \scnstructheader{Структурная классификация sc-констант}
    \begin{scnsubstruct}
        \scntext{примечание}{Приведенная типология полностью аналогична \textit{Структурной типологии sc-элементов\scnsupergroupsign}, в отличие от которой она описывает структурную классификацию только константных sc-элементов (\textit{sc-констант}).}
        \scniselement{sc-структура}
        \scnrelboth{аналог}{Структурная типология sc-элементов\scnsupergroupsign}
    
        \scnheader{sc-константа}
        \scnrelfrom{разбиение}{\scnkeyword{Структурная типология sc-констант\scnsupergroupsign}}
        \begin{scnindent}
            \begin{scneqtoset}
            \scnitem{sc-множество}
            \begin{scnindent}
                \begin{scnsubdividing}
                    \scnitem{sc-связка}
                    \begin{scnindent}
                        \begin{scnsubdividing}
                            \scnitem{sc-синглетон}
                            \scnitem{sc-пара}
                            \begin{scnindent}
                                \begin{scnsubdividing}
                                    \scnitem{неориентированная sc-пара}
                                    \scnitem{ориентированная sc-пара}
                                \end{scnsubdividing}
                            \end{scnindent}
                            \scnitem{sc-связка, не являющаяся ни синглетоном, ни парой}
                        \end{scnsubdividing}
                    \end{scnindent}
                    \scnitem{sc-класс}
                    \scnitem{sc-структура}
                \end{scnsubdividing}
            \end{scnindent}
            \scnitem{внешняя сущность}
            \begin{scnindent}
                \scnidtf{sc-элемент, являющийся знаком внешней сущности}
                \scnidtf{знак внешней сущности}
                \scnidtf{знак сущности, не являющейся sc-множеством (sc-конструкцией)}
            \end{scnindent} 
            \begin{scnindent}
                \begin{scnsubdividing}
                    \scnitem{файл}
                    \scnitem{информационная конструкция, не являющаяся ни sc-множеством, ни файлом}
                    \scnitem{внешняя сущность, не являющаяся информационной конструкцией}
                \end{scnsubdividing}
            \end{scnindent}
            \end{scneqtoset}
        \end{scnindent}

        \scnheader{ориентированная sc-пара}
        \begin{scnsubdividing}
            \scnitem{sc-пара принадлежности}
            \begin{scnindent}
                \begin{scnsubdividing}
                    \scnitem{sc-пара нечеткой принадлежности}
                    \scnitem{sc-пара позитивной принадлежности}
                    \begin{scnindent}
                        \scnsuperset{sc-пара постоянной позитивной принадлежности}
                        \begin{scnindent}
                            \begin{scnreltoset}{пересечение множеств}
                                \scnitem{sc-константа}
                                \scnitem{постоянная сущность}
                                \scnitem{статическая сущность}
                                \scnitem{sc-пара позитивной принадлежности}	
                            \end{scnreltoset}
                        \end{scnindent}
                        \scnsuperset{sc-пара временной позитивной принадлежности}
                        \begin{scnindent}
                            \begin{scnreltoset}{пересечение множеств}
                                \scnitem{sc-константа}
                                \scnitem{временная сущность}
                                \scnitem{динамическая сущность}
                                \scnitem{sc-пара позитивной принадлежности}	
                            \end{scnreltoset}
                        \end{scnindent}
                    \end{scnindent}
                    \scnitem{sc-пара негативной принадлежности}
                \end{scnsubdividing}
            \end{scnindent}
            \scnitem{ориентированнная sc-пара, не являющаяся парой принадлежности}	
        \end{scnsubdividing}
    \end{scnsubstruct}

    \scnstructheader{Семантическая классификация sc-элементов}
    \begin{scnsubstruct}
        \scnheader{sc-элемент}
        \begin{scnrelfromset}{базовые признаки классификации}
            \scnitem{структурный признак}
            \begin{scnindent}
                \scnrelfrom{смотрите}{\textbf{Структурная типология sc-элементов\scnsupergroupsign}}
            \end{scnindent}
            \scnitem{логико-семантический признак}
            \begin{scnindent}
                \scnrelfrom{смотрите}{\textbf{Типология sc-элементов по признаку константности\scnsupergroupsign}}
            \end{scnindent}
            \scnitem{темпоральная характеристика сущностей}
            \begin{scnindent}
                \scnidtf{темпоральная характеристика sc-элементов}
                \scnsuperset{постоянство существования обозначаемой сущности}
                \begin{scnindent}
                    \scnrelfrom{смотрите}{\textbf{Типология sc-элементов по постоянству обозначаемых сущностей\scnsupergroupsign}}
                \end{scnindent}
                \scnsuperset{временность существования обозначаемой сущности}
                \scnsuperset{статичность обозначаемой сущности}
                \begin{scnindent}
                    \scnrelfrom{смотрите}{\textbf{Типология sc-элементов по статичности обозначаемых сущностей\scnsupergroupsign}}
                    \scnidtf{стационарность обозначаемой сущности}
                \end{scnindent}
                \scnsuperset{динамичность обозначаемой сущности}
                \begin{scnindent}
                    \scnidtf{изменчивость обозначаемой сущности}
                \end{scnindent}
            \end{scnindent}
        \end{scnrelfromset}
        
        \scnheader{sc-элемент}
        \scnidtf{обозначение описываемой сущности}
        \scnrelfrom{разбиение}{\scnkeyword{Структурная типология sc-элементов\scnsupergroupsign}}
        \begin{scnindent}
            \begin{scneqtoset}
                \scnitem{обозначение терминальной сущности}
                \begin{scnindent}
                    \begin{scnsubdividing}
                        \scnitem{обозначение материальной сущности}
                        \begin{scnindent}
                            \scnidtf{обозначение внешней сущности, не являющейся информационной конструкцией}
                            \scntext{примечание}{К материальным сущностям относятся физические тела, поля, биологические объекты, технические системы и многое другое.}
                        \end{scnindent}
                        \scnitem{обозначение абстрактной терминальной сущности}
                        \begin{scnindent}
                            \scntext{примечание}{Примерами абстрактных терминальных сущностей являются предельно малые физические тела, точки различных пространств, числа.}
                        \end{scnindent}
                        \scnitem{обозначение дискретной информационной конструкции, не принадлежащей SC-коду}
                        \begin{scnindent}
                            \scnidtf{обозначение информационной конструкции, не являющейся ни sc-множеством, ни файлом}
                            \scnidtf{обозначение информационной конструкции, не являющейся конструкцией \textit{SC-кода} и тем более \textit{Ядра SC-кода}}
                            \scnidtf{обозначение \scnqq{инородной} для \textit{SC-кода} информационной конструкции}
                            \scnsuperset{обозначение файла}
                            \begin{scnindent}
                                \scnidtf{обозначение внешней информационной конструкции, представленной в электронной форме}
                            \end{scnindent}
                        \end{scnindent}
                    \end{scnsubdividing}
                \end{scnindent}
                \scnitem{обозначение sc-множества}
                \begin{scnindent}
                    \begin{scnsubdividing}
                        \scnitem{обозначение sc-связки}
                        \scnitem{обозначение sc-класса}
                        \scnitem{обозначение sc-структуры}
                    \end{scnsubdividing}
                \end{scnindent}
            \end{scneqtoset}
        \end{scnindent}
        
        \scnheader{обозначение sc-связки}
        \begin{scnsubdividing}
            \scnitem{обозначение небинарной пары}
            \begin{scnindent}
                \scnsuperset{обозначение sc-синглетона}
                \scnsuperset{обозначение sc-связки, не являющейся ни синглетоном, ни парой}
            \end{scnindent}
            \scnitem{обозначение sc-пары}
            \begin{scnindent}
                \begin{scnsubdividing}
                    \scnitem{обозначение неориентированной sc-пары}
                    \scnitem{обозначение ориентированной пары неизвестной направленности}
                    \scnitem{обозначение ориентированной sc-пары}
                    \begin{scnindent}
                        \scnsuperset{обозначение sc-пары принадлежности}
                        \begin{scnindent}
                            \begin{scnsubdividing}
                                \scnitem{обозначение позитивной sc-пары принадлежности}
                                \scnitem{обозначение негативной sc-пары принадлежности}
                                \scnitem{обозначение нечеткой sc-пары принадлежности}
                            \end{scnsubdividing}
                        \end{scnindent}
                        \scnsuperset{обозначение ориентированной sc-пары, не являющейся парой принадлежности}
                    \end{scnindent}
                \end{scnsubdividing}
            \end{scnindent}
        \end{scnsubdividing}
        
        \scnheader{sc-элемент}
        \scnrelfrom{разбиение}{\scnkeyword{Типология sc-элементов по признаку константности\scnsupergroupsign}}
            \begin{scnindent}
            \begin{scneqtoset}
                \scnitem{sc-константа}
                \begin{scnindent}
                    \scnidtf{sc-элемент, логико-семантическим значением которого является он сам}
                    \scnidtf{константный sc-элемент}
                    \scnidtf{обозначение конкретной (фиксированной) сущности}
                \end{scnindent}
                \scnitem{sc-переменная}
                \begin{scnindent}
                    \scnidtf{переменный sc-элемент}
                    \scnidtf{обозначение произвольной сущности из некоторого множества сущностей}
                    \scnidtf{sc-элемент, имеющий (принимающий) произвольное значение из некоторого множества sc-элементов}
                       \begin{scnsubdividing}
                           \scnitem{sc-переменная 1-го уровня}
                           \begin{scnindent}
                               \scnidtf{sc-элемент, областью возможных значений которого является множество sc-констант}
                           \end{scnindent}
                           \scnitem{sc-переменная 2-го уровня}
                           \begin{scnindent}
                               \scnidtf{sc-элемент, возможными значениями которого являются переменные 1-го уровня}
                               \scntext{примечание}{Такие переменные (метапеременные) необходимы для описания логических языков.}
                           \end{scnindent}
                           \scnitem{sc-переменная универсального типа}
                           \begin{scnindent}
                               \scnidtf{sc-переменная, на значения которой не накладывается никаких ограничений}
                           \end{scnindent}
                       \end{scnsubdividing}
                \end{scnindent}
            \end{scneqtoset}
        \end{scnindent}
        \scnrelfrom{разбиение}{\scnkeyword{Типология sc-элементов по постоянству обозначаемых сущностей\scnsupergroupsign}}
        \begin{scnindent}
            \begin{scneqtoset}
                \scnitem{обозначение постоянной сущности}
                \scnitem{обозначение временной сущности}
                \begin{scnindent}
                    \scnidtf{обозначение нестационарной сущности, факт существования которой зависит от времени}
                    \begin{scnsubdividing}
                        \scnitem{обозначение прошлой сущности}
                        \begin{scnindent}
                            \scnidtf{обозначение сущности, существовавшей до текущего момента времени}
                        \end{scnindent}
                        \scnitem{обозначение настоящей сущности}
                        \begin{scnindent}
                            \scnidtf{обозначение сущности, существующей в текущий момент времени}
                        \end{scnindent}
                        \scnitem{обозначение будущей сущности}
                        \begin{scnindent}
                            \scnidtf{обозначение сущности, существование которой прогнозируется или планируется в будущем}
                        \end{scnindent}
                    \end{scnsubdividing}
                    \begin{scnsubdividing}
                        \scnitem{обозначение внешней временной сущности}
                        \begin{scnindent}
                            \scnsuperset{обозначение внешней ситуации}
                            \scnsuperset{обозначение внешнего события}
                            \scnsuperset{обозначение внешнего процесса}
                        \end{scnindent}
                        \scnitem{обозначение временной сущности в sc-памяти}
                        \begin{scnindent}
                            \begin{scnsubdividing}
                                \scnitem{обозначение ситуации в sc-памяти}
                                \begin{scnindent}
                                    \scnidtf{обозначение ситуации, которая возникла или возникает в процессе обработки информации в sc-памяти}
                                \end{scnindent}
                                \scnitem{обозначение события в sc-памяти}
                                \begin{scnindent}
                                    \scnidtf{обозначение события, которое произошло или произойдет в процессе обработки информации в sc-памяти}
                                \end{scnindent}
                                \scnitem{обозначение информационного процесса в sc-памяти}
                                \begin{scnindent}
                                    \scnidtf{обозначение внутреннего процесса в sc-памяти, который происходит, произошел или будет происходить}
                                \end{scnindent}
                            \end{scnsubdividing}
                        \end{scnindent}
                    \end{scnsubdividing}
                \end{scnindent}
            \end{scneqtoset}
        \end{scnindent}
        \scnrelfrom{разбиение}{\scnkeyword{Типология sc-элементов по признаку статичности обозначаемых элементов\scnsupergroupsign}}
        \begin{scnindent}
            \begin{scneqtoset}
                \scnitem{обозначение статической сущности}
                \begin{scnindent}
                    \scnidtf{обозначение статичной сущности}
                    \scnidtf{обозначение стационарной сущности}
                    \scnidtf{обозначение сущности, изменения которой в рамках соответствующего интервала времени ее существования считаются несущественными}
                    \scnsuperset{обозначение статического sc-множества}
                \end{scnindent}
                \scnitem{обозначение динамической сущности}
                \begin{scnindent}
                    \scnidtf{обозначение сущности изменяющейся во времени}
                    \scnsuperset{обозначение динамического sc-множетсва}
                \end{scnindent}
            \end{scneqtoset}
        \end{scnindent} 
        \scnrelfrom{разбиение}{\scnkeyword{Типология sc-элементов по признаку доступности sc-элемента в процессе эксплуатации и эволюции базы знаний\scnsupergroupsign}}
        \begin{scnindent}
            \begin{scneqtoset}
                \scnitem{удаленный sc-элемент}
                \begin{scnindent}
                    \scnidtf{sc-элемент, считающийся логически удаленным, но присутствующим в описании истории эксплуатации и эволюции базы знаний}
                \end{scnindent}
                \scnitem{настоящий sc-элемент}
                \begin{scnindent}
                    \scnidtf{sc-элемент, входящий в состав эксплуатируемой части базы знаний}
                \end{scnindent}
                \scnitem{будущий sc-элемент}
                \begin{scnindent}
                    \scnidtf{sc-элемент, планируемый для включения в состав эксплуатируемой части базы знаний}
                \end{scnindent}
            \end{scneqtoset}
            \begin{scnrelfromset}{следует отличать}
                \scnfileitem{Временный характер присутствия любого \textit{sc-элемента} в составе той \textit{базы знаний} (в той \textit{sc-памяти}) \textit{ostis-системы}, в которой он находится (когда-то он появляется, когда-то может быть удален из \textit{sc-памяти}).}
                \scnfileitem{Временный характер присутствия в \textit{sc-памяти} всей заданной \textit{sc-конструкции} (заданного множества sc-элементов) --- такую \textit{sc-конструкцию} будем называть \textit{ситуацией в sc-памяти}.}
                \scnfileitem{Временный характер существования \textit{внешней сущности}, которую \textit{sc-элемент} обозначает.}
                \scnfileitem{Статичный или динамичный (изменчивый) характер \textit{внешней сущности}, обозначаемой \textit{sc-элементом}. Динамический характер внешней сущности, предполагает наличие в \textit{sc-памяти} описания процесса изменения состояния или конфигурации указанной \textit{внешней сущности}.}
                \scnfileitem{\textit{динамическое sc-множество} (динамическая sc-конструкция), являющееся отражением (динамической моделью) соответствующего внешнего процесса (процесса, происходящего во внешней среде).}
                \scnfileitem{\textit{динамическое sc-множество} (динамическая sc-конструкция), являющееся отражением (динамической моделью) соответствующего внутреннего процесса (информационного процесса, происходящего в той же \textit{sc-памяти}, в которой находится \textit{sc-элемент}, обозначающий указанное динамическое \textit{sc-множество}).}
            \end{scnrelfromset}
        \end{scnindent}
        \scnheader{обозначение множества}
        \scnidtf{обозначение множества sc-элементов}
        \begin{scnsubdividing}
            \scnitem{произвольное множество}
            \begin{scnindent}
                      \scnidtf{sc-переменная, обозначающая произвольное множество из некоторого семейства множеств}
                \scnidtf{переменное множество}
            \end{scnindent}
            \scnitem{множество}
            \begin{scnindent}
                      \scnidtf{конкретное (константное, фиксированное) множество sc-элементов}
            \end{scnindent}
        \end{scnsubdividing}
        
        \scnheader{множество}
        \scnidtf{множество sc-элементов}
        \begin{scnsubdividing}
            \scnitem{множество sc-констант}
            \begin{scnindent}
                      \scnidtf{множество, элементами которого являются только sc-константы}
            \end{scnindent}
            \scnitem{множество sc-переменных}
            \begin{scnindent}
                \scnidtf{множество, элементами которого являются только sc-переменные}
                \scnsuperset{sc-переменная}
                \begin{scnindent}
                    \scnidtf{множество, элементами которого являются всевозможные sc-переменные и только они}
                    \scnsuperset{произвольное множество}
                    \begin{scnindent}
                        \scnidtf{sc-переменная, значениями которой являются всевозможные обозначения множеств и только они}
                    \end{scnindent}
                \end{scnindent}
            \end{scnindent}
            \scnitem{множество sc-констант и sc-переменных}
            \begin{scnindent}
                      \scnidtf{множество, в число элементов которого входят как sc-константы, так и sc-переменные}
                \scnsuperset{обозначение множества}
                \begin{scnindent}
                    \scnidtf{множество, элементами которого являются всевозможные \mbox{sc-переменные} и \mbox{sc-константы}, обозначающие множества и только они}
                \end{scnindent}
            \end{scnindent}
        \end{scnsubdividing}
        
        \scnheader{обозначение sc-связки}
        \scnidtf{обозначение связи между sc-элементами и/или обозначаемыми ими сущностями}
        \begin{scnsubdividing}
            \scnitem{произвольная sc-связка}
            \begin{scnindent}
                \scnidtf{sc-переменная, значениями которой являются обозначения sc-связок}
            \end{scnindent}
            \scnitem{sc-связка}
            \begin{scnindent}
                \scnidtf{конкретная sc-связка sc-элементов}
            \end{scnindent}
        \end{scnsubdividing}
        \scnsuperset{обозначение sc-пары}
        \begin{scnindent}
            \scnidtf{обозначение sc-связки двух sc-элементов либо одного sc-элемента с самим собой}
            \scnsuperset{sc-пара}
            \scnidtf{конкретная sc-пара}
            \scnsubset{sc-константа}
            \scnsuperset{sc-коннектор}
            \scnsuperset{ориентированная sc-пара}
            \begin{scnindent}
                \scnsuperset{sc-пара принадлежности}
                \begin{scnindent}
                    \begin{scnsubdividing}
                        \scnitem{позитивная sc-пара принадлежности}
                        \begin{scnindent}
                            \scnsuperset{позитивная постоянная sc-пара принадлежности}
                            \begin{scnindent}
                                \scnsuperset{базовая sc-дуга}
                            \end{scnindent}
                        \end{scnindent}
                        \scnitem{негативная sc-пара принадлежности}
                        \scnitem{нечеткая sc-пара принадлежности}
                    \end{scnsubdividing}
                \end{scnindent}
            \end{scnindent}
        \end{scnindent}

        \scnheader{обозначение класса}
        \scnidtf{обозначение множества sc-элементов, которые в соответствующем смысле эквивалентны друг другу, т.е. имеют одинаковые свойства}
        \begin{scnsubdividing}
            \scnitem{произвольный класс}
            \begin{scnindent}
                      \scnsubset{sc-переменная}
                \scniselement{sc-константа}
            \end{scnindent}
            \scnitem{класс}
            \begin{scnindent}
                      \scnsubset{sc-константа}
            \end{scnindent}
        \end{scnsubdividing}
        
        \scnheader{класс}
        \begin{scnsubdividing}
            \scnitem{класс терминальных сущностей}
            \scnitem{класс множеств}
            \begin{scnindent}
                \begin{scnsubdividing}
                    \scnitem{класс связок}
                    \begin{scnindent}
                        \scnsuperset{sc-отношение}
                    \end{scnindent}
                    \scnitem{класс классов}
                    \begin{scnindent}
                        \scnsuperset{параметр}
                    \end{scnindent}
                    \scnitem{класс структур}
                    \begin{scnindent}
                        \scnsuperset{sc-язык}
                        \begin{scnindent}
                            \scnidtf{специализированный язык, являющийся подъязыком SC-кода, и обеспечивающий представление всевозможных знаний в рамках соответствующей предметной области, которая, в свою очередь, специфицируется соответствующей комплексной онтологией}
                        \end{scnindent}
                    \end{scnindent}
                \end{scnsubdividing}
            \end{scnindent}
        \end{scnsubdividing}
        \scnhaselement{обозначение sc-множества}
        \begin{scnindent}
            \scnsuperset{sc-множество}
        \end{scnindent}
        \scnhaselement{множество}
        \scnhaselement{обозначение sc-связки}
        \begin{scnindent}
            \scnsuperset{sc-связка}
        \end{scnindent}
        \scnhaselement{sc-связка}
        \scnhaselement{обозначение sc-класса}
        \begin{scnindent}
            \scnsuperset{sc-класс}
        \end{scnindent}
        \scnhaselement{sc-класс}
        \scnhaselement{обозначение sc-структуры}
        \begin{scnindent}
            \scnsuperset{sc-структура}
        \end{scnindent}
        \scnhaselement{sc-структура}
        \scnhaselement{обозначение дискретной информационной конструкции}
        \begin{scnindent}
            \scnsuperset{дискретная информационная конструкция}
            \begin{scnindent}
                \scnsuperset{файл}
                \begin{scnindent}
                    \scnsuperset{файл ostis-системы}
                    \begin{scnindent}
                        \scnsuperset{внутренний файл ostis-системы}
                    \end{scnindent}
                \end{scnindent}
            \end{scnindent}
        \end{scnindent}
            \scnsuperset{обозначение sc-структуры}
        \scntext{примечание}{Все семантически и синтаксически выделяемые классы sc-элементов, а также всевозможные подклассы этих классов являются экземплярами (элементами) \textit{класса}}
        
        \scnheader{обозначение sc-структуры}
        \scnidtf{обозначение sc-множества, не являющегося ни sc-связкой, ни sc-классом}
        \begin{scnsubdividing}
            \scnitem{произвольная sc-структура}
            \begin{scnindent}
                \scnsubset{sc-переменная}
            \end{scnindent}
            \scnitem{структура}
            \begin{scnindent}
                \scnidtf{конкретная sc-структура}
                \scnsubset{sc-константа}
            \end{scnindent}
        \end{scnsubdividing}
    \end{scnsubstruct}



    \scnstructheader{Соотношение между семантически и синтаксически выделяемыми классами sc-элементов в рамках Ядра SC-кода}
    \begin{scnsubstruct}
        \scnheader{семантически выделяемый класс sc-элементов}
        \begin{scnsubstruct}
            \scnidtf{класс sc-элементов, определяемый сущностями, которые обозначаются этими sc-элементами, также доступностью (активностью использования) sc-элементов в процессе эксплуатации и эволюции базы знаний}
            \scniselement{обозначение терминальной сущности}
            \begin{scnindent}
                \scnsuperset{\scnkeyword{sc-узел общего вида}}
            \end{scnindent}
            \scniselement{обозначение небинарной sc-связки}
            \begin{scnindent}
                \scnsuperset{\scnkeyword{sc-узел общего вида}}
            \end{scnindent}
            \scniselement{обозначение sc-пары}
            \begin{scnindent}
                \scnrelboth{пара пересекающихся множеств}{sc-узел общего вида}
                \scnsuperset{\scnkeyword{sc-коннектор}}
                \scntext{примечание}{\textit{обозначение sc-пары} может быть представлено либо \textit{sc-узлом общего вида}, либо \textit{sc-коннектором}. При этом каждый \textit{sc-коннектор} представляет собой \textit{обозначение sc-пары}.}
            \end{scnindent}
            \scniselement{обозначение неориентированной sc-пары}
            \begin{scnindent}
                \scnidtf{обозначение бинарной неориентированной связи между sc-элементами}
                \begin{scnrelbothlist}{пара пересекающихся множеств}
                    \scnitem{sc-узел общего вида}
                    \scnitem{sc-ребро общего вида}
                \end{scnrelbothlist}
                \scntext{примечание}{Каждый \textit{sc-элемент}, принадлежащий этому классу, связывается с элементами обозначаемого им множества}
                \scntext{примечание}{\textit{обозначение неориентированной sc-пары} может быть представлено либо \textit{sc-узлом общего вида}, либо \textit{sc-ребром}. При этом не каждое \textit{sc-ребро} представляет обозначение \textit{неориентированный sc-пары}. Некоторые из них представляют \textit{обозначения ориентированных sc-пар неизвестной направленности}.}
            \end{scnindent}
            \scniselement{обозначение ориентированной sc-пары неизвестной направленности}
            \begin{scnindent}
                \begin{scnrelbothlist}{пара пересекающихся множеств}
                    \scnitem{sc-узел общего вида}
                    \scnitem{sc-ребро общего вида}
                \end{scnrelbothlist}
            \end{scnindent}
            \scniselement{обозначение ориентированной sc-пары не являющейся \textit{двумя} парами инцидентности постоянной позитивной sc-парой принадлежности}
            \begin{scnindent}
            \scntext{примечание}{Каждый \textit{sc-элемент}, принадлежащий этому классу, связывается с элементами обозначаемого им множества:
            	\begin{scnitemize}
            		\item \textit{одной} парой инцидентности, связывающей \textit{обозначение sc-пары} с ее компонентом;
            		\item \textit{одной} парой инцидентности, связывающей \textit{обозначение ориентированной sc-пары} с ее вторым компонентом
            	\end{scnitemize}}
            \end{scnindent}
            \scniselement{обозначение ориентированной sc-пары}
            \begin{scnindent}
                \scnidtf{обозначение бинарной ориентированной связи между sc-элементами}
                \begin{scnrelbothlist}{пара пересекающихся множеств}
                    \scnitem{sc-узел общего вида}
                    \scnitem{sc-ребро общего вида}
                \end{scnrelbothlist}
                \scnsuperset{sc-дуга общего вида}
            \end{scnindent}
            \scniselement{постоянная позитивная sc-пара принадлежности}
            \begin{scnindent}
            \scntext{примечание}{Каждый элемент этого класса, как и любое другое \textit{обозначение ориентированной sc-пары}, является первым компонентом \textit{пары инцидентности обозначения sc-пары с ее компонентом}, а также первым компонентом \textit{пары инцидентности обозначения ориентированной sc-пары с ее вторым компонентом}}
            \end{scnindent}
            \scniselement{константная постоянная позитивная sc-пара принадлежности}
            \begin{scnindent}
                \begin{scnrelbothlist}{пара пересекающихся множеств}
                    \scnitem{sc-узел общего вида}
                    \scnitem{sc-ребро общего вида}
                    \scnitem{sc-дуга общего вида}
                \end{scnrelbothlist}
                \scnsuperset{базовая sc-дуга}
                \begin{scnreltoset}{пересечение множеств}
                    \scnitem{sc-константа}
                    \scnitem{обозначение постоянной сущности}
                    \begin{scnindent}
                        \scnidtf{обозначение постоянно существующей сущности}
                    \end{scnindent}
                    \scnitem{обозначение sc-пары принадлежности}
                \end{scnreltoset}
            \end{scnindent}
            \scniselement{обозначение класса}
            \begin{scnindent}
                \scnsubset{sc-узел общего вида}
            \end{scnindent}
            \scniselement{обозначение структуры}
            \begin{scnindent}
                \scnsubset{sc-узел общего вида}
            \end{scnindent}
            \scniselement{файл}
            \begin{scnindent}
	            \scnidtf{знак файла}
	            \scntext{примечание}{Для \textit{sc-элементов} этого класса необходимо на \scnqqi{синтаксическом} уровне обеспечить возможность связи этого \textit{sc-элемента} с обозначаемым им \textit{файлом}, хранимым в \textit{файловой памяти} этой же \textit{ostis-системы}}
	        \end{scnindent}
            \scniselement{sc-элемент, не являющийся ни знаком файла, ни обозначением sc-пары}
            \begin{scnindent}
	            \scnsuperset{обозначение sc-синглетона}
	            \scnsuperset{обозначение sc-связки, не являющейся ни синглетоном, ни парой}
	            \scnsuperset{обозначение sc-класса}
	            \scnsuperset{обозначение sc-структуры}
	            \scnsuperset{обозначение внешней сущности, не являющейся файлом}
	        \end{scnindent}
        \end{scnsubstruct}



        \scnstructheader{Соотношение между семантически и синтаксически выделяемыми классами sc-элементов в рамках Ядра SC-кода}
        \begin{scnsubstruct}
            \scnheader{Ядро SC-кода}
            \scnrelfrom{семантические правила}{\scnstructidtf{Семантические правила Ядра SC-кода}}
            \begin{scnindent}
                \begin{scnhassubset}
                    \scnfileitem{Каждый sc-элемент является знаком (обозначением) некоторой описываемой сущности.}
                    \scnfileitem{Любая сущность может быть обозначена sc-элементом и, соответственно, описана в виде конструкции Ядра SC-кода.}
                    \scnfileitem{С помощью sc-элементов можно описать любые связи между sc-элементами и/или между сущностями, которые обозначаются этими sc-элементами. При этом указанные связи трактуются как множества связываемых sc-элементов и обозначаются sc-ребрами, sc-дугами, а в случае небинарных связей --- sc-узлами.}
                    \scnfileitem{Поскольку каждый \mbox{sc-коннектор} семантически трактуется как обозначение пары \mbox{sc-элементов}, связываемых (соединяемых) этим \mbox{sc-коннектором}, каждая пара инцидентности \mbox{sc-коннектора} семантически интерпретируется как обозначение пары принадлежности, связывающей \mbox{sc-коннектор} с одним из элементов обозначаемой им пары \mbox{sc-элементов}.}
                    \scnfileitem{\uline{Любая} описываемая сущность может быть обозначена sc-узлом общего вида, но обратное неверно, т.к. некоторые сущности могут быть обозначены sc-ребрами общего вида, sc-дугами общего вида, базовыми sc-дугами.}
                    \scnfileitem{Каждое sc-ребро является обозначением либо бинарной неориентированной связи между sc-элементами, либо бинарной ориентированной связи неизвестной направленности между sc-элементами.}
                    \scnfileitem{Любая бинарная неориентированная связь между sc-элементами может быть обозначена sc-ребром, но обратное неверно.}
                \end{scnhassubset}
            \end{scnindent}
     
            \scnheader{Правила синтаксической трансформации sc-элементов в рамках Ядра SC-кода}
            \scnidtf{Правила модификации синтаксического типа sc-элементов в рамках Ядра SC-кода}
            \begin{scnhassubset}
                \scnfileitem{Если \textit{sc-узел общего вида} является \textit{обозначением sc-пары}, то он трансформируется в \textit{sc-коннектор}}
                \scnfileitem{Если \textit{sc-узел общего вида} является \textit{обозначением неориентированной sc-пары} или \textit{обозначением ориентированной sc-пары неизвестной направленности}, то он трансформируется в \textit{sc-ребро общего вида}}
                \scnfileitem{Если \textit{sc-узел общего вида} или \textit{sc-ребро общего вида} являются \textit{обозначением ориентированной sc-пары} и при этом дополнительно указана направленность этой sc-пары, то она трансформируется в \textit{sc-дугу общего вида}.}
                \scnfileitem{Если \textit{sc-узел общего вида} или \textit{sc-ребро общего вида} или \textit{sc-дуга общего вида} являются \textit{константными постоянными позитивными парами принадлежности}, то они трансформируются в \textit{базовую sc-дугу}.}
            \end{scnhassubset}
            
            \scnheader{следует отличать*}
            \begin{scnhaselementset}
                \scnitem{синтаксически выделяемый класс sc-элементов в рамках Ядра SC-кода}
                \scnitem{синтаксически выделяемый класс sc-элементов в рамках SC-кода}
                \scnitem{семантически выделяемый класс sc-элементов}
            \end{scnhaselementset}

            \scnheader{Отношение инцидентности обозначений sc-пар с их компонентами*}
            \begin{scnrelfromlist}{часто используемый sc-идентификатор}
                \scnfileitem{пара инцидентности sc-элементов*}
                \scnfileitem{\textit{пара инцидентности обозначения sc-пары с ее компонентом}*}
            \end{scnrelfromlist}
            \begin{scnrelfromlist}{примечание}
                \scnfileitem{Каждая пара, принадлежащая данному отношению семантически трактуется как \textit{обозначение sc-пары принадлежности}, но синтаксически оформляется не в виде самостоятельного \textit{sc-элемента}, а в виде бинарной ориентированной связи между \textit{sc-элементами}, что аналогично бинарным ориентированным связям, описывающим последовательность символов в строке символов. Заметим при этом, что конфигурация \textit{sc-конструкций} в отличие от строк символов не является линейной. Заметим также, что уточнение семантической интерпретации пар инцидентности \textit{sc-элементов} полностью определяется семантической типологией первых компонентов этих пар инцидентности, то есть семантической типологией \textit{обозначений sc-пар}, являющихся первыми компонентами рассматриваемых пар инцидентности:
                    \begin{scnitemize}
                        \item если указанное \textit{обозначение sc-пары} является \textit{sc-константой}, то соответствующая пара инцидентности трактуется как \textit{пара константной принадлежности};
                        \item если указанное \textit{обозначение sc-пары} является \textit{sc-переменной}, то соответствующая пара инцидентности трактуется как \textit{пара переменной принадлежности};
                        \item если указанное \textit{обозначение sc-пары} является \textit{обозначением постоянной сущности}, то соответствующая пара инцидентности трактуется как \textit{пара постоянной принадлежности};
                        \item если указанное \textit{обозначение sc-пары} является \textit{обозначением временной сущности}, то соответствующая пара инцидентности трактуется как \textit{пара временной принадлежности}.
                \end{scnitemize}}
                \scnfileitem{Подчеркнем, что первыми компонентами пар инцидентности \textit{sc-элементов} всегда являются \textit{обозначения sc-пар}, но вторыми компонентами пар инцидентности \textit{sc-элементов} могут быть \textit{sc-элементы} любого типа (в том числе, и \textit{обозначения sc-пар})}
            \end{scnrelfromlist}
	\scnexplanation{Каждая \textit{sc-пара} (константная пара sc-элементов), каждая \textit{переменная sc-пара} и каждое \textit{обозначение sc-пары} связывается со своими элементами не явно вводимыми константными или переменными \textit{sc-парами позитивной принадлежности}, а реализуемыми на \scnqqi{физическом} уровне связями (парами) инцидентности. Таким образом \textit{пары инцидентности sc-элементов} --- это специальным образом синтаксически выделенные константные или переменные \textit{sc-пары позитивной принадлежности}, связывающие \textit{обозначения sc-пар} с элементами этих пар. Соответственно этому синтаксические особенности имеют и все \textit{обозначения sc-пар}, поскольку только из них могут выходить ориентированные \textit{пары инцидентности}. Поэтому с синтаксической точки зрения \textit{обозначения sc-пар} будем называть \textbf{\textit{sc-коннекторами}}, \textit{обозначения неориентированных sc-пар} --- \textbf{\textit{sc-ребрами}}, а \textit{обозначения ориентированных sc-пар} --- \textbf{\textit{sc-дугами}}. При этом из класса \textit{пар инцидентности sc-элементов} выделим подкласс пар, связывающих обозначения sc-дуг с теми sc-элементами, в которые эти дуги входят. Такую пару инцидентности будем называть \textbf{\textit{парой инцидентности входящей sc-дуги}}.}

        \end{scnsubstruct}
    \end{scnsubstruct}
\end{scnsubstruct}
\scnsourcecommentinline{Завершили Сегмент \scnqqi{Описание Ядра SC-кода}}

        \scnsegmentheader{SC-код как синтаксическое расширение Ядра SC-кода}
\begin{scnsubstruct}
        \scnstructheader{Сравнение SC-кода и Ядра SC-кода}
        \begin{scnsubstruct}
            \scnheader{следует отличать*}
            \begin{scnhaselementset}
                \scnitem{SC-код}
                \begin{scnindent}
                    \scnrelto{синтаксическое расширение языка}{Ядро SC-кода}
                    \scnidtf{Синтаксическое расширение Ядра SC-кода}
                    \begin{scnindent}
                        \scntext{примечание}{Синтаксическое расширение Ядра SC-кода заключается во введении дополнительного класса синтаксически эквивалентных элементарных фрагментов конструкций Ядра SC-кода --- sc-элементов, обозначающих \scnqq{внутренние} файлы, хранимые в памяти ostis-системы}
                    \end{scnindent}
                    \scnrelboth{семантическая эквивалентность языков}{Ядро SC-кода}
                    \scntext{примечание}{Семантическая эквивалентность \textit{SC-кода} и \textit{Ядра SC-кода} является следствием того, что \textit{SC-код} является \uline{синтаксическим} расширением \textit{Ядра SC-кода}.}
                    \scnidtf{Результат введения в \textit{Ядро SC-кода} sc-узлов, имеющих содержимое и обозначающих файлы, хранимые в памяти ostis-системы, т.е. внутренние файлы ostis-системы.}
                    \scntext{примечание}{Результатом просмотренного расширения \textit{Ядра SC-кода} является расширение \textit{Алфавита Ядра SC-кода}.}
                    \scntext{примечание}{Все \textit{файлы}, представляющие собой электронные образы инородных для \textit{SC-кода} информационных конструкций, можно представить в \textit{SC-коде} с помощью графовых структур, в которых \textit{sc-элементы} обозначают буквы текстов или пиксели изображений. Но такой вариант кодирования внешних для \textit{ostis-системы} информационных конструкций не дает возможности непосредственно использовать накопленный человечеством арсенал электронных информационных ресурсов.}
                    \scntext{примечание}{Важнейшим видом внутренних \textit{файлов ostis-систем} являются внутренние файлы \textit{внешних идентификаторов sc-элементов} (в частности, имен sc-элементов), представляющих \textit{sc-элементы} в текстах внешних языков (в том числе, в текстах \textit{SCs-кода} и \textit{SCn-кода}).}
                \end{scnindent}
                \scnitem{\scnnonamednode}
                \begin{scnindent}
                     \begin{scneqtoset}
                        \scnitem{Ядро SC-кода}
                        \scnitem{SC-код}
                    \end{scneqtoset}
                    \scntext{сравнение}{Множество всех элементов конструкций \textit{Ядра SC-кода} и Множество всех элементов конструкций \textit{SC-кода} полностью совпадают, т.к. для каждого элемента конструкции \textit{Ядра SC-кода} существует синонимичный ему элемент конструкции \textit{SC-кода} и наоборот. Из этого следует, что семантическая классификации \textit{элементов информационных конструкций} \textit{SC-кода} и \textit{Ядра SC-кода} также полностью совпадают.Семантика \textit{SC-кода} ничем не отличается от семантики \textit{Ядра SC-кода}. То есть все, что может быть обозначено и описано текстами \textit{SC-кода}, может быть обозначено и описано текстами \textit{Ядра SC-кода}. Отличие \textit{SC-кода} от \textit{Ядра SC-кода} заключается только в том, что в \textit{SC-код} добавляется новый синтаксически выделяемый класс sc-элементов --- класс sc-элементов, являющихся знаками конкретных (константных) файлов, хранимых в памяти ostis-системы. Такие \textit{внутренние файлы} необходимы для того, чтобы в \textit{памяти ostis-системы} можно было хранить и обрабатывать \textit{информационные конструкции}, не являющиеся текстами \textit{SC-кода}, что необходимо при вводе (восприятии) информации, поступающей извне, а также при генерации \textit{информационных конструкций}, передаваемых другим субъектам.
                    	\\Включение в \textit{SC-код} специальных \uline{синтаксически} выделяемых \textit{sc-узлов}, обозначающих хранимые в \textit{sc-памяти} электронные образы (файлы) различного вида \textit{информационных конструкций}, не являющихся конструкциями \textit{SC-кода}, дает возможность непосредственно в \textit{памяти ostis-системы}, то есть в одной и той же запоминающий среде обрабатывать не только конструкции \textit{SC-кода}, но и \scnqq{инородные} для него конструкции, что для необходимо для реализации \textit{интерфейса ostis-системы}, обеспечивающего ее взаимодействие с \textit{внешней средой}. Без такой реализации \textit{интерфейса ostis-системы} невозможно реализовать синтаксический анализ, семантический анализ и понимание, а также невозможно реализовать синтез (генерацию) внешних информационных конструкций, принадлежащих заданному внешнему языку и семантически эквивалентных заданному смыслу. 
                    	\\Поскольку все синтаксические и семантические свойства \textit{SC-кода} и \textit{Ядра SC-кода} являются весьма близкими, при описании \textit{SC-кода} акцентируется внимание на его отличия от \textit{Ядра SC-кода}, а также на более детальное рассмотрение семантической классификации элементов.}
                \end{scnindent}
            \end{scnhaselementset}
        \end{scnsubstruct}

        \scnstructheader{Синтаксис SC-кода}
        \begin{scnsubstruct}
            \scnheader{SC-код}
            \scnrelfrom{синтаксис}{синтаксис SC-кода}
            \begin{scnindent}
                \scntext{примечание}{\textit{Синтаксис SC-кода} отличается от \textit{Синтаксиса Ядра SC-кода} только тем, что в \textit{Алфавит \mbox{SC-кода}} дополнительно вводится класс sc-узлов, являющихся знаками \textit{файлов}, хранимых в памяти \textit{\mbox{ostis-системы}}}
            \end{scnindent}
            \scnrelfrom{множество всех экземпляров конструкций данного языка}{sc-элемент}
            \begin{scnindent}
                \scnidtf{элемент конструкции SC-кода}
                \scntext{примечание}{Множество всех элементов конструкций SC-кода совпадает со множеством всех элементов конструкций Ядра SC-кода. Просто в конструкциях SC-кода некоторые sc-элементы, имеющие \scnqq{синтаксическую метку} (синтаксический тип) \textit{sc-узла общего вида}, будут иметь \scnqq{метку} sc-узла, являющегося знаком \textit{внутреннего файла},  хранимого в памяти \textit{ostis-системы}.}
            \end{scnindent}
            \scntext{пояснение}{\textit{Синтаксис SC-кода} задается
            \begin{scnitemize}
                \item типологией (алфавитом) sc-элементов (атомарных фрагментов текстов SC-кода);
                \item правилами соединения (инцидентности) sc-элементов (например, sc-элементы каких типов не могут быть инцидентными друг другу);
                \item типологией конфигураций sc-элементов (связки, классы, структуры), связями между конфигурациями sc-элементов (в частности, теоретико-множественными).
            \end{scnitemize}
            }
            
            \scnheader{Алфавит SC-кода}
            \scnrelto{алфавит}{SC-код}
            \scnrelfrom{разбиение}{sc-элемент}
            \begin{scneqtoset}
                  	\scnitem{(Алфавит Ядра SC-кода $\cup$ \scnkeyword{внутренний файл ostis-системы})}
            \end{scneqtoset}
            \begin{scneqtoset}
                \scnitem{sc-узел общего вида}
                \scnitem{\textit{\scnkeyword{внутренний файл ostis-системы}}} 
                \scnitem{sc-ребро общего вида}
                \scnitem{sc-дуга общего вида}
                \scnitem{базовая sc-дуга}
            \end{scneqtoset}
            
            \scnheader{Алфавит SC-кода}
            \scnidtf{Алфавит sc-элементов в рамках SC-кода}
            \scnidtf{Семейство всех максимальных множеств синтаксически эквивалентных (в рамках SC-кода) sc-элементов}
            \scnidtf{Семейство классов синтаксически эквивалентных sc-элементов SC-кода}
            \scnidtf{Семейство всех множеств, в каждое из которых входят все синтаксически эквивалентные друг другу (в рамках SC-кода) sc-элементы и только они}
            \scnidtf{Фактор-множество отношения \scnqq{синтаксическая эквивалентность sc-элементов в рамках SC-кода}}
            \scneq{фактор-множество*(синтаксическая эквивалентность sc-элементов в рамках SC-кода*)}
            \begin{scnindent}
                \scniselement{сложный внешний идентификатор sc-элемента}
            \end{scnindent}
            \scnidtf{Семейство множеств sc-элементов, являющихся результатом разбиения максимального множества \mbox{sc-элементов} SC-кода (класса всевозможных sc-элементов) по признаку синтаксической эквивалентности sc-элементов}
            \scnidtf{Признак (параметр) синтаксической эквивалентности sc-элементов}
            
            \scnheader{внутренний файл ostis-системы}
            \scnidtf{sc-узел, имеющий содержимое}
            \scnidtf{sc-ссылка}
            \scnidtf{множество всевозможных sc-узлов, имеющих содержимое, хранимое в памяти ostis-системы}
            \scnidtf{внутренний файл, хранимый в памяти ostis-системы}
            \scnidtf{внутренний файл для заданной ostis-системы (той ostis-системы, в памяти которой хранится sc-узел, обозначающий этот файл)}
            \scnidtf{sc-узел, являющийся знаком конкретного файла, хранимого в той же sc-памяти (в качестве содержимого sc-узла), в которой находится и сам указанный sc-узел}
            \scnidtf{файл, знак которого находится в той же sc-памяти, в которой находится и сам файл}
            \scnidtf{свой файл ostis-системы}
            \scnsubset{внутренняя информационная конструкция}
            
            \scnheader{синтаксически выделяемый класс sc-элементов в рамках SC-кода}
            \scnidtf{класс sc-элементов, определяемый на основе Алфавита SC-кода}
            \scnsuperset{Алфавит SC-кода}
            \scnhaselement{sc-узел, не являющийся знаком внутреннего файла ostis-системы}
        \end{scnsubstruct}
        
        \scnstructheader{Синтаксическая классификация sc-элементов в рамках SC-кода}
        \begin{scnsubstruct}                   
            \scnheaderlocal{sc-элемент}
            \begin{scnsubdividing}
                \scnitem{sc-узел общего вида}
                \begin{scnindent}
                    \begin{scnsubdividing}
                        \scnitem{sc-узел, не являющийся знаком внутреннего файла ostis-системы}
                        \scnitem{внутренний файл ostis-системы}
                    \end{scnsubdividing}
                \end{scnindent}
                \scnitem{sc-коннектор}
                \begin{scnindent}
                    \begin{scnsubdividing}
                        \scnitem{sc-ребро общего вида}
                        \scnitem{sc-дуга}
                        \begin{scnindent}
                            \begin{scnsubdividing}
                                \scnitem{sc-дуга общего вида}
                                \scnitem{базовая sc-дуга}
                            \end{scnsubdividing}
                        \end{scnindent}
                    \end{scnsubdividing}
                \end{scnindent}
            \end{scnsubdividing}
            \scntext{примечание}{Данная \textit{Синтаксическая классификация sc-элементов} от \textit{Синтаксической классификации sc-элементов Ядра SC-кода} отличается только дополнительным уточнением синтаксической типологии \textit{sc-узлов}.}
        \end{scnsubstruct}

        \scnstructheader{Денотационная семантика SC-кода}
        \begin{scnsubstruct}
            \scnheader{Денотационная семантика SC-кода}
            \scntext{аннотация}{\textit{Денотационную семантику SC-кода} рассмотрим как расширение и уточнение \textit{Денотационной семантики Ядра SC-кода} (смотрите предыдущий сегмент \scnqq{\textit{Описание Ядра SC-кода}}). Изложение построим как последовательное уточнение следующих понятий:
            \begin{scnitemize}
                \item \textit{sc-переменная}
                \item \textit{обозначение дискретной информационной конструкции}
                \item \textit{дискретная информационная конструкция} (рассмотрим различные параметры и отношения, заданные на множестве дискретных информационных конструкций)
                \item \textit{знание} (как частный вид дискретных информационных конструкций)
                \item \textit{файл} (как \textit{sc-константа}, являющаяся \textit{обозначением файла})
                \item \textit{внутренний файл ostis-системы}
                \item \textit{структура} (как \textit{дискретная информационная конструкция}, принадлежащая \textit{SC-коду})
            \end{scnitemize}
            }
            
            \scnheader{SC-код}
            \scnrelfrom{денотационная семантика}{Денотационная семантика SC-кода}
            \begin{scnindent}
                \scntext{пояснение}{\textit{Денотационная семантика SC-кода} задается
                \begin{scnitemize}
                    \item семантической интерпретацией sc-элементов и их конфигураций;
                    \item семантической интерпретацией инцидентности sc-элементов;
                    \item иерархической системой \textit{предметных областей};
                    \item структурой используемых понятий в каждой предметной области (исследуемые классы объектов, исследуемые отношения, исследуемые классы объектов отношений из смежных предметных областей, ключевые экземпляры исследуемых классов объектов);
                    \item \textit{онтологиями предметных областей};
                \end{scnitemize}
                }
            \end{scnindent}
        \end{scnsubstruct}    

        \scnstructheader{Классификация sc-переменных}
        \begin{scnsubstruct}
            \scnheader{sc-переменная}
            \scnidtf{sc-элемент, представляющий собой обозначение произвольной (переменной) сущности из некоторого дополнительно уточняемого множества обозначений других сущностей, которые считаются возможными значениями указанной произвольной сущности}
            \scnrelto{область задания}{значение переменной*}
            \begin{scnindent}
                \scnidtf{Бинарное ориентированное отношение, связывающее sc-переменные с их возможными значениями*}
                \scntext{пояснение}{Это одно из отношений, заданных на множестве sc-переменных}
            \end{scnindent}
            \scnrelfrom{разбиение}{\scnkeyword{Структурная типология sc-переменных}}
            \begin{scnindent}
                \begin{scneqtoset}
                    \scnitem{произвольная терминальная сущность}
                        \begin{scnindent}
                            \scnidtf{sc-переменная, обозначающая терминальную сущность}
                            \scnidtf{sc-переменная, значением или значением значения и т.д. которой является терминальная сущность}
                            \scnidtf{sc-переменная, \scnqq{конечным} значением которой является терминальная сущность}
                            \scnidtf{обозначение произвольной терминальной сущности}
                        \end{scnindent}
                    \scnitem{произвольное множество sc-элементов}
                \end{scneqtoset}
            \end{scnindent}
            \begin{scnsubdividing}
                \scnitem{sc-переменная, у которой логический уровень всех ее значений одинаков}
                \begin{scnindent}
                      	\scnsuperset{первичная sc-переменная}
                    \begin{scnindent}
                        \scnidtf{sc-переменная, все значения которой являются sc-константами}
                    \end{scnindent}
                    \scnsuperset{вторичная sc-переменная}
                    \begin{scnindent}
                        \scnidtf{sc-переменная, все значения которой являются первичными sc-переменными}
                    \end{scnindent}
                    \scnsuperset{sc-переменная третьего уровня}
                    \begin{scnindent}    
                        \scnidtf{sc-переменная, все значения которой являются вторичными sc-переменными}
                    \end{scnindent}
                \end{scnindent}
                \scnitem{sc-переменная, значения которой имеют различный логический уровень}
            \end{scnsubdividing}
            \begin{scnsubdividing}
                \scnitem{sc-переменная, у которой синтаксический тип всех её значений одинаков}
                \begin{scnindent}
                    \scnsuperset{переменный sc-узел}
                    \begin{scnindent}
                        \scnidtf{sc-переменная, все значения которой являются sc-узлами}
                    \end{scnindent}
                    \scnsuperset{переменное sc-ребро}
                    \scnsuperset{переменная sc-дуга}
                \end{scnindent}
                \scnitem{sc-переменная, значения которой имеют различный синтаксический тип}
            \end{scnsubdividing}
            
            \scnheader{обозначение дискретной информационной конструкции}
            \begin{scnsubdividing}
                \scnitem{обозначение дискретной информационной конструкции, не принадлежащей SC-коду}
                \scnitem{\scnkeyword{обозначение структуры}}
                \begin{scnindent}
                    \scnidtf{обозначение дискретной информационной конструкции, принадлежащей SC-коду}
                    \scnidtf{обозначение sc-конструкции}
                \end{scnindent}
            \end{scnsubdividing}
            \begin{scnsubdividing}
                \scnitem{произвольная дискретная информационная конструкция}
                \begin{scnindent}
                      	\scnidtf{sc-переменная, обозначающая дискретную информационную конструкцию}
                \end{scnindent}
                \scnitem{\scnkeyword{дискретная информационная структура}}
                \begin{scnindent}
                    \scnidtf{sc-константа, обозначающая конкретную дискретную информационную конструкцию}
                \end{scnindent}
            \end{scnsubdividing}
        \end{scnsubstruct}

        \scnstructheader{Описание параметров и отношений, заданных на дискретных информационных конструкциях}
        \begin{scnsubstruct}
            \scnheader{параметр, заданный на множестве дискретных информационных конструкций\scnsupergroupsign}
            \scnhaselement{типология дискретных информационных конструкций, определяемая их носителем\scnsupergroupsign}
            \begin{scnindent}
                \scnhaselement{некомпьютерная форма представления дискретных информационных конструкций\scnsupergroupsign}
                \scnhaselement{файл}
                \begin{scnindent}
                    \scnidtf{компьютерная форма предcтавления дискретных информационных конструкций в линейной адресной памяти}
                \end{scnindent}
                \scnhaselement{структура}
                \begin{scnindent}
                    \scnidtf{компьютерная форма представления дискретных информационных конструкций в графодинамической ассоциативной памяти}
                    \scnidtf{представление дискретных информационных конструкций в виде конструкций SC-кода в памяти ostis-систем}
                \end{scnindent}
            \end{scnindent}
            \scnhaselement{типология дискретных информационных конструкций, определяемая их соотношением с памятью ostis-систем\scnsupergroupsign}
            \begin{scnindent}
                \scnhaselement{внешняя дискретная информационная конструкция ostis-системы}
                \begin{scnindent}
                    \scnidtf{дискретная информационная конструкция, которая находится вне памяти той ostis-системы, в которой находится sc-узел, обозначающий эту информационную конструкцию}
                    \begin{scnsubdividing}
                        \scnitem{некомпьютерная форма представления дискретных информационных конструкций}
                        \begin{scnindent}
                            \scntext{примечание}{Очевидно, что информационные конструкции такого вида принципиально не могут быть внутренними информационными конструкциями ostis-систем, хранимыми в их памяти.}
                        \end{scnindent}
                        \scnitem{внешний файл ostis-системы}
                        \begin{scnindent}
                            \begin{scnsubdividing}
                                \scnitem{файл компьютерной системы, которая не является ostis-системой}
                                \begin{scnindent}
                                    \scnidtf{файл, который не хранится в памяти данной ostis-системы, но о которой известно, какая система, не являющаяся ostis-системой, им \scnqq{владеет} и как его \scnqq{скачать}}
                                    \scnidtf{внешний файл ostis-системы, принадлежащий компьютерной системе, которая не является ostis-системой}
                                \end{scnindent}
                                \scnitem{файл другой ostis-системы}
                                \begin{scnindent}
                                    \scnidtf{файл, который не является внутренним файлом данной ostis-системы, в памяти которой находится знак этого файла, но является внутренним знаком другой ostis-системы}
                                    \scnidtf{внешний файл ostis-системы, принадлежащий другой ostis-системе}
                                \end{scnindent}
                            \end{scnsubdividing}
                        \end{scnindent}
                        \scnitem{внешняя структура ostis-системы}
                        \begin{scnindent}
                            \scnidtf{структура, хранимая в памяти другой ostis-системы}
                            \scnidtf{структура другой ostis-системы}
                        \end{scnindent}
                    \end{scnsubdividing}
                \end{scnindent}
            \end{scnindent}
            \scnhaselement{внутренняя информационная конструкция ostis-системы}
            \begin{scnindent}
                \scnidtf{внутренняя для заданной ostis-системы информационная конструкция}
                \scnidtf{внутренняя информационная конструкция той ostis-системы, в памяти (sc-памяти) которой хранится знак (sc-узел) этой информационной конструкции}
                \scntext{примечание}{Внутренние информационные конструкции ostis-систем (т.е. конструкции, обрабатываемые в их памяти) могут быть только дискретными, хотя и не обязательно знаковыми.}
                \begin{scnsubdividing}
                    \scnitem{внутренний файл ostis-системы}
                    \scnitem{внутренняя структура}
                    \begin{scnindent}
                        \scnidtf{структура, которой в памяти данной ostis-системы соответствует не только знак этой структуры, но и она сама}
                        \scnidtf{структура, хранимая и обрабатываемая в памяти данной ostis-системы}
                        \scnidtf{внутренняя структура ostis-системы}
                    \end{scnindent}
                \end{scnsubdividing}
                \begin{scnsubdividing}
                    \scnitem{сформированная внутренняя информационная конструкция ostis-системы}
                    \scnitem{частично сформированная внутренняя информационная конструкция ostis-системы}
                    \scnitem{внутренняя информационная конструкция ostis-системы на начальной стадии формирования}
                \end{scnsubdividing}
            \end{scnindent}
            \scnhaselement{типология дискретных информационных конструкций, определяемая правилами, которым они должны удовлетворять\scnsupergroupsign}
            \scnhaselement{информационная конструкция Русского языка}
            \scnhaselement{информационная конструкция Английского языка}
            \scnhaselement{структура}
            \begin{scnindent}
                \scnidtf{информационная конструкция SC-кода}
            \end{scnindent}
            \scnhaselement{sc.g-конструкция}
            \begin{scnindent}
                \scnidtf{информационная конструкция SCg-кода}
            \end{scnindent}
            \scnhaselement{sc.s-конструкция}
            \begin{scnindent}
                \scnidtf{информационная конструкция SCs-кода}
            \end{scnindent}
            \scnhaselement{sc.n-конструкция}
            \begin{scnindent}
                \scnidtf{информационная конструкция SCn-кода}
            \end{scnindent}
            \scnhaselement{наличие синтаксической связности\scnsupergroupsign}
            \begin{scnindent}
                \scnhaselement{синтаксически связная дискретная информационная конструкция}
                \begin{scnindent}
                    \scnidtftext{определение}{дискретная информационная конструкция, у которой для каждой пары её элементов существует маршрут, соединяющий эти элементы и проходящий по связям их инцидентности}
                \end{scnindent}
                \scnhaselement{синтаксически несвязная дискретная информационная конструкция}
                \scntext{примечание}{Можно оценивать \scnqq{силу} синтаксической связности --- наличие и число \scnqq{мостов} в графе инцидентности элементов дискретной информационной конструкции, наличие и число точек \scnqq{сочленения}, минимальное число элементов конструкции, удаление которых приводит к несвязности. Можно также оценивать уровень синтаксической несвязности дискретной информационной конструкции числом компонентов связности этой конструкции.}
            \end{scnindent}
            \scnhaselement{наличие семантической связности\scnsupergroupsign}
            \begin{scnindent}
                \scntext{примечание}{Свойством семантической связности могут обладать только знаковые конструкции.}
                \scnhaselement{семантически связная дискретная информационная конструкция}
                \begin{scnindent}
                    \scntext{определение}{Это конструкция, которая обладает следующим свойством: для любой ее декомпозиции на два синтаксически правильных компонента всегда найдется пара синонимичных знаков, один из которых находится в одном компоненте, а другой --- в другом.}
                \end{scnindent}
                \scnhaselement{семантически несвязная дискретная информационная конструкция}
            \end{scnindent}
            
            \scnheader{отношение, заданное на множестве дискретных информационных конструкций\scnsupergroupsign}
            \scnhaselement{дискретная информационная конструкция заданного языка*}
            \begin{scnindent}
                \scniselement{отношение, заданное на множестве языков\scnsupergroupsign}
                \begin{scnsubdividing}
                    \scnitem{синтаксически неправильная дискретная информационная конструкция заданного языка*}
                    \begin{scnindent}
                        \begin{scnreltoset}{объединение}
                            \scnitem{синтаксически некорректная дискретная информационная конструкция заданного языка*}
                            \scnitem{синтаксически нецелостная дискретная информационная конструкция заданного языка*}
                        \end{scnreltoset}
                    \end{scnindent}
                    \scnitem{\scnkeyword{текст заданного языка*}}
                    \begin{scnindent}
                        \begin{scnreltoset}{пересечение}
                            \scnitem{синтаксически корректная дискретная информационная конструкция заданного языка*}
                            \scnitem{синтаксически целостная дискретная информационная конструкция заданного языка*}
                        \end{scnreltoset}
                        \begin{scnsubdividing}
                            \scnitem{семантически неправильный текст заданного языка*}
                            \begin{scnindent}
                                \begin{scnreltoset}{объединение}
                                    \scnitem{семантически некорректный текст заданного языка*}
                                    \scnitem{семантически нецелостный текст заданного языка*}
                                \end{scnreltoset}
                            \end{scnindent}
                            \scnitem{знание, представленное в заданном языке*}
                            \begin{scnindent}
                                \begin{scnreltoset}{пересечение}
                                    \scnitem{семантически корректный текст заданного языка*}
                                    \scnitem{семантически целостный текст заданного языка*}
                                \end{scnreltoset}
                            \end{scnindent}
                        \end{scnsubdividing}
                    \end{scnindent}
                \end{scnsubdividing}
            \end{scnindent}

            \scnheader{обозначение файла}
            \begin{scnsubdividing}
                \scnitem{произвольный файл}
                \begin{scnindent}
                      	\scnidtf{sc-переменная, каждым значением которой является обозначение файла}
                    \scnidtf{обозначение произвольного файла}
                    \scnidtf{sc-переменнная, обозначающая файл}
                \end{scnindent}
                \scnitem{\scnkeyword{файл}}
                \begin{scnindent}
                      	\scnidtf{знак конкретного файла}
                    \scnidtf{sc-константа, обозначающая конкретный файл}
                \end{scnindent}
            \end{scnsubdividing}
            \begin{scnsubdividing}
                \scnitem{обозначение внешнего файла ostis-системы}
                \begin{scnindent}
                    \begin{scnsubdividing}
                        \scnitem{произвольный внешний файл ostis-системы}
                        \scnitem{внешний файл ostis-системы}
                    \end{scnsubdividing}
                \end{scnindent}
                \scnitem{обозначение внутреннего файла ostis-системы}
                \begin{scnindent}
                    \begin{scnsubdividing}
                        \scnitem{произвольный внутренний файл ostis-системы}
                        \scnitem{внутренний файл ostis-системы}
                    \end{scnsubdividing}
                \end{scnindent}
            \end{scnsubdividing}
            
            \scnheader{файл}
            \scnidtf{sc-узел, обозначающий файл}
            \scnidtf{знак файла}
            \begin{scnsubdividing}
                \scnitem{ея-файл}
                \begin{scnindent}
                      	\scnidtf{естественно-языковой файл}
                \end{scnindent}
                \scnitem{файл, являющийся текстом формального языка}
                \begin{scnindent}
                      	\scnsuperset{sc.g-файл}
                    \scnsuperset{sc.s-файл}
                    \scnsuperset{sc.n-файл}
                \end{scnindent}
                \scnitem{файл, отражающий процесс изменения sc.g-текста}
                \scnitem{графический файл}
                \scnitem{файл-изображение}
                \scnitem{видео-файл}
                \scnitem{аудио-файл}
            \end{scnsubdividing}
            \begin{scnsubdividing}
                \scnitem{файл-экземпляр}
                \begin{scnindent}
                      	\scnidtf{файл, являющийся конкретным электронным документом или электронным образом конкретной внешней информационной конструкции}
                \end{scnindent}
                \scnitem{файл-образец}
                \begin{scnindent}
                      	\scnidtf{файл-класс ostis-системы}
                    \scnidtf{файл, являющийся одновременно также и знаком множества всевозможных экземпляров (копий) этого файла}
                \end{scnindent}
            \end{scnsubdividing}
            \begin{scnsubdividing}
                \scnitem{внешний файл ostis-системы}
                \scnitem{\scnkeyword{внутренний файл ostis-системы}}
            \end{scnsubdividing}
            
            \scnheader{внутренний файл ostis-системы}
            \scniselement{синтаксически выделяемый класс sc-элементов в рамках SC-кода}
            \scniselement{семантически выделяемый класс sc-элементов в рамках SC-кода}
            \scntext{примечание}{Данный класс sc-элементов, являющихся знаками файлов, хранимых в памяти ostis-систем, в отличие от других синтаксически выделяемых классов sc-элементов, представляет собой одновременно  синтаксически и семантически выделяемый класс sc-элементов. Это обусловлено (1) тем, что каждый экземпляр данного класса sc-элементов является знаком конкретного файла, хранимого в памяти ostis-системы, и (2) тем, что каждый файл, хранимый в памяти ostis-системы, может и должен быть обозначен только таким sc-элементом, который является экземпляром рассматриваемого класса sc-элементов.}
            \scntext{примечание}{sc-узел может быть знаком файла, находящегося в памяти другой ostis-системы (не в той, в которой хранится этот sc-узел). Но в этом случае указанный sc-узел не будет принадлежать рассматриваемому классу sc-узлов.}
            \scnidtf{знак файла ostis-системы, хранимого в \scnqq{моей} памяти}
            \scntext{примечание}{Следует отличать синтаксическую эквивалентность файлов, семантическую эквивалентность файлов и совпадение файлов (когда речь идет об одном и том же файле). Т.е. копия файла и один и тот же файл --- это разные вещи.}
        \end{scnsubstruct}


        \scnstructheader{Классификация структур}
        \begin{scnsubstruct}
            \scnheader{структура}
            \scnidtf{структура, элементами которой являются sc-элементы}
            \scnidtf{множество sc-элементов (множество), не являющиеся ни связкой (связкой sc-элементов), ни классом (множеством всех sc-элементов, эквивалентных в определенном смысле)}
            \scnidtf{знак конкретной (константной) структуры}
            \scnidtf{Класс всех тех и только тех sc-элементов, каждый из которых является знаком конкретной структуры}
            \scnidtf{Знак класса всех sc-элементов, являющихся знаками конкретных структур}
            \scnidtf{Константный sc-элемент (точнее, sc-узел), являющийся знаком конкретного класса всех sc-элементов, являющихся знаками конкретных структур}
            \scnidtf{sc-конструкция}
            \scnidtf{информационная конструкция, принадлежащая SC-коду}
            \scnsuperset{\scnkeyword{sc-текст}}
            \begin{scnindent}
                \scnidtf{структура, удовлетворяющая синтаксическим правилам SC-кода}
                \scnsuperset{\scnkeyword{знание}}
                \begin{scnindent}
                    \scnidtf{семантически корректный и целостный sc-текст}
                \end{scnindent}
            \end{scnindent}        
            \scnrelfrom{разбиение}{\scnkeyword{Наличие sc-переменных, входящих в состав структуры}}
            \begin{scnindent}
                \begin{scneqtoset}
                    \scnitem{структура, в составе которой sc-переменные не входят}
                    \scnitem{структура, в состав которой входят sc-переменные}
                    \begin{scnindent}
                        \scntext{примечание}{Такие структуры при представлении логических высказываний в SC-коде являются аналогами атомарных логических формул.}
                    \end{scnindent}
                \end{scneqtoset}
            \end{scnindent}
            \scnrelfrom{разбиение}{\scnkeyword{Темпоральная характеристика структур}}
            \begin{scnindent}
                \begin{scneqtoset}
                    \scnitem{ситуативная структура}
                    \begin{scnindent}
                        \scnidtf{ситуация, представленная в SC-коде}
                        \scnidtf{ситуация}
                        \scnidtf{структура, в состав которой входят знаки временных сущностей и которая сама является временной сущностью (при этом время существования такой структуры совпадает с временем одновременного существования всех временных сущностей, знаки которых входят в состав этой ситуативной структуры).}
                        \begin{scnsubdividing}
                            \scnitem{ситуация во внешней среде}
                            \scnitem{ситуация в sc-памяти}
                        \end{scnsubdividing}
                    \end{scnindent}
                    \scnitem{структура, не содержащая знаков временных сущностей}
                    \scnitem{динамическая структура}
                    \begin{scnindent}
                        \scntext{пояснение}{В отличие от ситуативной структуры конфигурация динамической структуры меняется во времени в зависимости от момента появления и момента завершения существования каждой временной сущности (в том числе временной связи), знак которой входит в состав динамической структуры. Каждой динамической структуре можно поставить в соответствие темпоральную последовательность состояний (ситуаций) и событий.}
                    \end{scnindent}
                \end{scneqtoset}
            \end{scnindent}
            \scnrelfrom{разбиение}{\scnkeyword{Наличие связности структур}}
            \begin{scnindent}
                \begin{scneqtoset}
                    \scnitem{связная структура}
                    \begin{scnindent}
                        \scnrelfrom{разбиение}{Связность структур\scnsupergroupsign}
                        \begin{scnindent}
                            \scnidtf{Минимальное число sc-элементов, удаление которых преобразует связную структуру в несвязную}
                            \scniselement{одно-связная структура}
                            \begin{scnindent}
                                \scnrelto{разбиение}{Признак классификации структур по числу точек сочленения\scnsupergroupsign}
                                \scnrelto{разбиение}{Признак классификации структур по числу мостов\scnsupergroupsign}
                            \end{scnindent}
                        \end{scnindent}
                    \end{scnindent}
                    \scnitem{несвязная структура}
                    \begin{scnindent}
                        \scnrelto{разбиение}{Признак классификации структур по числу компонентов связности\scnsupergroupsign}
                        \scnsubset{тривиальная структура}
                        \begin{scnindent}
                            \scnidtf{структура, в состав элементов которой sc-коннекторы не входят}
                        \end{scnindent}
                    \end{scnindent}
                \end{scneqtoset}
                \scntext{примечание}{Важнейшей особенностью SC-кода является то, что для конструкций SC-кода (для структур) нет необходимости противопоставлять синтаксическую и семантическую связность, то есть все синтаксически связные структуры являются также и семантически связными и наоборот.}
            \end{scnindent}
            \scnrelfrom{разбиение}{\scnkeyword{Рефлексивность структур}}
            \begin{scnindent}
                \begin{scneqtoset}
                    \scnitem{рефлексивная структура}
                    \begin{scnindent}
                        \scnidtf{структура, в число элементов которой входит sc-узел, обозначающий саму эту структуру}
                        \scnsubset{рефлексивное множество}
                    \end{scnindent}
                    \scnitem{нерефлексивная структура}
                \end{scneqtoset}
            \end{scnindent}
            \scnrelfrom{разбиение}{\scnkeyword{Целостность структур по связкам}}
            \begin{scnindent}
                \begin{scneqtoset}
                    \scnitem{структура, содержащая все компоненты всех своих связок}
                    \scnitem{структура, не содержащая все компоненты всех своих связок}
                \end{scneqtoset}
            \end{scnindent}
            \begin{scnsubdividing}
                \scnitem{синтаксически неправильная структура}
                \begin{scnindent}
                      	\scnidtf{синтаксически неправильно построенная структура}
                    \begin{scnreltoset}{объединение}
                        \scnitem{синтаксически некорректная структура}
                        \begin{scnindent}
                            \scnidtf{структура, содержащая фрагменты, противоречащие \textit{Синтаксическим правилам SC-кода} (ошибочные фрагменты)}
                        \end{scnindent}
                        \scnitem{синтаксически нецелостная структура}
                        \begin{scnindent}
                            \scnidtf{структура, в которой имеется синтаксически выявленная недостаточность, неполнота (то есть имеется некоторое количество информационных дыр)}
                        \end{scnindent}
                    \end{scnreltoset}
                    \scntext{примечание}{Разделение \textit{Синтаксических правил SC-кода} на правила анализа синтаксической корректности и правила анализа синтаксической целостности (полноты) существенно упрощает процедуру синтаксического анализа \textit{структур}.}
                \end{scnindent}
                \scnitem{\scnkeyword{sc-текст}}
                \begin{scnindent}
                    \scnidtf{синтаксически правильная структура}
                    \scnidtf{синтаксически правильно построенная структура}
                    \begin{scnreltoset}{пересечение}
                        \scnitem{синтаксически корректная структура}
                        \scnitem{синтаксически целостная структура}
                    \end{scnreltoset}
                    \begin{scnsubdividing}
                        \scnitem{семантически неправильный sc-текст}
                        \begin{scnindent}
                            \begin{scnreltoset}{объединение}
                                \scnitem{семантически некорректный sc-текст}
                                \scnitem{семантически нецелостный sc-текст}
                            \end{scnreltoset}
                        \end{scnindent}
                        \scnitem{\scnkeyword{знание}}
                        \begin{scnindent}
                            \scnidtf{семантически правильно построенный sc-текст}
                            \begin{scnreltoset}{пересечение}
                                \scnitem{семантически корректный sc-текст}
                                \scnitem{семантически целостный sc-текст}
                            \end{scnreltoset}
                        \end{scnindent}
                    \end{scnsubdividing}
                \end{scnindent}
            \end{scnsubdividing}

            \scnheader{знание}
            \scnidtf{дискретная информационная конструкция, являющаяся знанием, представленная в некотором (дополнительно уточняемом) языке}
            \scnrelto{второй домен}{знание, представленное в заданном языке*}
            \scnsubset{знаковая конструкция}
            \scntext{примечание}{Каждое знание является знаковой конструкцией, но не каждая знаковая конструкция является знанием, а только та, смысловое представление которой удовлетворяет определенным требованиям корректности и целостности.}
            \scniselement{семантически выделяемый класс дискретных информационных конструкций\scnsupergroupsign}
            
            \scnheader{следует отличать*}
            \begin{scnhaselementset}
                \scnitem{\scnnonamednode}
                \begin{scnindent}
                    \begin{scneqtoset}
                        \scnitem{дискретная информационная конструкция}
                        \scnitem{текст}
                        \scnitem{знание}
                    \end{scneqtoset}
                    \scniselement{следует отличать*}
                \end{scnindent}
                \scnitem{\scnnonamednode}
                \begin{scnindent}
                    \begin{scneqtoset}
                        \scnitem{структура}
                        \scnitem{sc-текст}
                        \scnitem{знание}
                    \end{scneqtoset}
                    \scniselement{следует отличать*}
                \end{scnindent}
            \end{scnhaselementset}
        \end{scnsubstruct}
\end{scnsubstruct}
\scnsourcecommentinline{Завершили Сегмент \scnqqi{SC-код как синтаксическое расширение Ядра SC-кода}}

        \scnsegmentheader{Использование SC-кода для формального описания собственного синтаксиса}
\begin{scnsubstruct}
    \scnheader{SC-код}
    \scntext{примечание}{В предыдущем сегменте \scnqqi{\textbf{SC-код как синтаксическое расширение Ядра SC-кода}} рассмотрен \textbf{Синтаксис SC-кода} путём:
    \begin{scnitemize}
        \item введения \textit{синтаксически выделяемых классов sc-элементов} в рамках \textit{SC-кода};
        \item описания \textit{теоретико-множественных связей} между указанными классами \textit{sc-элементов} (к такому описанию, в частности, относится \textit{Синтаксическая классификация sc-элементов в рамках SC-кода});
        \item введения двух отношений инцидентности \textit{sc-элементов} --- \textit{Отношения инцидентности sc-коннекторов*} и \textit{Отношения инцидентности входящих sc-дуг*};
        \item описания \textit{Синтаксических правил SC-кода}, которые, прежде всего, описывают формальные свойства указанных выше отношений инцидентности \textit{sc-элементов}.
    \end{scnitemize}
    Однако для того, чтобы получить возможность \uline{все} (!) \textit{Синтаксические правила SC-кода} записать средствами самого \textit{SC-кода}, необходимо иметь \uline{явное} представление \textit{пар} отношений инцидентности \textit{sc-элементов} в виде \textit{sc-дуг}, принадлежащим этим отношениям. В случае, если указанные \textit{sc-дуги} инцидентности являются \textit{sc-переменными}, логико-семантических проблем не возникнет. И этого, кстати, вполне достаточно, чтобы \textit{Синтаксические правила SC-кода}, сформулированные в виде \textit{логических высказываний}, записать средствами \textit{SC-кода}. Но, если разрешить \textit{sc-дугам} инцидентности быть \textit{sc-константами}, то, во-первых, в \textit{Синтаксические правила SC-кода} необходимо добавить Правило удаления \textit{константной sc-дуги инцидентности}, если эта инцидентность представлена неявно, а, во-вторых, в \textit{Правила синтаксической трансформации sc-элементов} необходимо добавить Правило трансформации (замены) \textit{константной sc-дуги инцидентности} на неявное представление этой инцидентности.В теоретическом и, возможно, даже в практическом плане может быть интересна такая синтаксическая модификация (синтаксическое расширение) \textit{SC-кода}, в котором: 
    \begin{scnitemize}
        \item \uline{все} неявно представленные \textit{пары инцидентности sc-элементов} заменяются на \textit{константные sc-дуги инцидентности} --- неявно представленными \textit{парами инцидентности} остаются \uline{только} \textit{пары инцидентности} константных \textit{sc-дуг} инцидентности с компонентами этих \textit{sc-дуг};
        \item В \textbf{Алфавит SC-кода} вводятся два новых \textit{синтаксически выделяемых класса sc-элементов} --- \textit{класс sc-дуг инцидентности sc-коннекторов}, а также \textit{класс sc-дуг инцидентности входящих sc-дуг}.
    \end{scnitemize}
    В результате такого преобразования конструкций \textit{SC-кода} конструкции \textit{SC-кода} перестают быть графовыми конструкциями нетрадиционного вида, в которых рёбра, гиперрёбра, дуги могут быть инцидентны другим рёбрам, гиперребрам и дугам, а становятся классическими графами с двумя типами дуг (с \textit{sc-дугами инцидентности sc-коннекторов} и с \textit{sc-дугами инцидентности входящих sc-дуг}) и с пятью типами вершин (с вершинами, представляющими \textit{sc-узлы общего вида}, с вершинами, представляющими \textit{sc-узлы}, являющиеся знаками \textit{внутренних файлов ostis-системы}, с вершинами, представляющими \textit{sc-рёбра общего вида}, с вершинами, представляющими \textit{sc-дуги общего вида}, с вершинами, представляющими \textit{базовые sc-дуги}).}
    \scntext{примечание}{Рассмотренное преобразование конструкций \textit{SC-кода} в теории графов называется поздразделением или подразбиением графа.}
    \begin{scnindent}
    	\scnrelfrom{смотрите}{\scncite{Trudeau1993}}
    \end{scnindent}
\end{scnsubstruct}
\scnsourcecommentinline{Завершили Сегмент \scnqqi{Использование SC-кода для формального описания собственного синтаксиса}}

        \scnsegmentheader{Уточнение смысла выделенных классов sc-элементов}
\begin{scnsubstruct}

\scnstructheader{Уточнение смысла выделенных классов sc-элементов в Структурной классификации sc-элементов}
\begin{scnsubstruct}
	\scnheader{sc-элемент}
	\scnidtf{обозначение множества}
	\scnidtf{sc-обозначение множества, представимого в SC-коде}
	\begin{scnsubdividing}
		\scnitem{обозначение sc-множества}
		\begin{scnindent}
			\scnidtf{обозначение множества \textit{sc-элементов}}
			\scnidtf{обозначение множества, все элементы которого являются \textit{sc-элементами}}
			\scnidtf{обозначение внутренней для sc-памяти сущности, то есть сущности, хранимой в sc-памяти}
		\end{scnindent}	
		\scnitem{обозначение внешней сущности}
		\begin{scnindent}
			\scnidtf{обозначение синглетона внешней сущности}
			\scnidtf{терминальный \textit{sc-элемент}}
		\end{scnindent}	
	\end{scnsubdividing}
	\begin{scnrelfromlist}{примечание}
			\scnfileitem{Каждый \textit{sc-элемент} является обозначением соответствующего множества.}
			\scnfileitem{Ко множествам, представимым в \textit{SC-коде}, относятся либо \textit{sc-множества}, элементами которых являются \textit{sc-элементы}, либо синглетоны, элементами которых являются сущности, не являющиеся \textit{sc-элементами} (синглетоны внешних сущностей). Таким образом, строго говоря, не каждое множество может быть обозначено соответствующим \textit{sc-элементом} и представлено в SC-коде. Но каждое множество, не являющееся \textit{sc-множеством} или синглетоном указанного выше вида может быть однозначно преобразовано в \textit{sc-множество} и описано средствами \textit{SC-кода}. При этом теоретико-множественные свойства \scnqq{нестандартных} для \textit{SC-кода} множеств совпадают со свойствами соответствующих им \scnqq{стандартных} для \textit{SC-кода} множеств.}
			\scnfileitem{Тот факт, что \textit{каждый} \textit{sc-элемент} является обозначением соответствующего множества (частным случае которых являются синглетоны \textit{внешних} описываемых сущностей), означает то, что базовым видом объектов, которыми оперирует \textit{SC-код} на синтаксическом, семантическом и логическом уровне являются множества знаков, обозначающих различные множества. В этом смысле \textit{SC-код} имеет базовую теоретико-множественную основу.}
	\end{scnrelfromlist}
	\scnrelfrom{правила построения внешних идентификаторов sc-элементов заданного класса}{Общие правила построения внешних идентификаторов sc-элементов}
	\begin{scnindent}
		\scnidtf{Общие правила идентификации \textit{sc-элементов}}
		\begin{scneqtoset}
			\scnfileitem{Принадлежность идентифицируемого \textit{sc-элемента} некоторым \textit{классам} \textit{sc-элементов} (sc-классам) явно указывается во внешнем идентификаторе этого \textit{sc-элемента} (в \textit{sc-идентификаторе}) с помощью соответствующих условных признаков.}
			\begin{scnindent}
				\begin{scnsubdividing}
					\scnfileitem{Если первым символом \textit{sc-идентификатора} является знак подчеркивания, то идентифицируемый \textit{sc-элемент} принадлежит Классу \textit{sc-переменных}. По умолчанию считается, что идентифицируемый \textit{sc-элемент} принадлежит Классу \textit{sc-констант}.}
					\scnfileitem{Если последним символом \textit{sc-идентификатора} является символ \scnqqi{звездочка}, то идентифицируемый \textit{sc-элемент} принадлежит Классу обозначений \textit{неролевых отношений}.}
					\scnfileitem{Если последним символом \textit{sc-идентификатора} является апостроф, то идентифицируемый \textit{sc-элемент} принадлежит Классу обозначений \textit{ролевых отношений}, каждое из которых является подмножеством Отношения принадлежности, то есть Класса всех \textit{константных позитивных sc-пар принадлежности}.}
					\scnfileitem{Если последним символом \textit{sc-идентификатора} является символ \scnqqi{\scnsupergroupsign}, то идентифицируемый \textit{sc-элемент} принадлежит Классу обозначений \textit{параметров}.}
				\end{scnsubdividing}
			\end{scnindent}
			\scnfileitem{Слово \scnqqi{обозначение} в \textit{sc-идентификаторе} означает то, что обозначаемая сущность может быть как константной, так и переменной.}
			\scnfileitem{В \textit{sc-идентификаторах} можно делать следующие сокращения.}
			\begin{scnindent}
				\begin{scnsubdividing}
					\scnfileitem{\scnqqi{sc-элемент}, обозначающий \ldots  --- \scnqqi{обозначение}}
					\scnfileitem{\scnqqi{обозначение константного} --- \scnqqi{знак константного}}
					\scnfileitem{\scnqqi{знак константного} --- \scnqqi{константный}}
					\scnfileitem{слово \scnqqi{константный} в \textit{sc-идентификаторах} можно опускать, так как константность подразумевается по умолчанию}
				\end{scnsubdividing}
			\end{scnindent}
			\scnfileitem{Для каждого \textit{sc-элемента} можно построить \textit{sc-идентификатор}, являющийся \textit{именем собственным}, которое всегда начинается с большой буквы (заглавной) буквы.}
			\scnfileitem{Если \textit{sc-элемент} является обозначением некоторого класса \textit{sc-элементов}, то этому \textit{sc-элементу} можно поставить в соответствие не только \textit{имя собственное}, но и \textit{имя нарицательное}, которое начинается маленькой (строчной) буквы. В спецификацию каждого sc-класса (каждого понятия) входит перечень эквивалентных (синонимичных) \textit{sc-идентификатор}, среди которых есть как \textit{имена собственные}, так и \textit{имена нарицательные}.}
		\end{scneqtoset}
	\end{scnindent} 

	\scnheader{обозначение sc-множества}
	\scnidtf{SC-элемент, являющийся знаком множества всевозможных \textit{обозначений sc-множеств}}
	\begin{scnindent}
		\scniselement{имя собственное}
	\end{scnindent} 
	\scnidtf{Знак множества всевозможных \textit{обозначений sc-множеств}}
	\begin{scnindent}
		\scniselement{имя собственное}
	\end{scnindent} 
	\scnidtf{Множество всевозможных \textit{обозначений sc-множеств}}
	\begin{scnindent}
		\scniselement{имя собственное}
	\end{scnindent} 
	\scnidtf{Класс \textit{обозначений sc-множеств}}
	\begin{scnindent}
		\scniselement{имя собственное}
	\end{scnindent} 
	\scnidtf{sc-элемент, являющийся обозначением множества \textit{sc-элементов}}
	\begin{scnindent}
		\scniselement{имя нарицательное}
	\end{scnindent} 
	\scnidtf{sc-обозначение множества \textit{sc-элементов}}
	\begin{scnindent}
		\scniselement{имя нарицательное}
	\end{scnindent}
	\scnidtf{обозначение множества, каждый элемент которого является \textit{sc-элементом}}
	\scnidtf{обозначение информационной конструкции, принадлежащей \textit{SC-коду}}
	\scnidtftext{часто используемый sc-идентификатор}{обозначение \textit{sc-конструкции}}
	\begin{scnsubdividing}
		\scnitem{sc-множество}
		\begin{scnindent}
			\scnidtf{знак константного \textit{sc-множества}}
			\scneq{\textit{(}обозначение sc-множества $ \bigcap $ sc-константа\textit{)}}
		\end{scnindent} 
		\scnitem{переменное sc-множество}
		\begin{scnindent}
			\scneq{\textit{(}обозначение sc-множества $ \bigcap $ sc-переменная\textit{)}}
		\end{scnindent}
	\end{scnsubdividing}

	\scnheader{следует отличать*}
	\begin{scnhaselementset}
		\scnitem{обозначение sc-множества}
		\begin{scnindent}
			\scnidtf{\textit{обозначение sc-множества}, которое может быть как константным sc-множеством, так и переменным sc-множеством}
			\scnidtf{обозначение внутренней для \textit{sc-памяти} сущности}
			\scnidtf{обозначение внутренней для \textit{sc-памяти} информационной конструкции (\textit{sc-конструкции})}
			\begin{scnsubdividing}
			\scnitem{sc-множество}
			\begin{scnindent}
				\scnidtf{обозначение конкретного множества}
				\scnidtf{знак множества}
				\scneq{\textit{(}sc-константа $ \bigcap $ обозначение sc-множества\textit{)}}
				\scnidtf{конкретное \textit{sc-множество}}
				\scnidtf{знак константного \textit{sc-множества}}
				\scnidtf{константное \textit{sc-множество}}
			\end{scnindent}
			\scnitem{переменное sc-множество}
			\begin{scnindent}
				\scnidtf{произвольное \textit{sc-множества}}
				\scnidtf{обозначение произвольного \textit{sc-множества}}
				\scneq{\textit{(}sc-переменная $ \bigcap $ обозначение sc-множества\textit{)}}
			\end{scnindent} 
			\end{scnsubdividing}
		\end{scnindent}
		\scnitem{sc-множество}
		\scnitem{переменное sc-множество}
		\scnitem{обозначение внешней сущности}
		\begin{scnindent}
			\scnidtf{обозначение сущности, не являющейся множеством sc-элементов (\textit{sc-множеством})}
			\scnsuperset{обозначение файла}
			\begin{scnindent}
				\scnidtf{\textit{обозначение файла}, хранимого либо в файловой памяти той же \textit{ostis-системы}, в \textit{sc-памяти} которой хранится знак этого \textit{файла}, либо в файловой памяти другой дополнительно указываемой \textit{компьютерной системы}}
			\end{scnindent} 
			\scnsuperset{обозначение информационной конструкции, не являющейся ни sc-множеством, ни файлом}
			\begin{scnindent}
				\scnnote{Примерами такой информационной конструкции являются напечатанный текст, речевое сообщение, которой следует отличать от его записи в виде аудио-файла.}
			\end{scnindent}
			\scnsuperset{обозначение внешней сущности, не являющейся информационной конструкцией}
			\begin{scnindent}
				\scnnote{Примером такой внешней сущности является любой материальный объект, не являющийся информационной конструкцией}
			\end{scnindent}
		\end{scnindent} 
	\end{scnhaselementset}

	\scnheader{sc-множество}
	\scnidtf{\textit{sc-конструкция}} 
	\scnidtf{информационная конструкция, принадлежащая \textit{SC-коду}} 
	\scnidtftext{часто используемый sc-идентификатор}{\textit{SC-код}} 
	\begin{scnindent}
		\scniselement{имя собственное}
	\end{scnindent} 
	\scnidtf{Множество всевозможных \textit{sc-конструкций}}
	\scnidtf{множество \textit{sc-элементов}, которые могут быть (но не обязательно) связаны между собой бинарными ориентированными \textit{парами инцидентности}, каждая из которых связывает некоторый \textit{sc-коннектор} с \textit{sc-элементами}, которые связываются этим \textit{sc-коннектором}}
	\scnidtf{информационная конструкция, каждый элемент (атомарный фрагмент) которой входит в состав некоторого текста, принадлежащего \textit{SC-коду}, но при этом \textit{конфигурация} всей указанной информационной конструкции не всегда позволяет считать ее \textit{текстом SC-кода}, удовлетворяющим целому ряду синтаксических и семантических требований}
	\begin{scnsubdividing}
		\scnitem{синтаксически корректная sc-конструкция}
		\scnitem{синтаксически некорректная sc-конструкция}
	\end{scnsubdividing}
	
	\scnheader{синтаксически корректная sc-конструкция}
	\scnidtf{синтаксически правильно построенная \textit{sc-конструкция}}
	
	\scnheader{правило построения синтаксически корректных sc-конструкций}
	\scnidtf{синтаксическое правило SC-кода}
	\scnidtf{требование (одно из требований), предъявляемое к \textit{синтаксически корректным sc-конструкциям}}
	\begin{scnhaselementset}
		\scnfileitem{Каждая \textit{sc-пара принадлежности}, связывающая \textit{sc-элемент}, обозначающий пару \textit{sc-элементов}, с компонентом этой пары (то есть с \textit{sc-элементом}, связываемым этой \textit{sc-парой} с другими \textit{sc-элементом}) синтаксически \scnqq{преобразуется} из \textit{sc-элемента}, обозначающего \textit{sc-пару принадлежности} в \textit{пару инцидентности sc-элементов}, которая синтаксически уже не является \textit{sc-элементом}.}
		\scnfileitem{Поскольку для каждого о \textit{бозначения sc-пары} осуществляется \textit{синтаксическая} замена \textit{sc-пар принадлежности} их элементов на \textit{пары инцидентности} этих элементов соответствующее синтаксическое преобразование происходит и с самими \textit{обозначениями sc-пар} --- они \scnqq{превращаются} в \textit{sc-коннекторы}. Соответственно этому \textit{обозначения неориентированных sc-пар} \scnqq{преобразуется} в \textit{sc-ребра}, а \textit{обозначения ориентированных sc-па}р --- в \textit{sc-дуги}.}
	\end{scnhaselementset}
	
	\scnheader{синтаксически некорректная sc-конструкция}
	\scnidtf{\textit{sc-конструкция}, содержащая одну или несколько синтаксических ошибок}
	\scnsuperset{минимальная синтаксически некорректная sc-конструкция}
	\begin{scnindent}
		\scnidtf{\textit{sc-конструкция}, не содержащая подструктур, являющихся \textit{синтаксически некорректными} \textit{sc-конструкциями}}
		\scntext{примечание}{Каждой \textit{минимальной синтаксически некорректной sc-конструкции} ставится в соответствие одно из синтаксических правил \textit{SC-кода}, которому указанная \textit{sc-конструкция} противоречит.}
	\end{scnindent}
	\scntext{примечание}{Строго говоря, \textit{синтаксически некорректные sc-конструкции} не являются \textit{sc-текстами}, то есть информационными конструкциями, принадлежащими \textit{SC-коду}}
	\scnrelto{невключение}{sc-текст}
	\begin{scnindent}
		\scnidtf{\textit{sc-конструкция} принадлежащая SC-коду}
	\end{scnindent} 
	
	\scnheader{обозначение sc-связки}
	\begin{scnsubdividing}
		\scnitem{sc-связка}
		\scnitem{переменная sc-связка}
	\end{scnsubdividing}

	\scnheader{sc-связка}
	\scnidtf{знак связи (связки) между \textit{sc-элементами}} 
	\scnnote{Если элементами \textit{sc-связки} являются знаки \textit{внешних сущностей}, то \textit{sc-связка} является отображением (моделью) некоторой связи, которая связывает указанные \textit{внешние сущности}}
	\scntext{пояснение}{Понятие \textit{sc-связки} --- это попытка формализации понятия \textit{целостности}, понятия перехода некоторой совокупности сущностей в некоторое новое качество, которое не сводится к свойствам каждой сущности, входящей в эту совокупность.
		Таким образом, связками следует считать:
		\begin{scnitemize}
			\item множество всех чисел, являющихся слагаемыми для заданного числа;
			\item множество всех сотрудников заданной организации, в заданный момент времени;
			\item множество всех сотрудников заданной организации, которые работают или работали в ней.
		\end{scnitemize}}
	\scntext{примеры}{Примерами \textit{sc-связок} являются:
		\begin{scnitemize}
			\item конкретная окружность, (множество \textit{всех} точек, равноудаленных от некоторой заданной точки);
			\item конкретный отрезок (множество \textit{всех} точек, лежащих между двумя заданными точками с включением этих точек);
			\item конкретный линейный треугольник (множество \textit{всех} точек, лежащих между каждыми двумя из трех заданных точек с включением этих точек);
			\item пары граничных точек различных отрезков;
			\item тройки вершин различных треугольников.
		\end{scnitemize}}

	\scnheader{обозначение sc-синглетона}
	\begin{scnsubdividing}
		\scnitem{sc-синглетон}
		\scnitem{переменный sc-синглетон}
	\end{scnsubdividing}

	\scnheader{sc-синглетон}
	\scnidtf{\textit{sc-множество}, являющиеся синглетоном}
	\scnidtf{одномощное \textit{sc-множество}}
	\scnidtf{\textit{sc-множество}, имеющее мощность, равную единице}
	\scnidtf{\textit{sc-элемент}, являющийся знаком унарной \textit{sc-связки}}
	\scnidtf{знак унарной \textit{sc-связки}}
	\scnidtf{унарная \textit{sc-связка}}
	\scnidtf{знак одномощного множества, единственный элемент которого является \textit{sc-элементом}}

	\scnheader{обозначение sc-пары}
	\scniselement{sc-константа}
	\scniselement{sc-класс}
	\begin{scnsubdividing}
		\scnitem{\textbf{sc-пара}}
		\begin{scnindent}
			\scnidtf{константная sc-пара}
			\scnsubset{sc-константа}
			\scniselement{sc-константа}
			\scniselement{sc-класс}
		\end{scnindent} 
		\scnitem{\textbf{переменная sc-пара}}
		\begin{scnindent}
			\scnsubset{sc-переменная}
			\scniselement{sc-константа}
			\scniselement{sc-класс}
		\end{scnindent} 
	\end{scnsubdividing}

	\scnheader{обозначение неориентированной sc-пары}
	\scniselement{sc-константа}
	\scniselement{sc-класс}
	\begin{scnsubdividing}
		\scnitem{\textbf{неориентированная sc-пара}}
		\begin{scnindent}
			\scnidtf{константная sc-пара}
			\scnsubset{sc-константа}
			\scniselement{sc-константа}
			\scniselement{sc-класс}
		\end{scnindent} 
		\scnitem{\textbf{переменная неориентированная sc-пара}}
		\begin{scnindent}
			\scnsubset{sc-переменная}
			\scniselement{sc-константа}
			\scniselement{sc-класс}
		\end{scnindent} 
	\end{scnsubdividing}

	\scnheader{обозначение ориентированной sc-пары}
	\scniselement{sc-константа}
	\scniselement{sc-класс}
	\begin{scnsubdividing}
		\scnitem{\textbf{ориентированная sc-пара}}
		\begin{scnindent}
			\scnidtf{константная sc-пара}
			\scnsubset{sc-константа}
			\scniselement{sc-константа}
			\scniselement{sc-класс}
		\end{scnindent} 
		\scnitem{\textbf{переменная ориентированна sc-пара}}
		\begin{scnindent}
			\scnsubset{sc-переменная}
			\scniselement{sc-константа}
			\scniselement{sc-класс}
		\end{scnindent} 
	\end{scnsubdividing}

	\scnheader{обозначение sc-пары принадлежности}
	\scniselement{sc-константа}
	\scniselement{sc-класс}
	\begin{scnsubdividing}
		\scnitem{\textbf{sc-пара принадлежности}}
		\begin{scnindent}
			\scnidtf{константная sc-пара}
			\scnsubset{sc-константа}
			\scniselement{sc-константа}
			\scniselement{sc-класс}
		\end{scnindent} 
		\scnitem{\textbf{переменная sc-пара принадлежности}}
		\begin{scnindent}
			\scnsubset{sc-переменная}
			\scniselement{sc-константа}
			\scniselement{sc-класс}
		\end{scnindent} 
	\end{scnsubdividing}

	\scnheader{обозначение sc-пары нечеткой принадлежности}
	\scniselement{sc-константа}
	\scniselement{sc-класс}
	\begin{scnsubdividing}
		\scnitem{\textbf{sc-пара нечеткой принадлежности}}
		\begin{scnindent}
			\scnidtf{константная sc-пара}
			\scnsubset{sc-константа}
			\scniselement{sc-константа}
			\scniselement{sc-класс}
		\end{scnindent} 
		\scnitem{\textbf{переменная sc-пара нечеткой принадлежности}}
		\begin{scnindent}
			\scnsubset{sc-переменная}
			\scniselement{sc-константа}
			\scniselement{sc-класс}
		\end{scnindent} 
	\end{scnsubdividing}

	\scnheader{обозначение sc-пары  позитивной принадлежности}
	\scniselement{sc-константа}
	\scniselement{sc-класс}
	\begin{scnsubdividing}
		\scnitem{\textbf{sc-пара позитивной принадлежности}}
		\begin{scnindent}
			\scnidtf{константная sc-пара}
			\scnsubset{sc-константа}
			\scniselement{sc-константа}
			\scniselement{sc-класс}
		\end{scnindent} 
		\scnitem{\textbf{переменная sc-пара позитивной принадлежности}}
		\begin{scnindent}
			\scnsubset{sc-переменная}
			\scniselement{sc-константа}
			\scniselement{sc-класс}
		\end{scnindent} 
	\end{scnsubdividing}

	\scnheader{sc-пара постоянной позитивной принадлежности}
	\scnidtf{константная позитивная постоянная sc-пара принадлежности}
	\scnidtf{sc-пара константной постоянной позитивной принадлежности}

	\scnheader{sc-пара временной позитивной принадлежности}
	\scnidtf{sc-пара константной временной позитивной принадлежности}

	\scnheader{обозначение sc-пары негативной принадлежности}
	\scniselement{sc-константа}
	\scniselement{sc-класс}
	\begin{scnsubdividing}
		\scnitem{\textbf{sc-пара негативной принадлежности}}
		\begin{scnindent}
			\scnidtf{константная sc-пара}
			\scnsubset{sc-константа}
			\scniselement{sc-константа}
			\scniselement{sc-класс}
		\end{scnindent} 
		\scnitem{\textbf{переменная sc-пара негативной принадлежности}}
		\begin{scnindent}
			\scnsubset{sc-переменная}
			\scniselement{sc-константа}
			\scniselement{sc-класс}
		\end{scnindent} 
	\end{scnsubdividing}

	\scnheader{обозначение sc-пары, не являющейся парой принадлежности}
	\scniselement{sc-константа}
	\scniselement{sc-класс}
	\begin{scnsubdividing}
		\scnitem{\textbf{sc-пара, не являющаяся парой принадлежности}}
		\begin{scnindent}
			\scnidtf{константная sc-пара}
			\scnsubset{sc-константа}
			\scniselement{sc-константа}
			\scniselement{sc-класс}
		\end{scnindent} 
		\scnitem{\textbf{переменная sc-пара, не являющаяся парой принадлежности}}
		\begin{scnindent}
			\scnsubset{sc-переменная}
			\scniselement{sc-константа}
			\scniselement{sc-класс}
		\end{scnindent} 
	\end{scnsubdividing}

	\scnheader{обозначение sc-связки, не являющейся ни синглетоном, ни парой}
	\scniselement{sc-константа}
	\scniselement{sc-класс}
	\begin{scnsubdividing}
		\scnitem{\textbf{sc-связка, не являющаяся ни синглетоном, ни парой}}
		\begin{scnindent}
			\scnidtf{константная sc-пара}
			\scnsubset{sc-константа}
			\scniselement{sc-константа}
			\scniselement{sc-класс}
		\end{scnindent} 
		\scnitem{\textbf{переменная sc-связка, не являющаяся ни синглетоном, ни парой}}
		\begin{scnindent}
			\scnsubset{sc-переменная}
			\scniselement{sc-константа}
			\scniselement{sc-класс}
		\end{scnindent} 
	\end{scnsubdividing}

	\scnheader{обозначение sc-класса}
	\begin{scnsubdividing}
		\scnitem{sc-класс}
		\scnitem{переменный sc-класс}
	\end{scnsubdividing}
	\begin{scnsubdividing}
		\scnitem{обозначение sc-класса обозначений sc-связок}
		\scnitem{обозначение sc-класса обозначений sc-классов}
		\scnitem{обозначение sc-класса обозначений sc-структор}
		\scnitem{обозначение sc-классов обозначений внешних сущностей}
	\end{scnsubdividing}

	\scnheader{sc-класс}
	\begin{scnsubdividing}
		\scnitem{sc-класс sc-связок}
		\begin{scnindent}
			\scnsuperset{sc-отношение}
			\begin{scnindent}
				\scnsuperset{бинарное sc-отношение}
				\begin{scnindent} 
					\begin{scnsubdividing}
						\scnitem{бинарное неориентированное sc-отношение}
						\scnitem{бинарное ориентированное sc-отношение}
						\begin{scnindent}
							\scnsuperset{ролевое sc-отношение}
						\end{scnindent} 
					\end{scnsubdividing}
				\end{scnindent}
			\end{scnindent}
		\end{scnindent}
		\scnitem{sc-класс sc-классов}
		\begin{scnindent}
			\scnsuperset{sc-параметр}
		\end{scnindent}
		\scnitem{sc-класс sc-структур}
		\scnitem{sc-класс внешних сущностей}
		\begin{scnindent}
			\scnsuperset{sc-класс файлов}
			\scnidtf{\textit{sc-класс} sc-элементов, являющихся знаками \textit{внешних сущностей}}
		\end{scnindent} 
		\scnitem{sc-класс sc-констант разного структурного типа}
		\begin{scnindent}
			\begin{scnhaselementrolelist}{пример}
				\scnitem{sc-константа}
				\scnitem{постоянная сущность}
			\end{scnhaselementrolelist}
		\end{scnindent} 
	\end{scnsubdividing}
	\scntext{пояснение}{Требованием, предъявляемым к каждому \textit{sc-классу} является наличие \textit{общего} свойства, присущего \textit{всем} элементам этого \textit{sc-класса}. Формулировку указанного общего свойства обычно называют \textit{определением} соответствующего \textit{sc-класса} (в частности, \textit{понятия}). Некоторые \textit{sc-классы} могут быть заданы с помощью \textit{отношений эквивалентности}, если эти классы являются \textit{классами эквивалентности} соответствующих \textit{отношений эквивалентности}, то есть являются элементами \textit{фактор-множеств}, соответствующих этим \textit{отношениям}.}

	\scnheader{следует отличать*}
	\begin{scnhaselementset}
		\scnitem{sc-связка}
		\scnitem{sc-класс}
	\end{scnhaselementset}
	\begin{scnindent}
		\scntext{сравнение}{В отличие от \textit{sc-связки} принципом формирования \textit{sc-класса} является наличие общего свойства, присущего \textit{всем} элементам этого \textit{sc-класса} \textit{и только им}, (или присущего всем сущностям, которые обозначаются указанными \textit{sc-элементами}). Таким общим свойством может быть \textit{определение \textit{sc-класса}} либо принадлежность одному из значений некоторого параметра, то есть одному из элементов \textit{фактор-множества}, соответствующего некоторому \textit{отношению эквивалентности} или толерантности.}
		\scntext{пояснение}{Примерами \textit{связок} являются:
		\begin{scnitemize}
			\item множество людей живущих сейчас (динамическое множество);
			\item множество сотрудников некоторой  конкретной организации (динамическое множество);
			\item конкретный отрезок, конкретный треугольник.
		\end{scnitemize}
		Здесь речь не идет об эквивалентности свойств самих людей и геометрических точек безотносительно к тому, в состав чего они входят. Поэтому это не является \textit{sc-классом}.}
	\end{scnindent}
	
	\begin{scnhaselementset}
		\scnitem{sc-класс эквивалентности}
		\begin{scnindent}
			\scnexplanation{В \textit{sc-класс эквивалентности} входит не просто некоторое количество попарно эквивалентных между собой сущностей, а абсолютно \textit{все} такие сущности.}
		\end{scnindent}
		\scnitem{sc-связка попарно эквивалентных сущностей}
	\end{scnhaselementset}

	\begin{scnhaselementset}
		\scnitem{множество \textit{всех} треугольников, подобных одному из них}
		\begin{scnindent}
			\scnsubset{sc-класс}
		\end{scnindent}
		\scnitem{конечное множество подобных треугольников}
		\begin{scnindent}
			\scnsubset{sc-связка попарно эквивалентных треугольников}
		\end{scnindent}
	\end{scnhaselementset}
	
	\begin{scnhaselementset}
		\scnitem{sc-параметр}
		\begin{scnindent}
			\scnidtftext{часто используемый sc-идентификатор}{параметр}
			\scnsubset{sc-класс sc-классов}
		\end{scnindent}
		\scnitem{признак различия}
		\begin{scnindent}
			\scnidtf{признак классификации}
		\end{scnindent}
	\end{scnhaselementset}
	\begin{scnindent}
		\scnrelfrom{пояснение}{\scnnonamednode}
		\begin{scnindent}
			\begin{scneqtoset}
				\scnitem{параметр}
				\begin{scnindent}
					\scnsubset{бесконечное множество}
				\end{scnindent}
				\scnitem{признак различия}
				\begin{scnindent}
					\scnsubset{конечное множество}
					\begin{scnhaselementrolelist}{пример}
						\scnitem{Признак конечности множеств}
						\begin{scnindent}
							\begin{scneqtoset}
								\scnitem{конечное множество}
								\scnitem{бесконечное множество}				
							\end{scneqtoset}
						\end{scnindent}
						\scnitem{Признак наличия кратных элементов}
						\begin{scnindent}
							\begin{scneqtoset}
								\scnitem{мультимножество}
								\scnitem{множество без кратных вхождений элементов}
							\end{scneqtoset}
						\end{scnindent}
					\end{scnhaselementrolelist}
				\end{scnindent}
			\end{scneqtoset}
		\end{scnindent}
	\end{scnindent}

	\scnheader{sc-класс}
	\scnrelfrom{правила построения внешних идентификаторов sc-элементов заданного класса}{Правила построения внешних идентификаторов sc-элементов, являющихся знаками sc-классов}
	\begin{scnindent}
		\begin{scneqtoset}
			\scnfileitem{Слово \scnqqi{обозначение} в начале идентификатора используется тогда, когда в идентифицируемый класс sc-элементов включаются знаки как константных, так и переменных сущностей соответствующего вида.}
			\scnfileitem{Слово \scnqqi{переменный} в начале идентификатора используется, когда элементами идентифицируемого sc-класса являются только sc-переменные.}
			\scnfileitem{Слово \scnqqi{константный} в начале идентификатора можно опустить, так как константность подразумевается по умолчанию.}
		\end{scneqtoset}
	\end{scnindent}

	\scnheader{обозначение sc-структуры}
	\scniselement{sc-константа}
	\scniselement{sc-класс}
	\begin{scnsubdividing}
		\scnitem{\textbf{sc-структура}}
		\begin{scnindent}
			\scnidtf{константная sc-пара}
			\scnsubset{sc-константа}
			\scniselement{sc-константа}
			\scniselement{sc-класс}
		\end{scnindent} 
		\scnitem{\textbf{переменная sc-структура}}
		\begin{scnindent}
			\scnsubset{sc-переменная}
			\scniselement{sc-константа}
			\scniselement{sc-класс}
		\end{scnindent} 
	\end{scnsubdividing}

	\scnheader{следует отличать*}
	\begin{scnhaselementset}
		\scnitem{sc-структура}
		\scnitem{sc-связка}
	\end{scnhaselementset}
	\begin{scnindent} 
		\begin{scnrelfromset}{сравнение}
			\scnfileitem{В отличие от \textit{sc-связок} в каждую \textit{sc-структуру} должна входить по крайней мере одна \textit{sc-связка} вместе с компонентами этой \textit{sc-связки}.}
		\end{scnrelfromset}
	\end{scnindent}

	\scnheader{обозначение внешней сущности}
	\scnidtf{обозначение сущности, не являющейся sc-множеством}

	\scnheader{внешняя сущность}
	\scnidtf{синглетон внешней сущности}
	\scnidtf{сущность, не являющаяся sc-множеством}
	\scnidtf{обозначение синглетона внешней сущности}
	\scnidtf{\textit{sc-элемент}, обозначающий синглетон, элементом которого является некоторая внешняя описываемая сущность}
	\scnidtf{множество, являющееся 1-мощным множеством, единственным элементом которого является сущность, внешняя по отношению к sc-памяти, то есть сущность, не являющаяся \textit{sc-элементом}}
	\begin{scnrelfromlist}{примечание}
		\scnfileitem{Обозначение внешней сущности, то есть \textit{sc-элемент}, обозначающий соответствующий синглетон, можно также трактовать как \textit{sc-элемент}, обозначающий соответствующую внешнюю описываемую сущность, которую, в свою очередь, можно считать денотатом указанного \textit{sc-элемента}.}
		\scnfileitem{Очевидно, что пара принадлежности, связывающая \textit{sc-элемент}, обозначающий синглетон внешней сущности, не может быть непосредственно представлена в виде соответствующей \textit{sc-пары принадлежности}, так как второй компонент этой \textit{sc-пары} не находится в \textit{sc-памяти}.}
	\end{scnrelfromlist}

	\scnheader{следует отличать*}
	\begin{scnhaselementset}
		\scnitem{внешня сущность}
		\scnitem{sc-синглетон}
		\begin{scnindent}
			\scnidtf{синглетон, единственным элементом которого является некоторый \textit{sc-элемент}}
			\scnsubset{sc-множество}
			\begin{scnindent}
				\scnidtf{\textit{sc-элемент}, обозначающий множество, элементами которого являются \textit{только} sc-элементы}
				\scnidtf{множество \textit{sc-элементов}}
			\end{scnindent} 
		\end{scnindent}
	\end{scnhaselementset}

	\scnheader{обозначение файла}
	\scniselement{sc-константа}
	\scniselement{sc-класс}
	\begin{scnsubdividing}
		\scnitem{\textbf{файл}}
		\begin{scnindent}
			\scnidtf{константная sc-пара}
			\scnsubset{sc-константа}
			\scniselement{sc-константа}
			\scniselement{sc-класс}
		\end{scnindent} 
		\scnitem{\textbf{переменный файл}}
		\begin{scnindent}
			\scnsubset{sc-переменная}
			\scniselement{sc-константа}
			\scniselement{sc-класс}
		\end{scnindent} 
	\end{scnsubdividing}

	\scnheader{файл}
	\scnidtf{внутренний образ (копия) информационной конструкции, хранимый в \textit{файловой памяти ostis-системы}}
	\scnidtf{файл \textit{ostis-системы}}
	\begin{scnrelfromlist}{примечание}
		\scnfileitem{\textit{файловая память ostis-системы}, хранящая различного рода \textit{информационные конструкции} (образы, модели), не являющиеся \textit{sc-конструкциями}, должна быть тесно связана с \textit{sc-памятью} этой же \textit{ostis-системы}. Как минимум, каждый \textit{файл ostis-системы} должен быть связан с тем \textit{sc-элементом}, которых является знаком этого \textit{файла} (точнее, знаком синглетона, элементом которого является указанный файл).}
	\end{scnrelfromlist}
\end{scnsubstruct}


\scnstructheader{Уточнение смысла выделенных классов sc-элементов в Логической классификации sc-элементов}
\begin{scnsubstruct}
	\scnheader{sc-константа}
	\scnidtf{sc-элемент, обозначающий константную сущность}
	\begin{scnindent}
		\scntext{сокращение}{обозначение константной сущности}
	\end{scnindent}
	\scnidtf{обозначение константной сущности}
	\scnidtf{знак константной сущности}
	\begin{scnindent}
		\scntext{сокращение}{константная сущность}
		\begin{scnindent}
			\scntext{сокращение}{сущность}
		\end{scnindent} 
	\end{scnindent}
	\scnidtf{константная сущность}
	\scnidtf{конкретная сущность}
	\scnidtf{сущность}
	\scnidtf{константный sc-элемент}
	\scnidtf{sc-элемент, имеющий одно логико-семантическое значение, каковым является он сам}
	\scnidtf{sc-элемент, являющийся знаком константной (конкретной, фиксированной) сущности}
	\begin{scnindent}
		\scntext{сокращение}{знак константной (конкретной, фиксированной) сущности}
			\begin{scnindent} 
				\scntext{сокращение}{константная (конкретная, фиксированная) сущность}
				\begin{scnindent} 
					\scntext{сокращение}{константная сущность}
				\end{scnindent}
			\end{scnindent}
	\end{scnindent}

	\scnheader{sc-переменная}
	\scnidtf{переменный sc-элемент}
	\scnidtf{sc-элемент, являющийся обозначением некоторой произвольной (нефиксируемой, переменной) сущности}
	\begin{scnindent}
		\scntext{сокращение}{обозначение произвольной (переменной) сущности}
		\begin{scnindent}
			\scntext{сокращение}{переменная сущность}
		\end{scnindent}
	\end{scnindent}
	\scniselement{sc-константа}
	\scniselement{sc-класс}
	\scnnote{Сам \textit{sc-элемент}, имеющий внешний идентификатор \scnqqi{\textit{sc-переменная}} является \textit{sc-константой} (константным sc-элементом), которая является знаком соответствующего класса sc-элементов.}

	\scnheader{sc-элемент}
	\scnidtf{обозначение константной или переменной сущности}
	\scnidtf{константная или переменная сущность}
	\scnidtf{sc-константа или sc-переменная}
	\scnidtf{обозначение описываемой сущности, которая может быть как константной, так и переменной сущностью, как внутренней, так и внешней sc-конструкцией для заданной ostis-системы, как информационной конструкцией, так и сущностью которая информационной конструкцией не является, как временной сущностью, так и постоянной, как динамической, так и статической сущностью}

	\scnheader{обозначение sc-множества}
	\begin{scnsubdividing}
		\scnitem{sc-множество}
		\begin{scnindent}
			\scnidtftext{часто используемый sc-идентификатор}{множество sc-элементов}
			\scnidtf{константное (конкретное) sc-множество}
			\scnidtf{обозначение (знак) конкретного множества}
			\scnsubset{sc-константа}
			\scniselement{sc-константа}
			\begin{scnsubdividing}
				\scnitem{sc-множество sc-констант}
				\begin{scnindent}
					\scnidtf{sc-множество, элементами которого являются только sc-константы}
					\scnidtf{множество, являющееся подмножеством Множества всевозможных констант}
				\end{scnindent} 
				\scnitem{sc-множество sc-переменных}
				\begin{scnindent}
					\scnidtf{sc-множество, элементами которого являются только sc-переменные}
				\end{scnindent} 
				\scnitem{sc-множество sc-констант и sc-переменных}
				\begin{scnindent}
					\scnidtf{множество, элементами которого являются как константы, так и переменные}
					\scniselement{sc-константа}
					\scnrelboth{следует отличать}{sc-множество sc-переменных}
				\end{scnindent} 
			\end{scnsubdividing}
		\end{scnindent}
		\scnitem{переменное sc-множество}
		\begin{scnindent}
			\scnidtf{обозначение переменного (произвольного) sc-множества}
		\end{scnindent}
	\end{scnsubdividing}
\end{scnsubstruct}


\scnstructheader{Уточнение смысла выделенных классов sc-элементов в Классификации sc-элементов по темпоральным характеристикам обозначаемых ими сущностей}
\begin{scnsubstruct}
	\begin{scnrelfromlist}{ключевой знак}
		\scnitem{формируемое sc-множество}
		\scnitem{sc-множество, элементы которого не известны}
		\scnitem{сформированный файл}
		\scnitem{формируемый файл}
		\scnitem{файл, структура которого не известна}
	\end{scnrelfromlist}
	
	\scnheader{обозначение временной сущности}
	\begin{scnsubdividing}
		\scnitem{обозначение временной сущности существующей сейчас}
		\begin{scnindent}
			\scnidtf{обозначение временной сущности, существующей в текущий (настоящий) момент}
		\end{scnindent} 
		\scnitem{обозначение прошлой временной сущности}
		\begin{scnindent}
			\scnidtf{обозначение бывшей временной сущности}
			\scnidtf{обозначение временной сущности, которая уже перестала существовать, прекратила свое существование}
		\end{scnindent} 
		\scnitem{обозначение будущей временной сущности}
		\begin{scnindent}
			\scnidtf{обозначение временной сущности, появление которой прогнозируется (планируется, обеспечивается)}
			\scnnote{проектирование и производство новых, ранее не существующих полезных сущностей --- это основное направление человеческой деятельности}
		\end{scnindent} 
	\end{scnsubdividing}
	\begin{scnindent}
		\scnnote{ostis-системы должны постоянно мониторить состояние временных сущностей, так как в процессе их функционирования будущие сущности становятся настоящими, а настоящие --- прошлыми.}
	\end{scnindent} 

	\scnheader{динамическое sc-множество}
	\scnidtf{sc-процесс}
	\scnidtf{процесс}
	\scntext{определение}{\textit{sc-множество}, у которого некоторые позитивные пары принадлежности, связывающие знак этого множества с его элементами, носят временный характер}
	\scnnote{Сами элементы \textit{динамического sc-множества}, связанные с ним временными позитивными парами принадлежности, могут обозначать как временные, так и постоянные сущности. Но чаще всего такими временными элементами динамического sc-множества являются знаки временных связок.}
	\begin{scnsubdividing}
		\scnitem{внешний процесс}
		\scnitem{процесс в sc-памяти}
	\end{scnsubdividing}

	\scnheader{темпоральная декомпозиция динамического sc-множества}
	\scnidtf{покадровая развертка динамического sc-множества}
	\scnidtf{представление sc-множества в виде кортежа (последовательности) ситуаций}

	\scnheader{следует отличать*}
	\begin{scnhaselementset}
		\scnitem{временная сущность}
		\scnitem{обозначение временной сущности}
		\scnitem{переменная временная сущность}
	\end{scnhaselementset}

	\scnheader{обозначение временной сущности}
	\begin{scnsubdividing}
		\scnitem{временная сущность}
		\begin{scnindent}
			\scnidtf{знак конкретной (константной) временной сущности}
		\end{scnindent} 
		\scnitem{переменная временная сущность}
		\begin{scnindent}
			\scnidtf{обозначение произвольной временной сущности}
		\end{scnindent} 
	\end{scnsubdividing}

	\scnheader{сформированное sc-множество}
	\scnidtf{sc-множество, у которого в текущем состоянии sc-памяти перечислены все его элементы}
	\scniselement{динамическое sc-множество}
	\scntext{пояснение}{Очевидно, что сформированным sc-множеством может стать только конечное sc-множество.}

\end{scnsubstruct}

\scnstructheader{Уточнение смысла семантически выделяемых классов \textit{sc-элементов}, которые необходимо ввести дополнительно к выше рассмотренным классам \textit{sc-элементов}}
\begin{scnsubstruct}

	\begin{scnrelfromlist}{ключевой знак}
		\scnitem{sc-элемент, не являющийся ни sc-синглетоном, ни sc-парой}
	\end{scnrelfromlist}
	
	\scnheader{sc-элемент, копируемый в других компьютерных системах}
	\scnidtf{\textit{sc-элемент}, имеющий в других компьютерных системах свои копии и/или копии обозначаемой им информационной конструкции}

	\scnheader{отношение, заданное на множестве sc-элементов, копируемых в других компьютерных системах}
	\scnhaselement{ostis-система, в sc-памяти которой хранится копия заданного sc-элемента*}
	\scnhaselement{компьютерная система, в файловой памяти которой хранится заданный файл*}
	\begin{scnindent}
		\scnnote{Указанная компьютерная система назначается хранителем файла.}
	\end{scnindent}
	\scnhaselement{ostis-система, в sc-памяти которой хранится копия знака заданного sc-множества и все известные в текущий момент его элементы*}
	\begin{scnindent}
		\scnnote{Указанная ostis-система назначается основным хранителем указанного sc-множества.}
	\end{scnindent}

	\scnheader{информационная конструкция}
	\begin{scnsubdividing}
		\scnitem{sc-множество}
		\begin{scnindent}
			\scnidtf{sc-конструкция}
			\scnidtf{информационная конструкция \textit{SC-кода}}
			\scnidtf{внутренняя информационная конструкция \textit{ostis-системы}, хранимая в ее \textit{sc-памяти}}
		\end{scnindent}
		\scnitem{файл}
		\begin{scnindent}
			\scnidtf{файл ostis-системы}
			\scnidtf{информационная конструкция \textit{ostis-системы}, хранимая в ее файловой памяти}
			\scnnote{файл, может храниться в памяти другой компьютерной системы и, в частности, в файловой памяти другой \textit{ostis-системы}}
		\end{scnindent}
		\scnitem{внешняя информационная конструкция, не являющаяся ни файлом, ни sc-конструкцией}
	\end{scnsubdividing}

	\scnheader{sc-идентификатор}
	\scnidtf{внешний идентификатор sc-элемента}
	\scnsuperset{файл}
	\begin{scnsubdividing}
		\scnitem{основной идентификатор}
		\scnitem{часто используемый sc-идентификатор}
		\scnitem{дополнительный sc-идентификатор}
	\end{scnsubdividing}

	\scnheader{sc-идентификатор*}
	\scnidtf{бинарное ориентированное отношение, связывающее \textit{sc-элементы} с их внешними идентификаторами}
\end{scnsubstruct}  

\end{scnsubstruct}

        
\scnsegmentheader{Структура базовой семантической спецификации sc-элемента}
\begin{scnsubstruct}
	\scnheader{базовая семантическая спецификация sc-элемента}
	\scnidtfexp{Класс \textit{sc-структур}, каждая из которых описывает базовые семантические свойства (характеристики) соответствующего (описываемого, специфицируемого) \textit{sc-элемента}}
	\scnsubset{sc-структура}
	\begin{scnindent}
		\scnsubset{sc-спецификация}
		\begin{scnindent}
			\scnidtf{представленная в \textit{SC-коде} семантическая окрестность (спецификация) некоторого (специфицируемого) \textit{sc-элемента}}
		\end{scnindent}
	\end{scnindent}
	\scnsubset{sc-спецификация}
	\scnrelto{второй домен}{базовая семантическая спецификация sc-элемента*}
	\begin{scnindent}
		\scnidtfexp{бинарное ориентированное отношение, каждая пара которого связывает \textit{sc-элемент} с его базовой семантической спецификацией*}
	\end{scnindent}
	\scnidtfexp{хранимая в \textit{sc-памяти} ostis-системы спецификация каждого \textit{sc-элемента}, необходимая для эффективной обработки этого \textit{sc-элемента}}
	\scnnote{базовая спецификация \textit{sc-элементов} осуществляется как явно с помощью соответствующих sc-конструкций, так и неявно с помощью соответствующих семантических меток, приписываемых sc-элементам}
	\scntext{пояснение}{Базовая семантическая спецификация каждого \textit{sc-элемента} включает в себя:
		\begin{scnitemize}
			\item перечисление всех тех \textit{базовых классов sc-элементов}, которым принадлежит специфицируемый \textit{sc-элемент};
			\item уточнение \scnqq{привязки} временной сущности, обозначаемой специфицируемым sc-элементом к текущему моменту и другим моментам времени;
			\item уточнение того, какие важные характеристики специфицируемого \textit{sc-элемента} в текущем состоянии \textit{sc-памяти} и файловой памяти \textit{ostis-системы} не известны.
		\end{scnitemize}
	}
    \scntext{примечание}{\textbf{\textit{базовая семантическая спецификация sc-элемента, обозначающего временную сущность}} включает в себя указание дополнительных темпоральных характеристик, позволяющих уточнить темпоральные \scnqq{координаты} этих временных сущностей (то есть их \scnqq{координаты} во времени), а также их основные темпоральные связи с другими временными сущностями.}
    
	\scnheader{базовая семантическая спецификация sc-элемента, обозначающего временную сущность}
	\begin{scnrelfromset}{включение}
		\scnitem{момент времени\scnsupergroupsign}
		\scnitem{Текущий момент времени}
		\scnitem{прошлая сущность}
		\scnitem{будущая сущность} 
		\scnitem{момент начала*}
		\scnitem{момент завершения*}
		\scnitem{внешняя ситуация}
		\scnitem{ситуация в sc-памяти}
		\scnitem{внешнее событие}
		\scnitem{событие в sc-памяти}
		\scnitem{внешний процесс}
		\scnitem{процесс в sc-памяти}
    \end{scnrelfromset}

	\scnheader{момент времени\scnsupergroupsign}
	\scniselement{параметр}
	\scniselement{параметр, заданный на множестве временных сущностей}
	\scnidtf{глобальная приблизительно точная ситуация\scnsupergroupsign}
	\scnidtf{глобальная ситуация пренебрежительно малого отрезка времени\scnsupergroupsign}
	\scnidtf{множество (класс) \textit{всех} временных сущностей, существующих одновременно в соответствующий момент времени\scnsupergroupsign}
	\scnnote{момент времени, соответствующий глобальной точечной ситуации может быть задан с различной и \textit{дополни-} \textit{тельно указываемой} степенью точности --- с точностью до секунды, до минуты, до часа, до даты, до календарного месяца, до календарного года и так далее. В том смысле корректнее говорить не о моменте времени, а об интервале времени, длительность которого считается пренебрежимо малой для рассмотрения описываемых процессов}

	\scnheader{Текущий момент времени}
	\scnidtf{Глобальная ситуация текущего (настоящего) момента времени}
	\scnidtf{Глобальная ситуация, имеющая место сейчас}
	\scnidtf{Класс всех сущностей, существующих в настоящий момент времени}
	\scniselement{sc-синглетон}
	\scniselement{динамическое sc-множество}
	\scnrelto{включение множества}{момент времени}
	\scnexplanation{Из знака \textit{Текущего момента времени} (который является также знаком \textit{sc-синглетона}) \scnqq{выходит} sc-пара \textit{временной} принадлежности, представляющая собой, образно говоря, \scnqq{стрелку} внутренних часов \textit{ostis-системы}, которая всегда указывает только на один элемент множества моментов времени, но в разные моменты времени указывает на разные элементы этого множества}

	\scnheader{прошлая сущность}
	\scnidtf{временная сущность, уже завершившая свое существование}

	\scnheader{будущая сущность}
	\scnidtf{прогнозируемая, планируемая или создаваемая временная сущность}

	\scnheader{момент начала*}
	\scnidtf{момент времени, соответствующий началу существования заданной временной сущности}
	\scnidtf{бинарное ориентированное отношение, каждая пара которого, связывает (1) знак некоторой временной сущности и (2) глобальную точечную ситуацию (значение параметра \scnqq{\textit{момент времени}\scnsupergroupsign}), элементом которой является условно точечная временная сущность, представляющая собой начальный этап существования временной сущности, указанной в первом компоненте рассматриваемой ориентированной пары}
	\scnnote{Начальный этап существования временной сущности (переходный процесс от небытия к реальному существованию) может рассматриваться с любой степенью детализации}
	\scnrelfrom{первый домен}{временная сущность}
	\scnrelfrom{второй домен}{момент времени\scnsupergroupsign}

	\scnheader{момент завершения*}
	\scnidtf{момент времени, соответствующий завершению существования заданной временной сущности}

	\scnheader{ситуация}
	\begin{scnsubdividing}
		\scnitem{внешняя ситуация}
		\scnitem{ситуация в sc-памяти}
	\end{scnsubdividing}

	\scnheader{событие}
	\begin{scnsubdividing}
		\scnitem{внешнее событие}
		\scnitem{событие в sc-памяти}
	\end{scnsubdividing}

	\scnheader{динамическое sc-множество}
	\begin{scnsubdividing}
		\scnitem{внешний процесс}
		\begin{scnindent}
			\scnidtf{процесс, происходящий в окружающей среде ostis-системы}
		\end{scnindent}
		\scnitem{процесс в sc-памяти}
	\end{scnsubdividing}

	\scnheader{внешняя ситуация}
	\scnidtf{ситуация во внешней среде}
	\scnidtf{ситуация \textit{одновременного} существования (в соответствующий период времени) указанных временных внешних сущностей}
	\scnsubset{временная сущность}
	\scnsubset{sc-структура}
	\scnsubset{sc-константа}
	\scnsubset{обозначение внешней ситуации}
	\scniselement{sc-класс}

	\scnheader{класс внешних ситуаций}
	\scnnote{В простейшем случае внешние ситуации, входящие в класс внешних ситуаций являются изоморфными}

	\scnheader{внешний процесс}
	\scnidtf{темпоральная детализация внешней динамической сущности}

	\scnheader{внешнее событие}
	\scnidtf{факт появления (возникновения) некоторой внешней сущности (в том числе некоторой внешней ситуации) или факт завершения существования некоторой внешней сущности (в том числе некоторой внешней ситуации)}

	\scnheader{ситуация в sc-памяти}
	\scnidtf{внутренняя ситуация}
	\begin{scnindent}
		\scnidtf{sc-ситуация}
		\scnidtf{хранимый в sc-памяти фрагмент базы знаний, рассматриваемый в контексте его появления в sc-памяти или его исчезновения (из-за удаления некоторые sc-элементов)}
	\end{scnindent}

	\scnheader{класс ситуаций в sc-памяти}
	\scnidtf{класс внутренних ситуаций}

	\scnheader{обобщенное описание класса ситуаций в sc-памяти}

	\scnheader{процесс в sc-памяти}
	\scnidtf{внутренний процесс}
	\scnidtf{информационный процесс, происходящий в sc-памяти}
	\scnidtf{sc-процесс}	

	\scnheader{событие в sc-памяти}

    \scnheader{базовая семантическая спецификация sc-элемента}
    \scntext{примечание}{Важной частью \textbf{\textit{базовой семантической спецификации sc-элемента}} является фиксация того, что \textit{ostis-система} \textit{знает и чего она не знает} о специфицируемом \textit{sc-элементе} или об обозначенной им сущности.}
        \begin{scnindent}
            \begin{scnrelfromset}{пояснение}
                \scnfileitem{Если в спецификации \textit{sc-элемента} указывается его принадлежность к некоторому классу \textit{sc-элементов}, но не указывается его принадлежность \textit{одному} из подклассов, на которые \textit{разбивается} указанный выше класс, то это означает, что в текущий момент времени \textit{ostis-система} этого \textit{не знает}.}
                \scnfileitem{Если специфицируемый \textit{sc-элемент} является обозначением \textit{конечного} множества sc-элементов (в частности, пары \textit{sc-элементов}), и если в текущий момент времени \textit{ostis-системе} не известны \textit{все} этого множества (то есть специфицируемый \textit{sc-элемент} не соединен соответствующими парами принадлежности со \textit{всеми} элементами обозначаемого им множества \textit{sc-элементов}), то этот специфицируемый \textit{sc-элемент} следует отнести к \textit{sc-классу} \scnqqi{\textbf{\textit{обозначение несформированного sc-множества}}}.}
                \scnfileitem{Если специфицируемый \textit{sc-элемент} является обозначением ориентированной \textit{sc-пары} и если в текущий момент времени \textit{ostis-системе} не известна \textit{направленность} этой ориентированной пары \textit{sc-элементов} (то есть не известно, какой элемент этой пары является первым ее компонентом, а какой ее элемент является ее вторым компонентом), то этот специфицируемый \textit{sc-элемент} следует отнести к \textit{sc-классу} \scnqqi{\textbf{\textit{обозначение ориентированной sc-пары неизвестной направленности}}}.}
            \end{scnrelfromset}
        \end{scnindent}
    \begin{scnrelfromset}{понятия, используемые для описания полноты}
		\scnitem{\textit{обозначение бесконечного sc-множества}}
		\scnitem{\textit{обозначение конечного sc-множества}}
		\scnitem{\textit{мощность обозначаемого sc-множества*}}
		\scnitem{\textit{обозначение sc-множества неизвестной мощности}}
		\scnitem{\textit{обозначение sc-множества, о котором не известно, является ли оно sc-парой}}
		\scnitem{\textit{обозначение sc-пары, о которой не известно, является ли она ориентированной или нет}}
		\scnitem{\textit{обозначение ориентированной sc-пары неизвестной направленности}}
		\scnitem{\textit{обозначение сформированного sc-множества}}
		\scnitem{\textit{обозначение несформированного sc-множества}}
		\scnitem{\textit{обозначение частично сформированного sc-множества}}
		\scnitem{\textit{обозначение полностью несформированного sc-множества}}
		\scnitem{\textit{обозначение сформированного файла}}
		\scnitem{\textit{обозначение несформированного файла}}
		\scnitem{\textit{обозначение частично сформированного файла}}
		\scnitem{\textit{обозначение полностью несформированного файла}}
    \end{scnrelfromset}

    \scntext{примечание}{Подчеркнем то, что базовую семантическую спецификацию должны иметь абсолютно все \textit{sc-элементы}, хранимые в \textit{sc-памяти} в текущий момент времени, в том числе и все \textit{sc-элементы}, являющиеся ключевыми знаками в рамках \textbf{\textit{Предметной области Базовой денотационной семантики SC-кода}}.}
    \begin{scnindent}
        \scnrelfrom{пример}{обозначение sc-множества}
    \end{scnindent}
    
	\scnheader{обозначение sc-множества}
	\scnidtf{Множество всевозможных sc-элементов, обозначающих sc-множества}
	\begin{scnindent}
		\scniselement{имя собственное}
	\end{scnindent}
	\scniselement{обозначение sc-множества}
	\scnnote{Одним из элементов данного множества является знак, обозначающий это множество. Это означает, это данное множество является \textit{рефлексивным множеством}}
	\scniselement{обозначение множества sc-элементов разного структурного типа}
	\scntext{примечание}{Элементами данного множества являются обозначения различных:
		\begin{scnitemize}
			\item sc-синглетонов;
			\item sc-пар;
			\item sc-связок, не являющихся ни sc-синглетонами, ни sc-парами;
			\item sc-классов;
			\item sc-структур.
		\end{scnitemize}
	}
	\scniselement{обозначение множества sc-элементов, содержащего как константные, так и переменные sc-элементы}
	\scniselement{sc-константа}
	\scnnote{Само данное множество является константным, несмотря на то, что его элементами являются как sc-константы, так и sc-переменные}
	\scniselement{обозначение множества sc-элементов, содержащего sc-элементы, обозначающие как постоянные, так и временные сущности.}
	\scniselement{постоянная сущность}
	\scnnote{Следует отличать постоянство~/~временность сущности, обозначаемой sc-элементом и постоянство~/~временность sc-множества, одним из элементов которого указанный sc-элемент является.}
	\scniselement{обозначение множества sc-элементов, содержащего sc-элементы, обозначающие как статические, так и динамические sc-множества}
	\scniselement{статическое sc-множество}
	\begin{scnrelfromlist}{примечание}
\scnfileitem{Следует отличать статичность~/динамичность sc-множества, обозначаемого соответствующим sc-элементом и статичность~/динамичность sc-множества, одним из элементов которого указанный выше sc-элемент является.}
		\scnfileitem{Напомним, что статический характер sc-множества означает отсутствие временных sc-пар принадлежности (временных sc-дуг принадлежности), выходящих из знака этого sc-множества.}
	\end{scnrelfromlist}
	\scniselement{sc-класс}
	\scnidtf{Класс всевозможных sc-элементов, обозначающих sc-множества}
	\scnidtf{Класс обозначающий sc-множеств}
	\scnnote{Следует отличать разные sc-элементы, являющиеся обозначениями соответствующих sc-множеств, и класс, элементами которого являются \textit{всевозможные} такие sc-элементы.}
\end{scnsubstruct}

        \scnsegmentheader{Онтологическая формализация Базовой денотационной семантики SC-кода}
\begin{scnsubstruct}
	\begin{scnrelfromlist}{ключевой знак}
		\scnitem{sc-память}
		\scnitem{база знаний ostis-системы}
	\end{scnrelfromlist}
	\scntext{пояснение}{Суть онтологической формализации различных областей знаний, различных фрагментов \textit{баз знаний} интеллектуальных компьютерных систем заключается в следующем.}
	\begin{scnhaselementset}
		\scnfileitem{Достаточно большой \textit{семантически целостный} фрагмент \textit{баз знаний}.}
		\begin{scnindent}
			\begin{scnrelfromlist}{включение}
				\scnfileitem{Все элементы некоторого одного ключевого класса рассматриваемых объектов (объектов исследования) или \textit{конечного} числа таких ключевых классов объектов исследования.}
				\scnfileitem{\textit{Все связи} между выделенными объектами исследования, соответствующие заданному \textit{семейству} отношений, параметров и классов структур, которое условно будем называть предметом исследования.}
			\end{scnrelfromlist}
		\end{scnindent}
		\scnfileitem{Указанный семантически целостный фрагмент 	\textit{базы знаний}, являющийся чаще всего \textit{бесконечной} структурой, будем называть \textbf{\textit{предметной областью}}.}
		\scnfileitem{Сама формальная \textbf{\textit{онтология}} представляет собой формальную спецификацию выделенной \textit{предметной области} и включает в себя следующие \textbf{\textit{частные онтологии}}.}
		\begin{scnindent}
			\begin{scnsubdividing}
				\scnitem{структурная спецификация предметной области}
				\begin{scnindent}
					\scntext{примечание}{спецификация предметной области, в которой указываются роли всех ключевых элементов (ключевых знаков), входящих в состав \textit{предметной области}}
					\begin{scnrelfromlist}{включение}
						\scnitem{максимальный класс объектов исследования\scnrolesign}
						\scnitem{немаксимальный класс объектов исследования\scnrolesign}
						\scnitem{ключевой объект исследования\scnrolesign}
						\scnitem{исследуемый класс связок\scnrolesign}
						\scnitem{исследуемый класс классов\scnrolesign}
						\scnitem{исследуемый класс структур\scnrolesign}
						\scnitem{неисследуемый класс\scnrolesign}
						\begin{scnindent}
							\scnidtf{\textit{sc-класс}, исследуемый в другой (смежной) \textit{предметной области}}
						\end{scnindent}
					\end{scnrelfromlist}
				\end{scnindent}
				\scnitem{теоретико-множественная онтология}
				\begin{scnindent}
					\scntext{примечание}{спецификация предметной области, в которой описываются теоретико-множественные связи между всеми классами (\textit{sc-классами}), исследуемыми в рамках заданной (специфицируемой) \textit{предметной области}}
				\end{scnindent}
				\scnitem{логическая онтология}
				\begin{scnindent}
					\begin{scnrelfromlist}{включение}
						\scnfileitem{определения исследуемых классов (исследуемых понятий)}
						\scnfileitem{логическая иерархию исследуемых понятий, которая связывает каждое понятие со множеством тех понятий, которые явно используются в определении этого понятия}
						\scnfileitem{аксиомы и теоремы, описывающие свойства специфицируемой предметной области}
						\scnfileitem{тексты доказательств теорем}
						\scnfileitem{логическая иерархия теорем, которая связывает каждую теорему со множеством теорем, на основе которых она доказывается}
					\end{scnrelfromlist}	
				\end{scnindent}	
				\scnitem{терминологическая спецификация предметной области}
				\begin{scnindent}
					\scntext{примечание}{спецификация предметной области, в которой указывается \textit{sc-идентификаторы} всех ключевых \textit{sc-элементов} специфицируемой \textit{предметной области}, а также приводятся правила построения \textit{основных sc-идентификаторов} для элементов всех \textit{sc-классов} (понятий), исследуемых в рамках специфицируемой \textit{предметной области}}
				\end{scnindent}
				\scnitem{дидактическая спецификация предметной области}
				\begin{scnindent}
					\scntext{примечание}{спецификация предметной области, в которой приводится дополнительная информация, предназначенная для того, чтобы пользователи и разработчики (инженеры знаний), которые используют или совершенствуют специфицируемую \textit{предметную область} и ее \textit{онтологию}, могли быстрее усвоить их особенности.}
					\scnrelfrom{смотрите}{Предметная область и онтология предметных областей}
				\end{scnindent}
				\scnfileitem{проектная спецификация предметной области и соответствующей ей онтологии}
				\begin{scnindent}
					\scntext{примечание}{спецификация предметной области, в которой приводится информация об истории эволюции этой \textit{предметной области и онтологии}, а также о направлениях и планах организации дальнейшего их развития.}
				\end{scnindent}
			\end{scnsubdividing}
		\end{scnindent}	
	\end{scnhaselementset}
	\begin{scnrelfromset}{смотрите}
		\scnitem{Предметная область и онтология онтологий}
		\scnitem{Предметная область и онтология предметных областей}
	\end{scnrelfromset}
	\scntext{примечание}{Онтологическая формализация \textit{базовой денотационной семантики SC-кода} трактуется нами как \textit{формальная онтология}, представленная в \textit{SC-коде} и описывающая детонационную семантику \textit{семантически корректных sc-конструкций}. Указанную \textit{формальную онтологию} будем называть \textbf{\textit{Базовой денотационной семантикой SC-кода}}. Для того, чтобы уточнить \textit{предметную область}, специфицируемую этой \textit{онтологией}, введем следующие понятия:
		\begin{itemize}
			\item синонимия sc-элементов,
			\item отношение эквивалентности,
			\item sc-память,
			\item база знаний ostis-системы,
			\item ostis-система,
			\item $\scnleftsquarebrace$ sc-конструкция $\scnrightsquarebrace$,
			\item sc-знание,
			\item интеграция sc-конструкций*,
			\item sc-пространство.
		\end{itemize}}

	\scnheader{синонимия sc-элементов}
	\scnidtf{бинарное ориентированное \textit{отношение эквивалентности}, каждая пара которого связывает два \textit{sc-элемента}, обозначающие одну и ту же сущность*}
	\scnnote{Синонимия двух \textit{sc-элементов} возможна только в том случае, если эти \textit{sc-элементы} хранятся в \textit{sc-памяти} (входят в состав \textit{баз знаний}) \textit{разных} \textit{ostis-систем}. В рамках каждой \textit{ostis-системы} синонимичные \textit{sc-элементы} совпадают (отождействляются, склеиваются, считаются одним и тем же \textit{sc-элементом}).}

	\scnheader{отношение эквивалентности}
	\scnrelto{ключевое понятие}{\textsection 2.4.2. Формальная онтология связок и отношений}

	\scnheader{ostis-система}
	\begin{scnsubdividing}
		\scnitem{индивидуальная ostis-система}
		\scnitem{коллективная ostis-система}
	\end{scnsubdividing}

	\scnheader{$\scnleftsquarebrace$ sc-конструкция $\scnrightsquarebrace$}
	\scnrelto{часто используемый sc-идентификатор}{\textbf{sc-множество}}
	\begin{scnindent}
	\scnidtf{информационная конструкция, представляющая собой множество \textit{sc-элементов}}
	\scnsuperset{sc-текст}
	\begin{scnindent}
		\scnidtf{текст SC-кода}
		\scnidtf{\textit{sc-конструкция}, являющаяся семантически корректной по отношению к \textit{Базовой денотационной семантике SC-кода}}
		\scnidtf{\textit{sc-конструкция}, удовлетворяющая (соответствующая) правилам \textit{Базовой денотационной семантики SC-кода}}
		\scnidtftext{часто используемый sc-идентификатор}{\textit{SC-код}}
		\begin{scnindent}
			\scniselement{имя собственное}
			\scnidtf{Класс (Множество всевозможных) sc-текстов}
		\end{scnindent}
		\scnsuperset{sc-знание}
	\end{scnindent} 
	\end{scnindent}

	\scnheader{sc-знание}
	\scnidtf{\textit{sc-текст}, являющийся либо фрагментом (подструктурой) соответствующей \textit{предметной области}, либо \textit{высказыванием}, описывающим некоторое свойство (в частности, некоторую закономерность) этой \textit{предметной области}}
	\scnidtf{знание, представленное в \textit{SC-коде}}
	\scnidtf{\textit{sc-текст}, обладающий истинным значением по отношению к соответствующей \textit{предметной области}}
	\scnsubset{связная sc-конструкция}
	\scnnote{Разные \textit{sc-знания} могут противоречить друг другу, то есть отражать разные точки зрения на некоторую \textit{предметную область}, но любое \textit{sc-знание} должно быть \textit{sc-текстом}, то есть не должно противоречить правилам \textit{Базовой денотационной семантики SC-кода}.}

	\scnheader{интеграция sc-конструкций*}
	\scnidtf{объединение sc-конструкций*}
	\scnidtf{объединение sc-множеств*}
	\scnnote{При интеграции sc-конструкций sc-элементы, обозначающие одну и ту же сущность, то есть синонимичные sc-элементы, считаются одинаковыми (совпадающими, тождественными) и, следовательно, должны склеиваться (отождествляться).}

	\scnheader{SC-пространство}
	\scnidtf{Результат интеграции \textit{всевозможных} sc-конструкций, \textit{семантически корректных} по отношению к \textit{Базовой денотационной семантики SC-кода}}
	\scnidtf{Предметная область, специфицируемая (описываемая) \textit{Базовой денотационной семантикой SC-кода}, которая является формальной онтологией, представленной средствами SC-кода}
	\scnidtf{Результат интеграции всевозможных sc-текстов (текстов SC-кода)}
	\scnidtf{Максимальный sc-текст}
	\scnidtf{Текст SC-кода, включающий в себя всевозможные sc-тексты}
	\scnidtf{Пространство sc-конструкций, семантически корректных по отношению к \textit{Базовой денотационной семантике SC-кода}}
	\begin{scnrelfromlist}{примечание}
		\scnfileitem{Особенностью \textit{SC-пространство} является то, что оно включает в себя и формальную онтологию, описывающую его свойства.}
		\scnfileitem{очевидно, что \textit{SC-пространство} является \textit{бесконечным} \textit{sc-текстом}, то есть текстом, содержащим бесконечное количество \textit{sc-элементов}. В частности, в состав \textit{SC-пространства} входят \textit{все} \textit{sc-элементы}, являющиеся элементами \textit{всех} \textit{sc-множеств}, знаки которых входят в состав \textit{SC-пространства}.}
		\scnfileitem{\textit{SC-пространство} является \scnqq{вместилищем} семантически корректных (по отношению к \textit{Базовой денотационной семантике SC-кода}) частей баз знаний всевозможных ostis-систем и, в том числе, глобальной (объединенной) \textit{Базы знаний Экосистемы OSTIS}. Подчеркнем при этом, что \textit{Экосистема OSTIS} является примером распределенных иерархических \textit{ostis-систем}.}
		\scnfileitem{Тот факт, что корректная (с точки зрения \textit{Базовой денотационной семантики SC-кода}) часть базы знаний \textit{каждой} \textit{ostis-системы} входит в состав \textit{SC-пространства}, позволяет трактовать описание соотношения между текущим состоянием \textit{базы знаний ostis-системы} и \textit{Sc-пространством} как описание того, что указанная \textit{ostis-система} в текущий момент времени не знает. Например, \textit{ostis-система} в некоторый момент времени может не знать (1) всех элементов некоторого конкретного \textit{конечного} \textit{sc-множества} (конечно sc-конструкции), (2) количества элементов указанного конечного \textit{sc-множества}, (3) какому подклассу заданного \textit{sc-класса} принадлежит указанный элемент этого \textit{sc-класса}.}
		\scnfileitem{В \textit{памяти ostis-системы} каждый \textit{sc-элемент} считается в рамках этой памяти \textit{временной} сущностью (имеется в виду сам \textit{sc-элемент}, а не обозначаемая им сущность), поскольку он появляется в \textit{памяти ostis-системы} и удаляется из нее независимо от того, что он обозначает. В отличие от этого в \textit{SC-пространстве} все sc-элементы считаются постоянными (\textit{постоянно} присутствующими) в рамках этого пространства.}
	\end{scnrelfromlist}

	\scnheader{Базовая денотационная семантика SC-кода}
	\scnidtf{Онтология Базовой денотационной семантики SC-кода}
	\scnidtf{Формальная \textit{онтология}, представленная в \textit{SC-коде} и являющаяся материнской \textit{онтологией} (онтологией самого высокого уровня) для всех остальных \textit{формальных онтологий}, представленных в \textit{SC-коде}}
	\scnidtf{Онтология SC-пространства}
	\scnidtf{Описание (представление) системы \textit{правил построения семантически корректируемых sc-конструкций}, удовлетворяющих требованиям Базовой денотационной семантики SC-кода}
	\scniselement{sc-онтология}
	\begin{scnindent}
		\scnidtf{формальная онтология, представленная в SC-коде}
	\end{scnindent} 
	\scnsuperset{\textbf{Семантическая классификация sc-элементов по базовым признакам}}
	\scnsuperset{\textbf{Уточнение смысла выделенных классов sc-элементов и связей между этими классами}}
	\scnsuperset{\textbf{Структура базовой семантической спецификации sc-элемента}}

	\scnheader{Логическая онтология SC-пространства}
	\scnrelto{логическая онтология}{Базовая денотационная семантика SC-пространства}
	\begin{scnrelfromset}{Правила, входящие в состав Логической онтологии SC-пространства}
		\scnfileitem{Вторыми компонентами \textit{sc-пар} константной парой принадлежности могут быть sc-элементы \textit{любого} типа (в том числе, и \textit{sc-переменные}), но первыми компонентами таких \textit{sc-пар} могут быть только \textit{константные} \textit{sc-множества}.}
		\scnfileitem{Знак \textit{sc-ситуации} связан с элементами этой ситуации \textit{sc-парами} константной \textit{постоянной} позитивной принадлежности. То есть позитивная принадлежность считается постоянной в рамках интервала времени существования соответствующей ситуации. В этом смысле ситуацию можно считать квазистатической.}
		\scnfileitem{Знак атомарной логической формулы связан со всеми элементами этой формулы \textit{sc-парами} \textit{константной} постоянной позитивной принадлежности, в том числе, и с теми элементами атомарной формулы, которые являются \textit{sc-переменными}.}
		\scnfileitem{Из переменного \textit{sc-множества} могут выходить только переменные \textit{sc-пары принадлежности}}
		\scnfileitem{Не существует sc-пар принадлежности выходящих из обозначений внешних сущностей и \textit{sc-пар}.}
		\end{scnrelfromset}
\end{scnsubstruct}

		\scnsegmentheader{Смысловое пространство ostis-систем}
\begin{scnsubstruct}

    \begin{scnrelfromlist}{ключевое понятие}
		\scnitem{обобщенная sc-связка}
		\scnitem{обобщенное sc-отношение}
		\scnitem{бинарное sc-отношение}
		\scnitem{слотовое sc-отношение}
		\scnitem{sc-структура*}
		\scnitem{элементарно представленный элемент\scnrolesign}
		\scnitem{полносвязно представленный элемент\scnrolesign}
		\scnitem{полностью представленный элемент\scnrolesign}
		\scnitem{sc-связка\scnrolesign}
		\scnitem{sc-отношение\scnrolesign}
		\scnitem{sc-класс\scnrolesign}
		\scnitem{сущностное замыкание*}
		\scnitem{содержательное замыкание*}
		\scnitem{sc-отношение сходства по слотовым отношениям*}
		\scnitem{sc-отношение семантического сходства по слотовым отношениям*}
		\scnitem{связная sc-структура*}
		\scnitem{семантическое сходство sc-структур*}
		\scnitem{семантическое непрерывное сходство sc-структур*}
		\scnitem{ключевой запрос\scnrolesign}
		\scnitem{минимальный ключевой запрос\scnrolesign}
		\scnitem{полная семантическая окрестность элемента*}
		\scnitem{интроспективный ключевой элемент\scnrolesign}
		\scnitem{топологическое пространство}
		\scnitem{топологическое пространство замыкания инцидентности коннекторов}
		\scnitem{топологическое пространство синтаксического замыкания}
		\scnitem{топологическое пространство сущностного замыкания}
		\scnitem{топологическое пространство содержательного замыкания}
		\scnitem{метрика}
		\scnitem{семантическая метрика}
		\scnitem{метрическое пространство}
		\scnitem{метрическое конечное синтаксическое пространство}
		\scnitem{метрическое конечное семантическое пространство}
		\scnitem{псевдометрика}
		\scnitem{псевдометрическое пространство}
		\scnitem{псевдометрическое конечное семантическое пространство}
	\end{scnrelfromlist}

\begin{scnrelfromvector}{примечание}
	\scnfileitem{Понятие \textbf{\textit{SC-пространства}} наряду с понятием \textbf{\textit{SC-кода}} является необходимым для уточнения и формализации понятия смысла \textit{информационных конструкций} и в \textit{унификации смыслового представления информации}. В \textbf{\textit{SC-пространстве}} можно выделять структуры, связанные как с синтаксическими свойствами текстов \textbf{\textit{SC-кода}}, так и с его семантикой. В последнем случае речь можно вести о \scnqqi{смысловом пространстве}. Смысл \textit{информационной конструкции}, в конечном счете, определяется (1) внутренними связями всех элементарных фрагментов этой конструкции и (2) ее внешними связями с элементами \textit{Cмыслового пространства} (ее положением в контексте). \textit{Смысловое пространство} является результатом естественного становления знаний в процессе их интеграции.}
    \scnfileitem{Важнейшим достоинством \textbf{\textit{SC-пространства}} является возможность уточнения таких понятий, как понятие аналогичности (сходства и отличия) различных описываемых "внешних"{} сущностей, аналогичности унифицированных \textit{семантических сетей} (текстов \textbf{\textit{SC-кода}}), понятие семантической близости описываемых сущностей (в том числе, и текстов \textbf{\textit{SC-кода}}).}
	\scnfileitem{Следует отметить, что в силу абстрактности языков модели \textit{унифицированного семантического представления знаний} и условности выбора меток элементов их текстов, нельзя исключить, что объединение двух произвольных текстов таких языков не будет текстом языка модели \textit{унифицированного семантического представления знаний}. Чтобы избежать результатов подобных эклектических объединений с точки зрения синтаксиса или семантики, для абстрактных языков следует рассматривать множество \scnqqi{смысловых пространств}. Однако, для конкретных (реальных) языков может оказаться достаточным рассмотрение одного \scnqqi{смыслового пространства}.}
    \scnfileitem{Далее рассмотрим:
    \begin{scnitemize}
        \item возможность перехода от sc-текстов к графовым структурам и от них к топологическому пространству;
        \item возможность перехода от sc-текстов к графовым структурам и от них к многообразию (топологическому пространству);
        \item возможность перехода от sc-текстов к графовым структурам и от них к метрическому пространству.
    \end{scnitemize}}
    \scnfileitem{Чтобы исследовать структурные свойства \textbf{\textit{SC-пространства}}, можно использовать уже разработанные модели пространств и связь их известными топологическими моделями, например, такими как \textit{графы}. При этом изначально не будем принимать в расчет динамические особенности, связанные с обработкой знаний, однако позже будет показано, что учет динамики в процессах обработки и при становлении знаний является необходимым для вычисления семантической метрики, являющейся одним из определяющих признаков знаний.
    Обратимся к исследованию структурно-топологических свойств пространства.}
    \scnfileitem{Структурно-топологические свойства могут свидетельствовать о синтаксических или семантических зависимостях обозначений текстов языка, позволяющих упростить работу со структурами за счет перехода к более простым структурам на уровнях управления данными или знаниями в характерных задачах управления для \textit{библиотеки многократно используемых компонентов ostis-систем}.}
    \begin{scnindent}
    	\scnrelfrom{смотрите}{Комплексная библиотека многократно используемых семантически совместимых компонентов ostis-систем}
    \end{scnindent}
    \scnfileitem{На множестве элементов, образующих \textbf{\textit{SC-пространство}}, можно изучать топологические свойства и рассматривать \textbf{\textit{SC-пространство}} как топологическое пространство. Следует заметить, что, несмотря на то, что \textbf{\textit{SC-код}} ориентирован на смысловое представление знаний, в силу наличия \textit{не-факторов}, не все смыслы могут быть представлены в некоторый момент времени и не будет известна структура соответствующего представления. Поэтому структурно-топологические свойства текстов \textit{языка представления знаний} скорее определяют синтаксическое пространство, нежели семантическое (смысловое). Хотя оба могут приближаться друг к другу по мере устранения неопределенностей, вызванных \textit{не-факторами}.}
    \scnfileitem{Рассмотрим следующие виды \textit{топологических пространств}:
    \begin{scnitemize}
        \item \textit{топологическое пространство замыкания инцидентности коннекторов};
        \item \textit{топологическое пространство синтаксического замыкания};
        \item \textit{топологическое пространство сущностного замыкания};
        \item \textit{топологическое пространство содержательного замыкания}.
    \end{scnitemize}}
\end{scnrelfromvector}
	
	\scnheader{топологическое пространство}
	\scntext{пояснение}{\textit{топологическое пространство} --- \textit{множество} с определенным над ним \textit{множеством} (семейством) (открытых) подмножеств, включая само \textit{множество} и \textit{пустое множество}. Для любого \textit{подмножества} семейства результат объединения принадлежит \textit{семейству множеств}, а для любого конечного \textit{подмножества семейства} результат пересечения также принадлежит \textit{семейству множеств}. Дополнения множеств семейства до наибольшего из множеств называются \textit{замкнутыми множествами}.}

	\scnheader{обобщенная sc-связка}
	\scnidtf{непустое sc-множество}

	\scnheader{обобщенное sc-отношение}
	\scnidtf{sc-множество непустых sc-множеств}
	\scnexplanation{обобщенное sc-отношение --- sc-множество обобщенных sc-связок.}

	\scnheader{бинарное sc-отношение}
	\scnexplanation{Бинарное sc-отношение --- sc-множество sc-пар (или обобщенных sc-связок, которым существуют две различные принадлежности sc-элементов или одного и того же sc-элемента).}

	\scnheader{узловая sc-пара}
	\scnexplanation{узловая sc-пара --- sc-пара, которая не может быть обозначена sc-дугой принадлежности (позитивной, негативной или нечеткой).}

    \scnheader{явление принадлежности}
    \scnexplanation{явление принадлежности --- множество явлений, каждое из которых является слотовым sc-отношением, которому постоянно непринадлежат sc-дуги постоянной непринадлежности.}

    \scnheader{становление*}
    \scnexplanation{становление* --- бинарное sc-отношение между событиями (состояниями) или явлениями.}

	\scnheader{непосредственно прежде\scnrolesign}
	\scnrelfrom{первый домен}{становление*}
	\scnrelfrom{второй домен}{установленное событие или явление}

	\scnheader{непосредственно после\scnrolesign}
	\scnrelfrom{первый домен}{становление*}
	\scnrelfrom{второй домен}{устанавливающее событие или явление}

    \scnheader{продолжительность*}
    \scnexplanation{продолжительность* --- транзитивное замыкание sc-отношения становления.}

	\scnheader{раньше\scnrolesign}
	\scnrelfrom{первый домен}{продолжительность*}
	\scnrelfrom{второй домен}{раннее событие или явление}

	\scnheader{позже\scnrolesign}
	\scnrelfrom{первый домен}{продолжительность*}
	\scnrelfrom{второй домен}{позднее событие или явление}

	\scnheader{слотовое sc-отношение}
	\scnexplanation{Слотовое sc-отношение --- бинарное sc-отношение (sc-множество (ориентированных) sc-пар), элементы которого не являются узловыми sc-парами.}

	\scnheader{sc-структура*}
	\scnexplanation{sc-структура* --- sc-множество, в котором есть непустое sc-подмножество-носитель (множество первичных элементов sc-структуры*).}

	\scnheader{sc-структура\scnrolesign}
	\scnrelfrom{первый домен}{sc-структура*}
	\scnrelfrom{второй домен}{непустое sc-множество}

	\scnheader{носитель sc-структуры\scnrolesign}
	\scnrelfrom{первый домен}{sc-структура*}
	\scnrelfrom{второй домен}{непустое sc-множество}

	\scnheader{элементарно представленное sc-множество\scnrolesign}
	\scnidtf{элементарно представленный элемент\scnrolesign}
	\scnexplanation{Элементарно представленный элемент\scnrolesign --- элемент sc-структуры*, sc-множество, все элементы которого являются элементами sc-структуры*.}

	\scnheader{полносвязно представленное sc-множество\scnrolesign}
	\scnidtf{полносвязно представленный элемент\scnrolesign}
	\scnexplanation{полносвязно представленный элемент\scnrolesign --- элемент sc-структуры*, sc-множество, все элементы и все принадлежности которому являются элементами sc-структуры*, или sc-дуга, являющаяся элементарно представленным элементом\scnrolesign этой sc-структуры*.}

	\scnheader{полностью представленное sc-множество\scnrolesign}
	\scnidtf{полностью представленный элемент\scnrolesign}
	\scnexplanation{Полностью представленный элемент\scnrolesign --- полносвязно представленный элемент\scnrolesign sc-структуры*, с любым элементом, не являющимся sc-дугой, выходящей из него, связанный принадлежащей этой sc-структуре* sc-дугой принадлежности или sc-дугой непринадлежности.}

	\scnheader{sc-связка\scnrolesign}
	\scnexplanation{sc-связка\scnrolesign --- полносвязно представленный элемент\scnrolesign sc-структуры*, принадлежащий sc-отношению\scnrolesign этой sc-структуры*, являющийся sc-связкой.}

	\scnheader{sc-отношение\scnrolesign}
	\scnexplanation{sc-отношение\scnrolesign --- полносвязно представленный элемент\scnrolesign sc-структуры*, sc-отношение, все элементы которого являются sc-связками\scnrolesign этой sc-структуры*.}

	\scnheader{sc-класс\scnrolesign}
	\scnexplanation{sc-класс\scnrolesign --- полносвязно представленный элемент\scnrolesign sc-структуры*, все элементы которого являются элементами sc-структуры*, не являющийся ни sc-отношением\scnrolesign, ни sc-связкой\scnrolesign этой sc-структуры*.}

	\scnheader{сущностное замыкание*}
	\scnexplanation{Сущностное замыкание* --- наименьшее надмножество* (структура*), в котором каждый элемент является элементарно представленным\scnrolesign.}

	\scnheader{сущностное замыкание\scnrolesign}
	\scnrelfrom{первый домен}{сущностное замыкание*}
	\scnrelfrom{второй домен}{сущностное замыкание}

	\scnheader{носитель сущностного замыкания\scnrolesign}
	\scnrelfrom{первый домен}{сущностное замыкание*}
	\scnrelfrom{второй домен}{непустое sc-множество}

	\scnheader{содержательное замыкание*}
	\scnexplanation{содержательное замыкание* --- наименьшее надмножество* (структура*), в котором каждый элемент является полносвязно представленным\scnrolesign}

	\scnheader{содержательное замыкание\scnrolesign}
	\scnrelfrom{первый домен}{содержательное замыкание*}
	\scnrelfrom{второй домен}{содержательное замыкание}

	\scnheader{носитель содержательного замыкания\scnrolesign}
	\scnrelfrom{первый домен}{содержательное замыкание*}
	\scnrelfrom{второй домен}{непустое sc-множество}

	\scnheader{sc-отношение сходства по слотовым отношениям*}
	\scnexplanation{sc-отношение сходства по слотовым sc-отношениям* --- sc-отношение, являющееся рефлексивным по этим слотовым отношениям, то есть для любого элемента, входящего в связку этого sc-отношения под одним из слотовых sc-отношений, найдется связка этого sc-отношения, в которую он входит под каждым из этих слотовых sc-отношений.}

	\scnheader{sc-отношение сходства по слотовым отношениям\scnrolesign}
	\scnrelfrom{первый домен}{sc-отношение сходства по слотовым отношениям*}
	\scnrelfrom{второй домен}{sc-отношение сходства по слотовым отношениям}

	\scnheader{слотовые отношения сходства sc-отношения\scnrolesign}
	\scnrelfrom{первый домен}{sc-отношение сходства по слотовым отношениям*}
	\scnrelfrom{второй домен}{слотовые отношения сходства sc-отношения}

	\scnheader{sc-отношение семантического сходства по слотовым отношениям*}
	\scnexplanation{sc-отношение семантического сходства по слотовым отношениям* --- sc-отношение сходства по слотовым sc-отношениям* si и sj, в котором каждый элемент под слотовым sc-отношением si, может быть преобразован к элементу синтаксического типа элемента под слотовым sc-отношением sj; два инцидентных sc-элемента под слотовым sc-отношением si, в рамках этого sc-отношения семантического сходства соответствуют инцидентным элементам соответственно под слотовым sc-отношением sj.}

	\scnheader{sc-отношение семантического сходства по слотовым отношениям\scnrolesign}
	\scnrelfrom{первый домен}{sc-отношение семантического сходства по слотовым отношениям*}
	\scnrelfrom{второй домен}{sc-отношение семантического сходства по слотовым отношениям}

	\scnheader{слотовые отношения семантического сходства sc-отношения\scnrolesign}
	\scnrelfrom{первый домен}{sc-отношение семантического сходства по слотовым отношениям*}
	\scnrelfrom{второй домен}{слотовые отношения семантического сходства sc-отношения}

	\scnheader{связная sc-структура*}
	\scnexplanation{Связная sc-структура* --- sc-структура*, являющаяся связной.}

	\scnheader{связная sc-структура\scnrolesign}
	\scnrelfrom{первый домен}{связная sc-структура*}
	\scnrelfrom{второй домен}{связное непустое sc-множество}

	\scnheader{носитель связной sc-структуры\scnrolesign}
	\scnrelfrom{первый домен}{связная sc-структура*}
	\scnrelfrom{второй домен}{непустое sc-множество}

	\scnheader{семантическое сходство sc-структур*}
	\scnidtf{семантическое подобие sc-структур*}
	\scnexplanation{Семантическое сходство sc-структур* --- связывает sc-множество sc-структур* с sc-структурой* sc-отношением семантического сходства по слотовым sc-отношениям si, sj так, что для каждой sc-структуры* из sc-множества найдется ее элемент и связка этого sc-отношения сходства, в которую он входит под слотовым sc-отношением si, а под слотовым sc-отношением sj входит элемент sc-структуры*, также для каждого элемента sc-структуры найдется связка этого sc-отношения сходства, в которую он входит под слотовым sc-отношением sj, а под слотовым sc-отношением si входит элемент sc-структуры* из sc-множества.}

	\scnheader{sc-отношение семантического сходства sc-структур\scnrolesign}
	\scnrelfrom{первый домен}{семантическое сходство sc-структур*}
	\scnrelfrom{второй домен}{sc-отношение семантического сходства по слотовым отношениям*}

	\scnheader{семантическое сходство sc-структур\scnrolesign}
	\scnrelfrom{первый домен}{семантическое сходство sc-структур*}
	\scnrelfrom{второй домен}{sc-структура семантического сходства sc-структур*}

	\scnheader{sc-структура семантического сходства sc-структур\scnrolesign}
	\scnrelfrom{первый домен}{sc-структура семантического сходства sc-структур*}
	\scnrelfrom{второй домен}{sc-структура семантического сходства sc-структур}

	\scnheader{множество семантически сходных sc-структур\scnrolesign}
	\scnrelfrom{первый домен}{sc-структура семантического сходства sc-структур*}
	\scnrelfrom{второй домен}{множество семантически сходных sc-структур}

	\scnheader{семантическое непрерывное сходство sc-структур*}
	\scnidtf{семантическое непрерывное подобие sc-структур*}
	\scnexplanation{Семантическое непрерывное сходство sc-структур* --- связывает sc-множество sc-структур* со связной sc-структурой* sc-отношением семантического сходства по слотовым sc-отношениям si, sj так, что для каждой sc-структуры* из sc-множества найдется ее элемент и связка этого sc-отношения сходства, в которую он входит под слотовым sc-отношением si, а под слотовым sc-отношением sj входит элемент связной sc-структуры*, также для каждого элемента связной sc-структуры найдется связка этого sc-отношения сходства, в которую он входит под слотовым sc-отношением sj, а под слотовым sc-отношением si входит элемент sc-структуры* из sc-множества.}

	\scnheader{sc-отношение семантического непрерывного сходства sc-структур\scnrolesign}
	\scnrelfrom{первый домен}{семантическое непрерывное сходство sc-структур*}
	\scnrelfrom{второй домен}{sc-отношение семантического непрерывного сходства по слотовым отношениям*}

	\scnheader{семантическое непрерывное сходство sc-структур\scnrolesign}
	\scnrelfrom{первый домен}{семантическое непрерывное сходство sc-структур*}
	\scnrelfrom{второй домен}{sc-структура семантического непрерывного сходства sc-структур*}

	\scnheader{sc-структура семантического непрерывного сходства sc-структур\scnrolesign}
	\scnrelfrom{первый домен}{sc-структура семантического непрерывного сходства sc-структур*}
	\scnrelfrom{второй домен}{sc-структура семантического непрерывного сходства sc-структур}

	\scnheader{множество семантически непрерывно сходных sc-структур\scnrolesign}
	\scnrelfrom{первый домен}{sc-структура семантического непрерывного сходства sc-структур*}
	\scnrelfrom{второй домен}{множество семантически непрерывно сходных sc-структур}

	\scnheader{ключевой запрос\scnrolesign}
	\scnrelfrom{первый домен}{ключевой запрос*}
	\scnrelfrom{второй домен}{ключевой запрос}
	\scnexplanation{Ключевой запрос\scnrolesign --- поисковый-проверочный запрос (от одного известного элемента), который выполняется хотя бы от одного элемента и не выполняется хотя бы от одного элемента.}

	\scnheader{элемент ключевого запроса\scnrolesign}
	\scnrelfrom{первый домен}{ключевой запрос*}
	\scnrelfrom{второй домен}{элемент ключевого запроса}

	\scnheader{минимальный ключевой запрос\scnrolesign}
	\scnsubset{ключевой запрос\scnrolesign}
	\scnexplanation{Минимальный ключевой запрос --- ключевой запрос, который находит sc-подмножества множеств элементов, находимых всеми другими ключевыми запросами, которые имеют те же области известных элементов выполнимости и невыполнимости.}

	\scnheader{элемент минимального ключевого запроса\scnrolesign}
	\scnrelfrom{первый домен}{минимальный ключевой запрос*}
	\scnrelfrom{второй домен}{элемент минимального ключевого запроса}

	\scnheader{полная семантическая окрестность элемента*}
	\scnexplanation{Полная семантическая окрестность элемента* --- все элементы, находимые выполнимыми минимальными ключевыми запросами от этого элемента (c учетом дизъюнктивного поиска и отрицания поиска).}

	\scnheader{полная семантическая окрестность элемента\scnrolesign}
	\scnrelfrom{первый домен}{полная семантическая окрестность элемента*}
	\scnrelfrom{второй домен}{полная семантическая окрестность элемента}

	\scnheader{элемент полной семантической окрестности\scnrolesign}
	\scnrelfrom{первый домен}{полная семантическая окрестность элемента*}
	\scnrelfrom{второй домен}{элемент полной семантической окрестности}

	\scnheader{интроспективный ключевой элемент\scnrolesign}
	\scnexplanation{интроспективный (базовый) ключевой элемент --- элемент множества (из класса наименьших таких множеств) элементов такого, что любая полная семантическая окрестность любого элемента является sc-подмножеством объединения их полных семантических окрестностей.}

	\scnheader{топологическое пространство замыкания инцидентности коннекторов}
	\scnexplanation{Топологическое пространство замыкания инцидентности коннекторов на множестве sc-элементов --- топологическое пространство, все замкнутые множества которого содержат все sc-элементы этого множества, до которых есть маршрут по ориентированным связкам отношения инцидентности коннекторов.}
	\scntext{примечание}{В общем случае не удовлетворяет аксиоме отделимости по Тихонову. Прагматика рассмотрения таких пространств обуславливается операциями удаления sc-элементов и коннекторов, которым они инцидентны. Удаление sc-элемента требует удаления всех коннекторов, замыканию любой открытой окрестности которых он принадлежит.}

	\scnheader{топологическое подпространство замыкания инцидентности коннекторов\scnrolesign}
	\scnrelfrom{первый домен}{включение топологических пространств замыкания инцидентности коннекторов*}
	\scnrelfrom{второй домен}{топологическое пространство замыкания инцидентности коннекторов}

	\scnheader{топологическое надпространство замыкания инцидентности коннекторов\scnrolesign}
	\scnrelfrom{первый домен}{включение топологических пространств замыкания инцидентности коннекторов*}
	\scnrelfrom{второй домен}{топологическое пространство замыкания инцидентности коннекторов}

	\scnheader{топологическое пространство синтаксического замыкания}
	\scnexplanation{Топологическое пространство синтаксического замыкания на множестве sc-элементов --- топологическое пространство, все замкнутые множества которого содержат все sc-элементы этого множества, до которых есть маршрут по ориентированным связкам отношения инцидентности.}
	\scntext{примечание}{В общем случае не удовлетворяет аксиоме отделимости по Колмогорову. В качестве основы замкнутых множеств топологического пространства можно выделить синтаксическое замыкание, однако в силу возможности проведения дуг из любого sc-узла в любой в итоговом случае (в итоге процесса устранения не-факторов) такое пространство является тривиальным (антидискретным) пространством. Отношение объединения топологических пространств синтаксического замыкания алгебраически не замкнуто на множестве топологических пространств синтаксического замыкания. По той же причине для любого неполного топологического пространства синтаксического замыкания можно рассмотреть топологическое пространство синтаксического замыкания, носитель которого является надмножеством носителя первого и которое не сохраняет замкнутые множества. В этом смысле топология на основе синтаксического замыкания не является устойчивой относительно процессов становления знаний и ее рассмотрение прагматически не оправдывается. Топология полного же топологического пространства синтаксического замыкания антидискретна (тривиальна). Таким образом, у полного топологического пространства синтаксического замыкания все топологические подпространства синтаксического замыкания обладают антидискретной (тривиальной) топологией.}

	\scnheader{топологическое пространство сущностного замыкания}
	\scnexplanation{Топологическое пространство сущностного замыкания на множестве sc-элементов --- топологическое пространство, все замкнутые множества которого являются сущностными замыканиями.}
	\scntext{примечание}{В общем случае не удовлетворяет аксиоме отделимости по Тихонову. В качестве носителя топологического (под)пространства можно выделить сущностное замыкание. Топологическое пространство сущностного замыкания сохраняет замкнутые множества любых топологических пространств сущностного замыкания, носитель которых является подмножеством его носителя и сущностным замыканием. Такие пространства образуют структуру топологических подпространств-топологических надпространств сущностного замыкания. Топология пространств в этой структуре разнообразна.}

	\scnheader{топологическое подпространство сущностного замыкания\scnrolesign}
	\scnrelfrom{первый домен}{включение топологических пространств сущностного замыкания*}
	\scnrelfrom{второй домен}{топологическое пространство сущностного замыкания}

	\scnheader{топологическое надпространство сущностного замыкания\scnrolesign}
	\scnrelfrom{первый домен}{включение топологических пространств сущностного замыкания*}
	\scnrelfrom{второй домен}{топологическое пространство сущностного замыкания}

	\scnheader{топологическое пространство содержательного замыкания}
	\scnexplanation{Топологическое пространство содержательного замыкания на множестве sc-элементов --- топологическое пространство, все замкнутые множества которого являются содержательными замыканиями.}
	\scntext{примечание}{В общем случае не удовлетворяет аксиоме отделимости по Тихонову. В качестве носителя топологического (под)пространства можно выделить содержательное замыкание. Топологическое пространство содержательного замыкания сохраняет замкнутые множества любых топологических пространств содержательного замыкания, носитель которых является подмножеством его носителя и содержательным замыканием. Такие пространства образуют структуру топологических подпространств-топологических надпространств содержательного замыкания. Топология пространств в этой структуре разнообразна.}
	
	\scnheader{топологическое подпространство содержательного замыкания\scnrolesign}
	\scnrelfrom{первый домен}{включение топологических пространств содержательного замыкания*}
	\scnrelfrom{второй домен}{топологическое пространство содержательного замыкания}


	\scnheader{топологическое надпространство содержательного замыкания\scnrolesign}
	\scnrelfrom{первый домен}{включение топологических пространств содержательного замыкания*}
	\scnrelfrom{второй домен}{топологическое пространство содержательного замыкания}
    \scntext{примечание}{Возможен переход от sc-структур к многообразиям и топологическим пространствам путем сведения sc-структур к графовым структурам.
        \\Для более сложных структур таких, как полная семантическая окрестность, топологические свойства подлежат дальнейшему изучению.
        \\Далее можно рассмотреть метрические пространства, в частности --- конечные подпространства с полностью представленными sc-элементами. }
        \begin{scnindent}
        	\scnrelfrom{смотрите}{\scncite{Ivashenko2022}}
        \end{scnindent}


	\scnheader{метрика}
	\scnexplanation{Метрика --- функция двух аргументов, принимающая значения на (линейно) упорядоченном носителе группы, неотрицательна, равна нейтральному элмененту (нулю) только при равенстве аргументов, симметрична, удовлетворяет неравенству треугольника.}

	\scnheader{метрическое пространство}
	\scnexplanation{Метрическое пространство --- множество, с определенной на нем функцией двух аргументов, являющейся метрикой, принимающей значения на упорядоченном носителе группы.}

	\scnheader{семантическая метрика}
	\scnidtf{семантическая близость}
	\scnexplanation{Семантическая метрика --- метрика, определенная на знаках и выражающая количественно близость их значений.}

	\scnheader{метрическое конечное синтаксическое пространство}
	\scnexplanation{Метрическое конечное синтаксическое пространство SC-кода --- метрическое пространство с конечным носителем, элементами которого являются обозначения (sc-элементы), а значение метрики может быть определено через отношения инцидентности элементов без учета их семантического типа.}

	\scnheader{метрическое конечное синтаксическое подпространство\scnrolesign}
	\scnrelfrom{первый домен}{включение метрических конечных синтаксических пространств*}
	\scnrelfrom{второй домен}{метрическое конечное синтаксическое пространство}

	\scnheader{метрическое конечное синтаксическое надпространство\scnrolesign}
	\scnrelfrom{первый домен}{включение метрических конечных синтаксических пространств*}
	\scnrelfrom{второй домен}{метрическое конечное синтаксическое пространство}

	\scnheader{метрическое конечное семантическое пространство}
	\scnexplanation{Метрическое конечное семантическое пространство SC-кода --- метрическое пространство с конечным носителем, элементами которого являются обозначения (sc-элементы), а значение метрики не может быть определено через отношения инцидентности элементов без учета их семантического типа.}

	\scnheader{метрическое конечное семантическое подпространство\scnrolesign}
	\scnrelfrom{первый домен}{включение метрических конечных семантических пространств*}
	\scnrelfrom{второй домен}{метрическое конечное семантическое пространство}

	\scnheader{метрическое конечное семантическое надпространство\scnrolesign}
	\scnrelfrom{первый домен}{включение метрических конечных семантических пространств*}
	\scnrelfrom{второй домен}{метрическое конечное семантическое пространство}

    \scntext{примечание}{Метрическое конечное синтаксическое пространство может быть построено в соответствии с моделью обработки строк и определениями метрики.}
	\begin{scnindent}
		\begin{scnrelfromset}{смотрите}
			\scnitem{\scncite{Ivashenko2022}}
			\scnitem{\scncite{Ivashenko2020String}}
		\end{scnrelfromset}
	\end{scnindent}
	
	\scnheader{псевдометрика}
	\scnexplanation{Псевдометрика --- функция двух аргументов, принимающая значения на (линейно) упорядоченном носителе группы, неотрицательна, симметрична, удовлетворяет неравенству треугольника.}

	\scnheader{псевдометрическое пространство}
	\scnexplanation{псевдометрическое пространство --- множество, с определенной на нем функцией двух аргументов, являющейся псевдометрикой, принимающей значения на упорядоченном носителе группы.}
	\begin{scnindent}
		\scnrelfrom{смотрите}{\scncite{Collatz1966}}
	\end{scnindent}

	\scnheader{псевдометрическое конечное семантическое пространство}
	\scnexplanation{Псевдометрическое конечное семантическое пространство SC-кода --- псевдометрическое пространство с конечным носителем, элементами которого являются обозначения (sc-элементы), а значение псевдометрики не может быть определено через отношения инцидентности элементов без учета их семантического типа.}
    \begin{scnrelfromvector}{примечание}
    	\scnfileitem{В силу неполноты выразительных средств для представления изменяющихся со временем знаний, отсутствия определенной пространственно-временной модели, наличия семантически неопределенных или слабоопределенных обозначений в текстах да и наличия недоопределенности самих текстов описанного в предыдущих разделах языка, на данном этапе в этом описании затруднительно предложить какую-либо модель метрического пространства для более сложных структур, учитывающих не-факторы, связанные с пространством-временем.}
        \scnfileitem{Предложенные модели полагались на представление, способное выразить семантику переменных обозначений и операционную семантику расширенными средствами алфавита. Для построения подобных моделей, кроме расширенных средств алфавита, предлагается полагаться на модели, описывающие процессы интеграции и становления знаний на средства спецификации знаний, ориентированные на рассмотрение финитных структур, что позволяет перейти к рассмотрению сложных метрических соотношений в рамках метамодели смыслового пространства.}
		\begin{scnindent}
			\begin{scnrelfromset}{смотрите}
				\scnitem{\scncite{Ivashenko2022}}
				\scnitem{\scncite{Ivashenko2017}}
			\end{scnrelfromset}
		\end{scnindent}
        \scnfileitem{В разных науках исследователи затрагивали вопросы касающиеся смыслов и их размещения и взаимосвязи. Можно выделить следующие работы, которые соотносятся с тремя подходами: экстериорный подход, интериорные подходы на основании количественных и структурно-динамических признаков.}
        \begin{scnindent}
        	\begin{scnrelfromset}{смотрите}
        		\scnitem{\scncite{Bohm1993}}
        		\scnitem{\scncite{Nalimov1995}}
        		\scnitem{\scncite{Bohm2002}}
        		\scnitem{\scncite{Martynov2004}}
        		\scnitem{\scncite{Nalimov1979}}
        		\scnitem{\scncite{Nalimov1989}}
        	\end{scnrelfromset}
        \end{scnindent}
        \scnfileitem{В современных работах в технических науках, возможно, наиболее близкими понятиями являются понятия, выражающие смысл термина \scnqqi{семантическое пространство} (интериорный подход).
        Общим во многих подходах к работе с \scnqqi{семантическим пространством} является рассмотрение словоформ или лексем (множеств словоформ) и их признаков. В литературе встречаются следующие подходы:
        \begin{scnitemize}
            \item подход на основе семантических осей и пространства признаков (бинарных $\left\lbrace 0,1\right\rbrace ^{n}$, монополярных $\left[0;1\right]^{n}$, биполярных $\left[-1;1\right]^{n}$);
            \item подход на основе семантических осей и нейронного кодирования места в поле смыслов (слова и словосочетания имеют области (подмножества) значений, связываясь другими частями речи как включением и пересечением, тексты соответствуют пути связанных областей, бинарное кодирование групп нейронов, распознающих смыслы);
            \item подход на основе модели \scnqqi{смысл-текст} (отражение неполноты семантических шкал и анализ синтагм и поверхностно-синтаксической структуры);
            \item нейролингвистические данные отражает процессы синтеза и восприятия речи в нейронных сетях (сеть лексического синтеза), близка к модели \scnqqi{смысл-текст};
            \item модели, построенные на основе статического анализа (корпусов) текстов (модель векторного пространства).
        \end{scnitemize}}
    	\begin{scnindent}
    		\begin{scnrelfromset}{смотрите}
    			\scnitem{\scncite{Manin2016}}
    			\scnitem{\scncite{Melchuk2016}}
    			\scnitem{\scncite{Harris1992}}
    		\end{scnrelfromset}
    	\end{scnindent}
        \scnfileitem{Статистический подход к обработке естественного языка противопоставляется интуиции и коммуникативному опыту ученых. В основе подхода лежит семантическая статистическая гипотеза, что смысл слов (лексем) определяется контекстом использования (его статистическим образом) в языке (с коммуникативной структурой.}
		\begin{scnindent}
			\scnrelfrom{смотрите}{\scncite{Manin2016}}
		\end{scnindent}
        \scnfileitem{Модель векторного пространства семантики. Модель рассматривается для двух случаев: большого словаря ($N\leq{M}$) и задачи информационного поиска ($M\leq{N}$). $M$ -- размер словаря, $N$ -- количество контекстов.}
        \begin{scnindent}
        	\scnrelfrom{смотрите}{\scncite{Manin2016}}
        \end{scnindent}
        \scnfileitem{На основе статистики строится матрица размерности $M\times{N}$ частот $p_{ij}$ появления лексемы (слова) $w_{i}$ в документе (контексте, подтексты, которые могут перекрываться) $c_{j}$.
        $$
        x_{ij}=\max{\left( \left\lbrace 0\right\rbrace \cup \left\lbrace \log{\left(\frac{p_{ij}}{\left( \Sigma_{j} p_{ij}\right) *\left( \Sigma_{i} p_{ij}\right) } \right) }\right\rbrace \right) }
        $$

        В знаменателе -- оценки вероятности слова и контекста соответственно.

        В случае невырожденной матрицы $r=N$ каждая такая матрица задает точку в грассманиане $N$-мерных подпространств $M$-мерного пространства ($N\leq{M}$).

        В случае невырожденной матрицы $r=M$ каждая такая матрица задает точку в грассманиане $M$-мерных подпространств $N$-мерного пространства ($M\leq{N}$).}
        \scnfileitem{Каждый текст -- точка в грассманиане, соответствующем проективному пространству $P^{M-1}=Gr\left( \left\langle1,M \right\rangle \right)$, относительно одного выделенного контекста. Для всех контекстов получая ориентированную $N$-ку, в соответствии с порядком контекстов в текстах, можно построить маршрут (путь), соединяя геодезическими соседние точки в $N$-ке. Для двух текстов $T$ и $T'$ это будут две ломанные, между которыми можно вычислить метрику Фреше, используя метрику Фубини-Штуди в $P^{M-1}$, для этого следует параметризовать пути $\Gamma\left( T \right)$ и $\Gamma\left( T' \right)$ через $t$ ($\gamma\in\Gamma\left( T \right)^{\left[ 0;1\right] }$,$\gamma' \in\Gamma\left( T'\right)^{\left[ 0;1\right] }$): 
        $$
        \delta\left( \left\langle \Gamma\left( T \right),\Gamma\left( T'\right)\right\rangle \right) =\inf_{\gamma,\gamma'}\max_{t\in\left[ 0;1\right] }\left(  \left\lbrace d_{FS}\left( \left\langle \gamma\left( t\right) ,\gamma'\left( t\right) \right\rangle \right) \right\rbrace \right).
        $$}
        \begin{scnindent}
        	\begin{scnrelfromset}{смотрите}
        		\scnitem{\scncite{Alt1995}}
        		\scnitem{\scncite{Study1905}}
        		\scnitem{\scncite{Harris1992}}
        	\end{scnrelfromset}
        \end{scnindent}
        \scnfileitem{Другой способ задать линейный порядок -- это рассмотреть фильтрацию в $\mathbb{R}^{M}$, заданную расширяющимися контекстами. В итоге для текста получаем точки (флаги) во флаговом многообразии. Для флаговых многообразий тоже можно вычислить метрику Фубини-Штуди.}
        \begin{scnindent}
        	\begin{scnrelfromset}{смотрите}
        		\scnitem{\scncite{Kostrikin1997}}
        		\scnitem{\scncite{Study1905}}
        	\end{scnrelfromset}
        \end{scnindent}
        \scnfileitem{Этот порядок соответствует временному измерению (процессу коммуникации во времени), что может быть существенным. Другой порядок может быть не зависимым от этого, например алфавитный или порядок в соответствии с законом Ципфа.}
        \begin{scnindent}
        	\begin{scnrelfromset}{смотрите}
        		\scnitem{\scncite{Lowe2001}}
        		\scnitem{\scncite{Manin2014}}
        	\end{scnrelfromset}
        \end{scnindent}
	\end{scnrelfromvector}
	\scnheader{Таблица. Сравнение подходов к построению \scnqqi{семантических пространств}\\=}


\begin{tabular}{|>{\centering\arraybackslash}m{3cm}|>{\centering\arraybackslash}m{3cm}|>{\centering\arraybackslash}m{5cm}|>{\centering\arraybackslash}m{5.25cm}|}
	\hline
	& экстериорный (физический) подход
	& интериорный (абстрактный, логико-семиотический) подход на основании анализа количественных признаков (вероятностных (аддитивных) мер)
	& интериорный (абстрактный, логико-семиотический) подход на основании анализа структурно-динамических признаков
	\\
	\hline
	анализ когнитивных процессов (интроспекция)
	& +
	& -
	& ?
	\\
	\hline
	адаптация
	& +
	& -
	& +
	\\
	\hline
	унификация
	& -
	& +
	& +
	\\
	\hline
\end{tabular}

\begin{tabular}{|>{\centering\arraybackslash}m{3cm}|>{\centering\arraybackslash}m{2cm}|>{\centering\arraybackslash}m{2cm}|>{\centering\arraybackslash}m{3cm}|>{\centering\arraybackslash}m{3cm}|>{\centering\arraybackslash}m{3cm}|}
	\hline
	& семанти-ческие оси и простран-ства признаков
	& семанти-ческие оси и нейронное кодирование признаков
	& модель \scnqqi{смысл-текст}
	& нейролингвисти-ческое кодирование
	& статистическая модель (модель векторного пространства семантики)
	\\
	\hline
	определенные семантические оси
	& +
	& +
	& -
	& -
	& -
	\\
	\hline
	динамическая (вычислительная) декомпозиция
	& -
	& +
	& -
	& +
	& -
	\\
	\hline
	анализ когнитивных процессов (интроспекция)
	& -
	& +
	& -
	& +
	& -
	\\
	\hline
	учет не-факторов (неполнота)
	& -
	& -
	& +
	& +
	& +
	\\
	\hline
\end{tabular}


\scntext{примечание}{Вопросы соотнесения смыслов, их формализации, развития языков в пространстве и времени рассмотрены в работах В.В. Мартынова (см. \scncite{Martynov2004}, \scncite{Martynov2009}, \scncite{Gordey2014}).}


\end{scnsubstruct}


        \begin{scnrelfromvector}{заключение}
            \scnfileitem{\textit{SC-код} может быть использован в качестве метаязыка для описания \textit{собственной} денотационной семантики и синтаксиса.}
            \scnfileitem{Аспекты представления знаний в памяти \textit{интеллектуальных компьютерных систем}, которые требуют особой аккуратности.}
            \begin{scnindent}
            	\begin{scnrelfromvector}{разбиение}
	                \scnfileitem{Константный и переменный характер денотационной семантики знаков, хранимых в памяти \textit{интеллектуальных компьютерных систем}.}
	                \scnfileitem{Динамический характер знаковых конструкций, хранимых в памяти, обусловленный либо выполняемыми в памяти \textit{информационными процессами}, либо динамическим характером структур внешних объектов, описываемых этими знаковыми конструкциями.}
	                \scnfileitem{Временный характер существования внешних описываемых объектов и временный характер существования различных конфигураций знаковых конструкций и даже самих \textit{знаков, хранимых в памяти}.}
	                \scnfileitem{Наличие информационного мусора (излишеств) в хранимых знаниях.}
	                \scnfileitem{Неизвестность (отсутствие в памяти) востребованной информации различного вида, неполнота знаний, их недостаточность для решения актуальных задач.}
	                \scnfileitem{Синтаксическая и семантическая некорректность, неадекватность и,  в частности, противоречивость некоторых имеющихся знаний.}
                \end{scnrelfromvector}
                \scntext{примечание}{Последние из перечисленных аспектов представления знаний называют \textit{не-факторами представления знаний}}
                \begin{scnindent}
                	\scnrelfrom{смотрите}{\scncite{Narinjani2000}}
                \end{scnindent}
            \end{scnindent}
        \end{scnrelfromvector}
        \scnrelfrom{пример}{Типология \textit{sc-конструкций} с точки зрения \textit{Денотационной семантики и Синтаксиса SC-кода}}

        \scnheader{Типология \textit{sc-конструкций} с точки зрения \textit{Денотационной семантики и Синтаксиса SC-кода}}
        \begin{scneqtoset}
            \scnitem{sc-множество}
			\begin{scnindent}
				\scnidtf{\textit{sc-конструкция}}
				\scnidtf{информационная конструкция, принадлежащая \textit{SC-коду}}
				\scnsuperset{sc-структура}
				\begin{scnindent}
					\scnsuperset{sc-текст}
					\begin{scnindent}
						\scnidtftext{часто используемый sc-идентификатор}{\textit{SC-код}}
						\begin{scnindent}
							\scniselement{имя собственное}
						\end{scnindent} 
						\scnidtf{синтаксически целостная и синтаксически корректная (правильно построенная) информационная конструкция SC-кода}
						\scnidtf{Класс (Множество всевозможных) sc-текстов}
						\scnsuperset{sc-знание}
						\begin{scnindent}
							\scnidtf{семантически целостный и семантически корректный \textit{sc-текст}, являющийся адекватным фрагментом соответствующей \textit{предметной области} или ее спецификации (онтологии)}
						\end{scnindent} 
					\end{scnindent} 
				\end{scnindent}
			\end{scnindent}
        \end{scneqtoset}
\end{scnsubstruct}
\scnendcurrentsectioncomment

\end{SCn}
\scsubsubsection{Предметная область и онтология синтаксиса внутреннего языка ostis-систем}
\label{sd_sc_code_syntax}
\scsubsubsection{Предметная область и онтология базовой денотационной семантики внутреннего языка ostis-систем}
\label{sd_sc_code_semantic}
