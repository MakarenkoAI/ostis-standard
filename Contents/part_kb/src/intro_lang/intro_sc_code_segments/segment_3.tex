\scnsegmentheader{SC-код как синтаксическое расширение Ядра SC-кода}
\begin{scnsubstruct}
        \scnstructheader{Сравнение SC-кода и Ядра SC-кода}
        \begin{scnsubstruct}
            \scnheader{следует отличать*}
            \begin{scnhaselementset}
                \scnitem{SC-код}
                \begin{scnindent}
                    \scnrelto{синтаксическое расширение языка}{Ядро SC-кода}
                    \scnidtf{Синтаксическое расширение Ядра SC-кода}
                    \begin{scnindent}
                        \scntext{примечание}{Синтаксическое расширение Ядра SC-кода заключается во введении дополнительного класса синтаксически эквивалентных элементарных фрагментов конструкций Ядра SC-кода --- sc-элементов, обозначающих \scnqq{внутренние} файлы, хранимые в памяти ostis-системы}
                    \end{scnindent}
                    \scnrelboth{семантическая эквивалентность языков}{Ядро SC-кода}
                    \scntext{примечание}{Семантическая эквивалентность \textit{SC-кода} и \textit{Ядра SC-кода} является следствием того, что \textit{SC-код} является \uline{синтаксическим} расширением \textit{Ядра SC-кода}.}
                    \scnidtf{Результат введения в \textit{Ядро SC-кода} sc-узлов, имеющих содержимое и обозначающих файлы, хранимые в памяти ostis-системы, т.е. внутренние файлы ostis-системы.}
                    \scntext{примечание}{Результатом просмотренного расширения \textit{Ядра SC-кода} является расширение \textit{Алфавита Ядра SC-кода}.}
                    \scntext{примечание}{Все \textit{файлы}, представляющие собой электронные образы инородных для \textit{SC-кода} информационных конструкций, можно представить в \textit{SC-коде} с помощью графовых структур, в которых \textit{sc-элементы} обозначают буквы текстов или пиксели изображений. Но такой вариант кодирования внешних для \textit{ostis-системы} информационных конструкций не дает возможности непосредственно использовать накопленный человечеством арсенал электронных информационных ресурсов.}
                    \scntext{примечание}{Важнейшим видом внутренних \textit{файлов ostis-систем} являются внутренние файлы \textit{внешних идентификаторов sc-элементов} (в частности, имен sc-элементов), представляющих \textit{sc-элементы} в текстах внешних языков (в том числе, в текстах \textit{SCs-кода} и \textit{SCn-кода}).}
                \end{scnindent}
                \scnitem{\scnnonamednode}
                \begin{scnindent}
                     \begin{scneqtoset}
                        \scnitem{Ядро SC-кода}
                        \scnitem{SC-код}
                    \end{scneqtoset}
                    \scntext{сравнение}{Множество всех элементов конструкций \textit{Ядра SC-кода} и Множество всех элементов конструкций \textit{SC-кода} полностью совпадают, т.к. для каждого элемента конструкции \textit{Ядра SC-кода} существует синонимичный ему элемент конструкции \textit{SC-кода} и наоборот. Из этого следует, что семантическая классификации \textit{элементов информационных конструкций} \textit{SC-кода} и \textit{Ядра SC-кода} также полностью совпадают.Семантика \textit{SC-кода} ничем не отличается от семантики \textit{Ядра SC-кода}. То есть все, что может быть обозначено и описано текстами \textit{SC-кода}, может быть обозначено и описано текстами \textit{Ядра SC-кода}. Отличие \textit{SC-кода} от \textit{Ядра SC-кода} заключается только в том, что в \textit{SC-код} добавляется новый синтаксически выделяемый класс sc-элементов --- класс sc-элементов, являющихся знаками конкретных (константных) файлов, хранимых в памяти ostis-системы. Такие \textit{внутренние файлы} необходимы для того, чтобы в \textit{памяти ostis-системы} можно было хранить и обрабатывать \textit{информационные конструкции}, не являющиеся текстами \textit{SC-кода}, что необходимо при вводе (восприятии) информации, поступающей извне, а также при генерации \textit{информационных конструкций}, передаваемых другим субъектам.
                    	\\Включение в \textit{SC-код} специальных \uline{синтаксически} выделяемых \textit{sc-узлов}, обозначающих хранимые в \textit{sc-памяти} электронные образы (файлы) различного вида \textit{информационных конструкций}, не являющихся конструкциями \textit{SC-кода}, дает возможность непосредственно в \textit{памяти ostis-системы}, то есть в одной и той же запоминающий среде обрабатывать не только конструкции \textit{SC-кода}, но и \scnqq{инородные} для него конструкции, что для необходимо для реализации \textit{интерфейса ostis-системы}, обеспечивающего ее взаимодействие с \textit{внешней средой}. Без такой реализации \textit{интерфейса ostis-системы} невозможно реализовать синтаксический анализ, семантический анализ и понимание, а также невозможно реализовать синтез (генерацию) внешних информационных конструкций, принадлежащих заданному внешнему языку и семантически эквивалентных заданному смыслу. 
                    	\\Поскольку все синтаксические и семантические свойства \textit{SC-кода} и \textit{Ядра SC-кода} являются весьма близкими, при описании \textit{SC-кода} акцентируется внимание на его отличия от \textit{Ядра SC-кода}, а также на более детальное рассмотрение семантической классификации элементов.}
                \end{scnindent}
            \end{scnhaselementset}
        \end{scnsubstruct}

        \scnstructheader{Синтаксис SC-кода}
        \begin{scnsubstruct}

            \scnheader{Синтаксис SC-кода}
            \scnidtf{Онтология синтаксиса SC-кода}
            \scnidtf{Описание правил построения \textit{синтаксически корректных sc-конструкций}}
            \scnidtf{Описание требований, предъявляемых к \textit{синтаксически корректным sc-конструкциям}}
            \scniselement{sc-онтология}

            \scnheader{SC-код}
            \scnrelfrom{синтаксис}{синтаксис SC-кода}
            \begin{scnindent}
                \scntext{примечание}{\textit{Синтаксис SC-кода} отличается от \textit{Синтаксиса Ядра SC-кода} только тем, что в \textit{Алфавит \mbox{SC-кода}} дополнительно вводится класс sc-узлов, являющихся знаками \textit{файлов}, хранимых в памяти \textit{\mbox{ostis-системы}}}
            \end{scnindent}
            \scnrelfrom{множество всех экземпляров конструкций данного языка}{sc-элемент}
            \begin{scnindent}
                \scnidtf{элемент конструкции SC-кода}
                \scntext{примечание}{Множество всех элементов конструкций SC-кода совпадает со множеством всех элементов конструкций Ядра SC-кода. Просто в конструкциях SC-кода некоторые sc-элементы, имеющие \scnqq{синтаксическую метку} (синтаксический тип) \textit{sc-узла общего вида}, будут иметь \scnqq{метку} sc-узла, являющегося знаком \textit{внутреннего файла},  хранимого в памяти \textit{ostis-системы}.}
            \end{scnindent}

            \scnheader{Синтаксис SC-кода}
            \begin{scnrelfromvector}{быть заданным}
                \scnfileitem{типология (алфавит) sc-элементов (атомарных фрагментов текстов SC-кода)}
                \scnfileitem{правила соединения (инцидентности) sc-элементов (например, sc-элементы каких типов не могут быть инцидентными друг другу)}
                \scnfileitem{типология конфигураций sc-элементов (связки, классы, структуры), связи между конфигурациями sc-элементов (в частности, теоретико-множественными)}
            \end{scnrelfromvector}
            
            \scnheader{sc-ребро}
            \scnidtf{Класс \textit{sc-элементов}, имеющих в рамках \textit{Ядра SC-кода} синтаксическую метку \textit{обозначений неориентированных sc-пар}}
            \scnidtf{Синтаксическая метка \textit{обозначения неориентированной sc-пары}, используемая в рамках \textit{Ядра SC-кода}}
            \scniselement{синтаксически выделяемый sc-класс в рамках Ядра SC-кода}
        
            \scnheader{sc-дуга общего вида}
            \scnidtf{Класс \textit{sc-элементов}, имеющих в рамках \textit{Ядра SC-кода} синтаксическую метку \textit{обозначений ориентированных sc-пар, не являющихся постоянными позитивными sc-парами принадлежности}}
        
            \scnheader{базовая sc-дуга}
            \scnidtf{Класс \textit{sc-элементов}, имеющих в рамках \textit{Ядра SC-кода} синтаксическую метку \textit{постоянных позитивных sc-пар принадлежности}}
    
            \scnheader{sc-узел, являющийся знаком файла}
            \scnidtf{\textit{sc-элементов}, имеющий в рамках \textit{Ядра SC-кода} синтаксическую метку \textit{sc-элементов}, являющихся знаками \textit{файлов}}
            
            \scnheader{sc-узел, не являющийся знаком файла}
	        \scnidtf{\textit{sc-узел}, имеющий в рамках \textit{Ядра SC-кода} синтаксическую метку \textit{sc-элементов, не являющихся ни знаками файлов, ни обозначениями sc-пар}}

            \scnheader{Алфавит SC-кода}
            \scnrelto{алфавит}{SC-код}
            \scnrelfrom{разбиение}{sc-элемент}
            \begin{scneqtoset}
                  	\scnitem{(Алфавит Ядра SC-кода $\cup$ \scnkeyword{внутренний файл ostis-системы})}
            \end{scneqtoset}
            \begin{scneqtoset}
                \scnitem{sc-узел общего вида}
                \scnitem{внутренний файл ostis-системы}
                \scnitem{sc-ребро общего вида}
                \scnitem{sc-дуга общего вида}
                \scnitem{базовая sc-дуга}
            \end{scneqtoset}
            
            \scnheader{Алфавит SC-кода}
            \scnidtf{Алфавит sc-элементов в рамках SC-кода}
            \scnidtf{Семейство всех максимальных множеств синтаксически эквивалентных (в рамках SC-кода) sc-элементов}
            \scnidtf{Семейство классов синтаксически эквивалентных sc-элементов SC-кода}
            \scnidtf{Семейство всех множеств, в каждое из которых входят все синтаксически эквивалентные друг другу (в рамках SC-кода) sc-элементы и только они}
            \scnidtf{Фактор-множество отношения \scnqq{синтаксическая эквивалентность sc-элементов в рамках SC-кода}}
            \scneq{фактор-множество*(синтаксическая эквивалентность sc-элементов в рамках SC-кода*)}
            \begin{scnindent}
                \scniselement{сложный внешний идентификатор sc-элемента}
            \end{scnindent}
            \scnidtf{Семейство множеств sc-элементов, являющихся результатом разбиения максимального множества \mbox{sc-элементов} SC-кода (класса всевозможных sc-элементов) по признаку синтаксической эквивалентности sc-элементов}
            \scnidtf{Признак (параметр) синтаксической эквивалентности sc-элементов}
            
            \scnheader{внутренний файл ostis-системы}
            \scnidtf{sc-узел, имеющий содержимое}
            \scnidtf{sc-ссылка}
            \scnidtf{множество всевозможных sc-узлов, имеющих содержимое, хранимое в памяти ostis-системы}
            \scnidtf{внутренний файл, хранимый в памяти ostis-системы}
            \scnidtf{внутренний файл для заданной ostis-системы (той ostis-системы, в памяти которой хранится sc-узел, обозначающий этот файл)}
            \scnidtf{sc-узел, являющийся знаком конкретного файла, хранимого в той же sc-памяти (в качестве содержимого sc-узла), в которой находится и сам указанный sc-узел}
            \scnidtf{файл, знак которого находится в той же sc-памяти, в которой находится и сам файл}
            \scnidtf{свой файл ostis-системы}
            \scnsubset{внутренняя информационная конструкция}
            
            \scnheader{синтаксически выделяемый sc-класс}
            \scnidtf{класс \textit{sc-элементов}, имеющих общий (одинаковый) синтаксический признак, который задается либо одной синтаксической меткой, каждая из которых семантически эквивалентна (синонимична) \textit{sc-элементу}, обозначающему соответствующий синтаксически выделяемый \textit{класс sc-элементов}, либо набором таких меток}
            \begin{scnindent}
                \begin{scnrelfromlist}{примечание}
                    \scnfileitem{Наличие у двух разных \textit{sc-элементов} одной и той же синтаксической метки означает то, что оба эти \textit{sc-элемента} принадлежат \textit{sc-классу}, знаком которого является \textit{sc-элемент}, семантически эквивалентный указанной метке}
                    \scnfileitem{Если \textit{sc-элементу} приписывается \textit{несколько} меток, то \textit{синтаксически выделяемым sc-классом} является \textit{пересечение} \textit{sc-классов}, синтаксически выделяемых по каждой из этих меток}
                \end{scnrelfromlist}
            \end{scnindent}
            \scnidtf{\textit{sc-элемент}, обозначающий \textit{sc-класс}, принадлежность которому может быть представлена либо с помощью \textit{sc-пары постоянной позитивной принадлежности}, либо с помощью соответствующей метки, приписываемой этому \textit{sc-элементу}, или набора таких меток}
            \begin{scnindent}
                \scntext{примечание}{Приписывание \textit{sc-элементам} меток ускоряет проверку принадлежности sc-элементов соответствующим классам}
            \end{scnindent}
            \scnidtf{sc-класс, каждому sc-элементу которого приписывается соответствующая этому sc-классу синтаксическая метка, которая является неявной (синтаксической) формой указания факта принадлежности указанного \textit{sc-элемента} указанному \textit{sc-классу}}
            \begin{scnsubdividing}
                \scnitem{синтаксически выделяемый sc-класс в рамках Ядра SC-кода}
                \scnitem{синтаксически выделяемый sc-класс в рамках расширения Ядра SC-кода}
            \end{scnsubdividing}
            \scntext{примечание}{Каждый \textit{синтаксически выделяемый sc-класс} можно считать элементом алфавита соответствующей \textit{Синтаксической модификации элементов SC-кода}. Но, в отличие от других языков, синтаксически выделяемые \textit{sc-классы} большинства синтаксических модификаций SC-кода могут \textit{пересекаться}.
                \\В отличие от этого, особенностью привычных нам языков является то, что каждый элементарный (атомарный) фрагмент информационных конструкций может иметь только \textit{одну} метку, то есть может быть элементом только одного синтаксически выделяемого класса элементарных фрагментов.}
            
            \scnheader{синтаксически выделяемый класс sc-элементов в рамках SC-кода}
            \scnidtf{класс sc-элементов, определяемый на основе Алфавита SC-кода}
            \scnsuperset{Алфавит SC-кода}
            \scnhaselement{sc-узел, не являющийся знаком внутреннего файла ostis-системы}
        \end{scnsubstruct}
        
        \scnstructheader{Синтаксическая классификация sc-элементов в рамках SC-кода}
        \begin{scnsubstruct}                   
            \scnheaderlocal{sc-элемент}
            \begin{scnsubdividing}
                \scnitem{sc-узел общего вида}
                \begin{scnindent}
                    \begin{scnsubdividing}
                        \scnitem{sc-узел, не являющийся знаком внутреннего файла ostis-системы}
                        \scnitem{внутренний файл ostis-системы}
                    \end{scnsubdividing}
                \end{scnindent}
                \scnitem{sc-коннектор}
                \begin{scnindent}
                    \begin{scnsubdividing}
                        \scnitem{sc-ребро общего вида}
                        \scnitem{sc-дуга}
                        \begin{scnindent}
                            \begin{scnsubdividing}
                                \scnitem{sc-дуга общего вида}
                                \scnitem{базовая sc-дуга}
                            \end{scnsubdividing}
                        \end{scnindent}
                    \end{scnsubdividing}
                \end{scnindent}
            \end{scnsubdividing}
            \scntext{примечание}{Данная \textit{Синтаксическая классификация sc-элементов} от \textit{Синтаксической классификации sc-элементов Ядра SC-кода} отличается только дополнительным уточнением синтаксической типологии \textit{sc-узлов}.}
        \end{scnsubstruct}

        \scnstructheader{Денотационная семантика SC-кода}
        \begin{scnsubstruct}
            \scnheader{Денотационная семантика SC-кода}
            \scntext{аннотация}{\textit{Денотационную семантику SC-кода} рассмотрим как расширение и уточнение \textit{Денотационной семантики Ядра SC-кода} (смотрите предыдущий сегмент \scnqq{\textit{Описание Ядра SC-кода}}). Изложение построим как последовательное уточнение следующих понятий:
            \begin{scnitemize}
                \item \textit{sc-переменная}
                \item \textit{обозначение дискретной информационной конструкции}
                \item \textit{дискретная информационная конструкция} (рассмотрим различные параметры и отношения, заданные на множестве дискретных информационных конструкций)
                \item \textit{знание} (как частный вид дискретных информационных конструкций)
                \item \textit{файл} (как \textit{sc-константа}, являющаяся \textit{обозначением файла})
                \item \textit{внутренний файл ostis-системы}
                \item \textit{структура} (как \textit{дискретная информационная конструкция}, принадлежащая \textit{SC-коду})
            \end{scnitemize}
            }
            
            \scnheader{SC-код}
            \scnrelfrom{денотационная семантика}{Денотационная семантика SC-кода}
            \begin{scnindent}
                \begin{scnrelfromvector}{быть заданным}
                    \scnfileitem{семантическая интерпретация sc-элементов и их конфигураций}
                    \scnfileitem{семантическая интерпретация инцидентности sc-элементов}
                    \scnfileitem{иерархическая система \textit{предметных областей}}
                    \scnfileitem{структура используемых понятий в каждой предметной области (исследуемые классы объектов, исследуемые отношения, исследуемые классы объектов отношений из смежных предметных областей, ключевые экземпляры исследуемых классов объектов)}
                    \scnfileitem{\textit{онтология предметных областей}}
                \end{scnrelfromvector}
            \end{scnindent}
        \end{scnsubstruct}    

        \scnstructheader{Классификация sc-переменных}
        \begin{scnsubstruct}
            \scnheader{sc-переменная}
            \scnidtf{sc-элемент, представляющий собой обозначение произвольной (переменной) сущности из некоторого дополнительно уточняемого множества обозначений других сущностей, которые считаются возможными значениями указанной произвольной сущности}
            \scnrelto{область задания}{значение переменной*}
            \begin{scnindent}
                \scnidtf{Бинарное ориентированное отношение, связывающее sc-переменные с их возможными значениями*}
                \scntext{пояснение}{Это одно из отношений, заданных на множестве sc-переменных}
            \end{scnindent}
            \scnrelfrom{разбиение}{\scnkeyword{Структурная типология sc-переменных}}
            \begin{scnindent}
                \begin{scneqtoset}
                    \scnitem{произвольная терминальная сущность}
                        \begin{scnindent}
                            \scnidtf{sc-переменная, обозначающая терминальную сущность}
                            \scnidtf{sc-переменная, значением или значением значения и т.д. которой является терминальная сущность}
                            \scnidtf{sc-переменная, \scnqq{конечным} значением которой является терминальная сущность}
                            \scnidtf{обозначение произвольной терминальной сущности}
                        \end{scnindent}
                    \scnitem{произвольное множество sc-элементов}
                \end{scneqtoset}
            \end{scnindent}
            \begin{scnsubdividing}
                \scnitem{sc-переменная, у которой логический уровень всех ее значений одинаков}
                \begin{scnindent}
                      	\scnsuperset{первичная sc-переменная}
                    \begin{scnindent}
                        \scnidtf{sc-переменная, все значения которой являются sc-константами}
                    \end{scnindent}
                    \scnsuperset{вторичная sc-переменная}
                    \begin{scnindent}
                        \scnidtf{sc-переменная, все значения которой являются первичными sc-переменными}
                    \end{scnindent}
                    \scnsuperset{sc-переменная третьего уровня}
                    \begin{scnindent}    
                        \scnidtf{sc-переменная, все значения которой являются вторичными sc-переменными}
                    \end{scnindent}
                \end{scnindent}
                \scnitem{sc-переменная, значения которой имеют различный логический уровень}
            \end{scnsubdividing}
            \begin{scnsubdividing}
                \scnitem{sc-переменная, у которой синтаксический тип всех её значений одинаков}
                \begin{scnindent}
                    \scnsuperset{переменный sc-узел}
                    \begin{scnindent}
                        \scnidtf{sc-переменная, все значения которой являются sc-узлами}
                    \end{scnindent}
                    \scnsuperset{переменное sc-ребро}
                    \scnsuperset{переменная sc-дуга}
                \end{scnindent}
                \scnitem{sc-переменная, значения которой имеют различный синтаксический тип}
            \end{scnsubdividing}
            
            \scnheader{обозначение дискретной информационной конструкции}
            \begin{scnsubdividing}
                \scnitem{обозначение дискретной информационной конструкции, не принадлежащей SC-коду}
                \scnitem{\scnkeyword{обозначение структуры}}
                \begin{scnindent}
                    \scnidtf{обозначение дискретной информационной конструкции, принадлежащей SC-коду}
                    \scnidtf{обозначение sc-конструкции}
                \end{scnindent}
            \end{scnsubdividing}
            \begin{scnsubdividing}
                \scnitem{произвольная дискретная информационная конструкция}
                \begin{scnindent}
                      	\scnidtf{sc-переменная, обозначающая дискретную информационную конструкцию}
                \end{scnindent}
                \scnitem{\scnkeyword{дискретная информационная структура}}
                \begin{scnindent}
                    \scnidtf{sc-константа, обозначающая конкретную дискретную информационную конструкцию}
                \end{scnindent}
            \end{scnsubdividing}
        \end{scnsubstruct}

        \scnstructheader{Описание параметров и отношений, заданных на дискретных информационных конструкциях}
        \begin{scnsubstruct}
            \scnheader{параметр, заданный на множестве дискретных информационных конструкций\scnsupergroupsign}
            \scnhaselement{типология дискретных информационных конструкций, определяемая их носителем\scnsupergroupsign}
            \begin{scnindent}
                \scnhaselement{некомпьютерная форма представления дискретных информационных конструкций\scnsupergroupsign}
                \scnhaselement{файл}
                \begin{scnindent}
                    \scnidtf{компьютерная форма предcтавления дискретных информационных конструкций в линейной адресной памяти}
                \end{scnindent}
                \scnhaselement{структура}
                \begin{scnindent}
                    \scnidtf{компьютерная форма представления дискретных информационных конструкций в графодинамической ассоциативной памяти}
                    \scnidtf{представление дискретных информационных конструкций в виде конструкций SC-кода в памяти ostis-систем}
                \end{scnindent}
            \end{scnindent}
            \scnhaselement{типология дискретных информационных конструкций, определяемая их соотношением с памятью ostis-систем\scnsupergroupsign}
            \begin{scnindent}
                \scnhaselement{внешняя дискретная информационная конструкция ostis-системы}
                \begin{scnindent}
                    \scnidtf{дискретная информационная конструкция, которая находится вне памяти той ostis-системы, в которой находится sc-узел, обозначающий эту информационную конструкцию}
                    \begin{scnsubdividing}
                        \scnitem{некомпьютерная форма представления дискретных информационных конструкций}
                        \begin{scnindent}
                            \scntext{примечание}{Очевидно, что информационные конструкции такого вида принципиально не могут быть внутренними информационными конструкциями ostis-систем, хранимыми в их памяти.}
                        \end{scnindent}
                        \scnitem{внешний файл ostis-системы}
                        \begin{scnindent}
                            \begin{scnsubdividing}
                                \scnitem{файл компьютерной системы, которая не является ostis-системой}
                                \begin{scnindent}
                                    \scnidtf{файл, который не хранится в памяти данной ostis-системы, но о которой известно, какая система, не являющаяся ostis-системой, им \scnqq{владеет} и как его \scnqq{скачать}}
                                    \scnidtf{внешний файл ostis-системы, принадлежащий компьютерной системе, которая не является ostis-системой}
                                \end{scnindent}
                                \scnitem{файл другой ostis-системы}
                                \begin{scnindent}
                                    \scnidtf{файл, который не является внутренним файлом данной ostis-системы, в памяти которой находится знак этого файла, но является внутренним знаком другой ostis-системы}
                                    \scnidtf{внешний файл ostis-системы, принадлежащий другой ostis-системе}
                                \end{scnindent}
                            \end{scnsubdividing}
                        \end{scnindent}
                        \scnitem{внешняя структура ostis-системы}
                        \begin{scnindent}
                            \scnidtf{структура, хранимая в памяти другой ostis-системы}
                            \scnidtf{структура другой ostis-системы}
                        \end{scnindent}
                    \end{scnsubdividing}
                \end{scnindent}
            \end{scnindent}
            \scnhaselement{внутренняя информационная конструкция ostis-системы}
            \begin{scnindent}
                \scnidtf{внутренняя для заданной ostis-системы информационная конструкция}
                \scnidtf{внутренняя информационная конструкция той ostis-системы, в памяти (sc-памяти) которой хранится знак (sc-узел) этой информационной конструкции}
                \scntext{примечание}{Внутренние информационные конструкции ostis-систем (т.е. конструкции, обрабатываемые в их памяти) могут быть только дискретными, хотя и не обязательно знаковыми.}
                \begin{scnsubdividing}
                    \scnitem{внутренний файл ostis-системы}
                    \scnitem{внутренняя структура}
                    \begin{scnindent}
                        \scnidtf{структура, которой в памяти данной ostis-системы соответствует не только знак этой структуры, но и она сама}
                        \scnidtf{структура, хранимая и обрабатываемая в памяти данной ostis-системы}
                        \scnidtf{внутренняя структура ostis-системы}
                    \end{scnindent}
                \end{scnsubdividing}
                \begin{scnsubdividing}
                    \scnitem{сформированная внутренняя информационная конструкция ostis-системы}
                    \scnitem{частично сформированная внутренняя информационная конструкция ostis-системы}
                    \scnitem{внутренняя информационная конструкция ostis-системы на начальной стадии формирования}
                \end{scnsubdividing}
            \end{scnindent}
            \scnhaselement{типология дискретных информационных конструкций, определяемая правилами, которым они должны удовлетворять\scnsupergroupsign}
            \scnhaselement{информационная конструкция Русского языка}
            \scnhaselement{информационная конструкция Английского языка}
            \scnhaselement{структура}
            \begin{scnindent}
                \scnidtf{информационная конструкция SC-кода}
            \end{scnindent}
            \scnhaselement{sc.g-конструкция}
            \begin{scnindent}
                \scnidtf{информационная конструкция SCg-кода}
            \end{scnindent}
            \scnhaselement{sc.s-конструкция}
            \begin{scnindent}
                \scnidtf{информационная конструкция SCs-кода}
            \end{scnindent}
            \scnhaselement{sc.n-конструкция}
            \begin{scnindent}
                \scnidtf{информационная конструкция SCn-кода}
            \end{scnindent}
            \scnhaselement{наличие синтаксической связности\scnsupergroupsign}
            \begin{scnindent}
                \scnhaselement{синтаксически связная дискретная информационная конструкция}
                \begin{scnindent}
                    \scnidtftext{определение}{дискретная информационная конструкция, у которой для каждой пары её элементов существует маршрут, соединяющий эти элементы и проходящий по связям их инцидентности}
                \end{scnindent}
                \scnhaselement{синтаксически несвязная дискретная информационная конструкция}
                \scntext{примечание}{Можно оценивать \scnqq{силу} синтаксической связности --- наличие и число \scnqq{мостов} в графе инцидентности элементов дискретной информационной конструкции, наличие и число точек \scnqq{сочленения}, минимальное число элементов конструкции, удаление которых приводит к несвязности. Можно также оценивать уровень синтаксической несвязности дискретной информационной конструкции числом компонентов связности этой конструкции.}
            \end{scnindent}
            \scnhaselement{наличие семантической связности\scnsupergroupsign}
            \begin{scnindent}
                \scntext{примечание}{Свойством семантической связности могут обладать только знаковые конструкции.}
                \scnhaselement{семантически связная дискретная информационная конструкция}
                \begin{scnindent}
                    \scntext{определение}{Это конструкция, которая обладает следующим свойством: для любой ее декомпозиции на два синтаксически правильных компонента всегда найдется пара синонимичных знаков, один из которых находится в одном компоненте, а другой --- в другом.}
                \end{scnindent}
                \scnhaselement{семантически несвязная дискретная информационная конструкция}
            \end{scnindent}
            
            \scnheader{отношение, заданное на множестве дискретных информационных конструкций\scnsupergroupsign}
            \scnhaselement{дискретная информационная конструкция заданного языка*}
            \begin{scnindent}
                \scniselement{отношение, заданное на множестве языков\scnsupergroupsign}
                \begin{scnsubdividing}
                    \scnitem{синтаксически неправильная дискретная информационная конструкция заданного языка*}
                    \begin{scnindent}
                        \begin{scnreltoset}{объединение}
                            \scnitem{синтаксически некорректная дискретная информационная конструкция заданного языка*}
                            \scnitem{синтаксически нецелостная дискретная информационная конструкция заданного языка*}
                        \end{scnreltoset}
                    \end{scnindent}
                    \scnitem{\scnkeyword{текст заданного языка*}}
                    \begin{scnindent}
                        \begin{scnreltoset}{пересечение}
                            \scnitem{синтаксически корректная дискретная информационная конструкция заданного языка*}
                            \scnitem{синтаксически целостная дискретная информационная конструкция заданного языка*}
                        \end{scnreltoset}
                        \begin{scnsubdividing}
                            \scnitem{семантически неправильный текст заданного языка*}
                            \begin{scnindent}
                                \begin{scnreltoset}{объединение}
                                    \scnitem{семантически некорректный текст заданного языка*}
                                    \scnitem{семантически нецелостный текст заданного языка*}
                                \end{scnreltoset}
                            \end{scnindent}
                            \scnitem{знание, представленное в заданном языке*}
                            \begin{scnindent}
                                \begin{scnreltoset}{пересечение}
                                    \scnitem{семантически корректный текст заданного языка*}
                                    \scnitem{семантически целостный текст заданного языка*}
                                \end{scnreltoset}
                            \end{scnindent}
                        \end{scnsubdividing}
                    \end{scnindent}
                \end{scnsubdividing}
            \end{scnindent}

            \scnheader{обозначение файла}
            \begin{scnsubdividing}
                \scnitem{произвольный файл}
                \begin{scnindent}
                      	\scnidtf{sc-переменная, каждым значением которой является обозначение файла}
                    \scnidtf{обозначение произвольного файла}
                    \scnidtf{sc-переменнная, обозначающая файл}
                \end{scnindent}
                \scnitem{\scnkeyword{файл}}
                \begin{scnindent}
                      	\scnidtf{знак конкретного файла}
                    \scnidtf{sc-константа, обозначающая конкретный файл}
                \end{scnindent}
            \end{scnsubdividing}
            \begin{scnsubdividing}
                \scnitem{обозначение внешнего файла ostis-системы}
                \begin{scnindent}
                    \begin{scnsubdividing}
                        \scnitem{произвольный внешний файл ostis-системы}
                        \scnitem{внешний файл ostis-системы}
                    \end{scnsubdividing}
                \end{scnindent}
                \scnitem{обозначение внутреннего файла ostis-системы}
                \begin{scnindent}
                    \begin{scnsubdividing}
                        \scnitem{произвольный внутренний файл ostis-системы}
                        \scnitem{внутренний файл ostis-системы}
                    \end{scnsubdividing}
                \end{scnindent}
            \end{scnsubdividing}
            
            \scnheader{файл}
            \scnidtf{sc-узел, обозначающий файл}
            \scnidtf{знак файла}
            \begin{scnsubdividing}
                \scnitem{ея-файл}
                \begin{scnindent}
                      	\scnidtf{естественно-языковой файл}
                \end{scnindent}
                \scnitem{файл, являющийся текстом формального языка}
                \begin{scnindent}
                      	\scnsuperset{sc.g-файл}
                    \scnsuperset{sc.s-файл}
                    \scnsuperset{sc.n-файл}
                \end{scnindent}
                \scnitem{файл, отражающий процесс изменения sc.g-текста}
                \scnitem{графический файл}
                \scnitem{файл-изображение}
                \scnitem{видео-файл}
                \scnitem{аудио-файл}
            \end{scnsubdividing}
            \begin{scnsubdividing}
                \scnitem{файл-экземпляр}
                \begin{scnindent}
                      	\scnidtf{файл, являющийся конкретным электронным документом или электронным образом конкретной внешней информационной конструкции}
                \end{scnindent}
                \scnitem{файл-образец}
                \begin{scnindent}
                      	\scnidtf{файл-класс ostis-системы}
                    \scnidtf{файл, являющийся одновременно также и знаком множества всевозможных экземпляров (копий) этого файла}
                \end{scnindent}
            \end{scnsubdividing}
            \begin{scnsubdividing}
                \scnitem{внешний файл ostis-системы}
                \scnitem{\scnkeyword{внутренний файл ostis-системы}}
            \end{scnsubdividing}
            
            \scnheader{внутренний файл ostis-системы}
            \scniselement{синтаксически выделяемый класс sc-элементов в рамках SC-кода}
            \scniselement{семантически выделяемый класс sc-элементов в рамках SC-кода}
            \scntext{примечание}{Данный класс sc-элементов, являющихся знаками файлов, хранимых в памяти ostis-систем, в отличие от других синтаксически выделяемых классов sc-элементов, представляет собой одновременно  синтаксически и семантически выделяемый класс sc-элементов. Это обусловлено (1) тем, что каждый экземпляр данного класса sc-элементов является знаком конкретного файла, хранимого в памяти ostis-системы, и (2) тем, что каждый файл, хранимый в памяти ostis-системы, может и должен быть обозначен только таким sc-элементом, который является экземпляром рассматриваемого класса sc-элементов.}
            \scntext{примечание}{sc-узел может быть знаком файла, находящегося в памяти другой ostis-системы (не в той, в которой хранится этот sc-узел). Но в этом случае указанный sc-узел не будет принадлежать рассматриваемому классу sc-узлов.}
            \scnidtf{знак файла ostis-системы, хранимого в \scnqq{моей} памяти}
            \scntext{примечание}{Следует отличать синтаксическую эквивалентность файлов, семантическую эквивалентность файлов и совпадение файлов (когда речь идет об одном и том же файле). Т.е. копия файла и один и тот же файл --- это разные вещи.}
        \end{scnsubstruct}


        \scnstructheader{Классификация структур}
        \begin{scnsubstruct}
            \scnheader{структура}
            \scnidtf{структура, элементами которой являются sc-элементы}
            \scnidtf{множество sc-элементов (множество), не являющиеся ни связкой (связкой sc-элементов), ни классом (множеством всех sc-элементов, эквивалентных в определенном смысле)}
            \scnidtf{знак конкретной (константной) структуры}
            \scnidtf{Класс всех тех и только тех sc-элементов, каждый из которых является знаком конкретной структуры}
            \scnidtf{Знак класса всех sc-элементов, являющихся знаками конкретных структур}
            \scnidtf{Константный sc-элемент (точнее, sc-узел), являющийся знаком конкретного класса всех sc-элементов, являющихся знаками конкретных структур}
            \scnidtf{sc-конструкция}
            \scnidtf{информационная конструкция, принадлежащая SC-коду}
            \scnsuperset{\scnkeyword{sc-текст}}
            \begin{scnindent}
                \scnidtf{структура, удовлетворяющая синтаксическим правилам SC-кода}
                \scnsuperset{\scnkeyword{знание}}
                \begin{scnindent}
                    \scnidtf{семантически корректный и целостный sc-текст}
                \end{scnindent}
            \end{scnindent}        
            \scnrelfrom{разбиение}{\scnkeyword{Наличие sc-переменных, входящих в состав структуры}}
            \begin{scnindent}
                \begin{scneqtoset}
                    \scnitem{структура, в составе которой sc-переменные не входят}
                    \scnitem{структура, в состав которой входят sc-переменные}
                    \begin{scnindent}
                        \scntext{примечание}{Такие структуры при представлении логических высказываний в SC-коде являются аналогами атомарных логических формул.}
                    \end{scnindent}
                \end{scneqtoset}
            \end{scnindent}
            \scnrelfrom{разбиение}{\scnkeyword{Темпоральная характеристика структур}}
            \begin{scnindent}
                \begin{scneqtoset}
                    \scnitem{ситуативная структура}
                    \begin{scnindent}
                        \scnidtf{ситуация, представленная в SC-коде}
                        \scnidtf{ситуация}
                        \scnidtf{структура, в состав которой входят знаки временных сущностей и которая сама является временной сущностью (при этом время существования такой структуры совпадает с временем одновременного существования всех временных сущностей, знаки которых входят в состав этой ситуативной структуры).}
                        \begin{scnsubdividing}
                            \scnitem{ситуация во внешней среде}
                            \scnitem{ситуация в sc-памяти}
                        \end{scnsubdividing}
                    \end{scnindent}
                    \scnitem{структура, не содержащая знаков временных сущностей}
                    \scnitem{динамическая структура}
                    \begin{scnindent}
                        \scntext{пояснение}{В отличие от ситуативной структуры конфигурация динамической структуры меняется во времени в зависимости от момента появления и момента завершения существования каждой временной сущности (в том числе временной связи), знак которой входит в состав динамической структуры. Каждой динамической структуре можно поставить в соответствие темпоральную последовательность состояний (ситуаций) и событий.}
                    \end{scnindent}
                \end{scneqtoset}
            \end{scnindent}
            \scnrelfrom{разбиение}{\scnkeyword{Наличие связности структур}}
            \begin{scnindent}
                \begin{scneqtoset}
                    \scnitem{связная структура}
                    \begin{scnindent}
                        \scnrelfrom{разбиение}{Связность структур\scnsupergroupsign}
                        \begin{scnindent}
                            \scnidtf{Минимальное число sc-элементов, удаление которых преобразует связную структуру в несвязную}
                            \scniselement{одно-связная структура}
                            \begin{scnindent}
                                \scnrelto{разбиение}{Признак классификации структур по числу точек сочленения\scnsupergroupsign}
                                \scnrelto{разбиение}{Признак классификации структур по числу мостов\scnsupergroupsign}
                            \end{scnindent}
                        \end{scnindent}
                    \end{scnindent}
                    \scnitem{несвязная структура}
                    \begin{scnindent}
                        \scnrelto{разбиение}{Признак классификации структур по числу компонентов связности\scnsupergroupsign}
                        \scnsubset{тривиальная структура}
                        \begin{scnindent}
                            \scnidtf{структура, в состав элементов которой sc-коннекторы не входят}
                        \end{scnindent}
                    \end{scnindent}
                \end{scneqtoset}
                \scntext{примечание}{Важнейшей особенностью SC-кода является то, что для конструкций SC-кода (для структур) нет необходимости противопоставлять синтаксическую и семантическую связность, то есть все синтаксически связные структуры являются также и семантически связными и наоборот.}
            \end{scnindent}
            \scnrelfrom{разбиение}{\scnkeyword{Рефлексивность структур}}
            \begin{scnindent}
                \begin{scneqtoset}
                    \scnitem{рефлексивная структура}
                    \begin{scnindent}
                        \scnidtf{структура, в число элементов которой входит sc-узел, обозначающий саму эту структуру}
                        \scnsubset{рефлексивное множество}
                    \end{scnindent}
                    \scnitem{нерефлексивная структура}
                \end{scneqtoset}
            \end{scnindent}
            \scnrelfrom{разбиение}{\scnkeyword{Целостность структур по связкам}}
            \begin{scnindent}
                \begin{scneqtoset}
                    \scnitem{структура, содержащая все компоненты всех своих связок}
                    \scnitem{структура, не содержащая все компоненты всех своих связок}
                \end{scneqtoset}
            \end{scnindent}
            \begin{scnsubdividing}
                \scnitem{синтаксически неправильная структура}
                \begin{scnindent}
                      	\scnidtf{синтаксически неправильно построенная структура}
                    \begin{scnreltoset}{объединение}
                        \scnitem{синтаксически некорректная структура}
                        \begin{scnindent}
                            \scnidtf{структура, содержащая фрагменты, противоречащие \textit{Синтаксическим правилам SC-кода} (ошибочные фрагменты)}
                        \end{scnindent}
                        \scnitem{синтаксически нецелостная структура}
                        \begin{scnindent}
                            \scnidtf{структура, в которой имеется синтаксически выявленная недостаточность, неполнота (то есть имеется некоторое количество информационных дыр)}
                        \end{scnindent}
                    \end{scnreltoset}
                    \scntext{примечание}{Разделение \textit{Синтаксических правил SC-кода} на правила анализа синтаксической корректности и правила анализа синтаксической целостности (полноты) существенно упрощает процедуру синтаксического анализа \textit{структур}.}
                \end{scnindent}
                \scnitem{\scnkeyword{sc-текст}}
                \begin{scnindent}
                    \scnidtf{синтаксически правильная структура}
                    \scnidtf{синтаксически правильно построенная структура}
                    \begin{scnreltoset}{пересечение}
                        \scnitem{синтаксически корректная структура}
                        \scnitem{синтаксически целостная структура}
                    \end{scnreltoset}
                    \begin{scnsubdividing}
                        \scnitem{семантически неправильный sc-текст}
                        \begin{scnindent}
                            \begin{scnreltoset}{объединение}
                                \scnitem{семантически некорректный sc-текст}
                                \scnitem{семантически нецелостный sc-текст}
                            \end{scnreltoset}
                        \end{scnindent}
                        \scnitem{\scnkeyword{знание}}
                        \begin{scnindent}
                            \scnidtf{семантически правильно построенный sc-текст}
                            \begin{scnreltoset}{пересечение}
                                \scnitem{семантически корректный sc-текст}
                                \scnitem{семантически целостный sc-текст}
                            \end{scnreltoset}
                        \end{scnindent}
                    \end{scnsubdividing}
                \end{scnindent}
            \end{scnsubdividing}

            \scnheader{знание}
            \scnidtf{дискретная информационная конструкция, являющаяся знанием, представленная в некотором (дополнительно уточняемом) языке}
            \scnrelto{второй домен}{знание, представленное в заданном языке*}
            \scnsubset{знаковая конструкция}
            \scntext{примечание}{Каждое знание является знаковой конструкцией, но не каждая знаковая конструкция является знанием, а только та, смысловое представление которой удовлетворяет определенным требованиям корректности и целостности.}
            \scniselement{семантически выделяемый класс дискретных информационных конструкций\scnsupergroupsign}
            
            \scnheader{следует отличать*}
            \begin{scnhaselementset}
                \scnitem{\scnnonamednode}
                \begin{scnindent}
                    \begin{scneqtoset}
                        \scnitem{дискретная информационная конструкция}
                        \scnitem{текст}
                        \scnitem{знание}
                    \end{scneqtoset}
                    \scniselement{следует отличать*}
                \end{scnindent}
                \scnitem{\scnnonamednode}
                \begin{scnindent}
                    \begin{scneqtoset}
                        \scnitem{структура}
                        \scnitem{sc-текст}
                        \scnitem{знание}
                    \end{scneqtoset}
                    \scniselement{следует отличать*}
                \end{scnindent}
            \end{scnhaselementset}
        \end{scnsubstruct}

        \scnstructheader{Синтаксические расширения Ядра SC-кода}
        \begin{scnsubstruct}

            \scntext{примечание}{Способ кодирования \textit{sc-конструкций} в различных вариантах реализации \textit{sc-памяти} может быть различным. То есть каждому варианту реализации \textit{sc-памяти} может соответствовать своя \textit{синтаксическая модификация} \textit{SC-кода}. При этом она может касаться не только способа представления меток \textit{sc-элементов}, но и представления (кодирования) \textit{отношений инцидентности sc-элементов}. В любом случае каждая такая модификация должна быть четко описана.}

            \scntext{примечание}{Расширение семейства синтаксически выделяемых классов sc-элементов целесообразно:
            \begin{scnitemize}
                \item Для того, чтобы ускорить определение семантического типа каждого \textit{sc-элемента} в ходе обработки \textit{sc-конструкций}
                \item Чтобы быстро уточнить содержание \textit{базовой спецификации sc-элемента} (что необходимо о нем знать, чтобы с ним эффективно работать). В частности, необходимо знать то, что о нем не известно.
            \end{scnitemize}}
            
            \scntext{примечание}{Число меток, приписываемых sc-элементу должно быть \textit{минимизировано}, то есть эти метки должны быть \textit{информативными}.}
            
            \scntext{примечание}{Поскольку \textit{SC-код} является языком \textit{внутреннего} представления информации в \textit{sc-памяти} ostis-системы, \textit{Синтаксис SC-кода} является уточнением \textit{формы} такого представления и, как следствие уточнением того, как устроена \textbf{\textit{sc-память ostis-систем}}. Поскольку хранимая в \textit{sc-памяти} информационная конструкция представляет собой множество \textit{sc-элементов}, можно ввести понятие \textbf{\textit{ячейки sc-памяти}}, каждая из которых обеспечивает хранение одного из \textit{sc-элементов}.}
            
            \scnheader{ячейка sc-памяти}
            \scntext{пояснение}{фрагмент \textit{sc-памяти}, в котором может храниться один \textit{sc-элемент} (точнее, основная информация об этом sc-элементе) и который должен содержать:
                \begin{scnitemize}
                    \item набор синтаксических меток, приписываемых хранимому \textit{sc-элементу};
                    \item уникальный (взаимнооднозначный) идентификатор хранимого \textit{sc-элемента} (аналог адреса ячейки в адресной памяти);
                    \item связи хранимого \textit{sc-элемента} со смежными \textit{sc-элементами} (пары инцидентности);
                    \item ссылка на хранимый файл, если хранимый \textit{sc-элемент} является \textit{знаком файла}, хранимого в файловой памяти этой же индивидуальной \textit{ostis-системы}.
                \end{scnitemize}}
           
            \scntext{примечание}{Формы представления меток \textit{sc-элементов} могут быть разными:
                \begin{scnitemize}
                    \item приписывание \textit{sc-идентификатора} того \textit{sc-класса} которому принадлежит данный \textit{sc-элемент};
                    \item формирование вектора признаков в некотором пространстве признаков (Каждому признаку ставится в соответствии свое \uline{поле}, в которое записывается соответствующее значение признака --- в качестве этого значения тоже можно использовать sc-идентификатор соответствующего значения этого признака).
                \end{scnitemize}}
        \end{scnsubstruct}
\end{scnsubstruct}
\scnsourcecommentinline{Завершили Сегмент \scnqqi{SC-код как синтаксическое расширение Ядра SC-кода}}
