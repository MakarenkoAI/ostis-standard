
\scnsegmentheader{Структура базовой семантической спецификации sc-элемента}
\begin{scnsubstruct}
	\scnheader{базовая семантическая спецификация sc-элемента}
	\scnidtfexp{Класс \textit{sc-структур}, каждая из которых описывает базовые семантические свойства (характеристики) соответствующего (описываемого, специфицируемого) \textit{sc-элемента}}
	\scnsubset{sc-структура}
	\begin{scnindent}
		\scnsubset{sc-спецификация}
		\begin{scnindent}
			\scnidtf{представленная в \textit{SC-коде} семантическая окрестность (спецификация) некоторого (специфицируемого) \textit{sc-элемента}}
		\end{scnindent}
	\end{scnindent}
	\scnsubset{sc-спецификация}
	\scnrelto{второй домен}{базовая семантическая спецификация sc-элемента*}
	\begin{scnindent}
		\scnidtfexp{бинарное ориентированное отношение, каждая пара которого связывает \textit{sc-элемент} с его базовой семантической спецификацией*}
	\end{scnindent}
	\scnidtfexp{хранимая в \textit{sc-памяти} ostis-системы спецификация каждого \textit{sc-элемента}, необходимая для эффективной обработки этого \textit{sc-элемента}}
	\scnnote{базовая спецификация \textit{sc-элементов} осуществляется как явно с помощью соответствующих sc-конструкций, так и неявно с помощью соответствующих семантических меток, приписываемых sc-элементам}
	\scntext{пояснение}{Базовая семантическая спецификация каждого \textit{sc-элемента} включает в себя:
		\begin{scnitemize}
			\item перечисление всех тех \textit{базовых классов sc-элементов}, которым принадлежит специфицируемый \textit{sc-элемент};
			\item уточнение \scnqq{привязки} временной сущности, обозначаемой специфицируемым sc-элементом к текущему моменту и другим моментам времени;
			\item уточнение того, какие важные характеристики специфицируемого \textit{sc-элемента} в текущем состоянии \textit{sc-памяти} и файловой памяти \textit{ostis-системы} не известны.
		\end{scnitemize}
	}
    \scntext{примечание}{\textbf{\textit{базовая семантическая спецификация sc-элемента, обозначающего временную сущность}} включает в себя указание дополнительных темпоральных характеристик, позволяющих уточнить темпоральные \scnqq{координаты} этих временных сущностей (то есть их \scnqq{координаты} во времени), а также их основные темпоральные связи с другими временными сущностями.}
    
	\scnheader{базовая семантическая спецификация sc-элемента, обозначающего временную сущность}
	\begin{scnrelfromset}{включение}
		\scnitem{момент времени\scnsupergroupsign}
		\scnitem{Текущий момент времени}
		\scnitem{прошлая сущность}
		\scnitem{будущая сущность} 
		\scnitem{момент начала*}
		\scnitem{момент завершения*}
		\scnitem{внешняя ситуация}
		\scnitem{ситуация в sc-памяти}
		\scnitem{внешнее событие}
		\scnitem{событие в sc-памяти}
		\scnitem{внешний процесс}
		\scnitem{процесс в sc-памяти}
    \end{scnrelfromset}

	\scnheader{момент времени\scnsupergroupsign}
	\scniselement{параметр}
	\scniselement{параметр, заданный на множестве временных сущностей}
	\scnidtf{глобальная приблизительно точная ситуация\scnsupergroupsign}
	\scnidtf{глобальная ситуация пренебрежительно малого отрезка времени\scnsupergroupsign}
	\scnidtf{множество (класс) \textit{всех} временных сущностей, существующих одновременно в соответствующий момент времени\scnsupergroupsign}
	\scnnote{момент времени, соответствующий глобальной точечной ситуации может быть задан с различной и \textit{дополни-} \textit{тельно указываемой} степенью точности --- с точностью до секунды, до минуты, до часа, до даты, до календарного месяца, до календарного года и так далее. В том смысле корректнее говорить не о моменте времени, а об интервале времени, длительность которого считается пренебрежимо малой для рассмотрения описываемых процессов}

	\scnheader{Текущий момент времени}
	\scnidtf{Глобальная ситуация текущего (настоящего) момента времени}
	\scnidtf{Глобальная ситуация, имеющая место сейчас}
	\scnidtf{Класс всех сущностей, существующих в настоящий момент времени}
	\scniselement{sc-синглетон}
	\scniselement{динамическое sc-множество}
	\scnrelto{включение множества}{момент времени}
	\scnexplanation{Из знака \textit{Текущего момента времени} (который является также знаком \textit{sc-синглетона}) \scnqq{выходит} sc-пара \textit{временной} принадлежности, представляющая собой, образно говоря, \scnqq{стрелку} внутренних часов \textit{ostis-системы}, которая всегда указывает только на один элемент множества моментов времени, но в разные моменты времени указывает на разные элементы этого множества}

	\scnheader{прошлая сущность}
	\scnidtf{временная сущность, уже завершившая свое существование}

	\scnheader{будущая сущность}
	\scnidtf{прогнозируемая, планируемая или создаваемая временная сущность}

	\scnheader{момент начала*}
	\scnidtf{момент времени, соответствующий началу существования заданной временной сущности}
	\scnidtf{бинарное ориентированное отношение, каждая пара которого, связывает (1) знак некоторой временной сущности и (2) глобальную точечную ситуацию (значение параметра \scnqq{\textit{момент времени}\scnsupergroupsign}), элементом которой является условно точечная временная сущность, представляющая собой начальный этап существования временной сущности, указанной в первом компоненте рассматриваемой ориентированной пары}
	\scnnote{Начальный этап существования временной сущности (переходный процесс от небытия к реальному существованию) может рассматриваться с любой степенью детализации}
	\scnrelfrom{первый домен}{временная сущность}
	\scnrelfrom{второй домен}{момент времени\scnsupergroupsign}

	\scnheader{момент завершения*}
	\scnidtf{момент времени, соответствующий завершению существования заданной временной сущности}

	\scnheader{ситуация}
	\begin{scnsubdividing}
		\scnitem{внешняя ситуация}
		\scnitem{ситуация в sc-памяти}
	\end{scnsubdividing}

	\scnheader{событие}
	\begin{scnsubdividing}
		\scnitem{внешнее событие}
		\scnitem{событие в sc-памяти}
	\end{scnsubdividing}

	\scnheader{динамическое sc-множество}
	\begin{scnsubdividing}
		\scnitem{внешний процесс}
		\begin{scnindent}
			\scnidtf{процесс, происходящий в окружающей среде ostis-системы}
		\end{scnindent}
		\scnitem{процесс в sc-памяти}
	\end{scnsubdividing}

	\scnheader{внешняя ситуация}
	\scnidtf{ситуация во внешней среде}
	\scnidtf{ситуация \textit{одновременного} существования (в соответствующий период времени) указанных временных внешних сущностей}
	\scnsubset{временная сущность}
	\scnsubset{sc-структура}
	\scnsubset{sc-константа}
	\scnsubset{обозначение внешней ситуации}
	\scniselement{sc-класс}

	\scnheader{класс внешних ситуаций}
	\scnnote{В простейшем случае внешние ситуации, входящие в класс внешних ситуаций являются изоморфными}

	\scnheader{внешний процесс}
	\scnidtf{темпоральная детализация внешней динамической сущности}

	\scnheader{внешнее событие}
	\scnidtf{факт появления (возникновения) некоторой внешней сущности (в том числе некоторой внешней ситуации) или факт завершения существования некоторой внешней сущности (в том числе некоторой внешней ситуации)}

	\scnheader{ситуация в sc-памяти}
	\scnidtf{внутренняя ситуация}
	\begin{scnindent}
		\scnidtf{sc-ситуация}
		\scnidtf{хранимый в sc-памяти фрагмент базы знаний, рассматриваемый в контексте его появления в sc-памяти или его исчезновения (из-за удаления некоторые sc-элементов)}
	\end{scnindent}

	\scnheader{класс ситуаций в sc-памяти}
	\scnidtf{класс внутренних ситуаций}

	\scnheader{обобщенное описание класса ситуаций в sc-памяти}

	\scnheader{процесс в sc-памяти}
	\scnidtf{внутренний процесс}
	\scnidtf{информационный процесс, происходящий в sc-памяти}
	\scnidtf{sc-процесс}	

	\scnheader{событие в sc-памяти}

    \scnheader{базовая семантическай спецификация sc-элемента}
    \scntext{примечание}{Важной частью \textbf{\textit{базовой семантической спецификации sc-элемента}} является фиксация того, что \textit{ostis-система} \textit{знает и чего она не знает} о специфицируемом \textit{sc-элементе} или об обозначенной им сущности.}
        \begin{scnindent}
            \begin{scnsubdividing}
                \scnfileitem{Если в спецификации \textit{sc-элемента} указывается его принадлежность к некоторому классу \textit{sc-элементов}, но не указывается его принадлежность \textit{одному} из подклассов, на которые \textit{разбивается} указанный выше класс, то это означает, что в текущий момент времени \textit{ostis-система} этого \textit{не знает}.}
                \scnfileitem{Если специфицируемый \textit{sc-элемент} является обозначением \textit{конечного} множества sc-элементов (в частности, пары \textit{sc-элементов}), и если в текущий момент времени \textit{ostis-системе} не известны \textit{все} этого множества (то есть специфицируемый \textit{sc-элемент} не соединен соответствующими парами принадлежности со \textit{всеми} элементами обозначаемого им множества \textit{sc-элементов}), то этот специфицируемый \textit{sc-элемент} следует отнести к \textit{sc-классу} \scnqqi{\textbf{\textit{обозначение несформированного sc-множества}}}.}
                \scnfileitem{Если специфицируемый \textit{sc-элемент} является обозначением ориентированной \textit{sc-пары} и если в текущий момент времени \textit{ostis-системе} не известна \textit{направленность} этой ориентированной пары \textit{sc-элементов} (то есть не известно, какой элемент этой пары является первым ее компонентом, а какой ее элемент является ее вторым компонентом), то этот специфицируемый \textit{sc-элемент} следует отнести к \textit{sc-классу} \scnqqi{\textbf{\textit{обозначение ориентированной sc-пары неизвестной направленности}}}.}
            \end{scnsubdividing}
        \end{scnindent}
    \scntext{примечание}{Понятия, используемые для описания \textit{полноты} \textbf{\textit{базовой семантической спецификации sc-элемента}}}
        \begin{scnindent}
        	\begin{scnrelfromlist}{включение}
            \scnitem{\textit{обозначение бесконечного sc-множества}}
            \scnitem{\textit{обозначение конечного sc-множества}}
            \scnitem{\textit{мощность обозначаемого sc-множества*}}
            \scnitem{\textit{обозначение sc-множества неизвестной мощности}}
            \scnitem{\textit{обозначение sc-множества, о котором не известно, является ли оно sc-парой}}
            \scnitem{\textit{обозначение sc-пары, о которой не известно, является ли она ориентированной или нет}}
            \scnitem{\textit{обозначение ориентированной sc-пары неизвестной направленности}}
            \scnitem{\textit{обозначение сформированного sc-множества}}
            \scnitem{\textit{обозначение несформированного sc-множества}}
            \scnitem{\textit{обозначение \textit{частично} сформированного sc-множества}}
            \scnitem{\textit{обозначение \textit{полностью} несформированного sc-множества}}
            \scnitem{\textit{обозначение сформированного файла}}
            \scnitem{\textit{обозначение несформированного файла}}
            \scnitem{\textit{обозначение частично сформированного файла}}
            \scnitem{\textit{обозначение полностью несформированного файла}}
            \end{scnrelfromlist}
         \end{scnindent}
    \scntext{примечание}{Подчеркнем то, что базовую семантическую спецификацию должны иметь абсолютно все \textit{sc-элементы}, хранимые в \textit{sc-памяти} в текущий момент времени, в том числе и все \textit{sc-элементы}, являющиеся ключевыми знаками в рамках \textbf{\textit{Предметной области Базовой денотационной семантики SC-кода}}.}
    \begin{scnindent}
        \scnrelfrom{пример}{обозначение sc-множества}
    \end{scnindent}
    
	\scnheader{обозначение sc-множества}
	\scnidtf{Множество всевозможных sc-элементов, обозначающих sc-множества}
	\begin{scnindent}
		\scniselement{имя собственное}
	\end{scnindent}
	\scniselement{обозначение sc-множества}
	\scnnote{Одним из элементов данного множества является знак, обозначающий это множество. Это означает, это данное множество является \textit{рефлексивным множеством}}
	\scniselement{обозначение множества sc-элементов разного структурного типа}
	\scntext{примечание}{Элементами данного множества являются обозначения различных:
		\begin{scnitemize}
			\item sc-синглетонов;
			\item sc-пар;
			\item sc-связок, не являющихся ни sc-синглетонами, ни sc-парами;
			\item sc-классов;
			\item sc-структур.
		\end{scnitemize}
	}
	\scniselement{обозначение множества sc-элементов, содержащего как константные, так и переменные sc-элементы}
	\scniselement{sc-константа}
	\scnnote{Само данное множество является константным, несмотря на то, что его элементами являются как sc-константы, так и sc-переменные}
	\scniselement{обозначение множества sc-элементов, содержащего sc-элементы, обозначающие как постоянные, так и временные сущности.}
	\scniselement{постоянная сущность}
	\scnnote{Следует отличать постоянство~/~временность сущности, обозначаемой sc-элементом и постоянство~/~временность sc-множества, одним из элементов которого указанный sc-элемент является.}
	\scniselement{обозначение множества sc-элементов, содержащего sc-элементы, обозначающие как статические, так и динамические sc-множества}
	\scniselement{статическое sc-множество}
	\begin{scnrelfromlist}{примечание}
\scnfileitem{Следует отличать статичность~/динамичность sc-множества, обозначаемого соответствующим sc-элементом и статичность~/динамичность sc-множества, одним из элементов которого указанный выше sc-элемент является.}
		\scnfileitem{Напомним, что статический характер sc-множества означает отсутствие временных sc-пар принадлежности (временных sc-дуг принадлежности), выходящих из знака этого sc-множества.}
	\end{scnrelfromlist}
	\scniselement{sc-класс}
	\scnidtf{Класс всевозможных sc-элементов, обозначающих sc-множества}
	\scnidtf{Класс обозначающий sc-множеств}
	\scnnote{Следует отличать разные sc-элементы, являющиеся обозначениями соответствующих sc-множеств, и класс, элементами которого являются \textit{всевозможные} такие sc-элементы.}
\end{scnsubstruct}
