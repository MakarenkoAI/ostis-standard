\scnsegmentheader{Онтологическая формализация Базовой денотационной семантики SC-кода}
\begin{scnsubstruct}

	\scntext{пояснение}{Суть онтологической формализации различных областей знаний, различных фрагментов \textit{баз знаний} интеллектуальных компьютерных систем заключается в следующем.}
	\begin{scnhaselementset}
			\scnfileitem{Достаточно большой \textit{семантически целостный} фрагмент \textit{баз знаний}.}
			\begin{scnindent}
				\begin{scnrelfromlist}{включение}
					\scnfileitem{Все элементы некоторого одного ключевого класса рассматриваемых объектов (объектов исследования) или \textit{конечного} числа таких ключевых классов объектов исследования.}
					\scnfileitem{\textit{Все связи} между выделенными объектами исследования, соответствующие заданному \textit{семейству} отношений, параметров и классов структур, которое условно будем называть предметом исследования.}
				\end{scnrelfromlist}
			\end{scnindent}
			\scnfileitem{Указанный семантически целостный фрагмент 	\textit{базы знаний}, являющийся чаще всего \textit{бесконечной} структурой, будем называть \textbf{\textit{предметной областью}}.}
			\scnfileitem{Сама формальная \textbf{\textit{онтология}} представляет собой формальную спецификацию выделенной \textit{предметной области} и включает в себя следующие \textbf{\textit{частные онтологии}}.}
			\begin{scnindent}
				\begin{scnsubdividing}
					\scnitem{структурная спецификация предметной области}
					\begin{scnindent}
						\scntext{примечание}{спецификация предметной области, в которой указываются роли всех ключевых элементов (ключевых знаков), входящих в состав \textit{предметной области}}
						\begin{scnrelfromlist}{включение}
							\scnitem{максимальный класс объектов исследования\scnrolesign}
							\scnitem{немаксимальный класс объектов исследования\scnrolesign}
							\scnitem{ключевой объект исследования\scnrolesign}
							\scnitem{исследуемый класс связок\scnrolesign}
							\scnitem{исследуемый класс классов\scnrolesign}
							\scnitem{исследуемый класс структур\scnrolesign}
							\scnitem{неисследуемый класс\scnrolesign}
							\begin{scnindent}
								\scnidtf{\textit{sc-класс}, исследуемый в другой (смежной) \textit{предметной области}}
							\end{scnindent}
						\end{scnrelfromlist}
					\end{scnindent}
					\scnitem{теоретико-множественная онтология}
					\begin{scnindent}
						\scntext{примечание}{спецификация предметной области, в которой описываются теоретико-множественные связи между всеми классами (\textit{sc-классами}), исследуемыми в рамках заданной (специфицируемой) \textit{предметной области}}
					\end{scnindent}
					\scnitem{логическая онтология}
					\begin{scnindent}
						\begin{scnrelfromlist}{включение}
							\scnfileitem{определения исследуемых классов (исследуемых понятий)}
							\scnfileitem{логическая иерархию исследуемых понятий, которая связывает каждое понятие со множеством тех понятий, которые явно используются в определении этого понятия}
							\scnfileitem{аксиомы и теоремы, описывающие свойства специфицируемой предметной области}
							\scnfileitem{тексты доказательств теорем}
							\scnfileitem{логическая иерархия теорем, которая связывает каждую теорему со множеством теорем, на основе которых она доказывается}
						\end{scnrelfromlist}	
					\end{scnindent}	
					\scnitem{терминологическая спецификация предметной области}
					\begin{scnindent}
						\scntext{примечание}{спецификация предметной области, в которой указывается \textit{sc-идентификаторы} всех ключевых \textit{sc-элементов} специфицируемой \textit{предметной области}, а также приводятся правила построения \textit{основных sc-идентификаторов} для элементов всех \textit{sc-классов} (понятий), исследуемых в рамках специфицируемой \textit{предметной области}}
					\end{scnindent}
					\scnitem{дидактическая спецификация предметной области}
					\begin{scnindent}
						\scntext{примечание}{спецификация предметной области, в которой приводится дополнительная информация, предназначенная для того, чтобы пользователи и разработчики (инженеры знаний), которые используют или совершенствуют специфицируемую \textit{предметную область} и ее \textit{онтологию}, могли быстрее усвоить их особенности.}
						\scnrelfrom{смотрите}{Предметная область и онтология предметных областей}
					\end{scnindent}
					\scnfileitem{проектная спецификация предметной области и соответствующей ей онтологии}
					\begin{scnindent}
						\scntext{примечание}{спецификация предметной области, в которой приводится информация об истории эволюции этой \textit{предметной области и онтологии}, а также о направлениях и планах организации дальнейшего их развития.}
					\end{scnindent}
				\end{scnsubdividing}
			\end{scnindent}	
	\end{scnhaselementset}
	\begin{scnrelfromset}{смотрите}
		\scnitem{Предметная область и онтология онтологий}
		\scnitem{Предметная область и онтология предметных областей}
	\end{scnrelfromset}
	\scntext{примечание}{Онтологическая формализация \textit{базовой денотационной семантики SC-кода} трактуется нами как \textit{формальная онтология}, представленная в \textit{SC-коде} и описывающая детонационную семантику \textit{семантически корректных sc-конструкций}. Указанную \textit{формальную онтологию} будем называть \textbf{\textit{Базовой денотационной семантикой SC-кода}}. Для того, чтобы уточнить \textit{предметную область}, специфицируемую этой \textit{онтологией}, введем следующие понятия:
		\begin{scnitemize}
			\item синонимия sc-элементов,
			\item отношение эквивалентности,
			\item sc-память,
			\item база знаний ostis-системы,
			\item ostis-система,
			\item \textup{[}sc-конструкция\textup{]},
			\item sc-знание,
			\item интеграция sc-конструкций*,
			\item sc-пространство.
		\end{scnitemize}}

	\scnheader{синонимия sc-элементов}
	\scnidtf{бинарное ориентированное \textit{отношение эквивалентности}, каждая пара которого связывает два \textit{sc-элемента}, обозначающие одну и ту же сущность*}
	\scnnote{Синонимия двух \textit{sc-элементов} возможна только в том случае, если эти \textit{sc-элементы} хранятся в \textit{sc-памяти} (входят в состав \textit{баз знаний}) \textit{разных} \textit{ostis-систем}. В рамках каждой \textit{ostis-системы} синонимичные \textit{sc-элементы} совпадают (отождействляются, склеиваются, считаются одним и тем же \textit{sc-элементом}).}

	\scnheader{отношение эквивалентности}
	\scnrelto{ключевое понятие}{\textsection~2.4.2. Формальная онтология связок и отношений}

	\scnheader{sc-память}

	\scnheader{база знаний ostis-системы}

	\scnheader{ostis-система}
	\begin{scnsubdividing}
		\scnitem{индивидуальная ostis-система}
		\scnitem{коллективная ostis-система}
	\end{scnsubdividing}

	\scnheader{\textup{[}sc-конструкция\textup{]}}
	\scnrelto{часто используемый sc-идентификатор}{\textbf{sc-множество}}
	\begin{scnindent}
	\scnidtf{информационная конструкция, представляющая собой множество \textit{sc-элементов}}
	\scnsuperset{sc-текст}
	\begin{scnindent}
		\scnidtf{текст SC-кода}
		\scnidtf{\textit{sc-конструкция}, являющаяся семантически корректной по отношению к \textit{Базовой денотационной семантике SC-кода}}
		\scnidtf{\textit{sc-конструкция}, удовлетворяющая (соответствующая) правилам \textit{Базовой денотационной семантики SC-кода}}
		\scnidtftext{часто используемый sc-идентификатор}{\textit{SC-код}}
		\begin{scnindent}
			\scniselement{имя собственное}
			\scnidtf{Класс (Множество всевозможных) sc-текстов}
		\end{scnindent}
		\scnsuperset{sc-знание}
	\end{scnindent} 
	\end{scnindent}

	\scnheader{sc-знание}
	\scnidtf{\textit{sc-текст}, являющийся либо фрагментом (подструктурой) соответствующей \textit{предметной области}, либо \textit{высказыванием}, описывающим некоторое свойство (в частности, некоторую закономерность) этой \textit{предметной области}}
	\scnidtf{знание, представленное в \textit{SC-коде}}
	\scnidtf{\textit{sc-текст}, обладающий истинным значением по отношению к соответствующей \textit{предметной области}}
	\scnsubset{связная sc-конструкция}
	\scnnote{Разные \textit{sc-знания} могут противоречить друг другу, то есть отражать разные точки зрения на некоторую \textit{предметную область}, но любое \textit{sc-знание} должно быть \textit{sc-текстом}, то есть не должно противоречить правилам \textit{Базовой денотационной семантики SC-кода}.}

	\scnheader{интеграция sc-конструкций*}
	\scnidtf{объединение sc-конструкций*}
	\scnidtf{объединение sc-множеств*}
	\scnnote{При интеграции sc-конструкций sc-элементы, обозначающие одну и ту же сущность, то есть синонимичные sc-элементы, считаются одинаковыми (совпадающими, тождественными) и, следовательно, должны склеиваться (отождествляться).}

	\scnheader{SC-пространство}
	\scnidtf{Результат интеграции \textit{всевозможных} sc-конструкций, \textit{семантически корректных} по отношению к \textit{Базовой денотационной семантики SC-кода}}
	\scnidtf{Предметная область, специфицируемая (описываемая) \textit{Базовой денотационной семантикой SC-кода}, которая является формальной онтологией, представленной средствами SC-кода}
	\scnidtf{Результат интеграции всевозможных sc-текстов (текстов SC-кода)}
	\scnidtf{Максимальный sc-текст}
	\scnidtf{Текст SC-кода, включающий в себя всевозможные sc-тексты}
	\scnidtf{Пространство sc-конструкций, семантически корректных по отношению к \textit{Базовой денотационной семантике SC-кода}}
	\begin{scnrelfromlist}{примечание}
		\scnfileitem{Особенностью \textit{SC-пространство} является то, что оно включает в себя и формальную онтологию, описывающую его свойства.}
		\scnfileitem{очевидно, что \textit{SC-пространство} является \textit{бесконечным} \textit{sc-текстом}, то есть текстом, содержащим бесконечное количество \textit{sc-элементов}. В частности, в состав \textit{SC-пространства} входят \textit{все} \textit{sc-элементы}, являющиеся элементами \textit{всех} \textit{sc-множеств}, знаки которых входят в состав \textit{SC-пространства}.}
		\scnfileitem{\textit{SC-пространство} является \scnqq{вместилищем} семантически корректных (по отношению к \textit{Базовой денотационной семантике SC-кода}) частей баз знаний всевозможных ostis-систем и, в том числе, глобальной (объединенной) \textit{Базы знаний Экосистемы OSTIS}. Подчеркнем при этом, что \textit{Экосистема OSTIS} является примером распределенных иерархических \textit{ostis-систем}.}
		\scnfileitem{Тот факт, что корректная (с точки зрения \textit{Базовой денотационной семантики SC-кода}) часть базы знаний \textit{каждой} \textit{ostis-системы} входит в состав \textit{SC-пространства}, позволяет трактовать описание соотношения между текущим состоянием \textit{базы знаний ostis-системы} и \textit{Sc-пространством} как описание того, что указанная \textit{ostis-система} в текущий момент времени не знает. Например, \textit{ostis-система} в некоторый момент времени может не знать (1) всех элементов некоторого конкретного \textit{конечного} \textit{sc-множества} (конечно sc-конструкции), (2) количества элементов указанного конечного \textit{sc-множества}, (3) какому подклассу заданного \textit{sc-класса} принадлежит указанный элемент этого \textit{sc-класса}.}
		\scnfileitem{В \textit{памяти ostis-системы} каждый \textit{sc-элемент} считается в рамках этой памяти \textit{временной} сущностью (имеется в виду сам \textit{sc-элемент}, а не обозначаемая им сущность), поскольку он появляется в \textit{памяти ostis-системы} и удаляется из нее независимо от того, что он обозначает. В отличие от этого в \textit{SC-пространстве} все sc-элементы считаются постоянными (\textit{постоянно} присутствующими) в рамках этого пространства.}
	\end{scnrelfromlist}

	\scnheader{Базовая денотационная семантика SC-кода}
	\scnidtf{Онтология Базовой денотационной семантики SC-кода}
	\scnidtf{Формальная \textit{онтология}, представленная в \textit{SC-коде} и являющаяся материнской \textit{онтологией} (онтологией самого высокого уровня) для всех остальных \textit{формальных онтологий}, представленных в \textit{SC-коде}}
	\scnidtf{Онтология SC-пространства}
	\scnidtf{Описание (представление) системы \textit{правил построения семантически корректируемых sc-конструкций}, удовлетворяющих требованиям Базовой денотационной семантики SC-кода}
	\scniselement{sc-онтология}
	\begin{scnindent}
		\scnidtf{формальная онтология, представленная в SC-коде}
	\end{scnindent} 
	\scnsuperset{\textbf{Семантическая классификация sc-элементов по базовым признакам}}
	\scnsuperset{\textbf{Уточнение смысла выделенных классов sc-элементов и связей между этими классами}}
	\scnsuperset{\textbf{Структура базовой семантической спецификации sc-элемента}}

	\scnheader{Логическая онтология SC-пространства}
	\scnrelto{логическая онтология}{Базовая денотационная семантика SC-пространства}
	\begin{scnrelfromset}{Правила, входящие в состав Логической онтологии SC-пространства}
		\scnfileitem{Вторыми компонентами \textit{sc-пар} константной парой принадлежности могут быть sc-элементы \textit{любого} типа (в том числе, и \textit{sc-переменные}), но первыми компонентами таких \textit{sc-пар} могут быть только \textit{константные} \textit{sc-множества}.}
		\scnfileitem{Знак \textit{sc-ситуации} связан с элементами этой ситуации \textit{sc-парами} константной \textit{постоянной} позитивной принадлежности. То есть позитивная принадлежность считается постоянной в рамках интервала времени существования соответствующей ситуации. В этом смысле ситуацию можно считать квазистатической.}
		\scnfileitem{Знак атомарной логической формулы связан со всеми элементами этой формулы \textit{sc-парами} \textit{константной} постоянной позитивной принадлежности, в том числе, и с теми элементами атомарной формулы, которые являются \textit{sc-переменными}.}
		\scnfileitem{Из переменного \textit{sc-множества} могут выходить только переменные \textit{sc-пары принадлежности}
		\item Не существует sc-пар принадлежности выходящих из обозначений внешних сущностей и \textit{sc-пар}.}
		\end{scnrelfromset}
\end{scnsubstruct}
