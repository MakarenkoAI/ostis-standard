\scnsegmentheader{Комплекс свойств, определяющих качество решателя задач
    кибернетической системы}

\begin{scnsubstruct}
    \scnheader{качество решателя задач кибернетической системы}
    \scnidtf{интегральная качественная оценка множества задач (действий), которые
        кибернетическая система способна выполнять в заданный момент}
    \scnidtf{качество навыков, приобретенных кибернетической системой}
    \scntext{примечание}{Основным свойством и назначением \textit{решателя задач
            кибернетической системы} является способность решать \textit{задачи} на основе
        накапливаемых (приобретаемых) \textit{кибернетической системой} различного вида
        \textit{навыков} с использованием \textit{процессора кибернетической системы},
        являющегося универсальным интерпретатором всевозможных накопленных
        \textit{навыков}. При этом качество (уровень развития, уровень совершенства)
        указанной способности определяется целым рядом дополнительных факторов
        (свойств).}
    \scnidtf{интеллектуальный уровень качества решателя задач
        кибернетической системы}
    \scnidtf{интегральное качество умений (навыков), приобретенных
        \textit{кибернетической системой} к текущему моменту}

    \begin{scnrelfromlist}{свойство-предпосылка}

        \scnitem{общая характеристика решателя задач кибернетической системы}
        \scnitem{качество логико-семантической организации памяти кибернетической
            системы}
        \scnitem{качество решения интерфейсных задач в кибернетической системе}

    \end{scnrelfromlist}
    
    \scnheader{общая характеристика решателя задач кибернетической системы}
    \begin{scnrelfromlist}{свойство-предпосылка}

        \scnitem{общий объем задач, решаемых кибернетической системой}
        \scnitem{многообразие видов задач, решаемых кибернетической системой}
        \scnitem{способность кибернетической системы к анализу решаемых задач}
        \scnitem{способность кибернетической системы к решению задач, методы решения
            которых в текущий момент известны}
        \scnitem{способность кибернетической системы к решению задач, методы решения
            которых ей в текущий момент не известны}
        \scnitem{множество навыков, используемых кибернетической системой}
        \scnitem{степень конвергенции и интеграции различного вида моделей решения
            задач, используемых кибернетической системой}
        \scnitem{качество организации взаимодействия процессов решения задач в
            кибернетической системе}
        \scnitem{быстродействие решателя задач кибернетической системы}
        \scnitem{способность кибернетической системы решать задачи, предполагающие
            использование информации, обладающей различного рода не-факторами}
        \scnitem{многообразие и качество решения задач информационного поиска}
        \scnitem{способность кибернетической системы генерировать ответы на вопросы
            различного вида в случае, если они целиком или частично отсутствуют в текущем
            состоянии информации, хранимой в памяти}
        \scnitem{способность кибернетической системы к рассуждениям различного вида}
        \scnitem{качество целеполагания}
        \scnitem{качество реализации планов собственных действий}
        \scnitem{способность кибернетической системы к локализации такой области
            информации,хранимой в ее памяти, которой достаточно для обеспечения решения
            заданной задачи}
        \scnitem{способность кибернетической системы к выявлению существенного в
            информации, хранимой в ее памяти}
        \scnitem{активность кибернетической системы}

    \end{scnrelfromlist}

    \scnheader{общий объем задач, решаемых кибернетической системой}
    \scnidtf{общий объем задач, которые кибернетическая система способна решать}
    \scnidtf{общий объем (множество), задач (действий), которые кибернетическая
        система способна (может, умеет) решать (выполнять) в заданный (в том числе, в
        текущий) момент}
    \scnrelfrom{свойство-предпосылка}{мощность языка представления задач, решаемых
        кибернетической системой}

    \scnheader{мощность языка представления задач, решаемых кибернетической системой}
    \scnidtf{мощность языка спецификации (описания) различного вида действий,
        выполняемых кибернетической системой}
    \scntext{примечание}{\textit{мощность языка представления задач} прежде всего
        определяется многообразием видов представляемых задач (многообразием видов
        описываемых действий).}
    \scnrelto{свойство-предпосылка}{многообразие видов
        задач, решаемых кибернетической системой}

    \begin{scnrelfromlist}{частное свойство}

        \scnitem{мощность языка представления задач, решаемых в памяти кибернетической системы}
        \begin{scnindent}
            \scnrelto{свойство-предпосылка}{многообразие видов задач, решаемых в
                памяти кибернетической системы}
        \end{scnindent}
        \scnitem{мощность языка представления задач, решаемых во внешней среде кибернетической системы}
        \begin{scnindent}
            \scnrelto{свойство-предпосылка}{многообразие видов
                задач, решаемых во внешней среде кибернетической системы}
        \end{scnindent}

        \scnitem{мощность языка представления задач, решаемых в рамках физической оболочки кибернетической системы}
        \begin{scnindent}
            \scnrelto{свойство-предпосылка}{многообразие
                видов задач, решаемых в рамках физической оболочки кибернетической системы}
        \end{scnindent}

    \end{scnrelfromlist}

    \scnheader{многообразие видов задач, решаемых кибернетической системой}
    \scnidtf{многообразие видов действий, которые кибернетическая система способна
        выполнять}
    \scntext{примечание}{Подчеркнем, что каждая задача есть спецификация соответствующего
        (описываемого) действия. Поэтому рассмотрение \textit{многообразия видов задач,
        решаемых кибернетической системой}, полностью соответствует многообразию видов
        деятельности, осуществляемой этой системой. Важно заметить, что есть виды
        \textit{деятельности кибернетической системы}, которые определяют качество и, в
        частности, \textit{уровень интеллекта кибернетической системы.}}
    \scnrelfrom{свойство-предпосылка}{мощность языка представления задач в
        памяти кибернетической системы}

    \begin{scnrelfromset}{комплекс частных свойств}

        \scnitem{многообразие видов задач, решаемых в памяти кибернетической системы}
        \scnitem{многообразие видов задач, решаемых во внешней среде кибернетической
            системы}
        \scnitem{многообразие видов задач, решаемых в рамках физической оболочки
            кибернетической системы}

    \end{scnrelfromset}

    \scnheader{способность кибернетической системы к анализу решаемых задач}
    \scnidtf{способность кибернетической системы осмысливать (ведать) то, что она
        творит}
    \scnidtf{способность анализировать свои цели и, соответственно, решаемые задачи
        на предмет:
        \begin{scnitemize}

            \item сложности достижения;
            \item целесообразности достижения (нужности, важности, приоритетности);
            \item соответствия цели существующим нормам (правилам) соответствующей
            деятельности
        \end{scnitemize}
    }

    \scnheader{способность кибернетической системы к решению задач, методы решения
        которых ей в текущий момент известны}
    \scntext{примечание}{Указанными методами могут быть не только алгоритмы, но также и
        функциональные программы, продукционные системы, логические исчисления,
        генетические алгоритмы, искусственные нейронные сети различного вида.}
    \begin{scnrelfromlist}{свойство-предпосылка}

        \scnitem{способность кибернетической системы к поиску хранимых в своей памяти
            методов решения инициированных задач}
        \scnitem{способность кибернетической системы к интерпретации хранимых в своей
            памяти методов решения задач}

    \end{scnrelfromlist}

    \scnheader{способность кибернетической системы к решению задач, методы решения
        которых ей в текущий момент не известны}
    \scnidtf{способность кибернетической системы к решению задач, для которых не
        найдены соответствующие (релевантные) им методы их решения}
    \scnidtf{способность кибернетической системы строить цепочку цель-план
        достижения цели-система действий}
    \scntext{примечание}{Задачи, для которых не находятся соответствующие им методы,
        решаются с помощью метаметодов (стратегий) решения задач, направленных:
        \begin{scnitemize}

            \item на генерацию нужных исходных данных (нужного контекста), необходимых для
            решения каждой задачи;
            \item на генерацию плана решения задачи, описывающего сведение исходной задачи
            к подзадачам (до тех подзадач, методы решения которых системы известны);
            \item на сужение области решения задачи (на сужения контекста задачи,
            достаточного для ее решения).
        \end{scnitemize}
    }
    
    \scnheader{множество навыков, используемых кибернетической системой}
    \scnidtf{объем и многообразие навыков, приобретенных кибернетической системой к
        текущему моменту (с помощью учителей-разработчиков или полностью
        самостоятельно)}
    \scnidtf{возможности, навыки, приобретенные кибернетической системой}
    \scnidtf{опыт, приобретенный кибернетической системой}
    \scntext{примечание}{Новые навыки могут приобретаться кибернетической системой либо
        полностью самостоятельно, либо с помощью учителей, которые в простейшем случае
        просто сообщают обучаемой системе полностью сформулированные навыки. Для
        компьютерных систем учителями является их разработчики.}\bigskip
    \begin{scnrelfromlist}{частное свойство}

        \scnitem{множество методов решения задач, используемых кибернетической
            системой}
        \scnitem{множество моделей решения задач, используемых кибернетической
            системой}
        \scnitem{мощность языка представления в памяти кибернетической системы методов
            и моделей решения задач}

    \end{scnrelfromlist}
    
    \scnheader{множество методов решения задач, используемых кибернетической
        системой}
    \scnidtf{множество методов решения задач, используемых кибернетической системой
        и хранимых в ее памяти}
    \scnrelto{частное свойство}{многообразие видов знаний, хранимых в памяти
        кибернетической системы}
    
        \scnheader{метод решения задач}
    \scntext{пояснение}{\textbf{\textit{метод решения задач}} --- это \textit{вид
            знаний}, хранимых в \textit{памяти кибернетической системы} и содержащих
        информацию, которой достаточно либо для сведения каждой \textit{задачи} из
        соответствующего \textit{класса задач} к \textit{полной системе подзадач*},
        решение которых гарантирует решение исходной \textit{задачи}, \uline{либо} для
        окончательного решения этой \textit{задачи} из указанного \textit{класса задач}}
            
    \scnheader{множество моделей решений задач, используемых кибернетической
        системой}
    \scnidtf{способность кибернетической системы к использованию различных видов
        методов решения задач, соответствующих различным моделям решения задач}
    \scnidtf{многообразие методов решения задач, используемых кибернетической
        системой}
    \scnrelfrom{свойство-предпосылка}{мощность языка представления в памяти
        кибернетической системы методов и моделей решения задач}
    
    \scnheader{множество моделей решения задач, используемых кибернетической
        системой}
    \begin{scnrelfromset}{примечание}
        \scnitem{следует отличать*}
        \begin{scnindent}
            \begin{scnhaselementset}
                \scnitem{вид задач}
                \scnitem{модель решения задач}
                \begin{scnindent}
                    \scntext{пояснение}{каждая \textit{модель решения задач} задается
                        \begin{scnitemize}
                            \item \textit{языком}, обеспечивающим представление в \textit{памяти
                                кибернетической системы} некоторого класса \textit{методов решения задач}
                            \item интерпретатором указанных \textit{методов}, определяющим
                            \textit{операционную семантику} указанного \textit{языка}
                        \end{scnitemize}
                    }
                \end{scnindent}
                \scnitem{метод решения задач}
                \scnitem{класс задач}
                \begin{scnindent}
                    \scnidtf{\textit{множество} всех тех и только тех
                        \textit{задач}, которые решаются с помощью соответствующего \textit{метода}}
                \end{scnindent}
            \end{scnhaselementset}
        \end{scnindent}
    \end{scnrelfromset}

    \scnheader{Степень конвергенции и интеграции различного вида моделей решения
        задач, используемых кибернетической системой}
    \scntext{примечание}{Необходим переход от эклектики никак не связанных друг с другом
        \textit{моделей решения задач} к их \textit{конвергенции}, это предполагает:
        \begin{scnitemize}
            \item разработку общего (базового) для всех \textit{моделей решения задач}
            языка описания \textit{операционной семантики} языков описания методов,
            соответствующих различным \textit{моделям решения задач};
            \item включение всех языков описания \textit{методов решения задач} в общую
            систему языков, связанных между собой отношением \scnqqi{язык-подъязык*}.
        \end{scnitemize}
    }
    
    \scnheader{качество организации взаимодействия процессов решения задач в
        кибернетической системе}
    \begin{scnrelfromlist}{частное свойство}

        \scnitem{качество управления информационным процессом в памяти кибернетической системы}
        \begin{scnindent}
            \scnrelfrom{свойство-предпосылка}{обеспечение процессором
                кибернетической системы качественного управления информационными процессами в
                памяти}
        \end{scnindent}
        
        \scnitem{качество организации взаимодействия процессов решения задач во внешней
            среде или в физической оболочке кибернетической системы}
        \begin{scnindent}
            \begin{scnrelfromlist}{свойство-предпосылка}

                \scnitem{последовательность/параллельность процессов решения задач в
                    кибернетической системе}
                \scnitem{ синхронность/асинхронность процессов решения задач в кибернетической
                    системе}
                \scnitem{ централизованной/децентрализованность управления процессами решения
                    задач в кибернетической системе}

            \end{scnrelfromlist}
        \end{scnindent}

    \end{scnrelfromlist}
    \scntext{примечание}{Качество решения каждой \textit{задачи} определяется:
        \begin{scnitemize}

            \item временем её решения (чем быстрее \textit{задача} решается, тем выше
            качество её решения);
            \item полнотой и корректностью результата решения \textit{задачи};
            \item затраченными для решения \textit{задачи} ресурсами памяти (объемом
            фрагмента хранимой информации, используемой для решения задачи);
            \item затраченным для решения \textit{задачи} ресурсами решателя задач
            (количеством используемых внутренних агентов).
        \end{scnitemize}
        Таким образом, повышение качества процесса решения каждой конкретной
        \textit{задачи}, а также каждого \textit{класса задач} (путем совершенствования
        соответствующего метода, в частности, алгоритма) является важным фактором
        повышения качества \textit{решателя задач} в
        целом.}
        
    \scnheader{агентно-ориентированная модель обработки информации в памяти}
    \scnidtf{агентно-ориентированная модель управления действиями кибернетической
        системы, выполняемыми ею в своей памяти}
    \scntext{пояснение}{Перспективным вариантом построения \textit{решателя задач
            кибернетической системы} является реализация \textit{агентно-ориентированной
            модели обработки информации}, т.е. построение \textit{решателя задач} в виде
        \textit{многоагентной системы}, агенты которой осуществляют обработку
        \textit{информации, хранимой в памяти} кибернетической системы, и управляются
        этой информацией (точнее, её текущим состоянием). Особое место среди этих
        \textit{агентов} занимают сенсорные (рецепторные) и эффекторные
        \textit{агенты}, которые, соответственно, воспринимают информацию о текущем
        состоянии \textit{внешней среды} и воздействуют на \textit{внешнюю среду}, в
        частности, путем изменения состояния \textit{физической оболочки
            кибернетической системы}.
            \\Подчеркнем, что указанная агентно-ориентированная
        модель организации взаимодействия процессов решения задач в
        \textit{кибернетической системе} по сути есть не что иное, как модель
        ситуационного управления процессами решения задач, решаемых
        \textit{кибернетической системой} как в своей \uline{внешней среде}, так и в
        своей памяти.}
        \begin{scnindent}
            \scntext{детализация}{\nameref{sd_agents}}
        \end{scnindent}
        
    \scnheader{модель инициирования действий кибернетической системы}
    \scnidtf{модель управления поведением кибернетической системы}

    \begin{scnsubdividing}

        \scnitem{стимульно-реактивная модель инициирования действий}
        \begin{scnindent}
            \scntext{пояснение}{от комбинации \textit{исходных сигналов},
                формируемых, например, априори известным набором сенсоров (рецепторов) к
                комбинации выходных \textit{сигналов}, управляющих, например, априори известным набором эффекторов}
        \end{scnindent}
        \scnitem{ситуационная модель инициирования действий без учета предыстории ситуаций и событий}
        \begin{scnindent}    
            \scntext{пояснение}{действие инициируется возникновением
                в памяти \textit{ситуации} априори известной конфигурации или априори
                известного события}
        \end{scnindent}
        \scnitem{ситуационная модель инициирования действий с учетом предыстории ситуаций и событий}
        \begin{scnindent}
            \scntext{пояснение}{действие инициируется не только
                текущей \textit{ситуацией} но и предшествующими \textit{ситуациями}, т.е.
                событиями перехода от одних \textit{ситуаций} к другим}
        \end{scnindent}

    \end{scnsubdividing}
    \scntext{примечание}{Речь идет о действиях, выполняемых \textit{кибернетической
            системой} как во внешней среде, так и в своей внутренней \textit{среде} (в
        своей памяти).}
        
    \scnheader{последовательность/параллельность процессов решения
        задач в кибернетической системе}
    \scnidtf{способность одновременно решать несколько разных задач, некоторые из
        которых могут быть подзадачами одной и той же задачи}
    \scnidtf{способность одновременно решать несколько разных задач, некоторые из
        которых могут быть подзадачами одной и той же задачи}

    \begin{scnrelfromlist}{свойство-предпосылка}

        \scnitem{максимально возможное количество действий, одновременно выполняемых
            кибернетической системой}
        \scnitem{способность кибернетической системы к одновременному выполнению
            взаимосвязанных действий}
        \begin{scnindent}    
            \scnidtf{способность кибернетической системы к
                одновременному выполнению действий, выполнение каждого из которых может
                помешать выполнению другого}
            \scnidtf{способность кибернетической системы к эквилибристике}
        \end{scnindent}

    \end{scnrelfromlist}

    \begin{scnrelfromset}{комплекс частных свойств}

        \scnitem{физическая последовательность/параллельность процессов решения задач в
            кибернетической системе}
        \scnitem{ логическая последовательность/параллельность процессов решения задач
            в кибернетической системе}
        \begin{scnindent}
            \scntext{пояснение}{Логическая параллельность выполняемых процессов
                (действий) предполагает возможность существования \uline{выполняемых} процессов
                в двух режимах:
                \begin{scnitemize}

                    \item в активном режиме --- в режиме непосредственного выполнения
                    \item в режиме прерывания --- в режиме ожидания	условий (событий и/или
                    ситуаций) при возникновении которых прерванный процесс переходит в режим
                    активного процесса.
                \end{scnitemize}
            }
        \end{scnindent}

    \end{scnrelfromset}

    \begin{scnrelfromset}{комплекс частных свойств}

        \scnitem{последовательность/параллельность информационных процессов в памяти
            кибернетической системы}
        \scnitem{ последовательность/параллельность процессов решения задач во внешней
            среде или в физической оболочке кибернетической системы}

    \end{scnrelfromset}
    \scntext{примечание}{Подчеркнем, что есть целый ряд задач, решаемых кибернетической
        системой, процессы решения которых носят перманентный (постоянный) характер. К
        таким задачам относятся:
        \begin{scnitemize}

            \item поддержка высокого качества базы знаний (устранение противоречий,
            информационного мусора);
            \item поддержка семантической совместимости с другими компьютерными системами;
            \item мониторинг и анализ состояния внешней среды;
            \item обеспечение собственной безопасности;
            \item самообучение.
        \end{scnitemize}
    }
    
    \scnheader{быстродействие решателя задач кибернетической системы}
    \scnidtf{скорость решения задач в кибернетической системе}
    \scnidtf{быстродействие решателя задач кибернетической системы}
    \scnidtf{скорость реакции кибернетической системы на различные задачные
        ситуации}
    \scnrelfrom{свойство-предпосылка}{быстродействие процессора кибернетической
        системы}
    
    \scnheader{способность кибернетической системы решать задачи, предполагающие
        использование информации, обладающей различного рода не-факторами}
    \scnidtf{способность кибернетических систем решать задачи, которые:
        \begin{scnitemize}

            \item либо нечетко сформулированы (делай то, не знаю что);
            \item либо решаются в условиях неполноты, неточности, противоречивости исходных
            данных;
            \item либо являются задачами, принадлежащими классам задач, для которых
            практически невозможно построить соответствующие алгоритмы.
        \end{scnitemize}
    }
    \scnidtf{способность кибернетической системы решать труднорешаемые,
        трудноформализуемые задачи}
    \scnidtf{способность решать интеллектуальные (трудноформализуемые) задачи, для
        которых характерна:
        \begin{scnitemize}

            \item неточность и недостоверность исходных данных;
            \item отсутствие критерия качества результата;
            \item невозможность или высокая трудоемкость разработки алгоритма;
            \item необходимость учета контекста задачи.
        \end{scnitemize}
    }

    \scnheader{задача, предполагающая использование информации, обладающей
        различного рода не-факторами}
    \scnidtf{трудноформализуемая задача}
    \scnsuperset{задача проектирования}
    \scnsuperset{задача распознавания}
    \scnsuperset{задача прогнозирования}
    \scnsuperset{задача целеполагания}
    \scnsuperset{задача планирования}
    
    \scnheader{многообразие и качество решения задач информационного поиска}
    \scnrelfrom{свойство-предпосылка}{семантический уровень доступа к информации,
        хранимой в памяти кибернетической системы}
    \scnrelto{частное свойство}{многообразие видов задач, решаемых кибернетической
        системой}
    \scnidtf{способность кибернетической системы качественно решать широкое
        многообразие задач информационного поиска в рамках текущего состояния хранимой
        информации}
    \scnidtf{способность кибернетической системы находить в текущем состоянии
        хранимой информации релевантные ответы на запросы (вопросы) самого различного
        вида}
    
    \scnheader{вопрос}
    \scnidtf{запрос}
    \scnsuperset{запрос изоморфных или гомоморфных фрагментов хранимой информации
        по заданному образцу с указанием знаков известных сущностей}
    \begin{scnindent}
        \scnrelfrom{класс частных вопросов}{запрос всех связок различных отношений,
            обязывающих заданную сущность с другими}
            \begin{scnindent}
                \scnrelfrom{класс частных вопросов}{запрос всех связок заданных отношений,
                    связывающих заданную сущность с другими}
            \end{scnindent}
    \end{scnindent}
    \scnsuperset{вопрос типа \scnqqi{как связаны между собой заданные две сущности}}
    \begin{scnindent}
    \scntext{пояснение}{Две сущности будем считать связанными в том и только в
        том случае, если существует маршрут, соединяющий указанные две сущности, в
        состав которого входят связки, принадлежащие в общем случаем разным
        отношениям}\scntext{примечание}{Здесь принципиально важным является учет
        \textit{семантической силы связей} между сущностями, которая определяется
        \textit{семантической силой отношений}, которым принадлежат связки, входящие в
        состав связей (маршрутов) между сущностями.}
    \scnrelto{класс частных
        вопросов}{вопрос типа \scnqqi{как связаны между собой заданные сущности}}
        \begin{scnindent}
            \scntext{примечание}{Здесь имеется в виду произвольное количество связываемых
                сущностей, а это предполагает, что ответом на данный запрос является
                \uline{связный граф}, вершинами которого являются знаки заданных сущностей.}
        \end{scnindent}
    \end{scnindent}
    \scnsuperset{вопрос типа \scnqqi{что это такое}}
    \begin{scnindent}
        \scnidtf{запрос спецификации (описания) заданной сущности}
        \scnrelfrom{класс частных вопросов}{запрос определения}
        \begin{scnindent}
            \scnidtf{запрос определения заданного понятия}
        \end{scnindent}
        \scnrelfrom{класс частных вопросов}{запрос документации заданного объекта}
    \end{scnindent}
    \scnsuperset{почему-вопрос}
    \begin{scnindent}
        \scnsuperset{запрос причины возникновения заданной ситуации или события}
        \scnsuperset{запрос логического обоснования заданного высказывания}
            \begin{scnindent}
                \scnidtf{запрос объяснения корректности заданного высказывания, которое, в
                    частности, может быть порождено (сгенерировано) в процессе решения некоторой
                    задачи с помощью некоторого метода (алгоритма, искусственной нейронной сети
                    логического исчисления и т.п.)}
                \scnsuperset{запрос доказательства заданной теоремы}
            \end{scnindent}
    \end{scnindent}
    \scnsuperset{запрос возможных последствий заданной ситуации или события}
    \scnsuperset{запрос того, что логически следует из заданного высказывания}
    \scnsuperset{запрос метода решения данной задачи}
    \scnsuperset{запрос плана решения данной задачи}
    \begin{scnindent}
        \scnidtf{запрос декомпозиции данной задачи на систему и/или подзадач}
    \end{scnindent}
    \scnsuperset{зачем-вопрос}
    \begin{scnindent}
        \scnidtf{каково назначение заданной сущности}
        \scnidtf{для решения какой задачи (для чего, достижения какой цели) нужна
            данная сущность}
    \end{scnindent}
    \scnsuperset{запрос аналогов заданной сущности}
    \scnsuperset{запрос антиподов заданной сущности}
    \scnsuperset{запрос сходств и отличий двух связанных сущностей}
    \scnsuperset{запрос сравнительного анализа заданной сущности}
    \begin{scnindent}
        \scnsuperset{запрос достоинств заданной сущности}
        \scnsuperset{запрос недостатков заданной сущности}
    \end{scnindent}
    \scnsuperset{где-вопрос}
    \begin{scnindent}
        \scnidtf{запрос информации о местоположении заданной пространственной сущности
            примечание}
        \scntext{примечание}{Здесь запрашивается любая информация о пространственных связях
            заданной сущности}
    \end{scnindent}    
    \scnsuperset{когда-вопрос}
    \begin{scnindent}
        \scnidtf{запрос информации о темпоральных свойствах и связях заданной временной
            сущности (о моменте начала, о моменте завершения, о длительности)}
    \end{scnindent}

   \scnheader{cпособность кибернетической системы генерировать ответы на вопросы
        различного вида в случае, если они целиком или частично отсутствуют в текущем
        состоянии информации, хранимой в памяти}
    \scnidtf{способность кибернетической системы генерировать (порождать, строить,
        синтезировать, выводить) ответы на самые различные вопросы и, в частности, на
        вопросы типа \scnqqi{что это такое}, на почему-вопросы, это означает способность
        кибернетической системы \uline{объяснять} (обосновывать корректность) своих
        действий}
    \begin{scnrelfromlist}{свойство-предпосылка}
        \scnitem{семантическая гибкость информации, хранимой в памяти кибернетической
            системы}
        \scnitem{ способность кибернетической системы к рассуждениям различного вида}
    \end{scnrelfromlist}
    \begin{scnindent}
        \scntext{детализация}{\nameref{sd_sc_quest_lang}}
    \end{scnindent}

    \scnheader{способность кибернетической системы к рассуждениям различного вида}
    \scnidtf{способность кибернетической системы к целенаправленному порождению
        (генерации) новых истинных или правдоподобных знаний (следствий) на основе
        имеющихся знаний (посылок)}

    \begin{scnrelfromlist}{частное свойство}

        \scnitem{способность кибернетической системы к дедуктивному выводу}
        \scnitem{способность кибернетической системы к индуктивному выводу}
        \scnitem{способность кибернетической системы к абдуктивному выводу}

    \end{scnrelfromlist}

    \scnheader{качество целеполагания}
    \scnidtf{качество реализации первого этапа решения сложных задач --- этапа
        генерации (построения) планов решения сложных задач}
    \scnidtf{качество генерации планов выполнения сложных действий:
        \begin{scnitemize}

            \item как внутренних действий (в памяти кибернетической системы), так и внешних
            действий (во внешней среде)
            \item как собственных действий, так и действий других субъектов
        \end{scnitemize}
    }
    \scnidtf{качество генерации планов действий кибернетической системы и, в
        частности, трудоемкость процесса генерации этих планов}
    \scnidtf{качество организации целенаправленной деятельности кибернетической
        системы}
    \scnidtf{качество построения цепочек цель-план-действие }
    \scnidtf{качество генерации, анализа и инициирования собственных целей
        (собственных задач)}
    \scnidtf{способность кибернетической системы к целеполаганию}

    \begin{scnrelfromlist}{свойство-предпосылка}
        \scnitem{самостоятельность целеполагания}
        \begin{scnindent}
            \scnidtf{самостоятельность генерации
                и инициирования целей (задач), направленных на создание условий достижения
                соответствующих стратегических целей (сверхзадач)}
        \end{scnindent}
        \scnitem{целенаправленность целеполагания}
        \begin{scnindent}
            \scnidtf{степень соответствия
                (степень полезности) генерируемых целей (задач) для достижения соответствующих
                стратегических целей (сверхзадач)}
        \end{scnindent}
        \scnitem{сбалансированность целеполагания}
        \begin{scnindent}
            \scnidtf{качество расстановки
                приоритетов у сгенерированных и инициированных целей (задач) для обеспечения
                баланса между тактическими и стратегическими целями}
        \end{scnindent}
    \end{scnrelfromlist}

    \scnheader{самостоятельность целеполагания}
    \scnidtf{способность \textit{кибернетической системы} генерировать, инициировать и
        решать задачи, которые не являются подзадачами, инициированными внешними
        (другими) субъектами, а также способность на основе анализа своих возможностей
        отказаться от выполнения задачи, инициированной извне, переадресовав её другой
        \textit{кибернетической системе}, либо на основе анализа самой этой задачи обосновать её
        нецелесообразность или некорректность}
    \scnidtf{способность к самостоятельному целеполаганию (генерации идей) и к
        инициированию процессов их достижения (т.е. к принятию решений), способность
        свободно (в определенных рамках) выбирать (ставить перед собой цели)}
    \scnidtf{уровень самостоятельности}
    \scnidtf{способность решать задачи в комплексе, включая создание всех
        необходимых условий для их решения с учетом конкретных обстоятельств}
    \scnidtf{умение решать задачи в условиях сильных помех (в осложненных
        обстоятельствах)}
    \scntext{примечание}{Повышение \textit{уровня самостоятельности} существенно расширяет
        возможности \textit{кибернетической системы}, т.е. объем тех задач, которые она может
        решать не только в идеальных  условиях, но и в реальных (осложненных)
        обстоятельствах.}
    \scnidtf{степень свободы выбора целей, подлежащих достижению,
        а также свободы генерации целей, не являющихся подцелями извне поставленных
        целей}

    \scnheader{целенаправленность целеполагания}
    \scnidtf{целеустремленность}
    \scnidtf{целенаправленность}
    \scnidtf{степень целостности деятельности}
    \scnidtf{степень соответствия между тактическими и стратегическими уровнями
        деятельности}
    \scnidtf{общее соотношение между временем, затраченным на лишние  (ненужные,
        нецелесообразные, нецеленаправленные) действия и полезные действия}
    \scnidtf{целесообразность деятельности}
    \scnidtf{способность адекватно расставлять приоритеты своим целям и не
        распыляться  на достижение неприоритетных (несущественных) целей}
    
        \scnheader{качество реализации планов собственных действий}
    \scnidtf{качество реализации целенаправленной деятельности на основе
        построенных планов}
    \scnidtf{качество реализации построенных в памяти кибернетической системы
        планов выполнения сложных собственных действий, которые могут предполагать
        участие других субъектов}
    
    \scnheader{способность кибернетической системы к локализации такой области
        информации, хранимой в ее памяти, которой достаточно для обеспечения решения
        заданной задачи}
    \scnidtf{способность кибернетической системы к сужению области решения каждой
        решаемой ею задачи, что существенно минимизирует затраты кибернетической
        системы на учет и анализ факторов, априори незначимых (несущественных) для
        решения каждой решаемой задачи}
    \scntext{примечание}{Для реализации данной способности важное значение имеет
        качественная стратификация базы знаний кибернетической системы на предметные
        области и соответствующие им онтологии.}
        
    \scnheader{способность кибернетической системы к выявлению существенного в информации, хранимой в ее памяти}
    \scnidtf{способность к выявлению (обнаружению, выделению) таких фрагментов
        информации, хранимой в памяти кибернетической системы, которые существенны
        (важны) для достижения соответствующих целей}
    \scntext{примечание}{Понятие существенного (важного) фрагмента информации, хранимой в
        памяти кибернетической системы, относительно и определяется соответствующей
        задачей. Тем не менее, есть важные перманентно (постоянно) решаемые задачи, в
        частности задачи анализа качества информации, хранимой в памяти кибернетической
        системы. Существенные фрагменты хранимой информации, выделяемые в процессе
        решения этих задач, являются относительными не столько по отношению к решаемой
        задаче, сколько по отношению к текущему состоянию хранимой информации.
        Примерами таких фрагментов являются:
        \begin{scnitemize}
            \item обнаруженные противоречия (ошибки) с явным указанием того, что чему
            противоречит;
            \item обнаруженные информационные дыры, точнее точная спецификация этих дыр;
            \item обнаруженные мусорные фрагменты, которые либо носят вспомогательный
            характер, либо могут быть легко восстановлены (воспроизведены).
        \end{scnitemize}
    }
    
    \scnheader{следует отличать*}
    \begin{scnhaselementset}
        \scnitem{способность кибернетической системы к выявлению существенного в
            информации, хранимой в ее памяти}
        \begin{scnindent}    
            \scntext{примечание}{Здесь кибернетическая система
                выделяет информацию, которая необходима, но не обязательно достаточна для
                решения соответствующей задачи.}
        \end{scnindent}
        \scnitem{способность кибернетической системы к локализации такой области
            информации, хранимой в ее памяти, которой достаточно для обеспечения решения
            заданной задачи}
        \begin{scnindent}    
            \scntext{примечание}{Здесь кибернетическая система отбрасывает
                (исключает) информацию, которая априори несущественна для решения
                соответствующей (заданной) задачи.}
        \end{scnindent}
    \end{scnhaselementset}

    \scnheader{активность кибернетической системы}
    \scnidtf{уровень активности кибернетической системы}
    \scnidtf{уровень мотивации к деятельности в различных направлениях}
    \scnidtf{уровень желания  действовать}
    \scnidtf{активность/пассивность кибернетической системы}
    \scnidtf{уровень инициативности, пассионарности, мотивированности}
    \scntext{примечание}{Уровень активности кибернетической системы может быть разным для
        разных решаемых задач, для разных классов выполняемых действий, для разных
        видов деятельности.}\scntext{примечание}{Следует отличать уровень активности
        (мотивации, желания) и направленность этой активности.}\scntext{примечание}{Чем выше
        активность кибернетической системы, тем (при прочих равных условиях) она больше
        успевает сделать, следовательно, тем выше ее качество
        (эффективность).}\scnrelboth{обратное свойство}{пассивность}
    \begin{scnindent}
        \scnidtf{уровень бездеятельности, медлительности, вялости, ленивости}
    \end{scnindent}
    \begin{scnrelfromlist}{частное свойство}
        \scnitem{познавательная активность}
        \scnitem{социальная активность}
    \end{scnrelfromlist}
    
    \scnheader{качество логико-семантической организации памяти
        кибернетической системы}
    \scnidtf{качество базовых семантически целостных действий в памяти
        кибернетической системы}
    \scnidtf{качество семантически элементарных (законченных, целостных)
        информационных процессов, выполняемых кибернетической системой в своей памяти}
    \scnidtf{интегральная оценка того, насколько способствует (насколько близка)
        организация памяти кибернетической системы реализации осмысленных
        преобразований, хранимых в памяти знаний}
    \scnidtf{степень приспособленности решателя задач кибернетической системы к
        обработке сложноструктурированных баз знаний}
    \scnidtf{степень приспособленности решателя задач кибернетической системы к
        обработке хранимой в её памяти информации, имеющий высокий уровень качества как
        по форме представления информации, так и по её содержанию --- по многообразию
        представляемых знаний и по уровню их конвергенции и интеграции}

    \begin{scnrelfromlist}{свойство-предпосылка}
        \scnitem{семантический уровень доступа к информации, хранимой в памяти
            кибернетической системы}
        \scnitem{семантическая гибкость информации, хранимой в памяти кибернетической
            системы}
        \scnitem{степень конвергенции и интеграции представления навыков, хранимых в
            памяти кибернетической системы, с представлением обрабатываемой информации}
    \end{scnrelfromlist}

    \scnheader{семантический уровень доступа к информации, хранимой в памяти
        кибернетической системы}
    \scnidtf{степень ассоциативности доступа к информации, хранимой в памяти
        кибернетической системы}
    \scnidtf{способность кибернетической системы локализовывать (находить)
        требуемый (запрашиваемый) фрагмент информации, хранимой в её памяти, не на
        основании известного адреса запрашиваемой информации (её местоположения в
        памяти), а на основании:
        \begin{scnitemize}
            \item известного типа запрашиваемой информации;
            \item известных сущностей, знаки которых входят в состав запрашиваемой
            информации;
            \item полностью или частично известной конфигурации запрашиваемой информации
            (т.е. конфигурации связей между известными и искомыми сущностями)
        \end{scnitemize}
    }
    \scntext{пояснение}{\textit{уровень доступа к информации, хранимой в памяти
            кибернетической системы} определяется тем, что нам достаточно знать об искомой
        в памяти кибернетической системы информации (в частности, об искомом знаке
        некоторой интересующей нас сущности). Мы можем знать место в памяти (ячейку
        памяти, область памяти), где находится интересующая нас информация. Такой
        доступ называется \uline{адресным}. Мы можем знать имя интересующей нас
        сущности, но не знать, где находится информация, описывающая эту сущность. Мы
        можем не знать имени интересующей нас сущности, но знать, как эта сущность
        связана с другими известными нам сущностями.}
    \scntext{пояснение}{Пусть нам необходимо локализовать (выделить) хранимую в памяти информацию, описывающую
        известные нам сущности, связанные известными нам отношениями, но
        местонахождение этой информации в памяти нам не известно. Если организация
        памяти нам представляет такую возможность, то такую память будем называть
        ассоциативной, т.е. памятью, обеспечивающей семантический доступ к хранимой в
        ней информации.}
    \scntext{примечание}{Для того, чтобы построить информационную модель
        среды, в которой действует (функционирует) кибернетическая система, необходимо,
        с одной стороны, разложить  эту информационную модель по полочкам , превратить
        её в некую систему из компонентов этой информационной модели, а, с другой
        стороны, обеспечить быстрый поиск нужного фрагмента указанной информационной
        модели, не зная, на каких полочках   находятся компоненты этого искомого
        фрагмента, который при этом может иметь произвольную конфигурацию и
        произвольный размер. Это и есть высший уровень \uline{ассоциативности} доступа
        к информации, хранимой в памяти кибернетической
        системы.}
    \scntext{пояснение}{Данное свойство, данная характеристика
        организации информации, хранимой в памяти кибернетической системы, является
        важнейшей характеристикой \uline{внутреннего} языка представления информации в
        памяти кибернетической системы. Указанная характеристика внутреннего языка
        определяется \uline{простотой процедур поиска} востребованных (запрашиваемых)
        фрагментов хранимой информации --- например, процедуры поиска знаков всех
        сущностей, каждая из которых связана с заданными (известными) сущностями
        связями заданных (известных) типов, процедуры поиска (выделения) знаков всех
        сущностей, которые связаны с заданной (известной) сущностью связью неважно
        какого типа, процедуры поиска информационного фрагмента заданному образцу
        (шаблону) произвольного размера и конфигурации, в котором выделены знаки
        известных сущностей и условные обозначения искомых
        сущностей.}
    \scnrelfrom{свойство-предпосылка}{степень близости языка внутреннего
        представления информации в памяти кибернетической системы к смысловому
        представлению информации}

    \scnheader{семантическая гибкость информации, хранимой в памяти
        кибернетической системы}
    \scnrelfrom{свойство-предпосылка}{степень близости языка внутреннего
        представления информации в памяти кибернетической системы к смысловому
        представлению информации}
    \scnidtf{простота реализации базовых (элементарных), но семантически целостных
        (семантически значимых, осмысленных) действий (операций) преобразования
        (обработки) информации, хранимой в памяти кибернетической системы}
    
    \scnheader{базовое семантически целостное действие над информацией, хранимой в
        памяти кибернетической системы}
    \scnidtf{элементарная семантически значимая (осмысленная) операция над
        информацией, хранимой в памяти кибернетической системы}
    \scntext{примечание}{Здесь принципиальной является семантическая целостность
        (осмысленность) действия над хранимой информацией. Так, например, операция
        адресного доступа к требуемому фрагменту хранимой информации не является
        семантически целостной, так как смысл искомого (запрашиваемого) фрагмента
        хранимой информации не уточняется.}
    \scntext{примечание}{Разные кибернетические
        системы могут использовать разные наборы классов базовых семантически целостных
        действий над информацией, хранимой в их памяти.}
    \scntext{примечание}{Примерами
        \textit{базовых семантически целостных действий над информацией, хранимой в
            памяти кибернетической системы}, в частности, являются:
        \begin{scnitemize}

            \item операции поиска, генерации, удаления или замены связок между знаками
            известных сущностей;
            \item операции поиска, генерации, удаления или замены имен, приписываемых
            знакам известных сущностей.
        \end{scnitemize}
        Существенно подчеркнуть, что простота реализации такого рода операций (т.е.
        гибкость хранимой в памяти информации) во многом обеспечивается стремлением к
        локальности выполнения этих операций. Такая локальность означает то, что при
        выполнении \uline{каждой} из указанных операций меняется только обрабатываемый
        фрагмент хранимой информации и не требуется никакого переразмещения в памяти
        остальной части хранимой информации.}
        
    \scnheader{степень конвергенции и интеграции представления навыков, хранимых в памяти кибернетической системы, с
        представлением обрабатываемой информации}
    \scnrelto{частное свойство}{степень конвергенции и интеграции различного вида
        знаний, хранимых в памяти кибернетической системы}
    \begin{scnindent}
        \scnrelfrom{свойство-предпосылка}{степень близости языка внутреннего
            представления информации в памяти кибернетической системы к смысловому
            представлению информации}
    \end{scnindent}
    \scntext{примечание}{Навыки кибернетической системы являются частным видом знаний,
        хранимых в её памяти, поэтому степень конвергенции навыков и обрабатываемых
        знаний определяется глубиной  и объемом  \uline{общих} (одинаковых) принципов,
        лежащих в основе как представления навыков, так представления обрабатываемых
        знаний.}
        
    \scnheader{качество решения интерфейсных задач в
        кибернетической системе}
    \begin{scnrelfromlist}{частное свойство}
        \scnitem{способность кибернетической системы к пониманию сенсорной информации}
        \scnitem{способность кибернетической системы к пониманию принимаемых сообщений}
        \scnitem{способность кибернетической системы к самостоятельной деятельности во
            внешней среде}
            \begin{scnindent}
                \scnidtf{способность кибернетической системы к воздействию на
                внешнюю среду и к управлению своим поведением во внешней среде}
            \end{scnindent}
    \end{scnrelfromlist}

    \scnheader{интерфейсная задача}
    \scnsuperset{задача анализа введенной информации}
    \scnsuperset{задача анализа сенсорной информации}
    \begin{scnindent}
        \scnidtf{задача анализа информации, порождаемой (генерируемой) непосредственно
            сенсорами кибернетической системы}
        \scnsuperset{задача синтаксического анализа сенсорной информации}
        \scnsuperset{задача семантического анализа сенсорной информации}
        \begin{scnindent}
            \scnidtf{задача анализа сенсорной информации, направленного на
                \uline{понимание} этой информации --- на выявление (распознавание) в этой
                информации отображения (сенсорного описания) объектов, важных для
                кибернетической системы (т.е. объектов, описанных в базе знаний этой системы и,
                соответственно, представленных в этой базе знаний своими знаками либо знаками
                классов, которым эти объекты принадлежат), а также важных для кибернетических
                связей между указанными объектами}
            \scnidtf{задача генерации фрагмента базы знаний кибернетической системы,
                являющегося логическим следствием заданной сенсорной информации и
                представляющегося собой важную для кибернетической системы информацию}
            \scnidtf{задача извлечения из сенсорной информации (первичной информации)
                важной для кибернетической системы вторичной информации}
            \scnsuperset{задача анализа принимаемого вербального сообщения}
            \scnidtf{задача анализа введенных знаковых конструкций}
            \scnidtf{задача анализа сообщений, введенных в кибернетическую систему}
            \scnidtf{задача анализа внешних знаковых конструкций}
            \scnsuperset{задача синтаксического анализа принимаемого вербального сообщения}
            \scnsuperset{задача трансляции принимаемого вербального сообщения на внутренний
                язык кибернетической системы}
            \scnsuperset{задача погружения нового фрагмента в состав согласованной части
                базы знаний}
            \begin{scnindent}
                \scnidtf{задача интеграции (встраивания) нового фрагмента базы знаний в состав
                    базы знаний}
                \scnidtf{задача понимания нового фрагмента базы знаний в контексте её текущего
                    состояния, что, прежде всего, требует обеспечения семантической совместимости
                    (согласования понятий) между базой знаний и интегрируемым новым фрагментом}
            \end{scnindent}
        \end{scnindent}
    \end{scnindent}
    \scnsuperset{задача управления эффекторами кибернетической системы при
        выполнении сложных воздействий на внешнюю среду и/или физическую оболочку этой
        кибернетической системы}
    \begin{scnindent}
        \scnidtf{задача целенаправленной сенсорно-эффекторной (в частности, сенсомоторной) координации}
    \end{scnindent}

    \scnheader{сенсорная информация}
    \scnidtf{информация, генерируемая непосредственно некоторой группой
        (конфигурацией) сенсоров (рецепторов) кибернетической системы}
    \scnidtf{рецепторная информация}
    \scnidtf{первичная информация, получаемая (приобретаемая) кибернетической
        системой}
    \scnidtf{первичная знаковая конструкция, которая описывает те или иные свойства
        текущего состояния физической окружающей среды (внешней среды и физической
        оболочки) кибернетической системы}

    \scnheader{сенсор кибернетической системы}
    \scnidtf{рецептор кибернетической системы}
    \scntext{пояснение}{Компонент кибернетической системы, генерирующий в памяти
        этой системы информацию о текущем значении соответствующего этому компоненту
        свойства (характеристики, параметра) того фрагмента физической окружающей среды
        кибернетической системы, который непосредственно смежен (пограничен) указанному
        компоненту.}
        
    \scnheader{эффектор кибернетической системы}
    \scnidtf{компонент кибернетической системы, который способен менять своё
        состояние в целях непосредственного воздействия на свою физическую оболочку и
        на внешнюю среду}

    \scnheader{способность кибернетической системы к пониманию сенсорной
        информации}
    \scnidtf{способность к синтаксическому и семантическому анализу информации,
        формируемой сенсорами кибернетической системы, а также к погружению  этой
        информации в состав общей информационной модели внешней среды кибернетической
        системы (в состав общей картины внешнего мира)}
    \scnidtf{способность кибернетической системы к переходу от первичной
        (сенсорной) информации ко вторичной информации, которая описывает связи между
        вторичными объектами, каждый из которых представлен (описан) в первичной
        информации конфигурацией знаков своих частей с дополнительным описанием свойств
        каждой из этих частей}

    \scnheader{способность кибернетической системы к самостоятельной
        деятельности во внешней среде}
    \begin{scnrelfromlist}{свойство-предпосылка}
        \scnitem{уровень развития эффекторов, обеспечивающих самостоятельное
            перемещение кибернетической системы}
        \begin{scnindent}
            \begin{scnrelfromlist}{частное свойство}
                \scnitem{уровень развития эффекторов, обеспечивающих локальное перемещение
                    сенсоров кибернетической системы}
                \scnitem{уровень развития эффекторов, обеспечивающих функционирование
                    манипуляторов кибернетической системы}
                \scnitem{уровень развития эффекторов, обеспечивающих перемещение всей
                    физической оболочки кибернетической системы}
            \end{scnrelfromlist}
        \end{scnindent}
        \scnitem{качество управления поведением кибернетической системы во внешней
            среде}
        \begin{scnindent}
            \scnidtf{качество сенсорно-эффекторной координации действий
                кибернетической системы при выполнении сложных действий во внешней среде}
        \end{scnindent}
    \end{scnrelfromlist}
    \bigskip

\end{scnsubstruct}
\scnsourcecomment{Завершили Сегмент \scnqqi{Комплекс свойств, определяющих качество решателя задач кибернитической системы}}