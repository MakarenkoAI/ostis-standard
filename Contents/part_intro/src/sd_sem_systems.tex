\begin{SCn}
	\scnsectionheader{Предметная область и онтология интеллектуальных компьютерных систем нового поколения}
	\begin{scnsubstruct}

		\begin{scnrelfromlist}{дочерний раздел}
			\scnitem{\nameref{sd_sem_inf_rep}}
			\scnitem{\nameref{sd_agent_solvers}}
			\scnitem{\nameref{sd_sem_ui}}
		\end{scnrelfromlist}

		\scntext{аннотация}{Данный раздел и дочерние ему разделы являются
			уточнением и обоснованием наших предложений, направленных на построение
			компьютерных систем следующего поколения, основанных на смысловом представлении
			обрабатываемой информации. 
			\\В предметной области рассмотрены принципы построения интеллектуальных компьютерных систем нового поколения. В
			качестве ключевых свойств интеллектуальных систем нового поколения выделяются их самообучаемость,
			интероперабельность и семантическая совместимость. В главе рассматривается подход к обеспечению ука
			занных свойств на основе смыслового представления информации и многоагентных моделей обработки
			информации.}
		\scntext{основной тезис}{Для \uline{любой} \textit{компьютерной
				системы} можно построить эквивалентную ей логико-семантическую модель,
			основанную на смысловом представлении обрабатываемой информации}

			\begin{scnrelfromlist}{ключевой знак}
				\scnitem{Технология OSTIS}
				\scnitem{УСК}
			\end{scnrelfromlist}

			\begin{scnrelfromlist}{ключевое понятие}
				\scnitem{интеллектуальная компьютерная система нового поколения}
				\scnitem{интероперабельная интеллектуальная компьютерная систем}
				\scnitem{самообучаемая интеллектуальная компьютерная система}
				\scnitem{семантическая сеть}
				\scnitem{многоагентная система обработки информации в общей памяти}
			\end{scnrelfromlist}

		\scnheader{логико-семантическая модель компьютерной системы}
		\scntext{пояснение}{Главным фактором обеспечения совместимости
			различных видов знаний, различных моделей решения задач и различных
			компьютерных систем в целом является
			\begin{scnitemize}
				\item унификация (стандартизация) представления информации в памяти
				компьютерных систем;
				\item унификация принципов организации обработки информации в памяти
				компьютерных систем.
			\end{scnitemize}
			Унификация представления информации, используемой в компьютерных
			системах, предполагает:
			\begin{scnitemize}
				\item синтаксическую унификацию используемой информации  унификацию
				формы представления (кодирования) этой информации. При этом следует отличать:
				\begin{scnitemizeii}
					\item кодирование информации в памяти компьютерной системы (внутреннее
					представление информации);
					\item внешнее представление информации, обеспечивающее однозначность
					интерпретации (понимания, трактовки) этой информации разными пользователями и
					разными компьютерными системами;
				\end{scnitemizeii}
				\item семантическую унификацию используемой информации, в основе
				которой лежит согласование и точная спецификация всех (!) используемых понятий
				(концептов) с помощью иерархической системы формальных онтологий.
			\end{scnitemize}}

		\scnheader{стандарт}
		\scnhaselement{Стандарт OSTIS}
		\begin{scnindent}
			\scnidtf{Предлагаемый нами стандарт логико-семантических моделей
				компьютерных систем,  основанных на смысловом представлении информации, и
				технологии разработки таких моделей и соответствующих компьютерных систем}
		\end{scnindent}
		\scnidtf{знания о структуре и принципах функционирования искусственных
			систем соответствующего класса}
		\scnidtf{онтология искусственных систем некоторого класса}
		\scnidtf{теория искусственных систем некоторого класса}
		\scntext{пояснение}{Важно отметить, что грамотная унификация
			(стандартизация) должна не ограничивать творческую свободу разработчика, а
			гарантировать \uline{совместимость} его результатов с результатами других
			разработчиков. Подчеркнем также, что текущая версия любого \textit{стандарта}
			-- это не догма, а только опора для дальнейшего его совершенствования.Целью
			качественного \textit{стандарта} является не только обеспечения совместимости
			технических решений, но и минимизация дублирования (повторения) таких решений.
			Один из важных критериев качества \textit{стандарта} --- ничего
			лишнего.\textit{Стандарты}, как и другие важные для человечества
			\textit{знания}, должны быть формализованы и должны постоянно
			совершенствоваться с помощью специальных \textit{интеллектуальных компьютерных
				систем}, поддерживающих процесс эволюции стандартов путем согласования
			различных точек зрения.}
			
		\scnheader{семантическая совместимость компьютерных систем}
		\scntext{пояснение}{Уровень совместимости \textit{компьютерных
				систем} определяется трудоемкостью реализации процедур интеграции (объединения,
			соединения знаний этих систем), а также трудоемкостью и глубиной интеграции
			входящих в эти системы \textit{решателей задач} (интеграции навыков и
			интерпретаторов этих навыков). Подчеркнем при этом, что интеграция интеграции
			рознь --- от эклектики до гибридности и синергетичности дистанция огромного
			размера.
			
			Совместимые \textit{компьютерные системы} --- это компьютерные системы,
			для которых существует автоматически выполняемая процедура их интеграции,
			превращающая эти системы в единую \textit{гибридную систему}, в рамках которой
			каждая интегрируемая компьютерная система в процессе своего функционирования
			может свободно использовать любые необходимые информационные ресурсы (знания и
			навыки), входящие в состав другой интегрируемой компьютерной системы.
			
			Целостную \textit{компьютерную систему} можно рассматривать как решатель задач,
			интегрировавший несколько моделей решения задач и обладающий средствами
			взаимодействия с внешней для себя средой (с другими компьютерными системами, с
			пользователями).
			
			Таким образом, для того, чтобы повысить уровень совместимости
			\textit{компьютерных систем}, необходимо преобразовать их к виду
			\textit{многоагентных систем}, работающих над общей семантической памятью.
			Такие \textit{компьютерные системы} не всегда целесообразно непосредственно
			объединять (интегрировать) в более крупные \textit{компьютерные системы}.
			Иногда целесообразнее их объединять в \textit{коллективы взаимодействующих
			компьютерных систем}. Но при создании таких коллективов компьютерных систем
			унификация и совместимость таких систем также очень важны, т.к. существенно
			упрощают обеспечение высокого уровня их взаимопонимания. Так, например,
			противоречия между компьютерными системами, входящими в коллектив, можно
			обнаруживать путем анализа на непротиворечивость \textit{виртуальной
			объединенной базы знаний} этого коллектива. Более того, непротиворечивость
			указанной виртуальной базы знаний можно считать одним из критериев
			семантической совместимости систем, входящих в соответствующий
			коллектив.}
			
		\scnheader{компьютерная система, основанная на смысловом представлении информации}
		\scntext{пояснение}{Предлагаемое нами устранение проблем современных
			информационных технологий путем перехода к \textit{смысловому представлению
				информации} в памяти компьютерных систем фактически преобразует современные
			компьютерные системы (в том числе и современные интеллектуальные компьютерные
			системы) в \textit{компьютерные системы, основанные на смысловом представлении
				информации}, которые являются не альтернативной ветвью развития
			\textit{компьютерных систем}, а естественным этапом их эволюции, направленным
			на обеспечение высокого уровня их \textit{обучаемости} и, в первую очередь,
			\textit{совместимости}.
			
			Архитектура \textit{компьютерных систем, основанных на
			смысловом представлении информации} (см. \textit{Рис. Архитектура компьютерных
			систем, основанных на смысловом представлении информации}) практически
			совпадает с архитектурой \textit{интеллектуальных компьютерных систем},
			основанных на знаниях. Отличие здесь заключаются в том, что в
			\textit{компьютерных системах, основанных на смысловом представлении
			информации}:
			\begin{scnitemize}
				\item база знаний имеет смысловое представление;
				\item интерпретатор знаний и навыков представляет собой коллектив
				\textit{агентов}, осуществляющих обработку \textit{базы знаний}.
			\end{scnitemize}
			Как следствие этого, \textit{компьютерные системы, основанная на
			смысловом представлении информации}, обладают высоким уровнем
			\textit{обучаемости}, т.е. способностью быстро приобретать новые и
			совершенствовать уже приобретенные знания и навыки и при этом не иметь никаких
			ограничений на вид приобретаемых и совершенствуемых ею знаний и навыков, а
			также на их совместное использование.
			
			Более того, при согласовании соответствующих стандартов, а также при перманентном совершенствовании этих
			стандартов и при грамотной их поддержке в условиях интенсивной эволюции как
			самих стандартов, так и \textit{компьютерных систем, основанных на смысловом
			представлении информации} (речь идет о постоянной поддержке соответствия между
			текущим состоянием компьютерных систем и текущим состоянием эволюционируемых
			стандартов), \textit{компьютерные системы, основанные на смысловом
			представлении информации} и их компоненты обладают весьма высокой степенью
			\textit{совместимости}.
			
			Это, в свою очередь, практически исключает дублирование
			инженерных решений и дает возможность существенно ускорить разработку
			\textit{компьютерных систем, основанных на смысловом представлении информации}
			с помощью постоянно расширяемой библиотеки многократно используемых и
			совместимых между собой компонентов. 
			
			Основным лейтмотивом перехода от современных компьютерных систем (в том числе интеллектуальных) к
			\textit{компьютерным системам, основанным на смысловом представлении
				информации}, хранимой в ее памяти, является создание \textbf{\textit{общей
					семантической теории компьютерных систем}}, включающей в себя:
			\begin{scnitemize}
				\item cемантическую теорию \textit{знаний} и \textit{баз знаний};
				\item семантическую теорию \textit{задач} и \textit{моделей решения
					задач};
				\item cемантическую теорию \textit{взаимодействия информационных
					процессов};
				\item cемантическую теорию пользовательских и, в том числе,
				естественно-языковых интерфейсов;
				\item cемантическую теорию невербальных (сенсорно-эффекторных)
				интерфейсов;
				\item теорию универсальных интерпретаторов \textit{логико-семантических
					моделей компьютерных систем} и, в частности, теорию семантических компьютеров.
			\end{scnitemize}
			Эпицентром следующего этапа развития информационных технологий является
			решение проблемы обеспечения \textbf{\textit{семантической совместимости}}
			\textit{компьютерных систем} и их компонентов. Для решения этой проблемы
			необходим
			\begin{scnitemize}
				\item переход от традиционных компьютерных систем и от современных
				интеллектуальных компьютерных систем к \textit{компьютерным системам,
					основанным на смысловом представлении информации};
				\item разработка \textit{стандарта компьютерных систем, основанных на
					смысловом представлении информации}.
			\end{scnitemize}
			Очевидно, что \textit{компьютерные системы, основанных на смысловом
				представлении информации} являются компьютерными системами нового поколения,
			устраняющие многие недостатки современных компьютерных систем. Но для массовой
			разработки таких систем необходима соответствующая технология, которая должна
			включать в себя
			\begin{scnitemize}
				\item теорию \textit{компьютерных систем, основанных на смысловом
					представлении информации} и комплекс всех стандартов, обеспечивающих
				совместимость разрабатываемых систем;
				\item методы и средства проектирования \textit{компьютерных систем,
					основанных на смысловом представлении информации};
				\item методы и средства перманентного совершенствования самой
				технологии.
			\end{scnitemize}}
			\begin{scnindent}
				\scnrelfrom{иллюстрация}{\scnfileimage{Contents/part_intro/src/images/arch.pdf}}
					\begin{scnindent}
						\scnidtf{Рис. Архитектура компьютерных систем, \textit{основанных на смысловом представлении информации}}
					\end{scnindent}
			\end{scnindent}
		\bigskip

		\scnheader{уровень интеллекта индивидуальных интеллектуальных систем}
		\scntext{примечание}{Важнейшим направлением повышения уровня интеллекта индивидуальных интеллектуальных кибернетических
			систем является переход к коллективам индивидуальных интеллектуальных кибернетических систем и далее к
			иерархическим коллективам интеллектуальных кибернетических систем, членами которых являются как 
			индивидуальные интеллектуальные кибернетические системы, так и коллективы индивидуальных интеллектуальных
			кибернетических систем, а также иерархические коллективы интеллектуальных кибернетических систем.}
		\scntext{примечание}{Аналогичным образом необходимо повышать уровень интеллекта и индивидуальных интеллектуальных \uline{компьютерных}
			систем (искусственных кибернетических систем). Но при этом надо помнить, что далеко не каждое объединение
			\uline{интеллектуальных} кибернетических систем (в том числе и компьютерных систем) становится интеллектуальным
			коллективом. Для этого необходимо соблюдение дополнительных требований, предъявляемых \uline{ко всем} членам
			интеллектуальных коллективов. Важнейшим из них является требование высокого уровня интероперабельности,
			то есть способности к эффективному взаимодействию с другими членами коллектива. Переход от современных 
			интеллектуальных компьютерных систем к интероперабельным интеллектуальным компьютерным системам является
			ключевым фактором перехода к интеллектуальным компьютерным системам нового поколения, обеспечивающим
			существенное повышение уровня автоматизации человеческой деятельности.}

		\scnheader{комплексная задача}
		\scnidtf{задача, решение которой невозможно с помощью одной модели решения задач, одного вида знаний, одной интеллектуальной компьютерной системы}
		\scntext{примечание}{Расширение областей применения интеллектуальных компьютерных систем требует перехода к решению \uline{комплексных} задач.}
		\scntext{примечание}{Решение:
		\begin{scnitemize}
			\item{Переход к гибридным индивидуальным интеллектуальным компьютерным системам, в которых осуществляется конвергенция и интеграция различных моделей решения задач и различных видов знаний.}
			\item{Переход к \uline{коллективам} семантически совместимых самостоятельных интеллектуальных компьютерных систем, в которых обеспечивается:
			\begin{scnitemizeii}
				\item{интероперабельность объединяемых интеллектуальных компьютерных систем}
				\item{конвергенция объединяемых интеллектуальных компьютерных систем при сохранении их самостоятельности.}
			\end{scnitemizeii}}
		\end{scnitemize}}

		\scnsegmentheader{Предметная область и онтология требований, предъявляемых к интеллектуальным компьютерным системам нового поколения}

\begin{scnsubstruct}

    \begin{scnrelfromlist}{ключевое понятие}
    	\scnitem{интеллектуальная компьютерная система нового поколения}
    	\scnitem{интероперабельная интеллектуальная компьютерная система}
    	\scnitem{гибридная интеллектуальная компьютерная система}
    \end{scnrelfromlist}
    
    \begin{scnrelfromlist}{ключевое отношение}
    	\scnitem{соединение интеллектуальных компьютерных систем*}
    	\begin{scnindent}
    		\scnidtf{преобразование множества интеллектуальных компьютерных систем в коллектив, членами (агентами) которого являются эти системы*}
    	\end{scnindent}
    	\scnitem{глубокая интеграция интеллектуальных компьютерных систем*}
    	\begin{scnindent}
    		\scnidtf{быть результатом преобразования множества индивидуальных интеллектуальных компьютерных систем в одну интегрированную индивидуальную интеллектуальную компьютерную систему*}
    	\end{scnindent}
    \end{scnrelfromlist}
    
    \begin{scnrelfromlist}{ключевой параметр}
    	\scnitem{интероперабельность интеллектуальных компьютерных систем\scnsupergroupsign}
    	\scnitem{семантическая совместимость пар интеллектуальных компьютерных систем\scnsupergroupsign}
    \end{scnrelfromlist}
    
    \begin{scnrelfromlist}{ключевое знание}
    	\scnitem{Требования, предъявляемые к интеллектуальным компьютерным системам нового поколения}
    	\scnitem{Принципы, лежащие в основе интеллектуальных компьютерных систем нового поколения}
    	\scnitem{Отличие данных от знаний}
    \end{scnrelfromlist}

    \scnheader{уровень интероперабельности интеллектуальных компьютерных систем}
    \scntext{примечание}{Создание различных комплексов взаимодействующих интеллектуальных компьютерных систем \uline{требует} повышения качества не только самих этих систем, но также и качества их взаимодействия. Интеллектуальные компьютерные системы нового поколения должны иметь высокий уровень интероперабельности.}
    \scnidtf{уровень коммуникационной (социальной) совместимости интеллектуальных компьютерных систем, позволяющей им самостоятельно формировать коллективы интеллектуальных компьютерных систем и их пользователей, а также самостоятельно согласовывать и координировать свою деятельность в рамках этих коллективов при решении сложных задач в частично предсказуемых условиях}
    \scnidtf{уровень способности к эффективному, целенаправленному взаимодействию с себе подобными и с пользователями в процессе коллективного распределенного) и децентрализованного решения сложных задач}
    \scnidtf{уровень \scnqq{социализации} интеллектуальных компьютерных систем, полезности в рамках различных априори неизвестных сообществ (коллективов) \textit{интеллектуальных систем}}
    \scntext{примечание}{Повышение уровня \textit{интероперабельности} интеллектуальных компьютерных систем определяет переход к \textbf{\textit{интеллектуальным компьютерным системам нового поколения}}, без которых невозможна реализация таких проектов, как \textit{интеллектуальное-предприятие}, \textit{интеллектуальная-больница}, \textit{интеллектуальная-школа}, \textit{интеллектуальный-университет}, \textit{интеллектуальная-кафедра}, \textit{интеллектуальный-дом}, \textit{интеллектуальный-город}, \textit{интеллектуаль\-ное-общество}.}
    
    
    \scnheader{интеллектуальная компьютерная система}
    \scnidtf{интеллектуальная искусственная кибернетическая система}
    \begin{scnrelfromset}{разбиение}
    	\scnitem{индивидуальная интеллектуальная компьютерная система}
    	\scnitem{интеллектуальный коллектив интеллектуальных компьютерных систем}
    	\begin{scnindent}
    		\scnidtf{интеллектуальная \textit{многоагентная система}, агенты которой являются \textit{интеллектуальными компьютерными системами}}
    		\scntext{примечание}{Не каждый \textit{коллектив интеллектуальных компьютерных систем} может оказаться интеллектуальным, поскольку уровень интеллекта такого коллектива определяется не только уровнем интеллекта его членов, но также и эффективностью (качеством) \uline{их взаимодействия}.}
    		\begin{scnrelfromset}{разбиение}
    			\scnitem{интеллектуальный коллектив \uline{индивидуальных} интеллектуальных компьютерных систем}
    			\scnitem{иерархический интеллектуальный коллектив интеллектуальных компьютерных систем}
    			\begin{scnindent}
    				\scnidtf{\textit{интеллектуальный коллектив интеллектуальных компьютерных систем}, по крайней мере одним из членов которого является \textit{интеллектуальный коллектив интеллектуальных компьютерных систем}}
    			\end{scnindent}
    		\end{scnrelfromset}
    	\end{scnindent}
    \end{scnrelfromset}
    
    \scnheader{интеллектуальные компьютерные системы нового поколения}
    \begin{scnrelfromlist}{предъявляемые требования}
    	\scnitem{высокий уровень \textit{интероперабельности}}
    	\scnitem{высокий уровень \textit{обучаемости}}
    	\scnitem{высокий уровень \textit{гибридности}}
    	\scnitem{высокий уровень способности решать \textit{интеллектуальные задачи}}
        \begin{scnindent}
            {\textit{задачи}, \textit{методы} решения которых и/или требуемая для их решения исходная информация априори неизвестны}
        \end{scnindent}
        \scnitem{высокий уровень \textit{синергетичности}}
    \end{scnrelfromlist}
    
    \scnheader{интероперабельность\scnsupergroupsign}
    \scnidtf{способность к эффективному (целенаправленному) взаимодействию с другими самостоятельными субъектами}
    \scnidtf{способность к партнерскому взаимодействию в решении \textit{комплексных задач}, требующих \textit{коллективной деятельности}}
    \scnidtf{способность работать в коллективе (в команде)}
    \scnidtf{уровень социализации}
    \scnidtf{social skills}
    
    \scnheader{высокий уровень интероперабельности}
    \begin{scnrelfromlist}{обеспечивается}
    	\scnitem{высоким уровнем \textit{взаимопонимания}}
    	\begin{scnindent}
    		\begin{scnrelfromlist}{обеспечивается}
    			\scnitem{высоким уровнем \textbf{\textit{семантической совместимости}} заданного субъекта с другими субъектами заданного коллектива}
    			\scnitem{высоким уровнем \textit{способности понимать} сообщения и поведение партнеров}
    			\scnitem{высоким уровнем \textit{способности быть понятной} для партнеров:}
                \begin{scnindent}
                    \begin{scnrelfromlist}{обеспечивается}
                        \scnitem{способностью понятно и обоснованно формулировать свои предложения и информацию, полезную для решения текущих задач}
    			        \scnitem{способностью действовать и комментировать свои действия так, чтобы они и их мотивы были понятны партнерам}
                \end{scnrelfromlist}
                \end{scnindent}
    			\scnitem{высоким уровнем \textit{способности к повышению уровня семантической совместимости} со своими партнерами}
    		\end{scnrelfromlist}
    	\end{scnindent}
    	\scnitem{высоким уровнем \textit{договороспособности}, то есть способности согласовывать с партнерами свои планы и намерения в целях своевременного обеспечения высокого качества коллективного результата}
    	\scnitem{высоким уровнем \textit{способности к децентрализованной координации} своих действий с действиями партнеров в непредсказуемых (нештатных) обстоятельствах}
    	\scnitem{высоким уровнем способности разделять ответственность с партнерами}
    	\scnitem{высоким уровнем \textit{способности к минимизации негативных последствий конфликтных ситуаций} с другими субъектами}
    	\begin{scnindent}
    		\begin{scnrelfromlist}{обеспечивается}
    			\scnitem{высоким уровнем \textit{способности к предотвращению возникновения конфликтных ситуаций}}
    			\scnitem{\textit{соблюдением этических норм} и правил, препятствующих возникновению разрушительных последствий конфликтных ситуаций}
    			\scnitem{высоким уровнем \textit{способности разделять ответственность} с партнерами за своевременное и качественное достижение общей цели}
    		\end{scnrelfromlist}
    	\end{scnindent}
    \end{scnrelfromlist}
    
    \scnheader{семантическая совместимость\scnsupergroupsign}
    \scnidtf{степень согласованности (совпадения) систем \textit{понятий} и других \textit{ключевых знаков}, используемых заданными взаимодействующими субъектами}
    \scntext{примечание}{Обеспечение \textit{семантической совместимости} требует формализации \textit{смыслового представления информации}.}
    
    \scnheader{способность разделять ответственность с партнёрами}
    \scnidtf{необходимое условие децентрализованного управления коллективной деятельностью}
    \begin{scnrelfromlist}{обеспечивается}
    	\scnitem{\textit{способностью к мониторингу} и анализу коллективно выполняемой деятельности}
    	\scnitem{\textit{способностью оперативно информировать партнеров} о неблагоприятных ситуациях, событиях, тенденциях, а также инициировать соответствующие коллективные действия}
    \end{scnrelfromlist}
    
    \scnheader{высокий уровень обучаемости интеллектуальной компьютерной системы нового поколения}
    \scnexplanation{Важнейшим направлением повышения уровня автоматизации человеческой деятельности является повышение уровня автоматизации не только проектирования интеллектуальной компьютерной системы, но и комплексной поддержки всех остальных этапов жизненного цикла \textit{интеллектуальной компьютерной системы}. В частности, это касается модернизации (совершенствования, реинжиниринга) интеллектуальной компьютерной системы непосредственно в ходе их эксплуатации. Для того, чтобы обеспечить высокий уровень автоматизации такой модернизации, необходимо существенно повысить \textbf{\textit{уровень самообучаемости}} \textit{интеллектуальной компьютерной системы} для того, что они сами (самостоятельно) могли себя модернизировать (самосовершенствовать) в ходе своего целевого функционирования.}
    
    \scnheader{высокий уровень обучаемости}
    \begin{scnrelfromlist}{обеспечивается}
    	\scnitem{высоким уровнем \textit{гибкости информации}, хранимой в памяти интеллектуальной системы}
    	\scnitem{высоким уровнем \textit{качества} \textit{стратификации информации}, хранимой в памяти интеллектуальной системы (стратифицированностью \textit{базы знаний})}
    	\scnitem{высоким уровнем \textit{рефлексивности} интеллектуальной системы}
    	\scnitem{высоким уровнем \textit{способности исправлять свои ошибки} (в том числе устранять противоречия в своей \textit{базе знаний})}
    	\scnitem{высоким уровнем \textit{познавательной активности}}
    	\scnitem{низким уровнем \textit{ограничений на вид приобретаемых знаний и навыков} (отсутствие таких ограничений означает потенциальную \textit{универсальность} интеллектуальной системы и предполагает высокий уровень ее гибридности)}
    \end{scnrelfromlist}
    
    \scnheader{обучаемость\scnsupergroupsign}
    \scnidtf{способность быстро и качественно приобретать новые \textit{знания} и \textit{навыки}, а также совершенствовать уже приобретенные \textit{знания} и \textit{навыки}}
    
    \scnheader{гибридность\scnsupergroupsign}
    \scnidtf{степень многообразия используемых \textit{видов знаний} и \textit{моделей решения задач} и уровень эффективности их совместного использования}
    \scnidtf{индивидуальная способность решать \textit{комплексные задачи}, требующие использования различных \textit{видов знаний}, а также различных комбинаций различных \textit{моделей решения задач}}
    \scntext{пояснение}{\textit{Гибридность} и \textit{интероперабельность} \textit{интеллектуальных компьютерных систем нового поколения} предполагает отказ от известной парадигмы \scnqq{черных ящиков}, поскольку:
    \begin{scnitemize}
        \item все многообразие моделей решения задач \textit{гибридной интеллектуальной компьютерной системы} должно интерпретироваться на одной общей \textit{универсальной платформе};
    	\item
    	доступность информации о том, как устроен каждый используемый метод, модель решения задач, каждый субъект существенно повышает качество их \textit{координации} при \textit{совместном решении комплексных задач};
    	\item
    	появляется возможность некоторые методы, модели решения задач и целые субъекты (например, \textit{интеллектуальные компьютерные системы}) использовать для совершенствования (повышения качества) других методов, моделей и субъектов.
    \end{scnitemize}}
    
    \scnheader{высокий уровень гибридности}
    \begin{scnrelfromlist}{обеспечивается}
    	\scnitem{высокой степенью многообразия используемых \textit{видов знаний} и \textit{моделей решения задач}}
    	\scnitem{высокой степенью \textit{конвергенции} и глубокой \textit{интеграции} (степенью взаимопроникновения) различных \textit{видов знаний} и \textit{моделей решения задач}}
    	\scnitem{способностью неограниченно расширять уровень своей \textit{гибридности}}
    \end{scnrelfromlist}
    
    \scnheader{характеристики \textit{интеллектуальных компьютерных систем нового поколения}}
    \begin{scnhassubset}
        \scnfileitem{\textbf{\textit{Степень}} \textbf{\textit{конвергенции}}, унификации и стандартизации \textit{интеллектуальных компьютерных систем} и их компонентов и соответствующая этому \textbf{\textit{степень интеграции}} (глубина интеграции) \textit{интеллектуальных компьютерных систем} и их компонентов.}
        \scnfileitem{\textbf{\textit{Семантическая совместимость}} между \textit{интеллектуальными компьютерными системами} в целом и \textit{семантическая совместимость} между компонентами каждой \textit{интеллектуальной компьютерной системы} (в частности, совместимость между различными \textit{видами знаний} и различными \textit{моделями обработки знаний}), которые являются основными показателями степени \textbf{\textit{конвергенции}} (сближения) между \textit{интеллектуальными компьютерными системами} и их компонентами.}
    \end{scnhassubset}
    \scntext{пояснение}{Особенность указанных характеристик \textit{интеллектуальных компьютерных систем} их компонентов заключается в том, что они играют важную роль при решении всех ключевых задач современного этапа развития \textit{Искусственного интеллекта} и тесно связаны друг с другом.}
    \scntext{пояснение}{Перечисленные требования, предъявляемые к \textit{интеллектуальным компьютерным системам нового поколения}, направлены на преодоление проклятия \textit{вавилонского столпотворения} как внутри \textit{интеллектуальных компьютерных систем нового поколения} (между внутренними \textit{информационными процессами} решения различных задач), так и между взаимодействующими самостоятельными \textit{интеллектуальными компьютерными системами нового поколения} в процессе коллективного решения \textit{комплексных задач}.}
    
    \scnheader{интеллектуальная компьютерная система нового поколения}
    \scntext{примечание}{На современном этапе эволюции \textit{интеллектуальных компьютерных систем} для существенного расширения областей их применения и качественного повышения уровня автоматизации человеческой деятельности:
        \begin{scnitemize}
            \item{необходим переход к созданию \uline{семантически совместимых} \textbf{интеллектуальных компьютерных систем \uline{нового поколения}}, ориентированных не только на индивидуальное, но и на \uline{коллективное} (совместное) решение \textit{комплексных задач}, требующих скоординированной деятельности нескольких самостоятельных интеллектуальных компьютерных систем и использования различных моделей и методов в непредсказуемых комбинациях, что необходимо для существенного расширения сфер применения \textit{интеллектуальных компьютерных систем}, для перехода от автоматизации локальных видов и областей \textit{человеческой деятельности} к комплексной автоматизации более крупных (объединенных) видов и областей этой деятельности;}
            \item{необходима разработка \textbf{Общей формальной теории и стандарта интеллектуальных компьютерных систем нового поколения};}
            \item{необходима разработка \textbf{Технологии комплексной поддержки жизненного цикла интеллектуальных компьютерных систем нового поколения}, которая включает в себя поддержку \textit{проектирования} этих систем (как начального этапа их жизненного цикла) и обеспечение их \textit{совместимости} на всех этапах их жизненного цикла;}
            \item{необходима \textbf{конвергенция} и \textbf{унификация} \textit{интеллектуальных компьютерных систем нового поколения} и их компонентов;}
            \item{необходима реализация \scnqq{бесшовной}, {диффузной}, взаимопроникающей, \textbf{глубокой интеграции семантически смежных компонентов интеллектуальных компьютерных систем}, то есть интеграции, при которой отсутствуют четкие границы (\scnqq{швы}) интегрируемых (соединяемых) компонентов, и которая может осуществляться \uline{автоматически}. Это означает переход к \textbf{\uline{гибридным} интеллектуальным компьютерным системам};}
            \item{необходимо соблюдение \textbf{Принципа бритвы Оккама} — максимально возможное структурное упрощение \textit{интеллектуальных компьютерных систем нового поколения}, исключение \uline{эклектичных} решений;}
            \item{необходима ориентация на потенциально \textbf{универсальные} (то есть способные быстро приобретать \uline{любые} знания и навыки), \textbf{синергетические} \textit{интеллектуальные компьютерные системы} с \scnqq{сильным} интеллектом}
        \end{scnitemize}}
    \begin{scnrelfromlist}{принципы, лежащие в основе}
        \scnfileitem{\textit{смысловое представление знаний} в памяти \textit{интеллектуальных компьютерных систем}, предполагающее отсутствие \textit{омонимических знаков}, которые в разных контекстах обозначают разные сущности, а также отсутствие \textit{синонимии}, то есть пар синонимичных \textit{знаков}, которые обозначают одну и ту же сущность}
        \scnfileitem{смысловое представление информационной конструкции в общем случае имеет нелинейный (графовый) характер представления информации, который является \textit{рафинированной семантической сетью}}
        \scnfileitem{фрактальный характер (масштабируемое самоподобие) структуризации представляемых знаний в базах знаний}
        \scnfileitem{использование \uline{общего} для всех интеллектуальных компьютерных систем \textit{универсального языка смыслового представления знаний} в памяти \textit{интеллектуальных компьютерных систем}, обладающий максимально простым \textit{синтаксисом}, обеспечивающий представление любых \textit{видов знаний} и имеющий неограниченные возможности перехода от \textit{знаний} к \textit{метазнаниям}. Простота синтаксиса \textit{информационных конструкций} указанного \textit{языка} позволяет называть эти конструкции \textit{рафинированными семантическими сетями}}
        \scnfileitem{\textit{структурно-перестраиваемая (графодинамическая) организация памяти} интеллектуальных компьютерных систем, при которой обработка знаний сводится не столько к изменению состояния хранимых \textit{знаков}, сколько к изменению конфигурации связей между этими \textit{знаками}}
        \scnfileitem{\textit{семантически неограниченный ассоциативный доступ к информации}, хранимой в памяти \textit{интеллектуальных компьютерных систем}, по заданному образцу произвольного размера и произвольной конфигурации}
        \scnfileitem{универсальная ситуационная многоагентная модель обработки знаний, ориентированная на обработку смыслового представления информации в ассоциативной графодинамической памяти, \textit{децентрализованное ситуационное управление информационными процессами} в памяти \textit{интеллектуальных компьютерных систем}, реализованное с помощью \textit{агентно-ориентированной модели обработки баз знаний}, в котором \textit{инициирование} новых \textit{информационных процессов} осуществляется не путем передачи управления соответствующим априори известным процедурам, а в результате возникновения соответствующих \textit{ситуаций} или \textit{событий} \textit{в памяти интеллектуальной компьютерной системы}, поскольку \scnqqi{основная проблема компьютерных систем состоит не в накоплении знаний, а в умении активизировать нужные знания в процессе решения задач} (Поспелов Д.~А.). Такой многоагентный процесс обработки информации представляет собой \textit{деятельность}, выполняемую некоторым коллективом \uline{самостоятельных} \textit{информационных агентов} (агентов обработки информации), условием инициирования каждого из которых является появление в текущем состоянии \textit{базы знаний} соответствующей этому агенту \textit{ситуации} и/или \textit{события}.
            \scnqqi{Выбор многоагентных технологий объясняется тем, что в настоящее время любая сложная производственная, логистическая или другая система может быть представлена набором взаимодействий более простых систем до любого уровня детальности, что обеспечивает фрактально-рекурсивный принцип построения многоярусных систем, построенных как открытые цифровые колонии и экосистемы ИИ. В основе многоагентных технологий лежит распределенный или децентрализованный подход к решению задач, при котором динамически обновляющаяся информация в распределенной сети интеллектуальных агентов обрабатывается непосредственно у агентов вместе с локально доступной информацией от \scnqq{соседей}. При этом существенно сокращаются как ресурсные и временные затраты на коммуникации в сети, так и время на обработку и принятие решений в центре системы (если он все-таки есть).}}
        \scnfileitem{агентно-ориентированная модель обработки знаний в памяти интеллектуальной компьютерной системы, обеспечивающая высокую степень \textit{интероперабельности} между внутренними агентами индивидуальной интеллектуальной компьютерной системы, взаимодействующими через общую память (это, фактически, \scnqq{внутренняя} интероперабельность интеллектуальной компьютерной системы нового поколения)}
        \scnfileitem{Переход к \textit{семантическим} \textit{моделям решения задач}, в основе которых лежит учет не только синтаксических (структурных) аспектов обрабатываемой информации, но также и \uline{семантических} (смысловых) аспектов этой информации - \scnqqi{From data science to knowledge science}}
        \scnfileitem{\textbf{\textit{онтологическая модель баз знаний}} \textit{интеллектуальных компьютерных систем}, то есть онтологическая структуризация всей информации, хранимой в памяти \textit{интеллектуальной компьютерной системы}, предполагающая четкую \textit{стратификацию базы знаний} в виде иерархической системы \textit{предметных областей} и соответствующих им \textit{онтологий}, каждая из которых обеспечивает семантическую \textit{спецификацию} всех \textit{понятий}, являющихся ключевыми в рамках соответствующей \textit{предметной области}}
        \scnfileitem{\textbf{\textit{онтологическая локализация решения задач}} в \textit{интеллектуальных компьютерных системах}, предполагающая \uline{локализацию} \textit{области действия} каждого хранимого в памяти \textit{метода} и каждого \textit{информационного агента} в соответствии с \textit{онтологической моделью} обрабатываемой \textit{базы знаний}. Чаще всего, такой \textit{областью действия} является одна из \textit{предметных областей} либо одна из \textit{предметных областей} вместе с соответствующей ей \textit{онтологии}}
        \scnfileitem{\textbf{\textit{онтологическая модель интерфейса}} \textit{интеллектуальной компьютерной системы}}
        \begin{scnindent}
            \begin{scnrelfromlist}{входить в состав}
                \scnfileitem{онтологическое описание \textit{синтаксиса} всех языков, используемых \textit{интеллектуальной компьютерной системой} для общения с \textit{внешними субъектами}}
                \scnfileitem{онтологическое описание \textit{денотационной семантики} каждого языка, используемого \textit{интеллектуальной компьютерной системой} для \textit{общения} с внешними \textit{субъектами}}
                \scnfileitem{семейство \textit{информационных агентов}, обеспечивающих \textit{синтаксический анализ}, \textit{семантический анализ} (перевод на внутренний смысловой язык) и \textit{понимание} (погружение в \textit{базу знаний}) любого введенного \textit{сообщения}, принадлежащего любому \textit{внешнему языку}, полное онтологическое описание которого находится в базе знаний \textit{интеллектуальной компьютерной системы}}
                \scnfileitem{семейство \textit{информационных агентов}, обеспечивающих \textit{синтез сообщений}, которые (1) адресуются внешним субъектам, с которыми общается \textit{интеллектуальная компьютерная система}, (2) \textit{семантически эквивалентны} заданным \textit{фрагментам базы знаний} интеллектуальной компьютерной системы, определяющим \textit{смысл} передаваемых \textit{сообщений}, (3) принадлежат одному из \textit{внешних языков}, полное онтологическое описание которого находится в \textit{базе знаний} интеллектуальной компьютерной системы}
            \end{scnrelfromlist}
        \end{scnindent}
        \scnfileitem{\textit{семантически дружественный характер пользовательского интерфейса}, обеспечиваемый (1) формальным описание в базе знаний средства управления пользовательским интерфейсом и (2) введением в состав \textit{интеллектуальной компьютерной системы} соответствующих help-подсистем, обеспечивающих существенное снижение языкового барьера между пользователями и \textit{интеллектуальными компьютерными системами}, что существенно повысит эффективность \textit{эксплуатации интеллектуальных компьютерных систем}}
        \scnfileitem{\textit{минимизация негативного влияния человеческого фактора} на эффективность \textit{эксплуатации} \textit{интеллектуальных компьютерных систем} благодаря реализации интероперабельного (партнерского) стиля взаимодействия не только между самими \textit{интеллектуальными компьютерными системами}, но также и между \textit{интеллектуальными компьютерными системами} и их пользователями. Ответственность за качество совместной деятельности должно быть распределено между всеми партнерами}
        \scnfileitem{\textbf{\textit{мультимодальность}} (гибридный характер) \textit{интеллектуальной компьютерной системы}}
        \begin{scnindent}
            \begin{scnrelfromlist}{предполагает}
                \scnfileitem{многообразие \textit{видов знаний}, входящих в состав \textit{базы знаний} интеллектуальной компьютерной системы}
                \scnfileitem{многообразие \textit{моделей решения задач}, используемых \textit{решателем задач} интеллектуальной компьютерной системы}
                \scnfileitem{многообразие \textit{сенсорных каналов}, обеспечивающих \textit{мониторинг} состояния \textit{внешней среды} интеллектуальной компьютерной системы}
                \scnfileitem{многообразие \textit{эффекторов}, осуществляющих \textit{воздействие на внешнюю среду}}
                \scnfileitem{многообразие \textit{языков общения} с другими субъектами (с пользователями, с интеллектуальными компьютерными системами)}
            \end{scnrelfromlist}
        \end{scnindent}
        \scnfileitem{\textbf{\textit{внутренняя семантическая совместимость}} между компонентами \textit{интеллектуальной компьютерной системы} (то есть максимально возможное введение общих, совпадающих \textit{понятий} для различных фрагментов хранимой \textit{базы знаний}), являющаяся формой \textbf{\textit{конвергенции}} и \textit{глубокой интеграции} внутри \textit{интеллектуальной компьютерной системы} для различного вида \textit{знаний} и различных \textit{моделей решения задач}, что обеспечивает эффективную реализацию \textit{мультимодальности интеллектуальной компьютерной системы}}
        \scnfileitem{\textbf{\textit{внешняя семантическая совместимость}} между различными \textit{интеллектуальными компьютерными системами}, выражающаяся не только в общности используемых \textit{понятий}, но и в общности базовых \textit{знаний} и являющаяся необходимым условием обеспечения высокого уровня \textit{интероперабельности} интеллектуальных компьютерных систем}
        \scnfileitem{ориентация на использование \textit{интеллектуальных компьютерных систем} как \textit{когнитивных агентов} в составе \textbf{\textit{иерархических многоагентных систем}}}
        \scnfileitem{фрактальный характер (масштабируемое самоподобие) структуризации иерархических коллективов интеллектуальных компьютерных систем нового поколения}
        \scnfileitem{\textbf{\textit{платформенная независимость} интеллектуальных компьютерных систем}}
        \begin{scnindent}
            \begin{scnrelfromlist}{предполагает}
                \scnfileitem{четкую \textit{стратификацию} каждой \textit{интеллектуальной компьютерной системы} (1) на \textit{логико-семантическую модель}, представленную ее \textit{базой знаний}, которая содержит не только \textit{декларативные знания}, но и знания, имеющие \textit{операционную семантику}, и (2) на \textit{платформу}, обеспечивающую \textit{интерпретацию} указанной \textit{логико-семантической модели}}
                \scnfileitem{универсальность указанной \textit{платформы} интерпретации \textit{логико-семантической модели интеллектуальной компьютерной системы}, что дает возможность каждой такой \textit{платформе} обеспечивать интерпретацию любой \textit{логико-семантической модели интеллектуальной компьютерной системы}, если эта модель представлена на том же \textit{универсальном языке смыслового представления информации}}
                \scnfileitem{многообразие вариантов реализации \textit{платформ интерпретации логико-семантических моделей интеллектуальных компьютерных систем} — как вариантов, программно реализуемых на \textit{современных компьютерах}, так и вариантов, реализуемых в виде \textit{универсальных компьютеров нового поколения}, ориентированных на использование в \textit{интеллектуальных компьютерных системах нового поколения} (такие компьютеры мы назвали \textit{ассоциативными семантическими компьютерами})}
                \scnfileitem{легко реализуемую возможность переноса (переустановки) логико-семантической модели (\textit{базы знаний}) любой \textit{интеллектуальной компьютерной системы} на любую другую \textit{платформу интерпретации логико-семантических моделей}}
            \end{scnrelfromlist}
        \end{scnindent}
        \scnfileitem{изначальная ориентация \textit{интеллектуальных компьютерных систем нового поколения} на использование \textbf{\textit{универсальных ассоциативных семантических компьютеров}} (компьютеров нового поколения) в качестве \textit{платформы интерпретации логико-семантических моделей} (баз знаний) \textit{интеллектуальных компьютерных систем}}
    \end{scnrelfromlist}
    \scntext{примечание}{В настоящее время разработано большое количество различного вида моделей решения задач, моделей представления и обработки знаний различного вида. Но в разных \textit{интеллектуальных компьютерных системах} могут быть востребованы разные комбинации этих моделей. При разработке и реализации различных \textit{интеллектуальных компьютерных систем} соответствующие методы и средства должны гарантировать \textit{логико-семантическую совместимость} разрабатываемых компонентов и, в частности, их способность использовать общие \textit{информационные ресурсы}. Для этого, очевидно, необходима \textit{унификация} указанных моделей.}
    \scntext{примечание}{\uline{Многообразие} различных видов интеллектуальных компьютерных систем и, соответственно, многообразие используемых ими комбинаций моделей представления знаний и решения задач определяется:
        \begin{scnitemize}
            \item{многообразием назначения интеллектуальных компьютерных систем и вида окружающей их среды;}
            \item{многообразием различных видов хранимых знаний;}
            \item{многообразием моделей обработки знаний и решений задач;}
            \item{многообразием различных видов сенсорных и эффекторных подсистем.}
        \end{scnitemize}}

    \scnheader{аспекты \textit{совместимости} моделей представления и обработки знаний в \textit{интеллектуальных компьютерных системах}}
    \scnsuperset{синтаксический аспект}
    \scnsuperset{семантический аспект}
    \begin{scnindent}
    \scntext{примечание}{Cогласованность систем понятий, их денотационной семантики}
    \end{scnindent}
    \scnsuperset{функциональный аспект}
    \begin{scnindent}
        \scneq{операционный аспект}
    \end{scnindent}
    
    \scnheader{следует отличать*}
    \begin{scnhaselementset}
    	\scnitem{\textit{совместимость} между компонентами \textit{интеллектуальных компьютерных систем}}
        \scnitem{\textit{совместимость} между верхним логико-семантическим уровнем используемых моделей представления и обработки знаний и различными уровнями их интерпретации вплоть до аппаратного уровня}
        \scnitem{\textit{совместимость} между индивидуальными интеллектуальными компьютерными системами}
    	\scnitem{\textit{совместимость} между индивидуальными интеллектуальными компьютерными системами и их пользователями}
    	\scnitem{\textit{совместимость} между коллективами интеллектуальных компьютерных системам}
    \end{scnhaselementset}

	\scnheader{следует отличать*}
	\begin{scnhaselementset}
		\scnitem{данные}
		\begin{scnindent}
			\scnidtf{информационная конструкция, обрабатываемая с помощью программы традиционного языка программирования}
		\end{scnindent}
		\scnitem{знание}
		\begin{scnindent}
			\scnidtf{семантически целостный фрагмент базы знаний}
		\end{scnindent}
	\end{scnhaselementset}
	\begin{scnindent}
		\scntext{отличие}{Для каждого знания всегда известен язык, на котором это знание представлено и денотационная семантика которого задана. При этом указанный язык имеет достаточно большую семантическую мощность, а в идеале является универсальным языком. В отличие от этого структуризация данных для традиционных программ осуществляется в целях упрощения самих этих программ и, следовательно, для разных программ в общем случае осуществляется по-разному. Таким образом, при разработке традиционных программ представление обрабатываемых данных осуществляется в общем случае на разных языках, денотационная семантика которых нигде не документируется и известна только разработчикам программ. Другими словами, данные для разных программ имеют денотационную семантику не только разную, но еще и априори неизвестную. По сути это форма проявления \textit{вавилонского столпотворения} в традиционных языках программирования, которые образно говоря \scnqq{хромают на одну ногу}, формализуя методы обработки информации, но не формализуя семантику обрабатываемой информации.}
	\end{scnindent} 

\end{scnsubstruct}

		\scnsegmentheader{Предметная область и онтология принципов, лежащих в основе онтологических моделей 
    мультимодальных интерфейсов интеллектуальных компьютерных систем нового поколения}

\begin{scnsubstruct}
    \begin{scnrelfromlist}{ключевое понятие}
    	\scnitem{смысловая память}
    	\scnitem{графодинамическая память}
    	\scnitem{ассоциативная память с информационным доступом по образцу произвольного размера и
            конфигурации}
        \scnitem{система ситуационного децентрализованного управления информационными процессами}
    	\scnitem{многоагентная система обработки информации в общей памяти}
    	\scnitem{язык смыслового представления задач}
        \scnitem{универсальный язык смыслового представления знаний}
    	\scnitem{язык смыслового представления методов}
        \begin{scnindent}
    		\scnidtf{интегрированный язык смыслового представления различного вида программ}
    	\end{scnindent}
    	\scnitem{инсерционная программа}
    \end{scnrelfromlist}
   
    \begin{scnrelfromlist}{ключевое знание}
    	\scnitem{Принципы, лежащие в основе решателей задач индивидуальных интеллектуальных компьютерных
            систем нового поколения}
    \end{scnrelfromlist}

    \scnheader{решатель задач интеллектуальных компьютерных систем нового поколения}
    \begin{scnrelfromlist}{предъявляемые требования}
        \scnfileitem{решатель задач интеллектуальных компьютерных систем нового поколения должен уметь решать 
            интеллектуальные задачи}
            \begin{scnrelfromlist}{виды задач}
                \scnfileitem{некачественно сформулированная задача}
                \begin{scnindent}
                    \scnidtf{задача, формулировка которой содержит различные не-факторы (неполнота, нечеткость,
                        противоречивость (некорректность) и так далее)}
                \end{scnindent}
                \scnfileitem{задача, для решения которой, кроме самой формулировки задачи и соответствующего метода ее
                    решения необходима дополнительная, но априори неизвестно какая информация об объектах, указанных
                    в формулировке (постановке) задачи. При этом указанная дополнительная информация
                    может присутствовать, а может и отсутствовать в текущем состоянии базы знаний интеллектуальных
                    компьютерных систем. Кроме того, для некоторых задач может быть задана (указана) та область
                    базы знаний, использования которой достаточно для поиска или генерации (в частности, логического
                    вывода) указанной дополнительной требуемой информации. Такую область базы знаний будем
                    называть областью решения соответствующей задачи}
                \scnfileitem{задача, для которой соответствующий метод ее решения в текущий момент не известен}
                \begin{scnrelfromlist}{решение}
                    \scnfileitem{переформулировать задачу, то есть сгенерировать (логически вывести) логически эквивалентную
                        формулировку исходной задачи, для которой метод ее решения в текущий момент является
                        известным}
                    \scnfileitem{свести исходную задачу к семейству подзадач, для которых методы их решения в текущий
                        момент известны.}
                \end{scnrelfromlist}
            \end{scnrelfromlist}
        \scnfileitem{процесс решения задач в интеллектуальных компьютерных системах нового поколения реализуется коллективом
            информационных агентов, обрабатывающих базу знаний интеллектуальных компьютерных систем}
        \scnfileitem{управление информационными процессами в памяти интеллектуальных компьютерных систем нового
            поколения осуществляется децентрализованным образом по принципам ситуационного управления}
    \end{scnrelfromlist}

    \scnheader{ситуационное управление}
    \scnidtf{ситуационно-событийное управление}
    \scntext{пояснение}{управление последовательностью выполнения действий, при котором условием (scnqq{триггером}) инициирования
        указанных действий является:
        \begin{scnitemize}
            \item{возникновение некоторых ситуаций (условий, состояний);}
            \item{и/или возникновение некоторых событий.}
        \end{scnitemize}}
        
    \scnheader{ситуация}
    \scnidtf{структура, описывающая некоторую временно существующую конфигурацию связей между некоторыми
        сущностями}
    \scnidtf{описание временно существующего состояния некоторого фрагмента (некоторой части) некоторо
        динамической системы}

    \scnheader{событие}
    \scnsuperset{возникновение временной сущности}
    \begin{scnindent}
        \scnidtf{появление, рождение, начало существования некоторой временной сущности}
    \end{scnindent}
    \scnsuperset{исчезновение временной сущности}
    \begin{scnindent}
        \scnidtf{прекращение, завершение существования некоторой временной сущности}
    \end{scnindent}
    \scnsuperset{переход от одной ситуации к другой}
    \begin{scnindent}
        \scntext{примечание}{Здесь учитывается не только факт возникновения новой ситуации, но и ее предыстория — то есть та
        ситуация, которая ей непосредственно предшествует. Так, например, реагируя на аномальное значение
        какого-либо параметра, нам важно знать:
        \begin{scnitemize}
            \item{какова динамика изменения этого параметра (увеличивается он или уменьшается и с какой скоростью);}
            \item{какие меры были предприняты ранее для ликвидации этой аномалии.}
        \end{scnitemize}}
    \end{scnindent}

    \scnheader{решатель задач индивидуальной интеллектуальной компьютерной системы нового поколения}
    \begin{scnrelfromlist}{принципы, лежащие в основе}
        \scnfileitem{смысловое представление обрабатываемых знаний}
        \scnfileitem{семантически неограниченный ассоциативный доступ к различным фрагментам знаний, хранимым в
            памяти интеллектуальных компьютерных систем нового поколения (доступ по заданному образцу произвольного
            размера и произвольной конфигурации)}
        \scnfileitem{графодинамический характер обработки знаний в памяти, при котором обработка знаний сводится не
            только к изменению состояния атомарных фрагментов (ячеек) памяти, но и к изменению конфигурации
            связей между этими атомарными фрагментами}
        \scnfileitem{ситуационное децентрализованное управление процессом обработки знаний, а также процессом организации
            взаимодействия интеллектуальных компьютерных систем с внешней средой}
        \scnfileitem{использование семантически мощного языка задач, обеспечивающего представление формулировок самых
            различных задач, которые могут решаться либо в рамках памяти интеллектуальной компьютерной
            системы, либо во внешней среде и которые осуществляют инициирование соответствующих процессов
            решения задач}
        \scnfileitem{многоагентный характер реализации процессов решения инициированных задач, в основе которого лежит
            иерархическая система агентов, каждый из которых активизируются при возникновении в памяти
            интеллектуальной компьютерной системы соответствующий ситуации или соответствующего события}
    \end{scnrelfromlist}
\end{scnsubstruct}

		\scnsegmentheader{Предметная область и онтология принципов, лежащие в основе онтологических моделей мультимодальных
    интерфейсов интеллектуальных компьютерных систем нового поколения}

\begin{scnsubstruct}
    \begin{scnrelfromlist}{ключевое понятие}
    	\scnitem{мультимодальный интерфейс}
        \scnitem{вербальный интерфейс}
        \scnitem{естественно-языковой интерфейс}
        \scnitem{внешний язык}
    	\begin{scnindent}
    		\scnidtf{язык обмена сообщениями}
    	\end{scnindent}
        \scnitem{внутренний язык}
    	\begin{scnindent}
    		\scnidtf{язык представления информации в памяти кибернетической системы}
    	\end{scnindent}
        \scnitem{синтаксис внешнего языка}
        \scnitem{денотационная семантика внешнего языка}
        \scnitem{интерфейсная задача}
        \scnitem{понимание сообщения}
        \scnitem{синтез сообщения}
        \scnitem{невербальный интерфейс}
        \scnitem{сенсор}
    	\begin{scnindent}
    		\scnidtf{рецептор}
    	\end{scnindent}
        \scnitem{сенсорная подсистема}
        \scnitem{мультисенсорная подсистема}
        \scnitem{сенсорная информация}
        \scnitem{эффектор}
        \scnitem{мультиэффекторная подсистема}
        \scnitem{сенсо-моторная координация}
    \end{scnrelfromlist}
   
    \begin{scnrelfromlist}{ключевое знание}
    	\scnitem{Принципы, лежащие в основе интерфейсов интеллектуальных компьютерных систем нового
            поколения}
    \end{scnrelfromlist}

    \scnheader{интерфейс интеллектуальной компьютерной системы нового поколения}
    \begin{scnrelfromlist}{принципы, лежащие в основе}
        \scnfileitem{интерфейс \textit{интеллектуальной компьютерной системы нового поколения} рассматривается как решатель
        задач частного вида — \textit{интерфейсных задач}}
        \begin{scnrelfromlist}{основные зачачи}
            \scnfileitem{задачи понимания вербальной информации, приобретаемой интеллектуальной компьютерной системой 
                (синтаксический анализ, семантический анализ и погружение в базу знаний интеллектуальной
                компьютерной системы)}
            \scnfileitem{задачи понимания невербальной информации, воспринимаемой сенсорными подсистемами
                интеллектуальной компьютерной системы (анализ изображений, анализ аудио-сигналов, погружение 
                результатов анализа в базу знаний интеллектуальной компьютерной системы)}
            \scnfileitem{задачи синтеза сообщений, адресуемых внешним субъектам (кибернетическим системам)}
        \end{scnrelfromlist}
        \scnfileitem{тот факт, что интерфейс \textit{интеллектуальной компьютерной системы нового поколения} является 
            решателем частного вида \textit{задач интеллектуальной компьютерной системы нового поколения}, свойства,
            лежащие в основе решателей \textit{задач интеллектуальной компьютерной систем нового поколения}, наследуются
            интерфейсами \textit{интеллектуальной компьютерной систем нового поколения}}
        \begin{scnrelfromlist}{лежит в основе}
            \scnfileitem{смысловое представление накапливаемых (приобретаемых знаний)}
            \scnfileitem{трактовка семантического анализа приобретаемой вербальной информации как процесса перевода
                этой информации на внутренний язык смыслового представления знаний с последующим погружением
                (вводом, интеграцией) результата этого перевода в состав текущего состояния базы знаний
                \textit{интеллектуальной компьютерной системы нового поколения}}
            \scnfileitem{трактовка синтеза сообщений, адресуемых внешними субъектами как процесса обратного перевода
                некоторого фрагмента базы знаний с внутреннего языка смыслового представления информации на
                внешний язык, используемый для общения с заданным субъектом}
            \scnfileitem{агентно-ориентированная организация решения интерфейсных задач, реализуемая соответствующим
                коллективов внутренних агентов \textit{интерфейса интеллектуальных компьютерных систем нового 
                поколения}, взаимодействующих через общедоступную для них базу знаний \textit{интеллектуальной
                компьютерной системы нового поколения}}
        \end{scnrelfromlist}
        \scnfileitem{интерфейс \textit{интеллектуальной компьютерной системы нового поколения} трактуется как специализированная
            встроенная \textit{интеллектуальная компьютерная система нового поколения}, входящая в состав
            указанной выше интеллектуальной компьютерной системы, база знаний которой включает в себя:}
        \begin{scnrelfromlist}{включение}
            \scnfileitem{онтологию синтаксиса внутреннего языка смыслового преставления информации}
            \scnfileitem{онтологию денотационной семантики внутреннего языка смыслового представления информации}
            \scnfileitem{онтологию синтаксиса всех внешних языков, используемых для общения с внешними субъектами}
            \scnfileitem{онтологии денотационной семантики всех внешних языков, используемых для общения с внешними
                субъектами (каждая такая онтология с формальной точки зрения является описанием соответствия
                между текстами внешних языков и семантически эквивалентными им текстами внутреннего языка
                смыслового представления информации)}
        \end{scnrelfromlist}
        \scntext{примечание}{Подчеркнем при этом, что все указанные онтологии, входящие в состав базы знаний интерфейса
            интеллектуальных компьютерных систем нового поколения, как и вся остальная информация, входящая в
            состав этой базы знаний, представляется на внутреннем языке смыслового представления информации,
            который, соответственно используется в данном случае как метаязык}
    \end{scnrelfromlist}

    \scnheader{интерфейс индивидуальной интеллектуальной компьютерной системы нового поколения}
    \begin{scnrelfromlist}{принципы, лежащие в основе}
        \scnfileitem{интерфейс индивидуальной интеллектуальной компьютерной системы нового поколения является
            специализированным компонентом решателя задач интеллектуальной компьютерной системы нового поколения,
            то есть специализированной \uline{встроенной} (в индивидуальную интеллектуальную компьютерную систему
            нового поколения) интеллектуальной компьютерной системой нового поколения, ориентированной на
            решение интерфейсных задач, к которым относятся:}
        \begin{scnrelfromlist}{относится}
            \scnfileitem{понимание принятых сообщений (их перевод на язык внутреннего смыслового представления информации
                и погружения в текущее состояние базы знаний)}
            \scnfileitem{синтез передаваемых сообщений (перевод сформированного сообщения с внутреннего языка смыслового
                представления на используемый внешний язык)}
            \scnfileitem{первичный анализ приобретаемой сенсорной информации, предполагающий распознавание некоторого
                семейства первичных образов и сцен}
            \scnfileitem{сенсомоторная координация действий, выполняемых эффекторами интеллектуальной компьютерной системы}
        \end{scnrelfromlist}
        \scnfileitem{мультимодальный характер интерфейса — многообразие внешних языков, видов сенсоров и эффекторов}
        \scnfileitem{формальное онтологическое описание на языке внутреннего смыслового представления информации}
        \begin{scnrelfromlist}{виды информации}
            \scnfileitem{синтаксиса и денотационной семантики всех используемых внешних языков}
            \scnfileitem{первичных образов и сцен (ситуаций), являющихся результатом первичного анализа приобретаемой
                сенсорной информации}
            \scnfileitem{методов низкого уровня, непосредственно интерпретируемых эффекторами интеллектуальной компьютерной системы}
        \end{scnrelfromlist}
    \end{scnrelfromlist}
    \scntext{примечание}{Разговоры о дружественном и, в частности, адаптивном \textit{пользовательском интерфейсе} ведутся давно, но это, чаще
        всего, касается формы (scnqq{синтаксической} стороны) \textit{пользовательского интерфейса}, а не смыслового содержания
        взаимодействия с пользователями. В настоящее время \textit{пользовательские интерфейсы} компьютерных систем (в
        том числе и \textit{интеллектуальных компьютерных систем}) для широкого контингента пользователей не являются
        семантически (содержательно) дружественными (семантически комфортными). Организация взаимодействия
        пользователей с компьютерными системами (в том числе и с \textit{интеллектуальными компьютерными системами})
        является \scnqq{узким местом}, оказывающим существенное влияние на эффективность \textit{автоматизации человеческой
        деятельности}. В основе современной организации взаимодействия пользователя с компьютерной системой лежит
        парадигма \uline{грамотного} пользователя, который знает, чего он хочет от используемого им инструмента и несет полную
        ответственность за качество взаимодействия с этим инструментом. Эта парадигма лежит в основе деятельности
        лесоруба во взаимодействии с топором, всадника во взаимодействии с лошадью, автоводителя, летчика во взаимодействии
        с соответствующим транспортным средством, оператора атомной электростанции, железнодорожного диспетчера и так далее.}
    \scntext{примечание}{На современном этапе развития \textit{Искусственного интеллекта} для повышения эффективности взаимодействия
        необходим переход \uline{от парадигмы грамотного управления} используемым инструментом \uline{к парадигме равноправного
        сотрудничества}, партнерскому взаимодействию интеллектуальной компьютерной системы со своим пользователем.
        \textit{Интеллектуальная компьютерная система} должна повернуться \scnqq{лицом} к пользователю. Семантическая дружественность 
        пользовательского интерфейса должна заключаться в адаптивности к особенностям и квалификации пользователя, исключении 
        любых проблем для пользователя в процессе диалога с \textit{интеллектуальной компьютерной системой}, в перманентной заботе о 
        совершенствовании коммуникационных навыков пользователя.}
    \scntext{примечание}{При организации взаимодействия пользователя с \textit{Глобальной сетью} компьютерным системам необходимо перейти
        от парадигмы \scnqq{многооконного} интерфейса, в каждом \scnqq{окне} которого свои \scnqq{правила игры}, к парадигме \scnqq{одного
        окна}. Пользователь не должен знать, какое \scnqq{окно} ему надо \scnqq{открыть} (в какую систему ему надо войти) для
        удовлетворения той или иной его потребности.
        Пользователь не должен знать, какая конкретно система будет решать его задачу. Пользователь должен уметь с
        помощью \uline{универсальных} средств сформулировать свою задачу, а соответствующая компьютерная система, входящая 
        в \textit{Глобальную сеть} и способная решить эту задачу, должна сама инициироваться, реагируя на факт появления
        указанной задачи. Таким образом пользовательский интерфейс должен быть интерфейсом пользователя не с 
        конкретной компьютерной системой, а в целом со всей \textit{Глобальной сетью компьютерных систем}.}
\end{scnsubstruct}

		\scnsegmentheader{Предметная область и онтология достоинств предлагаемых принципов, лежащих в основе
интеллектуальных компьютерных систем нового поколения}

\begin{scnsubstruct}
    \scnheader{принципы, лежащих в основе интеллектуальных компьютерных систем нового поколения}
    \scntext{достоинство}{\textbf{смысловое представление информации} в памяти \textit{интеллектуальных компьютерных систем} обеспечивает
        устранение дублирования информации, хранимой в памяти \textit{интеллектуальной компьютерной системы}, то есть
        устранение многообразия форм представления одной и той же информации, запрещение появления в одной памяти 
        \textit{семантически эквивалентных информационных конструкций} и, в том числе, синонимичных \textit{знаков}. Это
        существенно снижает сложность и повышает качество:
    \begin{scnitemize}
        \item{разработки различных \textit{моделей обработки знаний} (так как нет необходимости учитывать многообразие форм
            представления одного и того же знания);}
        \item{\textit{семантического анализа} и \textit{понимания} информации, поступающей (передаваемой) от различных внешних
            субъектов (от пользователей, от разработчиков, от других \textit{интеллектуальных компьютерных систем});}
        \item{\textit{конвергенции} и \textit{интеграции} различных видов знаний в рамках каждой \textit{интеллектуальной компьютерной
            системы};}
        \item{обеспечения \textit{семантической совместимости} и \textit{взаимопонимания} между различными \textit{интеллектуальными
            компьютерными системами}, а также между \textit{интеллектуальными компьютерными системами} и их пользователями}
    \end{scnitemize}}

    \scntext{достоинство}{Понятие \textit{семантической сети} нами рассматривается не как красивая метафора сложноструктурированных
        \textit{знаковых конструкций}, а как формальное уточнение понятия \textit{смыслового представления информации}, как принцип
        представления информации, лежащей в основе нового поколения \textit{компьютерных языков} и самих \textit{компьютерных 
        систем} — \textit{графовых языков} и \textit{графовых компьютеров}. \textit{Семантическая сеть} — это нелинейная (графовая)
        \textit{знаковая конструкция}, обладающая следующими свойствами:
        \begin{scnitemize}
            \item{все элементы (то есть синтаксически элементарные фрагменты) этой \textit{графовой структуры} (узлы и связки)
                являются знаками описываемых сущностей и, в частности, \textit{знаками связей} между этими сущностями;}
            \item{все знаки, входящие в эту \textit{графовую структуру}, не имеют \textit{синонимов} в рамках этой структуры;}
            \item{\scnqq{внутреннюю} структуру (строение) \textit{знаков}, входящих в семантическую сеть не требуется учитывать при ее
                \textit{семантическом анализе} (понимании);}
            \item{смысл \textit{семантической сети} определяется денотационной семантикой всех входящих в нее знаков и конфигурацией
                \textit{связей инцидентности} этих знаков;}
            \item{из двух \textit{инцидентных знаков}, входящих в \textit{семантическую сеть}, по крайней мере один является знаком связи.}
        \end{scnitemize}}

    \scntext{достоинство}{\textit{рафинированная семантическая сеть} — это \textit{семантическая сеть}, имеющая максимально простую 
        \textit{синтаксическую структуру}, в которой, в частности,
        \begin{scnitemize}
            \item{используется \uline{конечный} \textit{алфавит} элементов \textit{семантической сети}, то есть конечное число синтаксически
                выделяемых типов (синтаксических меток), приписываемых этим элемента;}
            \item{внешние идентификаторы (в частности, имена), приписываемые элементам \textit{семантической сети} используются
                \uline{только} для ввода/вывода информации}
        \end{scnitemize}}

    \scntext{достоинство}{\textit{агентно-ориентированная модель обработки информации} в сочетании с \textit{децентрализованным ситуационным
        управлением процессом обработки информации}, а также со \textit{смысловым представлением информации} в памяти
        \textit{интеллектуальной компьютерной системы} существенно снижает сложность и повышает качество интеграции
        \begin{scnitemize}
        \item{беспечивает автоматизацию решения сложных комплексных задач, для которых требуется создание
            временных или постоянных \uline{коллективов};}
        \item{превращает \textit{интеллектуальные компьютерные системы} в \uline{самостоятельные} активные \textit{субъекты}, способные
            инициировать различные комплексные задачи и, собственно, инициировать для этого работо-
            способные коллективы, состоящие из людей и \textit{интероперабельных интеллектуальных компьютерных
            систем} требуемой квалификации}
        \end{scnitemize}}

    \scntext{достоинство}{Высокий уровень семантической гибкости информации, хранимой в памяти интеллектуальной компьютерной
        системы нового поколения, обеспечивается тем, что каждое удаление или добавление синтаксически элементарного
        фрагмента хранимой информации, а также удаление или добавление каждой связи инцидентности между такими
        элементами имеет четкую семантическую интерпретацию.}

    \scntext{достоинство}{Высокий уровень стратифицированности информации, хранимой в памяти интеллектуальной компьютерной
        системы нового поколения, обеспечивается онтологически ориентированной структуризацией базы знаний интеллектуальной
        компьютерной системы нового поколения.}

    \scntext{Высокий уровень индивидуальной обучаемости интеллектуальных компьютерных систем нового поколения (то
        есть их способности к быстрому расширению своих знаний и навыков) обеспечивается:
        \begin{scnitemize}
            \item{семантической гибкостью информации, хранимой в их памяти;}
            \item{стратифицированностью этой информации;}
            \item{рефлексивностью интеллектуальных компьютерных систем нового поколения.}
        \end{scnitemize}}

    \scntext{достоинство}{Высокий уровень коллективной обучаемости интеллектуальных компьютерных систем нового поколения
        обеспечивается высоким уровнем их интероперабельности (их социализации, способности к эффективному участию в
        деятельности различных коллективов, состоящих из интеллектуальных компьютерных систем нового поколения и
        людей) и, прежде всего, высоким уровнем их взаимопонимания.}

    \scntext{достоинство}{Высокий уровень интероперабельности интеллектуальных компьютерных систем нового поколения
        принципиально меняет характер взаимодействия компьютерных систем с людьми, автоматизацию деятельности которых они
        осуществляют, — от управления этими средствами автоматизации к равноправным партнерским осмысленным
        взаимоотношениям}

    \scntext{достоинство}{Каждая интеллектуальная компьютерная система нового поколения способна:
        \begin{scnitemize}
            \item{самостоятельно или по приглашению войти в состав коллектива, состоящего из интеллектуальных компьютерных 
                систем нового поколения и/или людей. Такие коллективы создаются на временной или постоянной основе
                для коллективного решения сложных задач;}
            \item{участвовать в распределении (в том числе в согласовании распределения) задач — как \scnqq{разовых} задач, так и
                долгосрочных задач (обязанностей);}
            \item{мониторить состояние всего процесса коллективной деятельности и координировать свою деятельность с
                деятельностью других членов коллектива при возможных непредсказуемых изменениях условий (состояния)
                соответствующей среды.}
        \end{scnitemize}}

    \scntext{достоинство}{Высокий уровень интеллекта интеллектуальных компьютерных систем нового поколения и, соответственно,
        высокий уровень их самостоятельности и целенаправленности позволяет им быть полноправными членами самых
        различных сообществ, в рамках которых интеллектуальные компьютерные системы нового поколения получают
        права самостоятельно инициировать (на основе детального анализа текущего положения дел и, в том числе,
        текущего состояния плана действий сообщества) широкий спектр действий (задач), выполняемых другими членами
        сообщества, и тем самым участвовать в согласовании и координации деятельности членов сообщества. Способность
        интеллектуальной компьютерной системы нового поколения согласовывать свою деятельность с другими
        подобными системами, а также корректировать деятельность всего коллектива интеллектуальных компьютерных
        систем нового поколения, адаптируясь к различного вида изменениям среды (условий), в которой эта деятельность
        осуществляется, позволяет существенно автоматизировать деятельность системного интегратора как на этапе
        создания коллектива интеллектуальных компьютерных систем нового поколения, так и на этапе его обновления
        (реинжиниринга)}
    
    \scntext{примечание}{Достоинства интеллектуальных компьютерных систем нового поколения обеспечиваются:
        \begin{scnitemize}
            \item{достоинствами языка внутреннего смыслового кодирования информации, хранимой в памяти этих систем;}
            \item{достоинствами организации графодинамической ассоциативной смысловой памяти интеллектуальных
                компьютерных систем нового поколения;}
            \item{достоинствами смыслового представления баз знаний интеллектуальных компьютерных систем нового
                поколения и средствами онтологической структуризации баз знаний этих систем;}
            \item{достоинствами агентно-ориентированных моделей решения задач, используемых в интеллектуальных
                компьютерных системах нового поколения в сочетании с децентрализованным управлением процессом обработки
                информации.}
        \end{scnitemize}}

    \begin{scnrelfromlist}{основные положения}
    \scnfileitem{основным практически значимым направлением развития современных интеллектуальных компьютерных
        систем является переход к интероперабельным интеллектуальным компьютерным системам, способным к эффективному
        взаимодействию между собой и с пользователями, что:
        \begin{scnitemize}
            \item{обеспечивает автоматизацию решения сложных комплексных задач, для которых требуется создание
                временных или постоянных \uline{коллективов;}}
            \item{превращает интеллектуальные компьютерные системы в \uline{самостоятельные} активные субъекты, способные
                инициировать различные комплексные задачи и, собственно, инициировать для этого работоспособные
                коллективы, состоящие из людей и интероперабельных интеллектуальных компьютерных систем требуе-
                мой квалификации.}
        \end{scnitemize}}
    \scnfileitem{коллективы, состоящие из самостоятельных \textit{интероперабельных интеллектуальных компьютерных систем}
        и людей, имеют хорошие перспективы стать \textit{синергетическими} системами}
    \scnfileitem{\textit{интероперабельность интеллектуальных компьютерных систем} обеспечивается:
        \begin{scnitemize}
            \item{высоким уровнем взаимопонимания и, соответственно, семантической совместимостью;}
            \item{высоким уровнем договороспособности, то есть способности предварительно согласовывать свои действия
                с действиями других субъектов;}
            \item{высоким уровнем способности оперативно координировать свои действия с действиями других субъектов
                в ходе их выполнения}
        \end{scnitemize}}
    \scnfileitem{к числу принципов, лежащих в основе построения \textit{интероперабельных интеллектуальных компьютерных
        систем}, относятся:
        \begin{scnitemize}
            \item{смысловое представление знаний в памяти \textit{интеллектуальных компьютерных систем} в виде рафинированных 
                семантических сетей;}
            \item{использование универсального языка внутреннего смыслового представления знаний;}
            \item{графодинамическая организация обработки знаний;}
            \item{агентно-ориентированные модели решения задач;}
            \item{структуризация и стратификация баз знаний в виде иерархической системы формальных онтологий;}
            \item{семантически дружественный пользовательский интерфейс.}
        \end{scnitemize}}
    \scnfileitem{для разработки большого количества интероперабельных семантически совместимых \textit{интеллектуальных
        компьютерных систем}, обеспечивающих переход на принципиально новый уровень автоматизации \textit{человеческой
        деятельности}, необходимо создание технологии, обеспечивающей массовое производство таких \textit{интеллектуальных
        компьютерных систем}, участие в котором доступно широкому контингенту разработчиков (в том
        числе разработчиков средней квалификации и начинающих разработчиков). Основными положениями такой
        технологии являются
        \begin{scnitemize}
            \item{стандартизация \textit{интероперабельных интеллектуальных компьютерных систем};}
            \item{широкое использование \textit{компонентного проектирования} на основе мощной библиотеки семантически
                совместимых многократно используемых (типовых) компонентов \textit{интероперабельных интеллектуальных
                компьютерных систем}}
        \end{scnitemize}}
    \scnfileitem{эффективная эксплуатация \textit{интероперабельных интеллектуальных компьютерных систем} требует создания
        не только \textit{технологии проектирования} таких систем, но также и семейства технологий поддержки всех
        остальных этапов их жизненного цикла. Особенно это касается технологии перманентной поддержки \textit{семантической
        совместимости} всех взаимодействующих \textit{интероперабельных интеллектуальных компьютерных систем} в ходе их эксплуатации}
    \end{scnrelfromlist}
\end{scnsubstruct}

	\end{scnsubstruct}
\end{SCn}
\scnsourcecomment{Завершили Раздел \scnqqi{Предметная область и онтология интеллектуальных компьютерных систем нового поколения}}
