\scnsegmentheader{Требования, предъявляемые к интеллектуальным компьютерным системам нового поколения}

\begin{scnsubstruct}

    \begin{scnrelfromlist}{ключевое понятие}
    	\scnitem{интеллектуальная компьютерная система нового поколения}
    	\scnitem{интероперабельная интеллектуальная компьютерная система}
    	\scnitem{гибридная интеллектуальная компьютерная система}
    \end{scnrelfromlist}
    
    \begin{scnrelfromlist}{ключевое отношение}
    	\scnitem{соединение интеллектуальных компьютерных систем*}
    	\begin{scnindent}
    		\scnidtf{преобразование множества интеллектуальных компьютерных систем в коллектив, членами (агентами) которого являются эти системы*}
    	\end{scnindent}
    	\scnitem{глубокая интеграция интеллектуальных компьютерных систем*}
    	\begin{scnindent}
    		\scnidtf{быть результатом преобразования множества индивидуальных интеллектуальных компьютерных систем в одну интегрированную индивидуальную интеллектуальную компьютерную систему*}
    	\end{scnindent}
    \end{scnrelfromlist}
    
    \begin{scnrelfromlist}{ключевой параметр}
    	\scnitem{интероперабельность интеллектуальных компьютерных систем\scnsupergroupsign}
    	\scnitem{семантическая совместимость пар интеллектуальных компьютерных систем\scnsupergroupsign}
    \end{scnrelfromlist}
    
    \begin{scnrelfromlist}{ключевое знание}
    	\scnitem{Требования, предъявляемые к интеллектуальным компьютерным системам нового поколения}
    	\scnitem{Принципы, лежащие в основе интеллектуальных компьютерных систем нового поколения}
    	\scnitem{Отличие данных от знаний}
    \end{scnrelfromlist}

    \scnheader{уровень интероперабельности интеллектуальных компьютерных систем}
    \scntext{примечание}{Создание различных комплексов взаимодействующих интеллектуальных компьютерных систем \uline{требует} повышения качества не только самих этих систем, но также и качества их взаимодействия. Интеллектуальные компьютерные системы нового поколения должны иметь высокий уровень интероперабельности.}
    \scnidtf{уровень коммуникационной (социальной) совместимости интеллектуальных компьютерных систем, позволяющей им самостоятельно формировать коллективы интеллектуальных компьютерных систем и их пользователей, а также самостоятельно согласовывать и координировать свою деятельность в рамках этих коллективов при решении сложных задач в частично предсказуемых условиях}
    \scnidtf{уровень способности к эффективному, целенаправленному взаимодействию с себе подобными и с пользователями в процессе коллективного (распределенного) и децентрализованного решения сложных задач}
    \begin{scnindent}
        \begin{scnrelfromlist}{источник}
            \scnitem{\scncite{Yaghoobirafi2022}}
            \scnitem{\scncite{Ouksel1999}}
            \scnitem{\scncite{Lanzenberger2008}}
            \scnitem{\scncite{Neiva2016}}
            \scnitem{\scncite{Pohl2004}}
            \scnitem{\scncite{Waters2009}}
        \end{scnrelfromlist}
    \end{scnindent}
    \scnidtf{уровень \scnqq{социализации} интеллектуальных компьютерных систем, полезности в рамках различных априори неизвестных сообществ (коллективов) \textit{интеллектуальных систем}}
    \scntext{примечание}{Повышение уровня \textit{интероперабельности} интеллектуальных компьютерных систем определяет переход к \textbf{\textit{интеллектуальным компьютерным системам нового поколения}}, без которых невозможна реализация таких проектов, как \textit{интеллектуальное-предприятие}, \textit{интеллектуальная-больница}, \textit{интеллектуальная-школа}, \textit{интеллектуальный-университет}, \textit{интеллектуальная-кафедра}, \textit{интеллектуальный-дом}, \textit{интеллектуальный-город}, \textit{интеллектуальное-общество}.}
        \begin{scnindent}
            \begin{scnrelfromlist}{источник}
                \scnitem{\scncite{Lopes2022}}
                \scnitem{\scncite{Hamilton2006}}
            \end{scnrelfromlist}
        \end{scnindent}
    
    \scnheader{интеллектуальная компьютерная система}
    \scnidtf{интеллектуальная искусственная кибернетическая система}
    \begin{scnrelfromset}{разбиение}
    	\scnitem{индивидуальная интеллектуальная компьютерная система}
    	\scnitem{интеллектуальный коллектив интеллектуальных компьютерных систем}
    	\begin{scnindent}
    		\scnidtf{интеллектуальная \textit{многоагентная система}, агенты которой являются \textit{интеллектуальными компьютерными системами}}
    		\scntext{примечание}{Не каждый \textit{коллектив интеллектуальных компьютерных систем} может оказаться интеллектуальным, поскольку уровень интеллекта такого коллектива определяется не только уровнем интеллекта его членов, но также и эффективностью (качеством) \uline{их взаимодействия}.}
    		\begin{scnrelfromset}{разбиение}
    			\scnitem{интеллектуальный коллектив \uline{индивидуальных} интеллектуальных компьютерных систем}
    			\scnitem{иерархический интеллектуальный коллектив интеллектуальных компьютерных систем}
    			\begin{scnindent}
    				\scnidtf{\textit{интеллектуальный коллектив интеллектуальных компьютерных систем}, по крайней мере одним из членов которого является \textit{интеллектуальный коллектив интеллектуальных компьютерных систем}}
    			\end{scnindent}
    		\end{scnrelfromset}
    	\end{scnindent}
    \end{scnrelfromset}
    
    \scnheader{интеллектуальные компьютерные системы нового поколения}
    \begin{scnrelfromlist}{предъявляемые требования}
    	\scnitem{высокий уровень \textit{интероперабельности}}
    	\scnitem{высокий уровень \textit{обучаемости}}
    	\scnitem{высокий уровень \textit{гибридности}}
    	\scnitem{высокий уровень способности решать \textit{интеллектуальные задачи}}
        \begin{scnindent}
            \scnidtf{\textit{задачи}, \textit{методы} решения которых и/или требуемая для их решения исходная информация априори неизвестны}
        \end{scnindent}
        \scnitem{высокий уровень \textit{синергетичности}}
    \end{scnrelfromlist}
    
    \scnheader{интероперабельность\scnsupergroupsign}
    \scnidtf{способность к эффективному (целенаправленному) взаимодействию с другими самостоятельными субъектами}
    \scnidtf{способность к партнерскому взаимодействию в решении \textit{комплексных задач}, требующих \textit{коллективной деятельности}}
    \scnidtf{способность работать в коллективе (в команде)}
    \scnidtf{уровень социализации}
    \scnidtf{social skills}
    
    \scnheader{высокий уровень интероперабельности}
    \begin{scnrelfromlist}{обеспечивается}
    	\scnfileitem{высоким уровнем \textit{взаимопонимания}}
    	\begin{scnindent}
    		\begin{scnrelfromlist}{обеспечивается}
    			\scnfileitem{высоким уровнем \textbf{\textit{семантической совместимости}} заданного субъекта с другими субъектами заданного коллектива}
    			\scnfileitem{высоким уровнем \textit{способности понимать} сообщения и поведение партнеров}
    			\scnfileitem{высоким уровнем \textit{способности быть понятной} для партнеров:}
                \begin{scnindent}
                    \begin{scnrelfromlist}{обеспечивается}
                        \scnfileitem{способностью понятно и обоснованно формулировать свои предложения и информацию, полезную для решения текущих задач}
    			        \scnfileitem{способностью действовать и комментировать свои действия так, чтобы они и их мотивы были понятны партнерам}
                \end{scnrelfromlist}
                \end{scnindent}
    			\scnfileitem{высоким уровнем \textit{способности к повышению уровня семантической совместимости} со своими партнерами}
    		\end{scnrelfromlist}
    	\end{scnindent}
    	\scnfileitem{высоким уровнем \textit{договороспособности}, то есть способности согласовывать с партнерами свои планы и намерения в целях своевременного обеспечения высокого качества коллективного результата}
    	\scnfileitem{высоким уровнем \textit{способности к децентрализованной координации} своих действий с действиями партнеров в непредсказуемых (нештатных) обстоятельствах}
    	\scnfileitem{высоким уровнем способности разделять ответственность с партнерами}
    	\scnfileitem{высоким уровнем \textit{способности к минимизации негативных последствий конфликтных ситуаций} с другими субъектами}
    	\begin{scnindent}
    		\begin{scnrelfromlist}{обеспечивается}
    			\scnfileitem{высоким уровнем \textit{способности к предотвращению возникновения конфликтных ситуаций}}
    			\scnfileitem{\textit{соблюдением этических норм} и правил, препятствующих возникновению разрушительных последствий конфликтных ситуаций}
    			\scnfileitem{высоким уровнем \textit{способности разделять ответственность} с партнерами за своевременное и качественное достижение общей цели}
    		\end{scnrelfromlist}
    	\end{scnindent}
    \end{scnrelfromlist}
    
    \scnheader{семантическая совместимость\scnsupergroupsign}
    \scnidtf{степень согласованности (совпадения) систем \textit{понятий} и других \textit{ключевых знаков}, используемых заданными взаимодействующими субъектами}
    \scntext{примечание}{Обеспечение \textit{семантической совместимости} требует формализации \textit{смыслового представления информации}.}
    
    \scnheader{способность разделять ответственность с партнёрами}
    \scnidtf{необходимое условие децентрализованного управления коллективной деятельностью}
    \begin{scnrelfromlist}{обеспечивается}
    	\scnfileitem{\textit{способностью к мониторингу} и анализу коллективно выполняемой деятельности}
    	\scnfileitem{\textit{способностью оперативно информировать партнеров} о неблагоприятных ситуациях, событиях, тенденциях, а также инициировать соответствующие коллективные действия}
    \end{scnrelfromlist}
    
    \scnheader{высокий уровень обучаемости интеллектуальной компьютерной системы нового поколения}
    \scnexplanation{Важнейшим направлением повышения уровня автоматизации человеческой деятельности является повышение уровня автоматизации не только проектирования интеллектуальной компьютерной системы, но и комплексной поддержки всех остальных этапов жизненного цикла \textit{интеллектуальной компьютерной системы}. В частности, это касается модернизации (совершенствования, реинжиниринга) интеллектуальной компьютерной системы непосредственно в ходе их эксплуатации. Для того, чтобы обеспечить высокий уровень автоматизации такой модернизации, необходимо существенно повысить \textbf{\textit{уровень самообучаемости}} \textit{интеллектуальной компьютерной системы} для того, что они сами (самостоятельно) могли себя модернизировать (самосовершенствовать) в ходе своего целевого функционирования.}
    
    \scnheader{высокий уровень обучаемости}
    \begin{scnrelfromlist}{обеспечивается}
    	\scnfileitem{высоким уровнем \textit{гибкости информации}, хранимой в памяти интеллектуальной системы}
    	\scnfileitem{высоким уровнем \textit{качества} \textit{стратификации информации}, хранимой в памяти интеллектуальной системы (стратифицированностью \textit{базы знаний})}
    	\scnfileitem{высоким уровнем \textit{рефлексивности} интеллектуальной системы}
    	\scnfileitem{высоким уровнем \textit{способности исправлять свои ошибки} (в том числе устранять противоречия в своей \textit{базе знаний})}
    	\scnfileitem{высоким уровнем \textit{познавательной активности}}
    	\scnfileitem{низким уровнем \textit{ограничений на вид приобретаемых знаний и навыков} (отсутствие таких ограничений означает потенциальную \textit{универсальность} интеллектуальной системы и предполагает высокий уровень ее гибридности)}
    \end{scnrelfromlist}
    
    \scnheader{обучаемость\scnsupergroupsign}
    \scnidtf{способность быстро и качественно приобретать новые \textit{знания} и \textit{навыки}, а также совершенствовать уже приобретенные \textit{знания} и \textit{навыки}}
    
    \scnheader{гибридность\scnsupergroupsign}
    \scnidtf{степень многообразия используемых \textit{видов знаний} и \textit{моделей решения задач} и уровень эффективности их совместного использования}
    \scnidtf{индивидуальная способность решать \textit{комплексные задачи}, требующие использования различных \textit{видов знаний}, а также различных комбинаций различных \textit{моделей решения задач}}
    \scntext{пояснение}{\textit{Гибридность} и \textit{интероперабельность} \textit{интеллектуальных компьютерных систем нового поколения} предполагает отказ от известной парадигмы \scnqq{черных ящиков}, поскольку:
    \begin{scnitemize}
        \item все многообразие моделей решения задач \textit{гибридной интеллектуальной компьютерной системы} должно интерпретироваться на одной общей \textit{универсальной платформе};
    	\item
    	доступность информации о том, как устроен каждый используемый метод, модель решения задач, каждый субъект существенно повышает качество их \textit{координации} при \textit{совместном решении комплексных задач};
    	\item
    	появляется возможность некоторые методы, модели решения задач и целые субъекты (например, \textit{интеллектуальные компьютерные системы}) использовать для совершенствования (повышения качества) других методов, моделей и субъектов.
    \end{scnitemize}}
    
    \scnheader{высокий уровень гибридности}
    \begin{scnrelfromlist}{обеспечивается}
    	\scnfileitem{высокой степенью многообразия используемых \textit{видов знаний} и \textit{моделей решения задач}}
    	\scnfileitem{высокой степенью \textit{конвергенции} и глубокой \textit{интеграции} (степенью взаимопроникновения) различных \textit{видов знаний} и \textit{моделей решения задач}}
    	\scnfileitem{способностью неограниченно расширять уровень своей \textit{гибридности}}
    \end{scnrelfromlist}
    
    \scnheader{характеристики \textit{интеллектуальных компьютерных систем нового поколения}}
    \scnhaselement{\textbf{\textit{cтепень}} \textbf{\textit{конвергенции}}, унификации и стандартизации \textit{интеллектуальных компьютерных систем} и их компонентов и соответствующая этому \textbf{\textit{степень интеграции}} (глубина интеграции) \textit{интеллектуальных компьютерных систем} и их компонентов.}
    \scnhaselement{\textbf{\textit{cемантическая совместимость}} между \textit{интеллектуальными компьютерными системами} в целом и \textit{семантическая совместимость} между компонентами каждой \textit{интеллектуальной компьютерной системы} (в частности, совместимость между различными \textit{видами знаний} и различными \textit{моделями обработки знаний}), которые являются основными показателями степени \textbf{\textit{конвергенции}} (сближения) между \textit{интеллектуальными компьютерными системами} и их компонентами.}
    \scntext{пояснение}{Особенность указанных характеристик \textit{интеллектуальных компьютерных систем} их компонентов заключается в том, что они играют важную роль при решении всех ключевых задач современного этапа развития \textit{Искусственного интеллекта} и тесно связаны друг с другом.}
    \scntext{пояснение}{Перечисленные требования, предъявляемые к \textit{интеллектуальным компьютерным системам нового поколения}, направлены на преодоление проклятия \textit{вавилонского столпотворения} как внутри \textit{интеллектуальных компьютерных систем нового поколения} (между внутренними \textit{информационными процессами} решения различных задач), так и между взаимодействующими самостоятельными \textit{интеллектуальными компьютерными системами нового поколения} в процессе коллективного решения \textit{комплексных задач}.}
    
    \scnheader{интеллектуальная компьютерная система нового поколения}
    \scntext{примечание}{На современном этапе эволюции \textit{интеллектуальных компьютерных систем} для существенного расширения областей их применения и качественного повышения уровня автоматизации человеческой деятельности:
        \begin{scnitemize}
            \item необходим переход к созданию \uline{семантически совместимых} \textbf{интеллектуальных компьютерных систем \uline{нового поколения}}, ориентированных не только на индивидуальное, но и на \uline{коллективное} (совместное) решение \textit{комплексных задач}, требующих скоординированной деятельности нескольких самостоятельных интеллектуальных компьютерных систем и использования различных моделей и методов в непредсказуемых комбинациях, что необходимо для существенного расширения сфер применения \textit{интеллектуальных компьютерных систем}, для перехода от автоматизации локальных видов и областей \textit{человеческой деятельности} к комплексной автоматизации более крупных (объединенных) видов и областей этой деятельности;
            \item необходима разработка \textbf{Общей формальной теории и стандарта интеллектуальных компьютерных систем нового поколения};
            \item необходима разработка \textbf{Технологии комплексной поддержки жизненного цикла интеллектуальных компьютерных систем нового поколения}, которая включает в себя поддержку \textit{проектирования} этих систем (как начального этапа их жизненного цикла) и обеспечение их \textit{совместимости} на всех этапах их жизненного цикла;
            \item необходима \textbf{конвергенция} и \textbf{унификация} \textit{интеллектуальных компьютерных систем нового поколения} и их компонентов;
            \item необходима реализация \scnqq{бесшовной}, \scnqq{диффузной}, взаимопроникающей, \textbf{глубокой интеграции семантически смежных компонентов интеллектуальных компьютерных систем}, то есть интеграции, при которой отсутствуют четкие границы (\scnqq{швы}) интегрируемых (соединяемых) компонентов, и которая может осуществляться \uline{автоматически}. Это означает переход к \textbf{\uline{гибридным} интеллектуальным компьютерным системам};
            \item необходимо соблюдение \textbf{Принципа бритвы Оккама} --- максимально возможное структурное упрощение \textit{интеллектуальных компьютерных систем нового поколения}, исключение \uline{эклектичных} решений;
            \itemнеобходима ориентация на потенциально \textbf{универсальные} (то есть способные быстро приобретать \uline{любые} знания и навыки), \textbf{синергетические} \textit{интеллектуальные компьютерные системы} с \scnqq{сильным} интеллектом
        \end{scnitemize}}
    \begin{scnrelfromlist}{принципы, лежащие в основе}
        \scnfileitem{\textit{смысловое представление знаний} в памяти \textit{интеллектуальных компьютерных систем}, предполагающее отсутствие \textit{омонимических знаков}, которые в разных контекстах обозначают разные сущности, а также отсутствие \textit{синонимии}, то есть пар синонимичных \textit{знаков}, которые обозначают одну и ту же сущность}
        \scnfileitem{смысловое представление информационной конструкции в общем случае имеет нелинейный (графовый) характер представления информации, который является \textit{рафинированной семантической сетью}}
        \scnfileitem{фрактальный характер (масштабируемое самоподобие) структуризации представляемых знаний в базах знаний}
        \scnfileitem{использование \uline{общего} для всех интеллектуальных компьютерных систем \textit{универсального языка смыслового представления знаний} в памяти \textit{интеллектуальных компьютерных систем}, обладающий максимально простым \textit{синтаксисом}, обеспечивающий представление любых \textit{видов знаний} и имеющий неограниченные возможности перехода от \textit{знаний} к \textit{метазнаниям}. Простота синтаксиса \textit{информационных конструкций} указанного \textit{языка} позволяет называть эти конструкции \textit{рафинированными семантическими сетями}}
        \scnfileitem{\textit{структурно-перестраиваемая (графодинамическая) организация памяти} интеллектуальных компьютерных систем, при которой обработка знаний сводится не столько к изменению состояния хранимых \textit{знаков}, сколько к изменению конфигурации связей между этими \textit{знаками}}
        \scnfileitem{\textit{семантически неограниченный ассоциативный доступ к информации}, хранимой в памяти \textit{интеллектуальных компьютерных систем}, по заданному образцу произвольного размера и произвольной конфигурации}
        \scnfileitem{универсальная ситуационная многоагентная модель обработки знаний, ориентированная на обработку смыслового представления информации в ассоциативной графодинамической памяти, \textit{децентрализованное ситуационное управление информационными процессами} в памяти \textit{интеллектуальных компьютерных систем}, реализованное с помощью \textit{агентно-ориентированной модели обработки баз знаний}, в котором \textit{инициирование} новых \textit{информационных процессов} осуществляется не путем передачи управления соответствующим априори известным процедурам, а в результате возникновения соответствующих \textit{ситуаций} или \textit{событий} \textit{в памяти интеллектуальной компьютерной системы}, поскольку \scnqqi{основная проблема компьютерных систем состоит не в накоплении знаний, а в умении активизировать нужные знания в процессе решения задач} (Поспелов Д.~А.). Такой многоагентный процесс обработки информации представляет собой \textit{деятельность}, выполняемую некоторым коллективом \uline{самостоятельных} \textit{информационных агентов} (агентов обработки информации), условием инициирования каждого из которых является появление в текущем состоянии \textit{базы знаний} соответствующей этому агенту \textit{ситуации} и/или \textit{события}.
            \scnqqi{Выбор многоагентных технологий объясняется тем, что в настоящее время любая сложная производственная, логистическая или другая система может быть представлена набором взаимодействий более простых систем до любого уровня детальности, что обеспечивает фрактально-рекурсивный принцип построения многоярусных систем, построенных как открытые цифровые колонии и экосистемы ИИ. В основе многоагентных технологий лежит распределенный или децентрализованный подход к решению задач, при котором динамически обновляющаяся информация в распределенной сети интеллектуальных агентов обрабатывается непосредственно у агентов вместе с локально доступной информацией от \scnqq{соседей}. При этом существенно сокращаются как ресурсные и временные затраты на коммуникации в сети, так и время на обработку и принятие решений в центре системы (если он все-таки есть).}}
        \scnfileitem{агентно-ориентированная модель обработки знаний в памяти интеллектуальной компьютерной системы, обеспечивающая высокую степень \textit{интероперабельности} между внутренними агентами индивидуальной интеллектуальной компьютерной системы, взаимодействующими через общую память (это, фактически, \scnqq{внутренняя} интероперабельность интеллектуальной компьютерной системы нового поколения)}
        \scnfileitem{Переход к \textit{семантическим} \textit{моделям решения задач}, в основе которых лежит учет не только синтаксических (структурных) аспектов обрабатываемой информации, но также и \uline{семантических} (смысловых) аспектов этой информации --- \scnqqi{From data science to knowledge science}}
        \scnfileitem{\textbf{\textit{онтологическая модель баз знаний}} \textit{интеллектуальных компьютерных систем}, то есть онтологическая структуризация всей информации, хранимой в памяти \textit{интеллектуальной компьютерной системы}, предполагающая четкую \textit{стратификацию базы знаний} в виде иерархической системы \textit{предметных областей} и соответствующих им \textit{онтологий}, каждая из которых обеспечивает семантическую \textit{спецификацию} всех \textit{понятий}, являющихся ключевыми в рамках соответствующей \textit{предметной области}}
        \scnfileitem{\textbf{\textit{онтологическая локализация решения задач}} в \textit{интеллектуальных компьютерных системах}, предполагающая \uline{локализацию} \textit{области действия} каждого хранимого в памяти \textit{метода} и каждого \textit{информационного агента} в соответствии с \textit{онтологической моделью} обрабатываемой \textit{базы знаний}. Чаще всего, такой \textit{областью действия} является одна из \textit{предметных областей} либо одна из \textit{предметных областей} вместе с соответствующей ей \textit{онтологии}}
        \scnfileitem{\textbf{\textit{онтологическая модель интерфейса}} \textit{интеллектуальной компьютерной системы}}
        \begin{scnindent}
            \begin{scnrelfromlist}{входить в состав}
                \scnfileitem{онтологическое описание \textit{синтаксиса} всех языков, используемых \textit{интеллектуальной компьютерной системой} для общения с \textit{внешними субъектами}}
                \scnfileitem{онтологическое описание \textit{денотационной семантики} каждого языка, используемого \textit{интеллектуальной компьютерной системой} для \textit{общения} с внешними \textit{субъектами}}
                \scnfileitem{семейство \textit{информационных агентов}, обеспечивающих \textit{синтаксический анализ}, \textit{семантический анализ} (перевод на внутренний смысловой язык) и \textit{понимание} (погружение в \textit{базу знаний}) любого введенного \textit{сообщения}, принадлежащего любому \textit{внешнему языку}, полное онтологическое описание которого находится в базе знаний \textit{интеллектуальной компьютерной системы}}
                \scnfileitem{семейство \textit{информационных агентов}, обеспечивающих \textit{синтез сообщений}, которые (1) адресуются внешним субъектам, с которыми общается \textit{интеллектуальная компьютерная система}, (2) \textit{семантически эквивалентны} заданным \textit{фрагментам базы знаний} интеллектуальной компьютерной системы, определяющим \textit{смысл} передаваемых \textit{сообщений}, (3) принадлежат одному из \textit{внешних языков}, полное онтологическое описание которого находится в \textit{базе знаний} интеллектуальной компьютерной системы}
            \end{scnrelfromlist}
        \end{scnindent}
        \scnfileitem{\textit{семантически дружественный характер пользовательского интерфейса}, обеспечиваемый (1) формальным описание в базе знаний средства управления пользовательским интерфейсом и (2) введением в состав \textit{интеллектуальной компьютерной системы} соответствующих help-подсистем, обеспечивающих существенное снижение языкового барьера между пользователями и \textit{интеллектуальными компьютерными системами}, что существенно повысит эффективность \textit{эксплуатации интеллектуальных компьютерных систем}}
        \scnfileitem{\textit{минимизация негативного влияния человеческого фактора} на эффективность \textit{эксплуатации} \textit{интеллектуальных компьютерных систем} благодаря реализации интероперабельного (партнерского) стиля взаимодействия не только между самими \textit{интеллектуальными компьютерными системами}, но также и между \textit{интеллектуальными компьютерными системами} и их пользователями. Ответственность за качество совместной деятельности должно быть распределено между всеми партнерами}
        \scnfileitem{\textbf{\textit{мультимодальность}} (гибридный характер) \textit{интеллектуальной компьютерной системы}}
        \begin{scnindent}
            \begin{scnrelfromlist}{предполагает}
                \scnfileitem{многообразие \textit{видов знаний}, входящих в состав \textit{базы знаний} интеллектуальной компьютерной системы}
                \scnfileitem{многообразие \textit{моделей решения задач}, используемых \textit{решателем задач} интеллектуальной компьютерной системы}
                \scnfileitem{многообразие \textit{сенсорных каналов}, обеспечивающих \textit{мониторинг} состояния \textit{внешней среды} интеллектуальной компьютерной системы}
                \scnfileitem{многообразие \textit{эффекторов}, осуществляющих \textit{воздействие на внешнюю среду}}
                \scnfileitem{многообразие \textit{языков общения} с другими субъектами (с пользователями, с интеллектуальными компьютерными системами)}
            \end{scnrelfromlist}
        \end{scnindent}
        \scnfileitem{\textbf{\textit{внутренняя семантическая совместимость}} между компонентами \textit{интеллектуальной компьютерной системы} (то есть максимально возможное введение общих, совпадающих \textit{понятий} для различных фрагментов хранимой \textit{базы знаний}), являющаяся формой \textbf{\textit{конвергенции}} и \textit{глубокой интеграции} внутри \textit{интеллектуальной компьютерной системы} для различного вида \textit{знаний} и различных \textit{моделей решения задач}, что обеспечивает эффективную реализацию \textit{мультимодальности интеллектуальной компьютерной системы}}
        \scnfileitem{\textbf{\textit{внешняя семантическая совместимость}} между различными \textit{интеллектуальными компьютерными системами}, выражающаяся не только в общности используемых \textit{понятий}, но и в общности базовых \textit{знаний} и являющаяся необходимым условием обеспечения высокого уровня \textit{интероперабельности} интеллектуальных компьютерных систем}
        \scnfileitem{ориентация на использование \textit{интеллектуальных компьютерных систем} как \textit{когнитивных агентов} в составе \textbf{\textit{иерархических многоагентных систем}}}
        \scnfileitem{фрактальный характер (масштабируемое самоподобие) структуризации иерархических коллективов интеллектуальных компьютерных систем нового поколения}
        \scnfileitem{\textbf{\textit{платформенная независимость} интеллектуальных компьютерных систем}}
        \begin{scnindent}
            \begin{scnrelfromlist}{предполагает}
                \scnfileitem{четкую \textit{стратификацию} каждой \textit{интеллектуальной компьютерной системы} (1) на \textit{логико-семантическую модель}, представленную ее \textit{базой знаний}, которая содержит не только \textit{декларативные знания}, но и знания, имеющие \textit{операционную семантику}, и (2) на \textit{платформу}, обеспечивающую \textit{интерпретацию} указанной \textit{логико-семантической модели}}
                \scnfileitem{универсальность указанной \textit{платформы} интерпретации \textit{логико-семантической модели интеллектуальной компьютерной системы}, что дает возможность каждой такой \textit{платформе} обеспечивать интерпретацию любой \textit{логико-семантической модели интеллектуальной компьютерной системы}, если эта модель представлена на том же \textit{универсальном языке смыслового представления информации}}
                \scnfileitem{многообразие вариантов реализации \textit{платформ интерпретации логико-семантических моделей интеллектуальных компьютерных систем} --- как вариантов, программно реализуемых на \textit{современных компьютерах}, так и вариантов, реализуемых в виде \textit{универсальных компьютеров нового поколения}, ориентированных на использование в \textit{интеллектуальных компьютерных системах нового поколения} (такие компьютеры мы назвали \textit{ассоциативными семантическими компьютерами})}
                \scnfileitem{легко реализуемую возможность переноса (переустановки) логико-семантической модели (\textit{базы знаний}) любой \textit{интеллектуальной компьютерной системы} на любую другую \textit{платформу интерпретации логико-семантических моделей}}
            \end{scnrelfromlist}
        \end{scnindent}
        \scnfileitem{изначальная ориентация \textit{интеллектуальных компьютерных систем нового поколения} на использование \textbf{\textit{универсальных ассоциативных семантических компьютеров}} (компьютеров нового поколения) в качестве \textit{платформы интерпретации логико-семантических моделей} (баз знаний) \textit{интеллектуальных компьютерных систем}}
    \end{scnrelfromlist}
    \scntext{примечание}{В настоящее время разработано большое количество различного вида моделей решения задач, моделей представления и обработки знаний различного вида. Но в разных \textit{интеллектуальных компьютерных системах} могут быть востребованы разные комбинации этих моделей. При разработке и реализации различных \textit{интеллектуальных компьютерных систем} соответствующие методы и средства должны гарантировать \textit{логико-семантическую совместимость} разрабатываемых компонентов и, в частности, их способность использовать общие \textit{информационные ресурсы}. Для этого, очевидно, необходима \textit{унификация} указанных моделей.}
    \scntext{примечание}{\uline{Многообразие} различных видов интеллектуальных компьютерных систем и, соответственно, многообразие используемых ими комбинаций моделей представления знаний и решения задач определяется:
        \begin{scnitemize}
            \item многообразием назначения интеллектуальных компьютерных систем и вида окружающей их среды;
            \item многообразием различных видов хранимых знаний;
            \item многообразием моделей обработки знаний и решений задач;
            \item многообразием различных видов сенсорных и эффекторных подсистем.
        \end{scnitemize}}

    \scnheader{аспекты \textit{совместимости} моделей представления и обработки знаний в \textit{интеллектуальных компьютерных системах}}
    \scnhaselement{синтаксический аспект}
    \scnhaselement{семантический аспект}
    \begin{scnindent}
    \scntext{примечание}{Cогласованность систем понятий, их денотационной семантики}
    \end{scnindent}
    \scnhaselement{функциональный аспект}
    \begin{scnindent}
        \scnidtf{операционный аспект}
    \end{scnindent}
    
    \scnheader{следует отличать*}
    \begin{scnhaselementset}
    	\scnitem{\textit{совместимость} между компонентами \textit{интеллектуальных компьютерных систем}}
        \scnitem{\textit{совместимость} между верхним логико-семантическим уровнем используемых моделей представления и обработки знаний и различными уровнями их интерпретации вплоть до аппаратного уровня}
        \scnitem{\textit{совместимость} между индивидуальными интеллектуальными компьютерными системами}
    	\scnitem{\textit{совместимость} между индивидуальными интеллектуальными компьютерными системами и их пользователями}
    	\scnitem{\textit{совместимость} между коллективами интеллектуальных компьютерных системам}
    \end{scnhaselementset}

	\scnheader{следует отличать*}
	\begin{scnhaselementset}
		\scnitem{данные}
		\begin{scnindent}
			\scnidtf{информационная конструкция, обрабатываемая с помощью программы традиционного языка программирования}
		\end{scnindent}
		\scnitem{знание}
		\begin{scnindent}
			\scnidtf{семантически целостный фрагмент базы знаний}
		\end{scnindent}
	\end{scnhaselementset}
	\begin{scnindent}
		\scntext{отличие}{Для каждого знания всегда известен язык, на котором это знание представлено и денотационная семантика которого задана. При этом указанный язык имеет достаточно большую семантическую мощность, а в идеале является универсальным языком. В отличие от этого структуризация данных для традиционных программ осуществляется в целях упрощения самих этих программ и, следовательно, для разных программ в общем случае осуществляется по-разному. Таким образом, при разработке традиционных программ представление обрабатываемых данных осуществляется в общем случае на разных языках, денотационная семантика которых нигде не документируется и известна только разработчикам программ. Другими словами, данные для разных программ имеют денотационную семантику не только разную, но еще и априори неизвестную. По сути это форма проявления \textit{вавилонского столпотворения} в традиционных языках программирования, которые образно говоря \scnqq{хромают на одну ногу}, формализуя методы обработки информации, но не формализуя семантику обрабатываемой информации.}
	\end{scnindent} 

\end{scnsubstruct}
