\begin{SCn}
	\scnsectionheader{Предметная область и онтология онтологических моделей интерфейсов интеллектуальных компьютерных систем, основанных на смысловом представлении информации}

	\begin{scnsubstruct}

		\scnrelfrom{соавтор}{Садовский М.Е.}
		\scnheader{Предметная область онтологических моделей интерфейсов
			интеллектуальных компьютерных систем, основанных на смысловом представлении
			информации}
		\scniselement{предметная область}

		\begin{scnhaselementrolelist}{класс объектов исследования}
			\scnitem{подход к построению пользовательского интерфейса}
		\end{scnhaselementrolelist}

		\begin{scnhaselementrolelist}{класс объектов исследования}
			\scnitem{подход к построению пользовательского интерфейса на
				основе специализированных языков описания}
			\scnitem{контекстно-зависимый подход к построению
				пользовательского интерфейса;подход к построению пользовательского интерфейса
				на основе данных}
			\scnitem{онтологический подход к построению пользовательского
				интерфейса}
			\scnitem{онтологический подход к построению пользовательского
				интерфейса на основе логико-семантической модели}
		\end{scnhaselementrolelist}

		\scnheader{подход к построению пользовательского интерфейса}
		\scnsuperset{подход к построению пользовательского интерфейса на основе
			специализированных языков описания}
		\scnsuperset{контекстно-зависимый подход к построению пользовательского
			интерфейса}
		\scnsuperset{подход к построению пользовательского интерфейса на основе
			данных}
		\scnsuperset{онтологический подход к построению пользовательского
			интерфейса}
			\begin{scnindent}
				\scnsuperset{онтологический подход к построению пользовательского
					интерфейса на основе логико-семантической модели}
			\end{scnindent}

		\scnheader{подход к построению пользовательского интерфейса на основе
			специализированных языков описания}

		\scntext{пояснение}{подход на основе специализированных языков
			описания предполагает представление конкретного пользовательского интерфейса в
			платформенно независимом виде. В качестве примеров языков описания интерфейса
			можно привести UIML (\cite{ABRAMS19991695}), UsiXML (\cite{UsiXML}), XForms
			(\cite{XForms}) и FXML (\cite{fxml}). Ключевой идеей представленных языков
			является создание модели диалогов и форм интерфейса в независимом от
			используемой технологии виде, описание визуальных элементов, а также
			взаимосвязей между ними и их свойств для создания конкретного пользовательского
			интерфейса.}
		\begin{scnrelfromset}{недостатки современного состояния}
			\scnfileitem{как  правило,  спецификация  модели  интерфейса является
				неполной,  что	приводит  к  сложности автоматизации процесса генерации
				пользовательского интерфейса}
			\scnfileitem{как правило, созданные модели специфичны для конкретной
				платформы либо конкретной реализации пользовательского интерфейса, что
				препятствует их повторному использованию для других целей.}
			\scnfileitem{решения,	которые   предлагают   платформенно независимое
				описание,  позволяют  генерировать лишь  простые  ограниченные  по
				функционалу пользовательские интерфейсы (приложения-опросники, диаграммы и
				т.д.).}
		\end{scnrelfromset}

		\scnheader{контекстно-зависимый подход к построению пользовательского
			интерфейса}
		\scntext{пояснение}{Контекстно-зависимый подход интегрирует
			использование структурного описания интерфейса на основе языков описания с
			поведенческой спецификацией, то есть генерация интерфейса основывается на
			действиях пользователя. В рамках подхода специфицируются переходы между
			различными видами конкретного пользовательского интерфейса. В качестве примеров
			языков, реализующих идеи такого подхода можно привести CAP3 (\cite{CAP3}) и
			MARIA (\cite{MARIA}).}
		\scnheader{подход к построению пользовательского интерфейса на основе
			данных}
		\scntext{пояснение}{подход на основе данных или моделеориентированный
			подход использует модель предметной области в качестве основы для создания
			пользовательских интерфейсов. Указанный подход реализуется в таких проектах,
			как JANUS (\cite{JANUS}) и Mecano (\cite{Mecano}).}
		\scnheader{онтологический подход к построению пользовательского
			интерфейса}
		\scntext{пояснение}{cуществующие онтологические подходы как правило
			основаны на представленных ранее подходах и используют онтологии в качестве
			способа представления информации о конкретном пользовательском интерфейсе.
			Например, по аналогии с подходом на основе специализированных языков описания,
			был предложен фреймворк  (\cite{ui_model-based-approach}), использующий
			онтологию для описания пользовательского интерфейса на основе понятий,
			хранящихся в базе знаний. По аналогии с контекстно-зависимым подходом в рамках
			работы \cite{gaulke} используется модель предметной области совместно с моделью
			пользовательского интерфейса, ассоциированная с онтологией действий. Проект
			ActiveRaUL (\cite{ActiveRaUL}) совмещает UIML с моделеориентированным подходом.
			В рамках данного проекта онтологическая модель предметной области
			сопоставляется с онтологическим представлением пользовательского интерфейса.
			Подход, предложенный в работе \cite{hitz}, совмещает данные приложения с
			онтологией пользовательского интерфейса для создания единого описания в базе
			знаний с целью последующей автоматической генерации различных вариантов
			интерфейса для приложений-опросников с готовыми сценариями взаимодействия с
			пользователем. Следует также отметить работы \cite{vladivostok1} и
			\cite{vladivostok2}, в рамках которых предложена концепция, позволяющая
			объединить однородную по содержанию информацию в компоненты модели интерфейса,
			освободить разработчика интерфейса от кодирования и формировать информацию для
			каждого компонента модели интерфейса с помощью редакторов, управляемых
			соответствующими моделями онтологий.}
		\begin{scnrelfromset}{недостатки современного состояния}
			\scnfileitem{актуальна проблема совместимости различных онтологий в
				рамках единой системы}
			\scnfileitem{отсутствие способности адаптироватьсяк запросам
				пользователя и анализировать его действия длясамостоятельного
				совершенствования.}
		\end{scnrelfromset}

		\begin{scnrelfromset}{достоинства}
			\scnfileitem{позволяет интегрировать ранее предложенные подходы за счет
				единого способа представления знаний.}
			\scnfileitem{позволяет создать наиболее полное описание различных
				аспектов пользовательского интерфейса.}
			\scnfileitem{упрощает повторное использование интерфейса.}
		\end{scnrelfromset}

		\scnheader{онтологический подход к построению пользовательского
			интерфейса на основе логико-семантической модели}
		\scntext{примечание}{для проектирования пользовательских интерфейсов
			предлагается использовать \textbf{\textit{онтологический подход к построению
					пользовательского интерфейса на основе логико-семантической модели}},
			обладающий рядом важных достоинств.}
		\begin{scnrelfromset}{достоинства}
			\scnfileitem{возможность переноса пользовательских интерфейсов с одной
				платформы реализации на другую.}
			\scnfileitem{наличие общих    принципов построения пользовательских
				интерфейсов, что позволяет повторно использовать уже разработанные компоненты
				и  снижает сроки  обучения  пользователя  новым  для  него пользовательским
				интерфейсам.}
			\scnfileitem{возможность модификации пользовательского интерфейса в
				процессе работы.}
			\scnfileitem{возможность гибкой адаптации пользовательского интерфейса
				под нужды конкретного пользователя.}
		\end{scnrelfromset}

		\scntext{пояснение}{подход предполагает создание полной семантической
			модели интерфейса, содержащей  лексическое описание  интерфейса(описание
			компонентов, из которых формируется интерфейс), синтаксическое	описание
			интерфейса(правила  формирования  корректного  и  полного интерфейса из его
			компонентов), но также и его семантическое описание (знание о том, знаком какой
			сущности является отображаемый компонент). При этом семантическое описание
			также включает всебя назначение, область применения компонентов интерфейса,
			описание интерфейсной деятельности пользователя.}
		\begin{scnrelfromset}{основные принципы}
			\scnfileitem{пользовательский интерфейс представляет собой
				специализированную ostis-систему, ориентированную на решение интерфейсных
				задач,и имеющую в составе базу знаний и машину обработки знаний
				пользовательского интерфейса,что даёт возможность пользователю адресовать
				пользовательскому интерфейсу различного рода вопросы}
			\scnfileitem{используется онтологический подход к проектированию
				пользовательского интерфейса, чтоспособствует чёткому разделению деятельности
				различных разработчиков пользовательских интерфейсов, а также унификации
				принципов проектирования}
			\scnfileitem{используется SC-код в качестве формальногоязыка
				внутреннего представления знаний (онтологий, предметных областей и др.),
				благодарячему обеспечивается легкость интерпретацииэтих знаний и системой, и
				человеком - пользователем или разработчиком, а также однозначность восприятия
				этой информации ими}
			\scnfileitem{средствами SC-кода с помощью соответствующих онтологий
				описываются синтаксис и семантика всевозможных используемых внешнихязыков}
			\scnfileitem{трансляции с внутреннего языка на внешний иобратно
				организовываются так, чтобы механизмы трансляции не зависели от внешнего языка,
				для реализации нового специализированногоязыка в таком случае необходимо будет
				толькоописать его синтаксис и семантику, универсальная же модель трансляции не
				будет зависеть отданного описания}
			\scnfileitem{предполагается выбор стилей визуализации,осуществляемый в
				зависимости от вида отображаемых знаний (например, использование различных
				элементов визуализации для одних видов знаний и других - для других), что
				позволит пользователю быстрее обучаться новымспециализированным языкам, а также
				сделатьпростым и понятным отображение знаний}
			\scnfileitem{модель пользовательского интерфейса строитсянезависимо от
				реализации платформы интерпретации такой модели, что позволяет легкопереносить
				разработанную модель на разныеплатформы.}
		\end{scnrelfromset}
		\bigskip
	\end{scnsubstruct}
\end{SCn}
%\scnsourcecomment{Завершили Раздел \scnqq{Предметная область и онтология логико-семантических моделей интерфейсов компьютерных систем, основанных на смысловом представлении информации}}
