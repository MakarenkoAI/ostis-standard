\scsectionfamily{Часть 3 Стандарта OSTIS. Многоагентные решатели задач интеллектуальных компьютерных систем нового поколения}
\label{part_solvers}

\scsection[
    \protect\scneditor{Шункевич Д.В.}
    \protect\scnmonographychapter{Глава 3.2. Ситуационное управление обработкой знаний в интеллектуальных компьютерных системах нового поколения}
    ]{Предметная область и онтология решателей задач ostis-систем}
\label{sd_ps}
\begin{SCn}
\scnsectionheader{Предметная область и онтология решателей задач ostis-систем}
\begin{scnsubstruct}
\begin{scnrelfromlist}{дочерний раздел}
	\scnitem{Предметная область и онтология действий, задач, планов, протоколов и методов, реализуемых ostis-системой, а также внутренних агентов, выполняющих эти действия}
	\scnitem{Предметная область и онтология Базового языка программирования ostis-систем}
	\scnitem{Предметная область и онтология искусственных нейронных сетей и соответствующая им предметная область и онтология действий по обработке искусственных нейронных сетей}
\end{scnrelfromlist}

\scnheader{решатель задач ostis-системы}
\scnidtf{совокупность всех навыков, которыми обладает ostis-система на текущий момент времени}
\scnrelto{семейство подмножеств}{навык}
\scntext{примечание}{Предлагаемый в рамках \textit{Технологии OSTIS} подход к построению решателей задач позволяет обеспечить их модифицируемость, что, в свою очередь, позволяет \textit{ostis-системе} при необходимости легко приобретать новые \textit{навыки}, модифицировать (совершенствовать) уже имеющиеся и даже избавляться от некоторых навыков с целью повышения производительности системы. Таким образом, имеет смысл говорить не о жестко фиксированном решателе задач, который разрабатывается один раз при создании первой версии системы и далее не меняется, а о совокупности навыков, фиксированной в каждый текущий момент времени, но постоянно эволюционирующей.}
\scnsuperset{объединенный решатель задач ostis-системы}
	\begin{scnindent}
		\scnidtf{полный решатель задач ostis-системы}
		\scnidtf{интегрированный решатель задач ostis-системы}
		\scnidtf{решатель задач ostis-системы, реализующий все ее функциональные возможности, как основные, так и вспомогательные}
		\scntext{пояснение}{В общем случае \textit{объединенный решатель задач ostis-системы} решает задачи, связанные с:
		\begin{itemize}
			\item обеспечением основных функциональных возможностей системы (например, решение явно сформулированных задач по требованию пользователя);
			\item обеспечением корректности и оптимизацией работы самой ostis-системы (перманентно на протяжении всего жизненного цикла ostis-системы);
			\item обеспечением повышения квалификации конечных пользователей и разработчиков ostis-системы;
			\item обеспечением автоматизации развития и управления развитием ostis-системы.
		\end{itemize}
		}
	\end{scnindent}
\scnsuperset{гибридный решатель задач ostis-системы}
	\begin{scnindent}
		\scnidtf{решатель задач ostis-системы, реализующий две и более модели решения задач}
	\end{scnindent}
		
\scnheader{машина обработки знаний}
\scnsubset{sc-агент}
\scntext{пояснение}{Под \textit{машиной обработки знаний} будем понимать совокупность интерпретаторов всех \textit{навыков}, составляющих некоторый \textit{решатель задач}. С учетом многоагентного подхода к обработке информации, используемого в рамках Технологии OSTIS, \textit{машина обработки знаний} представляет собой \textit{sc-агент} (чаще всего --- \textit{неатомарный sc-агент}), в состав которого входят более простые sc-агенты, обеспечивающие интерпретацию соответствующего множества \textit{методов}. Таким образом, \textit{машина обработки знаний} в общем случае представляет собой иерархическую систему \textit{sc-агентов}.}

\scnheader{решатель задач ostis-системы}
\scnhaselement{Решатель задач Метасистемы IMS.ostis}
\scnsuperset{решатель задач вспомогательной ostis-системы}
	\begin{scnindent}
		\scnsuperset{решатель задач интерфейса компьютерной системы}
			\begin{scnindent}
				\begin{scnsubdividing}
					%TODO: check by human--->
					\scnitem{решатель задач пользовательского интерфейса компьютерной системы}
					\scnitem{решатель задач интерфейса компьютерной системы с другими компьютерными системами}
					\scnitem{решатель задач интерфейса компьютерной системы с окружающей средой}
					%<---TODO: check by human
				\end{scnsubdividing}
			\end{scnindent}
		\scnsuperset{решатель задач ostis-подсистемы поддержки проектирования компонентов определенного класса}
		\begin{scnindent}
			\scnsuperset{решатель задач ostis-подсистемы поддержки проектирования баз знаний}
				\begin{scnindent}
					\scnsuperset{решатель задач повышения качества базы знаний}
						\begin{scnindent}
							\scnsuperset{решатель задач верификации базы знаний}
								\begin{scnindent}
									\scnsuperset{решатель задач поиска и устранения некорректностей в базе знаний}
									\scnsuperset{решатель задач поиска и устранения неполноты}
								\end{scnindent}
							\scnsuperset{решатель задач оптимизации структуры базы знаний}
							\scnsuperset{решатель задач выявления и устранения информационного мусора}
						\end{scnindent}
				\end{scnindent}
			\scnsuperset{решатель задач ostis-подсистемы поддержки проектирования решателей задач ostis-систем}
			\begin{scnindent}
					\begin{scnsubdividing}
						%TODO: check by human--->
						\scnitem{решатель задач ostis-подсистемы поддержки проектирования программ обработки знаний}
						\scnitem{решатель задач ostis-подсистемы поддержки проектирования агентов обработки знаний}
						%<---TODO: check by human
					\end{scnsubdividing}
				\end{scnindent}
		\end{scnindent}
		\scnsuperset{решатель задач подсистемы управления проектирования компьютерных систем и их компонентов}
	\end{scnindent}
\scnsuperset{решатель задач самостоятельной ostis-системы}

\scnheader{решатель задач ostis-системы}
\scnsuperset{решатель задач с использованием хранимых методов}
	\begin{scnindent}
		\scnidtf{решатель, способный решать задачи тех классов, для которых на данный момент времени известен соответствующий метод решения}
		\scnsuperset{решатель задач на основе нейросетевых моделей}
		\scnsuperset{решатель задач на основе генетических алгоритмов}
		\scnsuperset{решатель задач на основе императивных программ}
			\begin{scnindent}
				\scnsuperset{решатель задач на основе процедурных программ}
				\scnsuperset{решатель задач на основе объектно-ориентированных программ}
			\end{scnindent}
		\scnsuperset{решатель задач на основе декларативных программ}
			\begin{scnindent}
				\scnsuperset{решатель задач на основе логических программ}
				\scnsuperset{решатель задач на основе функциональных программ}
			\end{scnindent}
	\end{scnindent}
\scnsuperset{решатель задач в условиях, когда метод решения задач данного класса в текущий момент времени не известен}
	\begin{scnindent}
		\scnidtf{решатель, реализующий стратегии решения задач, позволяющие породить метод решения задачи, который в текущий момент времени не известен ostis-системе}
		\scnidtf{решатель, использующий для решения задач метаметоды, соответствующие более общим классам задач по отношению к заданной}
		\scnidtf{решатель задач, позволяющий породить метод, который является частным по отношению к какому-либо известному ostis-системе методу и интерпретируется соответствующей машиной обработки знаний}
		\scnsuperset{решатель, реализующий стратегию поиска путей решения задачи в глубину}
		\scnsuperset{решатель, реализующий стратегию поиска путей решения задачи в ширину}
		\scnsuperset{решатель, реализующий стратегию проб и ошибок}
		\scnsuperset{решатель, реализующий стратегию разбиения задачи на подзадачи}
		\scnsuperset{решатель, реализующий стратегию решения задач по аналогии}
		\scnsuperset{решатель, реализующий концепцию интеллектуального пакета программ}
	\end{scnindent}

\scnheader{машина обработки знаний}
\scnsuperset{машина логического вывода}
\begin{scnindent}
	\scnsuperset{машина дедуктивного вывода}
		\begin{scnindent}
			\scnsuperset{машина прямого дедуктивного вывода}
			\scnsuperset{машина обратного дедуктивного вывода}
		\end{scnindent}
	\scnsuperset{машина индуктивного вывода}
	\scnsuperset{машина абдуктивного вывода}
	\scnsuperset{машина нечеткого вывода}
	\scnsuperset{машина вывода на основе логики умолчаний}
	\scnsuperset{машина логического вывода с учетом фактора времени}
\end{scnindent}

\scnheader{решатель задач ostis-системы}
\scnsuperset{решатель задач информационного поиска}
	\begin{scnindent}
		\begin{scnsubdividing}
			%TODO: check by human--->
			\scnitem{решатель задач поиска информации, удовлетворяющей заданным критериям}
			\scnitem{решатель задач поиска информации, не удовлетворяющей заданным критериям}
			%<---TODO: check by human
		\end{scnsubdividing}
	\end{scnindent}
\scnsuperset{решатель явно сформулированных задач}
	\begin{scnindent}
		\scnidtf{решатель задач, для которых явно сформулирована цель}
		\scnsuperset{решатель задач поиска или вычисления значений заданного множества величин}
		\scnsuperset{решатель задач установления истинности заданного логического высказывания в рамках заданной формальной теории}
		\scnsuperset{решатель задач формирования доказательства заданного высказывания в рамках заданной формальной теории}
		\scnsuperset{машина верификации ответа на указанную задачу}
		\scnsuperset{машина верификации решения указанной задачи}
			\begin{scnindent}
				\scnsuperset{машина верификации доказательства заданного высказывания в рамках заданной формальной теории}
			\end{scnindent}
	\end{scnindent}
\scnsuperset{решатель задач классификации сущностей}
	\begin{scnindent}
		\scnsuperset{машина соотнесения сущности с одним из заданного множества классов}
		\scnsuperset{машина разделения множества сущностей на классы по заданному множеству признаков}
	\end{scnindent}
\scnsuperset{решатель задач синтеза информационных конструкций}
	\begin{scnindent}
		\scnsuperset{решатель задач синтеза естественно-языковых текстов}
		\scnsuperset{решатель задач синтеза изображений}
		\scnsuperset{решатель задач синтеза сигналов}
		\begin{scnindent}
			\scnsuperset{решатель задач синтеза речи}
		\end{scnindent}
	\end{scnindent}
\scnsuperset{решатель задач анализа информационных конструкций}
	\begin{scnindent}
		\scnsuperset{решатель задач анализа естественно-языковых текстов}
			\begin{scnindent}
				\scnsuperset{решатель задач понимания естественно-языковых текстов}
				\scnsuperset{решатель задач верификации естественно-языковых текстов}
			\end{scnindent}
		\scnsuperset{решатель задач анализа изображений}
			\begin{scnindent}
				\scnsuperset{решатель задач сегментации изображений}
				\scnsuperset{решатель задач понимания изображений}
			\end{scnindent}
		\scnsuperset{решатель задач анализа сигналов}
			\begin{scnindent}
				\scnsuperset{решатель задач анализа речи}
					\begin{scnindent}
						\scnsuperset{решатель задач понимания речи}
					\end{scnindent}
			\end{scnindent}
	\end{scnindent}

\bigskip
\end{scnsubstruct}
\end{SCn}


\scsubsection[
    \protect\scneditor{Шункевич Д.В.}
    \protect\scnmonographychapter{Глава 3.2. Ситуационное управление обработкой знаний в интеллектуальных компьютерных системах нового поколения}
    ]{Предметная область и онтология действий, задач, планов, протоколов и методов, реализуемых ostis-системой, а также внутренних агентов, выполняющих эти действия}
\label{sd_agents}
\begin{SCn}
\scnsectionheader{Предметная область и онтология действий, задач, планов, протоколов и методов, реализуемых ostis-системой, а также внутренних агентов, выполняющих эти действия}
\begin{scnsubstruct}

\scnheader{Предметная область и онтология действий, задач, планов, протоколов и методов, реализуемых ostis-системой в ее памяти, а также внутренних агентов, выполняющих эти действия}
\scniselement{предметная область}
\begin{scnhaselementrolelist}{максимальный класс объектов исследования}
    \scnitem{действие в sc-памяти}
    \scnitem{абстрактный sc-агент}
    \scnitem{sc-агент}
\end{scnhaselementrolelist}
\begin{scnhaselementrolelist}{класс объектов исследования}
    \scnitem{абстрактный sc-агент, не реализуемый на Языке SCP}
    \scnitem{абстрактный sc-агент, реализуемый на Языке SCP}
    \scnitem{Абстрактный программный sc-агент}
    \scnitem{неатомарный абстрактный sc-агент}
    \scnitem{атомарный абстрактный sc-агент}
    \scnitem{платформенно-независимый абстрактный sc-агент}
    \scnitem{платформенно-зависимый абстрактный sc-агент}
    \scnitem{внутренний абстрактный sc-агент}
    \scnitem{эффекторный абстрактный sc-агент}
    \scnitem{рецепторный абстрактный sc-агент}
    \scnitem{абстрактный sc-агент, не реализуемый на Языке SCP}
    \scnitem{абстрактный sc-агент, реализуемый на Языке SCP}
    \scnitem{абстрактный sc-агент интерпретации scp-программ}
    \scnitem{абстрактный программный sc-агент}
    \scnitem{абстрактный программный sc-агент, реализуемый на Языке SCP}
    \scnitem{абстрактный sc-метаагент}
    \scnitem{sc-агент}
    \scnitem{активный sc-агент}
    \scnitem{описание поведения sc-агента}
    \scnitem{тип блокировки}
    \scnitem{полная блокировка}
    \scnitem{блокировка на любое изменение}
    \scnitem{блокировка на удаление}
\end{scnhaselementrolelist}
\begin{scnhaselementrolelist}{исследуемое отношение}
    \scnitem{декомпозиция абстрактного sc-агента*}
    \scnitem{ключевые sc-элементы sc-агента*}
    \scnitem{программа sc-агента*}
    \scnitem{первичное условие инициирования*}
    \scnitem{условие инициирования и результат*}
    \scnitem{блокировка*}
\end{scnhaselementrolelist}

\scnheader{обработка информации в ostis-системах}
\begin{scnrelfromlist}{принципы, лежащие в основе}
    \scnfileitem{В основе решателя задач каждой \textit{ostis-системы} лежит многоагентная система, агенты которой взаимодействуют между собой \uline{только}(!) через общую для них \textit{sc-память} посредством спецификации в этой памяти выполняемых ими \textit{действий в sc-памяти}. При этом пользователи \textit{ostis-системы} также считаются агентами этой системы. Кроме того, \textit{sc-агенты} делятся на внутренние, рецепторные и эффекторные. Взаимодействие между агентами через общую \textit{sc-память} сводится к следующим видам действий:
        \begin{scnenumerate}
            \item К использованию общедоступной для соответствующей группы sc-агентов части хранимой базы знаний.
            \item К формированию (генерации) новых фрагментов базы знаний и/или к корректировке (редактированию) каких-либо фрагментов доступной части базы знаний.
            \item К интеграции (погружению) новых и/или обновленных фрагментов в состав доступной части базы знаний.
        \end{scnenumerate}
        Подчеркнем, что sc-агенты не общаются между собой напрямую путем отправки сообщений, как это делается в большинстве современных подходов к построению многоагентных систем. Кроме того, sc-агенты имеют доступ к общей для них базе знаний за счет чего гарантируется семантическая совместимость (взаимопонимание) между агентами, включая и пользователей ostis-систем.}
    \scnfileitem{Пользователь \textit{ostis-системы} не может сам непосредственно выполнить какое-либо действие в \mbox{sc-памяти}, но он может средствами пользовательского интерфейса инициировать построение (генерацию, формирование в \textit{sc-памяти}) \textit{sc-текста}, являющегося спецификацией \textit{действия в \mbox{sc-памяти}}, выполняемого либо одним \textit{атомарным sc-агентом} за один акт, либо одним \textit{атомарным sc-агентом} за несколько актов, либо коллективом \textit{sc-агентов} (\textit{неатомарным sc-агентом}). В спецификации каждого такого \textit{действия в sc-памяти}, инициированного пользователем, этот пользователь указывается как заказчик этого действия. Таким образом, пользователь \textit{ostis-системы} дает поручения (задания, команды) \textit{sc-агентам} этой системы на выполнение различных специфицируемых им действий в \textit{sc-памяти}.}
    \scnfileitem{Каждый \textit{sc-агент}, выполняя некоторое \textit{действие в sc-памяти}, должен помнить, что \textit{sc-память}, над которой он работает, является общим ресурсом не только для него, но и для всех остальных \textit{\mbox{sc-агентов}}, работающих над этой же \textit{sc-памятью}. Поэтому \textit{sc-агент} должен соблюдать определенную этику поведения в коллективе таких \textit{sc-агентов}, которая должна минимизировать помехи, которые он создает другим \textit{sc-агентам}.}
    \scnfileitem{Деятельность каждого агента \textit{ostis-системы} дискретна и представляет собой множество элементарных действий (актов). При этом при выполнении каждого акта агент может устанавливать блокировки нескольких типов на фрагменты базы знаний. Указанные блокировки позволяют запретить другим агентам изменять указанный фрагмент базы знаний или вообще сделать его невидимым для других агентов. Блокировки устанавливаются самим агентом при выполнении соответствующего акта и снимаются им же на последнем этапе выполнения этого акта или раньше, если это возможно.}
    \scnfileitem{Если некий \textit{sc-агент} выполняет некоторое \textit{действие в sc-памяти}, то он на время выполнения этого действия может:\\
        \begin{scnenumerate}
            \item Запретить другим \textit{sc-агентам} изменять состояние некоторых sc-элементов, хранимых в \textit{sc-памяти} --- удалять их, изменять тип.
            \item Запретить другим \textit{sc-агентам} добавлять или удалять элементы некоторых множеств, обозначаемых соответствующими \textit{sc-узлами}.
            \item Запретить другим \textit{sc-агентам} доступ на просмотр некоторых \textit{sc-элементов}, то есть эти \textit{\mbox{sc-элементы}} становятся полностью невидимыми (полностью заблокированными) для других \textit{sc-агентов}, но только на время выполнения соответствующего действия.
        \end{scnenumerate}
        Указанные блокировки должны быть полностью сняты до завершения выполнения соответствующего действия. Подчеркнем, что в число \textit{sc-элементов}, блокируемых на время выполнения некоторого действия, в основном входят атомарные и неатомарные связки, и не должны входить \textit{sc-узлы}, обозначающие бесконечные классы каких-либо сущностей, и, тем более, sc-узлы, обозначающие различные понятия (ключевые классы различных предметных областей).\\
        Этичное (неэгоистичное) поведение \textit{sc-агента}, касающееся блокировки \textit{sc-элементов} (то есть ограничения к ним доступа другим \textit{sc-агентам}) предполагает соблюдение следующих правил:\\
        \begin{scnenumerate}
            \item Не следует блокировать больше \textit{sc-элементов}, чем это необходимо для решения задачи.
            \item Как только для какого-либо \textit{sc-элемента} необходимость его блокировки отпадает до завершения выполнения соответствующего действия, этот \textit{sc-элемент} желательно сразу деблокировать (снять блокировку).
        \end{scnenumerate}
        Для того, чтобы \textit{sc-агент} имел возможность работы с каким-либо произвольным \textit{sc-элементом}, он должен либо убедиться в том, что этот \textit{sc-элемент} не входит во фрагмент базы знаний, входящий в \textit{полную блокировку}, либо убедиться в том, что эта блокировка не установлена самим этим агентом.\\
        Особой группой полностью заблокированных \textit{sc-элементов} (на время выполнения действия \textit{\mbox{sc-агентом}}) являются вспомогательные \textit{sc-элементы} (леса), создаваемые только на время выполнения этого действия. Эти sc-элементы в конце выполнения действия должны не деблокироваться, а удаляться.}
    \scnfileitem{Если \textit{действие в sc-памяти}, выполняемое \textit{sc-агентом}, завершилось (т.е. стало прошлой сущностью), то \textit{sc-агент} оформляет результат этого \textit{действия}, указывая (1) удаленные \textit{sc-элементы} и (2) сгенерированные sc-элементы. Это необходимо, если по каким-либо причинам придется сделать откат этого \textit{действия}, т.е возвратиться к состоянию базы знаний до выполнения указанного \textit{действия}.}
\end{scnrelfromlist}

\scnsegmentheader{Понятие действия в sc-памяти}
\begin{scnsubstruct}
    
\scnheader{действие в sc-памяти}
\scnidtf{внутреннее действие ostis-системы}
\scnidtf{действие, выполняемое в sc-памяти}
\scnidtf{действие, выполняемое в абстрактной унифицированной семантической памяти}
\scnidtf{действие, выполняемое машиной обработки знаний ostis-системы}
\scnidtf{действие, выполняемое агентом или коллективом агентов ostis-системы}
\scnidtf{информационный процесс над базой знаний, хранимой в sc-памяти}
\scnidtf{процесс решения информационной задачи в sc-памяти}
\scnsubset{процесс в sc-памяти}
\scntext{пояснение}{Каждое \textbf{\textit{действие в sc-памяти}} обозначает некоторое преобразование, выполняемое некоторым \textit{sc-агентом} (или коллективом \textit{sc-агентов}) и ориентированное на преобразование \textit{sc-памяти}. Спецификация действия после его выполнения может быть включена в протокол решения некоторой задачи.\\
    Преобразование состояния базы знаний включает, в том числе и информационный поиск, предполагающий (1) локализацию в базе знаний ответа на запрос, явное выделение структуры ответа и (2) трансляцию ответа на некоторый внешний язык.\\
    Во множество \textbf{\textit{действий в sc-памяти}} входят знаки действий самого различного рода, семантика каждого из которых зависит от конкретного контекста, т.е. ориентации действия на какие-либо конкретные объекты и принадлежности действия какому-либо конкретному классу действий.\\
    Следует четко отличать:
        \begin{scnitemize}
            \item каждое конкретное \textbf{\textit{действие в sc-памяти}}, представляющее собой некоторый переходный процесс, переводящий sc-память из одного состояния в другое;
            \item каждый тип \textbf{\textit{действий в sc-памяти}}, представляющий собой некоторый класс однотипных (в том или ином смысле) действий;
            \item sc-узел, обозначающий некоторое конкретное \textbf{\textit{действие в sc-памяти}};
            \item sc-узел, обозначающий структуру, которая является описанием, спецификацией, заданием, постановкой соответствующего действия.
        \end{scnitemize}}
\scnsuperset{действие в sc-памяти, инициируемое вопросом}
\scnsuperset{действие редактирования базы знаний ostis-системы}
\scnsuperset{действие установки режима ostis-системы}
\scnsuperset{действие редактирования файла, хранимого в sc-памяти}
\scnsuperset{действие интерпретации программы, хранимой в sc-памяти}

\scnheader{действие в sc-памяти, инициируемое вопросом}
\scnidtf{действие, направленное на формирование ответа на поставленный вопрос}
\scnsuperset{действие. cформировать заданный файл}
\scnsuperset{действие. cформировать заданную структуру}
    \begin{scnindent}
        \scnsuperset{действие. верифицировать заданную структуру}
        \begin{scnindent}
            \scnsuperset{действие. установить истинность или ложность указываемого логического высказывания}
            \scnsuperset{действие. установить корректность или некорректность указываемой структуры}
            \scnsuperset{действие. сформировать структуру, описывающую некорректности, имеющиеся в указываемой структуре}
        \end{scnindent}
        \scnsuperset{действие. уточнить тип заданного sc-элемента}
        \begin{scnindent}
            \scnsuperset{действие. установить позитивность/негативность указываемой sc-дуги принадлежности или непринадлежности}
        \end{scnindent}
        \scnsuperset{действие. сформировать семантическую окрестность}
        \begin{scnindent}
            \scnsuperset{действие. сформировать полную семантическую окрестность указываемой сущности}
            \scnsuperset{действие. сформировать базовую семантическую окрестность указываемой сущности}
            \scnsuperset{действие. сформировать частную семантическую окрестность указываемой сущности}
        \end{scnindent}
        \scnsuperset{действие. сформировать структуру, описывающую связи между указываемыми сущностями}
        \begin{scnindent}
            \scnsuperset{действие. сформировать структуру, описывающую сходства указываемых сущностей}
            \scnsuperset{действие. сформировать структуру, описывающую различия указываемых сущностей}
        \end{scnindent}
        \scnsuperset{действие. сформировать структуру, описывающую способ решения указываемой задачи}
        \scnsuperset{действие. сформировать план генерации ответа на указанный вопрос}
        \scnsuperset{действие. сформировать протокол выполнения указываемого действия}
        \scnsuperset{действие. сформировать обоснование корректности указываемого решения}
        \scnsuperset{действие. верифицировать обоснование корректности указываемого решения}
        \scnsuperset{действие, направленное на установление темпоральных характеристик указываемой сущности}
        \scnsuperset{действие, направленное на установление пространственных характеристик указываемой сущности}
    \end{scnindent}

\scnheader{действие редактирования базы знаний}
\scnsuperset{действие. изменить направление указанной sc-дуги}
\scnsuperset{действие. исправить ошибки в заданной структуре}
\scnsuperset{действие. преобразовать указанную структуру в соответствии с указанным правилом}
\scnsuperset{действие. отождествить два указанных sc-элемента}
\scnsuperset{действие. включить множество}
    \begin{scnindent}
        \scnidtf{сделать все элементы множества \textbf{\textit{Si}} явно принадлежащими множеству \textbf{\textit{Sj}}, то есть сгенерировать соответствующие sc-дуги принадлежности}
    \end{scnindent}
\scnsuperset{действие генерации sc-элементов}
    \begin{scnindent}
        \scnsuperset{действие генерации, одним из аргументов которого является некоторая обобщенная структура}
            \begin{scnindent}
                \scnsuperset{действие. сгенерировать структуру, изоморфную указываемому образцу}
            \end{scnindent}
        \scnsuperset{действие. сгенерировать sc-элемент указанного типа}
            \begin{scnindent}
                \scnsuperset{действие. сгенерировать sc-коннектор указанного типа}
                \scnsuperset{действие. сгенерировать sc-узел указанного типа}
            \end{scnindent}
        \scnsuperset{действие. сгенерировать файл с заданным содержимым}
        \scnsuperset{действие. установить указанный файл в качестве основного идентификатора указанного sc-элемента для указанного внешнего языка}
    \end{scnindent}
\scnsuperset{действие. обновить понятия}
    \begin{scnindent}
        \scnidtf{действие. заменить неосновные понятия на их определения через основные понятия}
    \end{scnindent}
\scnsuperset{действие. интегрировать информационную конструкцию в текущее состояние базы знаний}
    \begin{scnindent}
        \scnsuperset{действие. интегрировать содержимое указанного файла в текущее состояние базы знаний}
            \begin{scnindent}
                \scnsuperset{действие. протранслировать содержимое указанного файла в sc-память}
            \end{scnindent}
        \scnsuperset{действие. интегрировать указанную структуру в текущее состояние базы знаний}
    \end{scnindent}
\scnsuperset{действие. дополнить описание прошлого состояния ostis-системы}
    \begin{scnindent}
        \scnsuperset{действие. дополнить структуру, описывающую историю эволюции ostis-системы}
        \scnsuperset{действие. дополнить структуру, описывающую историю эксплуатации ostis-системы}
    \end{scnindent}
\scnsuperset{действие удаления sc-элементов}
    \begin{scnindent}
        \scnsuperset{действие. удалить указанные sc-элементы}
        \begin{scnindent}
            \scnsuperset{действие. удалить sc-элементы, входящие в состав указанной структуры и не являющиеся ключевыми узлами каких-либо sc-агентов}
        \end{scnindent}
    \end{scnindent}
        
\scnheader{действие. отождествить два указанных sc-элемента}
\scnidtf{действие. совместить два указанных sc-элемента}
\scnidtf{действие. склеить два указанных sc-элемента}
    \begin{scnsubdividing}
        \scnitem{действие. физически отождествить два указанных sc-элемента}
        \scnitem{действие. логически отождествить два указанных sc-элемента}
    \end{scnsubdividing}

\scnheader{действие. отождествить два указанных sc-элемента}
\scntext{пояснение}{Каждое \textbf{\textit{действие. отождествить два указанных sc-элемента}} может быть выполнено как \textit{действие. физически отождествить два указанных sc-элемента} или \textit{действие. логически отождествить два указанных sc-элемента}. В случае логического отождествления в протоколе деятельности агентов сохраняется само действие с его спецификацией, включающей обязательное указание того, какие элементы были сгенерированы, а какие удалены. В случае физического отождествления протокол действия не сохраняется.}

\scnheader{действие. обновить понятия}
\scnidtf{действие. заменить некоторое множество понятий на другое множество понятий}
\scntext{пояснение}{Каждое \textbf{\textit{действие. обновить понятия}} обозначает переход от какой-то группы понятий, использовавшихся ранее, к другой группе понятий, которые будут использоваться вместо первых и станут \textit{основными понятиями}. В общем случае \textbf{\textit{действие. обновить понятия}} состоит из следующих этапов:
    \begin{scnitemize}
        \item определить заменяемые понятия на основе заменяющих;
        \item внести соответствующие изменения в программы sc-агентов, ключевыми узлами которых являются обновляемые понятия;
        \item заменить все конструкции в базе знаний, содержащие заменяемые понятия, в соответствии с определениями этих понятий через заменяющие их понятия;
        \item при необходимости, \textit{sc-элементы}, обозначающие замененные таким образом понятия, могут быть полностью выведены из текущего состояния базы знаний.
    \end{scnitemize}
    Первым аргументом (входящим в знак \textit{действия} под атрибутом \textit{1\scnrolesign}) \textbf{\textit{действия. обновить понятия}} является знак множества \textit{sc-узлов}, обозначающих заменяемые понятия, вторым (входящим в знак \textit{действия} под атрибутом \textit{2\scnrolesign}) --- знак множества \textit{sc-узлов}, обозначающих заменяющие понятия. В общем случае любое или оба этих множества могут быть \textit{синглетонами}.}   
    
\scnheader{действие. удалить указанные sc-элементы}
\begin{scnsubdividing}
    \scnitem{действие. физически удалить указанные sc-элементы}
    \scnitem{действие. логически удалить указанные sc-элементы}
\end{scnsubdividing}
\scntext{пояснение}{Каждое \textbf{\textit{действие. удалить указанные sc-элементы}} может быть выполнено как \textit{действие. физически удалить указанные sc-элементы} или \textit{действие. логически удалить указанные sc-элементы}. \\
    В случае логического удаления в протоколе деятельности агентов сохраняется само действие с его спецификацией, включающей обязательное указание того, какие элементы были удалены, т.е. по сути, элементы просто исключаются из текущего состояния базы знаний.\\
    В случае физического удаления протокол действия не сохраняется. В случае удаления какого-либо \textit{sc-элемента}, инцидентные ему \textit{связки}, в том числе \textit{sc-коннекторы}, так же удаляются.}

\scnheader{действие. интегрировать указанную структуру в текущее состояние базы знаний}
\scntext{пояснение}{Для того, чтобы выполнить \textbf{\textit{действие. интегрировать указанную структуру в текущее состояние базы знаний}}, необходимо склеить \textit{sc-элементы}, входящие в интегрируемую \textit{структуру} с синонимичными им \textit{sc-элементами}, входящими в текущее состояние базы знаний, заменить неиспользуемые (например, устаревшие) понятия, входящие в интегрируемую \textit{структуру}, на используемые (т.е. заменить неиспользуемые понятия на их определения через используемые), явно включить все элементы интегрируемой \textit{структуры} в число элементов утвержденной части базы знаний и явно включить все элементы интегрируемой \textit{структуры} в число элементов одного из атомарных разделов утвержденной части базы знаний.}

\scnheader{действие интерпретации программы, хранимой в sc-памяти}
\scnsuperset{действие интерпретации scp-программы}

\scnheader{задача, решаемая в sc-памяти}
\scnsubset{задача}
\scnidtf{спецификация действия, выполняемого в sc-памяти}
\scnidtf{структура, являющая описанием (постановкой, заданием) соответствующего действия в sc-памяти, которое обладает достаточной полнотой для выполения указанного действия}
\scnidtf{семантическая окрестность некоторого действия в sc-памяти, обеспечивающая достаточно полное задание этого действия}

\scnheader{класс действий}
\scnsuperset{класс действий в sc-памяти}
\begin{scnindent}
    \scnrelto{семейство подмножеств}{действие в sc-памяти}
\end{scnindent}
\begin{scnsubdividing}
    \scnitem{класс логически атомарных действий}
    \begin{scnindent}
        \scnidtf{класс автономных действий}
    \end{scnindent}
    \scnitem{класс логически неатомарных действий}
    \begin{scnindent}
        \scnidtf{класс неавтономных действий}
    \end{scnindent}
\end{scnsubdividing}

\scnheader{класс логически атомарных действий}
\scntext{пояснение}{Каждое \textit{действие}, принадлежащее некоторому конкретному \textit{классу логически атомарных действий}, обладает двумя необходимыми свойствами:
    \begin{scnitemize}
        \item Выполнение действия не зависит от того, является ли указанное действие частью декомпозиции более общего действия. При выполнении данного действия также не должен учитываться тот факт, что данное действие предшествует каким-либо другим действиям или следует за ними (что явно указывается при помощи отношения \textit{последовательность действий*}).
        \item Указанное действие должно представлять собой логически целостный акт преобразования, например, в семантической памяти. Такое действие по сути является транзакцией, т. е. результатом такого преобразования становится новое состояние преобразуемой системы, а выполняемое действие должно быть либо выполнено полностью, либо не выполнено совсем, частичное выполнение не допускается.
    \end{scnitemize}
    В то же время логическая атомарность не запрещает декомпозировать выполняемое действие на более частные, каждое из которых, в свою очередь, также будет являться логически атомарным.}
\scnsuperset{класс логически атомарных действий в sc-памяти}
    \begin{scnindent}
        \scntext{пояснение}{На логически атомарные действия предлагается делить всю деятельность, направленную на решение каких-либо задач ostis-системой. Соответственно \textit{решатель задач ostis-системы} предлагается делить на компоненты, соответствующие таким \textit{классам логически атомарных действий в sc-памяти}, что является основой для обеспечения его \textit{модифицируемости}.}
    \end{scnindent}

\bigskip
\end{scnsubstruct}
\scnendsegmentcomment{Понятие действия в sc-памяти}

\scnsegmentheader{Понятие sc-агента и абстрактного sc-агента}
\begin{scnsubstruct}

\scnheader{sc-агент}
\scnidtf{единственный вид \textit{субъектов}, выполняющих преобразования в \textit{\textit{sc-памяти}}}
\scnidtf{\textit{субъект}, способный выполнять \textit{действия в sc-памяти}, принадлежащие некоторому определенному \textit{классу логически атомарных действий}}
\scntext{пояснение}{Логическая атомарность выполняемых sc-агентом действий предполагает, что каждый sc-агент реагирует на соответствующий ему класс ситуаций и/или событий, происходящих в sc-памяти, и осуществляет определенное преобразование sc-текста, находящегося в семантической окрестности обрабатываемой ситуации и/или события. При этом каждый sc-агент в общем случае не имеет информацию о том, какие еще sc-агенты в данный момент присутствуют в системе и осуществляет взаимодействие в другими sc-агентами исключительно посредством формирования некоторых конструкций (как правило  спецификаций действий) в общей sc-памяти. Таким сообщением может быть, например, вопрос, адресованный другим sc-агентам в системе (заранее не известно, каким конкретно), или ответ на поставленный другими sc-агентами вопрос (заранее не известно, каким конкретно). Таким образом, каждый sc-агент в каждый момент времени контролирует только фрагмент базы знаний в контексте решаемой данным агентом задачи, состояние всей остальной базы знаний в общем случае непредсказуемо для sc-агента.}

\scnheader{абстрактный sc-агент}
\scntext{примечание}{Поскольку предполагается, что копии одного и того же \textit{sc-агента} или функционально эквивалентные \textit{sc-агенты} могут работать в разных ostis-системах, будучи при этом физически разными sc-агентами, то целесообразно рассматривать свойства и классификацию не sc-агентов, а классов функционально эквивалентных sc-агентов, которые будем называть \textit{абстрактными sc-агентами}.}
\scntext{пояснение}{Под \textbf{\textit{абстрактным sc-агентом}} понимается некоторый класс функционально эквивалентных \textit{sc-агентов}, разные экземпляры (т.е. представители) которого могут быть реализованы по-разному. Каждый \textbf{\textit{абстрактный sc-агент}} имеет соответствующую ему спецификацию. В спецификацию каждого \textbf{\textit{абстрактного sc-агента}} входит:
    \begin{scnitemize}
        \item указание ключевых \textit{sc-элементов} этого \textit{sc-агента}, т.е. тех \textit{sc-элементов}, хранимых в \textit{sc-памяти}, которые для данного \textit{sc-агента} являются точками опоры;
        \item формальное описание условий инициирования данного \textit{sc-агента}, т.е. тех \textit{ситуаций} в \textit{sc-памяти}, которые инициируют деятельность данного \textit{sc-агента};
        \item формальное описание первичного условия инициирования данного \textit{sc-агента}, т.е. такой ситуации в \textit{sc-памяти}, которая побуждает \textit{sc-агента} перейти в активное состояние и начать проверку наличия своего полного условия инициирования (для \textit{внутренних абстрактных sc-агентов});
        \item строгое, полное, однозначно понимаемое описание деятельности данного \textit{sc-агента}, оформленное при помощи каких-либо понятных, общепринятых средств, не требующих специального изучения, например, на естественном языке;
        \item описание результатов выполнения данного \textit{sc-агента}.
    \end{scnitemize}
}
\begin{scnsubdividing}
    \scnitem{неатомарный абстрактный sc-агент}
    \scnitem{атомарный абстрактный sc-агент}
\end{scnsubdividing}
\begin{scnsubdividing}
    \scnitem{внутренний абстрактный sc-агент}
    \scnitem{эффекторный абстрактный sc-агент}
    \scnitem{рецепторный абстрактный sc-агент}
\end{scnsubdividing}
\begin{scnsubdividing}
    \scnitem{абстрактный sc-агент, не реализуемый на Языке SCP}
    \scnitem{абстрактный sc-агент, реализуемый на Языке SCP}
\end{scnsubdividing}
\begin{scnsubdividing}
    \scnitem{абстрактный sc-агент интерпретации scp-программ}
    \scnitem{абстрактный программный sc-агент}
    \scnitem{абстрактный sc-метаагент}
\end{scnsubdividing}
\begin{scnsubdividing}
    \scnitem{платформенно-зависимый абстрактный sc-агент}
    \begin{scnindent}
        \scnsuperset{абстрактный sc-агент, не реализуемый на Языке SCP}
    \end{scnindent}
    \scnitem{платформенно-независимый абстрактный sc-агент}
\end{scnsubdividing}

\scnheader{абстрактный sc-агент, не реализуемый на Языке SCP}
\scnidtf{абстрактный sc-агент, который не может быть реализован на платформенно-независимом уровне}
\begin{scnsubdividing}
    \scnitem{эффекторный абстрактный sc-агент}
    \scnitem{рецепторный абстрактный sc-агент}
    \scnitem{абстрактный sc-агент интерпретации scp-программ}
\end{scnsubdividing}

\scnheader{абстрактный sc-агент, реализуемый на Языке SCP}
\scnidtf{абстрактный sc-агент, который может быть реализован на платформенно-независимом уровне}
\begin{scnsubdividing}
    \scnitem{абстрактный sc-метаагент}
    \scnitem{абстрактный программный sc-агент, реализуемый на Языке SCP}
\end{scnsubdividing}
\scnheader{абстрактный программный sc-агент}
\begin{scnsubdividing}
    \scnitem{эффекторный абстрактный sc-агент}
    \scnitem{рецепторный абстрактный sc-агент}
    \scnitem{абстрактный программный sc-агент, реализуемый на Языке SCP}
\end{scnsubdividing}

\scnheader{неатомарный абстрактный sc-агент}
\scntext{пояснение}{Под \textbf{\textit{неатомарным абстрактным sc-агентом}} понимается \textit{абстрактный sc-агент}, который декомпозируется на коллектив более простых \textit{абстрактных sc-агентов}, каждый из которых в свою очередь может быть как \textit{атомарным абстрактным sc-агентом}, так и \textbf{\textit{неатомарным абстрактным sc-агентом}}. При этом в каком-либо варианте \textit{декомпозиции абстрактного sc-агента*} дочерний \textbf{\textit{неатомарный абстрактный sc-агент}} может стать \textit{атомарным абстрактным sc-агентом} и реализовываться соответствующим образом.}

\scnheader{атомарный абстрактный sc-агент}
\scntext{пояснение}{Под \textbf{\textit{атомарным абстрактным sc-агентом}} понимается \textit{абстрактный sc-агент}, для которого уточняется платформа его реализации, т.е. существует соответствующая связка отношения \textit{программа sc-агента*}.}
\begin{scnsubdividing}
    \scnitem{платформенно-независимый абстрактный sc-агент}
    \scnitem{платформенно-зависимый абстрактный sc-агент}
\end{scnsubdividing}

\scnheader{платформенно-независимый абстрактный sc-агент}
\scntext{пояснение}{К \textbf{\textit{платформенно-независимым абстрактным \mbox{sc-агентам}}} относят \textit{атомарные абстрактные sc-агенты}, реализованные на базовом языке программирования Технологии OSTIS, т.е. на \textit{Языке SCP}.\\
    При описании \textbf{\textit{платформенно-независимых абстрактных sc-агентов}} под платформенной независимостью понимается платформенная независимость с точки зрения Технологии OSTIS, т.е реализация на специализированном языке программирования, ориентированном на обработку семантических сетей (\textit{Языке SCP}), поскольку \textit{атомарные sc-агенты}, реализованные на указанном языке могут свободно переноситься с одной платформы интерпретации \textit{sc-моделей} на другую. При этом языки программирования, традиционно считающиеся платформенно-независимыми, в данном случае не могут считаться таковыми.\\
    Существуют \textit{sc-агенты}, которые принципиально не могут быть реализованы на платформенно-независимом уровне, например, собственно \textit{sc-агенты} интерпретации \textit{sc-моделей} или рецепторные и эффекторные \textit{sc-агенты}, обеспечивающие взаимодействие с внешней средой.}
    
\scnheader{платформенно-зависимый абстрактный sc-агент}
\scntext{пояснение}{К \textbf{\textit{платформенно-зависимым абстрактным sc-агентам}} относят \textit{атомарные абстрактные sc-агенты}, реализованные ниже уровня sc-моделей, т.е. не на \textit{Языке SCP}, а на каком-либо другом языке описания программ.\\
    Существуют \textit{sc-агенты}, которые принципиально должны быть реализованы на платформенно-зависимом уровне, например, собственно \textit{sc-агенты} интерпретации \textit{sc-моделей} или рецепторные и эффекторные \textit{sc-агенты}, обеспечивающие взаимодействие с внешней средой.}

\scnheader{внутренний абстрактный sc-агент}
\scntext{пояснение}{Каждый \textbf{\textit{внутренний абстрактный sc-агент}} обозначает класс \textit{sc-агентов}, которые реагируют на события в \textit{sc-памяти} и осуществляют преобразования исключительно в рамках этой же \textit{sc-памяти}.}

\scnheader{эффекторный абстрактный sc-агент}
\scntext{пояснение}{Каждый \textbf{\textit{эффекторный абстрактный sc-агент}} обозначает класс \textit{sc-агентов}, которые реагируют на события в \textit{sc-памяти} и осуществляют преобразования во внешней относительно данной \textit{ostis-системы} среде.}

\scnheader{рецепторный абстрактный sc-агент}
\scntext{пояснение}{Каждый \textbf{\textit{рецепторный абстрактный sc-агент}} обозначает класс \textit{sc-агентов}, которые реагируют на события во внешней относительно данной \textit{ostis-системы} среде и осуществляют преобразования в памяти данной системы.}

\scnheader{абстрактный sc-агент, не реализуемый на Языке SCP}
\scntext{пояснение}{Каждый \textbf{\textit{абстрактный sc-агент, не реализуемый на Языке SCP}} должен быть реализован на уровне платформы интерпретации sc-моделей, в том числе, аппаратной. К таким \textit{абстрактным sc-агентам} относятся абстрактные sc-агенты интерпретации scp-программ, а также эффекторные и рецепторные абстрактные sc-агенты.}

\scnheader{абстрактный sc-агент, реализуемый на Языке SCP}
\scntext{пояснение}{Каждый \textbf{\textit{абстрактный sc-агент, реализуемый на Языке SCP}} может быть реализован на Языке SCP, то есть на платформенно-независимом уровне, но при необходимости может реализовываться и на уровне платформы, например, с целью повышения производительности.}

\scnheader{абстрактный sc-агент интерпретации scp-программ}
\scntext{пояснение}{К \textbf{\textit{абстрактным sc-агентам интерпретации scp-программ}} относятся нереализуемые на платформенно-независимом уровне \textit{абстрактные sc-агенты}, обеспечивающие интерпретацию \textit{scp-программ} и \textit{scp-метапрограмм}, в том числе создание \textit{scp-процессов}, собственно интерпретацию \textit{scp-операторов}, а также другие вспомогательные действия. По сути, агенты данного класса обеспечивают работу sc-агентов более высоких уровней (программных sc-агентов и sc-метаагентов), реализованных на Языке SCP, в частности, обеспечивают соблюдение указанными агентами общих принципов синхронизации.}

\scnheader{абстрактный программный sc-агент}
\scntext{пояснение}{К \textbf{\textit{абстрактным программным sc-агентам}} относятся все \textit{абстрактные sc-агенты}, обеспечивающие основной функционал системы, то есть ее возможность решать те или иные задачи. Агенты данного класса должны работать в соответствии с общими принципами синхронизации деятельности субъектов в sc-памяти.}

\scnheader{абстрактный sc-метаагент}
\scntext{пояснение}{Задачей \textbf{\textit{абстрактных sc-метаагентов}} является координация деятельности \textit{абстрактных программных sc-агентов}, в частности, решение проблемы взаимоблокировок. Агенты данного класса могут быть реализованы на Языке SCP, однако для синхронизации их деятельности используются другие принципы, соответственно, для реализации таких агентов требуется Язык SCP другого уровня, типология операторов которого полностью аналогична типологии scp-операторов, однако эти операторы имеют другую операционную семантику, учитывающую отличия в принципах синхронизации (работы с \textit{блокировками*}). Программы такого языка будем называть \textit{scp-метапрограммами}, соответствующие им \mbox{\textit{процессы в sc-памяти} ---  \textit{scp-метапроцессами}}, операторы  --- \textit{scp-метаоператорами}.}

\scnheader{декомпозиция абстрактного sc-агента*}
\scniselement{отношение декомпозиции}
\scntext{пояснение}{Отношение \textbf{\textit{декомпозиции абстрактного sc-агента*}} трактует \textit{неатомарные абстрактные sc-агенты} как коллективы более простых \textit{абстрактных sc-агентов}, взаимодействующих через \textit{sc-память}.\\
    Другими словами, \textbf{\textit{декомпозиция абстрактного sc-агента*}} на \textit{абстрактные sc-агенты} более низкого уровня уточняет один из возможных подходов к реализации этого \textit{абстрактного sc-агента} путем построения коллектива более простых \textit{абстрактных sc-агентов}.}

\scnheader{sc-агент}
\scnidtf{агент над sc-памятью}
\scnsubset{субъект}
\scnrelfrom{семейство подмножеств}{абстрактный sc-агент}
\scntext{пояснение}{Под \textbf{\textit{sc-агентом}} понимается конкретный экземпляр (с теоретико-множественной точки зрения --- элемент) некоторого \textit{атомарного абстрактного sc-агента}, работающий в какой-либо конкретной интеллектуальной системе.\\
    Таким образом, каждый \textit{sc-агент} --- это субъект, способный выполнять некоторый класс однотипных действий либо только над \textit{sc-памятью}, либо над sc-памятью и внешней средой (для эффекторных \textit{sc-агентов}). Каждое такое действие инициируется либо состоянием или ситуацией в sc-памяти, либо состоянием или ситуацией во внешней среде (для рецепторных sc-агентов-датчиков),  соответствующей условию инициирования \textit{атомарного абстрактного sc-агента}, экземпляром которого является заданный \textit{sc-агент}. В данном случае можно провести аналогию между принципами объектно-ориентированного программирования, рассматривая \textit{атомарный абстрактный sc-агент} как класс, а конкретный \textit{sc-агент}  как экземпляр, конкретную имплементацию этого класса.\\
    Взаимодействие \textit{sc-агентов} осуществляется только через \textit{sc-память}. Как следствие, результатом работы любого \textit{sc-агента} является некоторое изменение состояния \textit{sc-памяти}, т.е. удаление либо генерация каких-либо \textit{sc-элементов}.\\
    В общем случае один \textit{sc-агент} может явно передать управление другому \textit{sc-агенту}, если этот \textit{sc-агент} априори известен. Для этого каждый \textit{sc-агент} в \textit{sc-памяти} имеет обозначающий его \textit{sc-узел}, с которым можно связать конкретную ситуацию в текущем состоянии базы знаний, которую инициируемый \textit{sc-агент} должен обработать.\\
    Однако далеко не всегда легко определить тот \textit{sc-агент}, который должен принять управление от заданного \textit{sc-агента}, в связи с чем описанная выше ситуация возникает крайне редко. Более того, иногда условие инициирования \textit{sc-агента} является результатом деятельности непредсказуемой группы \textit{sc-агентов}, равно как и одна и та же конструкция может являться условием инициирования целой группы \textit{sc-агентов}.\\
    При этом общаются через \textit{sc-память} не \textit{программы sc-агентов*}, а сами описываемые данными программами \textit{sc-агенты}.\\
    В процессе работы \textit{sc-агент} может сам для себя порождать вспомогательные \textit{sc-элементы}, которые сам же удаляет после завершения акта своей деятельности (это вспомогательные \textit{структуры}, которые используются в качестве информационных лесов только в ходе выполнения соответствующего акта деятельности и после завершения этого акта удаляются).}

\scnheader{активный sc-агент}
\scnsubset{sc-агент}
\scntext{пояснение}{Под \textbf{\textit{активным sc-агентом}} понимается \textit{sc-агент} ostis-системы, который реагирует на события, соответствующие его условию инициирования, и, как следствие, его \textit{первичному условию инициирования*}. Не входящие во множество \textbf{\textit{активных sc-агентов}} \textit{sc-агенты} не реагируют ни на какие события в \textit{sc-памяти}.}

\scnheader{ключевые sc-элементы sc-агента*}
\scntext{пояснение}{Связки отношения \textbf{\textit{ключевые sc-элементы sc-агента*}} связывают между собой \textit{sc-узел}, обозначающий \textit{абстрактный sc-агент} и \textit{sc-узел}, обозначающий множество \textit{sc-элементов}, которые являются ключевыми для данного \textit{абстрактного sc-агента}, то данные \textit{sc-элементы} явно упоминаются в рамках программ, реализующих данный \textit{абстрактный sc-агент}.}

\scnheader{программа sc-агента*}
\scntext{пояснение}{Связки отношения \textbf{\textit{программа sc-агента*}} связывают между собой \textit{sc-узел}, обозначающий \textit{атомарный абстрактный sc-агент} и \textit{sc-узел}, обозначающий множество программ, реализующих указанный \textit{атомарный абстрактный sc-агент}. В случае \textit{платформенно-независимого абстрактного sc-агента} каждая связка отношения \textit{программа sc-агента*} связывает \textit{sc-узел}, обозначающий указанный \textit{абстрактный sc-агент} с множеством \textit{scp-программ}, описывающих деятельность данного \textit{абстрактного sc-агента}. Данное множество содержит одну \textit{агентную scp-программу} и произвольное количество (может быть, и ни одной) \textit{scp-программ}, которые необходимы для выполнения указанной \textit{агентной scp-программы}.\\
    В случае \textit{платформенно-зависимого абстрактного sc-агента} каждая связка отношения \textit{программа \mbox{sc-агента*}} связывает \textit{sc-узел}, обозначающий указанный \textit{абстрактный sc-агент} с множеством файлов, содержащих исходные тексты программы на некотором внешнем языке программирования, реализующей деятельность данного \textit{абстрактного sc-агента}.}

\scnheader{первичное условие инициирования*}
\scntext{пояснение}{Связки отношения \textbf{\textit{первичное условие инициирования*}} связывают между собой \textit{sc-узел}, обозначающий \textit{абстрактный sc-агент} и бинарную ориентированную пару, описывающую первичное условие инициирования данного \textit{абстрактного sc-агента}, т.е. такую спецификацию \textit{ситуации} в \textit{sc-памяти}, возникновение которой побуждает \textit{sc-агента} перейти в активное состояние и начать проверку наличия своего полного условия инициирования.\\
    Первым компонентом данной ориентированной пары является знак некоторого класса \textit{элементарных событий в sc-памяти*}, например, \textit{событие добавления sc-дуги, выходящей из заданного sc-элемента*}.\\
    Вторым компонентом данной ориентированной пары является произвольный в общем случае \textit{sc-элемент}, с которым непосредственно связан указанный тип события в \textit{sc-памяти}, т.е., например, \textit{sc-элемент}, из которого выходит либо в который входит генерируемая либо удаляемая \textit{sc-дуга} либо \textit{файл}, содержимое которого было изменено.\\
    После того, как в \textit{sc-памяти} происходит некоторое событие, активизируются все \textit{активные sc-агенты}, \textbf{\textit{первичное условие инициирования*}} которых соответствует произошедшему событию.}

\scnheader{условие инициирования и результат*}
\scntext{пояснение}{Связки отношения \textbf{\textit{условие инициирования и результат*}} связывают между собой \textit{sc-узел}, обозначающий \textit{абстрактный sc-агент}, и бинарную ориентированную пару, связывающую условие инициирования данного \textit{абстрактного sc-агента} и результаты выполнения данного экземпляров данного \textit{sc-агента} в какой-либо конкретной системе.\\
    Указанную ориентированную пару можно рассматривать как логическую связку импликации, при этом на \textit{sc-переменные}, присутствующие в обеих частях связки, неявно накладывается квантор всеобщности, на \textit{sc-переменные}, присутствующие либо только в посылке, либо только в заключении неявно накладывается квантор существования.\\
    Первым компонентом указанной ориентированной пары является логическая формула, описывающая условие инициирования описываемого \textit{абстрактного sc-агента}, то есть конструкции, наличие которой в \textit{sc-памяти} побуждает \textit{sc-агент} начать работу по изменению состояния \textit{sc-памяти}. Данная логическая формула может быть как атомарной, так и неатомарной, в которой допускается использование любых связок логического языка.\\
    Вторым компонентом указанной ориентированной пары является логическая формула, описывающая возможные результаты выполнения описываемого абстрактного \textit{sc-агента}, то есть описание произведенных им изменений состояния \textit{sc-памяти}. Данная логическая формула может быть как атомарной, так и неатомарной, в которой допускается использование любых связок логического языка.}

\scnheader{описание поведения sc-агента}
\scnsubset{семантическая окрестность}
\scntext{пояснение}{\textbf{\textit{описание поведения sc-агента}} представляет собой \textit{семантическую окрестность}, описывающую деятельность \textit{sc-агента} до какой-либо степени детализации, однако такое описание должно быть строгим, полным и однозначно понимаемым. Как любая другая \textit{семантическая окрестность}, \textbf{\textit{описание поведения sc-агента}} может быть протранслировано на какие-либо понятные, общепринятые средства, не требующие специального изучения, например, на естественный язык.\\
    Описываемый \textit{абстрактный sc-агент} входит в соответствующее \textbf{\textit{описание поведения sc-агента}} под атрибутом \textit{ключевой sc-элемент\scnrolesign}.}

\bigskip
\end{scnsubstruct}
\scnendsegmentcomment{Понятие sc-агента и абстрактного sc-агента}

\scnsegmentheader{Принципы синхронизации деятельности sc-агентов}
\begin{scnsubstruct}
    
\scnheader{процесс в sc-памяти}
\scntext{примечание}{Понятия \textit{действие в sc-памяти} и \textit{процесс в sc-памяти} (информационный процесс, выполняемый агентом в семантической памяти), являются синонимичными, поскольку все процессы, протекающие в sc-памяти, являюся осознанными и выполняются каким-либо sc-агентами. Тем не менее, когда идет речь о синхронизации выполнения каких-либо преобразований в памяти компьютерной системы, в литературе принято использовать именно термины \textit{процесс}, взаимодействие процессов \cite{Dijkstra1972, Hoare1989}, в связи с чем будем использовать этот термин при описании принципов синхронизации деятельности sc-агентов при выполнении ими параллельных процессов в sc-памяти.}
\begin{scnsubdividing}
    \scnitem{процесс в sc-памяти, соответствующий платформенно-зависимому sc-агенту}
    \scnitem{scp-процесс}
\end{scnsubdividing}
\begin{scnsubdividing}
    \scnitem{scp-процесс, не являющийся scp-метапроцессом}
    \scnitem{scp-метапроцесс}
\end{scnsubdividing}

\scnheader{процесс в sc-памяти, соответствующий платформенно-зависимому sc-агенту}
\begin{scnsubdividing}
    \scnitem{процесс в sc-памяти, соответствующий платформенно-зависимому sc-агенту и не являющийся действием абстрактной scp-машины}
    \scnitem{действие абстрактной scp-машины
    \begin{scnindent}
        \scnsuperset{действие интерпретации scp-программы}
    \end{scnindent}}
\end{scnsubdividing}

\scnheader{блокировка*}
\scniselement{бинарное отношение}
\scntext{пояснение}{Для синхронизации выполнения \textit{процессов в sc-памяти} используется механизм блокировок. Отношение \textbf{\textit{блокировка*}} связывает знаки \textit{действий в sc-памяти} со знаками \textit{структур} (ситуативных), которые содержат элементы, заблокированные на время выполнения данного действия или на какую-то часть этого периода. Каждая такая \textit{структура} принадлежит какому-либо из \textit{типов блокировки}.\\
    Первым компонентом связок отношения \textbf{\textit{блокировка*}} является знак \textit{действия в sc-памяти}, вторым  знак заблокированной \textit{структуры}.}
\scnrelfrom{описание примера}{\scnfileimage[20em]{Contents/part_ps/src/images/sd_agents/lock.png}}

\scnheader{тип блокировки}
\scntext{пояснение}{Множество \textbf{\textit{тип блокировки}} содержит все возможные классы блокировок, т.е. \textit{структуры}, содержащие \textit{sc-элементы}, заблокированные каким-либо \textit{sc-агентом} на время выполнения им некоторого \textit{действия в sc-памяти}.}
\scnhaselement{полная блокировка}
\scnhaselement{блокировка на любое изменение}
\scnhaselement{блокировка на удаление}

\scnheader{полная блокировка}
\scntext{пояснение}{Каждая \textit{структура}, принадлежащая множеству \textbf{\textit{полная блокировка}} содержит \textit{sc-элементы}, просмотр и изменение (удаление, добавление инцидентных \textit{sc-коннекторов}, удаление самих \textit{sc-элементов}, изменение содержимого в  случае файла) которых запрещены всем \textit{sc-агентам}, кроме собственно \textit{sc-агента}, выполняющего соответствующее данной структуре \textit{действие в sc-памяти}, связанное с ней отношением \textit{блокировка*}.\\
    Для того, чтобы исключить возможность реализации \textit{sc-агентов}, которые могут внести изменения в конструкции, описывающие блокировки других \textit{sc-агентов}, все элементы этих конструкций, в том числе, сам знак \textit{структуры}, содержащей заблокированные \textit{sc-элементы} (принадлежащей как множеству \textbf{\textit{полная блокировка}}, так и любому другому \textit{типу блокировки}) и связки отношения \textit{блокировка*}, связывающие эту \textit{структуру} и конкретное \textit{действие в sc-памяти}, добавляются в \textbf{\textit{полную блокировку}}, соответствующую данному \textit{действию в sc-памяти}. Таким образом, каждой \textbf{\textit{полной блокировке}} соответствует петля принадлежности, связывающая ее знак с самим собой.}

\scnheader{блокировка на любое изменение}
\scntext{пояснение}{Каждая \textit{структура}, принадлежащая множеству \textbf{\textit{блокировка на любое изменение}} содержит \textit{sc-элементы}, изменение (физическое удаление, добавление инцидентных \textit{sc-коннекторов}, физическое удаление самих \textit{\mbox{sc-элементов}}, изменение содержимого в случае файл) которых запрещено всем \textit{sc-агентам}, кроме собственно \textit{sc-агента}, выполняющего соответствующее данной структуре \textit{действие в sc-памяти}, связанное с ней отношением \textit{блокировка*}. Однако не запрещен просмотр (чтение) этих \textit{sc-элементов} любым \textit{sc-агентом}.}

\scnheader{блокировка на удаление}
\scntext{пояснение}{Каждая \textit{структура}, принадлежащая множеству \textbf{\textit{блокировка на удаление}} содержит \textit{sc-элементы}, удаление которых запрещено всем \textit{sc-агентам}, кроме собственно \textit{sc-агента}, выполняющего соответствующее данной структуре \textit{действие в sc-памяти}, связанное с ней отношением \textit{блокировка*}. Однако не запрещен просмотр (чтение) этих \textit{sc-элементов} любым \textit{sc-агентом}, добавление инцидентных sc-коннекторов.}

\scnheader{блокировка*}
\begin{scnrelfromset}{принципы работы}
    \scnfileitem{В каждый момент времени одному процессу в sc-памяти может соответствовать только одна блокировка каждого типа.}
    \scnfileitem{В каждый момент времени одному процессу в sc-памяти может соответствовать только одна блокировка, установленная на некоторый конкретный sc-элемент.}
    \scnfileitem{При завершении выполнения любого процесса в sc-памяти все установленные им блокировки автоматически снимаются.}
    \scnfileitem{Для повышения эффективности работы системы в целом каждый процесс должен в каждый момент времени блокировать минимально необходимое множество sc-элементов, снимая блокировку с каждого sc-элемента сразу же, как это становится возможным (безопасным).}
    \scnfileitem{В случае когда в рамках \textit{процесса в sc-памяти} явно выделяются более частные подпроцессы (при помощи отношений \textit{темпоральная часть*, поддействие*, декомпозиция действия*} и т. д.), то каждый такой подпроцесс с точки зрения синхронизации выполнения рассматривается как самостоятельный процесс, которому в соответствие могт быть поставлены все необходимые блокировки.}
    \begin{scnindent}
        \begin{scnrelfromlist}{детализация}
            \scnfileitem{Все дочерние процессы в sc-памяти имеют доступ к блокировкам родительского процесса так же, как если бы это были блокировки соответствующие каждому из таких дочерних процессов.}
            \scnfileitem{В свою очередь, родительский процесс не имеет какого-либо привилегированного доступа к sc-элементам, заблокированным дочерними процессами, и работает с ними так же, как любой другой процесс в sc-памяти. Исключение составляют sc-элементы, обозначающие сами дочерние процессы, поскольку родительский процесс должен иметь возможность управления дочерним, например, приостановки или прекращения их выполнения.}
            \scnfileitem{Все дочерние процессы по отношению друг к другу работают так же, как и по отношению к любым другим процессам.}
            \scnfileitem{В случае, когда родительский процесс приостанавливает выполнение (становится \textit{отложенным действием}), \uline{все} его дочерние процессы также приостанавливают выполнение. В свою очередь, приостановка одного из дочерних процессов в общем случае не инициирует явно остановку всего родительского процесса и соответственно других дочерних.}
        \end{scnrelfromlist}
    \end{scnindent}
\end{scnrelfromset}

\scnheader{полная блокировка}
\begin{scnrelfromset}{принципы работы}
    \scnfileitem{Если sc-элемент, инцидентный некоторому sc-коннектору, попадает в какую-либо полную блокировку, то сам этот sc-коннектор по умолчанию также считается заблокированным этой же блокировкой. Обратное в общем случае неверно, т. к. часть sc-коннекторов, инцидентных некоторому sc-элементу, может быть полностью заблокирована, при этом сам этот элемент заблокирован не будет. Такая ситуация типична, например, для sc-узлов, обозначающих классы понятий.}
    \scnfileitem{Каждый процесс в sc-памяти может свободно изменять или удалять любые sc-элементы, попадающие в полную блокировку, соответствующую этому процессу.}
\end{scnrelfromset}
    \begin{scnindent}
        \scntext{примечание}{Принципы работы с \textit{полными блокировками}, с одной стороны, наиболее просты, поскольку все процессы, кроме установившего такую блокировку, не имеют доступа к заблокированным \mbox{sc-элементам} и конфликты возникнуть не могут. С другой стороны, частое использование блокировок такого типа может привести к тому, что система не сможет использовать в полной мере имеющиеся у нее знания и давать неполные или даже некорректные ответы на поставленные вопросы.}
    \end{scnindent}

\scnheader{блокировка на любое изменение}
\begin{scnrelfromset}{принципы работы}
    \scnfileitem{На один и тот же sc-элемент в один момент времени может быть установлена только одна блокировка одного типа, но разные процессы могут одновременно установить на один и тот же элемент блокировки двух разных типов. Это касается случая, когда первый процесс установил на некоторый sc-элемент блокировку на удаление, а второй процесс затем устанавливает блокировку на любое изменение. В других случаях возникает конфликт блокировок.}
    \scnfileitem{Установка блокировки любого типа также считается изменением, таким образом, если на некоторый \mbox{sc-элемент} была установлена блокировка на любое изменение, то другой процесс не сможет установить на этот же sc-элемент блокировку любого типа, пока первый процесс не снимет свою.}
    \scnfileitem{Если блокировка на удаление устанавливается на некоторый sc-коннектор, то по умолчанию та же блокировка устанавливается на инцидентные этому sc-коннектору sc-элементы, поскольку удаление этих элементов приведет к удалению этого коннектора.}
\end{scnrelfromset}
    \begin{scnindent}
        \scnrelto{принципы работы}{блокировка на удаление}
    \end{scnindent}

\scnheader{процесс в sc-памяти}
\scnidtf{действие в sc-памяти}
\scnrelfrom{разбиение}{Классификация процессов в sc-памяти с точки зрения синхронизации их выполнения}
\begin{scnindent}
    \begin{scneqtoset}
        \scnitem{действие поиска sc-элементов}
        \scnitem{действие генерации sc-элементов}
        \scnitem{действие удаления sc-элементов}
        \scnitem{действие установки блокировки некоторого типа на некоторый sc-элемент}
        \scnitem{действие снятия блокировки с некоторого sc-элемента}
    \end{scneqtoset}
\end{scnindent}

\scnheader{транзакция в sc-памяти}
\scntext{пояснение}{В некоторых случаях для того, чтобы обеспечить синхронизацию, необходимо объединять несколько элементарных действий над sc-памятью в одно неделимое действие (\textit{транзакцию в sc-памяти}), для которого гарантируется, что ни один сторонний процесс не сможет прочитать или изменить участвующие в этом действии sc-элементы, пока действие не завершится. При этом, в отличие от ситуации с полной блокировкой, процесс, пытающийся получить доступ к таким элементам, не продолжает выполнение так, как если бы этих элементов просто не было в sc-памяти, а ожидает завершения транзакции, после чего может выполнять с данными элементами любые действия согласно общим принципам синхронизации процессов. Проблема обеспечения транзакций не может быть решена на уровне SC-кода и требует реализации таких неделимых действий на уровне \textit{платформы интерпретации sc-моделей}.}

\scnheader{действие поиска sc-элементов}
\scntext{пояснение}{В случае осуществления поиска все найденные и сохраненные в рамках какого-либо процесса sc-элементы попадают в соответствующую данному процессу \textit{блокировку на любое изменение}. Таким образом, гарантируется целостность фрагмента базы знаний, с которым работает некоторый процесс в sc-памяти. При этом поиск и автоматическая установка такой блокировки должны быть реализованы как \textit{транзакция в sc-памяти}.\\
    Такой подход также позволяет избежать ситуации, когда один процесс заблокировал некоторый sc-элемент на любое изменение, а второй процесс пытается сгенерировать или удалить \textit{sc-коннектор}, инцидентный данному \textit{sc-элементу}. В таком случае второй процесс должен будет предварительно найти и заблокировать указанный \textit{sc-элемент} на любое изменение, что вызовет конфликт блокировок (\textit{взаимоблокировку*}).}

\scnheader{действие генерации sc-элементов}
\scntext{пояснение}{В случае генерации любого sc-элемента в рамках некоторого процесса он автоматически попадает в полную блокировку, соответствующую данному процессу. При этом генерация и автоматическая установка такой блокировки должны быть реализованы как \textit{транзакция в sc-памяти}. При необходимости сгенерированные элементы могут быть удалены (т. е. их временное существование вообще никак не отразится на деятельности других процессов) или разблокированы в случае, когда сгенерирована информация, которая может иметь некоторую ценность в дальнейшем.}

\scnheader{действие установки блокировки некоторого типа на некоторый sc-элемент}
\scntext{пояснение}{В случае, если какой-либо процесс пытается установить блокировку любого типа на какой-либо sc-элемент, уже заблокированный каким-либо другим процессом, то, с одной стороны, блокировка не может быть установлена, пока другой процесс не разблокирует указанный sc-элемент; с другой стороны, для того чтобы обеспечить возможность поиска и устранения \textit{взаимоблокировок}, необходимо явно указывать тот факт, что какой-либо процесс хочет получить доступ к какому-либо заблокированному другим процессом sc-элементу. Для того чтобы иметь возможность указать, какие процессы пытаются заблокировать уже заблокированный \textit{sc-элемент}, предлагается наряду с отношением \textit{блокировка*} использовать отношение \textit{планируемая блокировка*}, полностью аналогичное отношению \textit{блокировка*}.\\
    Описанный механизм регулирует также и процессы поиска, поскольку поиск и сохранение некоторого sc-элемента предполагает установку \textit{блокировки на любое изменение}. Кроме того, следует учитывать, что на один sc-элемент \textit{блокировка на любое изменение} может быть установлена после \textit{блокировки на удаление}, соответствующей другому процессу. В этом случае использовать отношение \textit{планируемые блокировки*} нет необходимости.}
\scntext{примечание}{Действие проверки наличия на некотором sc-элементе блокировки и в зависимости от результата проверки, установки блокировки или планируемой блокировки (с указанием приоритета при необходимости) должно быть реализовано как транзакция.}

\scnheader{планируемая блокировка*}
\scnsubset{блокировка*}
\scntext{пояснение}{Процесс, которому в соответствие поставлена \textit{планируемая блокировка*}, приостанавливает выполнение до тех пор, пока уже установленные блокировки не будут сняты, после чего \textit{планируемая блокировка*} становится реальной \textit{блокировкой*} и процесс продолжает выполнение в соответствии с общими правилами.}

\scnheader{приоритет блокировки*}
\scnrelfrom{область определения}{планируемая блокировка*}
\scntext{пояснение}{В случае, когда на один и тот же sc-элемент планируют установить блокировку сразу несколько процессов, используется отношение \textit{приоритет блокировки*}, связывающее между собой пары отношения \textit{планируемая блокировка*}. Как правило, приоритет блокировки определяется тем, какой из процессов раньше попытался установить блокировку на рассматриваемый sc-элемент, хотя в общем случае приоритет может устанавливаться или меняться в зависимости от дополнительных критериев.}

\scnheader{действие удаления sc-элементов}
\scntext{примечание}{В случае попытки удаления некоторого sc-элемента некоторым процессом удаление может быть осуществлено только в случае, когда на данный sc-элемент не установлена (и не планируется) ни одна блокировка каким-либо другим процессом.\\
    В других случаях необходимо обеспечить корректное завершение выполнения всех процессов, работающих с данным sc-элементом, и только потом удалить его физически.\\
    Для реализации такой возможности каждому процессу в соответствие может быть поставлено множество удаляемых данным процессом sc-элементов.}
\scntext{примечание}{Действие проверки наличия блокировок или планируемых блокировок на удаляемый sc-элемент и, собственно, его удаление или добавление во множество удаляемых sc-элементов для соответствующего процесса должно быть реализовано как транзакция.}

\scnheader{удаляемые sc-элементы*}
\scnrelfrom{первый домен}{процесс в sc-памяти}
\scntext{пояснение}{Sc-элементы, попавшие во множество удаляемых sc-элементов некоторого процесса в sc-памяти, доступны процессам, уже установившим (или планирующим установить) на эти sc-элементы блокировки ранее (до попытки его удаления), а для всех остальных процессов эти sc-элементы уже считаются удаленными. Процесс, пытающийся удалить sc-элемент, приостанавливает свое выполнение до того момента, пока все заблокировавшие и планирующие заблокировать данный sc-элемент процессы не разблокируют его. В общем случае один sc-элемент может входить во множества удаляемых элементов одновременно для нескольких процессов, в этом случае все такие процессы одновременно продолжат выполнение после снятия с этого sc-элемента всех блокировок. Если удаление пытается осуществить один из процессов, уже установивший на указанный sc-элемент блокировку, то алгоритм действий остается прежним --- sc-элемент добавляется во множество удаляемых данным процессом sc-элементов и будет физически удален, как только все остальные процессы, установившие на данный sc-элемент блокировки, снимут их.}

\scnheader{действие снятия блокировки с некоторого sc-элемента}
\begin{scnrelfromvector}{алгоритм выполнения}
    \scnfileitem{Если на данный sc-элемент установлена одна или несколько \textit{планируемых блокировок*}, то первая из них по приоритету (или единственная) становится \textit{блокировкой*}, соответствующий ей процесс продолжает выполнение (становится настоящей сущностью).}
    \scnfileitem{Связка отношения приоритет выполнения, соответствовавшая удаленной связке отношения \textit{планируемая блокировка*} также удаляется, т. е. приоритет смещается на одну позицию.}
    \scnfileitem{Если \textit{планируемых блокировок*}, установленных на данный sc-элемент, нет, но он попадает во множество удаляемых sc-элементов для одного или нескольких процессов, то рассматриваемый sc-элемент физически удаляется, а приостановленные до его удаления процессы продолжают свое выполнение (становятся настоящими сущностями).}
    \scnfileitem{Если на данный sc-элемент не установлены планируемые блокировки, и он не входит во множество удаляемых для какого-либо процесса, то блокировка просто снимается без каких-либо дополнительных изменений.}
\end{scnrelfromvector}

\scnheader{транзакция в sc-памяти}
\begin{scnsubdividing}
    \scnitem{поиск некоторой конструкции в sc-памяти и автоматическая установка блокировки на любое изменение на найденные sc-элементы}
    \scnitem{генерация некоторого sc-элемента и автоматическая установка на него полной блокировки}
    \scnitem{проверка наличия на некотором sc-элементе блокировки и в зависимости от результата проверки установка блокировки или планируемой блокировки}
    \scnitem{проверка наличия блокировок или планируемых блокировок на удаляемый sc-элемент и собственно его удаление или добавление во множество удаляемых sc-элементов для соответствующего процесса}
    \scnitem{снятие блокировки с заданного sc-элемента и при необходимости установка первой по приоритету планируемой блокировки или удаление данного sc-элемента, если он входит во множество удаляемых sc-элементов для некоторого процесса}
    \scnitem{поиск подпроцессов процесса и добавление их во множество отложенных действий в случае добавления самого процесса в данное множество}
    \scnitem{поиск подпроцессов процесса и удаление их из множества отложенных действий в случае удаления самого процесса из данного множества}
\end{scnsubdividing}

\scnheader{абстрактный программный sc-агент}
\scntext{примечание}{При реализации \textit{абстрактных программных sc-агентов} на \textit{языке SCP}, соблюдение всех принципов синхронизации соответствующих этим sc-агентам процессов обеспечивается на уровне \textit{sc-агентов интерпретации scp-программ}, т. е. средствами \textit{платформы интерпретации sc-моделей}. При реализации \textit{абстрактных программных sc-агентов} на уровне платформы, соблюдение всех принципов синхронизации возлагается, во-первых, непосредственно на разработчика агентов, во-вторых, --- на разработчика платформы. Так, например, платформа может предоставлять доступ к хранимым в sc-памяти элементам через некоторый программный интерфейс, уже учитывающий принципы работы с блокировками, что избавит разработчика агентов от необходимости учитывать все эти принципы вручную.}
\begin{scnrelfromset}{принципы работы}
    \scnfileitem{В результате появления в sc-памяти некоторой конструкции, удовлетворяющей условию инициирования какого-либо \textit{абстрактного sc-агента}, реализованного при помощи \textit{Языка SCP}, в \textit{sc-памяти} генерируется и инициируется \textit{scp-процесс}. В качестве шаблона для генерации используется \textit{агентная scp-программа}, соответствующая данному \textit{абстрактному sc-агенту}.}
    \scnfileitem{Каждый такой \textit{scp-процесс}, соответствующий некоторой \textit{агентной \mbox{scp-программе}}, может быть связан с набором структур, описывающих блокировки различных типов. Таким образом, синхронизация взаимодействия параллельно выполняемых \textit{scp-процесcов} осуществляется так же, как и в случае любых других \textit{действий в sc-памяти}.}
    \scnfileitem{Несмотря на то что каждый \textit{scp-оператор} представляет собой атомарное действие в sc-памяти, являющееся поддействием в рамках всего \textit{\mbox{scp-процесса}}, блокировки, соответствующие одному оператору, не вводятся, чтобы избежать громоздкости и избытка дополнительных системных конструкций, создаваемых при выполнении некоторого \textit{scp-процесса}. Вместо этого используются блокировки, общие для всего \textit{scp-процесса}. Таким образом, \textit{агенты интерпретации scp-программ} работают только с учетом блокировок, общих для всего интерпретируемого \textit{scp-процесса}.}
    \scnfileitem{Процессы, описывающие деятельность агентов интерпретации \textit{scp-программ}, как правило, не создаются, следовательно, и не вводятся соответствующие им блокировки. Поскольку такие агенты работают с уникальным scp-процессом и их число ограничено и известно, то использование блокировок для их синхронизации не требуется.}
    \scnfileitem{В случае приостановки \textit{scp-процесса} (добавления его во множество \textit{отложенных действий}) в соответствии с общими правилами синхронизации все его дочерние процессы также должны быть приостановлены. В связи с этим все \textit{scp-операторы}, которые в этот момент являются \textit{настоящими сущностями}, становятся \textit{отложенными действиями}.}
    \scnfileitem{Во избежание нежелательных изменений в самом теле \textit{scp-процесса}, вся конструкция, сгенерированная на основе некоторой \textit{scp-программы} (весь \textit{sc-текст}, описывающий декомпозицию \textit{scp-процесса} на \textit{scp-операторы}), должна быть добавлена в \textit{полную блокировку}, соответствующую данному \textit{scp-процессу}.}
    \scnfileitem{При необходимости разблокировать или заблокировать некоторую конструкцию каким-либо типом блокировки используются соответствующие \textit{scp-операторы} класса \textit{scp-оператор управления блокировками}.}
    \scnfileitem{После завершения выполнения некоторого scp-процесса его текст, как правило, удаляется из \textit{sc-памяти}, а все заблокированные конструкции освобождаются (разрушаются знаки структур, обозначавших блокировки).}
    \scnfileitem{Как правило, частный \textit{класс действий}, соответствующий конкретной \textit{scp-программе}, явно не вводится, а используется более общий класс \textit{scp-процесс}, за исключением тех случаев, когда введение специального \textit{класса действий} необходимо по каким-либо другим соображениям.}
\end{scnrelfromset}

\scnheader{блокировка*}
\scntext{примечание}{В общем случае весь механизм блокировок может описываться как на уровне SC-кода (для повышения уровня платформенной независимости), так и при необходимости может быть реализован на уровне \textit{платформы интерпретации sc-моделей}, например, для повышения производительности. Для этого каждому выполняемому в sc-памяти процессу на нижнем уровне может быть поставлена в соответствие некая уникальная таблица, в каждый момент времени содержащая перечень заблокированных элементов с указанием типа блокировки.}
\begin{scnrelfromvector}{пример применения}
    \scnitem{\scnfileimage[20em]{Contents/part_ps/src/images/sd_agents/plan_lock_1.png}}
    \begin{scnindent}
        \scntext{пояснение}{В данном примере \textit{Процесс1} непосредственно работает с sc-элементом \textit{\textbf{e1}}, \textit{Процесс2} и \textit{Процесс3} планируют установить блокировку на любое изменение и блокировку на удаление соответственно, причем \textit{Процесс2} попытался установить свою блокировку раньше, чем \textit{Процесс3}, поэтому согласно направлению связки отношения \textit{приоритет блокировки*}, его блокировка будет установлена раньше. \textit{Процесс4} и \textit{Процесс5} ожидают снятия всех блокировок и планируемых блокировок, после чего \textit{\textbf{e1}} будет удален и \textit{Процесс1} и \textit{Процесс2} продолжат свое выполнение. Никакие другие планируемые блокировки установлены быть уже не могут, поскольку \textit{\textbf{e1}} попал во множество удаляемых sc-элементов как минимум одного процесса и, в соответствии с изложенными выше правилами, все остальные процессы кроме \textit{Процесс1}-\textit{Процесс5}, уже не смогут получить доступ к этому sc-элементу. Выполняемый процесс принадлежит множеству настоящая сущность, приостановленные --- множеству отложенное действие.}
    \end{scnindent}
    \scnitem{\scnfileimage[20em]{Contents/part_ps/src/images/sd_agents/plan_lock_2.png}}
    \begin{scnindent}
        \scntext{пояснение}{После того как \textit{Процесс1} разблокировал sc-элемент \textit{\textbf{e1}}, этот элемент будет заблокирован \textit{Процессом2}, и \textit{Процесс2} продолжит выполнение. \textit{Планируемая блокировка*}, установленная \textit{Процессом2}, становится обычной \textit{блокировкой*}.}
    \end{scnindent}
    \scnitem{\scnfileimage[20em]{Contents/part_ps/src/images/sd_agents/plan_lock_3.png}}
    \begin{scnindent}
        \scntext{пояснение}{После того как \textit{Процесс2} разблокировал sc-элемент \textit{\textbf{e1}}, этот элемент будет заблокирован \textit{Процессом3}, и \textit{Процесс3} продолжит выполнение.}
    \end{scnindent}
    \scnitem{\scnfileimage[20em]{Contents/part_ps/src/images/sd_agents/plan_lock_4.png}}
    \begin{scnindent}
        \scntext{пояснение}{Когда все процессы снимут блокировки с sc-элемента \textit{\textbf{e1}}, он может быть физически удален и \textit{Процесс4} и \textit{Процесс5} продолжат выполнение.}
    \end{scnindent}
\end{scnrelfromvector}

\scnheader{взаимоблокировка*}
\scntext{пояснение}{В зависимости от конкретных \textit{типов блокировок}, установленных параллельно выполняемыми процессами на некоторые sc-элементы, и того, какие конкретно действия с этими \textit{sc-элементами} предполагается выполнить, далее в рамках выполнения этих процессов возможны ситуации взаимоблокировки, когда каждый из указанных процессов будет ожидать снятия блокировки вторым процессом с нужного \textit{sc-элемента}, не снимая при этом установленной им самим блокировки с \textit{sc-элемента}, доступ к которому необходим второму процессу.\\
    В случае, когда хотя бы одна из блокировок является \textit{полной блокировкой}, ситуация взаимоблокировки возникнуть не может, поскольку \textit{sc-элементы}, попавшие в \textit{полную блокировку} некоторого \textit{scp-процесса}, не доступны другим \textit{scp-процессам} даже для чтения и, таким образом, остальные \textit{scp-процессы} будут работать так, как будто заблокированные \textit{sc-элементы} просто отсутствуют в текущем состоянии \textit{sc-памяти}.\\
    В случаях, когда ни одна из установленных блокировок не является \textit{полной блокировкой}, возможно появление взаимоблокировок.}
\scntext{примечание}{Устранение \textit{взаимоблокировки} невозможно без вмешательства специализированного \textit{sc-метаагента}, который имеет право игнорировать блокировки, установленные другими процессами.\\
    В общем случае проблема конкретной взаимоблокировки может быть решена путем выполнения специализированным \textit{sc-метаагентом} следующих шагов:
        \begin{scnitemize}
            \item откат нескольких операций, выполненных одним из участвующих во взаимоблокировке процессов на столько шагов назад, насколько это необходимо для того, чтобы второй процесс получил доступ к необходимым \textit{sc-элементам} и смог продолжить выполнение;
            \item ожидание выполнения второго процесса вплоть до завершения или до снятия им всех блокировок с \textit{sc-элементов}, доступ к которым необходимо получить первому процессу;
            \item повторное выполнение в рамках первого процесса отмененных операций и продолжение его выполнения, но уже с учетом изменений в памяти, внесенных вторым процессом.
        \end{scnitemize}}

\scnheader{sc-метаагент}
\scntext{пояснение}{Для \textit{sc-метаагентов} все sc-элементы, в том числе описывающие блокировки, планируемые блокировки и т. д., полностью эквивалентны между собой с точки зрения доступа к ним, т. е. любой \textit{sc-метаагент} имеет доступ к любым sc-элементам, даже попавшим в полную блокировку для какого-либо другого процесса. Это необходимо для того, чтобы \textit{sc-метаагенты} смогли выявлять и устранять различные проблемы, например, описанную выше проблему взаимоблокировки.\\
    Таким образом, проблема синхронизации деятельности \textit{sc-метаагентов} требует введения дополнительных правил.\\
    Указанную проблему разделим на две более частные:
        \begin{scnitemize}
            \item обеспечение синхронизации деятельности \textit{sc-метаагентов} между собой;
            \item обеспечение синхронизации деятельности \textit{sc-метаагентов} и \textit{программных sc-агентов}.
        \end{scnitemize}
    Первую проблему предлагается решить за счет запрета параллельного выполнения \textit{sc-метаагентов}. Таким образом, в каждый момент времени в рамках одной \textit{ostis-системы} может существовать только один процесс, соответствующий \textit{sc-метаагенту} и являющийся \textit{настоящей сущностью}.\\
    Вторую проблему предлагается решить за счет введения дополнительных привилегий для \textit{sc-метаагентов} при обращении к какому-либо sc-элементу. Для этого достаточно одного правила:\\
    Если некоторый sc-элемент стал использоваться в рамках процесса, соответствующего \textit{sc-метаагенту} (например, стал элементом хотя бы одного scp-оператора, входящего в данный процесс), то все процессы, которым в соответствующие блокировки   попадает указанный sc-элемент, становятся отложенными действиями (приостанавливают выполнение). Как только указанный sc-элемент перестает использоваться в рамках процесса, соответствующего \textit{sc-метаагенту}, все приостановленные по этой причине процессы продолжают выполнение.\\
    Рассмотренные ограничения не ухудшают производительность ostis-системы существенно, поскольку \textit{sc-метаагенты} предназначены для решения достаточно узкого класса задач, которые, как показал опыт практической разработки прототипов различных \textit{ostis-систем}, возникают достаточно редко.}
    \scntext{примечание}{Стоит отметить, что возможна ситуация, при которой выполнение некоторого процесса в sc-памяти прервано по причине возникновения какой-либо ошибки. В таком случае существует вероятность того, что блокировка, установленная данным процессом не будет снята до тех пор, пока этого не сделает sc-метаагент, обнаруживший подобную ситуацию. Однако указанная проблема на уровне sc-модели может быть решена лишь частично. Для случаев, когда ошибка возникает при интерпретации scp-программы, проблема отслеживается scp-интепретатором и в памяти формируется соответствующая конструкция, сообщающая о проблеме sc-метаагенту. Случаи, когда возникла ошибка на уровне scp-интерпретатора или sc-хранилища, должны рассматриваться на уровне платформы интерпретации sc-моделей.}

\bigskip
\end{scnsubstruct}
\scnendsegmentcomment{Принципы синхронизации деятельности sc-агентов}

\bigskip
\end{scnsubstruct}
\scnendcurrentsectioncomment
\end{SCn}


\scsubsection[
    \protect\scneditor{Шункевич Д.В.}
    \protect\scnmonographychapter{Глава 3.2. Ситуационное управление обработкой знаний в интеллектуальных компьютерных системах нового поколения}
    ]{Предметная область и онтология Базового языка программирования ostis-систем}
\label{sd_scp}
\begin{SCn}
\scnsectionheader{Предметная область и онтология Базового языка программирования ostis-систем}
\begin{scnsubstruct}
	\begin{scnrelfromlist}{дочерний раздел}
		\scnitem{\nameref{sd_scp_denote_sem}}
		\scnitem{\nameref{sd_scp_oper_sem}}
	\end{scnrelfromlist}
	
\scnheader{Предметная область Базового языка программирования ostis-систем (языка SCP --- Semantic Code Programming)}
\scnidtf{Предметная область Базового языка программирования ostis-систем}
\scnidtf{Предметная область Языка SCP}
\scntext{примечание}{В данную предметную область включаются все тексты программ Языка SCP. В ней исследуется типология операторов этих программ и заданные на них отношения.}
\scniselement{предметная область}
\begin{scnhaselementrole}{максимальный класс объектов исследования}
	{scp-программа}
\end{scnhaselementrole}
\begin{scnhaselementrolelist}{класс объектов исследования}
	%TODO: check by human--->
	\scnitem{агентная scp-программа}
	\scnitem{scp-процесс}
	\scnitem{scp-оператор}
	\scnitem{атомарный тип scp-оператора}
	%<---TODO: check by human
\end{scnhaselementrolelist}
\begin{scnhaselementrolelist}{исследуемое отношение}
	%TODO: check by human--->
	\scnitem{начальный оператор\scnrolesign}
	\scnitem{параметр scp-программы\scnrolesign}
	\scnitem{in-параметр\scnrolesign}
	\scnitem{out-параметр\scnrolesign}
	\scnitem{scp-операнд\scnrolesign}
	%<---TODO: check by human
\end{scnhaselementrolelist}

\scnheader{Язык SCP}
\scnidtftext{часто используемый sc-идентификатор}{scp-программа}
\scntext{пояснение}{В качестве базового языка для описания программ обработки текстов\textit{SC-кода} предлагается \textit{Язык SCP}.\\
	\textit{Язык SCP} --- это графовый язык процедурного программирования, предназначенный для эффективной обработки \textit{sc-текстов}. \textit{Язык SCP} является языком параллельного асинхронного программирования.\\
	Языком представления данных для текстов \textit{Языка SCP} (\textit{scp-программ}) является \textit{SC-код} и, соответственно, любые варианты его внешнего представления. \textit{Язык SCP} сам построен на основе \textit{SC-кода}, в следствие чего \textit{scp-программы} сами по себе могут входить в состав обрабатываемых данных для \textit{scp-программ}, в т.ч. по отношению к самим себе. Таким образом, \textit{язык SCP} предоставляет возможность построения реконфигурируемых программ. Однако для обеспечения возможности реконфигурирования программы непосредственно в процессе ее интерпретации необходимо на уровне интерпретатора \textit{Языка SCP (Aбстрактной scp-машины)} обеспечить уникальность каждой исполняемой копии исходной программы. Такую исполняемую копию, сгенерированную на основе \textit{scp-программы}, будем называть \textit{scp-процессом}. Включение знака некоторого \textit{действия в sc-памяти} во множество \textit{scp-процессов} гарантирует тот факт, что в декомпозиции данного действия будут присутствовать только знаки элементарных действий (\textit{scp-операторов}), которые может интерпретировать реализация \textit{Aбстрактной scp-машины} (интерпретатора scp-программ).\\
	\textit{Язык SCP} рассматривается как ассемблер для семантического компьютера.}

\scnheader{Базовая модель обработки sc-текстов}
\begin{scnreltoset}{объединение}
	%TODO: check by human--->
	\scnitem{Предметная область Базового языка программирования ostis-систем}
	\scnitem{Модель Абстрактной scp-машины}
	%<---TODO: check by human
\end{scnreltoset}
\begin{scnrelfromset}{особенности}
	%TODO: check by human--->
	\scnfileitem{Тексты программ \textit{Языка SCP} записываются при помощи тех же унифицированных семантических сетей, что и обрабатываемая информация, таким образом, можно сказать, что \textit{Синтаксис языка SCP} на базовом уровне совпадает с \textit{Синтаксисом SC-кода}.}
	\scnfileitem{Подход к интерпретации \textit{scp-программ} предполагает создание при каждом вызове \textit{scp-программы} уникального \textit{scp-процесса}.}
	%<---TODO: check by human
\end{scnrelfromset}
\begin{scnrelfromset}{достоинства}
	%TODO: check by human--->
	\scnfileitem{Одновременно в общей памяти могут выполняться несколько независимых\textit{sc-агентов}, при этом разные копии \textit{sc-агентов} могут выполняться на разных серверах, за счет распределенной реализации интерпретатора sc-моделей (\textit{платформы реализации sc-моделей компьютерных систем}). Более того, \textit{Язык SCP} позволяет осуществлять параллельные асинхронные вызовы подпрограмм с последующей синхронизацией, и даже параллельно выполнять операторы в рамках одной \textit{scp-программы}.}
	\scnfileitem{Перенос \textit{sc-агента} из одной системы в другую заключается в простом переносе фрагмента базы знаний, без каких-либо дополнительных операций, зависящих от платформы интерпретации.}
	\scnfileitem{Тот факт, что спецификации \textit{sc-агентов} и их программы могут быть записаны на том же языке, что и обрабатываемые знания, существенно сокращает перечень специализированных средств, предназначенных для проектирования машин обработки знаний, и упрощает их разработку за счет использования более универсальных компонентов.}
	\scnfileitem{Тот факт, что для интерпретации \textit{scp-программы} создается соответствующий ей уникальный \textit{\mbox{scp-процесс}}, позволяет по возможности оптимизировать план выполнения перед его реализацией и даже непосредственно в процессе выполнения без потенциальной опасности испортить общий универсальный алгоритм всей программы. Более того, такой подход к проектированию и интерпретации программ позволяет говорить о возможности создания самореконфигурируемых программ.}
	%<---TODO: check by human
\end{scnrelfromset}

\scnheader{Абстрактная scp-машина}
\scnrelfrom{модель}{Модель Абстрактной scp-машины}
\scntext{примечание}{\textit{Абстрактная scp-машина} представляет собой интерпретатор \textit{scp-программ}, который должен являться частью \textit{платформы интерпретации sc-моделей компьютерных систем} (хотя в общем случае могут существовать варианты платформы, не содержащие такого интерпретатора, что, однако, не позволит использовать достоинства предлагаемой базовой модели}

\scnheader{scp-программа}
\scnsubset{программа в sc-памяти}
\scntext{пояснение}{Каждая \textbf{\textit{scp-программа}} представляет собой \textit{обобщенную структуру}, описывающую один из вариантов декомпозиции действий некоторого класса, выполняемых в sc-памяти. Знак \textit{sc-переменной}, соответствующей конкретному декомпозируемому действию является в рамках \textbf{\textit{scp-программы}} \textit{ключевым sc-элементом\scnrolesign}. Также явно указывается принадлежность данного знака множеству \textit{scp-процессов}.\\
	Принадлежность некоторого действия множеству \textit{scp-процессов} гарантирует тот факт, что в декомпозиции данного действия будут присутствовать только знаки элементарных действий (\textit{scp-операторов}), которые может интерпретировать реализация абстрактной scp-машины.\\
	Таким образом, каждая \textbf{\textit{scp-программа}} описывает в обобщенном виде декомпозицию некоторого \textit{\mbox{scp-процесса}} на взаимосвязанные \textit{scp-операторы}, с указанием, при их наличии, аргументов для данного \textit{scp-процесса}.\\
	По сути каждая \textbf{\textit{scp-программа}} представляет собой описание последовательности элементарных операций, которые необходимо выполнить над семантической сетью, чтобы выполнить более сложное действие некоторого класса.}
\scnrelfrom{описание примера}{\scnfileimage[20em]{Contents/part_ps/images/sd_scp/program_example.png}}
	\begin{scnindent}
		\scntext{пояснение}{В приведенном примере показана \textit{scp-программа}, состоящая из трех \textit{scp-операторов}. Данная программа проверяет, содержится ли в заданном множестве (первый параметр) заданный элемент (второй параметр), и, если нет, то добавляет его в это множество.}
	\end{scnindent}

\begin{scnhaselementrolelist}{пример}
	\scnitem{\_scp\_process}
	\scnisvarelement{scp-процесс}
	\scnhasvarelementrole{1;in-параметр}{\_set1}
	\scnhasvarelementrole{2;in-параметр}{\_element1}
	\scnvarrelto{декомпозиция действия}{\_...}
	\begin{scnindent}
		\scnhasvarelementrole{1}{\_ operator1}
		\begin{scnindent}
			\scnisvarelement{searchElStr3}
			\scnhasvarelementrole{1; scp-операнд с заданным значением; scp-константа}{\_set1}
			\scnhasvarelementrole{2; scp-операнд со свободным значением; scp-переменная; sc-дуга основного вида}{\_arc1}
			\scnhasvarelementrole{3; scp-операнд с заданным значением; scp-константа}{\_element1}
			\scnvarrelfrom{последовательность действий при отрицательном результате}{\_operator2}
			\scnvarrelfrom{последовательность действий при положительном результате}{\_operator3}
		\end{scnindent}
		\scnhasvarelement{\_operator2}
		\begin{scnindent}
			\scnisvarelement{genElStr3}
			\scnhasvarelementrole{1:: scp-операнд с заданным значением; scp-константа}{\_set1}
			\scnhasvarelementrole{2:: scp-операнд со свободным значением; scp-переменная; sc-дуга основного вида}{\_arc1}
			\scnhasvarelementrole{3:: scp-операнд с заданным значением; scp-константа}{\_element1}
			\scnvarrelfrom{следующий оператор}{\_operator3}
		\end{scnindent}
		\scnhasvarelement{\_operator3}
		\begin{scnindent}
			\scnisvarelement{return}
		\end{scnindent}
	\end{scnindent}
\end{scnhaselementrolelist}
\end{scnsubstruct}

\scnheader{агентная scp-программа}
\scnsubset{scp-программа}
\scntext{пояснение}{\textbf{\textit{агентные scp-программы}} представляют собой частный случай \textit{scp-программ} вообще, однако заслуживают отдельного рассмотрения, поскольку используются наиболее часто. \textit{Scp-программы} данного класса представляют собой реализации программ агентов обработки знаний и имеют жестко фиксированный набор параметров. Каждая такая программа имеет ровно два \textit{in-параметра\scnrolesign}. Значение первого параметра является знаком бинарной ориентированной пары, являющейся вторым компонентом связки отношения \textit{первичное условие инициирования*} для абстрактного \textit{sc-агента}, во множество \textit{программ sc-агента*} которого входит рассматриваемая \textbf{\textit{агентная scp-программа}}, и, по сути, описывает класс событий, на которые реагирует указанный sc-агент.\\
	Значением второго параметра является \textit{sc-элемент}, с которым непосредственно связано событие, в результате возникновения которого был инициирован соответствующий \textit{sc-агент}, т.е., например, сгенерированная либо удаляемая \textit{sc-дуга} или \textit{sc-ребро}.}

\scnheader{абстрактный sc-агент, реализуемый на Языке SCP}
\begin{scnrelfromset}{принципы реализации}
	%TODO: check by human--->
	\scnfileitem{Общие принципы организации взаимодействия \textit{sc-агентов} и пользователей \textit{ostis-системы} через общую\textit{sc-память}.}
	\scnfileitem{В результате появления в sc-памяти некоторой конструкции,удовлетворяющей условию инициирования какого-либо \textit{абстрактногоsc-агента}, реализованного при помощи \textit{Языка SCP}, в \textit{sc-памяти} генерируется и инициируется \textit{scp-процесс}. В качестве шаблона для генерации используется \textit{агентная scp-программа}, указанная во множестве программ соответствующего \textit{абстрактного sc-агента}.}
	\scnfileitem{Каждый такой \textit{scp-процесс}, соответствующий некоторой \textit{агентной scp-программе}, может быть связан с набором структур, описывающих блокировки различных типов. Таким образом, синхронизация взаимодействия параллельно выполняемых \textit{scp-процесcов} осуществляется так же, как и в случае любых других \textit{действий в sc-памяти}.}
	\scnfileitem{В рамках \textit{scp-процесса} могут создаваться дочерние \textit{scp-процессы}, однако синхронизация между ними при необходимости осуществляется посредством введения дополнительных внутренних блокировок. Таким образом, каждый \textit{scp-процесс} с точки зрения \textit{процессов в sc-памяти} является атомарным и законченным актом деятельности некоторого \textit{sc-агента}.}
	\scnfileitem{Во избежание нежелательных изменений в самом теле \textit{scp-процесса}, вся конструкция, сгенерированная на основе некоторой \textit{scp-программы} (весь текст \textit{scp-процесса}), должна быть добавлена в \textit{полную блокировку}, соответствующую данному \textit{scp-процессу}.}
	\scnfileitem{Все конструкции, сгенерированные в процессе выполнения \textit{scp-процесса}, автоматически попадают в \textit{полную блокировку}, соответствующую данному \textit{scp-процессу}. Дополнительно следует отметить, что знак самой этой структуры и вся метаинформация о ней также включаются в эту структуру.}
	\scnfileitem{При необходимости можно вручную разблокировать или заблокировать некоторую конструкцию каким-либо типом блокировки, используя соответствующие \textit{scp-операторы} класса \textit{scp-оператор управления блокировками}.}
	\scnfileitem{После завершения выполнения некоторого \textit{scp-процесса} его текст как правило, удаляется из \textit{\mbox{sc-памяти}}, а все заблокированные конструкции освобождаются (разрушаются знаки структур, обозначавших блокировки).}
	\scnfileitem{Несмотря на то, что каждый \textit{scp-оператор} представляет собой атомарное \textit{действие в sc-памяти}, дополнительные блокировки, соответствующие одному оператору не вводятся, чтобы избежать громоздкости и избытка дополнительных системных конструкций, создаваемых при выполнении некоторого \textit{scp-процесса}. Вместо этого используются блокировки, общие для всего \textit{scp-процесса}. Таким образом, агенты \textit{Абстрактной scp-машины} при интерпретации \textit{scp-операторов} работают только с учетом блокировок, общих для всего интерпретируемого \textit{scp-процесса}.}
	\scnfileitem{Как правило, частный \textit{класс действий}, соответствующий конкретной \textit{scp-программе} явно не вводится, а используется более общий класс \textit{scp-процесс}, за исключением тех случаев, когда введение специального \textit{класса действий} необходимо по каким-либо другим соображениям.}
	%<---TODO: check by human
\end{scnrelfromset}

\scnheader{scp-процесс}
\scntext{пояснение}{Под \textbf{\textit{scp-процессом}} понимается некоторое \textit{действие в sc-памяти}, однозначно описывающее конкретный акт выполнения некоторой \textit{scp-программы} для заданных исходных данных. Если \textit{scp-программа} описывает алгоритм решения какой-либо задачи в общем виде, то \textit{scp-процесс} обозначает конкретное действие, реализующее данный алгоритм для заданных входных параметро
	По сути, \textbf{\textit{scp-процесс}} представляет собой уникальную копию, созданную на основе \textit{scp-программы}, в которой каждой \textit{sc-переменной}, за исключением \textit{scp-переменных\scnrolesign}, соответствует сгенерированная \textit{sc-константа}.\\
	Принадлежность некоторого действия множеству \textit{scp-процессов} гарантирует тот факт, что в декомпозиции данного действия будут присутствовать только знаки элементарных действий (\textit{scp-операторов}), которые может интерпретировать реализация \textit{Абстрактной scp-машины}.}
\begin{scnrelfromvector}{пример выполнения}
	%TODO: check by human--->
	\scnitem{\scnfileimage[20em]{Contents/part_ps/images/sd_scp/process_example.png}}
		\begin{scnindent}
			\scntext{пояснение}{Осуществляется вызов \textit{scp-программы}. Генерируется соответствующий \textit{scp-процесс}. Происходит инициирование начального оператора scp-процесса \textit{Operator1}.} 
		\end{scnindent}
	\scnitem{\scnfileimage[20em]{Contents/part_ps/images/sd_scp/process_example2.png}}
		\begin{scnindent}	
			\scntext{пояснение}{Оператор \textit{Operator1} оказался безуспешно выполненным. Производится инициирование \textit{\mbox{scp-оператора} генерации трёхэлементной конструкции} \textit{Operator2}.}
		\end{scnindent}
	\scnitem{\scnfileimage[20em]{Contents/part_ps/images/sd_scp/process_example3.png}}
		\begin{scnindent}	
			\scntext{пояснение}{Оператор \textit{Operator2} выполнился. Производится инициирование \textit{scp-оператора завершения выполнения программы} \textit{Operator3}.}
		\end{scnindent}
	\scnitem{\scnfileimage[20em]{Contents/part_ps/images/sd_scp/process_example4.png}}
		\begin{scnindent}
			\scntext{пояснение}{Оператор \textit{Operator3} выполнился. Выполнение \textit{scp-процесса} завершается.}
		\end{scnindent}
%<---TODO: check by human

\end{scnrelfromvector}
\end{SCn}


\scsubsubsection[
    \protect\scnmonographychapter{Глава 3.2. Ситуационное управление обработкой знаний в интеллектуальных компьютерных системах нового поколения}
    ]{Предметная область и онтология синтаксиса Базового языка программирования ostis-систем}
\label{sd_scp_syntax}

\scsubsubsection[
    \protect\scnmonographychapter{Глава 3.2. Ситуационное управление обработкой знаний в интеллектуальных компьютерных системах нового поколения}
    ]{Предметная область и онтология денотационной семантики Базового языка программирования ostis-систем}
\label{sd_scp_denote_sem}
\begin{SCn}
\scnsectionheader{Предметная область и онтология денотационной семантики Базового языка программирования ostis-систем}
\begin{scnsubstruct}
	
\scnheader{Предметная область денотационной семантики языка SCP}
\scniselement{предметная область}
\scnhaselementrole{максимальный класс объектов исследования}{scp-оператор}
\begin{scnhaselementrolelist}{исследуемое отношение}
    \scnitem{scp-операнд\scnrolesign}
    \scnitem{scp-константа\scnrolesign}
    \scnitem{scp-переменная\scnrolesign}
    \scnitem{scp-операнд с заданным значением\scnrolesign}
    \scnitem{scp-операнд со свободным значением\scnrolesign}
    \scnitem{формируемое множество\scnrolesign}
    \scnitem{удаляемый sc-элемент\scnrolesign}
\end{scnhaselementrolelist}

\scnheader{scp-оператор}
\scnrelto{включение}{действие в sc-памяти}
\scnrelto{семейство подмножеств}{атомарный тип scp-оператора}
\begin{scnsubdividing}
	%TODO: check by human--->
	\scnitem{scp-оператор генерации конструкций}
		\begin{scnindent}
			\begin{scnsubdividing}
				%TODO: check by human--->
				\scnitem{scp-оператор генерации конструкции по произвольному образцу}
				\scnitem{scp-оператор генерации пятиэлементной конструкции}
				\scnitem{scp-оператор генерации трехэлементной конструкции}
				\scnitem{scp-оператор генерации одноэлементной конструкции}
				%<---TODO: check by human
			\end{scnsubdividing}
		\end{scnindent}
	\scnitem{scp-оператор ассоциативного поиска конструкций}
		\begin{scnindent}
			\begin{scnsubdividing}
				%TODO: check by human--->
				\scnitem{scp-оператор поиска конструкции по произвольному образцу}
				\scnitem{scp-оператор поиска пятиэлементной конструкции с формированием множеств}
				\scnitem{scp-оператор поиска трехэлементной конструкции с формированием множеств}
				\scnitem{scp-оператор поиска пятиэлементной конструкции}
				\scnitem{scp-оператор поиска трехэлементной конструкции}
				%<---TODO: check by human
			\end{scnsubdividing}
		\end{scnindent}
	\scnitem{scp-оператор удаления конструкций}
		\begin{scnindent}
			\begin{scnsubdividing}
				%TODO: check by human--->
				\scnitem{scp-оператор удаления множества элементов трехэлементной конструкции}
				\scnitem{scp-оператор удаления одноэлементной конструкции}
				\scnitem{scp-оператор удаления пятиэлементной конструкции}
				\scnitem{scp-оператор удаления трехэлементной конструкции}
				%<---TODO: check by human
			\end{scnsubdividing}
		\end{scnindent}
	\scnitem{scp-оператор проверки условий}
		\begin{scnindent}
			\begin{scnsubdividing}
				%TODO: check by human--->
				\scnitem{scp-оператор сравнения числовых содержимых файлов}
				\scnitem{scp-оператор проверки равенства числовых содержимых файлов}
				\scnitem{scp-оператор проверки совпадения значений операндов}
				\scnitem{scp-оператор проверки наличия содержимого у файла}
				\scnitem{scp-оператор проверки наличия значения у переменной}
				\scnitem{scp-оператор проверки типа sc-элемента}
				%<---TODO: check by human
			\end{scnsubdividing}
		\end{scnindent}
	\scnitem{scp-оператор управления значениями операндов}
		\begin{scnindent}	
			\begin{scnsubdividing}
				%TODO: check by human--->
				\scnitem{scp-оператор удаления значения переменной}
				\scnitem{scp-оператор присваивания значения переменной}
				%<---TODO: check by human
			\end{scnsubdividing}
		\end{scnindent}
	\scnitem{scp-оператор управления scp-процессами}
		\begin{scnindent}
			\begin{scnsubdividing}
				%TODO: check by human--->
				\scnitem{scp-оператор удаления значения переменной}
				\scnitem{scp-оператор завершения выполнения программы}
				\scnitem{конъюнкция предшествующих scp-операторов}
				\scnitem{scp-оператор ожидания завершения выполнения множества scp-программ}
				\scnitem{scp-оператор ожидания завершения выполнения scp-программы}
				\scnitem{scp-оператор асинхронного вызова подпрограммы}
				%<---TODO: check by human
			\end{scnsubdividing}
		\end{scnindent}
	\scnitem{scp-оператор управления событиями}
		\begin{scnindent}
		\begin{scnreltoset}{разбиение}
			%TODO: check by human--->
			\scnitem{scp-оператор ожидания события}
			%<---TODO: check by human
		\end{scnreltoset}
		\end{scnindent}
	\scnitem{scp-оператор обработки содержимых файлов}
		\begin{scnindent}
			\begin{scnsubdividing}
                %TODO: check by human--->
                \scnitem{scp-оператор вычисления арксинуса числового содержимого файла}
                \scnitem{scp-оператор вычисления арккосинуса числового содержимого файла}
                \scnitem{scp-оператор деления числовых содержимых файлов}
                \scnitem{scp-оператор умножения числовых содержимых файлов}
                \scnitem{scp-оператор вычитания числовых содержимых файлов}
                \scnitem{scp-оператор сложения числовых содержимых файлов}
                \scnitem{scp-оператор вычисления тангенса числового содержимого файла}
                \scnitem{scp-оператор вычисления косинуса числового содержимого файла}
                \scnitem{scp-оператор вычисления синуса числового содержимого файла}
                \scnitem{scp-оператор вычисления логарифма числового содержимого файла}
                \scnitem{scp-оператор возведения числового содержимого файла в степень}
                \scnitem{scp-оператор удаления содержимого файла}
                \scnitem{scp-оператор копирования содержимого файла}
                \scnitem{scp-оператор нахождения остатка от деления числовых содержимых файлов}
                \scnitem{scp-оператор нахождения целой части от деления числовых содержимых файлов}
                \scnitem{scp-оператор вычисления арктангенса числового содержимого файла}
                \scnitem{scp-оператор перевода в верхний регистр строкового содержимого файла}
                \scnitem{scp-оператор перевода в верхний регистр строкового содержимого файла}
                \scnitem{scp-оператор замены определенной части строкового содержимого файла на содержимое указанной файла}
                \scnitem{scp-оператор проверки совпадения конца строкового содержимого файла со строковом содержимым другого файла}
                \scnitem{scp-оператор проверки совпадения начальной части строкового содержимого файла со строковом содержимым другого файла}
                \scnitem{scp-оператор получения части строкового содержимого файла по индексам}
                \scnitem{scp-оператор поиска строкового содержимого файла в строковом содержимом другого файла}
                \scnitem{scp-оператор вычисления длины строкового содержимого файла}
                \scnitem{scp-оператор разбиения строки на подстроки}
                \scnitem{scp-оператор лексикографического сравнения строковых содержимых файлов}
                \scnitem{scp-оператор проверки равенства строковых содержимых файлов}
                %<---TODO: check by human
			\end{scnsubdividing}
		\end{scnindent}
	\scnitem{scp-оператор управления блокировками}
		\begin{scnindent}
			\begin{scnsubdividing}
                %TODO: check by human--->
                \scnitem{scp-оператор снятия всех блокировок данного scp-процесса}
                \scnitem{scp-оператор снятия блокировки с sc-элемента}
                \scnitem{scp-оператор установки полной блокировки на sc-элемент}
                \scnitem{scp-оператор установки блокировки на изменение sc-элемента}
                \scnitem{scp-оператор установки блокировки на удаление sc-элемента}
                \scnitem{scp-оператор снятия блокировки со структуры}
                \scnitem{scp-оператор установки полной блокировки на структуру}
                \scnitem{scp-оператор установки блокировки на изменение структуры}
                \scnitem{scp-оператор установки блокировки на удаление структуры}
                %<---TODO: check by human
			\end{scnsubdividing}
		\end{scnindent}
	%<---TODO: check by human
\end{scnsubdividing}

\scnheader{scp-операнд\scnrolesign}
\scnrelto{включение}{аргумент действия\scnrolesign}
\scniselement{неосновное понятие}
\scniselement{ролевое отношение}
\begin{scnsubdividing}
	%TODO: check by human--->
	\scnitem{scp-константа\scnrolesign}
	\scnitem{scp-переменная\scnrolesign}
	%<---TODO: check by human
\end{scnsubdividing}
\begin{scnsubdividing}
	%TODO: check by human--->
	\scnitem{scp-операнд с заданным значением\scnrolesign}
	\scnitem{scp-операнд со свободным значением\scnrolesign}
	%<---TODO: check by human
\end{scnsubdividing}
\begin{scnsubdividing}
	%TODO: check by human--->
	\scnitem{константный sc-элемент\scnrolesign}
	\scnitem{переменный sc-элемент\scnrolesign}
	%<---TODO: check by human
\end{scnsubdividing}
\begin{scnrelfromlist}{включение}
%TODO: check by human--->
	\scnitem{формируемое множество\scnrolesign}
		\begin{scnindent}
			\begin{scnsubdividing}
			%TODO: check by human--->
			\scnitem{формируемое множество 1\scnrolesign}
			\scnitem{формируемое множество 2\scnrolesign}
			\scnitem{формируемое множество 3\scnrolesign}
			\scnitem{формируемое множество 4\scnrolesign}
			\scnitem{формируемое множество 5\scnrolesign}
			%<---TODO: check by human
			\end{scnsubdividing}
		\end{scnindent}
	\scnitem{удаляемый sc-элемент\scnrolesign}
	\scnitem{тип sc-элемента\scnrolesign}
		\begin{scnindent}
			\begin{scnsubdividing}
				%TODO: check by human--->
				\scnitem{sc-узел\scnrolesign}
					\begin{scnindent}
						\begin{scnsubdividing}
							%TODO: check by human--->
							\scnitem{структура\scnrolesign}
							\scnitem{отношение\scnrolesign}
								\begin{scnindent}
									\scnrelfrom{включение}{ролевое отношение\scnrolesign}
								\end{scnindent}
							\scnitem{класс\scnrolesign}
							%<---TODO: check by human
						\end{scnsubdividing}
					\end{scnindent}
				\scnitem{sc-дуга\scnrolesign}
					\begin{scnindent}
						\begin{scnsubdividing}
							%TODO: check by human--->
							\scnitem{sc-дуга общего вида\scnrolesign}
							\scnitem{sc-дуга принадлежности\scnrolesign}
								\begin{scnindent}
									\scnrelfrom{включение}{sc-дуга основного вида\scnrolesign}
									\begin{scnsubdividing}
										%TODO: check by human--->
										\scnitem{позитивная sc-дуга принадлежности\scnrolesign}
										\scnitem{негативная sc-дуга принадлежности\scnrolesign}
										\scnitem{нечеткая sc-дуга принадлежности\scnrolesign}
										%<---TODO: check by human
									\end{scnsubdividing}
									\begin{scnsubdividing}
										%TODO: check by human--->
										\scnitem{временная sc-дуга принадлежности\scnrolesign}
										\scnitem{постоянная sc-дуга принадлежности\scnrolesign}
										%<---TODO: check by human
									\end{scnsubdividing}
								\end{scnindent}
							%<---TODO: check by human
						\end{scnsubdividing}
					\end{scnindent}
				\scnitem{sc-ребро\scnrolesign}
				\scnitem{файл\scnrolesign}
				%<---TODO: check by human
			\end{scnsubdividing}
		\end{scnindent}
	%<---TODO: check by human
\end{scnrelfromlist}
\scntext{пояснение}{Ролевое отношение \textit{scp-операнд\scnrolesign} является неосновным понятием и указывает на принадлежность аргументов \textit{scp-оператору}. Помимо указания какого-либо класса \textit{scp-операндов\scnrolesign} порядок аргументов \textit{scp-оператора} дополнительно уточняется \textit{ролевыми отношениями 1\scnrolesign}, \textit{2\scnrolesign} и т. д.}

\scnheader{scp-константа\scnrolesign}
\scntext{пояснение}{В рамках \textit{scp-программы} \textit{scp-константы\scnrolesign} явно участвуют в \textit{\mbox{scp-операторах}} в качестве элементов (в теоретико-множественном смысле) и напрямую обрабатываются при интерпретации \textit{scp-программы}. Константами в рамках \textit{scp-программы} могут быть \textit{sc-элементы} любого типа, как \textit{\mbox{sc-константы}}, так и \textit{\mbox{sc-переменные}}. Константа в рамках \textit{scp-программы} остается неизменной в течение всего срока интерпретации. Константа \textit{\mbox{scp-программы}} может быть рассмотрена как переменная, значение которой совпадает с самой переменной в каждый момент времени и изменено быть не может. Таким образом, далее будем считать, что \textit{scp-константа\scnrolesign} и ее значение это одно и то же. Каждый \textit{in-параметр\scnrolesign} при интерпретации каждой конкретной копии \textit{scp-программы} становится \textit{scp-константой\scnrolesign} в рамках всех ее операторов, хотя в исходном теле данной программы в каждом из этих операторов он является \textit{scp-переменной\scnrolesign}.}

\scnheader{scp-переменная\scnrolesign}
\scntext{пояснение}{В рамках \textit{scp-программы} \textit{scp-переменные\scnrolesign} не обрабатываются явно при интерпретации, обрабатываются значения переменных. Каждая переменная \textit{scp-программы} может иметь одно значение в каждый момент времени, т. е. представляет собой ситуативный \textit{синглетон}, элементом которого является текущее значение \textit{scp-переменной\scnrolesign}. Значение каждой \textit{scp-переменной\scnrolesign} может меняться в ходе интерпретации \textit{scp-программы}. При этом интерпретатор при обработке \textit{scp-оператора} работает непосредственно со значениями \textit{\mbox{scp-переменных\scnrolesign}}, а не самими \textit{scp-переменными\scnrolesign} (которые также являются узлами той же семантической сети).}

\scnheader{scp-операнд с заданным значением\scnrolesign}
\scntext{пояснение}{Значение операндов, помеченных ролевым отношением \textit{scp-операнд с заданным значением\scnrolesign}, считается заданным в рамках текущего \textit{scp-оператора}. Данное значение учитывается при выполнении \textit{scp-оператора} и остается неизменным после окончания выполнения \textit{scp-оператора}. Каждая \textit{scp-константа\scnrolesign} по умолчанию рассматривается как \textit{scp-операнд с заданным значением\scnrolesign}, в связи с чем явное использование данного ролевого отношения в таком случае является избыточным. В таком случае в качестве значения рассматривается непосредственно сам операнд. В случае, если отношением \textit{\mbox{scp-операнд} с заданным значением\scnrolesign} помечена \textit{scp-переменная\scnrolesign}, то осуществляется попытка поиска значения для данной \textit{scp-переменной\scnrolesign} (ее элемента). Если попытка оказалась безуспешной, то возникает ошибка времени выполнения, которая должна быть обработана соответствующим образом.\\
Любой \textit{scp-операнд с заданным значением\scnrolesign} независимо от конкретного типа \textit{scp-оператора} может быть \textit{scp-переменной\scnrolesign}.}

\scnheader{scp-операнд со свободным значением\scnrolesign}
\scntext{пояснение}{Значение операндов, помеченных ролевым отношением \textit{scp-операнд со свободным значением\scnrolesign}, считается свободным (не заданным заранее) в рамках текущего \textit{scp-оператора}. В начале выполнения \textit{scp-оператора} связь между \textit{scp-переменной\scnrolesign}, помеченной данным ролевым отношением, и ее элементом (значением) всегда удаляется. В результате выполнения данного оператора может быть либо сгенерировано новое значение \textit{scp-переменной\scnrolesign}, либо не сгенерировано, тогда \textit{scp-переменная\scnrolesign} будет считаться не имеющей значения. Ни одна \textit{scp-константа\scnrolesign} не может быть помечена как \textit{scp-операнд со свободным значением\scnrolesign}, поскольку константа не может изменять свое значение в ходе интерпретации \textit{scp-программы}.}
\end{scnsubstruct}
\end{SCn}

%В стандарте нет таблицы такой
%Таблица \ref{table_operands_roles} показывает возможные сочетания различных ролевых отношений, указывающих роль операнда в рамках scp-оператора:

%\begin{table}[H]
%  \caption{Роли операндов в рамках scp-оператора}\label{table_operands_roles}
%\begin{tabularx}{\hsize}{| p{43mm} | X | X |}
%  \hline
%  \textbf{Тип значения}
%  & \multirow{2}{*}{\textbf{\shortstack[l]{scp-операнд с\\ заданным значением\scnrolesign}}} & \multirow{2}{*}{\textbf{\shortstack[l]{scp-операнд со\\ свободным значением\scnrolesign}}} \\
%  \cline{0-0}
%  \textbf{Константность} & & \\
%\hline
%\textbf{scp-константа\scnrolesign} & Разрешено, может быть опущено & Запрещено \\
%\hline
%\textbf{scp-переменная\scnrolesign} & Разрешено, значение останется неизменным & Разрешено, значение переменной будет изменено либо потеряно\\
%\hline
%\end{tabularx}
%\end{table}
%}

\scsubsubsection[
    \protect\scnmonographychapter{Глава 3.2. Ситуационное управление обработкой знаний в интеллектуальных компьютерных системах нового поколения}
    ]{Предметная область и онтология операционной семантики Базового языка программирования ostis-систем}
\label{sd_scp_oper_sem}
\begin{SCn}
\scnsectionheader{Предметная область и онтология операционной семантики Базового языка программирования ostis-систем}
\begin{scnsubstruct}
	
\scnheader{Предметная область операционной семантики языка SCP}
\scniselement{предметная область}
\scnhaselementrole{класс объектов исследования}{Абстрактная scp-машина}
\begin{scnhaselementrolelist}{ключевой объект исследования}
	\scnitem{Абстрактный sc-агент создания scp-процессов}
	\scnitem{Абстрактный sc-агент интерпретации scp-операторов}
	\scnitem{Абстрактный sc-агент синхронизации процесса интерпретации scp-программ}
	\scnitem{Абстрактный sc-агент уничтожения scp-процессов}
	\scnitem{Абстрактный sc-агент синхронизации событий в sc-памяти и ее реализации}
	\scnitem{Абстрактный sc-агент трансляции сформированной спецификации события в sc-памяти во внутреннее представление}
	\scnitem{Абстрактный sc-агент обработки события в sc-памяти, инициирующего агентную scp-программу}
\end{scnhaselementrolelist}

\scnheader{Абстрактная scp-машина}
\begin{scnreltoset}{декомпозиция абстрактного sc-агента}
	%TODO: check by human--->
	\scnitem{Абстрактный sc-агент создания scp-процессов}
	\scnitem{Абстрактный sc-агент интерпретации scp-операторов}
	\scnitem{Абстрактный sc-агент синхронизации процесса интерпретации scp-программ}
	\scnitem{Абстрактный sc-агент уничтожения scp-процессов}
	\scnitem{Абстрактный sc-агент синхронизации событий в sc-памяти и ее реализации}
	%<---TODO: check by human
\end{scnreltoset}

\scnheader{Абстрактный sc-агент создания scp-процессов}
\scntext{пояснение}{Задачей \textit{Абстрактного} \textit{sc-агента создания scp-процессов} является создание \textit{scp-процессов}, соответствующих заданной \textit{scp-программе}. Данный \textit{\mbox{sc-агент}} активируется при появлении в \textit{sc-памяти} \textit{инициированного действия}, принадлежащего классу \textit{действие интерпретации scp-программы}.\\
	После проверки \textit{sc-агентом} условия инициирования выполняется создание \textit{scp-процесса} с учетом конкретных параметров интерпретации \textit{\mbox{scp-программы}}, после чего осуществляется поиск\textit{начального оператора \scnrolesign \mbox{scp-процесса}} и добавление его во множество \textit{настоящих сущностей}.}

\scnheader{Абстрактный sc-агент интерпретации scp-операторов}
\scntext{пояснение}{Задачей \textit{Абстрактного sc-агента интерпретации scp-операторов} является интерпретация операторов \textit{scp-программы}, то есть выполнение в \textit{sc-памяти} действий, описываемых конкретным \textit{\mbox{scp-оператором}}. Данный \textit{sc-агент} активируется при появлении в \textit{sc-памяти} \textit{scp-оператора}, принадлежащего классу \textit{настоящих сущностей}. После выполнения действия, описываемого \textit{scp-оператором}, \textit{scp-оператор} добавляется во множество \textit{прошлых сущностей}. В случае, когда семантика действия, описываемого \textit{\mbox{scp-оператором}}, предполагает возможность ветвления \textit{scp-программы} после выполнения данного \textit{\mbox{scp-оператора}}, то используется одно из подмножеств класса \textit{выполненных действий --- безуспешно выполненное действие} или \textit{успешно выполненное действие}.}

\scnheader{Абстрактный sc-агент синхронизации процесса интерпретации scp-программ}
\scntext{пояснение}{Задачей \textit{Абстрактного sc-агента синхронизации процесса интерпретации scp-программ} является обеспечение переходов между \textit{scp-операторами} в рамках одного \textit{scp-процесса}. Данный \textit{sc-агент} активизируется при добавлении некоторого \textit{scp-оператора} во множество \textit{прошлых сущностей}. Далее осуществляется переход по \textit{sc-дуге}, принадлежащей отношению \textit{последовательность действий*} (или более частным отношениям, в случае, если \textit{\mbox{scp-оператор}} был добавлен во множество \textit{успешно выполненных действий} или \textit{безуспешно выполненных действий}). При этом очередной \textit{scp-оператор} становится \textit{настоящей сущностью} (активным \textit{scp-оператором}) в том случае, если хотя бы один \textit{scp-оператор}, связанный с ним входящими \textit{sc-дугами}, принадлежащими отношению \textit{последовательность действий*} (или более частным отношениям), стал \textit{прошлой сущностью} (или, соответственно, подмножеством прошлых сущностей). В случае, когда необходимо дождаться завершения выполнения всех предыдущих операторов, для синхронизации используется оператор класса \textit{конъюнкция предшествующих операторов}.}

\scnheader{Абстрактный sc-агент уничтожения scp-процессов}
\scntext{пояснение}{Задачей \textit{Абстрактного sc-агента уничтожения scp-процессов} является уничтожение \textit{scp-процесса}, т. е. удаление из \textit{sc-памяти} всех составляющих его \textit{sc-элементов}. Данный \textit{sc-агент} активируется при появлении в \textit{sc-памяти} \textit{scp-процесса}, принадлежащего множеству \textit{прошлых сущностей}.\\
	При этом уничтожаемый \textit{scp-процесс} необязательно должен быть полностью сформирован. Необходимость уничтожения не до конца сформированного \textit{scp-процесса} может возникнуть в случае, если при создании \textit{scp-процесса} возникли проблемы, не позволяющие продолжить создание \textit{scp-процесса} и его выполнение.}

\scnheader{Абстрактный sc-агент синхронизации событий в sc-памяти и ее реализации}
\scntext{пояснение}{Задачей \textit{Абстрактного sc-агента синхронизации событий вsc-памяти и ее реализации} является обеспечение работы \textit{неатомарных sc-агентов}, реализованных на \textit{языке SCP}.}
\begin{scnreltoset}{декомпозиция абстрактного sc-агента}
	%TODO: check by human--->
	\scnitem{Абстрактный sc-агент трансляции сформированной спецификации события в sc-памяти во внутреннее представление}
	\scnitem{Абстрактный sc-агент обработки события в sc-памяти, инициирующего агентную scp-программу}
	%<---TODO: check by human
\end{scnreltoset}

\scnheader{Абстрактный sc-агент трансляции сформированной спецификации события в sc-памяти во внутреннее представление}
\scntext{пояснение}{Задачей \textit{\textbf{Абстрактного sc-агента трансляции сформированной спецификации события в sc-памяти во внутреннее представление}} является трансляция ориентированных пар, описывающих \textit{первичное условие инициирования*} некоторого \textit{\mbox{sc-агента}} во внутреннее представление элементарных событий на уровне \textit{\mbox{sc-хранилища}}, при условии, что этот \textit{sc-агент} реализован на платформенно-независимом уровне (с использованием \textit{языка SCP}). Условием инициирования данного \textit{sc-агента} является появление в \textit{\mbox{sc-памяти}} нового элемента множества \textit{активных sc-агентов}, для которого будет найдена и протранслирована соответствующая ориентированная пара.}

\scnheader{Абстрактный sc-агент обработки события в sc-памяти, инициирующего агентную scp-программу}
\scntext{пояснение}{Задачей \textit{Абстрактного sc-агента обработки события в sc-памяти, инициирующего агентную \mbox{scp-программу}}, является поиск \textit{агентной scp-программы}, входящей во множество \textit{программ sc-агента*} для каждого \textit{sc-агента}, первичное условие инициирования которого соответствует событию, произошедшему в \textit{sc-памяти}, а также генерация и инициирование действия, направленного на интерпретацию этой программы. В результате работы данного \textit{sc-агента} в \textit{sc-памяти} появляется \textit{инициированное действие}, принадлежащее классу \textit{действие} \textit{интерпретации scp-программы.}}
\end{scnsubstruct}
\end{SCn}


\scsubsection[
    \protect\scneditor{Сердюков Р.Е.}
    \protect\scnmonographychapter{Глава 3.2. Семантическая теория программ в интеллектуальных компьютерных системах нового поколения}
    ]{Предметная область и онтология программ и языков программирования для ostis-систем}
\label{sd_programs}
\begin{SCn}
\scnsectionheader{Предметная область и онтология программ и языков программирования для ostis-систем}
\begin{scnsubstruct}

\scnheader{Предметная область программ и языков программирования для ostis-систем}
\scniselement{предметная область}
\begin{scnrelfromset}{автор}
    \scnitem{Зотов Н.В.}
    \scnitem{Шункевич Д.В.}
\end{scnrelfromset}

\scntext{аннотация}{Несмотря на активное развитие и использование современных технологий и языков программирования, общей семантической теории программ, на основе которой можно было бы проектировать и разрабатывать прикладные системы, на данный момент не существует. В данной предметной области предлагается семантическая теория программ для ostis-систем. Рассматриваются особенности представления и ключевые моменты процесса интерпретации программ в ostis-системах.}

\begin{scnreltovector}{конкатенация сегментов}
    \scnitem{Проблемы текущего состояния в области разработки и применения языков программирования}
    \scnitem{Существующие онтологии языков программирования}
    \scnitem{Предлагаемый подход к разработке технологий программирования для ostis-систем}
    \scnitem{Синтаксис и семантика программ в ostis-системах}
    \scnitem{Методы и средства поддержки проектирования и разработки программ в ostis-системах}
    \scnitem{Комплекс свойств, определяющих эффективность программ в ostis-системах}
\end{scnreltovector}

\begin{scnhaselementrolelist}{ключевой знак}
    \scnitem{Семантическая теория программ}
	\scnitem{Семантическая теория программ для ostis-систем}
	\begin{scnindent}
		\scnidtf{Предлагаемый вариант теории для проектирования языков программирования и программ для интеллектуальных компьютерных систем нового поколения}
	\end{scnindent}
\end{scnhaselementrolelist}

\begin{scnhaselementrolelist}{класс объектов исследования}
	\scnitem{метод}
	\scnitem{язык представления методов}
	\scnitem{эффективность метода}
\end{scnhaselementrolelist}

\begin{scnhaselementrolelist}{исследуемое отношение}
	\scnitem{спецификация метода*}
	\scnitem{синтаксис метода*}
	\scnitem{денотационная семантика метода*}
	\scnitem{операционная семантика метода*}
	\scnitem{спецификация языка представления методов*}
	\scnitem{синтаксис языка представления методов*}
	\scnitem{денотационная семантика языка представления методов*}
	\scnitem{операционная семантика языка представления методов*}
\end{scnhaselementrolelist}

\begin{scnrelfromlist}{библиографическая ссылка}
	\scnitem{\scncite{Sebesta2012}}
	\scnitem{\scncite{Zapata2010}}
	\scnitem{\scncite{Golenkov2019a}}
	\scnitem{\scncite{Penta2020}}
	\scnitem{\scncite{Scalabrino2016}}
	\scnitem{\scncite{Golenkov2012}}
	\scnitem{\scncite{Brooks2021}}
	\scnitem{\scncite{Sellitto2022}}
	\scnitem{\scncite{Turner2007}}
	\scnitem{\scncite{Chaparro2014}}
	\scnitem{\scncite{Golenkov2022a}}
	\scnitem{\scncite{Golenkov2019}}
	\scnitem{\scncite{Eden2007}}
	\scnitem{\scncite{Lando2007}}
	\scnitem{\scncite{Lando2007a}}
	\scnitem{\scncite{Turner2014}}
	\scnitem{\scncite{Deikstra1978}}
	\scnitem{\scncite{Standart2021}}
	\scnitem{\scncite{Kasyanov2003}}
	\scnitem{\scncite{Petrov1978}}
	\scnitem{\scncite{Scott2006}}
	\scnitem{\scncite{Scott1972}}
	\scnitem{\scncite{Orlov2021}}
	\scnitem{\scncite{Lu2022}}
	\scnitem{\scncite{Gulykina2012}}
	\scnitem{\scncite{Pivovarchik2016}}
	\scnitem{\scncite{Ford2019}}
	\scnitem{\scncite{IMS}}
	\scnitem{\scncite{Pivovarchik2013}}
	\scnitem{\scncite{Tin1995}}
	\scnitem{\scncite{Schiitze1991}}
	\scnitem{\scncite{Black1993}}
	\scnitem{\scncite{Zotov2022a}}
\end{scnrelfromlist}

\begin{scnrelfromvector}{введение}
    \scnfileitem{За долгий период развития компьютерных систем практически сняты аппаратные ограничения на решение различных задач. Оставшиеся ограничения отводятся на долю программного обеспечения. Прежде всего эти ограничения связаны с текущими проблемами развития программного обеспечения}
    \begin{scnindent}
        \begin{scnrelfromset}{проблемы текущего состояния}
            \scnfileitem{\uline{Аппаратная сложность опережает} умение человечества строить \uline{программные компьютерные системы}, использующее потенциальные возможности аппаратуры.}
            \scnfileitem{Навыки и \uline{технологии} разработки программ \uline{отстают от требований}, предъявляемых к разработке программ нового поколения.}
            \scnfileitem{Возможностям эксплуатировать существующие программы угрожает \uline{низкое качество их разработки}.}
        \end{scnrelfromset}
        \scntext{решение}{Ключом к решению этих проблем является глубокое понимание и грамотное использование существующих \textit{языков программирования} как основного инструмента для массового создания \textit{программных компьютерных систем нового поколения}.}
    \end{scnindent}
    \scnfileitem{В данной предметной области акцент делается на достижение следующих результатов
        \begin{itemize}
            \item (1) изложение классических основ, отражающих накопленный мировой опыт в области разработки и применения современных \textit{языков программирования};
            \item и (2) систематизация основных результатов в этой области в виде единой унифицированной \textit{Семантической теории программ для интеллектуальных компьютерных систем нового поколения}, построенных по принципам \textit{Технологии OSTIS}.
        \end{itemize}}
    \scnfileitem{В данной предметной области подробно описываются проблемы текущего состояния в области \textit{технологий} и \textit{языков программирования}. Она посвящена базовым понятиям \textit{теории языков программирования}, дается обзорная характеристика областей применения \textit{языков программирования}, достаточно востребованных современным человеческим обществом, рассматриваются способы представления и интерпретации \textit{программ} различных \textit{языков программирования}, подробно описываются формы и содержание критериев для оценки \textit{эффективности языков}.}
\end{scnrelfromvector}

\scntext{примечание}{Проектирование и реализация \textit{программы} на каком-либо \textit{языке программирования} должна сводиться к описанию ее \textit{синтаксиса} и \textit{денотационной семантики} в базе знаний ostis-системы с помощью некоторой библиотеки предметных областей и онтологий программ, описываемой в рамках этой базы знаний. Для этого нужна онтология программ, которые позволили бы в достаточном объеме описывать программы на любых языках программирования в ostis-системах. Такой подход позволяет не только описывать сложноструктурированные объекты простым и понятным языком, но и позволяет унифицировать представление различных видов знаний. Тем самым, информация о программах и сами программы представляются на одном и том же языке (имеют один синтаксис), но содержательно описываются при помощи разных онтологий. Таким образом, \uline{решением всех проблем будет являться общая теория программ}, однозначно соответствующей некоторой онтологии программ, c помощью которых можно было бы описывать синтаксис и денотационную семантику любых программ в ostis-системах.}
\scntext{примечание}{Таким образом, результатом данной предметной области является \textit{Предметная область и онтология программ} (далее --- Предметная область и онтология методов), с помощью которой можно описывать синтаксис, денотационную и операционную семантику различных методов в ostis-системах. \textit{Предметная область и онтология методов} является дочерней предметной областью по отношению к \textit{Предметной области и онтологии информационных конструкций и языков}. Это означает, что она наследует все свойства исследуемых в ней понятий и отношений.}
	
\scnsegmentheader{Проблемы текущего состояния в области разработки и применения языков программирования}
\begin{scnsubstruct}
    \begin{scnhaselementset}
        \scnfileitem{Поскольку количество \textit{языков программирования} растет с увеличением потребности в них, то растут и потребности в описании этих \textit{языков программирования} для дальнейшего использования и проектирования прикладных систем. Это в свою очередь требует высокого уровня качества спецификации конкретного языка: и описания \textit{синтаксиса} и семантики конструкций этого языка, и описания средств и методов реновации инструментальных средств, обеспечивающих интерпретацию или трансляцию этого языка. То есть, с увеличением количества \textit{языков программирования} растет не только многообразие форм представления знаний (\textit{языков программирования}), но и количество \textit{программных компьютерных систем} на различных формах представления знаний.}
        \scnfileitem{Большое многообразие форм представления знаний, как говорилось выше, предоставляет большой спектр возможностей проектирования \textit{программных компьютерных систем} на каждой из них. Получается, чтобы произвести интеграцию нескольких \textit{программных компьютерных систем}, реализованных на разных \textit{языках программирования}, необходимо сделать так, чтобы системы могли коммуницировать между собой на каждом из тех языков, на котором они реализованы. Так, стремление к использованию существующих программных компонентов затрудняется реализацией самих компонентов, поскольку чтобы объединить эти компоненты необходимо изменить их программный код. Наличие многообразия форм затрудняет реализацию \textit{совместимых интероперабельных программных компьютерных систем}.}
        \scnfileitem{С ростом сложности программного кода, уменьшается количество способных понять его смысл. Современные разработчики создают \textit{программные компьютерные системы}, не учитывая полный ее жизненный цикл. Системы должны постоянно обновляться и совершенствоваться с развитием технологий, на которых она основана. Это должно обеспечиваться хорошей документацией реализации компонентов этих систем --- это снижает не только потребности в привлечении новых ресурсов и кадров, но и способствует снижению реинжиниринга \textit{программных компьютерных систем}.}
        \scnfileitem{Полная автоматизация проектирования \textit{программных компьютерных систем} невозможна, поскольку современные языки, на которых они проектируются не имеют свойства рефлексивности --- системы не могут познавать и понимать себя и развиваться в полной мере самостоятельно. Таким образом, существующие \textit{программные компьютерные системы} не являются как таковыми интеллектуальными, потому что не имеют необходимых им свойств.}
        \scnfileitem{Ключом к легкому и глубокому освоению конкретного языка как основного профессионального инструмента программиста является понимание общих принципов построения и применения языков программирования, описываемых их общей теорией. До сегодняшнего дня, общей \textit{Семантической теории языков программирования} до сих пор не существует, что затрудняет разработку, верификацию и использование новых и существующих \textit{языков программирования}. Без общей теории каждый может разрабатывать принципиально общие методы и средства так, как хочется, а не так, как требуется.}
        \scnfileitem{Достижение максимума услуг и средств при минимуме затрат возможно только путем глубокого понимания принципов построения \textit{языков программирования} за счет простоты средств и методов представления знаний. Сложное нужно сводить к простому и изъяснять простыми понятиями, не создавая дополнительной иллюзии важности.}
    \end{scnhaselementset}
    \begin{scnrelfromset}{смотрите}
        \scnitem{\scncite{Zapata2010}}
        \scnitem{\scncite{Golenkov2019a}}
        \scnitem{\scncite{Penta2020}}
        \scnitem{\scncite{Scalabrino2016}}
        \scnitem{\scncite{Golenkov2012}}
        \scnitem{\scncite{Brooks2021}}
        \scnitem{\scncite{Sellitto2022}}
        \scnitem{\scncite{Penta2020}}
        \scnitem{\scncite{Scalabrino2016}}
        \scnitem{\scncite{Turner2007}}
        \scnitem{\scncite{Golenkov2012}}
        \scnitem{\scncite{Sellitto2022}}
        \scnitem{\scncite{Chaparro2014}}
        \scnitem{Комплекс свойств, определяющий общий уровень качества кибернетической системы}
    \end{scnrelfromset}
    \begin{scnrelfromvector}{введение}
        \scnfileitem{В современную эру развития информационных технологий существует огромное количество \textit{языков программирования}, каждый из которых имеет свое важное назначение в области проектирования \textit{программных компьютерных систем}. Многообразие \textit{языков программирования} и решений, созданных на них, настолько велико, что очень легко потеряться в море информации о всех аспектах применения и проектирования \textit{языков программирования}. Кроме этого, основная проблема заключается не в количестве существующих решений в области разработки и применения современных \textit{языков программирования}, а количестве форм (!), на которых представляются конкретные \textit{языки программирования}. Так, \textit{декларативные знания}, то есть знания, являющиеся, например, спецификацией какой-то программы, и \textit{процедурные знания}, то есть знания, которые являются программами, принадлежащими какому-то \textit{языку программирования}, представляются совершенно различными способами, методами и средствами.}
        \begin{scnindent}
        	\begin{scnrelfromset}{смотрите}
        		\scnitem{\scncite{Sebesta2012}}
        	\end{scnrelfromset}
        \end{scnindent}
        \scnfileitem{Все рассматриваемые проблемы связаны и являются проблемами текущего состояния направлений развития в области \textit{Искусственного интеллекта}.}
        \begin{scnindent}
        	\begin{scnrelfromset}{смотрите}
        		\scnitem{\scncite{Golenkov2022a}}
        	\end{scnrelfromset}
        \end{scnindent}
        \scnfileitem{Итак, для решения перечисленных проблем необходимо создавать комфортные условия для реализации \textit{программных компьютерных систем}, семантически совместимых и интероперабельных между собой. В контексте \textit{языков программирования} необходима общая \textit{Семантическая теория программ для интеллектуальных компьютерных систем нового поколения}, которая:
        \begin{itemize}
            \item \uline{позволит} без больших усилий и затрат \uline{интегрировать имеющиеся решения} в области проектирования программ компьютерных систем;
            \item \uline{объединит формы представления знаний} декларативного и процедурного вида;
            \item \uline{будет иметь широкий спектр средств} не только для описания синтаксиса и семантики существующих языков программирования, но и для проектирования новых аналогов;
            \item \uline{будет понятна} не только человеку, но и машине;
            \item \uline{обозначит принципы}, по которым необходимо проектировать \textit{языки программирования нового поколения}.
        \end{itemize}}
    	\begin{scnindent}
    		\begin{scnrelfromset}{смотрите}
    			\scnitem{\scncite{Golenkov2019}}
    			\scnitem{\scncite{Zapata2010}}
    		\end{scnrelfromset}
    	\end{scnindent}
        \scnfileitem{К проектированию таких общих теорий, строго говоря, нужно подходить с высокой степенью важности. Проектируемые \textit{компьютерные системы} должны всегда иметь возможности использовать те свойства, которые им начертаны. Для того, чтобы и эта теория могла быть использована как некоторая система знаний о том, как надо проектировать и использовать \textit{языки программирования} и программы в \textit{программных компьютерных системах}, и том, как интерпретировать их \textit{программы}, необходимо, чтобы эта теория была описана средствами и методами, которыми проектируются эти \textit{программные компьютерные системы}. Речь идет о том, что принципиально важным подходом к проектированию общей теории программ является \textit{онтологический подход}.}
        \begin{scnindent}
        	\begin{scnrelfromset}{смотрите}
        		\scnitem{\scncite{Golenkov2019}}
        		\scnitem{\scncite{Zapata2010}}
        	\end{scnrelfromset}
        \end{scnindent}
        \scnfileitem{Для воплощения данных идей необходимо изучить и интегрировать опыт, накопленный в области разработки и применения \textit{современных языков программирования}. Поэтому далее будут рассмотрены результаты других исследований в области проектирования общей теории языков программирования и программ.}
    \end{scnrelfromvector}
\end{scnsubstruct}

\scnsegmentheader{Существующие онтологии языков программирования}
\begin{scnsubstruct}
    
    \scnheader{Семантическая теория программ}
    \scntext{примечание}{В большинстве, идеи, предлагаемые в научных работах по исследованию языков программирования, безусловно являются востребованными и полезными для проектирования \textit{программных компьютерных систем}. Так, идея о том, что языки программирования и программы, реализуемые на них, должны быть организованы в общую таксономию понятий, является основополагающей, поскольку обеспечивает наиболее качественную среду для проектирования и реализации \textit{программных компьютерных систем}. Общая теория программ нужна не только для того чтобы описывать термины и понятия как некоторую спецификацию, используемую для проектирования \textit{программных компьютерных систем} (что тоже немаловажно), но и для того, чтобы определять качество языков программирования и программ по таким вопросам, как: \scnqq{Является ли данный язык языком программирования}, \scnqq{Является ли данное знание программой}, \scnqq{Насколько эффективна данная программа}, \scnqq{Какова степень интеллекта данной программной системы} и так далее. Данные идеи предложены и рассмотрены в работах Raymond Turner.}
    \begin{scnindent}
    	\begin{scnrelfromset}{смотрите}
    		\scnitem{\scncite{Eden2007}}
    		\scnitem{\scncite{Turner2007}}
    	\end{scnrelfromset}
    \end{scnindent}
    
   	\scnheader{онтология языков программирования и программ}
    \scntext{примечание}{До сегодняшнего дня существует большое количество аналогов онтологий языков программирования и программ. Также стоит отметить разработанные онтологии программ, система понятий в которых определяется строго и однозначно на формальных языках: языках логики и языках описания грамматик формальных языков. Однако ни одна из них не является таким результатом, который можно было бы использовать при проектировании \textit{программных компьютерных систем} без существенных проблем. Разработанные онтологии сосредотачивают в себе лишь краткое описание связанных между собой понятий, но общей картины того, как данные онтологии можно использовать в конкретных задачах, почти не видно.}
    \begin{scnindent}
    	\begin{scnrelfromset}{смотрите}
    		\scnitem{\scncite{Lando2007}}
    		\scnitem{\scncite{Lando2007a}}
    		\scnitem{\scncite{Turner2014}}
    		\scnitem{\scncite{Turner2007}}
    	\end{scnrelfromset}
    \end{scnindent}
    
    \scnheader{язык представления методов}
    \scntext{примечание}{Сегодня встречаются и вовсе протовоположные суждения о назначении программ и языков программирования, противоречащие формальным основам Искусственного интеллекта. \textit{Программные компьютерные системы} должны быть не только понятны человеку, но и сами должны понимать себя, свои возможности, намерения, действия и цели, и понимать себе подобные кибернетические системы. Только таким образом человечество и результаты его деятельности в виде каких-то конкретных систем смогут работать сообща, дополняя друг друга и преумножая свои результаты.}
    \begin{scnindent}
    	\begin{scnrelfromset}{смотрите}
    		\scnitem{\scncite{Golenkov2012}}
    	\end{scnrelfromset}
    \scntext{примечание}{В результате анализа приведенных работ можно сделать вывод о том, что:
        \begin{itemize}
            \item \textit{общей теории программ и языков программирования}, которая могла быть задействована при решении любой прикладной задачи и представлении и реализации средств проектирования компьютерных систем, до сих пор не существует;
            \item унификация представления средств описания и реализации по этим описании как главный аргумент к оперированию смысловому представлению знаний, к полному взаимопониманию между \textit{программными компьютерными системами} вовсе не рассматривается;
            \item \textit{программы} и совокупности этих \textit{программ} в виде \textit{программных компьютерных систем} реализуются в большинстве случаев в индивидуальном порядке и плохо документируются, что усложняет их использование, интеграцию с другими программами и \textit{программными компьютерными системами}, тестирование и совершенствование.
        \end{itemize}}
	\end{scnindent}
\end{scnsubstruct}

\scnsegmentheader{Предлагаемый подход к разработке технологий программирования для ostis-систем}
\begin{scnsubstruct}
    \begin{scnrelfromlist}{ключевой знак}
        \scnitem{Принципы программирования в интеллектуальных компьютерных системах нового поколения}
    \end{scnrelfromlist}
    
    \scnheader{язык представления методов}
    \begin{scnrelfromset}{проблемы}
        \scnfileitem{Поскольку количество \textit{языков программирования} растет с увеличением потребности в них, то растут и потребности в описании этих \textit{языков программирования} для проектирования и разработки \textit{программных компьютерных систем} на этих языках. То есть, с увеличением количества \textit{языков программирования} растет не только многообразие форм представления знаний, но и количество \textit{программных компьютерных систем} на различных формах представления знаний.}
        \scnfileitem{Многообразие форм представления знаний в свою очередь требует не только качественной спецификации конкретного \textit{языка программирования} для разработки \textit{программ} на этом языке, но и новых требований к существующим разработчикам.}
        \scnfileitem{Новые требования к существующим разработчикам влекут за собой появление барьеров и для создания семантически совместимых и интероперабельных \textit{программных компьютерных систем}, и для обеспечения благоприятной среды для взаимодействия их разработчиков.}
    \end{scnrelfromset}
    \begin{scnindent}
	    \scntext{решение}{Для преодоления данных проблем нет необходимости пересматривать уже существующие решения в области разработки программного обеспечения. Необходимо создавать принципиально новые \textit{языки программирования}, а также реализовывать \textit{программные компьютерные системы} на них, в которых будут учтены и решены существующие проблемы. Для этого следует учитывать следующие \textit{принципы программирования} этих систем.}
        \begin{scnindent}
            \begin{scnrelfromset}{принципы программировния}
                \scnfileitem{Расширение многообразия форм представления знаний происходит за счет появления новых синтаксических конструкций в \textit{языках программирования}. Поэтому разработка \textit{языков программирования} должна сводиться к уточнению \textit{синтаксиса} и \textit{семантики} уже существующих \textit{языков программирования}. При этом все \textit{языки программирования} должны являться подъязыками некоторого базового \textit{языка программирования}.}
                \scnfileitem{Нет необходимости в создании дополнительных языков, с помощью которых можно описывать семантику программ на \textit{языках программирования}. Наоборот, \textit{язык программирования}, на котором разрабатываются программы, должен позволять своими же средствами описывать \textit{семантику} \textit{программ} на этом же языке.}
                \scnfileitem{Документирование \textit{программ}, в том числе \textit{программных компьютерных систем}, должно минимизироваться за счет этапов их качественного проектирования и разработки. Смысл конструкций \textit{программ} \textit{языков программирования} должен быть настолько ясным и понятным, чтобы использование \textit{программ} на этом \textit{языке программирования} не требовало дополнительных ресурсов и инструментов как и у разработчиков этих программ и систем, таких и у новых разработчиков.}
                \scnfileitem{Появление новых программ должно влечь за собой к расширению \textit{Библиотеки многократно используемых программ} и к уменьшению количества семантически эквивалетных программ. Таким образом, программы должны быть не только максимальным образом совместимыми между собой, но и открытыми для переиспользования в других \textit{программных компьютерных системах нового поколения}.}
                \scnfileitem{Полный жизненный цикл разработки новых программ должен обеспечиваться теми же средствами и \textit{языками программирования}, на которых разрабатываются эти программы.}
                \scnfileitem{Сложность программ и \textit{программных компьютерных систем} должна сводиться к минимуму. То, что выглядит сложно, должно и может быть сделано максимально просто.}
                \scnfileitem{Построение качественного коллектива \textit{программных компьютерных систем} может быть обеспечено только совместимостью и интероперабельностью самих систем, и коллективов тех разработчиков, которые их создают.}
                \scnfileitem{Ключом к решению всех этих проблем является общая \textit{Технология проектирования компьютерных систем нового поколения}, на базе которой можно построить общую \textit{Семантическую теорию программ} (дисциплину программирования) для \textit{интеллектуальных компьютерных систем нового поколения}, построенных по принципам \textit{Технологии OSTIS}.}
                \begin{scnindent}
                    \begin{scnrelfromset}{смотрите}
                        \scnitem{\scncite{Deikstra1978}}
                    \end{scnrelfromset}
                \end{scnindent}
            \end{scnrelfromset}
        \end{scnindent}
        \scntext{примечание}{Почему \textit{Технология OSTIS} является ключом к решению описанных проблем в области проектирования и применения \textit{языков программирования}?}
        \begin{scnindent}
            \begin{scnrelfromset}{ответ}
                \scnfileitem{Стандарт Технологии OSTIS уже реализует базовые средства, необходимые для проектирования и разработки интероперабельных \textit{программных компьютерных систем}, в основе которых лежит смысловое представление знаний. Это устраняет не только необходимость создания \textit{онтологий верхнего уровня}, которые должны быть использованы в общей теории программ как базовые для описания понятий этой теории, но и помогает проектировать решения согласованно с другими онтологиями. В результате формируется общая слаженная картина мира, которая (1) непротиворечива, то есть согласована, (2) однозначно трактуема, (3) универсальна и, (4) самое главное, понятна для каждого.}
                \scnfileitem{\textit{Технология OSTIS} проектируется одним языком унифицированного представления знаний, называемым \textit{SC-кодом}. Смысл \textit{программ} и \textit{языков программирования} понятен и однозначен тогда и только тогда, когда этот смысл описывается на одном общем языке, понятному любой \textit{кибернетической системе}.}
                \scnfileitem{\textit{SC-код} синтаксически минимален. Для описания объектов и связей между ними используется минимальное количество знаков. В то же время многообразие этих связей сводится к многобразию знаковых конструкций. Все это обеспечивается за счет представления информации в виде графовых структур.}
                \scnfileitem{SC-код не просто удобен для описания и проектирования каких-то сложных объектов --- с его помощью можно проектировать и реализовывать любые \textit{языки представления знаний}, в том числе программ, компьютерные системы и, вообще, описывать реальный мир.}
                \scnfileitem{Онтологический и компонентный подходы к проектированию любых сложных объектов обеспечивают выполнение главных принципов, по которым должны проектироваться современные системы. То, что реализовано и можно использовать, нужно переиспользовать везде.}
            \end{scnrelfromset}
            \begin{scnrelfromset}{смотрите}
                \scnitem{\scncite{Kasyanov2003}}
                \scnitem{\scncite{Petrov1978}}
            \end{scnrelfromset}
        \end{scnindent}
    \end{scnindent}

    \scnheader{Предметная область и онтология информационных конструкций и языков}
    \begin{scnrelfromlist}{дочерняя предметная область и онтология}
        \scnitem{Предметная область и онтология языков}
        \begin{scnindent}
            \begin{scnrelfromlist}{дочерняя предметная область и онтология}
                \scnitem{Предметная область и онтология естественных языков}
                \scnitem{Предметная область и онтология формальных языков}
            \end{scnrelfromlist}
        \end{scnindent}
    \end{scnrelfromlist}

    \scnheader{Предметная область и онтология формальных языков}
    \begin{scnrelfromlist}{дочерняя предметная область и онтология}
        \scnitem{Предметная область и онтология языков представления знаний}
        \begin{scnindent}
            \begin{scnrelfromlist}{дочерняя предметная область и онтология}
                \scnitem{\scnkeyword{Предметная область и онтология методов}}
            \end{scnrelfromlist}
        \end{scnindent}
    \end{scnrelfromlist}

    \scnheader{Предметная область и онтология методов}
    \begin{scnrelfromlist}{дочерняя предметная область и онтология}
        \scnitem{Предметная область и онтология методов ostis-систем}
        \begin{scnindent}
            \begin{scnrelfromlist}{дочерняя предметная область и онтология}
                \scnitem{Предметная область и онтология процедурных методов ostis-систем}
            \end{scnrelfromlist}
        \end{scnindent}
    \end{scnrelfromlist}

	\scnheader{язык программирования}
    \scntext{примечание}{Каждая теория должна быть согласована понятийно. Несмотря на то, что в литературе сложилась разное трактование понятия \textit{языка программирования}, должно быть одно универсальное. Для этого вместо языков программирования далее \uline{будем говорить о языках представления методов}, а вместо программ этих языков программирования --- о методах как знаковых конструкциях языков представления методов. Такое решение обосновывается тем, что обычно язык выступает в роли инструмента какого-то знания определенного вида, а термин \textit{языка программирования} является вырожденным, поскольку стоит говорить не о языках, на которых что-то можно программировать, а о языках, на которых можно представлять знания определенного вида, в данном случае --- знания процедурного типа. Сами термины \scnqqi{языка программирования} и \scnqqi{программы} будем считать неосновными идентификаторами понятий \scnqqi{языка представления методов} и \scnqqi{метода}, соответственно. Также это правило применяется на все понятия, используемые в данной главе и содержащие термин \scnqqi{метод}.}
    
    \scnheader{Семантическая теория программ в ostis-системах}
    \scntext{примечание}{Следует отметить, что общая \textit{Семантическая теория программ в ostis-системах} не отрицает весь накопленный опыт в сфере разработки современных \textit{технологий программирования}. Наоборот, предлагаемая в данной главе идея позволяет переиспользовать те проверенные инструменты и методы для наиболее быстрой и качественной реализации программ в сложных \textit{программных компьютерных системах}.}
\end{scnsubstruct}

\scntext{заключение}{Данная предметная область является началом \textit{Семантической теории программ для компьютерных систем нового поколения}. Логичным развитием данной предметной области будут:
    \begin{itemize}
        \item уточнение и дополнение понятий \textit{Предметной области и онтологии методов} для достижения полноты теории;
        \item описание дочерних предметных областей \textit{Предметной области и онтологии методов} для конкретных видов методов, а также уточнение денотационной и операционной семантики спецификации этих методов;
        \item описание возможных путей реализации метаметодов интерпретации методов различных я.п.м.;
        \item формализация математических моделей для подсчета оценок эффективности методов.
    \end{itemize}}

\end{scnsubstruct}
\end{SCn}


\scsubsubsection[
    \protect\scneditor{Сердюков Р.Е.}
    \protect\scnmonographychapter{Глава 3.2. Семантическая теория программ в интеллектуальных компьютерных системах нового поколения}
    ]{Предметная область и онтология интерпретации современных языков программирования в ostis-системах}
\label{sd_program_interpreting}
\begin{SCn}
\scnsectionheader{Предметная область и онтология интерпретации современных языков программирования в ostis-системах}
\begin{scnsubstruct}

\scnheader{Предметная область интерпретации современных языков программирования в ostis-система}
\scniselement{предметная область}
\begin{scnrelfromset}{автор}
    \scnitem{Зотов Н.В.}
    \scnitem{Шункевич Д.В.}
\end{scnrelfromset}

\begin{scnreltovector}{конкатенация сегментов}
    \scnitem{Синтаксис и семантика программ в ostis-системах}
    \scnitem{Методы и средства поддержки проектирования и разработки программ в ostis-системах}
    \scnitem{Комплекс свойств, определяющих эффективность программ в ostis-системах}
\end{scnreltovector}

\scnsegmentheader{Синтаксис и семантика программ в ostis-системах}
\begin{scnsubstruct}
    \begin{scnreltovector}{конкатенация сегментов}
        \scnitem{Синтаксис программ в ostis-системах}
        \scnitem{Денотационная семантика программ в ostis-системах}
        \scnitem{Операционная семантика программ в ostis-системах}
        \scnitem{Синтаксис и семантика языков программирования в ostis-системах}
    \end{scnreltovector}
    \begin{scnhaselementrolelist}{класс объектов исследования}
        \scnitem{спецификация метода*}
        \scnitem{синтаксис метода*}
        \scnitem{денотационная семантика метода*}
        \scnitem{операционная семантика метода*}
        \scnitem{спецификация языка представления методов*}
        \scnitem{синтаксис языка представления методов*}
        \scnitem{денотационная семантика языка представления методов*}
        \scnitem{операционная семантика языка представления методов*}
    \end{scnhaselementrolelist}
    \begin{scnhaselementrolelist}{ключевой знак}
        \scnitem{Принципы описания синтаксиса и семантики программ в ostis-системах}
        \scnitem{Язык SCP}
    \end{scnhaselementrolelist}
    
    \scnheader{семантика метода $\cup$ синтаксис метода}
    \scntext{примечание}{\textit{синтаксис} и \textit{семантика метода} составляют \textit{спецификацию*} этого \textit{метода}. \textit{Семантику метода} можно рассматривать в двух аспектах: как множество знаний, связанных между собой (то есть \textit{денотационную семантику} данного \textit{метода}), и как знание, которое может быть интерпретировано другим методом (то есть \textit{операционную семантику} данного \textit{метода}).}
\end{scnsubstruct}

\scnsegmentheader{Синтаксис программ в ostis-системах}
\begin{scnsubstruct}
	
	\scnheader{метод}
    \scntext{примечание}{Любой \textit{метод} состоит из \textit{информационных конструкций}, которые задают порядок действий в базе знаний, с помощью которых нужно перейти от исходного состояния к \uline{целевому}, решив таким образом какую-то конкретную задачу. Так, например, в процедурном методе любой такой оператор представляет собой некоторую математическую функцию. Для композиции этих функций в более крупные фрагменты используются выражения и операторы. В свою очередь, линейные последовательности операторов и условные ветвления также могут быть представлены функциями, составленными из функций отдельных компонентов этих конструкций. Цикл легко описывается рекурсивной функцией, составленной из компонентов, входящих в его тело.}

	\scnheader{Синтаксис языков представления методов}
    \scntext{примечание}{\textit{Синтаксис языков представления методов} в ostis-системах может быть формально описан различными способами. Так, например, можно использовать метаязык Бэкуса-Наура для описания синтаксиса любых \textit{языков представления методов}. Другими не менее известными формами представления методов являются контекстно-свободные грамматики, расширенная форма Бэкуса-Наура, синтаксические графы.}
    \begin{scnindent}
    	\begin{scnrelfromset}{смотрите}
    		\scnitem{\scncite{Scott1972}}
    		\scnitem{\scncite{Scott2006}}
    		\scnitem{\scncite{Sebesta2012}}
    	\end{scnrelfromset}
        \scntext{примечание}{Однако значительно более логично и целесообразно описывать \textit{синтаксис} других языков на универсальном \textit{языке представления знаний} --- \textit{SC-коде}. Такой подход позволит ostis-системам самостоятельно понимать, анализировать и генерировать тексты указанных языков на основе принципов, общих для любых форм внешнего представления информации, в том числе нелинейных. Таким образом, языки, написанные на \textit{SC-коде}, имеют такой же синтаксис как и сам \textit{SC-код}.}
            \begin{scnindent}
            \begin{scnrelfromset}{смотрите}
                \scnitem{\scncite{Petrov1978}}
            \end{scnrelfromset}
        \end{scnindent}
    \end{scnindent}

\end{scnsubstruct}

\scnsegmentheader{Денотационная семантика программ в ostis-системах}
\begin{scnsubstruct}
	
	\scnheader{семантика метода}
    \scntext{примечание}{\textit{семантика метода} разъясняет смысл \textit{синтаксических конструкций метода}. Наиболее распространенными методами описания семантики \textit{языков программирования} являются: денотационной, операционный, аксиоматический, алгебраический. На базе принципов Технологии OSTIS, под семантикой метода будем подразумевать объединение \textit{денотационной} и \textit{операционной семантики метода}.}
    \begin{scnindent}
    	\begin{scnrelfromset}{смотрите}
    		\scnitem{\scncite{Orlov2021}}
    	\end{scnrelfromset}
    \end{scnindent}
    \scntext{примечание}{С помощью \textit{SC-кода} можно представлять и те языки, которые не написаны на нем. Проблема будет в том, что форма и смысл языка и его методов будут разделены, то есть будут представлены по-разному. В данном случае \textit{SC-код} выступает мощным инструментом для интеграции спецификаций различных языков внешнего представления знаний. Однако стоит отметить, что в представлении различных форм методов, принадлежащих разным \textit{языкам представления методов}, в рамках \textit{Технологии OSTIS} нет необходимости. Это объясняется тем, что:
        \begin{itemize}
            \item \textit{SC-код} является достаточно универсальным языком для представления любых видов знаний. Это означает, что различные формы алгоритма решения одной и той же задачи можно свести к минимуму. В \textit{SC-коде} фундаментом является формальная теория, что обеспечивает универсальное представление различных видов декларативных и процедурных знаний. Так, \textit{логические программы} можно представлять в виде \textit{процедурных программ}, в которых в качестве операндов операторов будут не только \textit{логические формулы} и \textit{правила вывода}, но и другие методы, обеспечивающее интерпретацию этих \textit{логических формул} при помощи правил вывода. Таким образом, \textit{SC-код} можно называть не только языком унифицированного представления знания, но и языком, на котором можно решать различные классы задачи одним и тем же способом.
            \item Различные виды знаний в \textit{ostis-системах}, проектируемые по принципам \textit{Технологии OSTIS}, глубоко интегрированы между собой. Это дает не только простоту для создания этих систем на базе имеющихся языков, которые могут быть описаны на \textit{SC-коде}, но большие возможности для создания базовых \textit{языков программирования} для \textit{программных компьютерных систем нового поколения} таких, как, например, \textit{базового языка представления процедурных методов SCP}, \textit{базового языка представления продукционных методов} и других. Современные \textit{языки представления методов} создаются с целью упрощения описания какого-то алгоритма для быстрого и качественного решения определенного класса задач. В свою очередь, предлагаемые методики и модели позволяют проектировать \textit{языки представления методов} для \textit{компьютерных систем нового поколения} с помощью базовых \textit{языков представления знаний} таким образом, чтобы сама форма представления знаний не менялась. Методы разных \textit{языков представления методов} должны иметь одну универсальную форму представления, то есть один и тот же синтаксис, но могут давать возможности описывать и представлять разными способами \textit{денотационную} и \textit{операционную семантику} своих \textit{методов} с помощью одного и того же синтаксиса.
            \item Проектирование новых \textit{языков представления методов} должно сводится к их полному описанию на минимальном семействе \textit{языков SC-кода}: \textit{SCP}, \textit{SCL} и других. Речь идет о том, что чтобы спроектировать новый \textit{язык представления методов} достаточно разработать (неатомарный) метаметод на языках \textit{SCP} и \textit{SCL}, который будет интерпретировать методы проектируемых языков, а также описать \textit{денотационную семантику} этих методов. \textit{Метаметод интерпретации методов языков представления методов} можно называть интерпретатором этих языков, то есть некоторой абстрактной sc-машиной, на которой возможно выполнение методов определенного \textit{языка представления} этих \textit{методов}.
        \end{itemize}}
\end{scnsubstruct}

\scnsegmentheader{Операционная семантика программ в ostis-системах}
\begin{scnsubstruct}
	
	\scnheader{полная спецификация метода*}
    \scntext{примечание}{Полная \textit{спецификация метода*} кроме \textit{денотационной семантики этого метода*} должна включать \textit{операционную семантику этого метода*}, то есть формальное описание интерпретатора заданного метода. \textit{Операционная семантика языка представления методов} описывает выполнение \textit{метода}, составленного на данном языке, средствами виртуального компьютера. Виртуальный компьютер определяется как абстрактный автомат. Внутренние состояния этого автомата моделируют состояния вычислительного процесса при выполнении метода. Автомат транслирует исходный текст метода в набор формально определенных операций. Этот набор задает переходы автомата из исходного состояния в последовательность промежуточных состояний, изменяя значения переменных метода. Автомат завершает свою работу, переходя в некоторое конечное состояние. Таким образом, здесь идет речь о достаточно прямой абстракции возможного использования языка представления методов. \textit{операционная семантика языка} описывает смысл метода путем выполнения его операторов на простой машине-автомате. Изменения, происходящие в состоянии машины при выполнении данного оператора, определяют смысл этого оператора.}
    
    \scnheader{операционная семантика метода}
    \scntext{примечание}{\textit{Операционная семантика} конкретного \textit{метода} сводится к описанию \textit{метаметода}, который его интерпретирует, верифицирует и так далее.}
    
    \scnheader{метаметод}
    \scnsubset{метод}
    \scnidtf{метод, значениями параметров которого являются другие методы}

    \scnheader{операционная семантика метода}
    \scnhaselement{метаметод интерпретации*}
    \scnhaselement{метаметод верификации и оценки качества*}

    \scnheader{метаметод интерпретации*}
    \scntext{определение}{Отношение \textit{метаметод интерпретации*} представляет собой \textit{класс sc-связок} между \textit{sc-связкой}, обозначающей множество \textit{методов}, и sc-узлом, обозначающим \textit{метод}, который способен произвести интерпретацию заданного множества \textit{методов}.}
     
    \scnheader{метаметод верификации и оценки качества*}
    \scntext{определение}{Отношение \textit{метаметод верификации и оценки качества*} представляет собой класс sc-связок между \textit{sc-связкой}, обозначающей множество \textit{методов}, и sc-узлом, обозначающим метод, который способен произвести верификацию и оценку качества заданного множества \textit{методов}.}
    
    \scnheader{метаметод интерпретации методов базовых языков представления методов}
    \begin{scnrelfromlist}{класс подметодов}
        \scnitem{метаметод интерпретации методов Языка SCP}
        \scnitem{метаметод интерпретации методов Языка SCL}
        \scnitem{метаметод интерпретации методов языка представления продукционных методов}
        \scnitem{метаметод интерпретации методов языка представления функциональных методов}
        \scnitem{метаметод интерпретации методов языка представления нейросетей}
        \scnitem{метаметод интерпретации методов языка представления генетических алгоритмов}
    \end{scnrelfromlist}
    \scntext{примечание}{В рамках \textit{Технологии OSTIS} таких метаметодов может быть большое разнообразие. Каждый из них может состоять из множества атомарных и неатомарных подметодов. Это могут быть как метаметоды, интерпретирующие методы определенных \textit{языков представления методов}, так и метаметоды, верифицирующие и анализирующие качество этих методов. В том числе метаметоды могут производить операции над другими метаметодами.}

    \scnheader{метаметод верификации и оценки качества методов базовых языков представления методов}
    \begin{scnrelfromlist}{класс подметодов}
        \scnitem{метаметод верификации и оценки качества методов Языка SCP}
        \scnitem{метаметод верификации и оценки качества методов Языка SCL}
        \scnitem{метаметод верификации и оценки качества методов языка представления продукционных методов}
        \scnitem{метаметод верификации и оценки качества методов языка представления функциональных методов}
        \scnitem{метаметод верификации и оценки качества методов языка представления нейросетей}
        \scnitem{метаметод верификации и оценки качества методов языка представления генетических алгоритмов}
    \end{scnrelfromlist}
    \scntext{примечание}{Так, например, при реализации методов в оstis-системах метаметодами будут являться \textit{интепретатор Языка SCP}, а также интепретаторы, реализованные непосредственно на \textit{Языке SCP}.}
    \scntext{примечание}{Понятие \textit{синтакиса}, \textit{денотационной} и операционной \textit{семантики языков представления методов} сводятся к понятию синтаксиса, денотационной и операционной семантики вообще любого языка.}
\end{scnsubstruct}

\scnsegmentheader{Синтаксис и семантика языков программирования в ostis-системах}
\begin{scnsubstruct}
	\scnheader{язык представления методов}
    \scntext{примечание}{Для использования \textit{языка представления методов} следует описать каждую конструкцию языка в отдельности, а также ее применение в совокупности с другими конструкциями. В языке существует множество различных конструкций, точное определение которых необходимо как программисту, применяющему язык, так и разработчику компилятора для этого языка. Программисту эти знания позволяют прогнозировать вычисления, производимые операторами метода. Разработчику описания конструкций необходимы для создания правильной реализации компилятора.}
    \scntext{примечание}{Описание формальной модели \textit{языка представления методов} можно задать его \textit{спецификацией}. \textit{спецификация языка представления методов*} содержит описание \textit{синтаксиса}, \textit{денотационной}, \textit{операционной} \textit{семантики языка представления методов}.}

    \scnheader{спецификация языка представления методов*}
    \scnsuperset{отношение, заданное на множестве (язык представления методов)*}
    \begin{scnrelfromset}{разбиение}
        \scnitem{синтаксис языка представления методов*}
        \begin{scnindent}
            \scnsubset{синтаксис языка*}
            \scnidtf{теория правильно построенных информационных конструкций, принадлежащих заданному языку представления методов}
        \end{scnindent}
        \scnitem{денотационная семантика языка представления методов*}
        \begin{scnindent}
            \scnsubset{денотационная семантика языка*}
            \scnidtf{обобщенная формулировка классов задач, решаемых с помощью данного языка представления методов*}
        \end{scnindent}
        \scnitem{операционная семантика языка представления методов*}
        \begin{scnindent}
            \scnsubset{операционная семантика языка*}
            \scnidtf{перечень обобщенных агентов, обеспечивающих интерпретацию методов заданного языка представления методов*}
            \scnidtf{семейство методов интерпретации текстов данного языка представления методов*}
            \scnidtf{формальное описание интерпретатора заданного языка представления методов*}
        \end{scnindent}
    \end{scnrelfromset}
\end{scnsubstruct}

\scnsegmentheader{Методы и средства поддержки проектирования и разработки программ в ostis-системах}
\begin{scnsubstruct}

    \scntext{примечание}{Текущее состояние в области проектирования и разработки программного обеспечения говорит о том, что разработчики больше стремятся автоматизировать разработку методов на конкретных языках представления методов, чем обеспечить себя инструментальными обучающими средствами их проектирования, в том числе проектирования новых \textit{языков представления методов}. Это приводит к проблемам.}
    \begin{scnindent}
        \begin{scnrelfromset}{проблемы текущего состояния}
            \scnfileitem{В то время, как количество разработчиков, понимающих код какой-то сложной программной системы, уменьшается, требования к этой системе растут все быстрее и быстрее. Зачастую, разработчики сложных программных систем сами не в состоянии объяснить логику работы этих систем. По этой причине необходимо создавать инструментальные средства, которые будут позволять автоматизировать документирование программных систем.}
            \scnfileitem{Для обучения новых разработчиков навыкам работы с программными системы и их разработки необходимо привлекать ресурсы экспертов, понимающих принципы работы этих программных систем. Проблема решается разработкой справочной системы, которая будет позволять не только обучать пользователя тому, как проектировать методы решения задачи и программные системы на основе этих методов, но и указывать на пробелы в смежных дисциплинах, необходимых для достижения качественных результатов всей своей деятельности.}
            \scnfileitem{В инженерии часто разработчики проектируют и разрабатывают решения, которые уже когда-то были созданы другими специалистами. Таким образом, получаются функционально эквивалентные методы решения задач, а то, и вовсе, программные системы, решающие схожие проблемы. Ключом к решению данной проблемы является проектирование семантически мощной \textit{библиотеки многократно используемых методов решения различных задач}.}
        \end{scnrelfromset}
		\begin{scnrelfromset}{смотрите}
			\scnitem{\scncite{Lu2022}}
		\end{scnrelfromset}
	\end{scnindent}

	\scnheader{Cемантическая теория программ}
    \scntext{примечание}{Одной \textit{Cемантической теории программ} недостаточно. Кроме нее, для перманетного и беспрепятственного проектирования и разработки \textit{методов} различного класса необходимо разрабатывать:
    \begin{itemize}
        \item интеллектуальную систему поддержки проектирования и разработки методов, которая будет не только помогать разработчику верифицировать разрабатываемый метод, но и подсказывать способы его разработки;
        \item семантически мощную библиотеку многократно используемых компонентов для быстрого поиска существующих методов решения задач и их применения для решения других более комплексных задач.
    \end{itemize}}
	\begin{scnindent}
		\begin{scnrelfromset}{смотрите}
			\scnitem{\scncite{Gulykina2012}}
			\scnitem{\scncite{Pivovarchik2016}}
			\scnitem{\scncite{Ford2019}}
		\end{scnrelfromset}
	\end{scnindent}

	\scnheader{интеллектуальная система поддержки проектирования и разработки методов}
    \scntext{примечание}{Потенциальная система должна быть частью общего инструментального средства разработки интеллектуальны компьютерных систем нового поколения --- \textit{Метасистемы OSTIS} --- и может состоять из следующих компонентов:
        \begin{itemize}
            \item интеллектуальной help-системы по семантической теории программ;
            \item интеллектуальной help-системы по библиотеке многократно используемых методов решения задач,
            \item интеллектуальной help-системы по комплексу инструментальных средств проектирования  методов решения задач,
            \item интеллектуальной help-системы по методике обучения проектированию различных методов решения задач.
        \end{itemize}}
	\begin{scnindent}
		\begin{scnrelfromset}{смотрите}
			\scnitem{\scncite{IMS}}
		\end{scnrelfromset}
	\end{scnindent}
    \scntext{примечание}{Каждый компонент должен содержать:
        \begin{itemize}
            \item справочную подсистему,
            \item подсистему мониторинга и анализа деятельности разработчика методов решения задач,
            \item подсистему управления обучением.
        \end{itemize}}
    \scntext{примечание}{Каждая из подсистем взаимодействует с другими подсистемами, а также может функционировать автономно.}
    \scntext{примечание}{Справочная подсистема является консультантом-экспертом в области \textit{Семантической теории программ}, который может ответить на любой вопрос новичка или опытного пользователя. Каждая из таких систем может становиться индивидуальным помощников в обучении новых специалистов --- персональным ostis-ассистентом.}

\end{scnsubstruct}

\scnsegmentheader{Комплекс свойств, определяющих эффективность программ в ostis-системах}
\begin{scnsubstruct}

	\scnheader{язык представления методов}
    \scntext{примечание}{\textit{язык представления методов} можно определить множеством показателей, характеризующих отдельные его свойства. Возникает задача введения меры для оценки степени приспособленности языка представления методов к выполнению возложенных на него функций --- \textit{критериев эффективности}. Критерии эффективности методов приводятся на основе частных показателей эффективности этих методов (показателей качества). Способ связи между частными показателями определяет вид критерия эффективности.}
	\begin{scnindent}
		\begin{scnrelfromset}{смотрите}
			\scnitem{\scncite{Orlov2021}}
		\end{scnrelfromset}
	\end{scnindent}
	
    \scnheader{эффективность метода}
    \begin{scnrelfromlist}{свойство-предпосылка}
        \scnitem{легкость чтения и понимания метода}
        \scnitem{легкость представления метода}
        \scnitem{стоимость метода}
        \scnitem{общий объем задач, решаемых при помощи данного класса методов}
        \scnitem{многообразие видов задач, решаемых при помощи данного класса методов}
        \scnitem{надежность метода}
    \end{scnrelfromlist}

    \scnheader{легкость чтения метода}
    \scntext{примечание}{\textbf{\textit{легкость чтения метода}} должна способствовать легкому выделению основных понятий каждой части метода без обращения к его спецификации.}

    \scnheader{легкость чтения и понимания метода}
    \begin{scnrelfromlist}{свойство-предпосылка}
        \scnitem{простота синтаксиса языка представления методов}
        \scnitem{ортогональность информационных конструкций языка представления методов}
        \scnitem{структурированность потока управления в методе}
    \end{scnrelfromlist}

	\scnheader{язык представления методов}
    \scntext{примечание}{\textit{язык представления методов} должен предоставить \textit{простой} набор \textit{информационных конструкций}, которые могут быть использованы в качестве базисных элементов при создании методов.
        \\Сильное воздействие на простоту оказывает \textit{синтаксис языка}: он должен прозрачно отражать семантику конструкций, исключать двусмысленность и неоднозначность толкования.}

	\scnheader{ортогональность информационных конструкций языка представления методов}
    \scntext{примечание}{\textbf{\textit{ортогональность информационных конструкций языка представления методов}} означает, что любые возможные комбинации различных \textit{информационных конструкций} будут осмысленными, без неожиданного поведения, возникающего в результате взаимодействия конструкций или контекста использования.}

	\scnheader{поток управления}
    \scntext{примечание}{Порядок передач управления между операторами метода, то есть \textit{поток управления}, должен быть удобен для чтения и понимания человеком.}

	\scnheader{легкость создания метода}
    \scntext{примечание}{\textbf{\textit{легкость создания метода}} отражает удобство языка для представления этого метода в конкретной предметной области.}

    \scnheader{легкость представления метода}
    \begin{scnrelfromlist}{свойство-предпосылка}
        \scnitem{простота синтаксиса языка представления методов}
        \scnitem{естественность синтаксиса языка представления методов}
        \scnitem{ортогональность информационных конструкций языка представления методов}
        \scnitem{полнота и точность спецификации языка представления методов}
        \scnitem{согласованность и целостность спецификации языка представления методов}
    \end{scnrelfromlist}

	\scnheader{синтаксис метода}
    \scntext{примечание}{\textit{синтаксис метода} должен способствовать легкому и прозрачному отображению в нем алгоритмических структур предметной области. \textit{синтаксис языка представления методов} должен быть не только \textit{простым}, но и \textit{естественным}, и поддерживать \textit{ортогональность} информационных конструкций языка.}

	\scnheader{легкость представления метода}
    \scntext{примечание}{\textbf{\textit{легкость представления нового метода}} обеспечивается \textit{полной и точной, согласованной и целостной спецификацией} соответствующего языка. То есть необходимо достаточное количество \textit{информационных конструкций} в этом языке для того чтобы представить конкретный \textit{метод}. При этом \textit{спецификация языка} должна быть согласованной и целостной чтобы представлять на ней непротиворечивые \textit{методы}.}

    \scnheader{общая стоимость метода}
    \begin{scnrelfromlist}{свойство-предпосылка}
        \scnitem{стоимость применения метода}
        \scnitem{стоимость интерпретации метода}
        \scnitem{стоимость создания, тестирования и использования метода}
        \scnitem{стоимость сопровождения метода}
        \scnitem{согласованность и целостность спецификации языка представления методов}
    \end{scnrelfromlist}
	\scntext{примечание}{Все эти критерии можно применить и касательно самих \textit{языков представления методов}.}
	
	\scnheader{стоимость применения метода}
    \scntext{примечание}{\textbf{\textit{стоимость применения метода}} во многом зависит от структуры \textit{языка представления методов}. Язык, требующий многочисленных проверок синтаксических типов во время применения метода, будет препятствовать быстрой работе программы.}

	\scnheader{размер стоимости интерпретации метода}
    \scntext{примечание}{\textbf{\textit{размер стоимости интерпретации метода}} зависит от возможностей используемого метаметода интерпретации. Чем совершеннее методы оптимизации, тем дороже стоит интерпретация.}
    
    \scnheader{общая стоимость метода}
    \scntext{примечание}{Размер стоимости создания, тестирования и использования метода зависит от используемого метаметода верификации и оценки качества этого метода.}
    \scntext{примечание}{Многочисленные исследования показывают, что значительную часть стоимости используемого метода составляет не стоимость его разработки, а \textit{стоимость его сопровождения}. Связывая сопровождение методов с другими их характеристиками, следует выделить, прежде всего, зависимость от читабельности, поскольку сопровождение обычно происходит следующим поколением разработчиков.}
    \begin{scnindent}
        \begin{scnrelfromset}{смотрите}
            \scnitem{\scncite{Brooks2021}}
        \end{scnrelfromset}
    \end{scnindent}
    
    \scnheader{язык представления методов}
    \scntext{примечание}{\textbf{\textit{общий объем задач и многообразие видов задач, решаемых при помощи данного класса методов}}, являются не менее важными характеристиками и показывают степень универсальности соответствующего языка представления методов. Чем больше задач можно решить на \textit{я.п.м.}, тем он универсальнее.}

	\scnheader{надежность методов языка представления методов}
    \scntext{примечание}{\textbf{\textit{надежность методов языка представления методов}} должна обеспечиваться минимумом ошибок при работе конкретного метода.}
  
\end{scnsubstruct}

\end{scnsubstruct}
\end{SCn}


\scsubsection[
    \protect\scneditors{Бутрин С.В.;Шункевич Д.В.;Зотов Н.В.;Орлов М.К.}
    \protect\scnmonographychapter{Глава 3.4. Язык запросов для интеллектуальных компьютерных систем нового поколения}
    ]{Предметная область и онтология sc-языка вопросов}
\label{sd_sc_quest_lang}
\begin{SCn}

\scnsectionheader{Предметная область и онтология Языка вопросов для ostis-систем}
\begin{scnsubstruct}

	\scnheader{Предметная область Языка вопросов для ostis-систем}
	\scniselement{предметная область}
    \begin{scnrelfromset}{автор}
        \scnitem{Самодумкин С.А.}
        \scnitem{Зотов Н.В.}
        \scnitem{Шункевич Д.В.}
        \scnitem{Ивашенко В.П}
    \end{scnrelfromset}
    \begin{scnrelfromlist}{дочерняя предметная область}
        \scnitem{Предметная область синтаксиса Языка вопросов для ostis-систем}
        \scnitem{Предметная область денотационной семантики Языка вопросов для ostis-систем}
        \scnitem{Предметная область операционной семантики Языка вопросов для ostis-систем}
    \end{scnrelfromlist}
    
    \scntext{аннотация}{Возможности \textit{баз знаний} \textit{ostis-систем} позволяют не только представлять и структурировать знания об окружающем мире, но и быстро получать и формировать эти знания о нем, тем самым удовлетворяя информационную потребность пользователя. В данной предметной области уточнена формальная спецификация \textit{Языка вопросов для ostis-систем}, позволяющая описывать и интерпретировать любые классы \textit{вопросов} \textit{пользователей ostis-систем}.}

    \begin{scnrelfromlist}{ключевой знак}
        \scnitem{Язык вопросов для ostis-систем}
    \end{scnrelfromlist}
    
    \begin{scnhaselementrolelist}{класс объектов исследования}
        \scnitem{вопрос}
        \scnitem{ответ на вопрос}
        \scnitem{знак в рамках заданного вопроса}
        \scnitem{основной знак в рамках заданного вопроса}
        \scnitem{неосновной знак в рамках заданного вопроса}
        \scnitem{отношение в рамках заданного вопроса}
        \scnitem{базовое отношение в рамках заданного вопроса}
        \scnitem{интерпретатор Языка вопросов для ostis-систем}
    \end{scnhaselementrolelist}

    \begin{scnrelfromlist}{библиографическая ссылка}
        \scnitem{\scncite{Averyanov1993}}
        \scnitem{\scncite{Suleimanov2001}}
        \scnitem{\scncite{Suleimanov2014}}
        \scnitem{\scncite{Bukharev1990}}
        \scnitem{\scncite{Kwok2001}}
        \scnitem{\scncite{Emelyanov2007}}
        \scnitem{\scncite{Finn1976}}
        \scnitem{\scncite{Finn1981}}
        \scnitem{\scncite{Belnap1981}}
        \scnitem{\scncite{Sosnin2007}}
        \scnitem{\scncite{Zaharov2002}}
        \scnitem{\scncite{Hant1978}}
        \scnitem{\scncite{Lyubarsky1990}}
        \scnitem{\scncite{Samodumkin2009}}
        \scnitem{\scncite{Samodumkin2009a}}
    \end{scnrelfromlist}

    \begin{scnrelfromvector}{введение}
    	\scnfileitem{Одна из ключевых особенностей \textit{интеллектуальной системы} состоит в том, что \textit{пользователь} имеет возможность формулировать свою информационную потребность. Cпособом выражения такой потребности является \textit{вопрос}. В процессе общения всегда существует контекст, который определяет дополнительную информацию, способствующую правильному пониманию \textit{смысла} сообщения. Особенность представления информации в \textit{базах знаний} \textit{ostis-систем} упрощает формирование информационной потребности пользователя, так как представленная информация в \textit{базах знаний} уже структурирована и известны отношения, заданные на определенном понятии, в отношении которого разрешается вопросно-проблемная ситуация.}
    	\scnfileitem{Показано, что вопросно-проблемная ситуация не может быть решена в рамках формальной логики и природа вопроса может быть понятна в системе субъектно-объектных отношений. В связи с тем, что при формировании \textit{баз знаний} \textit{ostis-систем} происходит формирование субъектно-объектных отношений в рамках заданной \textit{предметной области}, тем самым упрощается выражение информационной потребности пользователем средствами \textit{SC-кода}.}
    	\begin{scnindent}
    		\begin{scnrelfromset}{источник}
    			\scnitem{\scncite{Averyanov1993}}
    		\end{scnrelfromset}
    	\end{scnindent}
        \scnfileitem{Целью разработки \textbf{\textit{Языка вопросов для ostis-систем}} и последующего его развития является реализация возможности понимания действий, осуществляемых \textit{ostis-системой}, при формировании ответа на поставленный \textit{вопрос}. В процессе формирования ответа на поставленный \textit{вопрос} возможны следующие варианты:
        \begin{itemize}
            \item \textit{ответ на} поставленный \textit{вопрос} существует в \textit{базе знаний} и происходит локализация \textit{фрагмента базы знаний} в контексте выраженной средствами \textit{SC-кода} информационной потребности \textit{пользователя};
            \item ответ связан с разрешением некоторой задачной ситуации, которая содержится в контексте \textit{вопроса} и формирование \textit{ответа на вопрос} возлагается на \textit{решатель задач}.
        \end{itemize}}
	    \begin{scnindent}
	    	\begin{scnrelfromset}{смотрите}
	    		\scnitem{Агентно-ориентированные модели гибридных решателей задач ostis-систем}
	    	\end{scnrelfromset}
	    \end{scnindent}
	\end{scnrelfromvector}

    \scnheader{Язык вопросов для ostis-систем}
    \scnidtf{Предлагаемый вариант языка для описания вопросов и ответов на них в ostis-системах}
    \scnidtf{sc-язык вопросов}
    \scniselement{sc-язык}
    \scnrelfrom{синтаксис языка}{Синтаксис Языка вопросов для ostis-систем}
    \begin{scnindent}
        \scnsubset{Синтаксис SC-кода}
    \end{scnindent}
    \scnrelfrom{денотационная семантика языка}{Денотационная семантика Языка вопросов для ostis-систем}
    \begin{scnindent}
        \scnidtf{Онтология классов знаков и отношений для описания формулировок вопросов на SC-коде}
        \scnsuperset{Семантическая классификация вопросов}
    \end{scnindent}
    \scnrelfrom{операционная семантика языка}{Операционная семантика Языка вопросов для ostis-систем}
    \begin{scnindent}
        \scnidtf{Коллектив sc-агентов вывода ответов на заданные вопросы пользователя ostis-системы}
    \end{scnindent}

\end{scnsubstruct}
\scnendcurrentsectioncomment
\end{SCn}


\scsubsubsection[
    \protect\scneditor{Бутрин С.В.}
    \protect\scnmonographychapter{Глава 3.4. Язык запросов для интеллектуальных компьютерных систем нового поколения}
    ]{Предметная область и онтология синтаксиса sc-языка вопросов}
\label{sd_syntax_sc_quest_lang}
\begin{SCn}

\scnsectionheader{Предметная область и онтология синтаксиса Языка вопросов для ostis-систем}
\begin{scnsubstruct}

    \scnheader{Предметная область синтаксиса Языка вопросов для ostis-систем}
	\scniselement{предметная область}
    \begin{scnhaselementrole}{максимальный класс объектов исследования}
        {синтаксис Языка вопросов для ostis-систем}
    \end{scnhaselementrole}
    
    \scnheader{синтаксис Языка вопросов для ostis-систем}
    \scntext{примечание}{\textit{Язык вопросов для ostis-систем} относится к семейству семантических совместимых языков --- \textit{sc-языков}, и предназначен для формального описания поискового предписания \textit{ostis-систем} с целью удовлетворения информационной потребности \textit{пользователя}. Поэтому \textbf{\textit{Синтаксис Языка вопросов для ostis-систем}}, как и \textit{синтаксис} любого другого \textit{sc-языка}, является \textit{Синтаксисом SC-кода}. Такой подход позволяет:
        \begin{itemize}
            \item унифицировать форму представления \textit{вопросов} и \textit{знаний}, с помощью которых строятся ответы на поставленные \textit{вопросы};
            \item использовать минимум средств для интерпретации заданных \textit{вопросов пользователей};
            \item сводить формирование ответов на большую часть заданных \textit{вопросов} к поиску информации в текущем состоянии \textit{базы знаний ostis-системы}.
        \end{itemize}}

\end{scnsubstruct}
\scnendcurrentsectioncomment 
\end{SCn}


\scsubsubsection[
    \protect\scneditor{Бутрин С.В.}
    \protect\scnmonographychapter{Глава 3.4. Язык запросов для интеллектуальных компьютерных систем нового поколения}
    ]{Предметная область и онтология денотационной семантики sc-языка вопросов}
\label{sd_denot_sem_sc_quest_lang}
\begin{SCn}

\scnsectionheader{Предметная область и онтология денотационной семантики Языка вопросов для ostis-систем}
\begin{scnsubstruct}

    \scnheader{Предметная область денотационной семантики Языка вопросов для ostis-систем}
	\scniselement{предметная область}
    \begin{scnhaselementrolelist}{класс объектов исследования}
        \scnitem{вопрос}
        \scnitem{ответ на вопрос}
        \scnitem{знак в рамках заданного вопроса}
        \scnitem{основной знак в рамках заданного вопроса}
        \scnitem{неосновной знак в рамках заданного вопроса}
        \scnitem{отношение в рамках заданного вопроса}
        \scnitem{базовое отношение в рамках заданного вопроса}
    \end{scnhaselementrolelist}
    \scnhaselementrole{ключевой знак}{Денотационная семантика Языка вопросов для ostis-систем}

    \scnheader{Денотационная семантика Языка вопросов для ostis-систем}
    \scntext{примечание}{\textbf{\textit{Денотационная семантика Языка вопросов для ostis-систем}} включает \textit{классы вопросов} и соответствующие \textit{классы ответов}, необходимые для спецификации формулировок \textit{вопросов} и \textit{ответов} на них, а также \textit{классы знаков} и \textit{отношений}, входящих в структуру любого \textit{вопроса}. В \textit{Семантической классификации вопросов} \textit{Языка вопросов для ostis-систем} заложена идея, описанная в работе \cite{Suleimanov2001}.}
    \begin{scnindent}
    	\begin{scnrelfromset}{источник}
    		\scnitem{\scncite{Suleimanov2001}}
    	\end{scnrelfromset}
    \end{scnindent}
    \scntext{примечание}{Любой \textbf{\textit{вопрос}} на \textit{Языке вопросов для ostis-систем} представляет собой \textit{спецификацию действия} на поиск или синтез \textit{знания}, удовлетворяющего информационную потребность \textit{пользователя}, инициирующего этот \textit{вопрос}. То есть \textit{вопрос} --- это ничто иное как \textit{задача}, с помощью которой выражается потребность пользователя в некоторой информации, возможно хранимой или выводимой в \textit{базе знаний} \textit{ostis-системы}.}
    \begin{scnindent}
    	\begin{scnrelfromset}{смотрите}
    		\scnitem{Формализация понятий действия, задачи, метода, средства, навыка и технологии}
    	\end{scnrelfromset}
    \end{scnindent}
    \scntext{примечание}{Каждому \textit{вопросу} можно взаимно однозначно сопоставить некоторое множество \textit{ответов на} этот \textit{вопрос}. Каждый \textit{ответ на вопрос} представляет собой некоторую \textit{sс-структуру} \textit{семантической окрестности основного знака}, раскрываемого в этом \textit{ответе на} заданный \textit{вопрос}.}

    \scnheader{вопрос}
    \scnidtf{запрос}
    \scnidtf{непроцедурная формулировка задачи на поиск (в текущем состоянии базы знаний) или на синтез знания, удовлетворяющего заданным требованиям}
    \scnidtf{запрос метода (способа) решения заданного (указываемого) \textit{класса задач} либо \textit{плана решения} конкретной указываемой \textit{задачи}}
    \scnidtf{задача, направленная на удовлетворение информационной потребности некоторого субъекта-заказчика}
    \scnsubset{задача}

    \scnheader{ответ на вопрос}
    \scnidtf{ответ на запрос}
    \scnidtf{результат запроса}
    \scnidtf{результат решения задачи на поиск или синтез знания, удовлетворяющий заданным требованиям}
    \scnidtf{семантическая окрестность \textit{основного знака}, знание в которой удовлетворяет информационную потребность пользователя}
    \scnidtf{знание в базе знаний ostis-системы, которое удовлетворяет информационную потребность пользователя}
    \scnsubset{знание}
    
    \scnheader{знак в рамках заданного вопроса}
    \scntext{примечание}{Среди всех классов \textit{знаков в рамках заданного вопроса} \textit{Языка вопросов для ostis-систем} можно выделить наиболее общие по иерархии классы \textit{знаков}.}
    \scnsubset{знак}
    \begin{scnrelfromset}{разбиение}
        \scnitem{основной знак в рамках заданного вопроса}
        \begin{scnindent}
            \scnidtf{ключевой sc-элемент в рамках заданного вопроса}
            \scnidtf{\textit{знак}, относительно которого задан вопрос}
        \end{scnindent}
        \scnitem{неосновной знак в рамках заданного вопроса}
        \begin{scnindent}
            \scnidtf{\textit{знак}, стоящий в некотором отношении с \textit{основным знаком в рамках заданного вопроса}}
        \end{scnindent}
    \end{scnrelfromset}
    \scntext{определение}{\textbf{\textit{знаком в рамках заданного вопроса}} является любой \textit{знак понятия} или \textit{сущности}, принадлежащий этому \textit{вопросу}.}
    \scntext{пояснение}{Между \textit{знаками в рамках заданного вопроса} задается множество связей \textit{отношений}, входящих в состав различных \textit{предметных областей}.}
    
    \scnheader{отношение в рамках заданного вопроса}
    \scnidtf{определенное отношение между знаками \textit{предметной области} в контексте \textit{вопроса}}
    \scnsubset{отношение}
    \scntext{определение}{\textbf{\textit{отношение в рамках заданного вопроса}} представляет собой \textit{отношение} между \textit{знаками} \textit{предметной области}, принадлежащих заданному \textit{вопросу}.}
    \scntext{пояснение}{Среди всех классов \textit{отношений в рамках заданного вопроса} можно выделить класс \textbf{\textit{базовых отношений в рамках заданного вопроса}} и класс \textbf{\textit{составных отношений в рамках заданного вопроса}}.}
    
    \scnheader{базовое отношение в рамках заданного вопроса}
    \scnidtf{\textit{класс отношений}, объединяющий \textit{отношения в заданном вопросе}, отражающие однотипный \textit{смысл} и раскрывающие определенный признак \textit{знаков} \textit{предметной области}}
    \scnsubset{отношение в рамках заданного вопроса}
    \begin{scnrelfromset}{декомпозиция}
        \scnitem{отношение состояния}
        \scnitem{отношение действия}
        \scnitem{отношение состава}
        \scnitem{теоретико-множественное отношение}
        \scnitem{темпоральное отношение}
        \scnitem{пространственное отношение}
        \scnitem{количественное отношение}
        \scnitem{качественное отношение}
    \end{scnrelfromset}
    \scntext{пример}{Например, \textit{отношения в рамках заданного вопроса} такие, как \scnqqi{играет*}, \scnqqi{спит*}, \scnqqi{плавает*}, объединяются в класс \textit{отношений состояния} по признаку выражать состояние знака (то есть данные отношения раскрывают признак \textit{знака} \textit{предметной области} --- \scnqqi{находиться в некотором состоянии}).}
   
    \scnheader{составное отношение в рамках заданного вопроса}
    \scnidtf{устойчивая комбинация двух \textit{отношений действия}: действия, направленного на \textit{параметр вопроса\scnrolesign}, и действия, направленного на \textit{ответ на вопрос*}}
    \scntext{пример}{Например, элемент \textit{составного отношения в рамках заданного вопроса} между \textit{знаками}: \scnqqi{\textit{Нефтеперерабатывающий завод}}, \scnqqi{\textit{нефть}} и \scnqqi{\textit{нефтепродукты}} --- может быть представлен как \scnqqi{Нефтеперерабатывающий завод перерабатывает нефть в нефтепродукты}.}
    
    \scnheader{вопрос}
    \scntext{примечание}{Смысловая классификация \textit{вопросов} дает возможность противопоставить каждому типу вопроса ограниченный набор допустимых, то есть \textit{семантически корректных информационных конструкций}, передающий правильный \textit{смысл} \textit{вопроса} в зависимости от класса \textit{вопроса}. При этом \textbf{\textit{Семантическая классификация вопросов}} позволяет разбить множество \textit{вопросов} на классы, в каждом из которых требуется раскрытие некоторого однотипного \textit{смысла}, заданного классом этого \textit{вопроса}.}
    \begin{scnrelfromset}{декомпозиция}
        \scnitem{вопрос, требующий вывода семантической окрестности \textit{основного знака}}
        \begin{scnindent}
            \begin{scnhaselementrolelist}{пример}
                \scnitem{Вопрос. Что такое \textit{Город Минск}}
            \end{scnhaselementrolelist}
        \end{scnindent}
        \scnitem{вопрос, требующий раскрытия в ответе \textit{базового отношения} \textit{основного знака}}
        \begin{scnindent}
            \begin{scnhaselementrolelist}{пример}
                \scnitem{Вопрос. Что легче: железо или дерево}
            \end{scnhaselementrolelist}
        \end{scnindent}
        \scnitem{вопрос, требующий раскрытия в ответе \textit{составного отношения} \textit{основного знака}}
        \begin{scnindent}
            \scntext{пояснение}{Такому классу \textit{вопросов} соответствуют классы \textit{ответов}, в которых \textit{главный знак} раскрывается через \textit{составное отношение}.}
            \begin{scnhaselementrolelist}{пример}
                \scnitem{Вопрос. Какие Принципы компонентного проектирования в интеллектуальных компьютерных системах нового поколения}
            \end{scnhaselementrolelist}
        \end{scnindent}
        \scnitem{вопрос, требующий раскрытия в ответе произвольной комбинации \textit{базового отношения} и/или \textit{составного отношения} \textit{основного знака}}
        \begin{scnindent}
            \begin{scnhaselementrolelist}{пример}
                \scnitem{Вопрос. Как определяется уровень интеллекта кибернетической системы}
            \end{scnhaselementrolelist}
        \end{scnindent}
        \scnitem{вопрос, требующий раскрытия в ответе более чем одного \textit{основного знака}}
        \begin{scnindent}
            \begin{scnhaselementrolelist}{пример}
                \scnitem{Вопрос. Докажите теорему Пифагора}
            \end{scnhaselementrolelist}
        \end{scnindent}
    \end{scnrelfromset}

    \scnheader{вопрос, требующий раскрытия в ответе \textit{базового отношения} \textit{основного знака}}
    \begin{scnrelfromset}{декомпозиция}
        \scnitem{вопрос, требующий раскрытия в ответе \textit{отношения состава} \textit{основного знака}}
        \begin{scnindent}
            \scnidtf{класс вопросов, в ответах на которые \textit{основной знак} \textit{S} раскрывается через его \textit{отношение состава} в связке с его составляющими знаками \textit{P} и \textit{Q}}
            \begin{scnhaselementrolelist}{пример}
                \scnitem{Вопрос. Какие административные районы входят в состав Города Витебск}
                \begin{scnindent}
                    \scneq{\scnfileimage[35em]{Contents/part_ps/src/images/sd_sc_quest_lang/question_about_vitebsk_regions.png}}
                    \scnrelfrom{ответ на вопрос}{\{Железнодорожный район Города Витебск, Октябрьский район Города Витебск, Первомайский район Города Витебск\}}
                    \begin{scnindent}
                        \scneq{\scnfileimage[35em]{Contents/part_ps/src/images/sd_sc_quest_lang/question_about_vitebsk_regions_answer.png}}
                    \end{scnindent}
                \end{scnindent}
            \end{scnhaselementrolelist}
        \end{scnindent}
        \scnitem{вопрос, требующий раскрытия в ответе \textit{теоретико-множественного отношения} \textit{основного знака}}
        \begin{scnindent}
            \scnidtf{класс вопросов, в ответах на которые \textit{основной знак} \textit{S} раскрывается через его \textit{теоретико-множественное отношение} в связке с другим знаком \textit{P}, содержащего \textit{S} как часть}
            \begin{scnhaselementrolelist}{пример}
                \scnitem{Вопрос. Частью какой области является Смолевичский район}
                \begin{scnindent}
                    \scneq{\scnfileimage[35em]{Contents/part_ps/src/images/sd_sc_quest_lang/question_about_smolevichi_inclusion.png}}
                    \scnrelfrom{ответ на вопрос}{\{Смолевичский район является частью Минской области\}}
                \end{scnindent}
            \end{scnhaselementrolelist}
        \end{scnindent}
        \scnitem{вопрос, требующий раскрытия в ответе \textit{отношения состояния} \textit{основного знака}}
        \begin{scnindent}
            \scnidtf{класс вопросов, в ответах на которые \textit{основной знак} \textit{S} раскрывается через его \textit{отношение состояния}}
            \begin{scnhaselementrolelist}{пример}
                \scnitem{Вопрос. Какие города современной территории Республики Беларусь имели Магдебургское право}
                \begin{scnindent}
                    \scneq{\scnfileimage[35em]{Contents/part_ps/src/images/sd_sc_quest_lang/question_about_minsk_district_town_with_mag_act.png}}
                    \scnrelfrom{ответ на вопрос}{\{Волковыск, Гродно, Мозырь и другие имели Магдебургское право\}}
                \end{scnindent}
            \end{scnhaselementrolelist}
        \end{scnindent}
        \scnitem{вопрос, требующий раскрытия в ответе \textit{отношения действия} \textit{основного знака}}
        \begin{scnindent}
            \scnidtf{класс вопросов, в ответах на которые \textit{основной знак} \textit{S} раскрывается через его \textit{отношение действия} в связке с другим знаком \textit{P}}
        \end{scnindent}
        \scnitem{вопрос, требующий раскрытия в ответе \textit{темпорального отношения} \textit{основного знака}}
        \begin{scnindent}
            \scnidtf{класс вопросов, в ответах на которые \textit{основной знак} \textit{S} раскрывается через его \textit{темпоральное отношение} в связке с другим знаком \textit{P} по некоторой временной шкале}
            \begin{scnhaselementrolelist}{пример}
                \scnitem{Вопрос. Какое событие произошло раньше: Первый раздел Речи Посполитой или Бородинское сражение}
                \begin{scnindent}
                    \scneq{\scnfileimage[35em]{Contents/part_ps/src/images/sd_sc_quest_lang/question_about_events.png}}
                    \scnrelfrom{ответ на вопрос}{\{Первый раздел Речи Посполитой был раньше Бородинского сражения\}}
                    \begin{scnindent}
                        \scneq{\scnfileimage[35em]{Contents/part_ps/src/images/sd_sc_quest_lang/question_about_event_answer.png}}
                    \end{scnindent}
                \end{scnindent}
            \end{scnhaselementrolelist}
        \end{scnindent}
        \scnitem{вопрос, требующий раскрытия в ответе \textit{пространственного отношения} \textit{основного знака}}
        \begin{scnindent}
            \scnidtf{класс вопросов, в ответах на которые \textit{основной знак} \textit{S} раскрывается через \textit{пространственное отношение}, отражающее его положение в пространстве относительно другого знака \textit{P}}
        \end{scnindent}
        \scnitem{вопрос, требующий раскрытия в ответе \textit{количественного отношения} \textit{основного знака}}
        \begin{scnindent}
            \scnidtf{класс вопросов, в ответах на которые раскрывается \textit{количественное отношение} \textit{основного знака}}
            \begin{scnhaselementrolelist}{пример}
                \scnitem{Вопрос. Какова высота Горы Дзержинская}
                \begin{scnindent}
                    \scneq{\scnfileimage[35em]{Contents/part_ps/src/images/sd_sc_quest_lang/question_about_mountain_length.png}}
                    \scnrelfrom{ответ на вопрос}{\{Высота Горы Дзержинская --- 345 м\}}
                \end{scnindent}
            \end{scnhaselementrolelist}
        \end{scnindent}
        \scnitem{вопрос, требующий раскрытия в ответе \textit{качественного отношения} \textit{основного знака}}
        \begin{scnindent}
            \scnidtf{класс вопросов, в ответах на которые раскрывается \textit{качественное отношение} \textit{основного знака} \textit{S} в связке с другим знаком \textit{P}}
            \begin{scnhaselementrolelist}{пример}
                \scnitem{Вопрос. Территория какой административной области больше: Минской или Брестской}
                \begin{scnindent}
                    \scneq{\scnfileimage[35em]{Contents/part_ps/src/images/sd_sc_quest_lang/question_about_district_squares.png}}
                    \scnrelfrom{ответ на вопрос}{\{Территория Минской области больше Брестской\}}
                    \begin{scnindent}
                        \scneq{\scnfileimage[35em]{Contents/part_ps/src/images/sd_sc_quest_lang/question_about_district_squares_answer.png}}
                    \end{scnindent}
                \end{scnindent}
            \end{scnhaselementrolelist}
        \end{scnindent}
    \end{scnrelfromset}

    \scnheader{вопрос, требующий раскрытия в ответе произвольной комбинации \textit{базового отношения} и/или \textit{составного отношения} \textit{основного знака}}
    \begin{scnrelfromset}{декомпозиция}
        \scnitem{вопрос, требующий раскрытия в ответе произвольной комбинации \textit{составного отношения описания} \textit{основного знака}}
        \begin{scnindent}
            \scnidtf{класс вопросов, в ответах на которые раскрываются произвольные комбинации \textit{базового отношения} и/или \textit{составного отношения} \textit{основного знака} \textit{S} в связке с другими знаками}
            \begin{scnhaselementrolelist}{пример}
                \scnitem{\{S состоит из P, Q, W. S переводит X и Y и выполняется раньше Z\}}
                \begin{scnindent}
                    \scnrelto{ответ на вопрос}{Вопрос. Что такое S}
                \end{scnindent}
            \end{scnhaselementrolelist}
        \end{scnindent}
        \scnitem{вопрос, требующий раскрытия в ответе произвольной комбинации \textit{составного отношения определения} \textit{основного знака}}
        \begin{scnindent}
            \scnidtf{класс ответов, в которых \textit{основной знак} \textit{S} раскрывается через \textit{первостепенное понятие} и его \textit{описание}}
            \begin{scnhaselementrolelist}{пример}
                \scnitem{\{Минск --- это столица, которая находится в РБ\}}
                \begin{scnindent}
                    \scnrelto{ответ на вопросы}{Вопрос. Как определяется город Минск}
                \end{scnindent}
            \end{scnhaselementrolelist}
        \end{scnindent}
        \scnitem{вопрос, требующий раскрытия в ответе произвольной комбинации \textit{составного отношения причины} \textit{основного знака}}
        \begin{scnindent}
            \scnidtf{класс вопросов, в ответах на которые раскрывается условие существования некоторых отношений \textit{основного знака} \textit{S} в связке с другими знаками}
            \begin{scnhaselementrolelist}{пример}
                \scnitem{Вопрос. Почему время в пути от города Минска до города Борисова меньше чем время в пути от города Минска до города Орша}
                \begin{scnindent}
                    \scnrelfrom{ответ на вопрос}{\{Время в пути от города Минска до города Борисова меньше чем время в пути от города Минска до города Орша, потому что расстояние от города Минска меньше до города Борисова, чем до города Орша\}}
                \end{scnindent}
            \end{scnhaselementrolelist}
        \end{scnindent}
        \scnitem{вопрос, требующий раскрытия в ответе произвольной комбинации \textit{составного отношения следствия} \textit{основного знака}}
        \scnidtf{класс вопросов, в ответах на которые раскрывается следствие от существования некоторых отношений \textit{основного знака} \textit{S} в связке с другими знаками}
        \begin{scnindent}
            \begin{scnhaselementrolelist}{пример}
                \scnitem{Вопрос. Что следует из того, что расстояние от города Минска до города Борисова меньше расстояния от города Минска до города Орша}
                \begin{scnindent}
                    \scnrelfrom{ответ на вопрос}{\{Расстояние от города Минска до города Борисова меньше расстояния от города Минска до города Орша, поэтому от города Минска до города Борисова время в пути меньше чем до города Орша\}}
                \end{scnindent}
            \end{scnhaselementrolelist}
        \end{scnindent}
    \end{scnrelfromset}

    \scnheader{вопрос, требующий раскрытия в ответе более чем одного \textit{основного знака}}
    \scnsuperset{вопрос, требующий раскрытия в ответе \textit{отношение детализации} знаков, стоящих в некоторых отношениях с \textit{основным знаком}}
    \begin{scnindent}
        \scnidtf{класс вопросов, в ответах на которые происходит детализация знаков, стоящих в некоторых отношениях с \textit{основным знаком} \textit{S}}
        \begin{scnhaselementrolelist}{пример}
            \scnitem{Вопрос. Какая связь водной сети существует между городом Минск и городом Светлогорск}
            \begin{scnindent}
                \scnrelfrom{ответ на вопрос}{\{Город Минск расположен на реке Свислочь, которая впадает в реку Березина, протекающую через город Светлогорск\}}
            \end{scnindent}
        \end{scnhaselementrolelist}
    \end{scnindent}
  
    \scnheader{вопрос}
    \scntext{примечание}{Таким образом, для каждого \textit{вопроса} \textit{пользователя ostis-системы} можно найти класс \textit{вопросов}, на котором можно реализовывать \textit{вывод ответов} на этот \textit{вопрос}. Описанная \textit{Семантическая классификация вопросов} позволяет:
    \begin{itemize}
        \item автоматически структурировать \textit{вопросы} \textit{пользователей} по описанию этих \textit{вопросов};
        \item а также формировать \textit{ответы на} эти \textit{вопросы} с учетом \textit{непроцедурных формулировок} этих \textit{вопросов}.
    \end{itemize}}

\end{scnsubstruct}

\scnendcurrentsectioncomment
    
\end{SCn}


\scsubsubsection[
    \protect\scneditor{Бутрин С.В.}
    \protect\scnmonographychapter{Глава 3.4. Язык запросов для интеллектуальных компьютерных систем нового поколения}
    ]{Предметная область и онтология операционной семантики sc-языка вопросов}
\label{sd_operat_sem_sc_quest_lang}
\begin{SCn}

\scnsectionheader{Предметная область и онтология операционной семантики Языка вопросов для ostis-систем}
\begin{scnsubstruct}

    \scnheader{Предметная область операционной семантики Языка вопросов для ostis-систем}
	\scniselement{предметная область}
    \begin{scnhaselementrolelist}{класс объектов исследования}
        \scnitem{вопрос}
        \scnitem{ответ на вопрос}
        \scnitem{знак в рамках заданного вопроса}
        \scnitem{основной знак в рамках заданного вопроса}
        \scnitem{неосновной знак в рамках заданного вопроса}
        \scnitem{отношение в рамках заданного вопроса}
        \scnitem{базовое отношение в рамках заданного вопроса}
    \end{scnhaselementrolelist}
   
    \scnheader{вопрос}
    \scntext{примечание}{Каждому классу \textit{вопросов} должен соответствовать определенный \textit{коллектив sc-агентов}, реализующих поиск или синтез из \textit{базы знаний} \textit{ostis-системы} соответствующих ответов на поставленные \textit{вопросы}. Следует отметить, что в зависимости от степени наполненности \textit{базы знаний} \textit{ответы} могут содержаться в \textit{базе знаний} либо отсутствовать в текущей версии \textit{базы знаний}. В случае наличия в \textit{базе знаний} \textit{ответа на} поставленный \textit{вопрос} информационная потребность пользователя реализуется \textit{информационно-поисковыми sc-агентами}, в противном случае --- в зависимости от \textit{классов вопросов} формирование ответов осуществляется специализированными \textit{sc-агентами}, которые в процессе работы дополнительно выполняют вычислительные задачи либо осуществляют синтез на основе \textit{логического вывода} или других \textit{моделей решения задач}.} 
    \begin{scnindent}
    	\begin{scnrelfromset}{смотрите}
    		\scnitem{Смысловое представление логических формул и высказываний в различного вида логиках}
    	\end{scnrelfromset}
    \end{scnindent}
    
    \scnheader{интерпретатор Языка вопросов для ostis-систем}
    \scniselement{неатомарный sc-агент}
    \begin{scnrelfromset}{декомпозиция абстрактного sc-агента}
        \scnitem{Абстрактный sc-агент поиска ответа на заданный вопрос}
        \begin{scnindent}
            \begin{scnrelfromset}{декомпозиция абстрактного sc-агента}
                \scnitem{Абстрактный sc-агент поиска семантической окрестности \textit{основного знака}}
                \scnitem{Абстрактный sc-агент поиска ответа на вопрос, требующий раскрытия в ответе \textit{отношения состава} для \textit{основного знака}}
                \scnitem{Абстрактный sc-агент поиска ответа на вопрос, требующий раскрытия в ответе \textit{теоретико-множественного отношения} для \textit{основного знака}}
                \scnitem{Абстрактный sc-агент поиска ответа на вопрос, требующий раскрытия в ответе \textit{отношения состояния} для \textit{основного знака}}	
                \scnitem{Абстрактный sc-агент поиска ответа на вопрос, требующий раскрытия в ответе \textit{отношения действия} для \textit{основного знака}}	
                \scnitem{Абстрактный sc-агент поиска ответа на вопрос, требующий раскрытия в ответе \textit{темпорального отношения} для \textit{основного знака}}
                \scnitem{Абстрактный sc-агент поиска ответа на вопрос, требующий раскрытия в ответе \textit{пространственного отношения} для \textit{основного знака}}
                \scnitem{Абстрактный sc-агент поиска ответа на вопрос, требующий раскрытия в ответе \textit{количественного отношения} для \textit{основного знака}}
                \scnitem{Абстрактный sc-агент поиска ответа на вопрос, требующий раскрытия в ответе \textit{качественного отношения} для \textit{основного знака}}
                \scnitem{Абстрактный sc-агент поиска ответа на вопрос, требующий раскрытия в ответе \textit{отношения описания} для \textit{основного знака}}
                \scnitem{Абстрактный sc-агент поиска ответа на вопрос, требующий раскрытия в ответе \textit{отношения определения} для \textit{основного знака}}
                \scnitem{Абстрактный sc-агент поиска ответа на вопрос, требующий раскрытия в ответе \textit{отношения причины} для \textit{основного знака}}
                \scnitem{Абстрактный sc-агент поиска ответа на вопрос, требующий раскрытия в ответе \textit{отношения следствия} для \textit{основного знака}}
                \scnitem{Абстрактный sc-агент поиска ответа на вопрос, требующий раскрытия в ответе \textit{отношения детализации} для \textit{основного знака}}
            \end{scnrelfromset}
        \end{scnindent}
        \scnitem{Абстрактный sc-агент синтеза ответа на заданный вопрос}
    \end{scnrelfromset}
    \scntext{примечание}{Все \textit{sc-агенты}, выводящие \textit{ответы на} поставленные \textit{вопросы}, формируют \textit{коллектив sc-агентов} --- \textbf{\textit{интерпретатор Языка вопросов для ostis-систем}}, с помощью которого можно интерпретировать любые классы \textit{вопросов}. \textit{интерпретатор Языка вопросов для ostis-систем} может быть реализован по-разному: в виде \textit{коллектива scp-агентов} или \textit{платформенно-зависимых sc-агентов}.}

\end{scnsubstruct}

\scntext{заключение}{Перечислим основные положения:
\begin{itemize}
    \item информационная потребность \textit{пользователей ostis-системы} может быть выражена в виде \textit{вопросов}, а удовлетворение этой информационной потребности --- в виде \textit{ответов на} заданные \textit{вопросы};
    \item вывод \textit{ответов на} заданные \textit{вопросы} \textit{пользователем ostis-системы} может быть осуществлен путем поиска \textit{знаний} в текущем состоянии \textit{базы знаний} этой \textit{ostis-системы}, либо синтеза новых знаний, отсутствующих в \textit{базе знаний} этой \textit{ostis-системы};
    \item каждый \textit{вопрос} может быть представлен в виде некоторой \textit{спецификации задачи}, инициированной \textit{пользователем ostis-системы} для удовлетворения своей информационной потребности, а \textit{ответ на} этот \textit{вопрос} --- в виде \textit{семантической окрестности} \textit{основного знака в рамках заданного вопроса};
    \item каждому \textit{вопросу} может быть сопоставлен соответствующий класс \textit{вопросов} в \textit{Семантической классификации вопросов};
    \item для синтеза отсутствующих \textit{ответов на} поставленные \textit{вопросы} могут быть использованы различные \textit{модели решения задач}, в том числе \textit{логические модели решения задач};
    \item \textit{ответы на} поставленные \textit{вопросы} могут быть транслированы в \textit{естественно-языковой текст} и визуализированы при помощи соответствующих \textit{естественно-языковых интерфейсов} для удобства выдачи информации любому пользователю.
\end{itemize}}
\bigskip
\scnendcurrentsectioncomment
\end{SCn}


\scsubsection[
    \protect\scneditors{Василевская А.П.;Зотов Н.В.;Орлов М.К.}
    \protect\scnmonographychapter{Глава 3.5. Логические, продукционные и функциональные модели решения задач в интеллектуальных компьютерных системах нового поколения}
    ]{Предметная область и онтология операционной семантики логических sc-языков}
\label{sd_operat_sem_sc_logical_lang}

\scsubsection[
    \protect\scneditors{Зотов Н.В.;Орлов М.К.}
    \protect\scnmonographychapter{Глава 3.5. Логические, продукционные и функциональные модели решения задач в интеллектуальных компьютерных системах нового поколения}
    ]{Предметная область и онтология sc-языков продукционного программирования}
\label{sd_sc_product_program_lang}

\scsubsubsection[
    \protect\scneditors{Зотов Н.В.;Орлов М.К.}
    \protect\scnmonographychapter{Глава 3.5. Логические, продукционные и функциональные модели решения задач в интеллектуальных компьютерных системах нового поколения}
    ]{Предметная область и онтология синтаксиса sc-языков продукционного программирования}
\label{sd_sc_product_program_lang_syntax}

\scsubsubsection[
    \protect\scneditors{Зотов Н.В.;Орлов М.К.}
    \protect\scnmonographychapter{Глава 3.5. Логические, продукционные и функциональные модели решения задач в интеллектуальных компьютерных системах нового поколения}
    ]{Предметная область и онтология денотационной семантики sc-языков продукционного программирования}
\label{sd_sc_product_program_lang_denot_sem}

\scsubsubsection[
    \protect\scneditors{Зотов Н.В.;Орлов М.К.}
    \protect\scnmonographychapter{Глава 3.5. Логические, продукционные и функциональные модели решения задач в интеллектуальных компьютерных системах нового поколения}
    ]{Предметная область и онтология операционной семантики sc-языков продукционного программирования}
\label{sd_sc_product_program_lang_oper_sem}

\scsubsection[
    \protect\scneditors{Ковалев М.В.;Крощенко А.А.;Михно Е.В.}
    \protect\scnmonographychapter{Глава 3.6. Конвергенция и интеграция искусственных нейронных сетей с базами знаний в интеллектуальных компьютерных системах нового поколения}
    ]{Предметная область и онтология sc-моделей искусственных нейронных сетей}
\label{sd_ann}
\begin{SCn}
\scnsectionheader{Предметная область и онтология sc-моделей искусственных нейронных сетей}
\begin{scnsubstruct}

\begin{scnrelfromlist}{соавтор}
	\scnitem{Головко В. А.}
	\scnitem{Ковалёв М. В.}
	\scnitem{Крощенко А. А.}
	\scnitem{Михно Е. В.}
\end{scnrelfromlist}

\begin{scnreltovector}{конкатенация сегментов}
	\scnitem{Предметная область и онтология искусственных нейронных сетей}
	\scnitem{Предметная область и онтология действий по обработке искусственных нейронных сетей}
\end{scnreltovector}

\scntext{введение}{В последнее десятилетие обозначилась устойчивая тенденция широкого применения методов машинного обучения в самых разных областях человеческой деятельности, обусловленная в первую очередь развитием теории искусственных нейронных сетей (и. н. с.), а
также аппаратных возможностей.\\
Преимущество и.н.с. заключается в том, что они могут работать с неструктурированными данными.\\
Главный недостаток и.н.с. --- это отсутствие понятной человеку обратной связи, которую можно было бы назвать цепочкой рассуждений, т.е. можно сказать, что и.н.с. работают как \textit{черный ящик} (\scncite{gastelvecchi2016}).\\
Сложность современных интеллектуальных систем, использующих нейросетевые модели, а также большой объём обрабатываемых ими данных обуславливают необходимость мониторинга, объяснения и понимания механизмов их работы с целью вербализации оценки и оптимизации их деятельности.\\
В связи с этим становится актуальна разработка нейросимволических подходов (описанных, в частности, в работе (\scncite{nesy1}), в частности, подходов по интеграции и.н.с и баз знаний, использующих онтологии. Такие интегрированные системы способны сочетать:\\
1) возможность семантической интерпретации обрабатываемых данных, используя представление решаемых и.н.с. прикладных задач, а так же спецификацию её входных и выходных данных;\\
2) с представлением самой структуры и.н.с., описанием её свойств и состояний, позволяющими упростить понимание её работы (\scncite{ann_ostis2018}).\\
Можно выделить два основных направления интеграции и.н.с. с базами знаний:\\
1) построение интеллектуальных систем, способных использовать нейросетевые методы наравне с другими имеющимися в системе методами для решения задач или подзадач системы. Такие системы смогут учитывать семантику решаемых задач на более высоком уровне, что сделает решение этих задач более структурированными и прозрачными.\\
2) построение интеллектуальной среды по разработке, обучению и интеграции различных и.н.с., совместимых с базами знаний через представление и.н.с. с помощью онтологических структур и их интерпретацию средствами представления знаний. Такая среда предоставит возможность интроспекции и.н.с, возможность сохранения состояний и.н.с. после обучения и реконфигурации сети. Это позволит производить более глубокий анализ работы и.н.с. Так же формальное описание знаний рамках предметной области и.н.с. поможет поможет уменьшить порог вхождения разработчиков в методы решения задач с помощью и.н.с.\\
Данный раздел посвящен предметной области и онтологии искусственных нейронных сетей и предметной области и онтологии действий по обработке искусственных нейронных сетей, которые являются основой развития обоих указанных направлений.
}

\scnsegmentheader{Предметная область и онтология искусственных нейронных сетей}
\begin{scnsubstruct}

\scnheader{Предметная область искусственных нейронных сетей}
\scnidtf{Предметная область и.н.с.}
\scniselement{предметная область}

\begin{scnhaselementrolelist}{максимальный класс объектов исследования}
	\scnitem{искусственная нейронная сеть}
\end{scnhaselementrolelist}
\begin{scnhaselementrolelist}{класс объектов исследования}
    \scnitem{искусственная нейронная сеть}
    \scnitem{искусственная нейронная сеть с прямыми связями}
    \scnitem{персептрон}
    \scnitem{персептрон Розенблатта}
    \scnitem{автоэнкодерная искусственная нейронная сеть}
    \scnitem{машина опорных векторов}
    \scnitem{искусственная нейронная сеть радиально-базисных функций}
    \scnitem{искусственная нейронная сеть с обратными связями}
    \scnitem{нейронная сеть Хопфилда}
    \scnitem{нейронная сеть Хэмминга}
    \scnitem{рекуррентная искусственная нейронная сеть}
    \scnitem{искусственная нейронная сеть Джордана}
    \scnitem{искусственная нейронная сеть Элмана}
    \scnitem{мультирекуррентная нейронная сеть}
    \scnitem{LSTM-элемент}
    \scnitem{GRU-элемент}
    \scnitem{полносвязная искусственная нейронная сеть}
    \scnitem{слабосвязная искусственная нейронная сеть}
    \scnitem{формальный нейрон}
    \scnitem{полносвязный формальный нейрон}
    \scnitem{сверточный формальный нейрон}
    \scnitem{рекуррентный формальный нейрон}
    \scnitem{синаптическая связь}
    \scnitem{параметр нейронной сети}
    \scnitem{настраиваемый параметр нейронной сети}
    \scnitem{весовой коэффициент}
    \scnitem{пороговое значение}
    \scnitem{ядро свертки}
    \scnitem{архитектурный параметр нейронной сети}
    \scnitem{количество слоев}
    \scnitem{количество формальных нейронов}
    \scnitem{количество синаптических связей}
    \scnitem{паттерн входной активности н.с.}
    \scnitem{признак}
    \scnitem{слой и.н.с.}
    \scnitem{полносвязный слой и.н.с}
    \scnitem{сверточный слой и.н.с}
    \scnitem{слой и.н.с. нелинейного преобразования}
    \scnitem{dropout слой и.н.с.}
    \scnitem{pooling слой и.н.с}
\end{scnhaselementrolelist}
\begin{scnhaselementrolelist}{исследуемое отношение}
    \scnitem{формальный нейрон\scnrolesign}
    \scnitem{пороговый формальный нейрон\scnrolesign}
    \scnitem{синаптическая связь\scnrolesign}
    \scnitem{входное значение формального нейрона*}
    \scnitem{выходное значение формального нейрона*}
    \scnitem{функция активации*}
    \scnitem{взвешенная сумма*}
    \scnitem{распределяющий слой*}
    \scnitem{обрабатывающий слой*}
    \scnitem{выходной слой*}
\end{scnhaselementrolelist}

\begin{scnrelfromlist}{частная предметная область}
	\scnitem{Предметная область ИНС с заданным направлением связей}
	\begin{scnindent}
		\begin{scnrelfromlist}{частная предметная область}
			\scnitem{Предметная область ИНС с прямым связями}
			\begin{scnindent}
				\begin{scnrelfromlist}{частная предметная область}
					\scnitem{Предметная область персептронов}
					\begin{scnindent}
						\begin{scnrelfromlist}{частная предметная область}
							\scnitem{Предметная область персептронов Розенблатта}
							\scnitem{Предметная область персептронов Румельхарта}
							\scnitem{Предметная область автоэнкодерных ИНС}
						\end{scnrelfromlist}
					\end{scnindent}
					\scnitem{Предметная область ИНС радиально-базисных функций}
					\scnitem{Предметная область машин опорных векторов}
				\end{scnrelfromlist}
			\end{scnindent}
			\scnitem{Предметная область ИНС с обратными связями}
			\begin{scnindent}
				\scnidtf{Предметная область рекуррентных ИНС}
				\begin{scnrelfromlist}{частная предметная область}
					\scnitem{Предметная область ИНС Джордана}
					\scnitem{Предметная область ИНС Элмана}
					\scnitem{Предметная область LSTM-элементов}
					\scnitem{Предметная область GRU-элементов}
				\end{scnrelfromlist}
			\end{scnindent}
		\end{scnrelfromlist}
	\end{scnindent}
	\scnitem{Предметная область обучения ИНС}
	\begin{scnindent}
		\begin{scnrelfromlist}{частная предметная область}
			\scnitem{Предметная область ИНС, обучающихся с учителем}
			\scnitem{Предметная область ИНС, обучающихся без учителя}
			\begin{scnindent}
				\begin{scnrelfromlist}{частная предметная область}
					\scnitem{Предметная область обучающихся автоэнкодерных ИНС}
					\scnitem{Предметная область ИНС глубокого доверия}
					\scnitem{Предметная область генеративно-состязательных ИНС}
					\scnitem{Предметная область самоорганизующихся карт Кохонена}
					\scnitem{Предметная область ИНС Хопфилда}
					\scnitem{Предметная область подкрепляющего обучения ИНС}
				\end{scnrelfromlist}
			\end{scnindent}
		\end{scnrelfromlist}
	\end{scnindent}
	\scnitem{Предметная область топологий ИНC}
	\begin{scnindent}
		\begin{scnrelfromlist}{частная предметная область}
			\scnitem{Предметная область полносвязных ИНC}
			\scnitem{Предметная область многослойных ИНC}
			\scnitem{Предметная область слабосвязных ИНC}
		\end{scnrelfromlist}
	\end{scnindent}
	\scnitem{Предметная область задач, решаемых с помощью ИНС}
	\begin{scnindent}
		\begin{scnrelfromlist}{частная предметная область}
			\scnitem{Предметная область ИНС, решающих задачу классификации}
			\scnitem{Предметная область ИНС, решающих задачу аппроксимации}
			\scnitem{Предметная область ИНС, решающих задачу управления}
			\scnitem{Предметная область ИНС, решающих задачу фильтрации}
			\scnitem{Предметная область ИНС, решающих задачу детекции}
			\scnitem{Предметная область ИНС, решающих задачу с ассоциативной памятью}
		\end{scnrelfromlist}
	\end{scnindent}
	\scnitem{Предметная область интеграции ИНС с базой знаний}
\end{scnrelfromlist}

\scnheader{искусственная нейронная сеть}
    \scnidtf{и.н.с.}
    \scnidtf{множество искусственных нейронных сетей}
    \scnidtf{нейронная сеть}
    \scndefinition{\textbf{\textit{искусственная нейронная сеть}} --- это совокупность нейронных элементов и связей между ними (\scncite{Golovko2017}).\\
        Искусственная нейронная сеть состоит из \textbf{\textit{формальных нейронов}}, которые связаны между собой посредством \textbf{\textit{синаптических связей}}. Нейроны организованы в \textbf{\textit{слои}}. Каждый нейрон слоя принимает сигналы со входящих в него синаптических связей, обрабатывает их единым образом с помощью заданной ему или всему слою \textbf{\textit{функции активации}} и передает результат на выходящие из него синаптические связи.}
    \scntext{пояснение}{\textbf{\textit{искусственная нейронная сеть}} --- это биологически инспирированная математическая модель, обладающая обобщающей способностью после выполнения процедуры обучения. Под обобщающей способностью понимается способность модели выдавать корректные результаты для паттернов входной активности, не входящих в обучающую выборку.}
    \scnsubset{математическая модель}
    \begin{scnindent}
        \scntext{пояснение}{\textbf{\textit{математическая модель}} --- это упрощенное описание объекта реального мира, выраженное с помощью математической символики}
    \end{scnindent}
    \scnrelfrom{описание примера}{\scnfileimage[30em]{Contents/part_ps/images/sd_ps/sd_ann/neural_network_scg.png}}
    \vspace{5\baselineskip}
    \scnrelfrom{изображение}{\scnfileimage[30em]{Contents/part_ps/images/sd_ps/sd_ann/neural_network.png}}
    \scnrelfrom{разбиение}{\scnkeyword{Типология и.н.с. по признаку направленности связей\scnsupergroupsign}}
        \begin{scnindent}
            \begin{scneqtoset}
                \scnitem{искусственная нейронная сеть с прямыми связями}
                \begin{scnindent}
                    \begin{scnsubdividing}
                        \scnitem{персептрон}
                        \begin{scnindent}
                            \begin{scnsubdividing}
                                \scnitem{персептрон Розенблатта}
                                \scnitem{автоэнкодерная искусственная нейронная сеть}
                            \end{scnsubdividing}
                        \end{scnindent}
                        \scnitem{машина опорных векторов}
                        \scnitem{искусственная нейронная сеть радиально-базисных функций}
                    \end{scnsubdividing}
                \end{scnindent}
                \scnitem{искусственная нейронная сеть с обратными связями}
                \begin{scnindent}
                    \begin{scnsubdividing}
                        \scnitem{нейронная сеть Хопфилда}
                        \scnitem{нейронная сеть Хэмминга}
                    \end{scnsubdividing}
                \end{scnindent}
                \scnitem{рекуррентная искусственная нейронная сеть}
                \begin{scnindent}
                    \begin{scnsubdividing}
                        \scnitem{искусственная нейронная сеть Джордана}
                        \scnitem{искусственная нейронная сеть Элмана}
                        \scnitem{мультирекуррентная нейронная сеть}
                        \scnitem{LSTM-элемент}
                        \scnitem{GRU-элемент}
                    \end{scnsubdividing}
                \end{scnindent}
            \end{scneqtoset}
        \end{scnindent}
    \scnrelfrom{разбиение}{\scnkeyword{Типология и.н.с. по признаку полноты связей\scnsupergroupsign}}
        \begin{scnindent}
            \begin{scneqtoset}
                \scnitem{полносвязная искусственная нейронная сеть}
                \scnitem{слабосвязная искусственная нейронная сеть}
            \end{scneqtoset}
        \end{scnindent}
    \scnrelfrom{решаемые задачи}{задачи, которые могут быть решены с помощью и.н.с. с приемлемой точностью}
        \begin{scnindent}
            \begin{scneqtoset}
                \scnitem{задача классификации}
                \begin{scnindent}
                    \scnsubset{задача}
                    \scntext{пояснение}{Задача построения классификатора, т.е. отображения $\tilde c: X \rightarrow C$, где $ X \in \mathbb{R}^m$ ---
                    признаковое пространство п.в.а., $C = \scnleftcurlbrace~C_1, C_2, ...C_k\scnrightcurlbrace$ --- конечное и обычно небольшое множество меток классов.}
                \end{scnindent}
                \scnitem{задача регрессии}
                \begin{scnindent}
                    \scnsubset{задача}
                    \scntext{пояснение}{Задача построения оценочной функции по примерам $(x_i, f(x_i))$, где $f(x)$ --- неизвестная функция}
                    \scntext{пояснение}{\textbf{\textit{оценочная функция}} --- отображение вида $\tilde{f}: X \rightarrow \mathbb{R}$, где $X \in \mathbb{R}^m$ --- признаковое пространство п.в.а.}
                \end{scnindent}
                \scnitem{задача кластеризации}
                \begin{scnindent}
                    \scnsubset{задача}
                    \scntext{пояснение}{Задача разбиения множества п.в.а. на группы (кластеры) по какой-либо метрике сходства.}
                \end{scnindent}
                \scnitem{задача понижения размерности}
                \begin{scnindent}
                    \scnsubset{задача}
                    \scnidtf{задача уменьшения размерности признакового пространства}
                \end{scnindent}
                \scnitem{задача управления}
                \begin{scnindent}
                    \scnsubset{задача}
                \end{scnindent}
                \scnitem{задача фильтрации}
                \begin{scnindent}
                    \scnsubset{задача}
                \end{scnindent}
                \scnitem{задача детекции}
                \begin{scnindent}
                    \scnsubset{задача}
                    \scnsubset{задача классификации}
                    \scnsubset{задача регрессии}
                \end{scnindent}
                \scnitem{задача с ассоциативной памятью}
                \begin{scnindent}
                    \scnsubset{задача}
                \end{scnindent}
            \end{scneqtoset}
        \end{scnindent}

\scnheader{формальный нейрон}
    \scnidtf{искусственный нейрон}
    \scnidtf{нейрон}
    \scnidtf{ф.н.}
    \scnidtf{нейронный элемент}
    \scnidtf{множество нейронов искусственных нейронных сетей}
    \scnidtf{математическая модель реального биологического нейрона}
    \scntext{примечание}{Отдельный формальный нейрон является искусственной нейронной сети с одним нейроном в единственном слое.}
    \scnsubset{искусственная нейронная сеть}
    \scntext{пояснение}{\textbf{\textit{формальный нейрон}} --- это основной элемент \textit{искусственной нейронной сети}, применяющий свою \textit{функцию активации} (\scncite{Golovko2017}) к сумме произведений входных сигналов на весовые коэффициенты:
        $y = F\left(\sum\textunderscore\scnleftcurlbrace~i=1\scnrightcurlbrace\scnsupergroupsign\scnleftcurlbrace~n\scnrightcurlbrace w_ix_i - T\right) = F(WX - T)$
        где $X = (x_1,x_2,...,x_n)^\scnleftcurlbrace~T\scnrightcurlbrace$ --- вектор входного сигнала; $W - (w_1,w_2,...,w_n)$ --- вектор весовых коэффициентов; \textit{T} --- пороговое значение;
        \textit{F} --- функция активации.}
    \scnrelfrom{изображение}{\scnfileimage[20em]{Contents/part_ps/images/sd_ps/sd_ann/neuron.png}}
    \scntext{примечание}{Формальные нейроны могут иметь полный набор связей с нейронами предшествующего слоя или неполный (разряженный) набор связей.}
    \begin{scnsubdividing}
        \scnitem{полносвязный формальный нейрон}
        \begin{scnindent}
            \scnidtf{нейрон, у которого есть полный набор связей с нейронами предшествующего слоя}
            \scntext{пояснение}{отдельный обрабатывающий элемент и.н.с., выполняющий функциональное преобразование взвешенной суммы элементов вектора входных значений с помощью функции активации}
        \end{scnindent}
        \scnitem{сверточный формальный нейрон}
        \begin{scnindent}
            \scntext{пояснение}{Отдельный обрабатывающий элемент и.н.с., выполняющий функциональное преобразование результата операции свертки матрицы входных значений с помощью функции активации.}
            \scntext{примечание}{Сверточный формальный нейрон может быть представлен полносвязным формальным нейроном.}
            \scntext{примечание}{Сверточный формальный нейрон с соответствующим ему ядром свертки может быть представлен нейроном с неполным набором связей.}
        \end{scnindent}
        \scnitem{рекуррентный формальный нейрон}
        \begin{scnindent}
            \scntext{пояснение}{Формальный нейрон, имеющий обратную связь с самим собой или с другими нейронами и.н.с.}
        \end{scnindent}
    \end{scnsubdividing}

\scnheader{формальный нейрон\scnrolesign}
    \scnidtf{формальный нейронный элемент\scnrolesign}
    \scnidtf{нейронный элемент\scnrolesign}
    \scnidtf{нейрон\scnrolesign}
    \scniselement{ролевое отношение}
    \scnrelfrom{первый домен}{искусственная нейронная сеть}
    \scnrelfrom{второй домен}{формальный нейрон}
    \scnrelfrom{область определения}{искусственная нейронная сеть}
    \scntext{пояснение}{\textbf{\textit{формальный нейрон\scnrolesign}} --- ролевое отношение, связывающее искусственную нейронную сеть с ее нейроном.}

\scnheader{пороговый формальный нейрон\scnrolesign}
    \scnidtf{пороговый нейронный элемент\scnrolesign}
    \scnidtf{пороговый нейрон\scnrolesign}
    \scniselement{ролевое отношение}
    \scnrelfrom{первый домен}{искусственная нейронная сеть}
    \scnrelfrom{второй домен}{формальный нейрон}
    \scnrelfrom{область определения}{искусственная нейронная сеть}
    \scntext{пояснение}{\textbf{\textit{пороговый формальный нейрон\scnrolesign}} --- ролевое отношение, связывающее искусственную нейронную сеть с таким ее нейроном, выходное значение которого всегда равно -1.}
    \scntext{пояснение}{Весовой коэффициент синаптической связи, выходящей из такого нейрона, является порогом для нейрона, в который данная синаптическая связь входит.}

\scnheader{синаптическая связь}
    \scnidtf{синапс}
    \scnsubset{ориентированная пара}
    \scndefinition{\textbf{\textit{синаптическая связь}} --- ориентированная пара, первым компонентом которой является нейрон, из которого исходит сигнал, а вторым компонентом --- нейрон, который принимает этот сигнал.}

\scnheader{синаптическая связь\scnrolesign}
    \scnidtf{синапс\scnrolesign}
    \scniselement{ролевое отношение}
    \scnrelfrom{первый домен}{искусственная нейронная сеть}
    \scnrelfrom{второй домен}{синаптическая связь}
	\scnrelfrom{область определения}{\scnnonamednode}
	\begin{scnindent}
		\begin{scnreltoset}{объединение}
			\scnitem{искусственная нейронная сеть}
			\scnitem{синапс}
		\end{scnreltoset}
	\end{scnindent}
   \scndefinition{\textbf{\textit{синаптическая связь\scnrolesign}} --- ролевое отношение, связывающее искусственную нейронную сеть с ее синапсом.}

\scnheader{параметр и.н.с.}
    \scnsubset{параметр}
    \begin{scnsubdividing}
        \scnitem{настраиваемый параметр и.н.с.}
        \begin{scnindent}
            \scnidtf{параметр и.н.с., значение которого изменяется в ходе обучения}
            \begin{scnsubdividing}
                \scnitem{весовой коэффициент синаптической связи}
                \scnitem{пороговое значение}
                \scnitem{ядро свертки}
                \begin{scnindent}
                    \scnidtf{квадратная матрица произвольного порядка, элементы которой изменяются в процессе обучения и.н.с.}
                    \scntext{примечание}{Если сверточный формальный нейрон представить в виде полносвязного формального нейрона, соответствующее ядро свертки преобразуется в вектор весовых коэффициентов.}
                \end{scnindent}
            \end{scnsubdividing}
        \end{scnindent}
        \scnitem{архитектурный параметр и.н.с.}
        \begin{scnindent}
            \scntext{примечание}{Параметр и.н.с., определяющий ее архитектуру.}
            \begin{scnsubdividing}
                \scnitem{количество слоев}
                \scnitem{количество нейронов}
                \scnitem{количество синапсов}
            \end{scnsubdividing}
        \end{scnindent}
    \end{scnsubdividing}

\scnheader{весовой коэффициент синаптической связи}
    \scnidtf{вес синапса}
    \scnidtf{сила синаптической связи}
    \scnsubset{настраиваемый параметр}
    \scntext{пояснение}{\textbf{\textit{весовой коэффициент синаптической связи}} --- это числовой коэффициент, который ставится в соответствие каждому
        синапсу нейронной сети и изменяется в процессе обучения.}
    \scntext{примечание}{Если сила синаптической связи отрицательна, то она называется \textit{тормозящей}. В противном случае она
        является \textit{усиливающей}.}

\scnheader{входное значение формального нейрона*}
    \scnidtf{входное значение нейрона*}
    \scnidtf{входное значение*}
    \scniselement{неролевое отношение}
    \scniselement{бинарное отношение}
    \scnrelfrom{первый домен}{формальный нейрон}
    \scnrelfrom{второй домен}{число}
    \scnrelfrom{область определения}{\scnnonamednode}
    \begin{scnindent}
    	\begin{scnreltoset}{объединение}
    		\scnitem{формальный нейрон}
    		\scnitem{число}
    	\end{scnreltoset}
    \end{scnindent}
    \scntext{пояснение}{\textbf{\textit{входное значение формального нейрона*}} --- неролевое отношение, связывающее нейрон входного слоя со значением признака п.в.а., который подается на вход нейронной сети.}
    \scntext{теоретическая неточность}{Использование множества как формы представления входных данных является серьезным допущением, так как на практике входные данные структурированы более сложно --- в многомерные массивы. Самым близким теоретическим аналогом здесь выступает тензор. К сожалению, описание теории нейронных сетей с помощью тензорного исчисления в литературе как таковое отсутствует, но активно используется на практике: например, во многих разрабатываемых нейросетевых фреймворках. Формализация нейронных сетей с помощью тензоров видится авторам наиболее вероятным направлением работы в ближайших изданиях \textit{стандарта OSTIS}.}

\scnheader{паттерн входной активности и.н.с.}
	\scnidtf{п.в.а.}
    \scniselement{мультимножество}
    \scniselement{кортеж}
    \scntext{пояснение}{\textbf{\textit{паттерн входной активности и.н.с.}} --- ориентированное мультимножество численных значений признаков некоторого объекта, которые могут выступать в качестве входных значений нейронов.}
	\scntext{примечание}{В текущей версии \textit{Стандарта OSTIS} предполагается, что п.в.а. содержит только предобработанные данные, то есть данные приведенные к численному виду и, возможно, преобразованные с помощью известных статистических методов (например, нормирования).}

\scnheader{признак}
    \scnidtf{feature}
    \scnidtf{множество признаков}
    \scnsubset{ролевое отношение}
    \scntext{пояснение}{\textbf{\textit{признак}} --- множество ролевых отношений, каждое из которых связывает некоторый п.в.а. с численным значением, которое характеризует данный п.в.а. с какой-либо стороны.}

\scnheader{функция активации*}
    \scnidtf{функция активации нейрона*}
    \scniselement{неролевое отношение}
    \scniselement{бинарное отношение}
    \scntext{пояснение}{\textbf{\textit{функция активации*}} --- неролевое отношение, связывающее формальный нейрон с функцией, результат применения которой к \textbf{\textit{взвешенной сумме нейрона}} определяет его \textbf{\textit{выходное значение}}.}
  	\scnrelfrom{область определения}{\scnnonamednode}
  	\begin{scnindent}
  		\begin{scnreltoset}{объединение}
  			\scnitem{формальный нейрон}
  			\scnitem{функция}
  		\end{scnreltoset}
  	\end{scnindent}
    \scnrelfrom{первый домен}{формальный нейрон}
    \scnrelfrom{второй домен}{функция}
    \begin{scnindent}
    \begin{scnsubdividing}
        \scnitem{линейная функция}
            \begin{scnindent}
                \scntext{формула}{
                    \begin{equation*}
                        y = kS
                    \end{equation*}
                    где \textit{k} --- коэффициент наклона прямой, \textit{S} --- в.с.}
            \end{scnindent}
        \scnitem{пороговая функция}
            \begin{scnindent}
                \scntext{формула}{
                    \begin{equation*}
                        y = sign(S) =
                        \begin{cases}
                            1, S > 0,\\
                            0, S \leq 0
                        \end{cases}
                    \end{equation*}}
            \end{scnindent}
        \scnitem{сигмоидная функция}
            \begin{scnindent}
                \scntext{формула}{
                    \begin{equation*}
                        y = \frac{1}{1+e^\scnleftcurlbrace-cS\scnrightcurlbrace}
                    \end{equation*}
                    где \textit{с} > 0 --- коэффициент, характеризующий ширину сигмоидной функции по оси абсцисс, \textit{S} --- в.с.}
            \end{scnindent}
        \scnitem{функция гиперболического тангенса}
            \begin{scnindent}
                \scntext{формула}{
                    \begin{equation*}
                        y = \frac{e^\scnleftcurlbrace~cS\scnrightcurlbrace-e^\scnleftcurlbrace-cS\scnrightcurlbrace}{e^\scnleftcurlbrace~cs\scnrightcurlbrace+e^\scnleftcurlbrace-cS\scnrightcurlbrace}
                    \end{equation*}
                    где \textit{с} > 0 --- коэффициент, характеризующий ширину сигмоидной функции по оси абсцисс, \textit{S} --- в.с.}
            \end{scnindent}
        \scnitem{функция softmax}
            \begin{scnindent}
                \scntext{формула}{
                    \begin{equation*}
                        y_j = softmax(S_j) = \frac{e^\scnleftcurlbrace~S_j\scnrightcurlbrace}{\sum_{j} e^\scnleftcurlbrace~S_j\scnrightcurlbrace}
                    \end{equation*}
                    где $S_j$ --- в.с. \textit{j}-го выходного нейрона.}
            \end{scnindent}
        \scnitem{функция ReLU}
            \begin{scnindent}
                \scntext{формула}{
                    \begin{equation*}
                        y = F(S) =
                        \begin{cases}
                            S, S > 0,\\
                            kS, S \leq 0
                        \end{cases}
                    \end{equation*}
                    где \textit{k} = 0 или принимает небольшое значение, например, 0.01 или 0.001.}
            \end{scnindent}
    \end{scnsubdividing}
    \end{scnindent}

\scnheader{взвешенная сумма*}
    \scnidtf{взвешенная сумма входных значений*}
    \scnidtf{в.с.}
    \scniselement{неролевое отношение}
    \scniselement{бинарное отношение}
    \scntext{пояснение}{\textbf{\textit{взвешенная сумма*}} --- неролевое отношение, связывающее формальный нейрон с числом, являющимся суммой произведений входных сигналов на весовые коэффициенты входящих в нейрон синаптических связей. }
     \scnrelfrom{область определения}{\scnnonamednode}
    \begin{scnindent}
    	\begin{scnreltoset}{объединение}
    		\scnitem{формальный нейрон}
    		\scnitem{число}
    	\end{scnreltoset}
    \end{scnindent}
    \scnrelfrom{первый домен}{формальный нейрон}
    \scnrelfrom{второй домен}{число}
    \scnrelfrom{формула}{
        \begin{equation*}
            S = \sum_{i=1}^\scnleftcurlbrace~n\scnrightcurlbrace w_ix_i - T
        \end{equation*}
        где \textit{n} --- размерность вектора входных значений, $w_i$ --- \textit{i}-тый элемент вектора весовых коэффициентов, $x_i$ --- \textit{i}-тый элемент вектора входных значений, \textit{T} --- пороговое значение.}

\scnheader{выходное значение формального нейрона*}
    \scnidtf{выходное значение нейрона*}
    \scnidtf{выходное значение*}
    \scniselement{неролевое отношение}
    \scniselement{бинарное отношение}
    \scnrelfrom{первый домен}{формальный нейрон}
    \scnrelfrom{второй домен}{число}
    \scnrelfrom{область определения}{\scnnonamednode}
    \begin{scnindent}
    	\begin{scnreltoset}{объединение}
    		\scnitem{формальный нейрон}
    		\scnitem{число}
    	\end{scnreltoset}
    \end{scnindent}
    \scntext{пояснение}{\textbf{\textit{входное значение*}} --- неролевое отношение, связывающее нейрон с числом, являющимся результатом применения функции активации нейрона к его взвешенной сумме.}
    \scntext{примечание}{Выходное значение нейрона является одним из входных сигналов для всех нейронов, в которые ведут выходящие из данного нейрона синапсы.}

\scnheader{слой и.н.с.}
    \scnidtf{слой}
    \scnidtf{слой искусственной нейронной сети}
    \scnidtf{множество слоев искусственных нейронных сетей}
    \scntext{примечание}{отдельный слой является искусственной нейронной сетью с одним слоем}
    \scnsubset{искусственная нейронная сеть}
    \scntext{пояснение}{\textbf{\textit{слой и.н.с}} --- это множество нейронных элементов, на которые в каждый такт времени параллельно поступает информация от других нейронных элементов сети (\scncite{Golovko2017})}
    \scntext{пояснение}{\textbf{\textit{слой и.н.с.}} --- это множество формальных нейронов, осуществляющих параллельную независимую обработку вектора или матрицы входных значений}
    \scntext{примечание}{функция активации слоя является функцией активации всех формальных нейронов этого слоя}
    \scntext{примечание}{конфигурация слоя задается типом, количеством формальных нейронов, функцией активации}
    \scntext{примечание}{описание последовательности слоев и.н.с. с конфигурацией каждого слоя задает архитектуру и.н.с.}
    \begin{scnsubdividing}
        \scnitem{полносвязный слой и.н.с.}
        \begin{scnindent}
            \scnidtf{слой, в котором каждый нейрон имеет связь с каждым нейроном предшествующего слоя}
            \scnidtf{слой, в котором каждый нейрон является полносвязным}
        \end{scnindent}
        \scnitem{сверточный слой и.н.с.}
        \begin{scnindent}
            \scnidtf{слой, в котором каждый нейрон является сверточным}
        \end{scnindent}
        \scnitem{слой и.н.с. нелинейного преобразования}
        \begin{scnindent}
            \scnidtf{слой, осуществляющий нелинейное преобразование входных данных}
            \scntext{пояснение}{Как правило, выделяются в отдельные слои только в программных реализациях. Фактически рассматриваются как финальный этап расчета выходной активности любого нейрона --- применение функции активации.}
            \scntext{примечание}{не изменяет размерность входных данных}
        \end{scnindent}
        \scnitem{dropout слой и.н.с.}
        \begin{scnindent}
            \scnidtf{слой, реализующий технику регуляризации dropout}
            \scntext{примечание}{Данный тип слоя функционирует только во время обучения и.н.с.}
            \scntext{пояснение}{Поскольку полносвязные слои имеют большое количество настраиваемых параметров, они подвержены эффекту переобучения. Один из способов устранить такой негативный эффект --- выполнить частичный отсев результатов на выходе полносвязного слоя. На этапе обучения техника dropout позволяет отбросить выходную активность некоторых нейронов с определенной, заданной вероятностью. Выходная активность \scnqqi{отброшенных} нейронов полагается равной нулю.}
        \end{scnindent}
        \scnitem{pooling слой и.н.с.}
        \begin{scnindent}
            \scnidtf{подвыборочный слой}
            \scnidtf{объединяющий слой}
            \scnidtf{слой, осуществляющий уменьшение размерности входных данных}
        \end{scnindent}
        \scnitem{слой и.н.с. батч-нормализации}
    \end{scnsubdividing}

\scnheader{распределяющий слой*}
    \scnidtf{входной слой*}
    \scniselement{неролевое отношение}
    \scniselement{бинарное отношение}
    \scndefinition{\textbf{\textit{распределяющий слой*}} --- неролевое отношение, связывающее искусственную нейронную сеть с ее слоем, нейроны которого принимают входные значения всей нейронной сети.}
    \scnrelfrom{область определения}{искусственная нейронная сеть}
    \scnrelfrom{первый домен}{искусственная нейронная сеть}
    \scnrelfrom{второй домен}{слой и.н.с.}

\scnheader{обрабатывающий слой*}
    \scniselement{неролевое отношение}
    \scniselement{бинарное отношение}
    \scndefinition{\textbf{\textit{обрабатывающий слой*}} --- неролевое отношение, связывающее искусственную нейронную сеть с ее слоем, нейроны которого принимают на вход выходные значения нейронов предыдущего слоя.}
    \scnrelfrom{область определения}{искусственная нейронная сеть}
    \scnrelfrom{первый домен}{искусственная нейронная сеть}
    \scnrelfrom{второй домен}{слой и.н.с}

\scnheader{выходной слой*}
    \scniselement{неролевое отношение}
    \scniselement{бинарное отношение}
    \scntext{пояснение}{\textbf{\textit{выходной слой*}} --- неролевое отношение, связывающее искусственную нейронную сеть с ее слоем, выходные значения нейронов которого являются выходными значениями всей нейронной сети.}
    \scnrelfrom{область определения}{искусственная нейронная сеть}
    \scnrelfrom{первый домен}{искусственная нейронная сеть}
    \scnrelfrom{второй домен}{слой и.н.с}

\bigskip
\end{scnsubstruct}
\scnendsegmentcomment{Предметная область и онтология искусственных нейронных сетей}

\scnsegmentheader{Предметная область и онтология действий по обработке искусственной нейронной сети}
\begin{scnsubstruct}

\scnheader{Предметная область действий по обработке искусственных нейронных сетей}
    \scnidtf{Предметная область действий по обработке и.н.с.}
    \scniselement{предметная область}
    \begin{scnhaselementrole}{максимальный класс объектов исследования}
        {действие по обработке искусственных нейронных сетей}
    \end{scnhaselementrole}
    \begin{scnhaselementrolelist}{класс объектов исследования}
        \scnitem{действие по обработке искусственных нейронных сетей}
        \scnitem{действие конфигурации весовых коэффициентов и.н.с.}
        \scnitem{действие конфигурации и.н.с.}
        \scnitem{действие интерпретации и.н.с.}
        \scnitem{метод обучения и.н.с.}
        \scnitem{метод обучения с учителем}
        \scnitem{метод обратного распространения ошибки}
        \scnitem{метод обучения без учителя}
        \scnitem{метод оптимизации}
        \scnitem{функция потерь}
        \scnitem{параметр обучения}
        \scnitem{скорость обучения}
        \scnitem{моментный параметр}
        \scnitem{параметр регуляризации}
        \scnitem{размер группы обучения}
        \scnitem{количество эпох обучения}
        \scnitem{выборка}
    \end{scnhaselementrolelist}
    \begin{scnhaselementrolelist}{исследуемое отношение}
        \scnitem{обучающая выборка\scnrolesign}
        \scnitem{тестовая выборка\scnrolesign}
        \scnitem{валидационная выборка\scnrolesign}
        \scnitem{метод обучения\scnrolesign}
        \scnitem{метод оптимизации\scnrolesign}
        \scnitem{функция потерь\scnrolesign}
    \end{scnhaselementrolelist}

\scnheader{действие по обработке искусственной нейронной сети}
    \scnidtf{действие по обработке и.н.с.}
    \scnidtf{действие с искусственной нейронной сетью}
    \scnsubset{действие}
    \scntext{пояснение}{В зависимости от того, является ли искусственная нейронная сеть знаком внешней по отношению к памяти системы сущности, элементы множества действие по обработке и.н.с. являются либо элементами множества \textbf{\textit{действие, выполняемое кибернетической системой в своей внешней среде}}, либо элементом множества \textbf{\textit{действие, выполняемое кибернетической системой в собственной памяти.}}.}
    \begin{scnsubdividing}
        \scnitem{действие конфигурации и.н.с.}
        \begin{scnindent}
        \begin{scnsubdividing}
            \scnitem{действие создания и.н.с.}
            \scnitem{действие редактирования и.н.с.}
            \scnitem{действие удаления и.н.с.}
            \scnitem{действие конфигурации слоя и.н.с.}
            \begin{scnindent}
                \begin{scnsubdividing}
                    \scnitem{действие добавления слоя в и.н.с.}
                    \scnitem{действие редактирования слоя и.н.с.}
                    \scnitem{действие удаления слоя и.н.с.}
                    \scnitem{действие установки функции активации нейронов слоя и.н.с.}
                    \scnitem{действие конфигурации нейрона в слое и.н.с.}
                    \begin{scnindent}
                        \begin{scnsubdividing}
                            \scnitem{действие добавления нейрона в слой и.н.с.}
                            \scnitem{действие редактирования нейрона в слое и.н.с.}
                            \scnitem{действие удаления нейрона из слоя и.н.с.}
                            \scnitem{действие установки функции активации нейрона в слое и.н.с.}
                        \end{scnsubdividing}
                    \end{scnindent}
                \end{scnsubdividing}
            \end{scnindent}
        \end{scnsubdividing}
        \end{scnindent}
        \scnitem{действие конфигурации весовых коэффициентов и.н.с.}
        \begin{scnindent}
            \scnsuperset{действие обучения и.н.с.}
            \scnsuperset{действие начальной инициализации весов и.н.с.}
            \begin{scnindent}
                \scnsuperset{действие начальной инициализации весов нейронов слоя и.н.с.}
                \begin{scnindent}
                    \scnsuperset{действие начальной инициализации весов нейрона и.н.с.}
                \end{scnindent}
            \end{scnindent}
        \end{scnindent}
        \scnitem{действие интерпретации и.н.с.}
    \end{scnsubdividing}
    \scntext{примечание}{Действия по обработке и.н.с осуществляет соответствующий коллектив агентов.}
    \scntext{пояснение}{Так как в результате действий по обработке и.н.с объект этих действий, конкретная и.н.с, может существенно меняться (меняется конфигурация сети, ее весовые коэффициенты), то и.н.с представляется в базе знаний как темпоральное объединение всех ее версий. Каждая версия является и.н.с. и темпоральной сущностью. На множестве этих темпоральных сущностей задается темпоральная последовательность с указанием первой и последней версии. Для каждой версии описываются специфичные знания. Общие для всех версий знания описываются для и.н.с, являющейся темпоральным объединением всех версий.}
    \begin{scnindent}
        \scnrelfrom{пример}{\scnfileimage[30em]{Contents/part_ps/images/sd_ps/sd_ann/temporal_neural_network_scg.png}}
    \end{scnindent}

\scnheader{действие обучения и.н.с.}
    \scnidtf{действие обучения искусственной нейронной сети}
    \scnsubset{действие конфигурации весовых коэффициентов и.н.с.}
    \scndefinition{\textbf{\textit{действие обучения и.н.с.}} --- действие, в ходе которого реализуется определенный метод обучения и.н.с. с заданными параметрами обучения и.н.с, методом оптимизации и функцией потерь.}
    \begin{scnrelfromset}{известные проблемы}
        \scnfileitem{Переобучение --- проблема, возникающая при обучении и.н.с., заключающаяся в том, что сеть хорошо адаптируется к п.в.а. из обучающей выборки, при этом теряя способность к обобщению. Переобучение возникает из-за применения неоправданно сложной модели при обучении и.н.с. Это происходит, когда количество настраиваемых параметров и.н.с. намного больше размера обучающей выборки. Возможные варианты решения проблемы заключаются в упрощении модели, увеличении выборки, использовании регуляризации (параметр регуляризации, техника dropout и т.д.).\\
            Обнаружение переобученности сложнее, чем недообученности. Как правило, для этого применяется кросс-валидация на валидационной выборке, позволяющая оценить момент завершения процесса обучения. Идеальным вариантом является достижение баланса между переобученностью и недообученностью.}
        \scnfileitem{Недообучение --- проблема, возникающая при обучении и.н.с., заключающаяся в том, что сеть дает одинаково плохие результаты на обучающей и контрольной выборках. Чаще всего такого рода проблема возникает при недостаточном времени, затраченном на обучение модели. Однако это может быть вызвано и слишком простой архитектурой модели либо малым размером обучающей выборки. Соответственно решение, которое может быть принято ML-инженером, заключается в устранении этих недостатков: увеличение времени обучения, использование модели с большим числом настраиваемых параметров, увеличение размера обучающей выборки, а также уменьшение регуляризации и более тщательный отбор признаков для обучающих примеров.}
    \end{scnrelfromset}
    \scnrelfrom{описание примера}{\scnfileimage{Contents/part_ps/images/sd_ps/sd_ann/ann_trainning_scg.png}}

\scnheader{выборка}
    \scnsubset{множество}
	\scntext{пояснение}{\textbf{\textit{выборка}} --- множество п.в.а., используемых в процессе обучения, тестирования и архитектурной настройки и.н.с.}

\scnheader{обучающая выборка\scnrolesign}
    \scnidtf{training set\scnrolesign}
    \scniselement{ролевое отношение}
    \scnrelfrom{первый домен}{действие обучения и.н.с.}
    \scnrelfrom{второй домен}{выборка}
    \scnrelfrom{область определения}{\scnnonamednode}
    \begin{scnindent}
    	\begin{scnreltoset}{объединение}
    		\scnitem{действие обучения и.н.с.}
    		\scnitem{выборка}
    	\end{scnreltoset}
    \end{scnindent}
    \scntext{пояснение}{\textbf{\textit{обучающая выборка\scnrolesign}} --- ролевое отношение, связывающее действие обучения и.н.с. с выборкой, используемой для изменения настраиваемых параметров и.н.с. в процессе ее обучения.}

\scnheader{тестовая выборка\scnrolesign}
    \scnidtf{test set\scnrolesign}
    \scniselement{ролевое отношение}
    \scnrelfrom{первый домен}{действие обучения и.н.с.}
    \scnrelfrom{второй домен}{выборка}
   	\scnrelfrom{область определения}{\scnnonamednode}
   	\begin{scnindent}
   		\begin{scnreltoset}{объединение}
   			\scnitem{действие обучения и.н.с.}
   			\scnitem{выборка}
   		\end{scnreltoset}
   	\end{scnindent}
    \scntext{пояснение}{\textbf{\textit{тестовая выборка\scnrolesign}} --- ролевое отношение, связывающее действие обучения и.н.с. с выборкой, используемой для проверки обобщающей способности обученной и.н.с.}
    \scntext{примечание}{Элементы контрольной выборки не используются в процессе обучения.}

\scnheader{валидационная выборка\scnrolesign}
    \scniselement{ролевое отношение}
    \scnrelfrom{первый домен}{действие обучения и.н.с.}
    \scnrelfrom{второй домен}{выборка}
    \scnrelfrom{область определения}{\scnnonamednode}
   	\begin{scnindent}
   		\begin{scnreltoset}{объединение}
   			\scnitem{действие обучения и.н.с.}
   			\scnitem{выборка}
		\end{scnreltoset}
   	\end{scnindent}
    \scntext{пояснение}{\textbf{\textit{валидационная выборка\scnrolesign}} --- ролевое отношение, связывающее действие обучения и.н.с. с выборкой, используемой для определения (настройки) архитектурных параметров и.н.с. и параметров обучения.}
    \scntext{примечание}{Элементы валидационной выборки не используются в процессе обучения (не входят в обучающую выборку).}

\scnheader{метод обучения и.н.с.}
    \scnsubset{метод}
	\scntext{пояснение}{\textbf{\textit{метод обучения и.н.с.}} --- метод итеративного поиска оптимальных значений настраиваемых параметров и.н.с., минимизирующих некоторую заданную функцию потерь.}
	\scntext{примечание}{Стоит отметить, что хотя целью применения метода обучения является минимизация функции потерь, \scnqqi{полезность} полученной после обучения модели можно оценить только по достигнутому уровню ее обобщающей способности. }
	\scnsuperset{метод обучения с учителем}
	\begin{scnindent}
		\scntext{пояснение}{\textbf{\textit{метод обучения с учителем}} --- метод обучения с использованием заданных целевых переменных. }
		\scnsuperset{метод обратного распространения ошибки}
		\begin{scnindent}
			\scnidtf{м.о.р.о.}
			\scntext{алгоритм}{\\
				\begin{minipage}{\linewidth}
					\begin{algorithm}[H]
						\KwData{$X$ --- данные, $Et$ --- желаемый отклик (метки), $E_m$ --- желаемая ошибка (в соответствии с выбранной функцией потерь)}
						\KwResult{обученная нейронная сеть \textit{Net}}
						инициализация весов \textit{W} и порогов \textit{T};\\
						\Repeat{$E<E_m$}{
							\ForEach{$x \in X$ $\And$ $e \in Et$}{
								фаза прямого распространения сигнала: вычисляются активации для всех слоев и.н.с.;\\
								фаза обратного распространения ошибки: вычисляются ошибки для последнего слоя и всех предшествующих слоев;\\
								изменение настраиваемых параметров и.н.с. в соответствии с вычисленными ошибками;\\
							}
							вычисление общей ошибки E на данной эпохе;
						}
					\end{algorithm}
				\end{minipage}
            }
			\scntext{примечание}{м.о.р.о. использует заданный метод оптимизации и заданную функцию потерь для реализации фазы обратного распространения ошибки и изменения настраиваемых параметров и.н.с. Одним из самых распространенных методов оптимизации является метод стохастического градиентного спуска. Приведенный м.о.р.о. используется для реализации последовательного варианта обучения.}
			\scntext{примечание}{Следует также отметить, что несмотря на то, что метод отнесен к методам обучения с учителем, в случае использования м.о.р.о. для обучения автокодировщиков в классических публикациях он рассматривается как метод обучения без учителя, поскольку в данном случае размеченные данные отсутствуют.}
		\end{scnindent}
	\end{scnindent}
	\scnsuperset{метод обучения без учителя}
	\begin{scnindent}
		\scntext{пояснение}{\textbf{\textit{метод обучения без учителя}} --- метод обучения без использования заданных целевых переменных (в режиме самоорганизации)}
		\scntext{пояснение}{В ходе выполнения алгоритма метода обучения без учителя выявляются полезные структурные свойства набора. Неформально его понимают как метод для извлечения информации из распределения, выборка для которого не была вручную аннотирована человеком (\scncite{Goodfellow2017}). }
	\end{scnindent}

\scnheader{метод обучения\scnrolesign}
    \scniselement{ролевое отношение}
    \scnrelfrom{первый домен}{действие обучения и.н.с.}
    \scnrelfrom{второй домен}{метод обучения и.н.с.}
    \scnrelfrom{область определения}{\scnnonamednode}
    \begin{scnindent}
    	\begin{scnreltoset}{объединение}
    		\scnitem{действие обучения и.н.с.}
    		\scnitem{метод обучения и.н.с.}
    	\end{scnreltoset}
    \end{scnindent}
    \scntext{пояснение}{\textbf{\textit{метод обучения\scnrolesign}} --- ролевое отношение, связывающее действие обучения и.н.с. с методом обучения,  использующимся для обучения и.н.с. в рамках этого действия.}

\scnheader{метод оптимизации}
    \scnsubset{метод}
	\scndefinition{\textbf{\textit{метод оптимизации}} --- метод для минимизации целевой функции потерь при обучении и.н.с.}
	\begin{scnrelfromlist}{включение}
		\scnitem{SGD}
			\begin{scnindent}
				\scnidtf{стохастический градиентный спуск}
				\scnidtf{с.г.с.}
				\scnidtf{stochastic gradient descent}
				\scntext{примечание}{В методе стохастического градиентного спуска корректировка настраиваемых параметров и.н.с. выполняется в направлении максимального уменьшения функции стоимости, т.е. в направлении, противоположном вектору градиента функции потерь (\scncite{Haykin2006})}
			\end{scnindent}
		\scnitem{Nesterov method}
			\begin{scnindent}
				\scnidtf{метод Нестерова}
				\scntext{примечание}{Обучение методом с.г.с. иногда происходит очень медленно. Импульсный метод позволяет ускорить обучение, особенно в условиях высокой кривизны, небольших, но устойчивых градиентов или зашумленных градиентов. В импульсном методе вычисляется экспоненциально затухающее скользящее среднее прошлых градиентов и продолжается движение в этом направлении. Метод Нестерова является вариантом импульсного алгоритма, в котором градиент вычисляется после применения текущей скорости (\scncite{Goodfellow2017})}
			\end{scnindent}
		\scnitem{AdaGrad}
			\begin{scnindent}
				\scnidtf{adaptive gradient}
				\scntext{примечание}{Данный метод по отдельности адаптирует скорости обучения всех настраиваемых параметров и.н.с., умножая их на коэффициент, обратно пропорциональный квадратному корню из суммы всех прошлых значений квадрата градиента (\scncite{Duchi2011})}
			\end{scnindent}
		\scnitem{RMSProp}
			\begin{scnindent}
				\scnidtf{root mean square propagation}
				\scntext{примечание}{Данный метод является модификацией AdaGrad, которая позволяет улучшить его поведение в невыпуклом случае путем изменения способа агрегирования градиента на экспоненциально взвешенное скользящее среднее. Использование экспоненциально взвешенного скользящего среднего гарантирует повышение скорости сходимости после обнаружения выпуклой впадины, как если бы внутри этой впадины алгоритм AdaGrad был инициализирован заново (\scncite{Goodfellow2017})}
			\end{scnindent}
		\scnitem{Adam}
			\begin{scnindent}
				\scnidtf{adaptive moments}
				\scntext{примечание}{Данный метод можно рассматривать как комбинацию RMSProp и AdaGrad 	(\scncite{Kingma2014}). Помимо усредненного первого момента, данный метод использует усредненное значение вторых моментов градиентов}
			\end{scnindent}
	\end{scnrelfromlist}
	\scntext{примечание}{Успешность применения методов оптимизации зависит главным образом от знакомства пользователя с соответствующим алгоритмом (\scncite{Goodfellow2017}). }


\scnheader{метод оптимизации\scnrolesign}
    \scniselement{ролевое отношение}
    \scnrelfrom{первый домен}{метод обучения и.н.с.}
    \scnrelfrom{второй домен}{метод оптимизации}
    \scnrelfrom{область определения}{\scnnonamednode}
    \begin{scnindent}
    	\begin{scnreltoset}{объединение}
    		\scnitem{метод обучения и.н.с.}
    		\scnitem{метод оптимизации}
    	\end{scnreltoset}
    \end{scnindent}
    \scntext{пояснение}{\textbf{\textit{метод оптимизации\scnrolesign}} --- ролевое отношение, связывающее метод обучения и.н.с. с методом оптимизации, использующимся для обучения и.н.с. с помощью данного метода.}

\scnheader{функция потерь}
    \scnsubset{функция}
	\scntext{пояснение}{\textbf{\textit{функция потерь}} --- функция, используемая для вычисления ошибки, рассчитываемой как разница между фактическим эталонным значением и прогнозируемым значением, получаемым и.н.с.}
    \begin{scnrelfromlist}{включение}
		\scnitem{MSE}
		\begin{scnindent}
			\scnidtf{mean square error}
			\scnidtf{средняя квадратичная ошибка}
			\scntext{формула}{
				\begin{equation*}
					MSE = \frac{1}{L} \sum_{l=1}^L \sum_{i=1}^m (y_i^l - e_i^l)^2
				\end{equation*}
				где $y_i^l$ --- прогноз модели, $e_i^l$ --- ожидаемый (эталонный) результат, \textit{m} --- размерность выходного вектора, \textit{L} --- объем обучающей выборки.}
		\end{scnindent}
		\scnitem{BCE}
		\begin{scnindent}
			\scnidtf{binary cross entropy}
			\scnidtf{бинарная кросс-энтропия}
			\scntext{формула}{
				\begin{equation*}
					BCE = - \sum_{l=1}^L (e^l \log(y^l) + (1 - e^l)\log(1 - y^l))
				\end{equation*}
				где $y^l$ --- прогноз модели, $e^l$ --- ожидаемый (эталонный) результат: \textit{0} или \textit{1}, \textit{L} --- объем обучающей выборки.}
			\scntext{примечание}{для бинарной кросс-энтропии в выходном слое и.н.с. будет находиться один нейрон}
		\end{scnindent}
		\scnitem{MCE}
		\begin{scnindent}
			\scnidtf{multi-class cross entropy}
			\scnidtf{мультиклассовая кросс-энтропия}
			\scntext{формула}{
				\begin{equation*}
					MCE = - \sum_{l=1}^L \sum_{i=1}^m e_\scnleftcurlbrace~i\scnrightcurlbrace^l \log(y_\scnleftcurlbrace~i\scnrightcurlbrace^l)
				\end{equation*}
				где $y_\scnleftcurlbrace~i\scnrightcurlbrace^l$ --- прогноз модели, $e_i^l$ --- ожидаемый (эталонный результат), \textit{m} --- размерность выходного вектора.}
			\scntext{примечание}{для мультиклассовой кросс-энтропии количество нейронов в выходном слое и.н.с. совпадает с количеством классов}
		\end{scnindent}
	\end{scnrelfromlist}
	\scntext{примечание}{Для решения задачи классификации рекомендуется использовать бинарную или мультиклассовую кросс-энтропийную функцию потерь, для решения задачи регрессии рекомендуется использовать среднюю квадратичную ошибку.}

\scnheader{функция потерь\scnrolesign}
    \scniselement{ролевое отношение}
    \scnrelfrom{первый домен}{метод обучения и.н.с.}
    \scnrelfrom{второй домен}{функция потерь}
    \scnrelfrom{область определения}{\scnnonamednode}
    \begin{scnindent}
    	\begin{scnreltoset}{объединение}
    		\scnitem{метод обучения и.н.с.}
    		\scnitem{функция потерь}
    	\end{scnreltoset}
    \end{scnindent}
    \scntext{пояснение}{\textbf{\textit{функция потерь\scnrolesign}} --- ролевое отношение, связывающее метод обучения и.н.с. с функцией потерь, использующимся для обучения и.н.с. с помощью данного метода.}

\scnheader{параметр обучения}
   \scnidtf{группа наиболее общих параметров метода обучения и.н.с.}
   \begin{scnrelfromset}{состав группы параметров обучения}
       \scnitem{скорость обучения}
          \begin{scnindent}
              \scntext{пояснение}{\textbf{\textit{скорость обучения}} --- параметр, определяющий скорость изменения параметров и.н.с. в процессе обучения.}
          \end{scnindent}
       \scnitem{моментный параметр}
          \begin{scnindent}
              \scnidtf{момент}
              \scnidtf{momentum}
              \scntext{пояснение}{\textbf{\textit{моментный параметр}} --- параметр, используемый в процессе обучения для устранения проблемы \scnqqi{застревания} алгоритма обучения в локальных минимумах минимизируемой функции потерь.}
              \scntext{пояснение}{При обучении и.н.с. частой является ситуация остановки процесса в определенной точке локального минимума без достижения желаемого уровня обобщающей способности и.н.с. Для устранения такого    нежелательного явления вводится дополнительный параметр (момент) позволяющий алгоритму обучения \scnqqi{перескочить} через локальный минимум и продолжить процесс.}
         \end{scnindent}
       \scnitem{параметр регуляризации}
         \begin{scnindent}
             \scntext{пояснение}{\textbf{\textit{параметр регуляризации}} --- параметр, применяемый для контроля уровня переобучения и.н.с.}
             \scntext{пояснение}{\textbf{\textit{регуляризация}} --- добавление дополнительных ограничений к правилам изменения настраиваемых параметров и.н.с. с целью предотвратить переобучение.}
         \end{scnindent}
       \scnitem{размер группы обучения}
         \begin{scnindent}
             \scntext{пояснение}{\textbf{\textit{размер группы обучения}} --- размер группы п.в.а., которая используется для изменения параметров и.н.с. на каждом элементарном шаге обучения.}
         \end{scnindent}
       \scnitem{количество эпох обучения}
       \begin{scnindent}
       		\scntext{пояснение}{\textbf{\textit{эпоха обучения}} --- одна итерация алгоритма обучения, в ходе которой все обучающие п.в.а. из обучающей выборки были однократно использованы.}
       \end{scnindent}
   \end{scnrelfromset}

\bigskip
\end{scnsubstruct}
\scnendsegmentcomment{Предметная область и онтология действий по обработке искусственной нейронной сети}

\bigskip
\end{scnsubstruct}
\scnendcurrentsectioncomment

\end{SCn}


\scsubsubsection[
    \protect\scneditor{Ковалев М.В.}
    \protect\scnmonographychapter{Глава 3.6. Конвергенция и интеграция искусственных нейронных сетей с базами знаний в интеллектуальных компьютерных системах нового поколения}
    ]{Предметная область и онтология синтаксиса sc-моделей искусственных нейронных сетей}
\label{sd_syntax_sc_model_ann}
\scnsegmentheader{Предметная область и онтология синтаксиса sc-моделей искусственных нейронных сетей}
\begin{scnsubstruct}

\scnendcurrentsectioncomment
\end{scnsubstruct}

\scsubsubsection[
    \protect\scneditor{Ковалев М.В.}
    \protect\scnmonographychapter{Глава 3.6. Конвергенция и интеграция искусственных нейронных сетей с базами знаний в интеллектуальных компьютерных системах нового поколения}
    ]{Предметная область и онтология денотационной семантики sc-моделей искусственных нейронных сетей}
\label{sd_denot_sem_sc_model_ann}
\begin{SCn}
\scnsectionheader{Предметная область и онтология денотационной семантики Языка представления нейросетевого метода решения задач}
\begin{scnsubstruct}
	
\scnheader{Денотационная семантика Языка представления нейросетевого метода решения задач}
\scntext{примечание}{Денотационная семантика Языка представления нейросетевого метода решения задач в базах знаний описывается в рамках предметной области и соответствующей ей онтологии нейросетевого метода.}
\scntext{примечание}{Так же в \textit{Предметную область нейросетевых методов} добавлены понятия для описания метрик эффективности \textit{нейросетевых методов}. Данные метрики учитываются \textit{решателем задач} при принятии решения об использовании того или иного \textit{нейросетевого метода}.}

\scnheader{искусственная нейронная сеть}
\scnidtf{и.н.с.}
\scnidtf{множество искусственных нейронных сетей}
\scnidtf{нейронная сеть}
\scnidtf{нейросетевой метод}
\scntext{определение}{\textbf{\textit{искусственная нейронная сеть}} --- это совокупность нейронных элементов и связей между ними.}
\scntext{примечание}{Искусственная нейронная сеть состоит из \textbf{\textit{формальных нейронов}}, которые связаны между собой посредством \textbf{\textit{синаптических связей}}. Нейроны организованы в \textbf{\textit{слои}}. Каждый нейрон слоя принимает сигналы со входящих в него синаптических связей, обрабатывает их единым образом с помощью заданной ему или всему слою \textbf{\textit{функции активации}} и передает результат на выходящие из него синаптические связи.}
\begin{scnindent}
	\begin{scnrelfromset}{источник}
		\scnitem{\scncite{Golovko2017}}
	\end{scnrelfromset}
\end{scnindent}

\scnheader{архитектура и.н.с.}
\scntext{примечание}{\textit{Архитектурой и.н.с.} будем называть совокупность информации о структуре ее слоев, формальных нейронов, синаптических связей и функций активаций. То есть то, что можно обучить и использовать для решения задач.}
\scnrelfrom{пример}{\scnfileimage[30em]{Contents/part_ps/src/images/sd_ps/sd_ann/neural_network.png}}
\begin{scnindent}
	\scntext{примечание}{Пример архитектуры и.н.с.}
\end{scnindent}

\scnheader{искусственная нейронная сеть}
\scntext{примечание}{В соответствии с тем, какая у и.н.с. архитектура, можно выделить соответствующую иерархию классов и.н.с.}
\scnrelfrom{разбиение}{\scnkeyword{Типология и.н.с. по признаку направленности связей\scnsupergroupsign}}
\begin{scnindent}
	\begin{scneqtoset}
		\scnitem{искусственная нейронная сеть с прямыми связями}
		\begin{scnindent}
			\begin{scnsubdividing}
				\scnitem{персептрон}
				\begin{scnindent}
					\begin{scnsubdividing}
						\scnitem{персептрон Розенблатта}
						\scnitem{автоэнкодерная искусственная нейронная сеть}
					\end{scnsubdividing}
				\end{scnindent}
				\scnitem{машина опорных векторов}
				\scnitem{искусственная нейронная сеть радиально-базисных функций}
				\scnitem{сверточная искусственная нейронная сеть}
			\end{scnsubdividing}
		\end{scnindent}
		\scnitem{искусственная нейронная сеть с обратными связями}
		\begin{scnindent}
			\begin{scnsubdividing}
				\scnitem{нейронная сеть Хопфилда}
				\scnitem{нейронная сеть Хэмминга}
			\end{scnsubdividing}
		\end{scnindent}
		\scnitem{рекуррентная искусственная нейронная сеть}
		\begin{scnindent}
			\begin{scnsubdividing}
				\scnitem{искусственная нейронная сеть Джордана}
				\scnitem{искусственная нейронная сеть Элмана}
				\scnitem{мультирекуррентная нейронная сеть}
				\scnitem{LSTM-элемент}
				\scnitem{GRU-элемент}
			\end{scnsubdividing}
		\end{scnindent}
	\end{scneqtoset}
\end{scnindent}
\scnrelfrom{разбиение}{\scnkeyword{Типология и.н.с. по признаку полноты связей\scnsupergroupsign}}
\begin{scnindent}
	\begin{scneqtoset}
		\scnitem{полносвязная искусственная нейронная сеть}
		\scnitem{слабосвязная искусственная нейронная сеть}
	\end{scneqtoset}
\end{scnindent}

\scnheader{формальный нейрон}
\scnidtf{искусственный нейрон}
\scnidtf{нейрон}
\scnidtf{ф.н.}
\scnidtf{нейронный элемент}
\scnidtf{множество нейронов искусственных нейронных сетей}
\scnidtf{математическая модель реального биологического нейрона}
\scntext{примечание}{Отдельный формальный нейрон является искусственной нейронной сети с одним нейроном в единственном слое.}
\scnsubset{искусственная нейронная сеть}
\scntext{пояснение}{\textbf{\textit{формальный нейрон}} --- это основной элемент \textit{искусственной нейронной сети}, применяющий свою \textit{функцию активации} к сумме произведений входных сигналов на весовые коэффициенты:
	\begin{equation*}
		y \eq F\left(\sum\underscore{i=1}\upperscore{n} w\underscore{i} x\underscore{i} - T\right) \eq F(WX - T)
	\end{equation*}
		где $X \eq (x\underscore{1},x\underscore{2},...,x\underscore{n})\upperscore{T}$ --- вектор входного сигнала; $W - (w\underscore{1},w\underscore{2},...,w\underscore{n})$ --- вектор весовых коэффициентов; $T$ --- пороговое значение;
	\textit{F} --- функция активации.}
\begin{scnindent}
	\begin{scnrelfromset}{источник}
		\scnitem{\scncite{Golovko2017}}
	\end{scnrelfromset}
\end{scnindent}
\scnrelfrom{изображение}{\scnfileimage[20em]{Contents/part_ps/src/images/sd_ps/sd_ann/neuron.png}}
\begin{scnindent}
	\scntext{примечание}{Схема модели формального нейрона.}
\end{scnindent}
\scntext{примечание}{Формальные нейроны могут иметь полный набор связей с нейронами предшествующего слоя или неполный (разряженный) набор связей.}
\begin{scnsubdividing}
	\scnitem{полносвязный формальный нейрон}
	\begin{scnindent}
		\scnidtf{нейрон, у которого есть полный набор связей с нейронами предшествующего слоя}
		\scntext{пояснение}{Отдельный обрабатывающий элемент и.н.с., выполняющий функциональное преобразование взвешенной суммы элементов вектора входных значений с помощью функции активации.}
	\end{scnindent}
	\scnitem{сверточный формальный нейрон}
	\begin{scnindent}
		\scntext{пояснение}{Отдельный обрабатывающий элемент и.н.с., выполняющий функциональное преобразование результата операции свертки матрицы входных значений с помощью функции активации.}
		\scntext{примечание}{Сверточный формальный нейрон может быть представлен полносвязным формальным нейроном.}
		\scntext{примечание}{Сверточный формальный нейрон с соответствующим ему ядром свертки может быть представлен нейроном с неполным набором связей.}
	\end{scnindent}
	\scnitem{рекуррентный формальный нейрон}
	\begin{scnindent}
		\scntext{пояснение}{Формальный нейрон, имеющий обратную связь с самим собой или с другими нейронами и.н.с.}
	\end{scnindent}
\end{scnsubdividing}

\scnheader{синаптическая связь}
\scnidtf{синапс}
\scnsubset{ориентированная пара}
\scndefinition{\textbf{\textit{синаптическая связь}} --- ориентированная пара, первым компонентом которой является нейрон, из которого исходит сигнал, а вторым компонентом --- нейрон, который принимает этот сигнал.}

\scnheader{слой и.н.с.}
\scnidtf{слой}
\scnidtf{слой искусственной нейронной сети}
\scnidtf{множество слоев искусственных нейронных сетей}
\scntext{примечание}{Отдельный слой является искусственной нейронной сетью с одним слоем. Следует отметить принципиальную важность этого замечания. Один слой и.н.с. уже является нейронной сетью, поскольку над ним можно производить все основные операции, которые производятся над \scnqq{большой} и.н.с. (его можно обучить и использовать для решения определенной задачи).}
\scnsubset{искусственная нейронная сеть}
\scntext{пояснение}{\textbf{\textit{слой и.н.с}} --- это множество нейронных элементов, на которые в каждый такт времени параллельно поступает информация от других нейронных элементов сети.}
\begin{scnindent}
	\begin{scnrelfromset}{источник}
		\scnitem{\scncite{Golovko2017}}
	\end{scnrelfromset}
\end{scnindent}
\scntext{пояснение}{\textbf{\textit{слой и.н.с.}} --- это множество формальных нейронов, осуществляющих параллельную независимую обработку вектора или матрицы входных значений}
\scnrelfrom{разбиение}{\scnkeyword{Типология слоев и.н.с. по признаку операции, осуществляемой слоем\scnsupergroupsign}}
\begin{scnindent}
	\begin{scneqtoset}
	\scnitem{полносвязный слой и.н.с.}
	\begin{scnindent}
		\scnidtf{слой, в котором каждый нейрон имеет связь с каждым нейроном предшествующего слоя}
		\scnidtf{слой, в котором каждый нейрон является полносвязным}
	\end{scnindent}
	\scnitem{сверточный слой и.н.с.}
	\begin{scnindent}
		\scnidtf{слой, в котором каждый нейрон является сверточным}
	\end{scnindent}
	\scnitem{слой и.н.с. нелинейного преобразования}
	\begin{scnindent}
		\scnidtf{слой, осуществляющий нелинейное преобразование входных данных}
		\scntext{пояснение}{Как правило, выделяются в отдельные слои только в программных реализациях. Фактически рассматриваются как финальный этап расчета выходной активности любого нейрона --- применение функции активации.}
		\scntext{примечание}{не изменяет размерность входных данных}
	\end{scnindent}
	\scnitem{dropout слой и.н.с.}
	\begin{scnindent}
		\scnidtf{слой, реализующий технику регуляризации dropout}
		\scntext{примечание}{Данный тип слоя функционирует только во время обучения и.н.с.}
		\scntext{пояснение}{Поскольку полносвязные слои имеют большое количество настраиваемых параметров, они подвержены эффекту переобучения. Один из способов устранить такой негативный эффект --- выполнить частичный отсев результатов на выходе полносвязного слоя. На этапе обучения техника dropout позволяет отбросить выходную активность некоторых нейронов с определенной, заданной вероятностью. Выходная активность \scnqqi{отброшенных} нейронов полагается равной нулю.}
	\end{scnindent}
	\scnitem{pooling слой и.н.с.}
	\begin{scnindent}
		\scnidtf{подвыборочный слой}
		\scnidtf{объединяющий слой}
		\scnidtf{слой, осуществляющий уменьшение размерности входных данных}
	\end{scnindent}
	\scnitem{слой и.н.с. батч-нормализации}
	\end{scneqtoset}
	\begin{scnindent}
		\scntext{примечание}{Нужно отметить, что данный перечень неполный --- разновидности слоев и.н.с. появляются практически в каждой заслуживающей внимания публикации по нейросетевым алгоритмам и на текущий момент их существует достаточно много, однако, как правило, при построении более традиционных архитектур ограничиваются только приведенными вариантами слоев.}
	\end{scnindent}
\end{scnindent}
\scntext{примечание}{слои и.н.с. также могут быть классифицированы по исполняемой роли в рамках архитектуры (место в последовательности слоев и.н.с.).\\
	\\Так, например, слой, расположенный первым, называется распределяющим. Слои, расположенные далее, за исключением последнего, называются обрабатывающими. Наконец, последний слой носит название выходного слоя и.н.с.}


\scnheader{функция активации*}
\scnidtf{функция активации нейрона*}
\scniselement{неролевое отношение}
\scniselement{бинарное отношение}
\scntext{примечание}{функция активации* --- последний архитектурный компонент и.н.с.}
\scntext{пояснение}{\textbf{\textit{функция активации*}} --- неролевое отношение, связывающее формальный нейрон с функцией, результат применения которой к \textbf{\textit{взвешенной сумме нейрона}} определяет его \textbf{\textit{выходное значение}}.}
  \scnrelfrom{область определения}{\scnnonamednode}
  \begin{scnindent}
	  \begin{scnreltoset}{объединение}
		  \scnitem{формальный нейрон}
		  \scnitem{функция}
	  \end{scnreltoset}
  \end{scnindent}
\scnrelfrom{первый домен}{формальный нейрон}
\scnrelfrom{второй домен}{функция}
\begin{scnindent}
\begin{scnsubdividing}
	\scnitem{линейная функция}
		\begin{scnindent}
			\scntext{формула}{
				\begin{equation*}
					y = kS
				\end{equation*}}
				\begin{scnindent}
					\scntext{примечание}{\textit{k} --- коэффициент наклона прямой, \textit{S} --- в.с.}
				\end{scnindent}
		\end{scnindent}
	\scnitem{пороговая функция}
		\begin{scnindent}
			\scntext{формула}{
				\begin{equation*}
					y \eq sign(S) \eq
					\begin{cases}
						1, S > 0,\\
						0, S \leq 0
					\end{cases}
				\end{equation*}}
		\end{scnindent}
	\scnitem{сигмоидная функция}
		\begin{scnindent}
			\scntext{формула}{
				\begin{equation*}
					y \eq \frac{1}{1+e\upperscore{-cS}}
				\end{equation*}}
				\begin{scnindent}
					\scntext{примечание}{\textit{с} > 0 --- коэффициент, характеризующий ширину сигмоидной функции по оси абсцисс, \textit{S} --- в.с.}
				\end{scnindent}				
		\end{scnindent}
	\scnitem{функция гиперболического тангенса}
		\begin{scnindent}
			\scntext{формула}{
				\begin{equation*}
					y \eq \frac{e\upperscore{cS}-e\upperscore{-cS}}{e\underscore{cs}+e\upperscore{-cS}}
				\end{equation*}}
				\begin{scnindent}
					\scntext{примечание}{\textit{с} > 0 --- коэффициент, характеризующий ширину сигмоидной функции по оси абсцисс, \textit{S} --- в.с.}
				\end{scnindent}
		\end{scnindent}
	\scnitem{функция softmax}
		\begin{scnindent}
			\scntext{формула}{
				\begin{equation*}
					y\underscore{j} \eq softmax(S\underscore{j}) \eq \frac{e\underscore{S\underscore{j}}}{\sum\underscore{j} e\upperscore{S\underscore{j}}}
				\end{equation*}}
				\begin{scnindent}
					\scntext{примечание}{$S\underscore{j}$ --- в.с. \textit{j}-го выходного нейрона.}
				\end{scnindent}
		\end{scnindent}
	\scnitem{функция ReLU}
		\begin{scnindent}
			\scntext{формула}{
				\begin{equation*}
					y \eq F(S) \eq
					\begin{cases}
						S, S > 0,\\
						kS, S \leq 0
					\end{cases}
				\end{equation*}}
				\begin{scnindent}
					\scntext{примечание}{\textit{k} = 0 или принимает небольшое значение, например, 0.01 или 0.001.}
				\end{scnindent}
		\end{scnindent}
\end{scnsubdividing}
\end{scnindent}

\scnheader{параметр и.н.с.}
    \scnsubset{параметр}
    \begin{scnsubdividing}
        \scnitem{настраиваемый параметр и.н.с.}
        \begin{scnindent}
            \scnidtf{параметр и.н.с., значение которого изменяется в ходе обучения}
            \begin{scnsubdividing}
                \scnitem{весовой коэффициент синаптической связи}
                \scnitem{пороговое значение}
                \scnitem{ядро свертки}
                \begin{scnindent}
                    \scnidtf{квадратная матрица произвольного порядка, элементы которой изменяются в процессе обучения и.н.с.}
                    \scntext{примечание}{Если сверточный формальный нейрон представить в виде полносвязного формального нейрона, соответствующее ядро свертки преобразуется в вектор весовых коэффициентов.}
                \end{scnindent}
            \end{scnsubdividing}
        \end{scnindent}
        \scnitem{архитектурный параметр и.н.с.}
        \begin{scnindent}
            \scntext{примечание}{Параметр и.н.с., определяющий ее архитектуру.}
            \begin{scnsubdividing}
                \scnitem{количество слоев}
                \scnitem{количество нейронов}
                \scnitem{количество синапсов}
            \end{scnsubdividing}
        \end{scnindent}
    \end{scnsubdividing}

\scnheader{метрика оценки качества и.н.с.}
\scntext{примечание}{Метрики могут быть классифицированы по типу решаемой задачи.}
\scnrelfrom{разбиение}{Типология метрик по признаку решаемой задачи\scnsupergroupsign}
\begin{scnindent}
	\begin{scneqtoset}
		\scnitem{классификационные метрики}
		\begin{scnindent}
			\begin{scnrelfromset}{декомпозиция}
				\scnitem{точность и.н.с.}
				\scnitem{полнота и.н.с.}
				\scnitem{F1-метрика}
			\end{scnrelfromset}
		\end{scnindent}
		\scnitem{регрессионные метрики}
		\begin{scnindent}
			\begin{scnrelfromset}{декомпозиция}
				\scnitem{MAE}
				\scnitem{MAPE}
				\scnitem{RMSE}
			\end{scnrelfromset}
		\end{scnindent}
	\end{scneqtoset}
\end{scnindent}

\scnheader{точность и.н.с.}
\scnidtf{precision}
\scnidtf{доля верно идентифицированных положительных исходов в общем числе исходов, которые были идентифицированы как положительные}
\scntext{формула}{
	\begin{equation*}
		PRE = \frac{TP}{TP + FP}
	\end{equation*}}
	\begin{scnindent}
        \scntext{примечание}{\textit{TP} и \textit{FP} --- число истинно-положительных и ложно-положительных предсказаний нейронной сети соответственно}
    \end{scnindent}

\scnheader{полнота и.н.с.}
\scnidtf{recall}
\scnidtf{доля верно идентифицированных положительных исходов в общем числе положительных исходов}
\scntext{формула}{
	\begin{equation*}
		REC = \frac{TP}{TP + FN}
	\end{equation*}}
	\begin{scnindent}
        \scntext{примечание}{\textit{TP} и \textit{FN} --- число истинно-положительных и ложно-отрицательных предсказаний нейронной сети соответственно}
    \end{scnindent}

\scnheader{F1-метрика}
\scntext{формула}{
	\begin{equation*}
		F1 = 2 * \frac{PRE * REC}{PRE + REC}
	\end{equation*}}
	\begin{scnindent}
        \scntext{примечание}{\textit{PRE} и \textit{REC} --- точность и полнота и.и.с. соответственно}
    \end{scnindent}

\scnheader{MAE}
\scnidtf{mean absolute error}
\scntext{формула}{$\frac{1}{N} \sum\underscore{i \eq 1}\upperscore{N} |y\underscore{etalon}\upperscore{i} - y\underscore{predicted}\upperscore{i}|$}
\begin{scnindent}
	\scntext{примечание}{$y\underscore{etalon}\upperscore{i}$ --- эталонное значение,\\ $y\underscore{predicted}\upperscore{i}$ --- значение, полученное и.н.с.,\\ \textit{N} --- объем обучающей выборки}
\end{scnindent}

\scnheader{MAPE}
\scnidtf{mean absolute percentage error}
\scntext{формула}{$\frac{1}{N} \sum\underscore{i \eq 1}\upperscore{N} \frac{|y\underscore{etalon}\upperscore{i} - y\underscore{predicted}\upperscore{i}|}{y\underscore{etalon}\upperscore{i}} * 100\percent$}
\begin{scnindent}
	\scntext{примечание}{$y\underscore{etalon}\upperscore{i}$ --- эталонное значение,\\ $y\underscore{predicted}\upperscore{i}$ --- значение, полученное и.н.с.,\\ \textit{N} --- объем обучающей выборки}
\end{scnindent} 

\scnheader{RMSE}
\scnidtf{root mean squared error}
\scntext{формула}{$\sqrt{\frac{1}{N} \sum\underscore{i \eq 1}\upperscore{N} (y\underscore{etalon}\upperscore{i} - y\underscore{predicted}\upperscore{i})\upperscore{2}}$}
\begin{scnindent}
	\scntext{примечание}{$y\underscore{etalon}\upperscore{i}$ --- эталонное значение,\\ $y\underscore{predicted}\upperscore{i}$ --- значение, полученное и.н.с.,\\ \textit{N} --- объем обучающей выборки}
\end{scnindent}

\scnheader{SCg-текст. Пример формализации архитектуры искусственной нейронной сети в базе знаний}
\scnrelfrom{изображение}{\scnfileimage[20em]{Contents/part_ps/src/images/sd_ps/sd_ann/neural_network_scg.png}}

\bigskip
\end{scnsubstruct}

\begin{scnrelfromvector}{заключение}
	\scnfileitem{С помощью выделенных понятий становится возможна формализация в \textit{базе знаний} архитектуры конкретных \textit{и.н.с.} В качестве примера, на рисунке \textit{SCg-текст. Пример формализации архитектуры искусственной нейронной сети в базе знаний} представлен пример формализации полносвязной двухслойной \textit{и.н.с.} с двумя нейронами на входном слое и одном нейроне на обрабатывающем слоев.}
	\scnfileitem{Следует отметить, что в практике авторов еще не было необходимости явно представлять и.н.с., как это показано на рисунке \textit{SCg-текст. Пример формализации архитектуры искусственной нейронной сети в базе знаний}. Чаще всего, представление и.н.с. сводилось к представлению ее операционной семантики в виде SCP-программы, как это будет показано далее.}
\end{scnrelfromvector}

\scnendcurrentsectioncomment
\end{SCn}


\scsubsubsection[
    \protect\scneditor{Ковалев М.В.}
    \protect\scnmonographychapter{Глава 3.6. Конвергенция и интеграция искусственных нейронных сетей с базами знаний в интеллектуальных компьютерных системах нового поколения}
    ]{Предметная область и онтология операционной семантики sc-моделей искусственных нейронных сетей}
\label{sd_oper_sem_sc_model_ann}
\begin{SCn}
\scnsectionheader{Предметная область и онтология операционной семантики Языка представления нейросетевого метода решения задач}
\begin{scnsubstruct}
		
\scnheader{Операционная семантика Языка представления нейросетевого метода решения задач}
\scntext{примечание}{Операционной семантикой любого языка представления методов решения задач является спецификация семейства агентов, обеспечивающих интерпретацию любого метода, принадлежащего соответствующему классу методов. Это семейство является интерпретатором соответствующего метода решения задач. В рамках технологии OSTIS такой интерпретатор называется моделью решения задач. Так как в рамках Технологии OSTIS используется многоагентный подход, то разработка нейросетевой модели решения задач сводится к разработке агентно-ориентированной	модели интерпретации и.н.с.}
\scntext{примечание}{\textbf{\textit{Операционная семантика Языка представления нейросетевого метода в базах знаний}} задается \textit{многоагентный подход} к интерпретации \textit{искусственных нейронных сетей} и спецификацией соответствующих действий.}
\scntext{примечание}{Нейросетевой метод описан в виде программы на некотором \textit{языке программирования}, который может быть как внешним по отношению к \textit{ostis-системе}, так и внутренним (на данный момент, \textit{Язык SCP}). Каждому такому \textit{языку программирования} соответствует некоторая дочерняя \textit{предметная область} \textit{Предметная область нейросетевых методов}}
\begin{scnindent}
	\scnrelfrom{источник}{\scncite{Kovalev2022}}
\end{scnindent}
\scntext{примечание}{В случае описания \textit{нейросетевого метода} на внешнем языке, такой метод описывается в соответствующей предметной области, в рамках которой также специфицируется действие интерпретации данного метода. Данному действию соответствует агент, реализованный на соответствующем \textit{языке программирования}.
	\\Однако для достижения конвергенции и интеграции необходимо описывать нейросетевые методы на внутреннем языке ostis-системы, которым является \textit{Язык SCP}.
	\\Интерпретация \textit{scp-программы} сводится к агентно-ориентированной обработке действий в sc-памяти. Этими действиями являются \textit{scp-операторы}.}

\scnheader{Предметная область нейросетевых методов}
\scnidtf{Предметная область искусственных нейронных сетей}
\begin{scnrelfromlist}{дочерняя предметная область}
    \scnitem{Предметная область нейросетевых методов SCP}
    \scnitem{Предметная область нейросетевых методов Python}
    \scnitem{Предметная область нейросетевых методов C++}
\end{scnrelfromlist}

\scnheader{действие интерпретации слоя и.н.с.}
\begin{scnrelfromset}{декомпозиция}
	\scnitem{действие вычисления взвешенной суммы всех нейронов слоя}
	\scnitem{действие вычисления функции активации всех нейронов слоя}
	\scnitem{действие интерпретации сверточного слоя}
	\scnitem{действие интерпретации пулинг слоя}
\end{scnrelfromset}
\scntext{примечание}{При необходимости задавать различные аргументы для нейронов одного и того же слоя, можно специфицировать соответствующие действия, однако на данный момент этого не было произведено из-за слабой изученности подобного рода \textit{нейросетевых моделей решения задач}.}

\scnheader{ориентированное множество чисел}
\scnidtf{ормножество чисел}
\scnrelto{включение}{число}
\scnrelto{включение}{ориентированное множество}
\scnrelto{первый домен}{строковое представление ормножества чисел*}
\scntext{примечание}{Для описания спецификации указанных действий необходимо ввести понятия \textit{ориентированного множества чисел} и \textit{матрицы}, с помощью которых задаются входные значения \textit{и.н.с.}, выходные значения \textit{и.н.с.}, матрицы весовых коэффициентов и прочее.
	\\Каждый элемент ориентированного множества чисел является некоторым числом. Числа могут быть представлены в виде sc-узлов, либо с помощью строкового представления всего множества, для чего используется специальное отношение \textit{строковое представление ормножества чисел*}, которое введено в целях оптимизации некоторых вариантов реализации агента, интерпретирующего действие, использующее понятие ориентированного множества чисел.}
	
\scnheader{матрица}
\scntext{примечание}{\textit{матрица} является \textit{ориентированным множеством} \textit{ориентированных множеств} чисел равной мощности.}

\scnheader{Действие вычисления взвешенной суммы всех нейронов слоя}
\scntext{примечание}{Аргументы (\textit{объекты\scnrolesign}) этого действия задаются следующими отношениями: \textit{входной вектор\scnrolesign}, \textit{матрица весовых коэффициентов нейронов слоя\scnrolesign}.}
\scnhaselementrole{результат}{ориентированное множество чисел, являющихся взвешенной суммой нейронов соответствующего слоя.}

\scnheader{входной вектор\scnrolesign}
\scnrelfrom{первый домен}{действие интерпретации и.н.с.}
\scnrelfrom{второй домен}{ориентированное множество чисел}

\scnheader{матрица весовых коэффициентов нейронов слоя\scnrolesign}
\scnrelfrom{первый домен}{действие по обработке и.н.с.}
\scnrelfrom{второй домен}{матрица}

\scnheader{SCg-текст. Пример действия вычисления взвешенной суммы всех нейронов слоя}
\scnrelfrom{описание примера}{\scnfileimage[30em]{Contents/part_ps/src/images/sd_ps/sd_ann/action_weighted_sum.png}}
\begin{scnindent}
	\scntext{примечание}{Пример спецификации действия вычисления взвешенной суммы всех нейронов слоя для слоя с двумя нейронами и входным вектором размерностью 2}
\end{scnindent}

\scnheader{Действие вычисления функции активации всех нейронов слоя}
\scntext{примечание}{Аргументы этого действия задаются следующими отношениями: \textit{вектор взвешенных сумм нейронов слоя\scnrolesign}, \textit{вектор порогов нейронов слоя\scnrolesign}, \textit{функция активации\scnrolesign}.}
\scnhaselementrole{результат}{ориентированное множество чисел, являющихся выходными значениями нейронов слоя}

\scnheader{вектор взвешенных сумм нейронов слоя\scnrolesign}
\scnrelfrom{первый домен}{действие по обработке и.н.с.}
\scnrelfrom{второй домен}{ориентированное множество чисел}

\scnheader{вектор порогов нейронов слоя\scnrolesign}
\scnrelfrom{первый домен}{действие по обработке и.н.с.}
\scnrelfrom{второй домен}{ориентированное множество чисел}

\scnheader{функция активации\scnrolesign}
\scnrelfrom{первый домен}{действие по обработке и.н.с.}
\scnrelfrom{второй домен}{функция}
\scntext{примечание}{Любой агент, интерпретирующий действия с заданными с помощью отношения \textit{функция активации\scnrolesign} аргументами, должен использовать интерпретатор математических функций. использующихся в качестве функций активации.}

\scnheader{Действие интерпретации сверточного слоя}
\scntext{примечание}{Аргументы этого действия задаются следующими отношениями: \textit{входная матрица\scnrolesign}, \textit{ядро свертки\scnrolesign}, \textit{шаг свертки\scnrolesign}.}
\scnhaselementrole{результат}{Результатом действия является матрица, полученная в результате свертки входной матрицы с ядром свертки.}

\scnheader{входная матрица\scnrolesign}
\scnrelfrom{первый домен}{действие интерпретации и.н.с.}
\scnrelfrom{второй домен}{матрица}

\scnheader{ядро свертки\scnrolesign}
\scnrelfrom{первый домен}{действие интерпретации сверточного слоя}
\scnrelfrom{второй домен}{матрица}

\scnheader{шаг свертки\scnrolesign}
\scnrelfrom{первый домен}{действие интерпретации сверточного слоя}
\scnrelfrom{второй домен}{число}

\scnheader{Действие интерпретации пулинг слоя}
\scntext{примечание}{Аргументы этого действия задаются следующими отношениями: \textit{шаг окна пулинга\scnrolesign}, \textit{размер окна пулинга\scnrolesign}, \textit{входная матрица\scnrolesign}}
\scnhaselementrole{результат}{матрица, полученная в результате пулинга входной матрицы.}

\scnheader{входная матрица\scnrolesign}
\scnrelfrom{первый домен}{действие интерпретации и.н.с.}
\scnrelfrom{второй домен}{матрица}

\scnheader{размер окна пулинга\scnrolesign}
\scnrelfrom{первый домен}{действие интерпретации пулинг слоя}
\scnrelfrom{второй домен}{матрица}

\scnheader{шаг окна пулинга\scnrolesign}
\scnrelfrom{первый домен}{действие интерпретации пулинг слоя}
\scnrelfrom{второй домен}{число}

\scnheader{интерпретатор искусственных нейронных сетей}
\scntext{примечание}{Спецификация агентов, соответствующих указанным действиям, задает агентно-ориентированную модель интерпретации искусственных нейронных сетей. Реализация этой модели будет называться интерпретатором искусственных нейронных сетей}
\scntext{примечание}{Реализация интерпретатора описанных в данной главе действий по построению \textit{и.н.с.} и описания в базе знаний экспертных знаний разработчиков\textit{и.н.с.} (а значит реализация интеллектуальной среды проектирования \textit{и.н.с.}) позволит автоматически, исходя из описания задачи, генерировать нейросетевые методы в памяти \textit{ostis-системы}, что является одним из ключевых направлений дальнейшего развития конвергенции и интеграции и.н.с. с базами знаний.}

\scnheader{Рисунок. Решение задачи \scnqq{ИСКЛЮЧАЮЩЕЕ ИЛИ}}
\scntext{примечание}{Рассмотрим пример описания \textit{нейросетевого метода}, решающего задачу, которая формулируется следующим образом: вычислить результат логической операции \scnqq{ИСКЛЮЧАЮЩЕЕ ИЛИ} для значений двух логических переменных. На рисунке представлено решение этой задачи с помощью сигнальной функции.}
\scnrelfrom{описание примера}{\scnfileimage[30em]{Contents/part_ps/src/images/sd_ps/sd_ann/strong_or_graphic.png}}

\scnheader{однослойный персептрон}
\scntext{примечание}{В работе описан однослойный персептрон, решающий поставленную задачу. Персептрон состоит из двух входных нейронов и одного выходного, с заданным порогом в 0,5 и сигнальной функцией активации:
	\begin{equation*}
		F(S) =
	 	\begin{cases}
	 		1, 0 < S < 0,\\
	 		0, else
	 	\end{cases}
	 \end{equation*}}
\begin{scnindent}
	\begin{scnrelfromset}{источник}
		\scnitem{\scncite{Golovko2017}}
	\end{scnrelfromset}
\end{scnindent}

\scnheader{Рисунок. Схема однослойного персептрона, решающего задачу \scnqq{ИСКЛЮЧАЮЩЕЕ ИЛИ}}
\scnrelfrom{описание примера}{\scnfileimage[30em]{Contents/part_ps/src/images/sd_ps/sd_ann/strong_or_ann.png}}

\scnheader{Рисунок. Метод, решающий задачу \scnqq{ИСКЛЮЧАЮЩЕЕ ИЛИ}, представленный с помощью языка представления нейросетевых методов SCP}
\scnrelfrom{описание примера}{\scnfileimage[30em]{Contents/part_ps/src/images/sd_ps/sd_ann/exclusive_or_ann_scp.png}}

\scnheader{SCg-текст. Представление сигнальной функции активации в памяти ostis-системы}
\scnrelfrom{описание примера}{\scnfileimage[30em]{Contents/part_ps/src/images/sd_ps/sd_ann/signal_function_def.png}}
\scntext{примечание}{Весовые коэффициенты синапсов входного слоя равны 1. На рисунке \textit{Рисунок. Схема однослойного персептрона, решающего задачу \scnqq{ИСКЛЮЧАЮЩЕЕ ИЛИ}} представлена схема персептрона.
	\\Данному персептрону соответствует метод, представленный в базе знаний ostis-системы на описанном в этой главе языке представления нейросетевых методов SCP. Данный метод представлен на рисунке \textit{Рисунок. Метод, решающий задачу \scnqq{ИСКЛЮЧАЮЩЕЕ ИЛИ}, представленный с помощью языка представления нейросетевых методов SCP}.
	\\Описание метода состоит из последовательности двух обобщенных спецификаций действий --- действия вычисления взвешенной суммы всех нейронов слоя и действия вычисления функции активации для всех нейронов слоя.
	\\Сигнальная функция активации, использующаяся в персептроне, в памяти ostis-системы определяется логической формулой, представленной на рисунке \textit{SCg-текст. Представление сигнальной функции активации в памяти ostis-системы}.}



\scnsectionheader{Логико-семантическая модель ostis-системы автоматизации проектирования искусственных нейронных сетей, семантически совместимых с базами знаний ostis-систем}
\begin{scnsubstruct}

\scnheader{Логико-семантическая модель ostis-системы автоматизации проектирования искусственных нейронных сетей}
\begin{scnhaselementrolelist}{класс объектов исследования}
	\scnitem{действие трансляции условия задачи}
	\scnitem{действие классификации задачи}
	\scnitem{действие поиска подходящей обучающей выборки}
	\scnitem{действие формирования требований к обучающей выборке}
	\scnitem{действие очистки выборки}
	\scnitem{действие выявления содержательных признаков}
	\scnitem{действие трансформации выборки}
	\scnitem{действие разбиения выборки}
	\scnitem{действие выбора класса нейросетевых методов}
	\scnitem{действие формирования спецификации входов и выходов и.н.с.}
	\scnitem{действие выбора метода оптимизации}
	\scnitem{действие выбора минимизируемой функции ошибки}
	\scnitem{действие начальной инициализации и.н.с.}
	\scnitem{действие выбора гиперпараметров и.н.с.}
	\scnitem{метод обучения с учителем}
	\scnitem{метод обучения без учителя}
	\scnitem{действие обучения и.н.с.}
\end{scnhaselementrolelist}

\scnheader{Язык представления нейросетевых методов в базах знаний}
\scniselement{язык представления методов}
\scntext{примечание}{Наличия \textit{Языка представления нейросетевых методов в базах знаний} и его интерпретатора позволяет обеспечить интерпретацию \textit{нейросетевого метода} в памяти \textit{ostis-системы}. Наличие в единой памяти не только экземпляров методов, но и понятий, их описывающих, создает основу для автоматизации процесса построения нейросетевых методов. 
	\\В памяти \textit{ostis-системы} хранятся знания о том, методы какого класса могут решить задачу заданного класса, но экземпляров класса этого метода может не быть представлено в системе. На этот случай система должна иметь возможность сообщить пользователю о возможности решения, для которого, однако, необходимо погрузить в систему определенный метод. Так как система хранит в единой памяти задачу и требования к методу ее решения, появляется возможность спроектировать необходимый метод. Для этого необходимо наличие среды проектирования методов соответствующих классов. В случае \textit{нейросетевого метода}, речь идет об интеллектуальной среде построения \textit{нейросетевых методов}.}

\scnheader{нейросетевой метод}
\scntext{примечание}{В основе интеллектуальной среды построения \textit{нейросетевых методов} лежат соответствующие другу другу иерархии действий, задач и методов построения \textit{и.н.с.} Наличие такой иерархии позволит описать язык представления методов построения \textit{и.н.с.} и разработать интерпретатор этого языка.}
\scntext{примечание}{Построение иерархии соответствующих действий построения \textit{и.н.с.} следует начать с изучения этапов проектирования и обучения \textit{и.н.с.}, которые, в общем случае, выполняют все разработчики и.н.с.:
	\\1. Постановка задачи\\
	2. Предобработка выборки: очистка\\
	3. Предобработка выборки: выявление содержательных признаков\\
	4. Предобработка выборки: трансформация\\
	5. Разбиение выборки на обучающую, валидационную и тестовую (контрольную)\\
	6. Выбор класса нейросетевых методов в соответствии со сформулированной задачей\\
	7. Формирование спецификации на входные и выходные данные\\
	8. Выбор метода оптимизации\\
	9. Выбор минимизируемой функции ошибки\\
	10. Начальная инициализация параметров нейронной сети\\
	11. Выбора гиперпараметров и.н.с.\\
	12. Обучение модели на обучающей выборке\\
	13. Оценка эффективности и.н.с}

\scnheader{Постановка задачи}
\scntext{примечание}{Постановка задачи включает в себя описание входных данных (изображения/видео, временные ряды, текст), выходных данных и требований к методу решения (скорость, затраты по памяти и так далее). Также описывается дополнительная информация, которая может помочь в построении метода решения задачи (к примеру, спецификация обучающей выборки, если таковая имеется). Обычно, на данном этапе разработчик и.н.с. определяет класс задачи, формирует требования к обучающей выборке, если она не предоставлена.
	\\Выполнение данного этапа средой проектирования \textit{и.н.с.} подразумевает выполнение следующих действий:
	\begin{itemize}
		\item \textbf{\textit{действие трансляции условия задачи}}. Действие транслирует заданное с помощью \textit{интерфейса ostis-системы} (к примеру, естественно-языкового интерфейса) описание задачи в память ostis-системы. Действие необходимо в случае, когда условие задачи задается пользователем. Необходимо понимать, что описание задачи поступает в базу знаний не только от \textit{пользовательского интерфейса}. К примеру, задача может быть сформулирована самой системой в ходе ее жизнедеятельности.
		Данное действие является общим для всех ostis-систем, поэтому его рассмотрение выходит за рамки рассмотрения процесса построения интеллектуальной среды проектирования \textit{и.н.с.}
		\item \textbf{\textit{действие классификации задачи}}. Действие определяет класс задачи (задача регрессии, детекции, кластеризации и так далее), исходя из описания задачи в базе знаний.
		\item \textbf{\textit{действие поиска подходящей обучающей выборки}}. В базе знаний может храниться набор спецификаций выборок, к которым у ostis-системы есть доступ. Действие производит поиск выборок, которые могут быть использованы в качестве обучающей выборки.
		\item \textbf{\textit{действие формирования требований к обучающей выборке}}. Если обучающая выборка не была предоставлена и не была найдена, то необходимо сформировать описание требований к обучающей выборке, которое можно будет транслировать на язык пользовательского интерфейса и запросить необходимую выборку у пользователя.
	\end{itemize}}

\scnheader{Предобработка выборки}
\scnhaselement{очистка}
\begin{scnindent}
	\scntext{примечание}{На этом этапе обнаруживаются признаки, которые имеют в общем случае некорректные значения (например, для каких-то образов значение признака может иметь неопределенное значение, либо значение, не совпадающее по типу, либо аномально большое или очень маленькое значение, которое встречается в редком числе случаев). Для признаков, имеющих неопределенное значение, может быть применены различные методы устранения, например, такие значения могут быть заменены средним значением этого признака, рассчитанным по всем образам (для непоследовательных данных), либо они могут быть заменены средним значением по соседним образам (в случае временных рядов), либо каким-то фиксированным значением. Радикальная мера решения проблемы --- удаление образов, имеющих неопределенные значения признаков из выборки. Однако его лучше применять, если образов с отсутствующими значениями признаков немного. Для выбросов и аномалий применяются схожие стратегии (но только в том случае, если задача не состоит в прогнозировании этих аномалий).
		\\В интеллектуальной среде проектирования данный этап соответствует выполнению \textbf{\textit{действия очистки выборки}}, которое выполняется в случае обработки выборки, которая ранее не была представлена в памяти системы (к примеру, была получена от пользователя).
		\\Реализация интерпретатора (агента) данного действия требует описания в памяти классификации стратегий очистки данных и реализации методов применения этих стратегий.}
\end{scnindent}
\scnhaselement{выявление содержательных признаков}
\begin{scnindent}
	\scntext{примечание}{Осуществляется инжиниринг признаков, состоящий в отборе признаков, влияющих на результат работы модели, несодержательные признаки, которые никак не коррелируют с выходом модели, удаляются. Цель этого этапа --- уменьшение размерности пространства признаков для снижения влияния эффекта переобучения на модель.
		\\Для снижения размерности признакового пространства может применяться методы отбора признаков и выделения признаков.
		\\При отборе признаков, осуществляется формирование подмножества из исходных признаков (алгоритм последовательного обратного отбора, рекурсивный алгоритм обратного устранения признаков,  алгоритмы с использованием случайных лесов).
		\\При выделении признаков из набора признаков извлекается информация для построения нового подпространства признаков (алгоритмы с использованием автоэнкодера).
		\\В интеллектуальной среде проектирования данный этап соответствует выполнению \textbf{\textit{действия выявления содержательных признаков}}. Реализация интерпретатора (агента) данного действия требует описания в памяти классификации стратегий уменьшения размерности признакового пространства и реализации методов применения этих стратегий.}
\end{scnindent}
\scnhaselement{трансформация}
\begin{scnindent}
	\begin{scnrelfromvector}{примечание}
		\scnfileitem{На этом этапе осуществляется подготовка данных к обучению.}
		\scnfileitem{Здесь следует уделить особое внимание наличию категориальных признаков, чаще всего заданных строковыми типами. Эти признаки могут быть номинальными и порядковыми. Для кодирования порядковых признаков чаще всего применяют последовательный числовой код (1, 2, 3,...). Для кодирования номинальных такое решение неверно, так как эти признаки равноправны и не могут сравниваться по числовому коду (например, пол --- 0/1). Для номинальных признаков применяется способ прямого кодирования, заключающийся в создании и использовании фиктивных признаков по количеству значений исходного. Например, признак пол (мужской, женский) преобразуется в два новых признака мужской и женский с соответствующими значениями для имеющихся образов.}
		\scnfileitem{Масштабирование признаков предполагает приведение значений признаков к одному общему интервалу --- это особенно актуально для признаков, имеющих несоразмерные выборочные средние значения по всем образам --- например, один признак в среднем имеет значение 10.000, а другой 12. Это может проявится в выполнении минимизации только по признаку с наибольшими значениями и плохой сходимости метода обучения. Чаще всего масштабирование соответствует выполнению нормализации на отрезок (min-max нормализация)}
		\begin{scnindent}
			\scnrelfrom{формула}{
				\begin{equation*}
					x\underscore{norm}\upperscore{i} = \frac{x\upperscore{i} - x\underscore{min}}{x\underscore{max} - x\underscore{min}}
				\end{equation*}}
			\begin{scnindent}
				\scntext{примечание}{$x\upperscore{i}$ --- значение признака для отдельно взятого образа \textit{i}, $x\underscore{min}$ --- наименьшее значение для признака, $x\underscore{max}$ --- наибольшее значение для признака.}
			\end{scnindent}
		\end{scnindent}
		\scnfileitem{Другой вариант масштабирования --- применение стандартизации признаков}
		\begin{scnindent}
			\scnrelfrom{формула}{
				\begin{equation*}
					x\underscore{std}\upperscore{i} = \frac{x\upperscore{i} - \mu(x)}{\sigma(x)}
				\end{equation*}}
			\begin{scnindent}
				\scntext{примечание}{$\mu(x)$ --- выборочное среднее отдельного признака, $\sigma(x)$ --- стандартное отклонение.}
			\end{scnindent}
		\end{scnindent}
		\scnfileitem{Стандартизация сохраняет полезную информацию о выбросах в исходных данных и делает алгоритм обучения менее чувствительным к ним.}
		\scnfileitem{Дискретизация применяется для перехода от вещественного признака к порядковому за счет кодирования интервалов одним значением (например, если признак отражает возраст человека, то может быть произведена дискретизация значений с выделением определенных возрастных групп, где каждая группа будет кодироваться одним целым числом).}
		\scnfileitem{В интеллектуальной среде проектирования данный этап соответствует выполнению \textbf{\textit{действия трансформации выборки}}. Реализация интерпретатора (агента) данного действия требует описания в памяти классификации методов масштабирования признаков и реализации методов применения этих стратегий.}
	\end{scnrelfromvector}
\end{scnindent}

\scnheader{Разбиение выборки на обучающую, валидационную и тестовую (контрольную)}
\scntext{примечание}{Производится разбиение всей выборки данных, на обучающую, тестовую и, в некоторых случаях, валидационную.
	\\Валидационная выборка используется для оценки влияния изменения гиперпараметров на результат обучения и может применяться как дополнительный инструмент для этого наравне с сеточным поиском.
	\\Разбиение проводится в соотношении 3:1:1, в процентах (60/20/20), если валидационная выборка не используется, то 80/20.
	\\В интеллектуальной среде проектирования данный этап соответствует выполнению \textbf{\textit{действия разбиения выборки}}.
	\\Все предыдущие этапы применялись к выборке, последующие этапы относятся к используемым моделям и.н.с.}

\scnheader{Выбор класса нейросетевых методов в соответствии со сформулированной задачей}
\scntext{примечание}{На этом этапе осуществляется выбор основной архитектуры и.н.с., которая будет использоваться при обучении. Однако, нужно отметить, что этот выбор относительно условный, то есть исследователь не ограничен использованием только одного типа и.н.с. для решения задачи (как, например, сверточной сети для изображений, поскольку изображения можно обрабатывать и обычным многослойным персептроном). Речь скорее идет именно о рекомендованной архитектуре, но это не исключает использование любых других вариантов архитектур и их сочетаний в рамках одной модели).
	\\Примерами таких рекомендаций являются:
	\begin{itemize}
		\item изображения/видео --- сверхточные нейронные сети;
		\item временные ряды --- многослойные персептроны или рекуррентные сети;
		\item текстовая информация --- многослойные персептроны или рекуррентные сети;
		\item наборы характеристик некоторых объектов (например, спецификации автомобилей) --- многослойный персептрон.
	\end{itemize}
	В интеллектуальной среде проектирования данный этап соответствует выполнению \textbf{\textit{действия выбора класса нейросетевых методов}}.}

\scnheader{Формирование спецификации на входные и выходные данные}
\scntext{примечание}{Выполняются дополнительные преобразования данных, связанные с изменением структур хранения (например, преобразование многомерного массива в одномерный, конвертация типов)
	\\В интеллектуальной среде проектирования данный этап соответствует выполнению \textbf{\textit{действия формирования спецификации входов и выходов и.н.с.}}.}

\scnheader{Выбор метода оптимизации}
\scntext{примечание}{В рамках ПрО и.н.с. описаны следующие методы оптимизации:
	\begin{itemize}
		\item стохастический градиентный спуск (stochastic gradient descent --- SGD);
		\item метод Нестерова;
		\item адаптивный градиент (adaptive gradient --- AdaGrad);
		\item адаптивная оценка момента (adaptive moment estimation --- Adam);
		\item среднеквадратическое распространение (root mean square propagation --- RMSProp).
	\end{itemize}
	В интеллектуальной среде проектирования данный этап соответствует выполнению \textbf{\textit{действия выбора метода оптимизации}}.}

\scnheader{Выбор минимизируемой функции ошибки}
\begin{scnrelfromvector}{примечание}
	\scnfileitem{На этом этапе задается функция ошибок, которая будет минимизироваться. К примеру, MSE лучше подходит для задач регрессии и для кластеризации, CE --- для классификационных задач. Далее приведены примеры.}
	\scnfileitem{
		\begin{equation*}
			MSE = \frac{1}{n} \sum\underscore{i=1}\upperscore{n} (Y\underscore{i} - \widetilde{Y\underscore{i}})\upperscore{2}
		\end{equation*}}
		\begin{scnindent}
			\scntext{примечание}{\textit{n} --- размер обучающей выборки, $Y\underscore{i}$ --- эталонное значение функции, $\widetilde{Y\underscore{i}}$ --- результат, полученный НС}
		\end{scnindent}
	\scnfileitem{
		\begin{equation*}
		CE = - \frac{1}{n} \sum\underscore{i=1}\upperscore{n} (Y\underscore{i}\log(\widetilde{Y\underscore{i}}) + (1-Y\underscore{i})\log(1 - \widetilde{Y\underscore{i}}))
		\end{equation*}}
		\begin{scnindent}
			\scntext{примечание}{\textit{n} --- размер обучающей выборки, $Y\underscore{i}$ --- эталонное значение функции, $\widetilde{Y\underscore{i}}$ --- результат, полученный НС.
				\\(случай 2-классовой классификации)}
		\end{scnindent}
	\scnfileitem{
		\begin{equation*}
			CE = - \frac{1}{n} \sum\underscore{i=1}\upperscore{n} \sum\underscore{c=1}\upperscore{M} Y\underscore{i}\upperscore{c} \log{\widetilde{Y}\underscore{i}\upperscore{c}}
		\end{equation*}}
		\begin{scnindent}
			\scntext{примечание}{случай многоклассовой классификации}
		\end{scnindent}
	\scnfileitem{В интеллектуальной среде проектирования данный этап соответствует выполнению \textbf{\textit{действия выбора минимизируемой функции ошибки}}.}
\end{scnrelfromvector}
	
\scnheader{Начальная инициализация параметров нейронной сети}
\scntext{пример}{Наиболее часто используемые варианты инициализации весовых коэффициентов и порогов нейронной сети}
\begin{scnindent}
	\begin{scnrelfromset}{разбиение}
		\scnfileitem{Инициализация значениями из равномерного распределения на каком-то небольшом интервале, например,  $\scnleftsquarebrace$ -0.1, 0.1$\scnrightsquarebrace$.}
		\scnfileitem{Инициализация значениями из стандартного нормального распределения.}
		\scnfileitem{Инициализация по методу Ксавье.}
		\begin{scnindent}
			\begin{scnrelfromset}{источник}
				\scnitem{\scncite{Glorot2010}}
			\end{scnrelfromset}
			\scntext{примечание}{Применяется для предотвращения резкого уменьшения или увеличения значений выхода нейронных элементов после применения функции активации при прямом прохождении образа через глубокую нейронную сеть. Фактически инициализация этим методом осуществляется посредством выбора значений из равномерного распределения на отрезке $\scnleftsquarebrace - \sqrt{6} / \sqrt{n\underscore{i}+n\underscore{i+1}}, \sqrt{6} / \sqrt{n\underscore{i}+n\underscore{i+1}}\scnrightsquarebrace $, где $n\underscore{i}$ --- это число входящих связей в данный слой, а $n\underscore{i}$ --- число исходящих связей из данного слоя. Таким образом, инициализация этим методом проводится для разных слоев нейронной сети из разных отрезков.}
		\end{scnindent}
		\scnfileitem{Инициализация, полученная из предобученной модели.}
		\begin{scnindent}
			\begin{scnrelfromset}{источник}
				\scnitem{\scncite{Glorot2010}}
			\end{scnrelfromset}
			\scntext{примечание}{Вариант инициализации, который предполагает использование в качестве \scnqq{стартовой} модели предобученной модели, взятой из некоторого репозитория предобученных моделей, обученную самим исследователем или в процессе работы интеллектуальной системы.}
		\end{scnindent}
		\scnfileitem{Инициализация по методу Кайминга.}
		\begin{scnindent}
			\begin{scnrelfromset}{источник}
				\scnitem{\scncite{He2015}}
			\end{scnrelfromset}
			\scntext{примечание}{Данный метод инициализации применяется для решения проблемы \scnqq{затухающего} градиента и \scnqq{взрывающегося} градиента. Производится посредством выбора значений из равномерного распределения на отрезке: 
				\begin{equation*}
					\scnleftsquarebrace -\sqrt{2} / \sqrt{(1+a\upperscore{2})fan}, \sqrt{2} / \sqrt{(1+a\upperscore{2})fan}\scnrightsquarebrace
				\end{equation*}
				где \textit{a} --- угол наклона к оси абсцисс для отрицательной части области определения функции активации типа ReLU (для обычной ReLU функции этот параметр равен 0), $fan$ --- параметр режима работы, который для фазы прямого распространения равен количеству входящих связей (для устранения эффекта \scnqq{взрывающегося} градиента), а для фазы обратного распространения --- количеству выходящих (для устранения эффекта \scnqq{затухающего} градиента).
				\\В интеллектуальной среде проектирования данный этап соответствует выполнению \textbf{\textit{действия начальной инициализации и.н.с.}}.}
		\end{scnindent}
	\end{scnrelfromset}
\end{scnindent}

\scnheader{Выбора гиперпараметров и.н.с.}
\scntext{примечание}{На практике некоторые гиперпараметры (такие как количество слоев, их типы, количество нейронов в слое) часто определяются экспериментально, в процессе итеративного поиска лучшего варианта решения задачи. Хотя способы частично автоматизировать этот процесс существуют, они все же рассчитаны на наличие некоторых предусловий проведения эксперимента, в частности интервалов изменения параметра (например, скорости обучения).
	\\К гиперпараметрам, подбираемым на этом этапе, относятся:
	\begin{itemize}
		\item параметры обучения \textit{и.н.с.} (скорость обучения, моментный параметр, размер мини-батча);
		\item архитектура модели \textit{и.н.с.}, опирающаяся на ранее сформулированные спецификации входных и выходных данных (например, количество нейронов в определенном слое (слоях) или конфигурации целых слоев).
	\end{itemize}
	Нахождение оптимальных гиперпараметров может быть получено, например, использованием метода сеточного поиска, который позволяет проверить значения гиперпараметров, взятые с определенным шагом или из определенного интервала (кортежа). С помощью этого метода выбирается оптимальный набор гиперпараметров, который дает лучшие результаты, он используется для последующего дообучения. Или же, если полученные результаты являются приемлемыми, то процесс дальнейшего обучения вообще не проводится. Следует отметить затратность данного метода, так как фактически осуществляется перебор различных значений параметров обучения. Для снижения объема работы применяется метод случайного поиска.
	\\Для оптимизации архитектуры определяются типы слоев нейронной сети, количество нейронных элементов в каждом слое, их характеристики --- функция активации, для сверточных элементов --- размер ядра, а также параметры padding и шаг свертки (stride).
	\\Здесь же может осуществляться оценка не только пользовательского варианта сети, но и предобученной архитектуры. Основное правило при выборе --- количество параметров модели не должно превышать размер обучающей выборки. Для предобученных архитектур это ограничение снимается.
	\\В интеллектуальной среде проектирования данный этап соответствует выполнению \textbf{\textit{действия выбора гипперпараметров и.н.с.}}. Действие использует классификацию и спецификации гиперпараметров \textit{и.н.с.}}

\scnheader{Обучение модели на обучающей выборке}
\scntext{примечание}{Производится обучение модели до достижения выбранной точности (оценивается на тестовой выборке) или по другим заданным критериям (достижение заданного количества эпох обучения, неизменность точности на протяжении заданного количества эпох, падение точности на валидационной выборке и так далее).}
\scntext{примечание}{В интеллектуальной среде проектирования данный этап соответствует выполнению \textbf{\textit{действия обучения и.н.с.}}. Действие обучения \textit{и.н.с.} --- действие, в ходе которого реализуется определенный метод обучения \textit{и.н.с.} с заданными параметрами обучения \textit{и.н.с.}, методом оптимизации и функцией потерь.
	\\При обучении возможно возникновение следующих проблем:
	\begin{itemize}
		\item \textit{переобучение} --- проблема, возникающая при обучении \textit{и.н.с.}, заключающаяся в том,
		что сеть хорошо адаптируется к паттернам входной активности из обучающей выборки, при этом теряя способность к обобщению.
		Переобучение возникает из-за применения неоправданно сложной модели при обучении \textit{и.н.с.} Это происходит,
		когда количество настраиваемых параметров \textit{и.н.с.} намного больше размера обучающей выборки. Возможные
		варианты решения проблемы заключаются в упрощении модели, увеличении выборки, использовании регуляризации
		(параметр регуляризации, техника dropout и так далее).\\
		Обнаружение переобученности сложнее, чем недообученности. Как правило, для этого применяется
		кросс-валидация на валидационной выборке, позволяющая оценить момент завершения процесса обучения.
		Идеальным вариантом является достижение баланса между переобученностью и недообученностью.
		\item \textit{недообучение} --- проблема, возникающая при обучении  \textit{и.н.с.}, заключающаяся в том,
		что сеть дает одинаково плохие результаты на обучающей и контрольной выборках.
		Чаще всего такого рода проблема возникает при недостаточном времени, затраченном на обучение модели.
		Однако это может быть вызвано и слишком простой архитектурой модели либо малым размером обучающей
		выборки. Соответственно решение, которое может быть принято ML-инженером, заключается в устранении
		этих недостатков: увеличение времени обучения, использование модели с большим числом настраиваемых
		параметров, увеличение размера обучающей выборки, а также уменьшение регуляризации и более тщательный
		отбор признаков для обучающих примеров.
	\end{itemize}}

\scnheader{метод обучения и.н.с.}
\scnsubset{метод}
\scnrelfrom{разбиение}{Классификация алгоритмов обучения}
\begin{scnindent}
	\begin{scneqtoset}
		\scnitem{метод обучения с учителем}
		\begin{scnindent}
			\scntext{пояснение}{\textbf{\textit{метод обучения с учителем}} --- метод обучения с использованием заданных целевых переменных.}
			\scnsuperset{метод обратного распространения ошибки}
			\begin{scnindent}
				\scnidtf{м.о.р.о.}
				\scntext{пояснение}{м.о.р.о. использует заданный метод оптимизации и заданную функцию потерь для реализации фазы обратного распространения ошибки и изменения настраиваемых параметров и.н.с. Одним из самых распространенных	методов оптимизации является метод стохастического градиентного спуска.}
				\scntext{пояснение}{Следует также отметить, что несмотря на то, что метод отнесен к методам обучения с учителем, в случае	использования м.о.р.о. для обучения автокодировщиков в классических публикациях он рассматривается как	метод обучения без учителя, поскольку в данном случае размеченные данные отсутствуют.}
			\end{scnindent}
		\end{scnindent}
		\scnitem{метод обучения без учителя}
		\begin{scnindent}
			\scntext{пояснение}{\textbf{\textit{метод обучения без учителя}} --- метод обучения без использования заданных целевых переменных(в режиме самоорганизации)}
			\scntext{пояснение}{В ходе выполнения алгоритма метода обучения без учителя выявляются полезные структурные свойства набора. Неформально его понимают как метод для извлечения информации из распределения, выборка для которого	не была вручную аннотирована человеком.}
			\begin{scnindent}
				\begin{scnrelfromset}{источник}
					\scnitem{\scncite{Goodfellow2017}}
				\end{scnrelfromset}
			\end{scnindent}
		\end{scnindent}
	\end{scneqtoset}
\end{scnindent}

\scnheader{метод обучения \textit{и.н.с.}}
\scntext{определение}{метод обучения \textit{и.н.с.} --- это процесс итеративного поиска оптимальных значений настраиваемых параметров \textit{и.н.с.}, минимизирующих некоторую заданную функцию потерь.}
\scntext{примечание}{Стоит отметить, что хотя целью применения метода обучения является минимизация функции потерь, \scnqq{полезность} полученной после обучения модели можно оценить только по достигнутому уровню ее обобщающей способности.}
\scntext{примечание}{\\Методы обучения могут быть поделены на две большие группы --- \textit{\textbf{методы обучения с учителем}} и \textit{\textbf{методы обучения без учителя}} (контролируемый и неконтролируемый методы обучения).
	\\\textit{метод обучения с учителем} --- метод обучения с использованием заданных целевых переменных.
	\\Одним из методов обучения с учителем является метод обратного распространения ошибки.}

\scnheader{метод обратного распространения ошибки}
\scntext{описание}{Приведем его описание в виде алгоритма:
	\begin{algorithm}[H]
		\KwData{$X$ --- данные, $E\underscore{t}$ --- желаемый отклик (метки), $E\underscore{m}$ --- желаемая ошибка (в соответствии с выбранной функцией потерь)}
		\KwResult{обученная нейронная сеть \textit{Net}}
		инициализация весов \textit{W} и порогов \textit{T};\\
		\Repeat{$E<E\underscore{m}$}{
			\ForEach{$x \in X$, $e \in E\underscore{t}$}{
				фаза прямого распространения сигнала: вычисляются активации для всех слоев и.н.с.;\\
				фаза обратного распространения ошибки: вычисляются ошибки для последнего слоя и всех предшествующих слоев;\\
				изменение настраиваемых параметров и.н.с. в соответствии с вычисленными ошибками;\\
			}
			вычисление общей ошибки E на данной эпохе;
		}
	\end{algorithm}
	\textit{метод обратного распространения ошибки} использует заданный метод оптимизации и заданную функцию потерь для реализации фазы обратного распространения ошибки и изменения настраиваемых параметров и.н.с. Одним из самых распространенных методов оптимизации является метод стохастического градиентного спуска. Приведенный метод используется для реализации последовательного варианта обучения.}
\scntext{примечание}{Следует также отметить, что несмотря на то, что метод отнесен к методам обучения с учителем, в случае его использования для обучения автокодировщиков в классических публикациях он рассматривается как метод обучения без учителя, поскольку в данном случае размеченные данные отсутствуют.}


\scnheader{метод обучения без учителя}
\scniselement{метод обучения}
\scntext{определение}{метод обучения без учителя --- это метод обучения без использования заданных целевых переменных (в режиме самоорганизации)}
\scntext{примечание}{В ходе выполнения алгоритма метода обучения без учителя выявляются полезные структурные свойства набора. Неформально его понимают как метод для извлечения информации из распределения, выборка для которого не была вручную аннотирована человеком. Метод обучения без учителя может рассматриваться как вспомогательный метод для начальной инициализации настраиваемых параметров и.н.с. В этом случае он является методом предобучения.}
\begin{scnindent}
	\begin{scnrelfromset}{источник}
		\scnitem{\scncite{Goodfellow2017}}
	\end{scnrelfromset}
\end{scnindent}

\scnheader{целевая функция}
\begin{scnrelfromvector}{методы оптимизации}
		\scnfileitem{SGD (стохастический градиентный спуск). В данном методе корректировка настраиваемых параметров и.н.с. выполняется в направлении максимального уменьшения функции стоимости, то есть в направлении, противоположном вектору градиента функции потерь.}
		\begin{scnindent}
			\begin{scnrelfromset}{источник}
				\scnitem{\scncite{Golovko2017}}
				\scnitem{\scncite{Haykin2006}}
			\end{scnrelfromset}
		\end{scnindent}
		\scnfileitem{Метод Нестерова. Обучение методом стохастического градиентного спуска не редко происходит очень медленно. Импульсный метод позволяет ускорить обучение, особенно в условиях высокой кривизны, небольших, но устойчивых градиентов или зашумленных градиентов. В импульсном методе вычисляется экспоненциально затухающее скользящее среднее прошлых градиентов и продолжается движение в этом направлении. Метод Нестерова является вариантом импульсного алгоритма, в котором градиент вычисляется после применения текущей скорости.}
		\begin{scnindent}
			\begin{scnrelfromset}{источник}
				\scnitem{\scncite{Goodfellow2017}}
			\end{scnrelfromset}
		\end{scnindent}
		\scnfileitem{AdaGrad: данный метод по отдельности адаптирует скорости обучения всех настраиваемых параметров и.н.с., умножая их на коэффициент, обратно пропорциональный квадратному корню из суммы всех прошлых значений квадрата градиента.}
		\begin{scnindent}
			\begin{scnrelfromset}{источник}
				\scnitem{\scncite{Duchi2011}}
			\end{scnrelfromset}
		\end{scnindent}
		\scnfileitem{RMSProp. Данный метод является модификацией AdaGrad, которая позволяет улучшить его поведение в невыпуклом случае путем изменения способа агрегирования градиента на экспоненциально взвешенное скользящее среднее. Использование экспоненциально взвешенного скользящего среднего гарантирует повышение скорости сходимости после обнаружения выпуклой впадины, как если бы внутри этой впадины алгоритм AdaGrad был инициализирован заново}
		\begin{scnindent}
			\begin{scnrelfromset}{источник}
				\scnitem{\scncite{Goodfellow2017}}
			\end{scnrelfromset}
		\end{scnindent}
		\scnfileitem{Adam. Данный метод можно рассматривать как комбинацию RMSProp и AdaGrad. Помимо усредненного первого момента, данный метод использует усредненное значение вторых моментов градиентов.}
		\begin{scnindent}
			\begin{scnrelfromset}{источник}
				\scnitem{\scncite{Kingma2014}}
			\end{scnrelfromset}
		\end{scnindent}
\end{scnrelfromvector}
\scntext{примечание}{Отметим, что успешность применения методов оптимизации зависит главным образом от знакомства пользователя с соответствующим алгоритмом.}
\begin{scnindent}
	\begin{scnrelfromset}{источник}
		\scnitem{\scncite{Goodfellow2017}}
	\end{scnrelfromset}
\end{scnindent}

\scnheader{функция потерь}
\scntext{примечание}{важный компонент, влияющий на процесс обучения нейросетевой модели}
\scntext{определение}{функция потерь --- это функция, используемая для вычисления ошибки, рассчитываемой как разница между фактическим эталонным значением и прогнозируемым значением, получаемым \textit{и.н.с.}}
\begin{scnrelfromvector}{примечание}
	\scnfileitem{Среди функций потерь, используемые в качестве целевых функций для применяемого метода оптимизации, можно выделить MSE, BCE, MCE}
	\scnfileitem{MSE --- средняя квадратичная ошибка}
	\begin{scnindent}
		\scnrelfrom{формула}{
			\begin{equation*}
				MSE = \frac{1}{L} \sum\underscore{l=1}\upperscore{L} \sum\underscore{i=1}\upperscore{m} (y\underscore{i}\upperscore{l} - e\underscore{i}\upperscore{l})\upperscore{2}
			\end{equation*}}
		\begin{scnindent}
			\scntext{примечание}{$y\underscore{i}\upperscore{l}$ --- прогноз модели, $e\underscore{i}\upperscore{l}$ --- ожидаемый (эталонный) результат, \textit{m} --- размерность выходного вектора, \textit{L} --- объем обучающей выборки}
		\end{scnindent}
	\end{scnindent}
	\scnfileitem{BCE --- бинарная кросс-энтропия (binary cross-entropy)}
	\begin{scnindent}
		\scnrelfrom{формула}{
			\begin{equation*}
				BCE = - \sum\underscore{l=1}\upperscore{L} (e\upperscore{l} \log(y\upperscore{l}) + (1 - e\upperscore{l})\log(1 - y\upperscore{l}))
			\end{equation*}}
		\begin{scnindent}
			\scntext{примечание}{$y\upperscore{l}$ --- прогноз модели, $e\upperscore{l}$ --- ожидаемый (эталонный) результат: \textit{0} или \textit{1}, \textit{L} --- объем обучающей выборки}
		\end{scnindent}
	\end{scnindent}
	\scnfileitem{MCE --- мультиклассовая кросс-энтропия (multiclass cross-entropy)}
	\begin{scnindent}
		\scnrelfrom{формула}{
			\begin{equation*}
				MCE = - \sum\underscore{l=1}\upperscore{L} \sum\underscore{i=1}\upperscore{m} e\underscore{i}\upperscore{l \log(y\underscore{i}\upperscore{l})}
			\end{equation*}}
		\begin{scnindent}
			\scntext{примечание}{$y\underscore{i}\upperscore{l}$ --- прогноз модели, $e\underscore{i}\upperscore{l}$ --- ожидаемый (эталонный результат), \textit{m} --- размерность выходного вектора}
		\end{scnindent}
	\end{scnindent}
\end{scnrelfromvector}

\scnheader{бинарная кросс-энтропия}
\scnidtf{BCE}
\scntext{примечание}{Отметим, что для бинарной кросс-энтропии в выходном слое \textit{и.н.с.} будет находиться один нейрон, а для для мультиклассовой кросс-энтропии количество нейронов в выходном \textit{слое и.н.с.} совпадает с количеством классов.}

\scnheader{задача классификации}
\scntext{примечание}{Для решения задачи классификации рекомендуется использовать бинарную или мультиклассовую кросс-энтропийную функцию потерь, для решения задачи регрессии рекомендуется использовать среднюю квадратичную ошибку.}

\scnheader{SCg-текст. Действие обучения и.н.с.}
\scnrelfrom{описание примера}{\scnfileimage[30em]{Contents/part_ps/src/images/sd_ps/sd_ann/ann_training_nn_scg.png}}
\begin{scnindent}
	\scntext{примечание}{Пример действия обучения \textit{и.н.с.}}
\end{scnindent}

\scnheader{Оценка эффективности и.н.с}
\scntext{примечание}{После выполнения обучения осуществляется оценка полученной модели с помощью метрик оценки качества.
	\\Далее результат оценки может быть визуализирован с помощью матрицы ошибок (confusion matrix) и ROC-кривой.}
\scntext{примечание}{В интеллектуальной среде проектирования данный этап соответствует выполнению \textit{действия оценки эффективности и.н.с.}.}

\scnheader{матрица ошибок}
\scntext{определение}{матрица ошибок --- это матрица, в которую помещены сведения о числе истинно-положительных, истинно-отрицательных, ложно-положительных и ложно-отрицательных предсказаниях классификатора.}
\scnrelfrom{смотрите}{Рисунок. Матрица ошибок}

\scnheader{Рисунок. Матрица ошибок}
\scnrelfrom{описание примера}{\scnfileimage[30em]{Contents/part_ps/src/images/sd_ps/sd_ann/conf_matrix.png}}

\scnheader{ROC-кривая}
\scnidtf{receiver operating characteristic}
\scntext{определение}{ROC-кривая --- это график, в котором, основываясь на заданном пороге решения классификатора, рассчитываются доли ложноположительных и истинно положительных исходов. Основываясь на ROC-кривой, высчитывается AUC-показатель (площадь под кривой), которая используется в качестве характеристики качества модели.}

\scnheader{Задача. Классификация цифр из выборки рукописных цифр MNIST}
\begin{scnrelfromvector}{пример}
	\scnfileitem{Рассмотрим пример выполнения описанных этапов разработчиком для конкретной задачи --- \textit{классификации цифр из выборки рукописных цифр MNIST}:
		\begin{itemize}
		\item Исходными данными задачи является: выборка из 70.000 изображений, предварительно разделенная на обучающую (60.000 изображений) и контрольную (10.000 изображений) выборки. Каждое изображение представлено двумерным массивом 28x28 чисел из интервала $\scnleftsquarebrace  0, 255\scnrightsquarebrace$, числа представляют определенный оттенок серого цвета. Помимо этого каждому изображению соответствует метка класса, соответствующая конкретной цифре от 0 до 9.
		\\Ставится задача: \textit{обучить модель, которая будет принимать на вход двумерный массив данных и возвращать метку класса, соответствующей распознанной цифре.}
		\\Таким образом, тип решаемой задачи --- \textbf{классификационная}, природа данных задачи --- \textbf{изображения}.
		\item В рассматриваемой выборке отсутствуют аномалии, ошибочные данные, признаки с отсутствующими значениями.
		\item В рассматриваемой задаче отсутствуют несодержательные признаки.
		\item В качестве метода предобработки данных используем масштабирование признаков, а именно нормализацию на отрезок $\scnleftsquarebrace 0, 1\scnrightsquarebrace$.
		\item Выполним разбиение обучающей части данных на обучающую и валидационную выборки в соотношении 4:1 (48.000 в обучающей и 12.000 в валидационной).
		\item Так как выборка включает в себя изображения, будем использовать сверточную нейронную сеть.
		\item Не требуется.
		\item В качестве оптимизационного алгоритма будем использовать метод стохастического градиентного спуска (SGD).
		\item Так как решается задача классификации, выберем в качестве минимизируемой функции кросс-энтропийную функцию потерь.
		\item В качестве начальной инициализации будем использовать инициализацию по методу Кайминга.
		\item На предыдущих этапах было определено, что для решения задачи будет использоваться сверточная нейронная сеть. При использовании one-hot кодирования в последнем полносвязном слое будет 10 нейронов по числу классов в задаче.
		\end{itemize}}
	\scnfileitem{Для упрощения будем использовать архитектуру, изображенную на \textit{Рисунок. Архитектура и.н.с., решающая задачу классификации цифр}, не содержащую промежуточные слои.}
	\scnfileitem{Для нахождения оптимального набора гиперпараметров будем применять метод случайного поиска.
		\\Перечислим кортежи, из которых будут сэмплироваться гиперпараметры:
		\begin{itemize}
			\item Скорость обучения --- (0.9, 0.1, 0.01, 0.001);
			\item Количество нейронов в сверточном слое --- (5, 10, 15, 20);
			\item Размер ядра свертки --- (3, 5, 7, 9);
			\item Моментный параметр --- (0, 0.5, 0.9);
			\item Размер мини-батча --- (16, 32, 64, 128).
		\end{itemize}}
	\scnfileitem{После определения данных параметров и оценки эффективности работы алгоритма, получим следующую таблицу: \textit{Таблица. Результаты решения задачи}}
	\scnfileitem{Можно заметить, что лучший результат (acc = 0.9839) по обобщающей способности на валидационной выборке был получен при следующих параметрах: mbs = 64, ks = 7, lr = 0.01, momentum = 0.9, cnc = 15.
		\begin{itemize}
		\item В качестве критерия останова нами был выбран самый простой критерий по достижению заданного количества эпох обучения. Дообучение не проводилось, для оценки обобщающей способности использовалась модель, полученная после выполнения процедуры подбора гиперпараметров. Обобщающая способность на тестовой выборке составила \textbf{0.9853}, то есть \textbf{98.53\percent}.
		\item Построив матрицу ошибок на основании обученной модели и тестовой выборки, получим результат, проиллюстрированный на рис. \textit{Рисунок. Матрица ошибок для задачи MNIST}
		\end{itemize}
		Мы получили матрицу с явно выраженным диагональным преобладанием, таким образом полученная модель делает относительно небольшое число ошибок.}
\end{scnrelfromvector}

\scnheader{Рисунок. Архитектура и.н.с., решающая задачу классификации цифр}
\scnrelfrom{описание примера}{\scnfileimage[20em]{Contents/part_ps/src/images/sd_ps/sd_ann/model.png}}

\scnheader{Таблица. Результаты решения задачи}
\scnrelfrom{описание примера}{\scnfileimage[20em]{Contents/part_ps/src/images/sd_ps/sd_ann/results_table.png}}
\begin{scnindent}
	\scntext{пояснение}{Используемые сокращения: mbs --- mini-batch size, ks --- kernel size, lr --- learning rate, cnc --- convolutional neurons count, acc --- accuracy, it --- iterations count}
\end{scnindent}

\scnheader{Рисунок. Матрица ошибок для задачи MNIST}
\scnrelfrom{описание примера}{\scnfileimage[30em]{Contents/part_ps/src/images/sd_ps/sd_ann/conf_matrix_result}}

\scnheader{действие по построению и.н.с.}
\scntext{примечание}{Исходя из анализа этапов построения и.н.с., которые выполняют разработчики, можно вывести следующую классификацию действий по построению и.н.с.}
\begin{scnrelfromset}{декомпозиция}
	\scnitem{действие по обработке выборки}
	\begin{scnindent}
		\begin{scnrelfromset}{декомпозиция}
			\scnitem{действие поиска подходящей обучающей выборки}
			\scnitem{действие формирования требований к обучающей выборке}
			\scnitem{действие очистки выборки}
			\scnitem{действие выявления содержательных признаков}
			\scnitem{действие трансформации выборки}
			\scnitem{действие разбиения выборки}
		\end{scnrelfromset}
	\end{scnindent}
	\scnitem{действие по проектированию и.н.с.}
	\begin{scnindent}
		\begin{scnrelfromset}{декомпозиция}
			\scnitem{действие выбора класса нейросетевых методов}
			\scnitem{действие формирования спецификации входов и выходов и.н.с.}
		\end{scnrelfromset}
	\end{scnindent}
	\scnitem{действие обучения и.н.с.}
	\begin{scnindent}
		\begin{scnrelfromset}{декомпозиция}
			\scnitem{действие выбора метода оптимизации}
			\scnitem{действие выбора минимизируемой функции ошибки}
			\scnitem{действие начальной инициализации и.н.с.}
			\scnitem{действие выбора гиперпараметров и.н.с.}
			\scnitem{действие обучения и.н.с.}
			\scnitem{действие оценки эффективности и.н.с.}
		\end{scnrelfromset}
	\end{scnindent}
\end{scnrelfromset}

\scnheader{темпоральность нейронной сети}
\scntext{примечание}{Так как в результате действий по построению \textit{и.н.с.} объект этих действий, конкретная \textit{и.н.с.}, может существенно меняться (меняется конфигурация сети, ее весовые коэффициенты), то \textit{и.н.с.} представляется в базе знаний как темпоральное объединение всех ее версий. Каждая версия является \textit{и.н.с.} и темпоральной сущностью. На множестве этих темпоральных сущностей задается темпоральная последовательность с указанием первой и последней версии. Для каждой версии описываются специфичные знания.
	\\Общие для всех версий знания описываются для \textit{и.н.с.}, являющейся темпоральным объединением всех версий (рисунок \textit{SCg-текст. Темпоральность нейронной сети})}

\scnheader{SCg-текст. Темпоральность нейронной сети}
\scnrelfrom{описание примера}{\scnfileimage[20em]{Contents/part_ps/src/images/sd_ps/sd_ann/temporal_neural_network_scg.png}}

\end{scnsubstruct}

\begin{scnrelfromvector}{заключение}
	\scnfileitem{В главе описан подход к \textit{интеграции и конвергенции искусственных нейронных сетей с базами знаний} в \textit{интеллектуальных компьютерных системах нового поколения} с помощью представления и интерпретации \textit{искусственной нейронной сети} в \textit{базе знаний}.}
	\scnfileitem{Описаны \textit{Синтаксис, Денотационная и Операционная семантика Языка представления нейросетевых методов в базах знаний}, который позволяет представить и интерпретировать в памяти интеллектуальной системы любую \textit{и.н.с.} Наличие такого языка порождает семантическую совместимость нейросетевого метода с другими методами, представленными в памяти системы, что позволяет анализировать саму \textit{и.н.с.} и этапы ее работы любыми другими методами системы.}
	\scnfileitem{Так же наличие языка представления нейросетевых методов позволяет описывать в памяти системы экспертные знания разработчиков \textit{и.н.с.} В главе приведены этапы построения \textit{и.н.с.}, которые выполняют разработчики \textit{и.н.с.} На основании этих этапов, c целью проектирования интеллектуальной среды построения \textit{нейросетевых методов}, в \textit{базе знаний} были классифицированы и описаны действия по построению \textit{и.н.с.}}
	\scnfileitem{Проектирования и реализация интеллектуальной среды построения \textit{и.н.с.} в \textit{базе знаний} системы является одним из двух основных направлений дальнейшего развития работу по конвергенции и интеграции и.н.с. с базами знаний.}
	\scnfileitem{Вторым основным направлением является разработка подхода к обработке фрагментов \textit{базы знаний} с помощью \textit{и.н.с.}, для чего необходимо разработать универсальный алгоритм взаимно-однозначного соответствия фрагментов базы знаний и входных векторов \textit{и.н.с.} Язык представления знаний способен представить любое знание. Наличие в системе нейросетевого метода, способного принимать на вход фрагменты знаний, позволит решить новые, слабо изученные классы задач.}
\end{scnrelfromvector}


\bigskip
\scnendcurrentsectioncomment
\end{scnsubstruct}
\end{SCn}

