\begin{SCn}
	\scnsectionheader{Предметная область и онтология многократно используемых компонентов решателей задач ostis-систем}
	
	\begin{scnrelfromvector}{введение}
		\scnfileitem{\textbf{\textit{Библиотека многократно используемых компонентов решателей задач в составе Метасистемы OSTIS}} является важнейшим фрагментом Метасистемы OSTIS, обеспечивающим надежность проектируемых решателей задач и повышение скорости их разработки.}
	\end{scnrelfromvector}
\scntext{заключение}{В данной предметной области рассмотрены многократно используемые компоненты \textit{решателей задач ostis-систем} и соответствующая им библиотека.}

	\begin{scnrelfromlist}{ключевое понятие}
		\scnitem{решатель задач ostis-системы}
		\scnitem{библиотека многократно используемых компонентов решателей задач}
		\scnitem{многократно используемый компонент решателей задач}
		\scnitem{метод}
		\scnitem{Средства автоматизации библиотеки многократно используемых компонентов решателей задач}
		\scnitem{отношение, специфицирующее многократно используемый компонент решателей задач ostis-систем}
		\scnitem{Решатель задач библиотеки многократно используемых компонентов решателей задач}
	\end{scnrelfromlist}

	\begin{scnsubstruct}

		\scnheader{библиотека многократно используемых компонентов решателей задач ostis-систем}
		\begin{scnrelfromlist}{включение}
			\scnitem{множество компонентов решателей задач}
			\scnitem{средства спецификации компонентов решателей задач}
			\scnitem{средства поиска компонентов решателей задач на основе их спецификации, уточняющие общие средства поиска компонентов в рамках библиотеки}
			\begin{scnindent}
				\scnrelfrom{смотрите}{Предметная область и онтология комплексной библиотеки многократно используемых семантически совместимых компонентов ostis-систем}
			\end{scnindent}
		\end{scnrelfromlist}


		\begin{scnrelfromset}{обобщенная декомпозиция}
			\scnitem{база знаний библиотеки многократно используемых компонентов решателей задач ostis-систем}
			\begin{scnindent}
				\scntext{примечание}{База знаний представляет собой иерархию многократно используемых компонентов решателей задач ostis-систем и их спецификацию.}
			\end{scnindent}
			\scnitem{решатель задач библиотеки многократно используемых компонентов решателей задач ostis-систем}
			\begin{scnindent}
				\scntext{примечание}{Решатель задач позволяет осуществлять поиск компонентов решателей задач, находить зависимости таких компонентов, конфликты между компонентами.}
			\end{scnindent}
			\scnitem{интерфейс библиотеки многократно используемых компонентов решателей задач ostis-систем}
			\begin{scnindent}
				\scntext{примечание}{Интерфейс библиотеки позволяет подключиться к библиотеке и получить доступ к компонентам хранящимся в ней и к ее функционалу.}
			\end{scnindent}
		\end{scnrelfromset}


		\scnheader{многократно используемый компонент решателей задач ostis-систем}
		\scnsuperset{метод}
		\scnsuperset{пакет программ}
		\scnsuperset{абстрактный sc-агент}
		\scnsuperset{решатель задач ostis-системы}
		\begin{scnindent}
			\scnnote{Целые решатели задач могут быть многократно используемыми компонентами в случае разработки интеллектуальных систем, назначение которых совпадает.}
		\end{scnindent}
		\scnnote{Если \textit{многократно используемый компонент решателей задач} является \textit{платформенно-зависимым многократно используемым компонентом ostis-системы}, то его интеграция производится в соответствии с инструкцией, предоставляемой разработчиком в зависимости от платформы, как и для любого компонента такого рода. В противном случае процесс интеграции можно конкретизировать в зависимости от подклассов данного типа компонентов.}

		\scnheader{абстрактный sc-агент}
		\begin{scnsubdividing}
			\scnitem{абстрактный sc-агент информационного поиска}
			\scnitem{абстрактный sc-агент погружения интегрируемого знания в базу знаний}
			\scnitem{абстрактный sc-агент выравнивания онтологии интегрируемого знания с основной онтологией текущего состояния базы знаний}
			\scnitem{абстрактный sc-агент планирования решения явно сформулированных задач}
			\scnitem{абстрактный sc-агент логического вывода}
			\scnitem{абстрактный sc-агент верификации базы знаний}
			\scnitem{абстрактный sc-агент редактирования базы знаний}
			\scnitem{абстрактный sc-агент автоматизации деятельности разработчиков базы знаний}
		\end{scnsubdividing}

		\scnnote{Под \textit{многократно используемым абстрактным sc-агентом} подразумевается компонент, соответствующий некоторому \textit{абстрактному sc-агенту}, который может быть использован в решателях задач других ostis-систем, возможно, в составе более сложных \textit{неатомарных абстрактных sc-агентов}. Спецификация многократно используемого \textit{абстрактного sc-агента} должна содержать всю информацию, необходимую для функционирования соответствующего \textit{sc-агента} в дочерней ostis-системе.}
		\begin{scnindent}

		\scnnote{Указанная спецификация формируется следующим образом:

		\begin{scnitemize}
			\item В спецификацию включается \textit{sc-узел}, обозначающий соответствующий \textit{абстрактный sc-агент}, и вся его спецификация, то есть, как минимум, указание \textit{ключевых sc-элементов sc-агента*}, \textit{условия инициирования и результат*}, \textit{первичного условия инициирования*}, \textit{sc-описание поведения sc-агента} и класса решаемых им задач;
			\item В случае, если специфицируется многократно используемый \textit{абстрактный sc-агент}, который рассматривается как \textit{неатомарный абстрактный sc-агент}, то его спецификация  будет содержать \textit{sc-узлы}, обозначающие все более частные \textit{абстрактные sc-агенты}, а также все их спецификации;
			\item Для каждого \textit{атомарного абстрактного sc-агента}, знак которого вошел в такую спецификацию, необходимо выбрать вариант его реализации (то есть элемент класса \textit{платформенно-независимый абстрактный sc-агент} или\\ \textit{платформенно-зависимый абстрактный sc-агент}, связанный с исходным \textit{атомарным абстрактным sc-агентом} связкой отношения \textit{включение*}) и включить в указанную спецификацию sc-узел, обозначающий указанную реализацию, а также знаки всех программ, входящие во множество, связанное с указанной реализацией отношением \textit{программа sc-агента*};
			\item В спецификацию компонента включаются также все связки отношений, связывающие уже включенные в его состав sc-элементы, а также сами знаки этих отношений (например, \textit{включение*}, \textit{программа sc-агента*} и так далее).
		\end{scnitemize}}
		\scnnote{После того как многократно используемый \textit{абстрактный sc-агент} был скопирован в дочернюю ostis-систему, необходимо сгенерировать sc-узел, обозначающий конкретный \textit{sc-агент}, работающий в данной системе и принадлежащий выбранной реализации \textit{абстрактного sc-агента}, и добавить его во множество \textit{активных sc-агентов} при необходимости.}
 	  \end{scnindent}


		\scnnote{Под \textit{многократно используемой программой} подразумевается компонент, соответствующий программе, записанной на произвольном языке программирования, которая ориентирована на обработку \textit{структур}, хранящихся в памяти \textit{ostis-системы}. Приоритетным в данном случае является использование \textit{scp-программ} по причине их платформенной независимости, за исключением случаев проектирования некоторых компонентов интерфейса, когда полная платформенная независимость невозможна (например, при проектировании \textit{эффекторных sc-агентов и рецепторных sc-агентов}).}
	\begin{scnindent}

		\scnnote{Каждую \textit{scp-программу}, попавшую в \textit{дочернюю ostis-систему} при копировании \textit{многократно используемого комопнента решателя задач ostis-систем}, необходимо добавить во множество \textit{корректных scp-программ} (корректность верифицируется при попадании в библиотеку компонентов).}
	\end{scnindent}
	
		\scnnote{Для удобства работы с библиотекой многократно используемых компонентов необходимы также средства автоматизации поиска компонентов на основе заданной спецификации, представляющие собой \textit{решатель задач ostis-системы} частного вида.}

		\scnheader{Средства автоматизации библиотеки многократно используемых компонентов решателей задач ostis-систем}
		\begin{scnrelfromset}{декомпозиция}
			\scnitem{Множество методов, входящих в состав средств автоматизации библиотеки многократно используемых компонентов решателей задач ostis-систем}
			\scnitem{Машина обработки знаний библиотеки многократно используемых компонентов решателей задач ostis-систем}
			\begin{scnindent}
				\begin{scnrelfromset}{декомпозиция абстрактного sc-агента}
					\scnitem{Абстрактный sc-агент формирования неатомарного компонента решателей задач ostis-систем из атомарных}
					\begin{scnindent}
						\scntext{задача}{Задача \textit{Абстрактного sc-агента формирования неатомарного компонента решателей задач ostis-систем из атомарных} --- формирование структуры, содержащей в себе полный sc-текст неатомарного компонента, включая спецификации всех \mbox{sc-агентов} в его составе, а также тексты всех необходимых scp-программ.}
						\begin{scnindent}
							\scnnote{Формирование такой структуры необходимо для того, чтобы упростить процесс копирования указанного компонента в другие ostis-системы.}
							\scntext{пояснение}{Под \textit{неатомарным компонентом решателей задач ostis-систем} понимается такой компонент, в составе которого можно выделить другие компоненты, которые могут использоваться самостоятельно, отдельно от исходного компонента. Чаще всего в роли таких неатомарных компонентов выступают неатомарные sc-агенты, в составе которых могут быть выделены самодостаточные sc-агенты, которые могут быть использованы отдельно от исходного неатомарного, или scp-программы, которые являются общими для нескольких агентов и могут быть использованы не только в составе неатомарного sc-агента.}
						\end{scnindent}
					\end{scnindent}
					\scnitem{Абстрактный sc-агент поиска всех неатомарных компонентов, частью которых является заданный атомарный компонент}
					\scnitem{Абстрактный sc-агент поиска всех сопутствующих компонентов}
					\begin{scnindent}
						\scntext{пояснение}{Под сопутствующим компонентом понимается компонент, который часто используется в ostis-системе одновременно с некоторым другим компонентом.}
						\begin{scnindent}
						\scnrelfrom{смотрите}{Предметная область и онтология комплексной библиотеки многократно используемых семантически совместимых компонентов ostis-систем}
						\end{scnindent}
					\end{scnindent}
					\scnitem{Абстрактный sc-агент поиска sc-агента по условию инициирования}
					\scnitem{Абстрактный sc-агент поиска sc-агента по результату работы}
					\scnitem{Абстрактный sc-агент поиска scp-программы по входным/выходным параметрам}
					\scnitem{Абстрактный sc-агент поиска sc-агентов, для которых элементы заданного множества являются ключевыми sc-элементами}\\
					\begin{scnindent}
						\scnnote{\textit{Абстрактный sc-агент поиска sc-агентов, для которых элементы заданного множества являются ключевыми sc-элементами} играет важную роль при внесении изменений в базу знаний, в частности, при переопределении каких-либо понятий. Указанный sc-агент позволяет выявить те sc-агенты, для которых могут потребоваться изменения в алгоритме работы в связи с изменением семантической трактовки каких-либо понятий.}
					\end{scnindent}
				\end{scnrelfromset}
			\end{scnindent}
		\end{scnrelfromset}


		\scnheader{отношение, специфицирующее многократно используемый компонент решателей задач ostis-систем}
		\scnsubset{отношение, специфицирующее многократно используемый компонент ostis-систем}
		\scnhaselement{первичное условие инициирования*}
		\begin{scnindent}
			\scnexplanation{Связки отношения первичное условие инициирования* связывают между собой sc-узел, обозначающий абстрактный sc-агент и бинарную ориентированную пару, описывающую первичное условие инициирования данного абстрактного sc-агента, то есть такой ситуации в sc-памяти, которая побуждает sc-агента перейти в активное состояние и начать проверку наличия своего полного условия инициирования.

			Первым компонентом данной ориентированной пары является знак некоторого подмножества понятия событие, например событие добавления выходящей sc-дуги, то есть по сути конкретный тип события в sc-памяти.

			Вторым компонентом данной ориентированной пары является произвольный в общем случае sc-элемент, с которым непосредственно связан указанный тип события в sc-памяти, то есть, например, sc-элемент, из которого выходит либо в который входит генерируемая либо удаляемая sc-дуга, либо файл ostis-системы, содержимое которого было изменено.

			После того, как в sc-памяти происходит некоторое событие, активизируются все активные sc-агенты, первичное условие инициирования* которых соответствует произошедшему событию.}
			\scnrelfrom{первый домен}{абстрактный sc-агент}
			\scnrelfrom{второй домен}{бинарная ориентированная пара}
		\end{scnindent}
		\scnhaselement{условие инициирования и результат*}
		\begin{scnindent}
			\scnexplanation{Связки отношения условие инициирования и результат* связывают между собой sc-узел, обозначающий абстрактный sc-агент и бинарную ориентированную пару, связывающую условие инициирования данного абстрактного sc-агента и результаты выполнения экземпляров данного sc-агента в какой-либо конкретной системе.

			Указанную ориентированную пару можно рассматривать как логическую связку импликации, при этом на sc-переменные, присутствующие в обеих частях связки, неявно накладывается квантор всеобщности, на sc-переменные, присутствующие либо только в посылке, либо только в заключении неявно накладывается квантор существования.

			Первым компонентом указанной ориентированной пары является логическая формула, описывающая условие инициирования описываемого абстрактного sc-агента, то есть конструкции, наличие которой в sc-памяти побуждает sc-агент начать работу по изменению состояния sc-памяти. Данная логическая формула может быть как атомарной, так и неатомарной, в которой допускается использование любых связок логического языка.

			Вторым компонентом указанной ориентированной пары является логическая формула, описывающая возможные результаты выполнения описываемого абстрактного sc-агента, то есть описание произведенных им изменений состояния sc-памяти. Данная логическая формула может быть как атомарной, так и неатомарной, в которой допускается использование любых связок логического языка.}
			\scnrelfrom{первый домен}{абстрактный sc-агент}
			\scnrelfrom{второй домен}{бинарная ориентированная пара}
		\end{scnindent}


	\end{scnsubstruct}

\end{SCn}
