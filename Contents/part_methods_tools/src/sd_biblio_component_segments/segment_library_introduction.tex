\begin{SCn}
	\scnsectionheader{Сегмент. Введение в Предметную область и онтологию комплексной библиотеки многократно используемых семантически совместимых компонентов ostis-систем}

	\begin{scnsubstruct}
	
	\scnheader{компонентное проектирование интеллектуальных компьютерных систем}
	\begin{scnrelfromset}{основные положения}
		\scnfileitem{Важнейшим этапом эволюции любой технологии является переход к компонентному проектированию на основе постоянно пополняемый библиотеки многократно используемых компонентов.}
		\begin{scnindent}
			\begin{scnrelfromset}{необходимые требования}
				\scnfileitem{Универсальный язык представления знаний.}
				\scnfileitem{Универсальная процедура интеграции знаний в рамках указанного языка.}
				\scnfileitem{Разработка стандарта, обеспечивающего семантическую совместимость интегрируемых знаний (таким стандартом является согласованная система используемых понятий).}
			\end{scnrelfromset}
		\end{scnindent}
		\scnfileitem{Повторное использование готовых компонентов широко применяется во многих отраслях, связанных с проектированием различного рода систем, поскольку позволяет уменьшить трудоемкость разработки и ее стоимость (путем минимизации количества труда за счет отсутствия необходимости разрабатывать какой-либо компонент), повысить качество создаваемого контента и снизить профессиональные требования к разработчикам компьютерных систем. Таким образом, осуществляется  переход от программирования компонентов или целых систем к их проектированию (дизайну, сборке) на основе готовых компонентов. \textbf{\textit{компонентное проектирование интеллектуальных компьютерных систем}} предполагает подбор существующих компонентов, способных решить поставленную задачу целиком или декомпозицию задачи на подзадачи с выделением компонентов для каждой из них.}
		\begin{scnindent}
			\begin{scnrelfromlist}{источник}
				\scnitem{\cite{Zhitko2011b}}
				\scnitem{\cite{Zalivako2012}}
				\scnitem{\cite{Borisov2014}}
			\end{scnrelfromlist}
		\end{scnindent}
	\end{scnrelfromset}
	\scntext{назначение}{Позволяет уменьшить трудоемкость создания компьютерных систем и их стоимость (путем минимизации количества труда за счет отсутствия необходимости разрабатывать какой-либо компонент), повысить качество создаваемых компьютерных систем и снизить профессиональные требования к разработчикам этих систем.}      
	\scntext{пояснение}{Компонентное проектирование интеллектуальных компьютерных систем предполагает подбор существующих компонентов, способных решить поставленную задачу целиком или декомпозицию задачи на подзадачи с выделением компонентов для каждой из них.}  
	\scntext{преимущество}{Проектируемые системы по предлагаемой технологии обладают высоким уровнем гибкости, их разработка осуществляется поэтапно, переходя от одной целостной версии системы к другой. При этом стартовая версия системы может быть ядром соответствующего класса систем, входящим в библиотеку многократно используемых компонентов.}
	\scnheader{технология компонентного проектирования интеллектуальных компьютерных систем}		
	\scnhaselementrole{главный ключевой sc-элемент}{библиотека совместимых многократно используемых компонентов}
	\scntext{преимущество}{Позволяет проектировать интеллектуальные системы, комбинируя уже существующие компоненты, выбирая нужные из соответствующих библиотек. Использование готовых компонентов предполагает, что распространяемый компонент верифицирован и документирован, а возможные ошибки и ограничения устранены либо специфицированы и известны. Создание \textit{библиотеки многократно используемых компонентов} не означает пересоздание всех уже существующих современных продуктов информационных технологий. Технология компонентного проектирования интеллектуальных компьютерных систем предполагает использование огромного опыта в разработке современных компьютерных систем, однако обязательным является \uline{спецификация} каждого компонента (как вновь созданного, так и интегрируемого существующего) для обеспечения возможности его установки и совместимости с другими компонентами и системами. Тем не менее эффективная технология компонентного проектирования появится только тогда, когда сформируется \scnqq{критическая масса} разработчиков прикладных систем, участвующих в пополнении \textit{библиотек многократно используемых компонентов} проектируемых систем.}
	\begin{scnindent}
		\begin{scnrelfromlist}{источник}
			\scnitem{\cite{Zhitko2011b}}
			\scnitem{\cite{Golenkov2013}}
		\end{scnrelfromlist}
	\end{scnindent}
	\scnrelfrom{проблемы текущего состояния}{Проблемы в реализации компонентного проектирования интеллектуальных компьютерных систем}
	\begin{scnindent}
		\scntext{примечание}{Проблемы реализации компонентного подхода к проектированию интеллектуальных компьютерных систем наследуют проблемы современных \textit{технологий проектирования интеллектуальных систем}.}
		\begin{scneqtoset}
			\scnfileitem{\uline{Несовместимость} компонентов, разработанных в рамках разных проектов, вследствие отсутствия унификации в принципах представления различных видов знаний в рамках одной \textit{базы знаний}, и, как следствие, отсутствие унификации в принципах выделения и спецификации \textbf{\textit{многократно используемых компонентов}}, которое приводит к несовместимости компонентов, разработанных в рамках разных проектов.}
			\begin{scnindent}
				\scntext{примечание}{Большинство существующих систем создано как автономные программные продукты, которые не могут быть использованы в качестве компонентов других систем. Необходимо использовать либо целую систему, либо ничего. Небольшое число систем поддерживает компонентно-ориентированную архитектуру способную интегрироваться с другими системами. Однако их интеграция возможна при условии использования одинаковых технологий и только при проектировании одной командой разработчиков.}
				\begin{scnindent}
					\begin{scnrelfromlist}{источник}
						\scnitem{\cite{Iyengar2021}}
						\scnitem{\cite{Ford2019}}
					\end{scnrelfromlist}
				\end{scnindent}
				\scntext{примечание}{Многократная повторная разработка уже имеющихся технических решений обусловлена либо тем, что известные технические решения \uline{плохо} интегрируются в разрабатываемую систему, либо тем, что эти технические решения трудно найти. Данная проблема актуальна как в целом в сфере разработки компьютерных систем, так и в сфере разработки систем, основанных на знаниях, поскольку в системах такого рода степень согласованности различных видов знаний влияет на возможность системы решать нетривиальные задачи.}  
			\end{scnindent}
			\scnfileitem{Невозможность автоматической интеграции компонентов в систему \uline{без} ручного вмешательства пользователя.}
			\scnfileitem{Автоматическое обновление компонентов приводит к рассогласованности как отдельных модулей компьютерных систем, так и самих систем между собой.}
			\scnfileitem{Отсутствие классификации компонентов на различных уровнях детализации.}
			\scnfileitem{Не ведётся разработка стандартов, обеспечивающих совместимость этих компонентов.}
			\scnfileitem{Не проводится тестирование, верификация и анализ качества компонентов, не выделяются преимущества, недостатки, ограничения компонентов.}
			\scnfileitem{Многие компоненты используют для идентификации язык разработчика (как правило, английский), и предполагается, что все пользователи будут использовать этот же язык. Однако для многих приложений это недопустимо --- понятные только разработчику идентификаторы должны быть скрыты от конечных пользователей, которые должны быть в состоянии выбрать язык для идентификаторов, которые они видят.}
			\scnfileitem{Отсутствие средств поиска компонентов, удовлетворяющих заданным критериям.}
		\end{scneqtoset}
		\scnrelfrom{источник}{\cite{Shunkevich2015a}}
	\end{scnindent}
	\scntext{примечание}{\scnkeyword{компонентное проектирование интеллектуальных компьютерных систем} возможно только в том случае, если отбор компонентов будет осуществляться на основе тщательного анализа качества этих компонентов. Одним из важнейших критериев такого анализа является уровень семантической совместимости анализируемых компонентов со всеми компонентами, имеющимися в текущей версии библиотеки.}
	\scnrelfrom{предъявляемые требования}{Требования к реализации компонентного проектирования интеллектуальных компьютерных систем}
	\begin{scnindent}
		\begin{scneqtoset}
			\scnfileitem{Обеспечение совместимости (интегрируемости) компонентов интеллектуальных компьютерных систем на основе унификации представления этих компонентов.}
			\scnfileitem{Четкое разделение процесса разработки формальных описаний интеллектуальных компьютерных систем и процесса их реализации по этому описанию.}
			\scnfileitem{Четкое разделение разработки формального описания проектируемой интеллектуальной системы от разработки различных вариантов интерпретации таких формальных описаний систем.}
			\scnfileitem{Наличие онтологии компонентного проектирования интеллектуальных компьютерных систем, включающей (1) описание методов компонентного проектирования, (2) модель \textit{библиотеки многократно используемых компонентов}, (3) модель \textit{спецификации многократно используемых компонентов}, (4) полную \textit{классификацию многократно используемых компонентов}, (5) описание средств взаимодействия разрабатываемой интеллектуальной компьютерной системы с \textit{библиотеками многократно используемых компонентов}.}
			\scnfileitem{Наличие \textit{библиотек многократно используемых компонентов интеллектуальных компьютерных систем}, включающих спецификации компонентов.}
			\scnfileitem{Наличие средств взаимодействия разрабатываемой интеллектуальной компьютерной системы с библиотеками многократно используемых компонентов для установки любых видов компонентов и управления ими в создаваемой системе.}
			\begin{scnindent}
				\scnnote{Под установкой компонента понимается его транспортировка в систему (копирование sc-элементов и/или скачивание файлов компонента), а также выполнение вспомогательных действий для того, чтобы компонент мог функционировать в создаваемой системе.}
			\end{scnindent}
		\end{scneqtoset}
		\begin{scnrelfromlist}{источник}
			\scnitem{\cite{Zalivako2011}}
			\scnitem{\cite{Golenkov2013}}
			\scnitem{\cite{Golenkov2014a}}
		\end{scnrelfromlist}
		\scntext{примечание}{Для того, чтобы решить возникшие проблемы при проектировании интеллектуальных систем и библиотек их многократно используемых компонентов, необходимо придерживаться общих принципов технологии проектирования интеллектуальных компьютерных систем, а также выполнить эти требования.}
	\end{scnindent}
	
		\bigskip
	\end{scnsubstruct}
	\scnsourcecomment{Завершили \scnqqi{Сегмент. Введение в Предметную область и онтологию комплексной библиотеки многократно используемых семантически совместимых компонентов ostis-систем}}
\end{SCn}