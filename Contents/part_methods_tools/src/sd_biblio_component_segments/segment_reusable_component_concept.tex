\begin{SCn}
	\scnsectionheader{Сегмент. Понятие многократно используемого компонента ostis-систем}

\begin{scnsubstruct}
	
	\scnheader{многократно используемый компонент ostis-систем}
	\scnidtf{типовой компонент ostis-систем}
	\scnidtf{повторно используемый компонент ostis-систем}
	\scnidtf{многократно используемый компонент OSTIS}
	\scnidtf{ip-компонент ostis-систем}
	\scnidtftext{часто используемый sc-идентификатор}{многократно используемый компонент}
	\scnrelfrom{аббревиатура}{\scnfilelong{МИК ostis-систем}}
	\scnsubset{sc-структура}
	\scnsubset{компонент ostis-системы}
	\begin{scnindent}
		\scntext{пояснение}{Целостная часть ostis-системы, которая содержит все те (и только те) sc-элементы, которые необходимы для её функционирования в ostis-системе.}
	\end{scnindent}
	\scntext{определение}{многократно используемый компонент ostis-систем --- компонент некоторой ostis-системы, который может быть использован в рамках другой ostis-системы.}
	\begin{scnindent}
		\scnrelfrom{источник}{\cite{Shunkevich2015a}}
	\end{scnindent}
	\scntext{пояснение}{Компонент ostis-системы, который может быть использован в других ostis-системах (\scnkeyword{дочерних ostis-системах}).}
	\scntext{пояснение}{Компонент некоторой \scnkeyword{материнской ostis-системы}, который может быть использован в некоторой \scnkeyword{дочерней ostis-системе}.}
	\scntext{пояснение}{многократно используемый компонент ostis-систем --- это общий компонент (общая часть) для некоторого множества ostis-систем, который многократно используется, дублируется и входит в состав некоторого множества ostis-систем.}
	\scntext{примечание}{Для включения многократно используемого компонента в некоторую систему, его необходимо установить в эту систему, то есть скопировать в нее все sc-элементы компонента и, при необходимости, вспомогательные файлы, такие как исходные или скомпилированные файлы компонента.}
	\scntext{примечание}{многократно используемый компонент ostis-систем должен иметь унифицированную спецификацию и иерархию для поддержки \uline{совместимости} с другими компонентами.}
	\scntext{примечание}{Совместимость многократно используемых компонентов приводит систему к новому качеству, к дополнительному расширению множества решаемых задач при интеграции компонентов.}
	\begin{scnrelfromset}{необходимые требования}
		\scnfileitem{Существует техническая возможность встроить многократно используемый компонент в \scnkeyword{дочернюю ostis-систему}.}
		\scnfileitem{Полнота многократно используемого компонента: компонент должен в полной мере выполнять свои функции, соответствовать своему назначению.}
		\scnfileitem{Связность многократно используемого компонента: компонент должен логически выполнять только одну задачу из предметной области, для которой он предназначен. Многократно используемый компонент должен выполнять свои функции наиболее общим образом, чтобы круг возможных систем, в которые он может быть встроен, был наиболее широким.}
		\scnfileitem{Совместимость многократно используемого компонента: компонент должен стремиться повышать уровень \uline{договороспособности} ostis-систем, в которые он встроен, и иметь возможность \uline{автоматической} интеграции в другие системы.}
		\scnfileitem{Самодостаточность компонентов (или групп компонентов) технологии, то есть способности их функционировать отдельно от других компонентов без утраты целесообразности их использования.}
	\end{scnrelfromset}
	\scnheader{следует отличать*}
	\begin{scnhaselementset}
		\scnitem{многократно используемый компонент ostis-систем}
		\scnitem{компонент ostis-системы}
	\end{scnhaselementset}
	\scntext{отличие}{многократно используемый компонент ostis-систем имеет \uline{спецификацию, достаточную для установки} этого компонента в \scnkeyword{дочернюю ostis-систему}. Спецификация является частью базы знаний \scnkeyword{библиотеки многократно используемых компонентов} соответствующей \scnkeyword{материнской ostis-системы}. Есть техническая возможность встроить многократно используемый компонент в дочернюю ostis-систему.}
	\scnheader{параметр, заданный на многократно используемых компонентах ostis-систем\scnsupergroupsign}
	\scnsubset{параметр}
	\scnhaselement{класс многократно используемого компонента ostis-систем\scnsupergroupsign}
	\begin{scnindent}
		\scntext{примечание}{класс многократно используемого компонента ostis-систем является важной частью спецификации компонента, позволяющей лучше понять назначение и область применения данного компонента, а также класс многократно используемого компонента является важнейшим критерием поиска компонентов в библиотеке ostis-систем.}
	\end{scnindent}
	\scnhaselement{начало\scnsupergroupsign}
	\scnhaselement{завершение\scnsupergroupsign}
	\scnheader{многократно используемый компонент ostis-систем}
	\scntext{примечание}{Интеллектуальная система, спроектированная по \textit{Технологии OSTIS}, представляет собой интеграцию \textit{многократно используемых компонентов баз знаний}, \textit{многократно используемых компонентов решателей задач} и \textit{многократно используемых компонентов интерфейсов}.}
	\begin{scnindent}
		\scnrelfrom{источник}{\cite{Pivovarchik2015}}
	\end{scnindent}
	\scntext{примечание}{Основной признак классификации многократно используемых компонентов является признак предметной области, к которой относится компонент. Здесь структура может быть довольно богатой в соответствии с иерархией областей человеческой деятельности. Существует также множество предметно-независимых многократно используемых компонентов, которые могут использоваться в любой предметной области.}
	\begin{scnindent}
		\scnrelfrom{источник}{\cite{Orlov2022a}}
	\end{scnindent}
	\begin{scnrelfromset}{разбиение}
		\scnitem{многократно используемый компонент базы знаний ostis-систем}
		\begin{scnindent}
			\scnidtf{многократно используемый компонент базы знаний}
			\scniselement{класс многократно используемого компонента ostis-систем\scnsupergroupsign}
			\scntext{примечание}{Важнейшим признаком классификации многократно используемых компонентов баз знаний является вид знаний.}
			\begin{scnindent}
				\scnrelfrom{смотрите}{вид знаний}
			\end{scnindent}
			\scnsuperset{семантическая окрестность}
			\begin{scnindent}
				\scnhaselement{Семантическая окрестность города Минска}
				\scnhaselement{Семантическая окрестность понятия множество}
			\end{scnindent}
			\scnsuperset{предметная область и онтология}
			\begin{scnindent}
				\scnhaselement{Предметная область и онтология треугольников}
			\end{scnindent}
			\scnsuperset{база знаний}
			\scnsuperset{шаблон типового компонента ostis-систем}
			\begin{scnindent}
				\scnhaselement{Шаблон описания предметной области}
				\scnhaselement{Шаблон описания отношения}
			\end{scnindent}
		\end{scnindent}
		\scnitem{многократно используемый компонент решателя задач ostis-систем}
		\begin{scnindent}
			\scnidtf{многократно используемый компонент решателя задач}
			\scniselement{класс многократно используемого компонента ostis-систем\scnsupergroupsign}
			\scntext{примечание}{Важнейшим признаком классификации многократно используемых компонентов баз решателя задач является используемая модель решения задачи.}
			\scnsuperset{атомарный абстрактный sc-агент}
			\begin{scnindent}
				\scnhaselement{Абстрактный sc-агент подсчета мощности множества}
			\end{scnindent}
			\scnsuperset{программа обработки знаний}
			\scnsuperset{scp-машина}
			\scnsuperset{scl-машина}
		\end{scnindent}
		\scnitem{многократно используемый компонент интерфейса ostis-систем}
		\begin{scnindent}
			\scnidtf{многократно используемый компонент интерфейса}
			\scniselement{класс многократно используемого компонента ostis-систем\scnsupergroupsign}
			\scntext{примечание}{Важнейшим признаком классификации многократно используемых компонентов баз решателя задач является вид интерфейса в соответствии с классификацией интерфейсов.}
			\scnsuperset{многократно используемый компонент пользовательских интерфейсов ostis-систем}
		\end{scnindent}
	\end{scnrelfromset}
	\scnrelfrom{разбиение}{\scnkeyword{Типология компонентов ostis-систем по атомарности\scnsupergroupsign}}
	\begin{scnindent}
		\scnsubset{класс многократно используемого компонента ostis-систем\scnsupergroupsign}
		\begin{scneqtoset}
			\scnitem{атомарный многократно используемый компонент ostis-систем}
			\begin{scnindent}
				\scnhaselement{Абстрактный sc-агент подсчета мощности множества}
				\scntext{пояснение}{Многократно используемый компонент, который в текущем состоянии библиотеки ostis-систем рассматривается как неделимый, то есть не содержит в своем составе других компонентов.}
				\scntext{примечание}{Принадлежность МИК ostis-систем классу атомарных компонентов зависит от того, как специфицирован этот компонент, и от существующих на данный момент компонентов в библиотеке.}
				\begin{scnindent}
					\scntext{примечание}{В библиотеку ostis-систем нельзя опубликовать многократно используемый компонент как атомарный, в составе которого есть какой-либо другой известный библиотеке ostis-систем компонент.}
				\end{scnindent}
				\scntext{примечание}{В общем случае атомарный компонент может перейти в разряд неатомарных в случае, если будет принято решение выделить какую-то из его частей в качестве отдельного компонента. Все вышесказанное, однако, не означает, что даже в случае его платформенной независимости, атомарный компонент всегда хранится в sc-памяти как сформированная sc-структура. Например, платформенно-независимая реализация sc-агента всегда будет представлена набором \textit{scp-программ}, но будет \textit{атомарным многократно используемым компонентом ostis-систем} в случае, если ни одна из этих программ не будет представлять интереса как самостоятельный компонент.}
			\end{scnindent}
			\scnitem{неатомарный многократно используемый компонент ostis-систем}
			\begin{scnindent}
				\scnidtf{составной многократно используемый компонент ostis-систем}
				\scnhaselement{Решатель задач по геометрии}
				\scntext{пояснение}{Многократно используемый компонент, который в текущем состоянии библиотеки ostis-систем содержит в своем составе другие атомарные или неатомарные компоненты.}
				\scntext{примечание}{Неатомарный многократно используемый компонент не зависит от своих частей. Без какой-либо части неатомарного компонента его назначение сужается.}
				\scntext{примечание}{В общем случае неатомарный компонент может перейти в разряд атомарных в случае, если будет принято решение по каким-либо причинам исключить все его части из рассмотрения в качестве отдельных компонентов. Следует отметить, что неатомарный компонент необязательно должен складываться \uline{полностью} из других компонентов, в его состав могут также входить и части, не являющиеся самостоятельными компонентами. Например, в состав реализованного на \textit{Языке SCP} \textit{sc-агента}, являющего \textit{неатомарным многократно используемым компонентом} могут входить как \textit{scp-программы}, которые могут являться многократно используемыми компонентами (а могут и не являться), а также агентная \textit{scp-программа}, которая не имеет смысла как многократно используемый компонент.}
				\scntext{примечание}{Спецификация неатомарного многократно используемого компонента должна содержать информацию о том, из каких компонентов он состоит, используя отношение декомпозиция*. При этом sc-структура, обозначающая неатомарный компонент не обязана содержать все sc-элементы компонентов, на которые она декомпозируется, достаточно, чтобы неатомарному компоненту принадлежали знаки всех тех компонентов, из которых он состоит. Должно быть полное перечисление составных компонентов.}
			\end{scnindent}
		\end{scneqtoset}
	\end{scnindent}
	\scnrelfrom{разбиение}{\scnkeyword{Типология компонентов ostis-систем по зависимости от других компонентов\scnsupergroupsign}}
	\begin{scnindent}
		\scnsubset{класс многократно используемого компонента ostis-систем\scnsupergroupsign}
		\begin{scneqtoset}
			\scnitem{зависимый многократно используемый компонент ostis-систем}
			\begin{scnindent}
				\scnhaselement{визуальный редактор системы по химии}
				\scntext{пояснение}{Многократно используемый компонент, который зависит хотя бы от одного другого компонента библиотеки ostis-систем, то есть не может быть встроен в дочернюю ostis-систему без компонентов, от которых он зависит.}
			\end{scnindent}
			\scnitem{независимый многократно используемый компонент ostis-систем}
			\begin{scnindent}
				\scnhaselement{Предметная область множеств}
				\scntext{пояснение}{Многократно используемый компонент, который не зависит ни от одного другого компонента библиотеки ostis-систем.}
			\end{scnindent}
		\end{scneqtoset}
	\end{scnindent}
	\scnrelfrom{разбиение}{\scnkeyword{Типология компонентов ostis-систем по способу их хранения\scnsupergroupsign}}
	\begin{scnindent}
		\scnsubset{класс многократно используемого компонента ostis-систем\scnsupergroupsign}
		\begin{scneqtoset}
			\scnitem{многократно используемый компонент ostis-систем, хранящийся в виде внешних файлов}
			\begin{scnindent}
				\begin{scnrelfromset}{разбиение}
					\scnitem{многократно используемый компонент ostis-систем, хранящийся в виде файлов исходных текстов}
					\scnitem{многократно используемый компонент ostis-систем, хранящийся в виде скомпилированных файлов}
				\end{scnrelfromset}
			\end{scnindent}
			\scnitem{многократно используемый компонент, хранящийся в виде sc-структуры}
		\end{scneqtoset}
		\scntext{примечание}{На данном этапе развития \textit{Технологии OSTIS} более удобным является хранение компонентов в виде исходных текстов.}
	\end{scnindent}
	\scnrelfrom{разбиение}{\scnkeyword{Типология компонентов ostis-систем по зависимости от платформы\scnsupergroupsign}}
	\begin{scnindent}
		\scnsubset{класс многократно используемого компонента ostis-систем\scnsupergroupsign}
		\begin{scneqtoset}
			\scnitem{платформенно зависимый многократно используемый компонент ostis-систем}
			\begin{scnindent}
				\scntext{пояснение}{Под платформенно-зависимым многократно используемым компонентом OSTIS понимается компонент, частично или полностью реализованный при помощи каких-либо сторонних с точки зрения \textit{Технологии OSTIS} средств.}
				\scntext{недостаток}{Интеграция таких компонентов в интеллектуальные системы может сопровождаться дополнительными трудностями, зависящими от конкретных средств реализации компонента.}
				\scntext{преимущество}{В качестве возможного преимущества платформенно-зависимых многократно используемых компонентов ostis-систем можно выделить их, как правило, более высокую производительность за счет реализации их на более приближенном к платформе уровне.}
				\scntext{примечание}{С точки зрения \textit{Технологии OSTIS} любая ostis-платформа является платформенно-зависимым многократно используемым компонентом.}
				\scntext{примечание}{В общем случае платформенно-зависимый многократно используемый компонент ostis-систем может поставляться как в виде набора исходных кодов, так и в скомпилированном виде. Процесс интеграции платформенно-зависимого многократно используемого компонента ostis-систем в дочернюю систему, разработанную по \textit{Технологии OSTIS}, сильно зависит от технологий реализации данного компонента и в каждом конкретном случае может состоять из различных этапов. Каждый платформенно-зависимый многократно используемый компонент ostis-систем должен иметь соответствующую подробную, корректную и понятную инструкцию по его установке и внедрению в дочернюю систему.}
				\scnsuperset{ostis-платформа}
				\scnsuperset{абстрактный sc-агент, не реализуемый на Языке SCP}
			\end{scnindent}
			\scnitem{платформенно-независимый многократно используемый компонент ostis-систем}
			\begin{scnindent}
				\scntext{пояснение}{Под платформенно-независимым многократно используемым компонентом ostis-систем понимается компонент, который полностью представлен на \textit{SC-коде}.}
				\scnsuperset{многократно используемый компонент базы знаний}
				\scnsuperset{платформенно-независимый scp-агент}
				\scnsuperset{scp-программа}
				\scntext{примечание}{В случае \textit{неатомарного многократно используемого компонента} платформенная независимость означает, что \uline{все} более простые компоненты, входящие в его состав также обязаны быть платформенно-независимыми многократно используемыми компонентами ostis-систем.}
				\scntext{примечание}{Процесс интеграции платформенно-зависимого многократно используемого компонента ostis-систем в дочернюю систему, разработанную по Технологии OSTIS, существенно упрощается за счет использования общей унифицированной формальной основы представления и обработки знаний.}
				\scntext{примечание}{Наиболее ценными являются платформенно-независимые многократно используемые компоненты ostis-систем.}
			\end{scnindent}
		\end{scneqtoset}
	\end{scnindent}
	\scnrelfrom{разбиение}{\scnkeyword{Типология компонентов ostis-систем по динамичности их установки\scnsupergroupsign}}
	\begin{scnindent}
		\scnsubset{класс многократно используемого компонента ostis-систем\scnsupergroupsign}
		\begin{scneqtoset}
			\scnitem{динамически устанавливаемый многократно используемый компонент ostis-систем}
			\begin{scnindent}
				\scnidtf{многократно используемый компонент, при установке которого система не требует перезапуска}
				\begin{scnrelfromset}{декомпозиция}
					\scnitem{многократно используемый компонент, хранящийся в виде скомпилированных файлов}
					\scnitem{многократно используемый компонент базы знаний}
				\end{scnrelfromset}
			\end{scnindent}
			\scnitem{многократно используемый компонент, при установке которого система требует перезапуска}
		\end{scneqtoset}
		\scntext{примечание}{Процесс интеграции компонентов разных видов на разных этапах жизненного цикла osits-систем бывает разным. Наиболее ценными являются компоненты, которые могут быть интегрированы в работающую систему \uline{без} прекращения её функционирования. Некоторые системы, особенно системы управления, нельзя останавливать, а устанавливать и обновлять компоненты нужно.}
	\end{scnindent}
	\scnsuperset{встраиваемая ostis-систем}
	\begin{scnindent}
		\scnidtf{типовая подсистема ostis-систем}
		\scnsubset{ostis-система}
		\scnsubset{неатомарный многократно используемый компонент ostis-систем}
		\scntext{пояснение}{\scnkeyword{встраиваемая ostis-система} --- это \textit{неатомарный многократно используемый компонент}, который состоит из \textit{базы знаний}, \textit{решателя задач} и \textit{интерфейса}.}
		\begin{scnrelfromset}{декомпозиция}
			\scnitem{многократно используемый компонент базы знаний ostis-систем}
			\scnitem{многократно используемый компонент решателей задач ostis-систем}
			\scnitem{многократно используемый компонент интерфейсов ostis-систем}
		\end{scnrelfromset}
		\scniselement{класс многократно используемого компонента ostis-систем\scnsupergroupsign}
		\scnhaselement{Среда коллективной разработки баз знаний ostis-систем}
		\scnhaselement{Визуальный web-ориентированный редактор sc.g-текстов}
		\scnhaselement{Естественно-языковой интерфейс ostis-системы}
		\scnsuperset{менеджер многократно используемых компонентов ostis-систем}
		\scnsuperset{интеллектуальная обучающая ostis-система}
		\scnsuperset{система тестирования и верификации ostis-систем}
		\scntext{примечание}{Особенность \textit{встраиваемых ostis-систем} в том, что интеграция целых интеллектуальных систем предполагает интеграцию баз знаний этих систем, интеграцию их решателей задач и интеграцию их интеллектуальных интерфейсов. При интеграции встраиваемых ostis-систем база знаний встраиваемой системы становится частью базы знаний системы, в которую она встраивается. Решатель задач встраиваемой ostis-системы становится частью решателя задач системы, в которую она встраивается. И интерфейс встраиваемой ostis-системы становится частью интерфейса системы, в которую она встраивается. При этом встраиваемая система является целостной и может функционировать отдельно от других ostis-систем, в отличие от других многократно используемых компонентов.}
		\scntext{примечание}{\textit{встраиваемые ostis-системы} зачастую являются предметно-независимыми многократно используемыми компонентами. Таким образом, например, встраиваемая ostis-система в виде среды проектирования баз знаний может быть встроена как в систему из предметной области по геометрии, так и в систему управления аграрными объектами.}
		\scntext{примечание}{\textit{встраиваемая ostis-система}, как и любой другой многократно используемый компонент ostis-систем, должна поддерживать семантическую совместимость ostis-систем. Как сама встраиваемая ostis-система, так и все ее компоненты должны быть специфицированы и согласованы. Компоненты встраиваемых ostis-систем могут быть заменены на другие, имеющие то же назначение, например, естественно-языковой интерфейс может иметь различные варианты базы знаний в зависимости от естественного языка, поддерживаемого системой, различные варианты интерфейса, в зависимости от требований и удобства пользователей и также различные варианты реализации решателя задач для обработки естественного языка, которые могут использовать различные модели, однако решать одну и ту же задачу. Встраиваемая ostis-система связывается с системой, в которую она встроена с помощью отношения \textbf{\textit{встроенная ostis-система*}}, которое является подмножеством отношения \textit{встроенная кибернетическая система*}.}
	\end{scnindent}
	
		\bigskip
	\end{scnsubstruct}
	\scnsourcecomment{Завершили \scnqqi{Сегмент. Понятие многократно используемого компонента ostis-систем}}
\end{SCn}