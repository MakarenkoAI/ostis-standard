\begin{SCn}
	\scnsectionheader{Предметная область и онтология действий и методик проектирования баз знаний ostis-систем}
	\scntext{аннотация}{В предметной области и онтологии рассматриваются актуальные проблемы текущего состояния средств проектирования и анализа качества \textit{баз знаний}, предложен подход к их решению на основе \textit{Технологии OSTIS}. Сформулированы принципы коллективного проектирования и разработки \textit{баз знаний}. Сформулированы принципы верификации \textit{баз знаний}.}
	
	\begin{scnrelfromvector}{введение}
		\scnfileitem{Разработка \textit{базы знаний} является трудоемким и продолжительным процессом, требующим высокого уровня квалификации \textit{разработчиков баз знаний}. Данный факт приводит к высокой себестоимости как самих \textit{баз знаний}, так и соответствующих им \textit{интеллектуальных систем}, а также к дефициту специалистов в области \textit{инженерии знаний}.}
		\scnfileitem{Расширение областей применения \textit{интеллектуальных систем} требует поддержки решения комплексных задач. Решение каждой такой задачи предполагает совместное использование различных видов знаний и моделей их представления, что приводит к компенсации недостатков одних моделей возможностями и достоинствами других.}
		\scnfileitem{Существующие \textit{средства создания баз знаний} предполагают, что процессы разработки и модификации базы знаний осуществляются отдельно от процесса ее использования, что приводит к дополнительному усложнению решения задачи обеспечения совместимости различного вида знаний. Отсутствие удовлетворительного решения этой задачи приводит к несовместимости \textit{компонентов баз знаний}, разрабатываемых для разных систем, и невозможности их повторного использования в других системах. Данный факт приводит к многократной повторной разработке содержательно одних и тех же компонентов для разных баз знаний.}
		\scnfileitem{Таким образом, актуальной является задача разработки модели \textit{баз знаний}, которая, с одной стороны, обеспечит общий унифицированный формальный фундамент для представления различных видов знаний в рамках одной \textit{базы знаний} и их совместного использования при решении комплексных задач, а с другой стороны, обеспечит возможность расширения числа видов знаний, используемых \textit{интеллектуальной системой}}
	\end{scnrelfromvector}
	\begin{scnindent}
		\begin{scnrelfromset}{источник}
			\scnitem{Davydenko2016}
		\end{scnrelfromset}
	\end{scnindent}
	
	\begin{scnrelfromlist}{ключевое понятие}
		\scnitem{база знаний}
		\scnitem{редактор баз знаний}
		\scnitem{разработка баз знаний}
		\scnitem{разработчик}
		\scnitem{противоречие}
		\scnitem{информационная дыра}
		\scnitem{многократно используемые компоненты}
	\end{scnrelfromlist}
	
\end{SCn}
