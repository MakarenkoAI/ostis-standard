\begin{SCn}
	\scnsectionheader{Предметная область и онтология действий и методик проектирования баз знаний ostis-систем}
	\scntext{аннотация}{В предметной области и онтологии рассматриваются актуальные проблемы текущего состояния средств проектирования и анализа качества \textit{баз знаний}, предложен подход к их решению на основе \textit{Технологии OSTIS}. Сформулированы принципы коллективного проектирования и разработки \textit{баз знаний}. Сформулированы принципы верификации \textit{баз знаний}.}
	
	\scntext{эпиграф}{Дырявая и запутанная сеть хорошего улова не принесет.}
	
	\begin{scnrelfromvector}{введение}
		\scnfileitem{Разработка \textit{базы знаний} является трудоемким и продолжительным процессом, требующим высокого уровня квалификации \textit{разработчиков баз знаний}. Данный факт приводит к высокой себестоимости как самих \textit{баз знаний}, так и соответствующих им \textit{интеллектуальных систем}, а также к дефициту специалистов в области \textit{инженерии знаний}.}
		\scnfileitem{Расширение областей применения \textit{интеллектуальных систем} требует поддержки решения комплексных задач. Решение каждой такой задачи предполагает совместное использование различных видов знаний и моделей их представления, что приводит к компенсации недостатков одних моделей возможностями и достоинствами других.}
		\scnfileitem{Существующие \textit{средства создания баз знаний} предполагают, что процессы разработки и модификации базы знаний осуществляются отдельно от процесса ее использования, что приводит к дополнительному усложнению решения задачи обеспечения совместимости различного вида знаний. Отсутствие удовлетворительного решения этой задачи приводит к несовместимости \textit{компонентов баз знаний}, разрабатываемых для разных систем, и невозможности их повторного использования в других системах. Данный факт приводит к многократной повторной разработке содержательно одних и тех же компонентов для разных баз знаний.}
		\scnfileitem{Таким образом, актуальной является задача разработки модели \textit{баз знаний}, которая, с одной стороны, обеспечит общий унифицированный формальный фундамент для представления различных видов знаний в рамках одной \textit{базы знаний} и их совместного использования при решении комплексных задач, а с другой стороны, обеспечит возможность расширения числа видов знаний, используемых \textit{интеллектуальной системой}}
	\end{scnrelfromvector}
	\begin{scnindent}
		\begin{scnrelfromset}{источник}
			\scnitem{Davydenko2016}
		\end{scnrelfromset}
	\end{scnindent}
	
	\begin{scnrelfromlist}{ключевое понятие}
		\scnitem{база знаний}
		\scnitem{редактор баз знаний}
		\scnitem{разработка баз знаний}
		\scnitem{разработчик}
		\scnitem{противоречие}
		\scnitem{информационная дыра}
		\scnitem{многократно используемые компоненты}
	\end{scnrelfromlist}
	
	\begin{scnrelfromlist}{библиографическая ссылка}
		\scnitem{\scncite{Davydenko2016}}
		\scnitem{\scncite{Davydenko2017}}
		\scnitem{\scncite{Davydenko2018}}
		\scnitem{\scncite{Knowledge-base-editor2022}}
		\scnitem{\scncite{Arshinskiy2020}}
		\scnitem{\scncite{Rybina2007}}
		\scnitem{\scncite{Zhang2023}}
		\scnitem{\scncite{Narinjani2004}}
		\scnitem{\scncite{Ivashenko2011}}
		\scnitem{\scncite{Davydenko2013}}
	\end{scnrelfromlist}

	\begin{scnsubstruct}
		
		\scnheader{база знаний}
		\scntext{пояснение}{\textit{База знаний} является ключевым компонентом \textit{интеллектуальной системы}, которая в систематизированном виде включает в себя все знания, необходимые \textit{интеллектуальной системе} для ее функционирования.}
		
		\scnnote{Для задач, решаемых \textit{интеллектуальными системами}, в общем случае неизвестно, какие данные и знания должны быть использованы для решения этих задач, какие должны быть использованы методы их решения.
		При решении таких задач необходима локализация \textit{фрагмента базы знаний}, содержащего данные и знания, достаточные для решения задачи, и исключающего те данные и знания, которые заведомо для этого не нужны, а также выделение из имеющегося многообразия методов решения задач тех \textit{методов}, которых достаточно для решения данной задачи.}
	
		\scnnote{Для обеспечения совместного использования различных видов знаний в единой \textit{базе знаний} необходимо обеспечить совместимость этих видов знаний и, как следствие, совместимость \textit{компонентов баз знаний}}
	
		\scnheader{совместимость компонентов баз знаний}
		\scnsuperset{синтаксическая совместимость знаний}
		\begin{scnindent}
			\scntext{пояснение}{унификация формы представления знаний}
		\end{scnindent}
		\scnsuperset{семантическая совместимость знаний}
		\begin{scnindent}
			\scntext{пояснение}{однозначная и единая для всех компонентов трактовка используемых понятий}
		\end{scnindent}
		
		\scntext{проблема}{Существующие подходы к разработке \textit{баз знаний}, как правило, предполагают решение задачи обеспечения синтаксической совместимости знаний путем соединения разнородных моделей представления знаний, а также разработки новых интегрированных моделей и новых языков представления знаний.
		
		Разработка \textit{базы знаний} таким методом приводит к дополнительным накладным расходам при интеграции и обработке разнородных знаний, и, как следствие, к резкому увеличению трудозатрат при модификации таких \textit{баз знаний} и добавлении новых видов знаний.}
		
		\scnnote{\textit{интеллектуальная система}, способная решать комплексные задачи, должна обладать способностью приобретать новые знания и навыки в процессе ее эксплуатации, сохраняя при этом \textit{корректность} и \textit{целостность} \textit{базы знаний}.
		
		Это обуславливает требование модицифируемости к такой \textit{базе знаний}, то есть снижения трудоемкости внесения изменений в базу знаний.}
		
		
		\scnheader{\textit{база знаний} \textit{системы, способной решать комплексные задачи}}
			
		\begin{scnrelfromvector}{требование}
		
			\scnfileitem{возможность согласованного использования различных видов знаний, хранимых в одной \textit{базе знаний}, при решении каждой комплексной задачи}
			
			\scnfileitem{возможность реализации различных аспектов спецификации \textit{сущностей}, описываемых в \textit{базе знаний}}
			
			\scnfileitem{\textit{модифицируемость базы знаний}, позволяющая непосредственно в процессе эксплуатации \textit{интеллектуальной системы} добавлять в базу знаний новые фрагменты, в том числе новые виды знаний без внесения изменений в существующую структуру \textit{базы знаний}}
			
			\scnfileitem{возможность неограниченного перехода в рамках каждой \textit{базы знаний} от знаний к \textit{метазнаниям}, от метазнаний к \textit{метаметазнаниям} и так далее, что, в частности, предоставляет неограниченные возможности типологии и систематизации знаний, хранимых в составе \textit{базы знаний}, и неограниченные возможности \textit{декомпозиции} и \textit{структуризации} самой базы знаний}
			
			\scnfileitem{наличие языковых средств, позволяющих в рамках базы знаний представлять метазнания, описывающие \textit{качество базы знаний} (противоречия, неполноту, избыточность)}
			
			\scnfileitem{наличие языковых средств, позволяющих в рамках базы знаний представлять метазнания, описывающие \textit{историю эволюции} и \textit{планы дальнейшей эволюции} базы знаний}
			
			\scnfileitem{возможность для каждой решаемой задачи явно задавать и уточнять в ходе решения задачи \textit{область решения}, то есть такой фрагмент базы знаний, использование которого является достаточным для решения этой задачи}
		\end{scnrelfromvector}
		
		\scnheader{функциональность \textit{системы поддержки коллективной разработки гибридных баз знаний}}
		
		\begin{scnrelfromvector}{требование}
			\scnfileitem{обеспечение возможности как ручного, так и автоматического \textit{редактирования баз знаний}}
			
			\scnfileitem{обеспечение возможности \textit{автоматической верификации базы знаний}, в том числе анализ \textit{корректности} и \textit{полноты} \textit{базы знаний}}
			
			\scnfileitem{обеспечение возможности создания \textit{базы знаний} распределенным \textit{коллективом разработчиков}, включая механизм согласования вносимых в \textit{базу знаний} изменений, а также механизм хранения истории вносимых изменений с указанием авторства}
		\end{scnrelfromvector}
		
		\scnnote{Реализация перечисленных возможностей подразумевает отказ от работы с \textit{файлами исходных текстов базы знаний}. В данном случае предполагается, что все изменения осуществляются непосредственно в памяти системы, где хранится вся база знаний, что позволяет осуществлять разработку базы знаний компьютерной системы в процессе ее эксплуатации}
		\begin{scnindent}
			\begin{scnrelfromset}{источник}
				\scnitem{Davydenko2017}
			\end{scnrelfromset}
		\end{scnindent}
		
		\scnheader{Проектирование базы знаний}
		\scnsuperset{индивидуальный аспект проектирования}
		\begin{scnindent}
			\scntext{пояснение}{Индивидуальный представляет собой наполнение изолировнной части \textit{базы знаний} конкретным пользователем}
			\scnnote{При этом \textit{процесс редактирования базы знаний} должен быть максимально удобным и понятным для пользователя.}
			\begin{scnrelfromset}{включение}
				\scnitem{редакторы баз знаний}
				\scnitem{трансляторы баз знаний}
			\end{scnrelfromset}
		\end{scnindent}
		\scnsuperset{коллективный аспект проектирования}
		\begin{scnindent}
			\scntext{пояснение}{Коллективный представляет собой согласование всей \textit{базы знаний} и включает в себя процесс внесения предложений и внедрения изменений}
			\begin{scnrelfromset}{включение}
				\scnitem{процессы создания предложений по изменению базы знаний}
				\scnitem{процессы рассмотрения предложений по изменению базы знаний}
				\scnitem{процессы внесения предложений по изменению базы знаний}
			\end{scnrelfromset}
		\end{scnindent}
		
		\scnnote{Так как в основе любой современной базы знаний лежат \textit{онтологии}, то методы и средства разработки онтологий являются важнейшей частью технологий разработки баз знаний.}
		
		\scnheader{методология разработки онтологий} 
		\scntext{поясненение}{представляет собой набор инструкций и руководств, описывающих процесс выполнения сложных процедур разработки онтологий.
		Она детализирует различные задачи, как они должны быть выполнены, в каком порядке и каким образом осуществлять документирование работы по созданию онтологий}
	
		\scnrelfrom{разбиение}{\scnkeyword{Типология методологий по поддержке коллективной разработки\scnsupergroupsign}}
		\begin{scnindent}
			\begin{scneqtoset}
				\scnitem{методология, поддерживающая совместную коллективную разработку онтологии}
				\scnitem{методология, не поддерживающая совместную коллективную разработку онтологии}
			\end{scneqtoset}
		\end{scnindent}
		
		\scnrelfrom{разбиение}{\scnkeyword{Типология методологий по степени зависимости от инструментария\scnsupergroupsign}}
		\begin{scnindent}
			\begin{scneqtoset}
				\scnitem{методология, зависимая от инструментария}
				\scnitem{методология, полузависимая от инструментария}
				\scnitem{методология, независимая от инструментария}
			\end{scneqtoset}
		\end{scnindent}
		
		\scnrelfrom{разбиение}{\scnkeyword{Типология методологий по типу используемой модели жизненного цикла онтологии\scnsupergroupsign}}
		\begin{scnindent}
			\begin{scneqtoset}
				\scnitem{методология без указания модели жизненного цикла онтологии}
				\scnitem{методология с итеративной моделью жизненного цикла онтологии}
				\scnitem{методология с моделью жизненного цикла онтологии на основе эволюционного прототипирования}
				\scnitem{методология с моделью жизненного цикла приложения}
			\end{scneqtoset}
		\end{scnindent}
		
		\scnrelfrom{разбиение}{\scnkeyword{Типология методологий по возможности формализации\scnsupergroupsign}}
		\begin{scnindent}
			\begin{scneqtoset}
				\scnitem{методология, предусматривающая методы формализации}
				\scnitem{методология, не предусматривающая формализации}
			\end{scneqtoset}
		\end{scnindent}
		
		\scnrelfrom{разбиение}{\scnkeyword{Типология методологий по возможности повторного использования разрабатываемых онтологий\scnsupergroupsign}}
		\begin{scnindent}
			\begin{scneqtoset}
				\scnitem{методология, поддерживающая повторное использование}
				\scnitem{методология, не поддерживающая повторного использования}
			\end{scneqtoset}
		\end{scnindent}
		
		\scnrelfrom{разбиение}{\scnkeyword{Типология методологий по стратегии выделения концептов предметной области\scnsupergroupsign}}
		\begin{scnindent}
			\begin{scneqtoset}
				\scnitem{методология снизу вверх (bottom-up)}
				\scnitem{методология сверху вниз (top-down)}
				\scnitem{методология от середины (middle-out)}
				\scnitem{методология, сочетающая различные стратегии}
			\end{scneqtoset}
		\end{scnindent}
		
		\scnrelfrom{разбиение}{\scnkeyword{Типология методологий по возможности поддержки совместимости разрабатываемых онтологий\scnsupergroupsign}}
		\begin{scnindent}
			\begin{scneqtoset}
				\scnitem{методология, поддерживающая совместимость}
				\scnitem{методология, не поддерживающая совместимость}
			\end{scneqtoset}
		\end{scnindent}
		
		
		\scnnote{Большинство методологий не поддерживают совместную разработку \textit{баз знаний}, поддержку совместимости разрабатываемых \textit{баз знаний} и, как следствие, поддержку повторного использования уже разработанных \textit{баз знаний} и их \textit{компонентов}}
		
		\scnnote{Подавляющее большинство \textit{методологий разработки баз знаний} описывают процесс разработки в общих чертах, не регламентируя действия участников на каждом этапе разработки \textit{онтологии}, не уточняя принципы согласования новых \textit{понятий} с уже существующими, высоким оказывается субъективное влияние разработчиков}
		
		\scnnote{Отличительной особенностью предлагаемой методики от существующих \textit{методологий разработки баз знаний} является совершенствование \textit{базы знаний} \textit{коллективом разработчиков} непосредственно в процессе ее использования, а также создание новых и использование уже имеющихся \textit{компонентов баз знаний} в процессе разработки каждой \textit{базы знаний}}
		
		\scnnote{Данная методика предполагает два основных этапа --- этап создания стартовой версии разрабатываемой \textit{ostis-системы}, \textit{база знаний} которой синтезируется из \textit{компонентов}, входящих в \textit{библиотеку многократно используемых компонентов баз знаний ostis-систем}, и этап расширения и совершенствования \textit{базы знаний} разрабатываемой \textit{ostis-системы}, осуществляемый в рамках этой системы.}
		\begin{scnindent}
			\begin{scnrelfromset}{смотрите}
				\scnitem{Предметная область и онтология многократно используемых компонентов баз знаний ostis-систем}
			\end{scnrelfromset}
		\end{scnindent}
		\scnheader{стартовая версия \textit{ostis-системы}}
		\scntext{пояснение}{Стартовая версия \textit{ostis-системы} содержит набор знаний и средств решения задач, достаточный для дальнейшего развития системы}
		
		\begin{scnrelfromvector}{этапы создания}
			\scnfileitem{выбор и установка \textit{ostis-платформы} для интерпретации \textit{sc-модели ostis-системы}}
			
			\scnfileitem{установка \textit{Ядра sc-модели базы знаний ostis-системы} из \textit{библиотеки многократно используемых компонентов баз знаний}}
			
			\scnfileitem{установка \textit{Ядра решателей задач из библиотеки многократно используемых компонентов решателей задач}, то есть набора базовых \textit{многократно используемых компонентов решателей задач}, необходимых для работы стартовой версии \textit{ostis-системы}}
			
			\scnfileitem{установка \textit{Ядра sc-моделей интерфейсов}, то есть набора базовых \textit{многократно используемых компонентов пользовательского интерфейса ostis-систем}, необходимых для работы стартовой версии \textit{ostis-системы}}
			
			\scnfileitem{установка \textit{системы поддержки коллективной разработки гибридных баз знаний}}
		\end{scnrelfromvector}
		
		\scnheader{Процесс разработки базы знаний}
		\begin{scnrelfromvector}{включение}
			\scnfileitem{Формирование начальной структуры \textit{гибридной базы знаний}}
			\begin{scnindent}
				\begin{scnrelfromvector}{этапы}
					\scnfileitem{Формирование структуры разделов базы знаний, соответствующей варианту структуризации \textit{базы знаний} с точки зрения разработчиков}
					\scnfileitem{Выявление описываемых \textit{предметных областей} и
					построение иерархической системы описываемых \textit{предметных областей}}
					\scnfileitem{Построение иерархии разделов \textit{базы знаний} в рамках предметной	части базы знаний, учитывающей построенную на предыдущем этапе иерархию \textit{предметных областей}}
				\end{scnrelfromvector}
			\end{scnindent}
			\scnfileitem{Выявление \textit{компонентов базы знаний}, которые могут быть заимствованы из \textit{библиотеки многократно используемых компонентов баз знаний}, и включение их в состав разрабатываемой \textit{базы знаний}}
			\scnfileitem{Формирование проектных заданий на разработку недостающих \textit{фрагментов базы знаний} и распределение заданий между разработчиками}
			\scnfileitem{Разработка и согласование \textit{фрагментов базы знаний}, которые, в свою	очередь, могут в дальнейшем быть включены в состав \textit{библиотеки многократно используемых компонентов баз знаний}}
			\scnfileitem{Верификация и отладка базы знаний}
		\end{scnrelfromvector}
		
		\begin{scnindent}
			\scnnote{Следует отметить, что в процессе совершенствования базы знаний этапы формирования проектных заданий, разработки и согласования фрагментов базы знаний, верификации и отлади базы знаний выполняются циклически}
		\end{scnindent}
		
		\scnnote{Для обеспечения свойства рефлексивности \textit{интеллектуальной системы}, в частности, возможности автоматизации анализа истории эволюции базы знаний и генерации планов по ее развитию, вся деятельность, связанная с разработкой базы знаний, должна специфицироваться в самой этой базе знаний теми же средствами, что и предметная часть}
		
		\begin{scnindent}
			\begin{scnrelfromset}{источник}
				\scnitem{Davydenko2018}
			\end{scnrelfromset}
		\end{scnindent}
		
	\end{scnsubstruct}
	
\end{SCn}
