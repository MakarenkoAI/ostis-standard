\begin{SCn}
\scnsectionheader{Предметная область и онтология программ и языков программирования для ostis-систем}
\begin{scnsubstruct}

\scnheader{Предметная область программ и языков программирования для ostis-систем}
\scniselement{предметная область}
\begin{scnrelfromset}{автор}
    \scnitem{Зотов Н.В.}
    \scnitem{Шункевич Д.В.}
\end{scnrelfromset}

\scntext{аннотация}{Несмотря на активное развитие и использование современных технологий и языков программирования, общей семантической теории программ, на основе которой можно было бы проектировать и разрабатывать прикладные системы, на данный момент не существует. В данной предметной области предлагается семантическая теория программ для ostis-систем. Рассматриваются особенности представления и ключевые моменты процесса интерпретации программ в ostis-системах.}

\begin{scnreltovector}{конкатенация сегментов}
    \scnitem{Проблемы текущего состояния в области разработки и применения языков программирования}
    \scnitem{Существующие онтологии языков программирования}
    \scnitem{Предлагаемый подход к разработке технологий программирования для ostis-систем}
    \scnitem{Синтаксис и семантика программ в ostis-системах}
    \scnitem{Методы и средства поддержки проектирования и разработки программ в ostis-системах}
    \scnitem{Комплекс свойств, определяющих эффективность программ в ostis-системах}
\end{scnreltovector}

\begin{scnhaselementrolelist}{ключевой знак}
    \scnitem{Семантическая теория программ}
	\scnitem{Семантическая теория программ для ostis-систем}
	\begin{scnindent}
		\scnidtf{Предлагаемый вариант теории для проектирования языков программирования и программ для интеллектуальных компьютерных систем нового поколения}
	\end{scnindent}
\end{scnhaselementrolelist}

\begin{scnhaselementrolelist}{класс объектов исследования}
	\scnitem{метод}
	\scnitem{язык представления методов}
	\scnitem{эффективность метода}
\end{scnhaselementrolelist}

\begin{scnhaselementrolelist}{исследуемое отношение}
	\scnitem{спецификация метода*}
	\scnitem{синтаксис метода*}
	\scnitem{денотационная семантика метода*}
	\scnitem{операционная семантика метода*}
	\scnitem{спецификация языка представления методов*}
	\scnitem{синтаксис языка представления методов*}
	\scnitem{денотационная семантика языка представления методов*}
	\scnitem{операционная семантика языка представления методов*}
\end{scnhaselementrolelist}

\begin{scnrelfromlist}{библиографическая ссылка}
	\scnitem{\scncite{Sebesta2012}}
	\scnitem{\scncite{Zapata2010}}
	\scnitem{\scncite{Golenkov2019a}}
	\scnitem{\scncite{Penta2020}}
	\scnitem{\scncite{Scalabrino2016}}
	\scnitem{\scncite{Golenkov2012}}
	\scnitem{\scncite{Brooks2021}}
	\scnitem{\scncite{Sellitto2022}}
	\scnitem{\scncite{Turner2007}}
	\scnitem{\scncite{Chaparro2014}}
	\scnitem{\scncite{Golenkov2022a}}
	\scnitem{\scncite{Golenkov2019}}
	\scnitem{\scncite{Eden2007}}
	\scnitem{\scncite{Lando2007}}
	\scnitem{\scncite{Lando2007a}}
	\scnitem{\scncite{Turner2014}}
	\scnitem{\scncite{Deikstra1978}}
	\scnitem{\scncite{Standart2021}}
	\scnitem{\scncite{Kasyanov2003}}
	\scnitem{\scncite{Petrov1978}}
	\scnitem{\scncite{Scott2006}}
	\scnitem{\scncite{Scott1972}}
	\scnitem{\scncite{Orlov2021}}
	\scnitem{\scncite{Lu2022}}
	\scnitem{\scncite{Gulykina2012}}
	\scnitem{\scncite{Pivovarchik2016}}
	\scnitem{\scncite{Ford2019}}
	\scnitem{\scncite{IMS}}
	\scnitem{\scncite{Pivovarchik2013}}
	\scnitem{\scncite{Tin1995}}
	\scnitem{\scncite{Schiitze1991}}
	\scnitem{\scncite{Black1993}}
	\scnitem{\scncite{Zotov2022a}}
\end{scnrelfromlist}

\begin{scnrelfromvector}{введение}
    \scnfileitem{За долгий период развития компьютерных систем практически сняты аппаратные ограничения на решение различных задач. Оставшиеся ограничения отводятся на долю программного обеспечения. Прежде всего эти ограничения связаны с текущими проблемами развития программного обеспечения}
    \begin{scnindent}
        \begin{scnrelfromset}{проблемы текущего состояния}
            \scnfileitem{\uline{Аппаратная сложность опережает} умение человечества строить \uline{программные компьютерные системы}, использующее потенциальные возможности аппаратуры.}
            \scnfileitem{Навыки и \uline{технологии} разработки программ \uline{отстают от требований}, предъявляемых к разработке программ нового поколения.}
            \scnfileitem{Возможностям эксплуатировать существующие программы угрожает \uline{низкое качество их разработки}.}
        \end{scnrelfromset}
        \scntext{решение}{Ключом к решению этих проблем является глубокое понимание и грамотное использование существующих \textit{языков программирования} как основного инструмента для массового создания \textit{программных компьютерных систем нового поколения}.}
    \end{scnindent}
    \scnfileitem{В данной предметной области акцент делается на достижение следующих результатов
        \begin{itemize}
            \item (1) изложение классических основ, отражающих накопленный мировой опыт в области разработки и применения современных \textit{языков программирования};
            \item и (2) систематизация основных результатов в этой области в виде единой унифицированной \textit{Семантической теории программ для интеллектуальных компьютерных систем нового поколения}, построенных по принципам \textit{Технологии OSTIS}.
        \end{itemize}}
    \scnfileitem{В данной предметной области подробно описываются проблемы текущего состояния в области \textit{технологий} и \textit{языков программирования}. Она посвящена базовым понятиям \textit{теории языков программирования}, дается обзорная характеристика областей применения \textit{языков программирования}, достаточно востребованных современным человеческим обществом, рассматриваются способы представления и интерпретации \textit{программ} различных \textit{языков программирования}, подробно описываются формы и содержание критериев для оценки \textit{эффективности языков}.}
\end{scnrelfromvector}

\scntext{примечание}{Проектирование и реализация \textit{программы} на каком-либо \textit{языке программирования} должна сводиться к описанию ее \textit{синтаксиса} и \textit{денотационной семантики} в базе знаний ostis-системы с помощью некоторой библиотеки предметных областей и онтологий программ, описываемой в рамках этой базы знаний. Для этого нужна онтология программ, которые позволили бы в достаточном объеме описывать программы на любых языках программирования в ostis-системах. Такой подход позволяет не только описывать сложноструктурированные объекты простым и понятным языком, но и позволяет унифицировать представление различных видов знаний. Тем самым, информация о программах и сами программы представляются на одном и том же языке (имеют один синтаксис), но содержательно описываются при помощи разных онтологий. Таким образом, \uline{решением всех проблем будет являться общая теория программ}, однозначно соответствующей некоторой онтологии программ, c помощью которых можно было бы описывать синтаксис и денотационную семантику любых программ в ostis-системах.}
\scntext{примечание}{Таким образом, результатом данной предметной области является \textit{Предметная область и онтология программ} (далее --- Предметная область и онтология методов), с помощью которой можно описывать синтаксис, денотационную и операционную семантику различных методов в ostis-системах. \textit{Предметная область и онтология методов} является дочерней предметной областью по отношению к \textit{Предметной области и онтологии информационных конструкций и языков}. Это означает, что она наследует все свойства исследуемых в ней понятий и отношений.}
	
\scnsegmentheader{Проблемы текущего состояния в области разработки и применения языков программирования}
\begin{scnsubstruct}
    \begin{scnhaselementset}
        \scnfileitem{Поскольку количество \textit{языков программирования} растет с увеличением потребности в них, то растут и потребности в описании этих \textit{языков программирования} для дальнейшего использования и проектирования прикладных систем. Это в свою очередь требует высокого уровня качества спецификации конкретного языка: и описания \textit{синтаксиса} и семантики конструкций этого языка, и описания средств и методов реновации инструментальных средств, обеспечивающих интерпретацию или трансляцию этого языка. То есть, с увеличением количества \textit{языков программирования} растет не только многообразие форм представления знаний (\textit{языков программирования}), но и количество \textit{программных компьютерных систем} на различных формах представления знаний.}
        \scnfileitem{Большое многообразие форм представления знаний, как говорилось выше, предоставляет большой спектр возможностей проектирования \textit{программных компьютерных систем} на каждой из них. Получается, чтобы произвести интеграцию нескольких \textit{программных компьютерных систем}, реализованных на разных \textit{языках программирования}, необходимо сделать так, чтобы системы могли коммуницировать между собой на каждом из тех языков, на котором они реализованы. Так, стремление к использованию существующих программных компонентов затрудняется реализацией самих компонентов, поскольку чтобы объединить эти компоненты необходимо изменить их программный код. Наличие многообразия форм затрудняет реализацию \textit{совместимых интероперабельных программных компьютерных систем}.}
        \scnfileitem{С ростом сложности программного кода, уменьшается количество способных понять его смысл. Современные разработчики создают \textit{программные компьютерные системы}, не учитывая полный ее жизненный цикл. Системы должны постоянно обновляться и совершенствоваться с развитием технологий, на которых она основана. Это должно обеспечиваться хорошей документацией реализации компонентов этих систем --- это снижает не только потребности в привлечении новых ресурсов и кадров, но и способствует снижению реинжиниринга \textit{программных компьютерных систем}.}
        \scnfileitem{Полная автоматизация проектирования \textit{программных компьютерных систем} невозможна, поскольку современные языки, на которых они проектируются не имеют свойства рефлексивности --- системы не могут познавать и понимать себя и развиваться в полной мере самостоятельно. Таким образом, существующие \textit{программные компьютерные системы} не являются как таковыми интеллектуальными, потому что не имеют необходимых им свойств.}
        \scnfileitem{Ключом к легкому и глубокому освоению конкретного языка как основного профессионального инструмента программиста является понимание общих принципов построения и применения языков программирования, описываемых их общей теорией. До сегодняшнего дня, общей \textit{Семантической теории языков программирования} до сих пор не существует, что затрудняет разработку, верификацию и использование новых и существующих \textit{языков программирования}. Без общей теории каждый может разрабатывать принципиально общие методы и средства так, как хочется, а не так, как требуется.}
        \scnfileitem{Достижение максимума услуг и средств при минимуме затрат возможно только путем глубокого понимания принципов построения \textit{языков программирования} за счет простоты средств и методов представления знаний. Сложное нужно сводить к простому и изъяснять простыми понятиями, не создавая дополнительной иллюзии важности.}
    \end{scnhaselementset}
    \begin{scnrelfromset}{смотрите}
        \scnitem{\scncite{Zapata2010}}
        \scnitem{\scncite{Golenkov2019a}}
        \scnitem{\scncite{Penta2020}}
        \scnitem{\scncite{Scalabrino2016}}
        \scnitem{\scncite{Golenkov2012}}
        \scnitem{\scncite{Brooks2021}}
        \scnitem{\scncite{Sellitto2022}}
        \scnitem{\scncite{Penta2020}}
        \scnitem{\scncite{Scalabrino2016}}
        \scnitem{\scncite{Turner2007}}
        \scnitem{\scncite{Golenkov2012}}
        \scnitem{\scncite{Sellitto2022}}
        \scnitem{\scncite{Chaparro2014}}
        \scnitem{Комплекс свойств, определяющий общий уровень качества кибернетической системы}
    \end{scnrelfromset}
    \begin{scnrelfromvector}{введение}
        \scnfileitem{В современную эру развития информационных технологий существует огромное количество \textit{языков программирования}, каждый из которых имеет свое важное назначение в области проектирования \textit{программных компьютерных систем}. Многообразие \textit{языков программирования} и решений, созданных на них, настолько велико, что очень легко потеряться в море информации о всех аспектах применения и проектирования \textit{языков программирования}. Кроме этого, основная проблема заключается не в количестве существующих решений в области разработки и применения современных \textit{языков программирования}, а количестве форм (!), на которых представляются конкретные \textit{языки программирования}. Так, \textit{декларативные знания}, то есть знания, являющиеся, например, спецификацией какой-то программы, и \textit{процедурные знания}, то есть знания, которые являются программами, принадлежащими какому-то \textit{языку программирования}, представляются совершенно различными способами, методами и средствами.}
        \begin{scnindent}
        	\begin{scnrelfromset}{смотрите}
        		\scnitem{\scncite{Sebesta2012}}
        	\end{scnrelfromset}
        \end{scnindent}
        \scnfileitem{Все рассматриваемые проблемы связаны и являются проблемами текущего состояния направлений развития в области \textit{Искусственного интеллекта}.}
        \begin{scnindent}
        	\begin{scnrelfromset}{смотрите}
        		\scnitem{\scncite{Golenkov2022a}}
        	\end{scnrelfromset}
        \end{scnindent}
        \scnfileitem{Итак, для решения перечисленных проблем необходимо создавать комфортные условия для реализации \textit{программных компьютерных систем}, семантически совместимых и интероперабельных между собой. В контексте \textit{языков программирования} необходима общая \textit{Семантическая теория программ для интеллектуальных компьютерных систем нового поколения}, которая:
        \begin{itemize}
            \item \uline{позволит} без больших усилий и затрат \uline{интегрировать имеющиеся решения} в области проектирования программ компьютерных систем;
            \item \uline{объединит формы представления знаний} декларативного и процедурного вида;
            \item \uline{будет иметь широкий спектр средств} не только для описания синтаксиса и семантики существующих языков программирования, но и для проектирования новых аналогов;
            \item \uline{будет понятна} не только человеку, но и машине;
            \item \uline{обозначит принципы}, по которым необходимо проектировать \textit{языки программирования нового поколения}.
        \end{itemize}}
    	\begin{scnindent}
    		\begin{scnrelfromset}{смотрите}
    			\scnitem{\scncite{Golenkov2019}}
    			\scnitem{\scncite{Zapata2010}}
    		\end{scnrelfromset}
    	\end{scnindent}
        \scnfileitem{К проектированию таких общих теорий, строго говоря, нужно подходить с высокой степенью важности. Проектируемые \textit{компьютерные системы} должны всегда иметь возможности использовать те свойства, которые им начертаны. Для того, чтобы и эта теория могла быть использована как некоторая система знаний о том, как надо проектировать и использовать \textit{языки программирования} и программы в \textit{программных компьютерных системах}, и том, как интерпретировать их \textit{программы}, необходимо, чтобы эта теория была описана средствами и методами, которыми проектируются эти \textit{программные компьютерные системы}. Речь идет о том, что принципиально важным подходом к проектированию общей теории программ является \textit{онтологический подход}.}
        \begin{scnindent}
        	\begin{scnrelfromset}{смотрите}
        		\scnitem{\scncite{Golenkov2019}}
        		\scnitem{\scncite{Zapata2010}}
        	\end{scnrelfromset}
        \end{scnindent}
        \scnfileitem{Для воплощения данных идей необходимо изучить и интегрировать опыт, накопленный в области разработки и применения \textit{современных языков программирования}. Поэтому далее будут рассмотрены результаты других исследований в области проектирования общей теории языков программирования и программ.}
    \end{scnrelfromvector}
\end{scnsubstruct}

\scnsegmentheader{Существующие онтологии языков программирования}
\begin{scnsubstruct}
    
    \scnheader{Семантическая теория программ}
    \scntext{примечание}{В большинстве, идеи, предлагаемые в научных работах по исследованию языков программирования, безусловно являются востребованными и полезными для проектирования \textit{программных компьютерных систем}. Так, идея о том, что языки программирования и программы, реализуемые на них, должны быть организованы в общую таксономию понятий, является основополагающей, поскольку обеспечивает наиболее качественную среду для проектирования и реализации \textit{программных компьютерных систем}. Общая теория программ нужна не только для того чтобы описывать термины и понятия как некоторую спецификацию, используемую для проектирования \textit{программных компьютерных систем} (что тоже немаловажно), но и для того, чтобы определять качество языков программирования и программ по таким вопросам, как: \scnqq{Является ли данный язык языком программирования}, \scnqq{Является ли данное знание программой}, \scnqq{Насколько эффективна данная программа}, \scnqq{Какова степень интеллекта данной программной системы} и так далее. Данные идеи предложены и рассмотрены в работах Raymond Turner.}
    \begin{scnindent}
    	\begin{scnrelfromset}{смотрите}
    		\scnitem{\scncite{Eden2007}}
    		\scnitem{\scncite{Turner2007}}
    	\end{scnrelfromset}
    \end{scnindent}
    
   	\scnheader{онтология языков программирования и программ}
    \scntext{примечание}{До сегодняшнего дня существует большое количество аналогов онтологий языков программирования и программ. Также стоит отметить разработанные онтологии программ, система понятий в которых определяется строго и однозначно на формальных языках: языках логики и языках описания грамматик формальных языков. Однако ни одна из них не является таким результатом, который можно было бы использовать при проектировании \textit{программных компьютерных систем} без существенных проблем. Разработанные онтологии сосредотачивают в себе лишь краткое описание связанных между собой понятий, но общей картины того, как данные онтологии можно использовать в конкретных задачах, почти не видно.}
    \begin{scnindent}
    	\begin{scnrelfromset}{смотрите}
    		\scnitem{\scncite{Lando2007}}
    		\scnitem{\scncite{Lando2007a}}
    		\scnitem{\scncite{Turner2014}}
    		\scnitem{\scncite{Turner2007}}
    	\end{scnrelfromset}
    \end{scnindent}
    
    \scnheader{язык представления методов}
    \scntext{примечание}{Сегодня встречаются и вовсе протовоположные суждения о назначении программ и языков программирования, противоречащие формальным основам Искусственного интеллекта. \textit{Программные компьютерные системы} должны быть не только понятны человеку, но и сами должны понимать себя, свои возможности, намерения, действия и цели, и понимать себе подобные кибернетические системы. Только таким образом человечество и результаты его деятельности в виде каких-то конкретных систем смогут работать сообща, дополняя друг друга и преумножая свои результаты.}
    \begin{scnindent}
    	\begin{scnrelfromset}{смотрите}
    		\scnitem{\scncite{Golenkov2012}}
    	\end{scnrelfromset}
    \scntext{примечание}{В результате анализа приведенных работ можно сделать вывод о том, что:
        \begin{itemize}
            \item \textit{общей теории программ и языков программирования}, которая могла быть задействована при решении любой прикладной задачи и представлении и реализации средств проектирования компьютерных систем, до сих пор не существует;
            \item унификация представления средств описания и реализации по этим описании как главный аргумент к оперированию смысловому представлению знаний, к полному взаимопониманию между \textit{программными компьютерными системами} вовсе не рассматривается;
            \item \textit{программы} и совокупности этих \textit{программ} в виде \textit{программных компьютерных систем} реализуются в большинстве случаев в индивидуальном порядке и плохо документируются, что усложняет их использование, интеграцию с другими программами и \textit{программными компьютерными системами}, тестирование и совершенствование.
        \end{itemize}}
	\end{scnindent}
\end{scnsubstruct}

\scnsegmentheader{Предлагаемый подход к разработке технологий программирования для ostis-систем}
\begin{scnsubstruct}
    \begin{scnrelfromlist}{ключевой знак}
        \scnitem{Принципы программирования в интеллектуальных компьютерных системах нового поколения}
    \end{scnrelfromlist}
    
    \scnheader{язык представления методов}
    \begin{scnrelfromset}{проблемы}
        \scnfileitem{Поскольку количество \textit{языков программирования} растет с увеличением потребности в них, то растут и потребности в описании этих \textit{языков программирования} для проектирования и разработки \textit{программных компьютерных систем} на этих языках. То есть, с увеличением количества \textit{языков программирования} растет не только многообразие форм представления знаний, но и количество \textit{программных компьютерных систем} на различных формах представления знаний.}
        \scnfileitem{Многообразие форм представления знаний в свою очередь требует не только качественной спецификации конкретного \textit{языка программирования} для разработки \textit{программ} на этом языке, но и новых требований к существующим разработчикам.}
        \scnfileitem{Новые требования к существующим разработчикам влекут за собой появление барьеров и для создания семантически совместимых и интероперабельных \textit{программных компьютерных систем}, и для обеспечения благоприятной среды для взаимодействия их разработчиков.}
    \end{scnrelfromset}
    \begin{scnindent}
	    \scntext{решение}{Для преодоления данных проблем нет необходимости пересматривать уже существующие решения в области разработки программного обеспечения. Необходимо создавать принципиально новые \textit{языки программирования}, а также реализовывать \textit{программные компьютерные системы} на них, в которых будут учтены и решены существующие проблемы. Для этого следует учитывать следующие \textit{принципы программирования} этих систем.}
        \begin{scnindent}
            \begin{scnrelfromset}{принципы программировния}
                \scnfileitem{Расширение многообразия форм представления знаний происходит за счет появления новых синтаксических конструкций в \textit{языках программирования}. Поэтому разработка \textit{языков программирования} должна сводиться к уточнению \textit{синтаксиса} и \textit{семантики} уже существующих \textit{языков программирования}. При этом все \textit{языки программирования} должны являться подъязыками некоторого базового \textit{языка программирования}.}
                \scnfileitem{Нет необходимости в создании дополнительных языков, с помощью которых можно описывать семантику программ на \textit{языках программирования}. Наоборот, \textit{язык программирования}, на котором разрабатываются программы, должен позволять своими же средствами описывать \textit{семантику} \textit{программ} на этом же языке.}
                \scnfileitem{Документирование \textit{программ}, в том числе \textit{программных компьютерных систем}, должно минимизироваться за счет этапов их качественного проектирования и разработки. Смысл конструкций \textit{программ} \textit{языков программирования} должен быть настолько ясным и понятным, чтобы использование \textit{программ} на этом \textit{языке программирования} не требовало дополнительных ресурсов и инструментов как и у разработчиков этих программ и систем, таких и у новых разработчиков.}
                \scnfileitem{Появление новых программ должно влечь за собой к расширению \textit{Библиотеки многократно используемых программ} и к уменьшению количества семантически эквивалетных программ. Таким образом, программы должны быть не только максимальным образом совместимыми между собой, но и открытыми для переиспользования в других \textit{программных компьютерных системах нового поколения}.}
                \scnfileitem{Полный жизненный цикл разработки новых программ должен обеспечиваться теми же средствами и \textit{языками программирования}, на которых разрабатываются эти программы.}
                \scnfileitem{Сложность программ и \textit{программных компьютерных систем} должна сводиться к минимуму. То, что выглядит сложно, должно и может быть сделано максимально просто.}
                \scnfileitem{Построение качественного коллектива \textit{программных компьютерных систем} может быть обеспечено только совместимостью и интероперабельностью самих систем, и коллективов тех разработчиков, которые их создают.}
                \scnfileitem{Ключом к решению всех этих проблем является общая \textit{Технология проектирования компьютерных систем нового поколения}, на базе которой можно построить общую \textit{Семантическую теорию программ} (дисциплину программирования) для \textit{интеллектуальных компьютерных систем нового поколения}, построенных по принципам \textit{Технологии OSTIS}.}
                \begin{scnindent}
                    \begin{scnrelfromset}{смотрите}
                        \scnitem{\scncite{Deikstra1978}}
                    \end{scnrelfromset}
                \end{scnindent}
            \end{scnrelfromset}
        \end{scnindent}
        \scntext{примечание}{Почему \textit{Технология OSTIS} является ключом к решению описанных проблем в области проектирования и применения \textit{языков программирования}?}
        \begin{scnindent}
            \begin{scnrelfromset}{ответ}
                \scnfileitem{Стандарт Технологии OSTIS уже реализует базовые средства, необходимые для проектирования и разработки интероперабельных \textit{программных компьютерных систем}, в основе которых лежит смысловое представление знаний. Это устраняет не только необходимость создания \textit{онтологий верхнего уровня}, которые должны быть использованы в общей теории программ как базовые для описания понятий этой теории, но и помогает проектировать решения согласованно с другими онтологиями. В результате формируется общая слаженная картина мира, которая (1) непротиворечива, то есть согласована, (2) однозначно трактуема, (3) универсальна и, (4) самое главное, понятна для каждого.}
                \scnfileitem{\textit{Технология OSTIS} проектируется одним языком унифицированного представления знаний, называемым \textit{SC-кодом}. Смысл \textit{программ} и \textit{языков программирования} понятен и однозначен тогда и только тогда, когда этот смысл описывается на одном общем языке, понятному любой \textit{кибернетической системе}.}
                \scnfileitem{\textit{SC-код} синтаксически минимален. Для описания объектов и связей между ними используется минимальное количество знаков. В то же время многообразие этих связей сводится к многобразию знаковых конструкций. Все это обеспечивается за счет представления информации в виде графовых структур.}
                \scnfileitem{SC-код не просто удобен для описания и проектирования каких-то сложных объектов --- с его помощью можно проектировать и реализовывать любые \textit{языки представления знаний}, в том числе программ, компьютерные системы и, вообще, описывать реальный мир.}
                \scnfileitem{Онтологический и компонентный подходы к проектированию любых сложных объектов обеспечивают выполнение главных принципов, по которым должны проектироваться современные системы. То, что реализовано и можно использовать, нужно переиспользовать везде.}
            \end{scnrelfromset}
            \begin{scnrelfromset}{смотрите}
                \scnitem{\scncite{Kasyanov2003}}
                \scnitem{\scncite{Petrov1978}}
            \end{scnrelfromset}
        \end{scnindent}
    \end{scnindent}

    \scnheader{Предметная область и онтология информационных конструкций и языков}
    \begin{scnrelfromlist}{дочерняя предметная область и онтология}
        \scnitem{Предметная область и онтология языков}
        \begin{scnindent}
            \begin{scnrelfromlist}{дочерняя предметная область и онтология}
                \scnitem{Предметная область и онтология естественных языков}
                \scnitem{Предметная область и онтология формальных языков}
            \end{scnrelfromlist}
        \end{scnindent}
    \end{scnrelfromlist}

    \scnheader{Предметная область и онтология формальных языков}
    \begin{scnrelfromlist}{дочерняя предметная область и онтология}
        \scnitem{Предметная область и онтология языков представления знаний}
        \begin{scnindent}
            \begin{scnrelfromlist}{дочерняя предметная область и онтология}
                \scnitem{\scnkeyword{Предметная область и онтология методов}}
            \end{scnrelfromlist}
        \end{scnindent}
    \end{scnrelfromlist}

    \scnheader{Предметная область и онтология методов}
    \begin{scnrelfromlist}{дочерняя предметная область и онтология}
        \scnitem{Предметная область и онтология методов ostis-систем}
        \begin{scnindent}
            \begin{scnrelfromlist}{дочерняя предметная область и онтология}
                \scnitem{Предметная область и онтология процедурных методов ostis-систем}
            \end{scnrelfromlist}
        \end{scnindent}
    \end{scnrelfromlist}

	\scnheader{язык программирования}
    \scntext{примечание}{Каждая теория должна быть согласована понятийно. Несмотря на то, что в литературе сложилась разное трактование понятия \textit{языка программирования}, должно быть одно универсальное. Для этого вместо языков программирования далее \uline{будем говорить о языках представления методов}, а вместо программ этих языков программирования --- о методах как знаковых конструкциях языков представления методов. Такое решение обосновывается тем, что обычно язык выступает в роли инструмента какого-то знания определенного вида, а термин \textit{языка программирования} является вырожденным, поскольку стоит говорить не о языках, на которых что-то можно программировать, а о языках, на которых можно представлять знания определенного вида, в данном случае --- знания процедурного типа. Сами термины \scnqqi{языка программирования} и \scnqqi{программы} будем считать неосновными идентификаторами понятий \scnqqi{языка представления методов} и \scnqqi{метода}, соответственно. Также это правило применяется на все понятия, используемые в данной главе и содержащие термин \scnqqi{метод}.}
    
    \scnheader{Семантическая теория программ в ostis-системах}
    \scntext{примечание}{Следует отметить, что общая \textit{Семантическая теория программ в ostis-системах} не отрицает весь накопленный опыт в сфере разработки современных \textit{технологий программирования}. Наоборот, предлагаемая в данной главе идея позволяет переиспользовать те проверенные инструменты и методы для наиболее быстрой и качественной реализации программ в сложных \textit{программных компьютерных системах}.}
\end{scnsubstruct}

\scntext{заключение}{Данная предметная область является началом \textit{Семантической теории программ для компьютерных систем нового поколения}. Логичным развитием данной предметной области будут:
    \begin{itemize}
        \item уточнение и дополнение понятий \textit{Предметной области и онтологии методов} для достижения полноты теории;
        \item описание дочерних предметных областей \textit{Предметной области и онтологии методов} для конкретных видов методов, а также уточнение денотационной и операционной семантики спецификации этих методов;
        \item описание возможных путей реализации метаметодов интерпретации методов различных я.п.м.;
        \item формализация математических моделей для подсчета оценок эффективности методов.
    \end{itemize}}

\end{scnsubstruct}
\end{SCn}
