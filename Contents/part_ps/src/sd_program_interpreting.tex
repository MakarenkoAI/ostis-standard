\begin{SCn}
\scnsectionheader{Предметная область и онтология интерпретации современных языков программирования в ostis-системах}
\begin{scnsubstruct}

\scnheader{Предметная область интерпретации современных языков программирования в ostis-система}
\scniselement{предметная область}
\begin{scnrelfromset}{автор}
    \scnitem{Зотов Н.В.}
    \scnitem{Шункевич Д.В.}
\end{scnrelfromset}

\begin{scnreltovector}{конкатенация сегментов}
    \scnitem{Синтаксис и семантика программ в ostis-системах}
    \scnitem{Методы и средства поддержки проектирования и разработки программ в ostis-системах}
    \scnitem{Комплекс свойств, определяющих эффективность программ в ostis-системах}
\end{scnreltovector}

\scnsegmentheader{Синтаксис и семантика программ в ostis-системах}
\begin{scnsubstruct}
    \begin{scnreltovector}{конкатенация сегментов}
        \scnitem{Синтаксис программ в ostis-системах}
        \scnitem{Денотационная семантика программ в ostis-системах}
        \scnitem{Операционная семантика программ в ostis-системах}
        \scnitem{Синтаксис и семантика языков программирования в ostis-системах}
    \end{scnreltovector}
    \begin{scnhaselementrolelist}{класс объектов исследования}
        \scnitem{спецификация метода*}
        \scnitem{синтаксис метода*}
        \scnitem{денотационная семантика метода*}
        \scnitem{операционная семантика метода*}
        \scnitem{спецификация языка представления методов*}
        \scnitem{синтаксис языка представления методов*}
        \scnitem{денотационная семантика языка представления методов*}
        \scnitem{операционная семантика языка представления методов*}
    \end{scnhaselementrolelist}
    \begin{scnhaselementrolelist}{ключевой знак}
        \scnitem{Принципы описания синтаксиса и семантики программ в ostis-системах}
        \scnitem{Язык SCP}
    \end{scnhaselementrolelist}
    
    \scnheader{семантика метода $\cup$ синтаксис метода}
    \scntext{примечание}{\textit{синтаксис} и \textit{семантика метода} составляют \textit{спецификацию*} этого \textit{метода}. \textit{Семантику метода} можно рассматривать в двух аспектах: как множество знаний, связанных между собой (то есть \textit{денотационную семантику} данного \textit{метода}), и как знание, которое может быть интерпретировано другим методом (то есть \textit{операционную семантику} данного \textit{метода}).}
\end{scnsubstruct}

\scnsegmentheader{Синтаксис программ в ostis-системах}
\begin{scnsubstruct}
	
	\scnheader{метод}
    \scntext{примечание}{Любой \textit{метод} состоит из \textit{информационных конструкций}, которые задают порядок действий в базе знаний, с помощью которых нужно перейти от исходного состояния к \uline{целевому}, решив таким образом какую-то конкретную задачу. Так, например, в процедурном методе любой такой оператор представляет собой некоторую математическую функцию. Для композиции этих функций в более крупные фрагменты используются выражения и операторы. В свою очередь, линейные последовательности операторов и условные ветвления также могут быть представлены функциями, составленными из функций отдельных компонентов этих конструкций. Цикл легко описывается рекурсивной функцией, составленной из компонентов, входящих в его тело.}

	\scnheader{Синтаксис языков представления методов}
    \scntext{примечание}{\textit{Синтаксис языков представления методов} в ostis-системах может быть формально описан различными способами. Так, например, можно использовать метаязык Бэкуса-Наура для описания синтаксиса любых \textit{языков представления методов}. Другими не менее известными формами представления методов являются контекстно-свободные грамматики, расширенная форма Бэкуса-Наура, синтаксические графы.}
    \begin{scnindent}
    	\begin{scnrelfromset}{смотрите}
    		\scnitem{\scncite{Scott1972}}
    		\scnitem{\scncite{Scott2006}}
    		\scnitem{\scncite{Sebesta2012}}
    	\end{scnrelfromset}
        \scntext{примечание}{Однако значительно более логично и целесообразно описывать \textit{синтаксис} других языков на универсальном \textit{языке представления знаний} --- \textit{SC-коде}. Такой подход позволит ostis-системам самостоятельно понимать, анализировать и генерировать тексты указанных языков на основе принципов, общих для любых форм внешнего представления информации, в том числе нелинейных. Таким образом, языки, написанные на \textit{SC-коде}, имеют такой же синтаксис как и сам \textit{SC-код}.}
            \begin{scnindent}
            \begin{scnrelfromset}{смотрите}
                \scnitem{\scncite{Petrov1978}}
            \end{scnrelfromset}
        \end{scnindent}
    \end{scnindent}

\end{scnsubstruct}

\scnsegmentheader{Денотационная семантика программ в ostis-системах}
\begin{scnsubstruct}
	
	\scnheader{семантика метода}
    \scntext{примечание}{\textit{семантика метода} разъясняет смысл \textit{синтаксических конструкций метода}. Наиболее распространенными методами описания семантики \textit{языков программирования} являются: денотационной, операционный, аксиоматический, алгебраический. На базе принципов Технологии OSTIS, под семантикой метода будем подразумевать объединение \textit{денотационной} и \textit{операционной семантики метода}.}
    \begin{scnindent}
    	\begin{scnrelfromset}{смотрите}
    		\scnitem{\scncite{Orlov2021}}
    	\end{scnrelfromset}
    \end{scnindent}
    \scntext{примечание}{С помощью \textit{SC-кода} можно представлять и те языки, которые не написаны на нем. Проблема будет в том, что форма и смысл языка и его методов будут разделены, то есть будут представлены по-разному. В данном случае \textit{SC-код} выступает мощным инструментом для интеграции спецификаций различных языков внешнего представления знаний. Однако стоит отметить, что в представлении различных форм методов, принадлежащих разным \textit{языкам представления методов}, в рамках \textit{Технологии OSTIS} нет необходимости. Это объясняется тем, что:
        \begin{itemize}
            \item \textit{SC-код} является достаточно универсальным языком для представления любых видов знаний. Это означает, что различные формы алгоритма решения одной и той же задачи можно свести к минимуму. В \textit{SC-коде} фундаментом является формальная теория, что обеспечивает универсальное представление различных видов декларативных и процедурных знаний. Так, \textit{логические программы} можно представлять в виде \textit{процедурных программ}, в которых в качестве операндов операторов будут не только \textit{логические формулы} и \textit{правила вывода}, но и другие методы, обеспечивающее интерпретацию этих \textit{логических формул} при помощи правил вывода. Таким образом, \textit{SC-код} можно называть не только языком унифицированного представления знания, но и языком, на котором можно решать различные классы задачи одним и тем же способом.
            \item Различные виды знаний в \textit{ostis-системах}, проектируемые по принципам \textit{Технологии OSTIS}, глубоко интегрированы между собой. Это дает не только простоту для создания этих систем на базе имеющихся языков, которые могут быть описаны на \textit{SC-коде}, но большие возможности для создания базовых \textit{языков программирования} для \textit{программных компьютерных систем нового поколения} таких, как, например, \textit{базового языка представления процедурных методов SCP}, \textit{базового языка представления продукционных методов} и других. Современные \textit{языки представления методов} создаются с целью упрощения описания какого-то алгоритма для быстрого и качественного решения определенного класса задач. В свою очередь, предлагаемые методики и модели позволяют проектировать \textit{языки представления методов} для \textit{компьютерных систем нового поколения} с помощью базовых \textit{языков представления знаний} таким образом, чтобы сама форма представления знаний не менялась. Методы разных \textit{языков представления методов} должны иметь одну универсальную форму представления, то есть один и тот же синтаксис, но могут давать возможности описывать и представлять разными способами \textit{денотационную} и \textit{операционную семантику} своих \textit{методов} с помощью одного и того же синтаксиса.
            \item Проектирование новых \textit{языков представления методов} должно сводится к их полному описанию на минимальном семействе \textit{языков SC-кода}: \textit{SCP}, \textit{SCL} и других. Речь идет о том, что чтобы спроектировать новый \textit{язык представления методов} достаточно разработать (неатомарный) метаметод на языках \textit{SCP} и \textit{SCL}, который будет интерпретировать методы проектируемых языков, а также описать \textit{денотационную семантику} этих методов. \textit{Метаметод интерпретации методов языков представления методов} можно называть интерпретатором этих языков, то есть некоторой абстрактной sc-машиной, на которой возможно выполнение методов определенного \textit{языка представления} этих \textit{методов}.
        \end{itemize}}
\end{scnsubstruct}

\scnsegmentheader{Операционная семантика программ в ostis-системах}
\begin{scnsubstruct}
	
	\scnheader{полная спецификация метода*}
    \scntext{примечание}{Полная \textit{спецификация метода*} кроме \textit{денотационной семантики этого метода*} должна включать \textit{операционную семантику этого метода*}, то есть формальное описание интерпретатора заданного метода. \textit{Операционная семантика языка представления методов} описывает выполнение \textit{метода}, составленного на данном языке, средствами виртуального компьютера. Виртуальный компьютер определяется как абстрактный автомат. Внутренние состояния этого автомата моделируют состояния вычислительного процесса при выполнении метода. Автомат транслирует исходный текст метода в набор формально определенных операций. Этот набор задает переходы автомата из исходного состояния в последовательность промежуточных состояний, изменяя значения переменных метода. Автомат завершает свою работу, переходя в некоторое конечное состояние. Таким образом, здесь идет речь о достаточно прямой абстракции возможного использования языка представления методов. \textit{операционная семантика языка} описывает смысл метода путем выполнения его операторов на простой машине-автомате. Изменения, происходящие в состоянии машины при выполнении данного оператора, определяют смысл этого оператора.}
    
    \scnheader{операционная семантика метода}
    \scntext{примечание}{\textit{Операционная семантика} конкретного \textit{метода} сводится к описанию \textit{метаметода}, который его интерпретирует, верифицирует и так далее.}
    
    \scnheader{метаметод}
    \scnsubset{метод}
    \scnidtf{метод, значениями параметров которого являются другие методы}

    \scnheader{операционная семантика метода}
    \scnhaselement{метаметод интерпретации*}
    \scnhaselement{метаметод верификации и оценки качества*}

    \scnheader{метаметод интерпретации*}
    \scntext{определение}{Отношение \textit{метаметод интерпретации*} представляет собой \textit{класс sc-связок} между \textit{sc-связкой}, обозначающей множество \textit{методов}, и sc-узлом, обозначающим \textit{метод}, который способен произвести интерпретацию заданного множества \textit{методов}.}
     
    \scnheader{метаметод верификации и оценки качества*}
    \scntext{определение}{Отношение \textit{метаметод верификации и оценки качества*} представляет собой класс sc-связок между \textit{sc-связкой}, обозначающей множество \textit{методов}, и sc-узлом, обозначающим метод, который способен произвести верификацию и оценку качества заданного множества \textit{методов}.}
    
    \scnheader{метаметод интерпретации методов базовых языков представления методов}
    \begin{scnrelfromlist}{класс подметодов}
        \scnitem{метаметод интерпретации методов Языка SCP}
        \scnitem{метаметод интерпретации методов Языка SCL}
        \scnitem{метаметод интерпретации методов языка представления продукционных методов}
        \scnitem{метаметод интерпретации методов языка представления функциональных методов}
        \scnitem{метаметод интерпретации методов языка представления нейросетей}
        \scnitem{метаметод интерпретации методов языка представления генетических алгоритмов}
    \end{scnrelfromlist}
    \scntext{примечание}{В рамках \textit{Технологии OSTIS} таких метаметодов может быть большое разнообразие. Каждый из них может состоять из множества атомарных и неатомарных подметодов. Это могут быть как метаметоды, интерпретирующие методы определенных \textit{языков представления методов}, так и метаметоды, верифицирующие и анализирующие качество этих методов. В том числе метаметоды могут производить операции над другими метаметодами.}

    \scnheader{метаметод верификации и оценки качества методов базовых языков представления методов}
    \begin{scnrelfromlist}{класс подметодов}
        \scnitem{метаметод верификации и оценки качества методов Языка SCP}
        \scnitem{метаметод верификации и оценки качества методов Языка SCL}
        \scnitem{метаметод верификации и оценки качества методов языка представления продукционных методов}
        \scnitem{метаметод верификации и оценки качества методов языка представления функциональных методов}
        \scnitem{метаметод верификации и оценки качества методов языка представления нейросетей}
        \scnitem{метаметод верификации и оценки качества методов языка представления генетических алгоритмов}
    \end{scnrelfromlist}
    \scntext{примечание}{Так, например, при реализации методов в оstis-системах метаметодами будут являться \textit{интепретатор Языка SCP}, а также интепретаторы, реализованные непосредственно на \textit{Языке SCP}.}
    \scntext{примечание}{Понятие \textit{синтакиса}, \textit{денотационной} и операционной \textit{семантики языков представления методов} сводятся к понятию синтаксиса, денотационной и операционной семантики вообще любого языка.}
\end{scnsubstruct}

\scnsegmentheader{Синтаксис и семантика языков программирования в ostis-системах}
\begin{scnsubstruct}
	\scnheader{язык представления методов}
    \scntext{примечание}{Для использования \textit{языка представления методов} следует описать каждую конструкцию языка в отдельности, а также ее применение в совокупности с другими конструкциями. В языке существует множество различных конструкций, точное определение которых необходимо как программисту, применяющему язык, так и разработчику компилятора для этого языка. Программисту эти знания позволяют прогнозировать вычисления, производимые операторами метода. Разработчику описания конструкций необходимы для создания правильной реализации компилятора.}
    \scntext{примечание}{Описание формальной модели \textit{языка представления методов} можно задать его \textit{спецификацией}. \textit{спецификация языка представления методов*} содержит описание \textit{синтаксиса}, \textit{денотационной}, \textit{операционной} \textit{семантики языка представления методов}.}

    \scnheader{спецификация языка представления методов*}
    \scnsuperset{отношение, заданное на множестве (язык представления методов)*}
    \begin{scnrelfromset}{разбиение}
        \scnitem{синтаксис языка представления методов*}
        \begin{scnindent}
            \scnsubset{синтаксис языка*}
            \scnidtf{теория правильно построенных информационных конструкций, принадлежащих заданному языку представления методов}
        \end{scnindent}
        \scnitem{денотационная семантика языка представления методов*}
        \begin{scnindent}
            \scnsubset{денотационная семантика языка*}
            \scnidtf{обобщенная формулировка классов задач, решаемых с помощью данного языка представления методов*}
        \end{scnindent}
        \scnitem{операционная семантика языка представления методов*}
        \begin{scnindent}
            \scnsubset{операционная семантика языка*}
            \scnidtf{перечень обобщенных агентов, обеспечивающих интерпретацию методов заданного языка представления методов*}
            \scnidtf{семейство методов интерпретации текстов данного языка представления методов*}
            \scnidtf{формальное описание интерпретатора заданного языка представления методов*}
        \end{scnindent}
    \end{scnrelfromset}
\end{scnsubstruct}

\scnsegmentheader{Методы и средства поддержки проектирования и разработки программ в ostis-системах}
\begin{scnsubstruct}

    \scntext{примечание}{Текущее состояние в области проектирования и разработки программного обеспечения говорит о том, что разработчики больше стремятся автоматизировать разработку методов на конкретных языках представления методов, чем обеспечить себя инструментальными обучающими средствами их проектирования, в том числе проектирования новых \textit{языков представления методов}. Это приводит к проблемам.}
    \begin{scnindent}
        \begin{scnrelfromset}{проблемы текущего состояния}
            \scnfileitem{В то время, как количество разработчиков, понимающих код какой-то сложной программной системы, уменьшается, требования к этой системе растут все быстрее и быстрее. Зачастую, разработчики сложных программных систем сами не в состоянии объяснить логику работы этих систем. По этой причине необходимо создавать инструментальные средства, которые будут позволять автоматизировать документирование программных систем.}
            \scnfileitem{Для обучения новых разработчиков навыкам работы с программными системы и их разработки необходимо привлекать ресурсы экспертов, понимающих принципы работы этих программных систем. Проблема решается разработкой справочной системы, которая будет позволять не только обучать пользователя тому, как проектировать методы решения задачи и программные системы на основе этих методов, но и указывать на пробелы в смежных дисциплинах, необходимых для достижения качественных результатов всей своей деятельности.}
            \scnfileitem{В инженерии часто разработчики проектируют и разрабатывают решения, которые уже когда-то были созданы другими специалистами. Таким образом, получаются функционально эквивалентные методы решения задач, а то, и вовсе, программные системы, решающие схожие проблемы. Ключом к решению данной проблемы является проектирование семантически мощной \textit{библиотеки многократно используемых методов решения различных задач}.}
        \end{scnrelfromset}
		\begin{scnrelfromset}{смотрите}
			\scnitem{\scncite{Lu2022}}
		\end{scnrelfromset}
	\end{scnindent}

	\scnheader{Cемантическая теория программ}
    \scntext{примечание}{Одной \textit{Cемантической теории программ} недостаточно. Кроме нее, для перманетного и беспрепятственного проектирования и разработки \textit{методов} различного класса необходимо разрабатывать:
    \begin{itemize}
        \item интеллектуальную систему поддержки проектирования и разработки методов, которая будет не только помогать разработчику верифицировать разрабатываемый метод, но и подсказывать способы его разработки;
        \item семантически мощную библиотеку многократно используемых компонентов для быстрого поиска существующих методов решения задач и их применения для решения других более комплексных задач.
    \end{itemize}}
	\begin{scnindent}
		\begin{scnrelfromset}{смотрите}
			\scnitem{\scncite{Gulykina2012}}
			\scnitem{\scncite{Pivovarchik2016}}
			\scnitem{\scncite{Ford2019}}
		\end{scnrelfromset}
	\end{scnindent}

	\scnheader{интеллектуальная система поддержки проектирования и разработки методов}
    \scntext{примечание}{Потенциальная система должна быть частью общего инструментального средства разработки интеллектуальны компьютерных систем нового поколения --- \textit{Метасистемы OSTIS} --- и может состоять из следующих компонентов:
        \begin{itemize}
            \item интеллектуальной help-системы по семантической теории программ;
            \item интеллектуальной help-системы по библиотеке многократно используемых методов решения задач,
            \item интеллектуальной help-системы по комплексу инструментальных средств проектирования  методов решения задач,
            \item интеллектуальной help-системы по методике обучения проектированию различных методов решения задач.
        \end{itemize}}
	\begin{scnindent}
		\begin{scnrelfromset}{смотрите}
			\scnitem{\scncite{IMS}}
		\end{scnrelfromset}
	\end{scnindent}
    \scntext{примечание}{Каждый компонент должен содержать:
        \begin{itemize}
            \item справочную подсистему,
            \item подсистему мониторинга и анализа деятельности разработчика методов решения задач,
            \item подсистему управления обучением.
        \end{itemize}}
    \scntext{примечание}{Каждая из подсистем взаимодействует с другими подсистемами, а также может функционировать автономно.}
    \scntext{примечание}{Справочная подсистема является консультантом-экспертом в области \textit{Семантической теории программ}, который может ответить на любой вопрос новичка или опытного пользователя. Каждая из таких систем может становиться индивидуальным помощников в обучении новых специалистов --- персональным ostis-ассистентом.}

\end{scnsubstruct}

\scnsegmentheader{Комплекс свойств, определяющих эффективность программ в ostis-системах}
\begin{scnsubstruct}

	\scnheader{язык представления методов}
    \scntext{примечание}{\textit{язык представления методов} можно определить множеством показателей, характеризующих отдельные его свойства. Возникает задача введения меры для оценки степени приспособленности языка представления методов к выполнению возложенных на него функций --- \textit{критериев эффективности}. Критерии эффективности методов приводятся на основе частных показателей эффективности этих методов (показателей качества). Способ связи между частными показателями определяет вид критерия эффективности.}
	\begin{scnindent}
		\begin{scnrelfromset}{смотрите}
			\scnitem{\scncite{Orlov2021}}
		\end{scnrelfromset}
	\end{scnindent}
	
    \scnheader{эффективность метода}
    \begin{scnrelfromlist}{свойство-предпосылка}
        \scnitem{легкость чтения и понимания метода}
        \scnitem{легкость представления метода}
        \scnitem{стоимость метода}
        \scnitem{общий объем задач, решаемых при помощи данного класса методов}
        \scnitem{многообразие видов задач, решаемых при помощи данного класса методов}
        \scnitem{надежность метода}
    \end{scnrelfromlist}

    \scnheader{легкость чтения метода}
    \scntext{примечание}{\textbf{\textit{легкость чтения метода}} должна способствовать легкому выделению основных понятий каждой части метода без обращения к его спецификации.}

    \scnheader{легкость чтения и понимания метода}
    \begin{scnrelfromlist}{свойство-предпосылка}
        \scnitem{простота синтаксиса языка представления методов}
        \scnitem{ортогональность информационных конструкций языка представления методов}
        \scnitem{структурированность потока управления в методе}
    \end{scnrelfromlist}

	\scnheader{язык представления методов}
    \scntext{примечание}{\textit{язык представления методов} должен предоставить \textit{простой} набор \textit{информационных конструкций}, которые могут быть использованы в качестве базисных элементов при создании методов.
        \\Сильное воздействие на простоту оказывает \textit{синтаксис языка}: он должен прозрачно отражать семантику конструкций, исключать двусмысленность и неоднозначность толкования.}

	\scnheader{ортогональность информационных конструкций языка представления методов}
    \scntext{примечание}{\textbf{\textit{ортогональность информационных конструкций языка представления методов}} означает, что любые возможные комбинации различных \textit{информационных конструкций} будут осмысленными, без неожиданного поведения, возникающего в результате взаимодействия конструкций или контекста использования.}

	\scnheader{поток управления}
    \scntext{примечание}{Порядок передач управления между операторами метода, то есть \textit{поток управления}, должен быть удобен для чтения и понимания человеком.}

	\scnheader{легкость создания метода}
    \scntext{примечание}{\textbf{\textit{легкость создания метода}} отражает удобство языка для представления этого метода в конкретной предметной области.}

    \scnheader{легкость представления метода}
    \begin{scnrelfromlist}{свойство-предпосылка}
        \scnitem{простота синтаксиса языка представления методов}
        \scnitem{естественность синтаксиса языка представления методов}
        \scnitem{ортогональность информационных конструкций языка представления методов}
        \scnitem{полнота и точность спецификации языка представления методов}
        \scnitem{согласованность и целостность спецификации языка представления методов}
    \end{scnrelfromlist}

	\scnheader{синтаксис метода}
    \scntext{примечание}{\textit{синтаксис метода} должен способствовать легкому и прозрачному отображению в нем алгоритмических структур предметной области. \textit{синтаксис языка представления методов} должен быть не только \textit{простым}, но и \textit{естественным}, и поддерживать \textit{ортогональность} информационных конструкций языка.}

	\scnheader{легкость представления метода}
    \scntext{примечание}{\textbf{\textit{легкость представления нового метода}} обеспечивается \textit{полной и точной, согласованной и целостной спецификацией} соответствующего языка. То есть необходимо достаточное количество \textit{информационных конструкций} в этом языке для того чтобы представить конкретный \textit{метод}. При этом \textit{спецификация языка} должна быть согласованной и целостной чтобы представлять на ней непротиворечивые \textit{методы}.}

    \scnheader{общая стоимость метода}
    \begin{scnrelfromlist}{свойство-предпосылка}
        \scnitem{стоимость применения метода}
        \scnitem{стоимость интерпретации метода}
        \scnitem{стоимость создания, тестирования и использования метода}
        \scnitem{стоимость сопровождения метода}
        \scnitem{согласованность и целостность спецификации языка представления методов}
    \end{scnrelfromlist}
	\scntext{примечание}{Все эти критерии можно применить и касательно самих \textit{языков представления методов}.}
	
	\scnheader{стоимость применения метода}
    \scntext{примечание}{\textbf{\textit{стоимость применения метода}} во многом зависит от структуры \textit{языка представления методов}. Язык, требующий многочисленных проверок синтаксических типов во время применения метода, будет препятствовать быстрой работе программы.}

	\scnheader{размер стоимости интерпретации метода}
    \scntext{примечание}{\textbf{\textit{размер стоимости интерпретации метода}} зависит от возможностей используемого метаметода интерпретации. Чем совершеннее методы оптимизации, тем дороже стоит интерпретация.}
    
    \scnheader{общая стоимость метода}
    \scntext{примечание}{Размер стоимости создания, тестирования и использования метода зависит от используемого метаметода верификации и оценки качества этого метода.}
    \scntext{примечание}{Многочисленные исследования показывают, что значительную часть стоимости используемого метода составляет не стоимость его разработки, а \textit{стоимость его сопровождения}. Связывая сопровождение методов с другими их характеристиками, следует выделить, прежде всего, зависимость от читабельности, поскольку сопровождение обычно происходит следующим поколением разработчиков.}
    \begin{scnindent}
        \begin{scnrelfromset}{смотрите}
            \scnitem{\scncite{Brooks2021}}
        \end{scnrelfromset}
    \end{scnindent}
    
    \scnheader{язык представления методов}
    \scntext{примечание}{\textbf{\textit{общий объем задач и многообразие видов задач, решаемых при помощи данного класса методов}}, являются не менее важными характеристиками и показывают степень универсальности соответствующего языка представления методов. Чем больше задач можно решить на \textit{я.п.м.}, тем он универсальнее.}

	\scnheader{надежность методов языка представления методов}
    \scntext{примечание}{\textbf{\textit{надежность методов языка представления методов}} должна обеспечиваться минимумом ошибок при работе конкретного метода.}
  
\end{scnsubstruct}

\end{scnsubstruct}
\end{SCn}
