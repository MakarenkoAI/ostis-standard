\begin{SCn}

\scnsectionheader{Предметная область и онтология Языка вопросов для ostis-систем}
\begin{scnsubstruct}

	\scnheader{Предметная область Языка вопросов для ostis-систем}
	\scniselement{предметная область}
    \begin{scnrelfromset}{автор}
        \scnitem{Самодумкин С.А.}
        \scnitem{Зотов Н.В.}
        \scnitem{Шункевич Д.В.}
        \scnitem{Ивашенко В.П}
    \end{scnrelfromset}
    \begin{scnrelfromlist}{дочерняя предметная область}
        \scnitem{Предметная область синтаксиса Языка вопросов для ostis-систем}
        \scnitem{Предметная область денотационной семантики Языка вопросов для ostis-систем}
        \scnitem{Предметная область операционной семантики Языка вопросов для ostis-систем}
    \end{scnrelfromlist}
    
    \scntext{аннотация}{Возможности \textit{баз знаний} \textit{ostis-систем} позволяют не только представлять и структурировать знания об окружающем мире, но и быстро получать и формировать эти знания о нем, тем самым удовлетворяя информационную потребность пользователя. В данной предметной области уточнена формальная спецификация \textit{Языка вопросов для ostis-систем}, позволяющая описывать и интерпретировать любые классы \textit{вопросов} \textit{пользователей ostis-систем}.}

    \begin{scnrelfromlist}{ключевой знак}
        \scnitem{Язык вопросов для ostis-систем}
    \end{scnrelfromlist}
    
    \begin{scnhaselementrolelist}{класс объектов исследования}
        \scnitem{вопрос}
        \scnitem{ответ на вопрос}
        \scnitem{знак в рамках заданного вопроса}
        \scnitem{основной знак в рамках заданного вопроса}
        \scnitem{неосновной знак в рамках заданного вопроса}
        \scnitem{отношение в рамках заданного вопроса}
        \scnitem{базовое отношение в рамках заданного вопроса}
        \scnitem{интерпретатор Языка вопросов для ostis-систем}
    \end{scnhaselementrolelist}

    \begin{scnrelfromlist}{библиографическая ссылка}
        \scnitem{\scncite{Averyanov1993}}
        \scnitem{\scncite{Suleimanov2001}}
        \scnitem{\scncite{Suleimanov2014}}
        \scnitem{\scncite{Bukharev1990}}
        \scnitem{\scncite{Kwok2001}}
        \scnitem{\scncite{Emelyanov2007}}
        \scnitem{\scncite{Finn1976}}
        \scnitem{\scncite{Finn1981}}
        \scnitem{\scncite{Belnap1981}}
        \scnitem{\scncite{Sosnin2007}}
        \scnitem{\scncite{Zaharov2002}}
        \scnitem{\scncite{Hant1978}}
        \scnitem{\scncite{Lyubarsky1990}}
        \scnitem{\scncite{Samodumkin2009}}
        \scnitem{\scncite{Samodumkin2009a}}
    \end{scnrelfromlist}

    \begin{scnrelfromvector}{введение}
    	\scnfileitem{Одна из ключевых особенностей \textit{интеллектуальной системы} состоит в том, что \textit{пользователь} имеет возможность формулировать свою информационную потребность. Cпособом выражения такой потребности является \textit{вопрос}. В процессе общения всегда существует контекст, который определяет дополнительную информацию, способствующую правильному пониманию \textit{смысла} сообщения. Особенность представления информации в \textit{базах знаний} \textit{ostis-систем} упрощает формирование информационной потребности пользователя, так как представленная информация в \textit{базах знаний} уже структурирована и известны отношения, заданные на определенном понятии, в отношении которого разрешается вопросно-проблемная ситуация.}
    	\scnfileitem{Показано, что вопросно-проблемная ситуация не может быть решена в рамках формальной логики и природа вопроса может быть понятна в системе субъектно-объектных отношений. В связи с тем, что при формировании \textit{баз знаний} \textit{ostis-систем} происходит формирование субъектно-объектных отношений в рамках заданной \textit{предметной области}, тем самым упрощается выражение информационной потребности пользователем средствами \textit{SC-кода}.}
    	\begin{scnindent}
    		\begin{scnrelfromset}{источник}
    			\scnitem{\scncite{Averyanov1993}}
    		\end{scnrelfromset}
    	\end{scnindent}
        \scnfileitem{Целью разработки \textbf{\textit{Языка вопросов для ostis-систем}} и последующего его развития является реализация возможности понимания действий, осуществляемых \textit{ostis-системой}, при формировании ответа на поставленный \textit{вопрос}. В процессе формирования ответа на поставленный \textit{вопрос} возможны следующие варианты:
        \begin{itemize}
            \item \textit{ответ на} поставленный \textit{вопрос} существует в \textit{базе знаний} и происходит локализация \textit{фрагмента базы знаний} в контексте выраженной средствами \textit{SC-кода} информационной потребности \textit{пользователя};
            \item ответ связан с разрешением некоторой задачной ситуации, которая содержится в контексте \textit{вопроса} и формирование \textit{ответа на вопрос} возлагается на \textit{решатель задач}.
        \end{itemize}}
	    \begin{scnindent}
	    	\begin{scnrelfromset}{смотрите}
	    		\scnitem{Агентно-ориентированные модели гибридных решателей задач ostis-систем}
	    	\end{scnrelfromset}
	    \end{scnindent}
	\end{scnrelfromvector}

    \scnheader{Язык вопросов для ostis-систем}
    \scnidtf{Предлагаемый вариант языка для описания вопросов и ответов на них в ostis-системах}
    \scnidtf{sc-язык вопросов}
    \scniselement{sc-язык}
    \scnrelfrom{синтаксис языка}{Синтаксис Языка вопросов для ostis-систем}
    \begin{scnindent}
        \scnsubset{Синтаксис SC-кода}
    \end{scnindent}
    \scnrelfrom{денотационная семантика языка}{Денотационная семантика Языка вопросов для ostis-систем}
    \begin{scnindent}
        \scnidtf{Онтология классов знаков и отношений для описания формулировок вопросов на SC-коде}
        \scnsuperset{Семантическая классификация вопросов}
    \end{scnindent}
    \scnrelfrom{операционная семантика языка}{Операционная семантика Языка вопросов для ostis-систем}
    \begin{scnindent}
        \scnidtf{Коллектив sc-агентов вывода ответов на заданные вопросы пользователя ostis-системы}
    \end{scnindent}

\end{scnsubstruct}
\scnendcurrentsectioncomment
\end{SCn}
