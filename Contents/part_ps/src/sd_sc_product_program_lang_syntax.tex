\begin{SCn}
\scnsectionheader{Предметная область и онтология синтаксиса sc-языков продукционного программирования}
\begin{scnsubstruct}

\scnheader{Предметная область синтаксиса sc-языков продукционного программирования}
\scniselement{предметная область}
\begin{scnrelfromlist}{соавтор}
    \scnitem{Орлов М. К.}
    \scnitem{Зотов Н. В.}
\end{scnrelfromlist}

\begin{scnrelfromvector}{библиография}
    \scnitem{\scncite{AIHandbookMM}}
\end{scnrelfromvector}

\scnheader{продукция}
\scntext{общий вид}{($i$); $Q$; $P$; $A$ $\Rightarrow$ $B$; $N$}
\begin{scnindent}
    \scnhaselement{i}
    \begin{scnindent}
        \scnnote{Имя продукции, с помощью которого данная продукция выделяется из всего множества продукций. В качестве имени может выступать некоторая лексема, отражающая суть данной продукции (например, "покупка книги"{} или "набор кода замка"{}), или порядковый номер продукции в их множестве, хранящемся в памяти системы. }
    \end{scnindent}
    \scnhaselement{Q}
    \begin{scnindent}
        \scnnote{Характеризует сферу применения продукции или же контекст.}
    \end{scnindent}
    \scnhaselement{P}
    \begin{scnindent}
        \scnnote{Условие применимости ядра продукции. Обычно $P$ представляет собой логическое выражение (как правило, предикат). Когда $P$ принимает значение «истина», ядро продукции активизируется. Если $P$ ложно, то ядро продукции не может быть использовано.}
        \scntext{пример невозможности применения продукции}{Если в продукции «НАЛИЧИЕ ДЕНЕГ; ЕСЛИ ХОЧЕШЬ КУПИТЬ ВЕЩЬ X, ТО ЗАПЛАТИ В КАССУ ЕЕ СТОИМОСТЬ И ОТДАЙ ЧЕК ПРОДАВЦУ» условие применимости ядра продукции ложно, то есть денег нет, то применить ядро продукции невозможно.}
    \end{scnindent}
    \scntext{подстрока}{$A$ $\Rightarrow$ $B$}
    \begin{scnindent}
        \scnnote{Ядро продукции. Интерпретация ядра продукции может быть различной и зависит от того, что стоит слева и справа от знака секвенции $\Rightarrow$.}
        \scntext{прочтение ядра}{ЕСЛИ $A$, ТО $B$}
        \scntext{прочтение более сложной конструкции ядра}{ЕСЛИ $A$, ТО $B_1$ ИНАЧЕ $B_2$}
        \scnnote{Секвенция может истолковываться в обычном логическом смысле как знак логического следования $B$ из истинного $A$ (если $A$ не является истинным выражением, то о $B$ ничего сказать нельзя). Возможны и другие интерпретации ядра продукции, например $A$ описывает некоторое условие, необходимое для того, чтобы можно было совершить действие $B$.}
    \end{scnindent}
    \scnhaselement{N}
    \begin{scnindent}
        \scnnote{Постусловия продукции. Они актуализируются только в том случае, если ядро продукции реализовалось. Постусловия продукции описывают действия и процедуры, которые необходимо выполнить после реализации В.}
        \scntext{пример использования}{после покупки некоторой вещи в магазине необходимо в описи товаров, имеющихся в этом магазине, уменьшить количество вещей такого типа на единицу. Выполнение $N$ может происходить не сразу после реализации ядра продукции.}
        \scnrelfrom{библиографический источник}{\scncite{AIHandbookMM}}
    \end{scnindent}
\end{scnindent}
\scnnote{Если в памяти системы хранится некоторый набор продукций, то они образуют систему продукций. В системе продукций должны быть заданы специальные процедуры управления продукциями, с помощью которых происходит актуализация продукций и выбор для выполнения той или иной продукции из числа актуализированных.}

\scnheader{структурная единица продукционного языка программирования}
\scnrelto{содержит}{продукционный язык программирования}
\scnhaselementrole{основной элемент}{продукция}
\scnhaselement{определение глобальной переменной}
\scnhaselement{оператор управления макрогенерацией}
\scnhaselement{вставка текста на алгоритмическом языке программирования}

\scnheader{системы, комбинирующие сетевые и продукционные модели}
\scnrelfrom{модель представления декларативных знаний}{сетевая модель}
\scnrelfrom{модель представления процедурных знаний}{продукционная модель}
\scntext{способ работы}{работа продукционной системы над семантической сетью}
\scnnote{Процедурные знания позволяют системе узнать, как можно использовать те или иные декларативные знания, в частности, знания о закономерностях той части действительности, в которой "живет"{} интеллектуальная система, для получения нужных системе результатов или тех результатов, которые ожидает от нее пользователь.}

\bigskip
\end{scnsubstruct}
\scnendsegmentcomment{Предметная область и онтология синтаксиса sc-языков продукционного программирования}

\end{SCn}
