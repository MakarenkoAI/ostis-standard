\begin{SCn}
    \scnsectionheader{Предметная область и онтология ostis-систем, являющихся персональными ассистентами пользователей, обеспечивающими организацию эффективного взаимодействия каждого пользователя с другими ostis-системами и пользователями, входящими в состав Экосистемы OSTIS}
    \begin{scnsubstruct}

        \scnheader{Предметная область ostis-систем, являющихся персональными ассистентами пользователей в рамках Экосистемы OSTIS}
        \scniselement{предметная область}
        \begin{scnhaselementrole}{максимальный класс объектов исследования}
            {персональный ostis-ассистент}
        \end{scnhaselementrole}

        \scnheader{персональный ostis-ассистент}
        \scnidtf{ostis-система, являющаяся персональным ассистентом пользователя в рамках Экосистемы OSTIS}
        \begin{scnrelfromset}{возможности}
            \scnfileitem{Возможность анализа деятельности пользователя и формирования рекомендаций по ее оптимизации.}
            \scnfileitem{Возможность адаптации под настроение пользователя, его личностные качества, общую окружающую обстановку, задачи, которые чаще всего решает пользователь.}
            \scnfileitem{Перманентное обучение самого ассистента в процессе решения новых задач, при этом обучаемость потенциально не ограничена.}
            \scnfileitem{Возможность вести диалог с пользователем на естественном языке, в том числе в речевой форме.}
            \scnfileitem{Возможность отвечать на вопросы различных классов, при этом если системе что-то не понятно, то она сама может задавать встречные вопросы.}
            \scnfileitem{Возможность автономного получения информации от всей окружающей среды, а не только от пользователя (в текстовой или речевой форме). При этом система может как анализировать доступные информационные источники (например, в интернете), так и анализировать окружающий ее физический мир, например, окружающие предметы или внешний вид пользователя.}
        \end{scnrelfromset}
        \begin{scnrelfromset}{достоинства}
            \scnfileitem{Пользователю нет необходимости хранить разную информацию в разной форме в разных местах, вся информация хранится в единой базе знаний компактно и без дублирований.}
            \scnfileitem{Благодаря неограниченной обучаемости ассистенты могут потенциально автоматизировать практически любую деятельность, а не только самую рутинную.}
            \scnfileitem{Благодаря базе знаний, ее структуризации и средствам поиска информации в базе знаний пользователь может получить более точную информацию более быстро.}
        \end{scnrelfromset}
        \scnsuperset{персональный ostis-ассистент учебного назначения}
        \begin{scnindent}
        	\scnidtf{персональный ассистент-учитель}
        \end{scnindent}
        \scnsuperset{персональный ostis-ассистент по здоровому образу жизни и здоровому питанию}
        \begin{scnindent}
        	\scnidtf{персональный фитнесс-тренер}
        \end{scnindent}
        \scnsuperset{персональный ostis-ассистент для ухода за пациентом}
        \scnsuperset{персональный секретарь-референт}
        \bigskip
        
    \end{scnsubstruct}
    \scnendcurrentsectioncomment
\end{SCn}
