\begin{SCn}
    \scnsectionheader{Предметная область и онтология деятельности в области Искусственного интеллекта}
    \begin{scnsubstruct}
        \begin{scnreltovector}{конкатенация сегментов}
    \scnitem{Структура деятельности в области Искусственного интеллекта}
    \scnitem{Текущее состояние и проблемы дальнейшего развития деятельности в области Искусственного интеллекта}
    \scnitem{Понятие Технологии OSTIS}
    \scnitem{Использование Технологии OSTIS для повышения качества человеческой деятельности в области Искусственного интеллекта}
    \scnitem{Понятие Экосистемы OSTIS}
\end{scnreltovector}
\begin{scnrelfromset}{рассматриваемые вопросы}
    \scnfileitem{Каковы основные стратегические цели (сверхзадачи) научно-технической деятельности в области \textit{Искусственного интеллекта}.}
    \scnfileitem{Какие проблемы являются на сегодняшний день актуальными для дальнейшего развития различных направлений \textit{Искусственного интеллекта} и для развития \textit{Искусственного интеллекта} в целом как общей (объединённой) \textit{научно-технической дисциплины}, а также для развития различных форм деятельности в этой области (научно-исследовательской деятельности создания технологий разработки интеллектуальных компьютерных систем, образовательной деятельности, бизнеса).}
    \scnfileitem{Какие проблемы являются на сегодняшний день актуальными для развития других \textit{научно-технических дисциплин} и являются ли эти проблемы аналогичными тем, которые актуальны для развития \textit{Искусственного интеллекта}.}
    \scnfileitem{Какие можно предложить подходы к решению указанных выше проблем и как для этого можно использовать создаваемый сейчас новый технологический уклад в области \textit{Искусственного интеллекта} (следующий уровень технологий искусственного интеллекта).}
    \scnfileitem{Как будет выглядеть на основе следующего уровня \textit{технологий Искусственного интеллекта} комплексная автоматизация всех \textit{видов человеческой деятельности}, а также взаимодействие различных \textit{видов человеческой деятельности}, т.е. как будет выглядеть архитектура \textit{smart-общества}.}
    \scnfileitem{Устраивает ли нас уровень семантической совместимости взаимопонимания между современными виртуальными компьютерными системами и что необходимо сделать для повышения этого уровня.}
    \scnfileitem{Устраивает ли нас уровень семантической совместимости взаимопонимания между современными интеллектуальными компьютерными системами их пользователями и что необходимо сделать для повышения этого уровня.}
\end{scnrelfromset}
\scntext{аннотация}{Предлагаемое вашему вниманию рассмотрение методологических проблем современного состояния работ в области \textit{Искусственного интеллекта} состоит из следующих частей:
    \begin{scnitemize}
        \item Анализ актуальных проблем, препятствующих дальнейшему развитию  \textit{Искусственного интеллекта} как \textit{научно-технической дисциплины}:
        \begin{scnitemizeii}
            \item Проблемы развития научных исследований в области \textit{Искусственного интеллекта}.
            \item Проблемы разработки технологий проектирования и реализации \textit{интеллектуальных компьютерных систем}.
            \item Проблемы формирования рынка \textit{интеллектуальных компьютерных систем}.
            \item Образовательные проблемы в области \textit{Искусственного интеллекта}.
            \item Проблемы развития бизнеса в области \textit{Искусственного интеллекта}.
        \end{scnitemizeii}
        \item Анализ проблем автоматизации сложных видов деятельности:
        \begin{scnitemizeii}
            \item научно-исследовательской деятельности в рамках различных научных дисциплин;
            \item создание \textit{технологий проектирования} и производства (реализации) сложных технических систем;
            \item \textit{инженерной деятельности} по разработке сложных технических систем;
            \item \textit{образовательной деятельности} по наукоёмким техническим специальностям.
        \end{scnitemizeii}
        \item Формулировка принципов, лежащих в основе \textit{Технологии OSTIS}, предназначенной для решения указанных выше проблем.
        \item Рассмотрение структуры \textit{Экосистемы OSTIS}, построенной по \textit{Технологии OSTIS} и обеспечивающей комплексную автоматизацию всех видов человеческой деятельности.
    \end{scnitemize}}
\begin{scnrelfromset}{используемые знаки общих понятий и иных сущностей}
    \scnitem{деятельность}
	    \begin{scnindent}
	    	\scnidtf{область деятельности}
	    	\scnsuperset{человеческая деятельность}
	    \end{scnindent}
    \scnitem{вид деятельности}
    \begin{scnindent}
        \scnhaselement{проектирование}
        \begin{scnindent}
            \scnidtf{проектная деятельность}
        \end{scnindent}
        \scnhaselement{производство}
        \begin{scnindent}
            \scnidtf{производственная деятельность}
        \end{scnindent}
        \scnhaselement{наука}
        \begin{scnindent}
            \scnidtf{научная деятельность}
        \end{scnindent}
    \end{scnindent}
    \scnitem{проект}
    \begin{scnindent}
        \scnsuperset{открытый проект}
    \end{scnindent}
    \scnitem{консорциум}
    \scnitem{технология}
    	\begin{scnindent}
        \scnsuperset{информационная технология}
        \begin{scnindent}
            \scnsuperset{технология искусственного интеллекта}
        \end{scnindent}
    \end{scnindent}
    \scnitem{кибернетическая система}
    \begin{scnindent}
        \scnsuperset{интеллектуальная система}
        \begin{scnindent}
            \scnsuperset{интеллектуальная компьютерная система}
            \begin{scnindent}
                \scnidtf{искусственная интеллектуальная система}
            \end{scnindent}
        \end{scnindent}
    \end{scnindent}
    \scnitem{конвергенция\scnsupergroupsign}
    \begin{scnindent}
        \scnidtf{уровень конвергенции (близости)}
        \scnsuperset{конвергенция кибернетических систем\scnsupergroupsign}
        %Ключевого знака в стандарте не было
        \begin{scnreltolist}{ключевой знак}
            \scnitem{\cite{Yankovskaya2017}}
            \scnitem{\cite{Palagin2013}}
            \scnitem{\cite{Yankovskaya2010}}
            \scnitem{\cite{Kovalchuk2011}}
        \end{scnreltolist}
    \end{scnindent}
    \scnitem{интеграция*}
    \begin{scnindent}
        \scnsuperset{интеграция кибернетических систем*}
        \scnsuperset{эклектичная интеграция*}
        \scnsuperset{глубокая интеграция*}
    \end{scnindent}
    \scnitem{интегрированная система}
    \begin{scnindent}
        \scnsuperset{эклектичная система}
        \scnsuperset{гибридная система}
    \end{scnindent}
    \scnitem{экосистема интеллектуальных компьютерных систем}
    \scnitem{рынок знаний}
    \begin{scnindent}
        \scnidtf{рыночная организация порождения эволюции и применения знаний}
    \end{scnindent}
    \scnitem{smart-общество}
    \begin{scnindent}
        \scnidtf{общество,в основе которого лежит экосистема интеллектуальных компьютерных систем и рынок знаний}
    \end{scnindent}
\end{scnrelfromset}
\begin{scnrelfromset}{ключевые знаки}
    \scnitem{Искусственный интеллект}
    \begin{scnindent}
        \scniselement{научно-техническая дисциплина}
        \begin{scnindent}
            \scnsubset{научно-техническая деятельность}
        \end{scnindent}
    \end{scnindent}
    \scnitem{интеллектуальная система}
    	\begin{scnindent}
        	\scnsuperset{интеллектуальная компьютерная система}
        \end{scnindent}
    \scnitem{Общая теория интеллектуальных систем}
    \scnitem{Базовая комплексная технология проектирования интеллектуальных компьютерных систем}
    \scnitem{Технология производства спроектированных интеллектуальных компьютерных систем}
    \scnitem{Специализированная инженерия в области Искусственного интеллекта}
    \scnitem{Образовательная деятельность в области Искусственного интеллекта}
    \scnitem{Бизнес-деятельность в области Искусственного интеллекта}

    \bigskip

    \scnitem{\scnkeyword{Технология OSTIS}}
    \scnitem{\scnkeyword{ostis-система}}
    \scnitem{смысловое преставление информации}
    \scnitem{агентно-ориентированная модель обработки информации в памяти}
    \scnitem{стандартизация ostis-систем}
    \scnitem{\scnkeyword{SC-код}}
    \scnitem{абстрактная sc-машина}
    \scnitem{конвергенция знаний в памяти}
    \scnitem{ostis-систем}
    \scnitem{конвергенция моделей решения задач в  ostis-системе}
    \scnitem{интеграция знаний в памяти  ostis-системы}
    \scnitem{интеграция моделей решения задач в  ostis-системе}
    \scnitem{ostis-сообщество}
    \scnitem{ostis-технология}
    \begin{scnindent}
        \scnsuperset{ostis-технология проектирования}
        \scnsuperset{ostis-технология производства}
        \scnsuperset{технология эксплуатации ostis-систем}
        \scnsuperset{технология реинжиниринга ostis-систем} 
    \end{scnindent}
    \scnitem{\scnkeyword{Ядро Технологии OSTIS}}  

    \bigskip

    \scnitem{OSTIS-портал научных знаний в области Искусственного интеллекта}
    \scnitem{Проект Метасистемы OSTIS}
    \scnitem{\scnkeyword{Метасистема OSTIS}}
    \scnitem{Проект Программной реализации универсальной абстрактной sc-машины}
    \scnitem{Проект разработки Универсального sc-компьютера}
    \scnitem{Специализированная инженерия, осуществляемая на основе Технологии OSTIS}
    \scnitem{Образовательная деятельность в области Искусственного интеллекта, осуществляемая на основе технологии OSTIS}
    \scnitem{\scnkeyword{Консорциум OSTIS}}

    \bigskip

    \scnitem{\scnkeyword{Экосистема OSTIS}}
    \begin{scnindent}
        \scnidtf{Симбиоз семантически совместимых и координирующих свою деятельность \textit{ostis-систем} и людей, направленный на существенное, качественное повышение уровня автоматизации всех \textit{видов человеческой деятельности}.}
        \scntext{примечание}{Семантически совместимая (понятийно согласованная) формализация всех(!) видов человеческой деятельности является органической частью \textit{Технологии OSTIS}(!). То есть формализуемые отраслевые стандарты \textit{всех видов человеческой деятельности} должны строго наследовать свойства всей системы базовых понятий и знаний, лежащих в основе Технологии OSTIS. Таким образом речь идет о строгой формальной модели \textit{Экосистемы OSTIS} как единого целого и здесь есть место всем приложениям, но приведенным в комплексную систему. Если к построению такой комплексной формальной модели \textit{всех видов человеческой деятельности} подходить системно, то все не так страшно, так как многие модели можно и нужно строить на основе аналогий, стратификации, наследования и свойств. Это придаст существенную динамику эволюции этих формальных моделей.\\
            К сожалению, современная наука психологически ориентирована на поиск отличий, на выявление принципиальной (научной) новизны своих результатов (что является необходимым фактором оценки этих решений). В этом ничего плохого нет, но для решения сиситемых проблем (в частности, для вывода \textit{Искусственного интеллекта} из кризисного состояния) необходимо существенно активизировать поиск сходств, аналогий, реализацию конвергентных процессов по построению комплексных интегрированных моделей. Это не менее значимые научные результаты, чем выявление принципиально новых свойств и закономерностей.}
    \end{scnindent}
    \scnitem{человеческая деятельность}
    \scnitem{вид человеческой деятельности}
    \scnitem{автоматизация человеческой деятельности}
    \scnitem{качество человеческой деятельности}
    \scnitem{субъект Экосистемы OSTIS}
    \scnitem{Рынок знаний, реализованный в рамках Экосистемы OSTIS}
    \scnitem{smart-общество}
\end{scnrelfromset}

\scntext{предисловие}{Анализ современного состояния работы в области \textit{Искусственного интеллекта} показывает то, что указанная область \textit{человеческой деятельности} находится в глубоком фундаментальным методологическом и трудновидимом кризисе. Поэтому основными целями данного раздела являются:
    \begin{scnitemize}
        \item выявление основных причин возникновения указанного кризиса;
        \item уточнение основных мер, направленных на устранение этого кризиса.
    \end{scnitemize}}
\scntext{основная цель}{Сформировать мотивацию и инфраструктуру для создания эволюции информационных технологий принципиально нового уровня, в основе которого лежат семантические совместимые \textit{интеллектуальные компьютерные системы}, способные согласовывать свои действия в заранее непредсказуемых обстоятельствах.}
\begin{scnindent}
    \scntext{примечание}{Сейчас актуально не столько обсуждать различные вопросы \textit{Искусственного интеллекта}, а обсуждать \uline{проблемы} и пути решения этих проблем. Нельзя делать вид, что всё хорошо.}
\end{scnindent}
	
\scnheader{Искусственный интеллект}
\scntext{примечание}{Современное кризисное состояние \textit{Искусственного интеллекта} вполне логично --- это естественный этап эволюции любых сложных систем и технологий:
    \begin{scnitemize}
        \item Сначала накопление большого количества конкретных решений;
        \item Анализ полученного многообразия и превращение его в стройную систему качественно более высокого уровня.
    \end{scnitemize}
    Кризисов не надо бояться --- их надо преодолевать. Диалектику и, в частности, переход количества в качество ещё никто не отменял.}
    
\scnheader{Современное состояние технологии Искусственного интеллекта}
\scntext{пояснение}{К настоящему моменту мы научились разрабатывать \textit{интеллектуальные компьютерные системы} самого различного назначения. Но для повышения уровня автоматизации всё более и более широких видов человеческой деятельности необходим \uline{качественный} переход к разработке не отдельных \textit{интеллектуальных компьютерных систем}, а целых комплексов самостоятельно взаимодействующих между собой \textit{интеллектуальных компьютерных систем}.Это требует фундаментального переосмысления теории технологии проектирования \textit{интеллектуальных компьютерных систем}. Эффективный переход количества в новое качество требует серьезных усилий.}
\scntext{эпиграф}{Необходим переход от зоопарка локальных идей, сервисов и информационных ресурсов к их системе.}
\scntext{эпиграф}{From data science to knowledge science.}
\scntext{эпиграф}{Одно дело --- создавать локальные шедевры и совсем другое дело --- двигаться ко всеобщей гармонии.}

\scnheader{Современное состояние информационных технологии}
\scntext{примечание}{Экспертам в процессе обсуждения инновационных вопросов приходится тратить много времени на формирование нового \uline{понятийного аппарата}. <...> Мировому сообществу есть смысл задуматься над созданием нового искусственного языка, <...> чтобы с учётом возможностей современного мира сформулировать новую среду экспертного общения.}
\scnrelfrom{автор}{Курбацкий А.Н.}

\scnheader{будущие технологии Искусственного интеллекта}
\scntext{примечание}{\uline{\textit{Смысл}} той \textit{информации}, которой оперирует \uline{каждая}(!) \textit{интеллектуальная компьютерная система}, а также \uline{\textit{смысл}} того, что она делает (какие \textit{задачи} она решает) должен быть четко формализован и является частью её \textit{базы знаний}. Формализация этого \textit{смысла} (в состав которой входит экспертное согласование системы используемых \textit{понятий}) представляет собой первый этап проектирования каждой \textit{интеллектуальной компьютерной системы}, обеспечивающий \textit{семантическую совместимость} (взаимопонимание) \textit{интеллектуальных компьютерных систем} и эффективное их взаимодействие (самостоятельную организацию коллективной деятельности).Таким образом, необходим \textit{\uline{универсальный} формальный язык}, который используется как экспертами, разработчиками, так и непосредственно самими \textit{интеллектуальными компьютерными системами}.}

\scnheader{Технология OSTIS}
\begin{scnrelfromset}{теоретический компонент}
    \scnfileitem{Комплексная семантическая теория интеллектуальных компьютерных систем (ostis-систем).}
    \scnfileitem{Комплексная семантическая теория человеческой деятельности как Экосистемы (симбиоза) с иерархическим комплексом интеллектуальных компьютерных систем.}
    \scnfileitem{Теория перманентной эволюции (реинжиниринга) указанной Экосистемы (с минимизацией этапов локальной приостановки).}
    \begin{scnindent}
        \scntext{примечание}{Cтандарты должны меняться быстро, а Экосистема должна быстро приводиться в соответствие с новыми стандартами.}
    \end{scnindent}
\end{scnrelfromset}

\scnheader{Подготовка специалистов в области Искусственного интеллекта}
\scntext{примечание}{Массовая подготовка высококвалифицированных \textit{специалистов в области Искусственного интеллекта}, способных преодолеть современное кризисное состояние \textit{Искусственного интеллекта}, фактически и является самым главным фактом преодоления указанного кризиса.\\
    Необходимым условием и эпицентром вывода \textit{Искусственного интеллекта} из кризисного состояния и повышения темпов эволюции технологий \textit{Искусственного интеллекта} является организация \textit{подготовки специалистов в области Искусственного интеллекта} на основе активного привлечения студентов, магистрантов и аспирантов к \uline{перманентному} процессу эволюции \textit{Технологии OSTIS}.\\
    Очевидно, этому должно способствовать объединение соответствующей учебно-методической базы для разных кафедр, осуществляющих такую подготовку.\\
    На современном этапе развития \textit{Искусственного интеллекта} требуется не просто подготовка специалистов в этой области --- а подготовка специалистов \uline{принципиально новой формации}, способных:
    \begin{scnitemize}
        \item рассматривать область \textit{Искусственного интеллекта} не просто как многообразие \textit{интеллектуальных компьютерных систем}, а как постоянно эволюционируемую \uline{\textit{Экосистему}} таких систем;
        \item эффективно участвовать в решении как фундаментальных, системных, технологических проблем, так и практических, прикладных проблем эволюции указанной \textit{Экосистемы}.
    \end{scnitemize}
    Все это требует существенного переосмысления организации учебного процесса и учебно-методического обеспечения.\\
    <<Часто, например, совершенствование программных систем сводится к программным заплаткам. Через какое-то время мы имеем программу со множеством заплаток, как правило уже громоздкую и малоэффективную. В итоге --- иногда её проще выбросить и создать новую.>> (\cite{Kurbatski})\\
    Современная разработка каждой сложной программной системы требует построения \uline{качественной} формальной (цифровой) модели объекта управления, объекта автоматизации, причем \textit{семантически совместимой} с соответствующими моделями в смежных системах.\\
    Здесь важна общая математическая культура и унификация такой формализации.\\
    В настоящее время методологический подход к инженерной деятельности при разработке компьютерных систем часто выглядит следующим образом: <<Поставьте мне четкую инженерную задачу и я ее выполню. Но ответственность за ее постановку я с себя снимаю и не хочу учитывать критерии качества постановки задачи на более высоком уровне>>.\\
    Для наукоемких проектов, реализуемых в рамках развивающихся технологий, это недопустимо.\\
    Каждый инженер должен \uline{понимать}, что он делает и каковы истинные более глубокие критерии качества его результата.\\
    Нужна принципиально новая психологическая установка.\\
    Необходимо учитывать не только желание заказчика, но и общие принципы и стандарты разрабатываемых \textit{интеллектуальных компьютерных систем}.\\
    В основе организации образовательной деятельности на современном этапе развития \textit{Искусственного интеллекта} лежит:
    \begin{scnitemize}
        \item четкое формальное описание того, чему мы учим (каким знаниям и навыкам) --- в нашем случае это описание текущей версии \textit{Стандарта OSTIS} и направлений эволюции этого стандарта;
        \item уточнение того, что должен делать студент, магистрант и любой специалист для быстрого и качественного приобретения этих знаний и навыков.
    \end{scnitemize}
    Нужна \uline{комплексная} учебная программа по специальности \textit{Искусственный интеллект}, а не мозаика отдельных учебных дисциплин. И, соответственно этому необходимо \uline{комплексное} учебно-методическое пособие, достаточно полно отражающее текущее состояние теории и технологии проектирования \textit{интеллектуальных компьютерных систем}.\\
    Использование проектного метода при подготовке специалистов в области Искусственного интеллекта предполагает составление систематизированного сборника упражнений и задач, в частности, направленных на эволюцию Технологии OSTIS и посильных для студентов специальности \textit{Искусственный интеллект}:
    \begin{scnitemize}
        \item представление конкретных фрагментов различных предметных областей и онтологий;
        \item представление конкретных специфицированных методов (пополнение библиотек используемых методов из разных предметных областей, например, из теории графов);
        \item спецификация библиографических источников (в контексте \textit{Базы знаний Метасистемы OSTIS});
        \item выявление синонимии, омонимии, противоречий;
        \item сравнительный анализ и обзор близких внешних публикаций.
    \end{scnitemize}
    Таким образом, фронт самостоятельных, весьма полезных и посильных для студентов работ весьма широк. Главное сформировать у студентов профессиональный интерес, познавательную активность, инициативность и самостоятельность.}
\scntext{проектный метод}{Для того, чтобы научиться разрабатывать \textit{интеллектуальные компьютерные системы}, необходимо приобрести достаточно большой опыт участия и \uline{завершения} разработки реально востребованных \textit{интеллектуальных компьютерных систем}.}
\scntext{проектный метод}{Для того, чтобы научиться разрабатывать и совершенствовать \textit{технологии искусственного интеллекта}, необходимо приобрести достаточно большой опыт успешного (!) участия в создании различных компонентов комплексной \textit{технологии Искусственного интеллекта}.}
\scntext{пояснение}{Итак, современного специалиста в области \textit{Искусственного интеллекта} необходимо учить:
    \begin{scnitemize}
        \item не только тому, как следует разрабатывать \textit{интеллектуальные компьютерные системы} с помощью имеющихся (существующих) \textit{методов} и \textit{средств}, т.е. с помощью имеющихся \textit{технологий},
        \item но и тому, как надо развивать (совершенствовать) имеющиеся \textit{технологии}.
    \end{scnitemize}
    \textit{Технология OSTIS} рассматривается нами не столько, как предлагаемая технология разработки \textit{интеллектуальных компьютерных систем}, а как предлагаемые \uline{принципы} построения технологии разработки \textit{интеллектуальных компьютерных систем} следующего поколения. Т.е. фактически мы предлагаем не саму технологию (\textit{Технологию OSTIS}), а участие в её создании и развитии, которое может привести даже к радикальным изменениям текущего состояния (текущей версии) этой \textit{технологии}. Это психологически снимает ощущение навязывания предлагаемой технологии и заменяет его на атмосферу партнерства, направленного на перманентную эволюцию указанной технологии. Такой подход создаст также условия для существенного повышения качества \textit{подготовки специалистов в области Искусственного интеллекта}, поскольку дает возможность осуществлять обучение путём непосредственного вовлечения студентов и магистрантов в реальные, практически значимые процессы разработки \textit{интеллектуальных компьютерных систем}, а также в процессы совершенствования (эволюции) соответствующих \textit{технологий}.}
\scnheader{Анализ методологических проблем современного состояния работ в области Искусственного интеллекта}
\begin{scnrelfromvector}{примечания}
    \scnfileitem{Самые тяжелые кризисные ситуации в научно-технической сфере --- это те, которые носят фундаментальный и не совсем очевидный характер. Развитие кибернетики, информатики и искусственного интеллекта подтверждает это. За впечатляющими практическими и теоретическими достижениями незаметно возрастает огромный вал накладных расходов при разработке сложных больших систем --- возрастает дублирование, нестыковки, несогласованности.}
    \scnfileitem{Нет ничего более грустного, чем созерцать активную творческую деятельность большого количества умных людей, которые по инерции, не отдавая себе отчета, накапливают проблемы, препятствующие дальнейшему качественному развитию этой деятельности. Вместо того, чтобы разгребать эти проблемы на благо всем.}
    \scnfileitem{Как только мы начнем серьезно относиться к \uline{формальному} уточнению и согласованию всего многообразия понятий, используемых в области Искусственного интеллекта и различных его приложений, как только мы начнем \uline{реальную}(!) \uline{совместную} работу по общей комплексной формальной теории интеллектуальных компьютерных систем и по созданию комплексной технологии их проектирования, многие современные проблемы \textit{Искусственного интеллекта} начнут решаться. Нет ничего практичнее хорошей теории.}
    \scnfileitem{Основной лейтмотив развития технологий \textit{Искусственного интеллекта} --- это не только создание компьютерной технологии разработки сей совместной \textit{интеллектуальной компьютерной системы}, но и создание \uline{Метатехнологии} перманентной \uline{эволюции}(!) такой технологии. Иначе --- эклектика, усугубляющая современный кризис. Для создания эффективно и самостоятельно взаимодействующих \textit{интеллектуальных компьютерных систем} несущественных мелочей не бывает --- дьявол кроется в деталях и тонкостях. Важен не столько инжиниринг, сколько реинжиниринг интеллектуальных компьютерных систем и человеческой деятельности в целом.}
    \scnfileitem{Решение рассматриваемых кризисных проблем требует:\\
	        \begin{scnitemize}
	            \item Существенного фундаментального общесистемного переосмысления всего того, что мы творим.
	            \item Осознания того, что кибернетика, информатика и искусственный интеллект --- это общая фундаментальная наука, требующая единого серьезного математического аппарата.
	            \item Осознания того, что сейчас требуется не расширяемость многообразия точек зрения, а учиться их согласовывать, совершенствуя соответствующие методы.
	        \end{scnitemize}}
    \scnfileitem{Нам необходимо переходить от автоматизации отдельных видов \textit{человеческой деятельности} к интегрированной автоматизации всего комплекса человеческой деятельности, к созданию и постоянной эволюции всей общечеловеческой \textit{экосистемы}, состоящей из самостоятельно взаимодействующих \textit{интеллектуальных компьютерных систем} как между собой, так и между людьми, автоматизацию деятельности которой они осуществляют. При этом надо помнить, что основные накладные расходы, основные проблемы, возникают на стыках при интеграции различных технических решений. Разработчик каждой подсистемы должен гарантировать отсутствие указанных накладных расходов.}
    \scnfileitem{Самое главное --- надо ориентироваться не на создание идеальной информационной \textit{экосистемы}, а на создание эффективной технологии, направленной на перманентную эволюцию(!) указанной экосистемы.}
    \scnfileitem{Уникальность современного кризиса в области кибернетики, информатики и искусственного интеллекта заключается в том, что, несмотря на глобальность этого кризиса, абсолютно реально создать локальный эпицентр по разрешению этого кризиса --- в частности, в Республике Беларусь. Для этого есть все предпосылки --- специальность \textit{Искусственный интеллект}, опыт разработки компьютерных систем, достаточный научный уровень.}
\end{scnrelfromvector}

        \begin{scnrelfromset}{подвопрос}
    \scnfileitem{Недостатки современных интеллектуальных компьютерных систем}
    \scnfileitem{Недостатки современной технологии Искусственного интеллекта}
    \scnfileitem{Каким требованиям должна удовлетворять качественная технология разработки интеллектуальных компьютерных систем}
    \begin{scnindent}
        \begin{scnrelfromset}{подвопрос}
            \scnfileitem{уточнить требования, представляемые к интеллектуальным компьютерным системам (что такое интеллектуальная компьютерная система)}
            \scnfileitem{уточнить, почему этого нет}
            \scnfileitem{как эти требования удовлетворить в рамках интеллектуальных компьютерных систем (принципы)}
            \scnfileitem{уточнить требования к технологии}
            \scnfileitem{понять, уточнить, почему, что мешает созданию технологии}
            \begin{scnindent}
                \begin{scnrelfromset}{причина}
                    \scnfileitem{сложность объекта}
                    \scnfileitem{отсутствие понимания того, что задача такой сложности требует создания принципиально нового творческого коллектива с принципиально новой организацией взаимодействия}
                \end{scnrelfromset}
            \end{scnindent}
            \scnfileitem{как это сделать (принципы, лежащие в основе создания технологии интеллектуальных компьютерных систем)}
        \end{scnrelfromset}
    \end{scnindent}
    \scnfileitem{Что такое ИИ (как наука)}
    \begin{scnindent}
        \scniselement{научно-техническая дисциплина}
    \end{scnindent}
    \scnfileitem{Что такое интеллектуальная кибернетическая  система}
    \begin{scnindent}
        \scnsubset{кибернетическая система}
    \end{scnindent}
    \scnfileitem{Что такое технология проектирования и реализации интеллектуальная кибернетическая система}
    \scnfileitem{проблемы создания технологии проектирования}
    \scnfileitem{технология реализации от традиционных компьютеров к компьютерам, ориентированным на реализацию интеллектуальных кибернетических систем}
    \scnfileitem{Результат использования технологии проектирования и реализации --- это не отдельные интеллектуальные компьютерные системы и Экосистема из интеллектуальных компьютерных систем и людей}
    \scnfileitem{структура Экосистемы --- иерархическая система специализированных сообществ}
    \scnfileitem{Чем нас не устраивают те, интеллектуальные компьютерные системы, которые мы разрабатываем сейчас}
    \scnfileitem{Чем нас не устраивают современные технологии ИИ}
    \scnfileitem{Какие интеллектуальные компьютерные системы нам нужны}
    \scnfileitem{Какими свойствами и способностями мы хотели бы их наделить}
    \scnfileitem{высокая степень обучаемости в разных направлениях}
    \scnfileitem{расширение знаний без введения новых понятий}
    \scnfileitem{введение новых понятий без расширения многообразия видов знаний}
    \scnfileitem{расширение многообразия видов знаний}
    \scnfileitem{расширение моделей решения задач (новый вид методов + их интерпретация)}
    \scnfileitem{Какие технологии нам нужны}
    \scnfileitem{Почему таких икс и технологий ещё нет}
    \scnfileitem{Что мешает?}
    \scnfileitem{Что делать?}
    \scnfileitem{Какие недостатки имеют современные интеллектуальные системы}
    \scnfileitem{недостаточно высокий уровень интеллектуальности}
    \scnfileitem{нет эффективного взаимодействия (координации)}
    \scnfileitem{высокая степень обучаемости в разных направлениях}
    \scnfileitem{Какие недостатки имеют современные технологии Искусственного интеллекта}
    \scnfileitem{Какова трудоёмкость разработки выбранных икс}
    \scnfileitem{Какова трудоёмкость системной интеграции икс и их компонентов}
    \scnfileitem{Обеспечивается ли совместимость компонентов икс, разрабатываемых с помощью различных}
\end{scnrelfromset}

        \scnsegmentheader{Анализ структуры Деятельности в области Искусственного интеллекта}
\begin{scnsubstruct}
    \scntext{аннотация}{Для того, чтобы рассмотреть проблемы дальнейшего развития \textit{деятельности} в области \textit{Искусственного интеллекта} как \textit{научно-технической дисциплины} и, в частности, проблемы комплексной автоматизации этой \textit{деятельности}, необходимо уточнить структуру указанной \textit{деятельности}.}
    
	\scnheader{Человеческая деятельность в области Искусственного интеллекта}
	\scntext{примечание}{Человеческая деятельность в области \textit{Искусственного интеллекта} направлена на исследование и создание \textit{интеллектуальных компьютерных систем} различного вида и различного назначения.}
	\begin{scnhaselementrolelist}{объект исследования}
		\scnitem{индивидуальные интеллектуальные компьютерные системы}
		\begin{scnindent}
			\scnhaselement{когнитивные агенты}
		\end{scnindent}
		\scnitem{многоагентные интеллектуальные компьютерные системы}
		\begin{scnindent}
			\scnhaselement{сообщества, состоящие из индивидуальных интеллектуальных компьютерных систем}
		\end{scnindent}
		\scnitem{человеко-машинные сообщества, состоящие из интеллектуальных компьютерных систем и их пользователей}
	\end{scnhaselementrolelist}

	\scnheader{Искусственный интеллект}
    \scniselement{область человеческой деятельности}
    \scniselement{научно-техническая дисциплина}
    \begin{scnindent}
    	\scniselement{вид человеческой деятельности}
    \end{scnindent}
    \begin{scnrelfromlist}{цель}
		\scnfileitem{построение теории интеллектуальных систем}
		\scnfileitem{создание технологии разработки интеллектуальных компьютерных систем (искусственных интеллектуальных систем)}
		\scnfileitem{переход на принципиально новый уровень комплексной автоматизации всех \textit{видов человеческой деятельности}, который основан на массовом применение \textit{интеллектуальных компьютерных систем}}
		\begin{scnindent}
			\begin{scnrelfromlist}{детализация}
				\scnfileitem{наличие \textit{интеллектуальных компьютерных систем}, способных понимать друг друга и согласовывать свою деятельность}
				\scnfileitem{построение \textit{Общей теории человеческой деятельности}, осуществляемый в условиях нового уровня её автоматизации, (теории деятельности \textit{smart-общества}), которая должна быть понятна \textit{используемым интеллектуальным компьютерным системам} и которая потребует существенного переосмысления современной организации \textit{человеческой деятельности}}
			\end{scnrelfromlist}
		\end{scnindent}
	\end{scnrelfromlist}
	\begin{scnrelfromlist}{определение}
		\scnfileitem{Научно-техническая деятельность, направленная на построение теории интеллектуальных систем, а также на создание технологии проектирования и производства искусственных интеллектуальных систем (\textit{интеллектуальных компьютерных систем}).}
		\scnfileitem{Научно-техническая деятельность в области \textit{Искусственного интеллекта}.}
		\scnfileitem{Деятельность в области \textit{Искусственного интеллекта}}
		\scnfileitem{Научно-техническая деятельность, направленная на исследование феномена \textit{интеллекта}, а также на создание искусственных интеллектуальных систем (\textit{интеллектуальных компьютерных систем}) и включающая в себя соответствующую научно-исследовательскую деятельность, инженерно-технологическую, инженерно-прикладную, образовательную и организационную деятельность.}
		\scnfileitem{Междисциплинарная (трансдисциплинарная) область \textit{научно-технической деятельности}, направленная на разработку и эксплуатацию \textit{интеллектуальных компьютерных систем}, обеспечивающих автоматизацию различных сфер \textit{человеческой деятельности}.}
		\scnfileitem{Научно-техническая дисциплина направленная на разработку теории индивидуальных \textit{интеллектуальных компьютерных систем} и \textit{интеллектуальных сообществ} (коллективов) таких систем, а также средств поддержки их проектирования и реализации).}
		\scnfileitem{\textit{Научно-техническая дисциплина}, являющаяся частью \textit{кибернетики} (теория кибернетических систем), объектом исследования которой являются \textit{интеллектуальные компьютерные системы} (искусственные \textit{интеллектуальные системы}) и целями которой являются (1) разработка \textit{теории интеллектуальных компьютерных систем}, (2) разработка \textit{технологии(методов и средств)} \textit{проектирования и производства компьютерных систем}, а также, (3) разработка конкретных интеллектуальных компьютерных систем различного назначения.}
    \end{scnrelfromlist}
	\scnrelfrom{декомпозиция}{Декомпозиция Искусственного интеллекта по формам деятельности}
    \begin{scnindent}
		\begin{scneqtoset}
			\scnitem{Научно-исследовательская деятельность в области Искусственного интеллекта}
			\begin{scnindent}
				\scnrelfrom{продукт}{Общая теория интеллектуальных систем}
				\begin{scnindent}
					\scnidtf{Теория, уточняющая структуру и принципы функционирования \textit{интеллектуальных систем}, а также акцентирующая внимание на причинах (предпосылках) возникновение свойства интеллектуальности (феномены \textit{интеллекта})}
				\end{scnindent}
				\scniselement{коллективная научно-техническая деятельность}
				\begin{scnindent}
					\scniselement{вид человеческой деятельности}
					\scnidtf{научно-исследовательская дисциплина или направление}
					\scnrelboth{следует отличать}{продукт научно-исследовательской деятельности}
				\end{scnindent}
				\scntext{примечание}{В процессе \textit{Научно-исследовательской деятельности в области Искусственного интеллекта} осуществляется конкуренция различных точек зрения и подходов к построению формальных моделей различных компонентов \textit{интеллектуальных компьютерных систем}. Конечной целью такой деятельности является постоянно развиваемая \textit{Общая теория} \textit{интеллектуальных} \textit{компьютерных систем}, объектами исследования которой являются \textit{интеллектуальные компьютерные системы} и их формальные \textit{логико-семантические модели}, включающие в себя формальные модели различного вида \textit{знаний}, входящих в состав \textit{баз знаний} интеллектуальных компьютерных систем, а также различные \textit{модели решения задач} (логические модели различного вида, нейросетевые, генетические, продукционные, функциональные и другие).}
			\end{scnindent}
			\scnitem{Разработка Стандарта интеллектуальных компьютерных систем}
			\begin{scnindent}
				\scntext{примечание}{\textit{Разработка Стандарта интеллектуальных компьютерных систем} включает в себя перманентную эволюцию этого стандарта и поддержку целостности каждой его версии. Текущая версия \textit{Стандарта интеллектуальных компьютерных систем} --- это \uline{согласованная} (общепризнанная) \uline{на текущий момент} часть \textit{Общей теории интеллектуальных компьютерных систем}.}
			\end{scnindent}
			\scnitem{Разработка Базовой комплексной технологии проектирования интеллектуальных компьютерных систем}
			\begin{scnindent}
				\scnrelfrom{продукт}{Базовая комплексная технология проектирования интеллектуальных компьютерных систем}
				\begin{scnindent}
					\scntext{примечание}{Комплексность данной \textit{технологии} заключается в том, что она ориентирована (1) на проектирование \textit{интеллектуальных компьютерных систем} в целом, а не только отдельных их компонентов и (2) на создание объединённой \textit{технологии}, объединяющей самые различные технологические подходы на основе их \textit{конвергенции} и глубокой \textit{интеграции}.}
				\end{scnindent}
				\scniselement{разработка технологии проектирования искусственных объектов заданного класса}
				\begin{scnindent}
					\scniselement{вид человеческой деятельности}
				\end{scnindent}
				\scntext{примечание}{\textit{Разработка технологии проектирования интеллектуальных компьютерных систем} включает в себя семейство методик проектирования, а также методов и средств автоматизации \textit{проектирования} различных \textit{компонентов} \textit{интеллектуальных компьютерных систем} и \textit{интеллектуальных компьютерных систем} в целом. Результатом проектирования \textit{интеллектуальных компьютерных систем} является полная формальная логико-семантическая модель этой системы.}
			\end{scnindent}
			\scnitem{Разработка Технологии производства спроектированных интеллектуальных компьютерных систем}
			\begin{scnindent}
				\scnhaselement{Разработка технологии реализации спроектированных интеллектуальных компьютерных систем}
				\scnhaselement{Разработка технологий эксплуатации и сопровождения интеллектуальных компьютерных систем}
				\scnrelfrom{продукт}{Технология производства спроектированных интеллектуальных компьютерных систем}
				\begin{scnindent}
					\scnidtf{технология реализации (сборки и установки) спроектированных интеллектуальных компьютерных систем}
					\scntext{примечание}{Очевидно, что данная \textit{технология} должна быть самым тесным образом связана с \textit{Базовой комплексной технологией проектирования интеллектуальных компьютерных систем} (по крайней мере данная \textit{технология} должна знать в какой форме ей на вход передаётся результат проектирования). Поэтому имеет смысл говорить об объединённой технологии проектирования и производства \textit{интеллектуальных компьютерных систем}}
				\end{scnindent}
				\scniselement{производства спроектированных искусственных объектов заданного класса}
				\begin{scnindent}
					\scniselement{вид человеческой деятельности}
				\end{scnindent}
				\scntext{примечание}{В основе технологии реализации (производства) спроектированных \textit{интеллектуальных компьютерных систем} лежит \textit{универсальный интерпретатор формальных логико-семантических моделей интеллектуальных компьютерных систем}, являющихся результатом проектирования указанных систем. Указанный универсальный интерпретатор может быть реализован либо в виде \textit{программной системы} на современных компьютерах, либо в виде \textit{универсального компьютера нового поколения}, ориентированного на интерпретацию формальных \textit{логико-семантических моделей интеллектуальных компьютерных систем}.}
			\end{scnindent}
			\scnitem{Специализированная инженерия в области Искусственного интеллекта}
			\begin{scnindent}
				\scnidtf{Прикладная инженерная деятельность в области Искусственного интеллекта}
				\scnidtf{Множество Процессов разработки (проектирования и производства) \textit{интеллектуальных компьютерных систем} различного назначения, кроме \textit{интеллектуальных компьютерных систем автоматизации проектирования} и автоматизации производства \textit{интеллектуальных компьютерных систем}}
				\scnrelfrom{продукт}{множество специализированных интеллектуальных компьютерных систем}
				\begin{scnindent}
					\scnrelfrom{основной sc-идентификатор}{прикладная интеллектуальная компьютерная система}
				\end{scnindent}
				\scnsubset{проектирование и производство искусственного объекта заданного класса на основе заданной технологии}
				\begin{scnindent}
					\scniselement{вид человеческой деятельности}
				\end{scnindent}
				\scntext{примечание}{Прикладная инженерная деятельность в области Искусственного интеллекта, то есть непосредственное проектирование, реализация и сопровождение включает в себя обновление (реинжиниринг), осуществляемое в ходе эксплуатации, конкретных \textit{интеллектуальных компьютерных систем.}}
			\end{scnindent}
			\scnitem{Образовательная деятельность в области Искусственного интеллекта}
			\begin{scnindent}
				\scnidtf{Учебная деятельность в области Искусственного интеллекта}
				\scnidtf{Деятельность, направленная на подготовку молодых специалистов области \textit{Искусственного интеллекта} на перманентное повышение квалификации действующих специалистов в этой области}
				\scntext{примечание}{Сложность и высокая степень наукоемкости задач, больших своего решения на текущем этапе развития \textit{Искусственного интеллекта}, добавляют к специалистам, работающим в этой области высокие требования к уровню их:
					\begin{itemize}
						\item математической культуры (культуры формализации),
						\item системной культуры,
						\item технологической культуры,
						\item инженерная культура,
						\item умения работать в коллективных наукоемких проектах.
					\end{itemize}}
				\scnsubset{образовательная деятельность}
				\begin{scnindent}
					\scniselement{вид человеческой деятельности}
				\end{scnindent}
				\scntext{примечание}{\textit{Учебная деятельность в области Искусственного интеллекта} направлена на подготовку специалистов области \textit{Искусственного интеллекта} и на перманентное повышение квалификации действующих специалистов в этой области. Без эффективной организации учебной деятельности в области \textit{Искусственного интеллекта} быстрый прогресс в этой области невозможен. Непосредственное включение учебной деятельности в общую структуру человеческой деятельности в области \textit{Искусственного интеллекта} обусловлено следующими обстоятельствами:
					\begin{itemize}
						\item необходимостью глубокой \textit{конвергенции} между различными направлениями и видами деятельности в области \textit{Искусственного интеллекта} и соответствующей спецификой требований, предъявляемых к специалистам в этой области --- каждый такой специалист должен быть достаточно компетентен и в научно-исследовательской деятельности в области \textit{Искусственного интеллекта}, и в разработке технологий (методик и средств) \textit{проектирования интеллектуальных компьютерных систем}, и в разработке технологий \textit{воспроизводства} (реализации) спроектированных \textit{интеллектуальных компьютерных систем}, а также технологий их \textit{эксплуатации} и \textit{сопровождения}, и в прикладной \textit{инженерной деятельности в области} \textit{Искусственного интеллекта};
						\item высокими темпами эволюции результатов в области \textit{Искусственного интеллекта}, что делает необходимой организацию обучения соответствующих специалистов путем их непосредственного подключения не к учебным (упрощенным) проектам, а к реальным проектам, реализуемым в текущий момент. Иначе подготовленные специалисты будут иметь квалификацию \scnqq{вчерашнего дня};
						\item существенным расширением объемов работ в области \textit{Искусственного интеллекта} и острой необходимостью массовой подготовки соответствующих специалистов.
					\end{itemize}}
			\end{scnindent}
			\scnitem{Бизнес-деятельность в области искусственного интеллекта}
			\begin{scnindent}
				\scntext{пояснение}{Речь идет о бизнес-деятельности в широком смысле как деятельности, направленный на создание инфраструктурных условий для качественного выполнения всех \textit{видов деятельности} в области \textit{Искусственного интеллекта}}
				\begin{scnindent}
					\begin{scnrelfromlist}{пример}
						\scnfileitem{разработка и реализация грамотной научно-технической политики, связывающие как тактические, так и стратегические цели}
						\scnfileitem{глубокая \textit{конвергенцию} всех форм и \textit{видов деятельности} в области \textit{Искусственного интеллекта}}
						\scnfileitem{организация взаимовыгодного сотрудничество различных школ и коллективов, работающих в области \textit{Искусственного интеллекта}}
						\scnfileitem{финансовое обеспечение}
						\scnfileitem{кадровое обеспечение}
						\scnfileitem{материально-техническое обеспечение}
						\scnfileitem{организация проведения различных мероприятий (конференций, выставок, семинаров)}
						\scnfileitem{публикационная деятельность и защита интеллектуальной собственности}
						\scnfileitem{материально-техническое обеспечение}
					\end{scnrelfromlist}
				\end{scnindent}
				\scnsubset{бизнес-деятельность научно-технической области}
				\begin{scnindent}
					\scniselement{вид человеческой деятельности}
				\end{scnindent}
			\end{scnindent}
			\scnitem{Организационная деятельность в области Искусственного интеллекта}
			\begin{scnindent}
				\scntext{примечание}{\textit{Организационная деятельность в области Искусственного интеллекта} направлена на создание инфраструктуры для качественного выполнения всех остальных видов деятельности в области \textit{Искусственного интеллекта}, а именно:
				\begin{itemize}
					\item для обеспечения глубокой \textit{конвергенции} между различными направлениями и видами деятельности в области \textit{Искусственного интеллекта} и, в частности, между теорией, технологиями и инженерной практикой в этой области;
					\item для обеспечения баланса между тактикой и стратегией в развитии деятельности в области \textit{Искусственного интеллекта} как ключевой основы существенного повышения уровня автоматизации всевозможных видов \textit{человеческой деятельности} и перехода к \textit{smart-обществу}.
				\end{itemize}}
			\end{scnindent}
		\end{scneqtoset}
    \end{scnindent}
    \scntext{оценка}{Декомпозиция \textit{человеческой} \textit{деятельности} в области \textit{Искусственного интеллекта} по \textit{видам} \textit{деятельности} не является традиционным признаком декомпозиции \textit{научно-технических дисциплин}. Обычно декомпозиция \textit{научно-технических дисциплин} осуществляется по содержательным направлениям, которые соответствуют декомпозиции \textit{технических систем}, исследуемых и разрабатываемых в рамках этих \textit{научно-технических дисциплин}, то есть соответствуют выделению в этих \textit{технических системах} различного вида компонентов.}
    
    \scnheader{Искусственный интеллект}
    \scnrelfrom{разбиение}{Декомпозиция Искусственного интеллекта по видам деятельности}
    \begin{scnindent}
	    \begin{scneqtoset} 
	    	\scnfileitem{исследование и разработка формальных моделей и языков представления знаний}
	    	\scnfileitem{исследование и разработка баз знаний}
	    	\scnfileitem{исследование и разработка логических моделей обработки знаний}
	   		\scnfileitem{исследование и разработка искусственных нейронных сетей}
	   		\scnfileitem{исследование и разработка подсистем компьютерного зрения}
	   		\scnfileitem{исследование и разработка подсистем обработки естественно-языковых текстов (синтаксический анализ, понимание, синтез)}
	    \end{scneqtoset}
    \end{scnindent}
    \scntext{примечание}{Важность декомпозиции \textit{Искусственного интеллекта} по \textit{видам} \textit{деятельности} определяется тем, что выделение различных \textit{видов деятельности} позволяет четко ставить задачу на разработку средств автоматизации этих \textit{видов деятельности}.}
    
   \scnheader{Cтруктура Человеческой деятельности в области Искусственного интеллекта}
   \begin{scnstruct}
    	\scnheader{Человеческая деятельность в области Искусственного интеллекта}
    	\scnidtf{Искусственный интеллект (как научно-техническая дисциплина)}
    	\scniselement{научно-техническая дисциплина}
    	\scnidtf{Искусственный интеллект (как научно-техническая дисциплина)}
    	\scnidtf{Человеческая деятельность в Предметной области интеллектуальных компьютерных систем}
    	\scniselement{деятельность}
    	\begin{scnrelfromset}{декомпозиция}
    		\scnitem{Интегральная деятельность по поддержке жизненного цикла всевозможных интеллектуальных компьютерных систем}
    		\begin{scnindent}
    			\scnrelfrom{декомпозиция}{поддержка жизненного цикла интеллектуальных компьютерных систем}
    			\begin{scnindent}
    				\scniselement{вид деятельности}
    				\scnsuperset{поддержка жизненного цикла ostis-систем}
    				\begin{scnrelfromset}{обобщенная декомпозиция}
    					\scnitem{проектирование интеллектуальных компьютерных систем}
    					\scnitem{производство интеллектуальных компьютерных систем}
    					\scnitem{начальное обучение интеллектуальных компьютерных систем}
    					\scnitem{мониторинг качества интеллектуальных компьютерных систем}
    					\scnitem{восстановление требуемого уровня качества интеллектуальных компьютерных систем}
    					\scnitem{реинжиниринг интеллектуальных компьютерных систем}
    					\scnitem{обеспечение безопасности интеллектуальных компьютерных систем}
    					\scnitem{эксплуатация интеллектуальных компьютерных систем конечными пользователями}
    				\end{scnrelfromset}
    			\end{scnindent}
    		\end{scnindent}
    		\scnitem{Поддержка жизненного цикла Общей теории интеллектуальных компьютерных систем}
    		\begin{scnindent}
    			\scniselement{научно-исследовательская деятельность}
    		\end{scnindent}        
    		\scnitem{Поддержка жизненного цикла Стандарта интеллектуальных компьютерных систем}
    		\begin{scnindent}
    			\scniselement{стандартизация}
    			\scnrelfrom{часть}{Поддержка жизненного цикла Стандарта ostis-систем}
    		\end{scnindent}        
    		\scnitem{Поддержка жизненного цикла Технологии комплексной поддержки жизненного цикла интеллектуальных компьютерных систем}
    		\begin{scnindent}
    			\scniselement{поддержка жизненного цикла технологий}
    			\begin{scnindent}
    				\scnidtf{создание и сопровождение технологий}
    			\end{scnindent}
    			\scnrelfrom{часть}{Поддержка жизненного цикла Технологии OSTIS}
    		\end{scnindent}
    		\scnitem{Поддержка жизненного цикла кадровых ресурсов для Человеческой деятельности в области Искусственного интеллекта}
    		\scnitem{Поддержка жизненного цикла системы комплексной организации взаимодействия между всеми направлениями Человеческой деятельности в области Искусственного интеллекта}
    		\begin{scnindent}
    			\scniselement{поддержка жизненного цикла метасистем комплексного управления поддержкой и обеспечением поддержки жизненного цикла сущностей соответствующего класса}
    		\end{scnindent}        
    	\end{scnrelfromset}
    \end{scnstruct}
    
    \scnheader{Человеческая деятельность в области Искусственного интеллекта}
    \begin{scnrelfromset}{практический результат}
    	\scnfileitem{Реорганизация и комплексная автоматизация \textit{человеческой деятельности в области Искусственного интеллекта} с помощью \textit{интеллектуальных компьютерных систем нового поколения}.}
    	\scnfileitem{\uline{Поэтапное} создание глобальной сети эффективно взаимодействующих \textbf{\textit{интеллектуальных компьютерных систем нового поколения}}, обеспечивающих \uline{комплексную} автоматизацию всевозможных видов и областей \textit{человеческой деятельности}.}
    \end{scnrelfromset}
    	
    \scnheader{Технология поддержки жизненного цикла интеллектуальных компьютерных систем}
    \scnrelfrom{вид деятельности}{поддержка жизненного цикла интеллектуальных компьютерных систем}
    \begin{scnrelfromset}{декомпозиция}
    	\scnitem{Технология проектирования интеллектуальных компьютерных систем}
    	\begin{scnindent}
    		\scnrelfrom{вид деятельности}{проектирование интеллектуальных компьютерных систем}
    	\end{scnindent}
    	\scnitem{Технология производства интеллектуальных компьютерных систем}
    	\begin{scnindent}
    		\scnrelfrom{вид деятельности}{производство интеллектуальных компьютерных систем}
    	\end{scnindent}
    	\scnitem{Технология начального обучения интеллектуальных компьютерных систем (адаптации к конкретной деятельности)}
    	\begin{scnindent}
    		\scnrelfrom{вид деятельности}{начальное обучение интеллектуальных компьютерных систем}
    	\end{scnindent}
    	\scnitem{Технология мониторинга качества интеллектуальных компьютерных систем}
    	\begin{scnindent}
    		\scnrelfrom{вид деятельности}{мониторинг качества интеллектуальных компьютерных систем}
    		\begin{scnindent}
    			\scnidtf{плановое тестирование и диагностика интеллектуальных компьютерных систем}
    		\end{scnindent}
    	\end{scnindent}
    	\scnitem{Технология восстановления требуемого уровня качества интеллектуальных компьютерных систем в ходе их эксплуатации}
    	\begin{scnindent}
    		\scnidtf{Технология выявления и исправления потенциально опасных ситуаций и событий в деятельности интеллектуальных компьютерных систем (ошибок, противоречий, и так далее)}
    		\scnrelfrom{вид деятельности}{восстановление требуемого уровня качества интеллектуальных компьютерных систем}
    	\end{scnindent}
    	\scnitem{Технология реинжиниринга  интеллектуальных компьютерных систем}
    	\begin{scnindent}
    		\scnidtf{Технология совершенствования, модернизации, обновления интеллектуальных компьютерных систем}
    		\scnrelfrom{вид деятельности}{реинжиниринг интеллектуальных компьютерных систем}
    	\end{scnindent}
    	\scnitem{Технология обеспечения безопасности интеллектуальных компьютерных систем}
    	\begin{scnindent}
    		\scnrelfrom{вид деятельности}{обеспечение безопасности интеллектуальных компьютерных систем}
    	\end{scnindent}
    	\scnitem{Технология эксплуатации интеллектуальных компьютерных систем конечными пользователями}
    	\begin{scnindent}
    		\scnrelfrom{вид деятельности}{эксплуатация интеллектуальных компьютерных систем конечными пользователями}
    	\end{scnindent}
    \end{scnrelfromset}
    
    \scnheader{Разработка базовой комплексной технологии проектирования интеллектуальных компьютерных систем}
    \begin{scnrelfromset}{декомпозиция}
        \scnitem{Разработка общей теории интеллектуальных компьютерных систем}
		\begin{scnindent}
			\scnrelfrom{продукт}{Общая теория интеллектуальных компьютерных систем}
			\scniselement{разработка теории искусственных объектов заданного класса}
			\begin{scnindent}
				\scniselement{вид человеческой деятельности}
			\end{scnindent}
		\end{scnindent}
        \scnitem{Разработка общей теории интеллектуальных компьютерных систем}
		\begin{scnindent}
			\scnrelfrom{продукт}{Теория проектирования интеллектуальных компьютерных систем}
			\begin{scnindent}
				\scntext{примечание}{В состав этой теории входят методы проектирования, библиотеки проектирования и спецификация используемых индустриальных средств.}
			\end{scnindent}
			\scnidtf{Разработка \textit{Теории проектной деятельности} по построению формальных моделей \textit{интеллектуальных компьютерных систем}}
			\scniselement{теории проектирования интеллектуальных объектов заданного класса}
			\begin{scnindent}
				\scniselement{вид человеческой деятельности}
			\end{scnindent}
		\end{scnindent}
        \scnitem{Разработка комплекса средств автоматизации проектирования интеллектуальных компьютерных систем}
		\begin{scnindent}
			\scniselement{разработка комплекса средств автоматизации проектирования искусственных объектов заданного класса}
			\begin{scnindent}
				\scniselement{вид человеческой деятельности}
			\end{scnindent}
		\end{scnindent}
    \end{scnrelfromset}
    
    \scnheader{Разработка Технологии производства спроектированных интеллектуальных компьютерных систем}
    \begin{scnrelfromset}{декомпозиция}
        \scnitem{Разработка Теории производства спроектированных интеллектуальных компьютерных систем}
		\begin{scnindent}
			\scnidtf{Разработка \textit{Теории производственной деятельности} по реализации (сборке и установке) \textit{интеллектуальных компьютерных систем}}
			\scnrelfrom{продукт}{Теория производства спроектированных интеллектуальных компьютерных систем}
			\begin{scnindent}
				\scntext{примечание}{В состав этой \textit{теории} входят \textit{методы производства} (сборки и установки) \textit{интеллектуальных компьютерных систем}, а также спецификация \textit{используемых инструментальных средств}.}
			\end{scnindent}
		\end{scnindent}
        \scnitem{Разработка комплекса средств автоматизации производства интеллектуальных компьютерных систем}
		\begin{scnindent}
			\scniselement{разработка комплекса средств автоматизации производства искусственных объектов заданного класса}
			\begin{scnindent}
				\scniselement{вид человеческой деятельности}
			\end{scnindent}
		\end{scnindent}
    \end{scnrelfromset}
    
    \scnheader{Специализированная инженерия в области Искусственного интеллекта}
    \scnidtf{проектирование и производство конкретной \textit{интеллектуальной компьютерной системы} по заданной технологии}
    \begin{scnrelfromset}{обобщенная декомпозиция}
        \scnitem{проектирование конкретных интеллектуальных компьютерные системы}
		\begin{scnindent}
			\begin{scnrelfromset}{обобщенное разбиение}
				\scnitem{проектирование интеллектуальной компьютерной системы автоматизации проектирования соответствующего класса интеллектуальных компьютерных систем}
				\scnitem{проектирование интеллектуальной компьютерной системы автоматизации проектирования соответствующего класса объектов, не являющихся интеллектуальными компьютерными системами}
				\scnitem{проектирование интеллектуальной компьютерной системы,не являющейся системой автоматизации проектирования}
			\end{scnrelfromset}
		\end{scnindent}
        \scnitem{производство конкретной спроектированной интеллектуальной компьютерной системы}
    \end{scnrelfromset}
    
    \scnheader{производство конкретной спроектированной интеллектуальной компьютерной системы}
    \scntext{примечание}{Производственная деятельность, направленная на \textit{производство} (реализацию) спроектированной \textit{интеллектуальной компьютерной системы} значительно уступает по уровню сложности деятельности по проектированию этой \textit{интеллектуальной компьютерной системы}, так как это производство сводится к сборке результата \textit{проектирования} (формальной логико-семантической модели разрабатываемой \textit{интеллектуальной компьютерной системы}) и загрузки этой модели в память компьютера или программного \textit{универсального интерпретатора логико-семантических моделей интеллектуальных компьютерных систем}, в качестве которого может быть использован:
        \begin{itemize}
            \item либо специально разработанный для этого \textit{компьютер}, ориентированные на обработку \textit{баз знаний} и интерпретацию различных \textit{интеллектуальных моделей решения задач},
            \item либо программная эмуляция такого \textit{компьютера}, реализованная на современных \textit{компьютерах} фон-неймановской архитектуры.
        \end{itemize}
        Простота производства спроектированных систем характерна для производства не только интеллектуальных, но и любых других \textit{компьютерных систем}.
        \\Мы выделяем производственный этап реализации \textit{интеллектуальных компьютерных систем} для того, чтобы по аналогии рассматривать этап производства (массового, мелкосерийного, разового производства) спроектированных искусственных объектов любого другого вида(микросхем, автомобилей, зданий, компьютеров).
        \\Очевидно, что массовое производство некоторых видов продукции может иметь весьма большой уровень сложности, но при этом суть \textit{производственной деятельности} как процесса перехода от проекта (спецификации) некоторого объекта к его реализации остаётся одной и той же независимо от уровня сложности реализуемого объекта (производимой продукции).}
        
    \scnheader{следует отличать*}
    \begin{scnhaselementset}
        \scnitem{специализированная интеллектуальная компьютерная система}
        \scnitem{интеллектуальная компьютерная система автоматизации проектирования интеллектуальных компьютерных систем}
        \scnitem{интеллектуальная компьютерная система автоматизации производства спроектированных интеллектуальных компьютерных систем}
    \end{scnhaselementset}
    \begin{scnhaselementset}
        \scnitem{человеческая деятельность}
		\begin{scnindent}
			\scnsuperset{научно-исследовательская деятельность}
			\scnsuperset{научно-техническая деятельность}
		\end{scnindent}
        \scnitem{продукт человеческой деятельности}
		\begin{scnindent}
			\scnsuperset{продукт научно-исследовательской деятельности}
			\scnsuperset{продукты научно-технической деятельности стакан}
		\end{scnindent}
    \end{scnhaselementset}
    
    \scnheader{Искусственный интеллект}
    \scnrelfrom{декомпозиция}{Декомпозиция Искусственного интеллекта по направлениям}
    \begin{scnindent}
	    \begin{scneqtoset}
	        \scnitem{Разработка теории представления знаний и технологии проектирования баз знаний актуальных компьютерных систем}
	        \scnitem{Разработка теории решения задач и технологии проектирования решателей задач интеллектуальных компьютерных систем}
			\begin{scnindent}
				\begin{scnrelfromset}{декомпозиция}
					\scnitem{Разработка теории решения интерфейсных задач и технологии проектирования соответствующих решателей}
					\begin{scnindent}
						\scnrelfrom{часть}{Разработка теории естественно языковых интерфейсов интеллектуальных компьютерных систем и технологии их проектирования}
					\end{scnindent}
					\scnitem{Разработка теории решения информационных задач в базах знаний интеллектуальных компьютерных систем и технологии проектирования соответствующих решателей}
					\scnitem{Разработка теории решения поведенческих задач во внешней среде интеллектуальных компьютерных систем и технологии проектирования соответствующих решателей}
				\end{scnrelfromset}
				\begin{scnrelfromset}{декомпозиция}
					\scnitem{Разработка логических моделей решения задач и технологии проектирования соответствующих решателей}
					\scnitem{Разработка нейросетевых моделей решение задач и технологии проектирования соответствующих решателей}
				\end{scnrelfromset}
			\end{scnindent}
	        \scnitem{Разработка универсальных интерпретаторов базовых моделей обработки баз знаний интеллектуальных компьютерных систем}
	        \scnitem{Разработка общей теории человеческой деятельности, автоматизируемой с помощью комплекса взаимодействующих интеллектуальных компьютерных систем}
	    \end{scneqtoset}
    \end{scnindent}
    \scntext{примечание}{Переход от современных интеллектуальных компьютерных систем к \textit{интеллектуальным компьютерным системам нового поколения} и к соответствующей комплексной технологии не требует от специалистов в области Искусственного интеллекта изменения сферы их научных интересов. От них требуется только преодолеть синдром \scnqqi{Вавилонского столпотворения}, оформляя свои научные результаты как часть общего коллективного продукта.}
    \scntext{примечание}{Проблемы текущего этапа развития \textit{Искусственного интеллекта}, направленного на создание Общей теории и технологии \textit{интеллектуальных компьютерных систем нового поколения}, требуют \uline{фундаментального} комплексного междисциплинарного подхода и принципиально новой организации научно-технической деятельности.}  

\bigskip
\end{scnsubstruct}
\scnendcurrentsectioncomment

        \scnsegmentheader{Анализ текущего состояния и проблем дальнейшего развития деятельности в области Искусственного интеллекта}
\begin{scnsubstruct}
    \scntext{аннотация}{Рассмотрим в каких направлениях должна происходить эволюция повышенного качества деятельности в области \textit{Искусственного интеллекта}, а также эволюция продуктов этой деятельности}
    \scntext{введение}{В настоящее время научные исследования в области \textit{Искусственного интеллекта} активно развиваются по широкому спектру различных направлений (\textit{модели представления знаний}, различного вида \textit{логики} --- дедуктивные, индуктивные, абдуктивные, четкие, нечеткие, различного вида \textit{искусственные нейронные сети}, машинное обучение, принятие решений, целеполагание, планирование поведения, ситуационное поведение, многоагентные системы, компьютерное зрение, распознавание, интеллектуальный анализ данных, мягкие вычисления и многое другое). В данной предметной области будут рассмотрены и проанализированы проблемы, связанные с научно-исследовательской деятельности в области Искусственного интеллекта.}

    \bigskip
    
    \scnheader{Современное состояние человеческой деятельности в области Искусственного интеллекта}
    \begin{scnrelfromvector}{оценка}
        \scnfileitem{Рассмотрим необходимость перехода организации \textit{человеческой деятельности} \textit{Искусственного интеллекта} на принципиально новый уровень, обеспечивающий формирование рынка \textit{семантически совместимых} \textit{интеллектуальных компьютерных систем} \textit{нового поколения}, разрабатываемых на основе принципиально нового комплекса \textit{семантически совместимых} \textit{технологий Искусственного интеллекта}.}
        \scnfileitem{Сейчас актуально исследовать не только \textit{модели решения интеллектуальных задач} в интеллектуальных компьютерных системах различного вида, но также методологические проблемы текущего состояния \textit{Искусственного интеллекта} в целом и пути решения этих проблем.}
        \scnfileitem{Анализ современного состояния работ в области \textit{Искусственного интеллекта} показывает то, что указанная научно-техническая дисциплина находится в серьезном методологическом кризисе. Поэтому необходимо:
            \begin{itemize}
                \item Выявление основных причин возникновения указанного кризиса;
                \item Уточнение основных мер, направленных на его устранение.
            \end{itemize}}
        \begin{scnindent}
            \begin{scnrelfromlist}{требование}
                \scnfileitem{Существенное фундаментальное общесистемное переосмысление всего того, \textit{что} мы делаем в области \textit{Искусственного интеллекта} и \textit{как} мы это делаем, то есть требование уточнения характеристик \textit{интеллектуальных компьютерных систем}, уточнения понятия сообщества, состоящего из \textit{интеллектуальных компьютерных систем} и взаимодействующих с ними пользователей, уточнения требований, предъявляемых к \textit{интеллектуальным компьютерным системам}, а также уточнения методик и средств их создания и использования.}
                \scnfileitem{Осознание того, что \textit{Кибернетика, Информатика} и \textit{Искусственный интеллект} --- это общая фундаментальная наука, требующая комплексного подхода к построению общих формальных моделей систем, основанных на обработке информации (\textit{кибернетических систем}), путем \textit{конвергенции} и \textit{интеграции} формальных моделей различных компонентов этих систем. Таким образом, современный этап развития \textit{Искусственного интеллекта} --- это переход от накопленного к текущему моменту многообразия моделей решения различного вида задач к преобразованию этого многообразия в стройную систему \textit{семантически совместимых} моделей.}
	            \begin{scnindent}
		            \begin{scnrelfromset}{смотрите}
	        			\scnitem{\scncite{Palagin2013}}
	        			\scnitem{\scncite{Yankovskaya2017}}
	                 \end{scnrelfromset}
	            \end{scnindent}
                \scnfileitem{Осознание того, что сейчас требуется не расширять многообразие точек зрения, а учиться их согласовывать, обеспечивать их \textit{семантическую совместимость}, совершенствуя соответствующие методы.}
            \end{scnrelfromlist}
        \end{scnindent}
        \scnfileitem{Обсуждая современную проблематику \textit{конвергенции} различных моделей в области \textit{Искусственного интеллекта} и построения интегрированных \textit{гибридных моделей}, уместно вспомнить \scnqqi{фантастический рассказ Д. А. Поспелова \scnqq{Соприкосновение}, посвященный контакту различных миров. В нем главный герой популярно излагает свою теорию \textit{концептуальных разломов} <...>. Эта теория напоминает историю долгого периода \uline{дифференциации} наук, когда различные научные дисциплины развивались \uline{независимо}, словно параллельные миры, лишь изредка соприкасаясь друг с другом, а отдельные ученые, получая все более узкую специализацию, мало что знали о достижениях даже своих \scnqqi{близких собратьев}. К счастью, в последние годы все чаще и чаще возникают новые области контакта между отдельными дисциплинами, происходит взаимопроникновение идей, установление \uline{аналогий} между полученными результатами и тенденциями развития. Во многом это объясняется появлением и широким внедрением во все сферы жизни общества передовых информационных и коммуникационных технологий <...>. Современные технологии опираются на достижения многих научно-технических дисциплин, среди которых на первый план выходят \uline{синтетические науки нового поколения} --- науки об искусственном}.}
        \begin{scnindent}
            \scnrelto{цитата}{\scncite{Tarasov2002}/с.13}
        \end{scnindent}
        \scnfileitem{Анализируя современное состояние работ в области \textit{Искусственного интеллекта (ИИ)}, следует констатировать то, что \textit{концептуальный разлом} между различными направлениями \textit{Искусственного интеллекта} является очевидным фактом. Это подтверждается следующей цитатой из книги В. Б. Тарасова \scnqqi{вновь, как и на заре ИИ, актуальными становятся формирование единых методологических основ ИИ, разработка теоретических проблем создания интеллектуальных систем новых поколений, развитие нетрадиционных аппаратно-программных средств. Здесь большие перспективы связаны с использованием идей и принципов синергетики в ИИ. Сам термин \scnqq{синергетика} происходит от слова \scnqq{синергия}, означающего совместное действие, сотрудничество. По мнению \scnqq{отца синергетики} Г. Хакена, такое название вполне подходит для современной теории сложных самоорганизующихся систем по двум причинам: а) исследуются совместные действия многих элементов развивающейся системы; б) для отыскания общих принципов самоорганизации требуется объединение усилий представителей различных дисциплин}.}
        \begin{scnindent}
            \scnrelto{цитата}{\scncite{Tarasov2002}/с.14}
        \end{scnindent}
        \scnfileitem{Для того, чтобы убедиться в наличии \textit{концептуального разлома} между различными направлениями \textit{Искусственного интеллекта}, достаточно просто перечислить основные направления работы конференций по тематике \textit{Искусственного интеллекта}, обращая внимание на то, что многие из них развиваются независимо от других.}
            \begin{scnindent}
            	\begin{scnrelfromlist}{пример}
                \scnfileitem{синергетические модели самоорганизации интеллектуальных компьютерных систем}
                \scnfileitem{гибридные интеллектуальные компьютерные системы}
                \scnfileitem{коллаборативные интеллектуальные компьютерные системы}
                \scnfileitem{мягкие вычисления, интеллектуальные вычисления}
                \scnfileitem{моделирование не-факторов}
                \scnfileitem{неклассические, многозначные, модальные, псевдофизические, индуктивные, нечеткие логики и приближенные рассуждения, логические программы}
                \scnfileitem{нечеткие множества, отношения, графы, алгоритмы}
                \scnfileitem{функциональные программы, нечеткие алгоритмы, генетические алгоритмы, продукционные модели}
                \scnfileitem{нейросетевые модели}
                \scnfileitem{параллельные асинхронные модели децентрализованного решения задач}
                \scnfileitem{обработка сигналов}
                \scnfileitem{мультисенсорная конвергенция, сенсо-моторная координация}
                \scnfileitem{модели ситуационного управления}
                \end{scnrelfromlist}
            \end{scnindent}
        \scnfileitem{Преодоление \textit{концептуального разлома} между различными направлениями исследований в области \textit{Искусственного интеллекта} --- это, своего рода \scnqq{прыжок} через \scnqq{концептуальную пропасть}, который требует особой концентрации усилий. Через пропасть нельзя перепрыгнуть двумя прыжками.}
        \scnfileitem{Если кратко охарактеризовать \textbf{текущее состояние} всего комплекса работ в области \textbf{\textit{Искусственного интеллекта}}, то это --- \textbf{иллюзия благополучия}. Происходит активное \uline{локальное} развитие самых различных направлений \textit{Искусственного интеллекта} (\textit{неклассические логики}, \textit{формальные онтологии}, \textit{искусственные нейронные сети}, \textit{машинное обучение}, \textit{мягкие вычисления}, \textit{многоагентные системы} и так далее), но \uline{комплексного} повышения уровня \textit{интеллекта} современных \textit{интеллектуальных компьютерных систем} не происходит. Для этого прежде всего требуется сближение и \textit{интеграция} \uline{всех} направлений \textit{Искусственного интеллекта} и соответствующее построение \textbf{\textit{Общей формальной теории интеллектуальных компьютерных систем}}, а также превращение современного многообразия \textit{инструментальных средств} (frameworks) разработки различных компонентов \textit{интеллектуальных компьютерных систем} в единую \textbf{\textit{Технологию комплексного проектирования и поддержки всего жизненного цикла интеллектуальных компьютерных систем}}, гарантирующую \uline{совместимость} всех разрабатываемых компонентов \textit{интеллектуальных компьютерных систем}, а также совместимость самих \textit{интеллектуальных компьютерных систем} как \uline{самостоятельных} субъектов (агентов, акторов), взаимодействующих между собой в рамках комплексных систем автоматизации сложных видов коллективной \textit{человеческой деятельности} (умных домов, умных больниц, умных школ, умных производственных предприятий, умных городов и так далее). Таким образом, эпиграфом текущего состояния работ в области \textit{Искусственного интеллекта} является известное высказывание из Экклезиаста: \scnqq{Время разбрасывать камни и время собирать камни --- всему свое время}.}
        \scnfileitem{\scnqqi{К сожалению, в современных дискуссиях по теме ИИ (Искусственного интеллекта) научные споры часто подменяются завышенными ожиданиями от скорого внедрения ИИ и значительным сужением темы ИИ, которая оказалась сведена лишь к \textit{машинному обучению} на основе \textit{искусственных нейронных сетей}. <...> При этом за бортом Национальной стратегии пока остались \textit{онтологии}, \textit{базы знаний}, \textit{методы рассуждений} и \textit{принятия решений}, \textit{методы синтеза и} \textit{анализа сложных конструкций}, умные кибер-физические системы, \textit{цифровые двойники}, \textit{автономные системы}, системы анализа как \scnqq{больших}, так и \scnqq{малых} данных. <...>}}
        \scnfileitem{Признавая всю важность \textit{машинного обучения} на базе \textit{искусственных} \textit{нейронных сетей}, научные и практические результаты мирового уровня следует искать на стыке разных дисциплин в \textbf{\textit{конвергенции}} различных технологий ИИ и \textbf{\textit{интеграции}} полипредметных \textit{знаний}. В этой связи формализация \textit{знаний} в виде \textit{онтологий} и \textit{баз знаний} в рамках \textit{Semantic Web} рассматривается как одно из фундаментальных направлений для создания \textit{Искусственного интеллекта}. Действительно, какой же может быть \textit{интеллект} без использования \textit{знаний} современных учебников, на основе чего ИИ будет понимать \textit{контекст ситуации}, делать \textit{выводы} и \textit{принимать решения}? <...>}
        \scnfileitem{Еще одной ключевой сферой ИИ, не нашедшей отражения в Российской стратегии по ИИ, является \textit{распределенное принятие решений}, которое все больше становится коллективным для стремительно развивающихся систем умного Интернета вещей и автономных систем управления, начиная с беспилотных автомобилей, самолетов, кораблей и так далее.}
        \begin{scnindent}
        	\scnrelto{цитата}{\scncite{Barinov2021}/с. 264-265}
        \end{scnindent}
        \scnfileitem{Компанией Гартнер 2020 год был объявлен годом \scnqq{автономных вещей}, которые по мнению компании уже прошли большую эволюцию от \scnqq{цифровых} к \scnqq{умным}. Ожидается, что на следующем этапе автономные вещи, обладающие собственным \textit{искусственным интеллектом}, \scnqq{заговорят} друг с другом и в научную повестку войдут вопросы \textbf{\textit{семантической интероперабельности}} систем \textit{Искусственного интеллекта}, которые будут не только обмениваться данными, но и вести переговоры для согласования решений. Дорожная карта научных исследований по \textit{Искусственному интеллекту} США в качестве ключевых выделяет такие направления, как \textit{связность} систем \textit{Искусственного интеллекта} (Integrated Intelligence) и их \textit{осмысленное взаимодействие} (Meaningful Interaction), наряду с разными видами \textit{самообучения} в системах (Self-Aware Learning).}
        \begin{scnindent}
            \scnrelto{цитата}{\scncite{Barinov2021}/с. 264-265}
        \end{scnindent}
        \scnfileitem{\textbf{Ключевой причиной} \textbf{методологических проблем} современного состояния \textit{Искусственного интеллекта} и серьезным вызовом для специалистов в этой области является проклятие \textbf{\textit{Вавилонского столпотворения}}, которое преследует нас на всех уровнях:
            \begin{itemize}
                \item на уровне внутренней организации \textit{решения задач} в \textit{интеллектуальных компьютерных системах};
                \item на уровне взаимодействия \textit{интеллектуальных компьютерных систем} как между собой, так и с пользователями;
                \item на уровне взаимодействия ученых, работающих в области \textit{Искусственного интеллекта}, что препятствует созданию \textit{Общей формальной теории и стандарта интеллектуальных компьютерных систем}, а также \textit{Технологии комплексного проектирования и поддержки всего жизненного цикла интеллектуальных компьютерных систем}
                \item на уровне взаимодействия между учеными, инженерами, разрабатывающими прикладные \textit{интеллектуальные компьютерные системы}, преподавателями ВУЗов, которые готовят специалистов в области \textit{Искусственного интеллекта}, а также студентами, магистрантами и аспирантами.
            \end{itemize}}
         \begin{scnindent}
        	\begin{scnrelfromset}{смотрите}
        		\scnitem{\scncite{Iliadis2019}}
        	\end{scnrelfromset}
        \end{scnindent}
        \scnfileitem{Сложность разрабатываемых в настоящее время \textit{интеллектуальных компьютерных систем} и технологий \textit{Искусственного интеллекта} достигла такого уровня, что для их разработки требуются не просто большие творческие коллективы, но и существенное повышение квалификации и качества этих коллективов. Как известно, квалификация коллектива разработчиков определяется не только квалификацией его членов, но также эффективностью и атмосферой их взаимодействия. Известно также, что качество любой технической системы является отражением качества того коллектива, который эту систему разработал. Может ли коллектив достаточно квалифицированных специалистов, многие из которых не обладают высоким уровнем \textit{интероперабельности}, разработать интеллектуальную компьютерную систему с высоким уровнем \textit{интероперабельности}, а тем более технологию комплексной поддержки всего жизненного цикла \textit{интеллектуальных компьютерных систем} такого уровня? Очевидный ответ на этот вопрос и очевидная сложность создания работоспособных творческих коллективов указывают на основной вызов, адресованный специалистам в области \textit{Искусственного интеллекта} в настоящее время. Таким образом, требования, предъявляемые к \textit{интеллектуальным компьютерным системам нового поколения} и определяющие их способность к индивидуальному и коллективному решению комплексных сложных задач, должны предъявляться и к разработчикам этих систем, а также к разработчикам любых других сложных объектов, поскольку все сложные виды и области человеческой деятельности являются коллективными и творческими.}
        \scnfileitem{Создание быстро развивающегося рынка семантически совместимых \textit{интеллектуальных компьютерных систем} --- это основная цель, адресованная специалистам в области \textit{Искусственного интеллекта}, требующая преодоления \textbf{\textit{Вавилонского столпотворения}} во всех его проявлениях, формирования высокой культуры договороспособности и унифицированной, согласованной формы представления коллективно накапливаемых, совершенствуемых и используемых знаний. Ученые, работающие в области \textit{Искусственного интеллекта}, должны обеспечить \textbf{\textit{конвергенцию}} результатов различных направлений \textit{Искусственного интеллекта} и построить \textit{Общую формальную теорию интеллектуальных компьютерных систем}, а также \textit{Комплексную технологию проектирования семантически совместимых интеллектуальных компьютерных систем,} включающую соответствующие стандарты \textit{интеллектуальных компьютерных систем} и их компонентов. Инженеры, \textit{разрабатывающие прикладные интеллектуальные компьютерные системы}, должны сотрудничать с учеными и участвовать в развитии \textit{Комплексной технологии проектирования семантически совместимых интеллектуальных компьютерных систем}, и поддержки всех последующих этапов жизненного цикла этих систем.}
        \scnfileitem{Разобщенность различных направлений исследований в области \textit{Искусственного интеллекта} является главным препятствием создания \textit{Комплексной технологии проектирования семантически совместимых интеллектуальных компьютерных систем}, а также \textit{Технологии комплексной поддержки} всех последующих этапов жизненного цикла \textit{интеллектуальных компьютерных систем}.}
    \end{scnrelfromvector}

    \scnheader{Научно-исследовательская деятельность в области Искусственного интеллекта}
    \begin{scnrelfromset}{проблемы текущего состояния}
        \scnfileitem{Отсутствует согласованность систем \textit{понятий} в разных направлениях \textit{Искусственного интеллекта} и, как следствие, отсутствует \textit{семантическая совместимость} и \textit{конвергенция} этих направлений, в результате чего ни о каком движении в направлении построения \textit{общей теории интеллектуальных систем} с высоким уровнем формализации и речи быть не может. Существование и продолжающееся увеличение высоты барьеров между различными направлениями исследований в области \textit{Искусственного интеллекта} проявляется в том, что специалист, работающий в рамках какого-либо направления \textit{Искусственного интеллекта}, посещая заседания не своей секции на конференции по \textit{Искусственному интеллекту}, мало что там может понять и, соответственно, извлечь полезного для себя.}
        \scnfileitem{Отсутствует мотивация и осознание острой необходимости в указанной \textit{конвергенции} между различными направлениями \textit{Искусственного интеллекта}.}
        \scnfileitem{Отсутствует реальное движение в направлении построения \textit{Общей теории интеллектуальных систем}, поскольку отсутствует соответствующая мотивация и осознание острой практической необходимости в этом.}
        \scnfileitem{Отсутствует строгое и согласованное уточнение понятия \textit{интеллектуальной компьютерные системы}. До сих пор для этого используется Тест Тьюринга. Поверхностная трактовка Теста Тьюринга породила различные имитации интеллекта в стиле \scnqqi{светского} разговора или разговора на \scnqqi{завалинке}. На самом деле должна учитываться содержательная, целевая установка диалога, в рамках которого интеллект \textit{интеллектуальной компьютерной системы} определяется как ее нетривиальный вклад в коллективное решение некоторой интеллектуальной (творческой) задачи.}
    \end{scnrelfromset}
    
    \scnheader{Текущее состояние унификации интеллектуальных компьютерных систем}
    \begin{scnrelfromvector}{примечание}
        \scnfileitem{Стандарты в самых различных областях являются важнейшим видом знаний, обеспечивающих согласованность различных видов массовой деятельности. Но для того, чтобы стандарты не тормозили прогресс, они должны постоянно совершенствоваться.}
        \scnfileitem{Стандарты должны эффективно и грамотно использоваться. Поэтому оформление стандартов в виде текстовых документов не удовлетворяет современным требованиям.}
        \scnfileitem{Стандарты должны быть оформлены в виде интеллектуальных справочных систем, которые способны отвечать на самые различные вопросы. Таким образом стандарты целесообразно оформлять в виде баз знаний, соответствующих интеллектуальных справочных систем. При этом указанные интеллектуальные справочные системы могут осуществлять координацию деятельности разработчиков стандартов, направленной на совершенствование этих стандартов.}
        \begin{scnindent}
            \begin{scnrelfromset}{смотрите}
                \scnitem{\scncite{Serenkov2004}}
                \scnitem{\scncite{ItApkit}}
                \scnitem{\scncite{Volkov2015}}
             \end{scnrelfromset}
        \end{scnindent}
        \scnfileitem{С семантической точки зрения каждый стандарт есть иерархическая онтология, уточняющих структуру и систем понятий соответствующих им предметных областей, которая описывает структуру и функционирование либо некоторого класса технических или иных искусственных систем, либо некоторого класса организаций, либо некоторого вида деятельности.}
        \scnfileitem{Очевидно, что для построения интеллектуальной справочной системы по стандарту и интеллектуальной системы поддержки коллективного совершенствования стандарта необходима формализация стандарта, в виде соответствующей формальной онтологии.}
        \scnfileitem{Конвергенция различных видов деятельности, а также конвергенция результатов этой деятельности требует глубокой семантической конвергенции (семантической совместимости) соответствующих стандартов, для чего также настоятельно необходима формализация стандартов.}
        \scnfileitem{Следует также заметить, что важнейшей методологической основой формализации стандартов и обеспечения их семантической совместимости и конвергенции является построение иерархической системы формальных онтологий и соблюдение \textit{Принципа Бритвы Оккама}.}
        \scnfileitem{В настоящее время необходимость унификации и стандартизации \textit{интеллектуальных компьютерных систем} не осознана, что существенно тормозит создание \textit{комплексной технологии} \textit{Искусственного интеллекта}.}
    \end{scnrelfromvector}

    \scnheader{Разработка базовой комплексной технологии проектирования интеллектуальных компьютерных систем}
    \scntext{текущее состояние}{Современная технология \textit{Искусственного интеллекта} представляет собой целое семейство всевозможных частных технологий, ориентированных на разработку и сопровождение различного вида компонентов \textit{интеллектуальных компьютерных систем}, реализующих самые различные модели представления и обработки информации, различные модели решения задач, ориентированных на разработку различных классов \textit{интеллектуальных компьютерных систем}.}
    \begin{scnrelfromset}{проблемы текущего состояния}
        \scnfileitem{Высокая трудоемкость разработки интеллектуальных компьютерных систем.}
        \scnfileitem{Необходимая высокая квалификация разработчиков.}
        \scnfileitem{Современные технологии \textit{Искусственного интеллекта} принципиально не обеспечивают разработки таких \textit{интеллектуальных компьютерных систем}, в которых устраняются недостатки современных \textit{интеллектуальных компьютерных систем}.}
        \scnfileitem{Совместимость частных технологий \textit{Искусственного интеллекта} практически отсутствует и, как следствие, отсутствует \textit{семантическая совместимость} разрабатываемых \textit{интеллектуальных компьютерных систем}, поэтому их системная интеграция осуществляется \uline{вручную}.}
        \scnfileitem{Разрабатываемые \textit{интеллектуальные компьютерные системы} не способны \uline{самостоятельно} координировать свою деятельность друг с другом следовательно:\\
            \begin{itemize}
                \item нет общей комплексной технологии проектирования интеллектуальных компьютерных систем;
                \item не обеспечивается совместимость и взаимодействие разрабатываемых систем (синтаксическая и семантическая совместимость);
                \item нет совместимости между существующими частными технологиями проектирования различных компонентов интеллектуальных компьютерных систем (базы знаний, нейросетевые модели, интеллектуальные интерфейсы и т.д.);
                \item есть инструментальные средства по разработке компонентов, но склеивать (соединять, интегрировать) разработанные компоненты надо вручную, то есть нет комплексных инструментальных средств, позволяющих разрабатывать интеллектуальные системы в целом.
            \end{itemize}}
    \end{scnrelfromset}
    
    \scnheader{Разработка технологии производства спроектированных интеллектуальных компьютерных систем}
    \scntext{текущее состояние}{Был сделан целый ряд попыток разработки \textit{компьютеров} нового поколения, ориентированных на использование в \textit{интеллектуальных компьютерных системах}. Но все они оказались неудачными, так как не были ориентированы на всё многообразие моделей решения задач в \textit{интеллектуальных компьютерных системах}. В этом смысле они не были \textit{\uline{универсальными} компьютерами} для \textit{интеллектуальных компьютерных систем}.}
    \begin{scnrelfromset}{проблемы текущего состояния}
        \scnfileitem{Разрабатываемые \textit{интеллектуальные компьютерные системы} могут использовать самые различные комбинации \textit{моделей решения интеллектуальных задач} (логических моделей, соответствующих различного вида логикам, нейросетевых моделей различного вида, моделей целеполагания, синтеза планов, моделей управления сложными объектами, моделей понимания и синтеза текстов естественного языка и т.д.). Современные (традиционные, фон-неймановские) \textit{компьютеры} не в состоянии достаточно производительно интерпретировать всё многообразие указанных моделей решения задач. При этом разработка специализированных \textit{компьютеров}, ориентированных на интерпретацию какой-либо одной модели решения задач (нейросетевой модели или какой-либо логической модели) проблему не решает, так как в \textit{интеллектуальной компьютерной системе} необходимо использовать сразу несколько разных моделей решения задач, причём в различных сочетаниях.}
        \scnfileitem{В настоящее время отсутствует комплексный подход к технологическому обеспечению всех этапов жизненного цикла \textit{интеллектуальных компьютерных систем} --- не только к поддержке проектирования и реализации (сборки, производства) \textit{интеллектуальных компьютерных систем}, но также и к технологической поддержке сопровождения, реинжиниринга и эксплуатации \textit{интеллектуальных компьютерных систем}.}
        \scnfileitem{Семантическая недружественность \textit{пользовательского интерфейса} и отсутствие встроенных интеллектуальных справочных систем, позволяющих запрашивать информацию об элементах интерфейса и возможностях системы, приводят к низкой эффективности эксплуатации всех возможностей \textit{интеллектуальной компьютерной системы}.}
    \end{scnrelfromset}
    
    \scnheader{Специализированная инженерия в области Искусственного интеллекта}
    \scnidtf{Деятельность, направленная на разработку \textit{интеллектуальных компьютерных систем} различного назначения с использованием имеющихся для этого моделей, методов и средств}
    \scnidtf{Деятельность по проектированию и производству \textit{интеллектуальных компьютерных систем}}
    \scnidtf{Деятельность, направленная на формирование рынка \textit{интеллектуальных компьютерных систем}}
    \scnrelfrom{в перспективе}{Специализированная инженерия в области \textit{Искусственного интеллекта}, осуществляемая специальной частью Экосистемы OSTIS}
	    \begin{scnindent}
		    \scnrelfrom{продукт}{Экосистема OSTIS}
		    \scnrelfrom{субъект действия}{часть Экосистемы OSTIS, осуществляющая специализированную инженерию в области \textit{Искусственного интеллекта}}
	    \end{scnindent}
    \scntext{текущее состояние}{Накоплен достаточно большой опыт разработки \textit{интеллектуальных компьютерных систем} самого различного назначения --- систем автоматизации медицинской диагностики, а также диагностики сложных технических систем, интеллектуальных обучающих, информационно-справочных и help-систем, систем естественно-языкового общения, интеллектуальных компьютерных персональных ассистентов, интеллектуальных корпоративных систем, интеллектуальных систем ситуационного управления различного рода сложными объектами, систем интеллектуального анализа больших данных, систем технического зрения и анализа сцен, интеллектуальных порталов знаний, интеллектуальных систем автоматизации проектирования различного вида сложных объектов, интеллектуальных систем автоматизации подготовки к производству спроектированной продукции различного вида, интеллектуальных автоматизированных систем управления производства различного вида продуктов, а также многих других \textit{интеллектуальных компьютерных систем}.}
    \begin{scnrelfromset}{проблемы текущего состояния}
        \scnfileitem{Уровень эффективности практического использования научных результатов в области \textit{Искусственного интеллекта} явно не соответствует современному уровню развития самих этих научных результатов. Для того чтобы повысить уровень эффективности практического использования указанных научных результатов, \uline{необходимы} \uline{совместные усилия} и ученых, создающих новые модели решения интеллектуальных задач, и разработчиков технологий проектирования и реализации \textit{интеллектуальных компьютерных систем}, и разработчиков прикладных \textit{интеллектуальных компьютерных систем.}}
        \scnfileitem{Отсутствует четкая систематизация многообразия \textit{интеллектуальных компьютерных систем}, соответствующая систематизации автоматизируемых \textit{видов человеческой деятельности}.}
        \scnfileitem{Отсутствует \textit{конвергенция} \textit{интеллектуальных компьютерных систем}, обеспечивающих автоматизацию \textit{областей человеческой деятельности}, принадлежащих одному и тому же \textit{виду человеческой деятельности}.}
        \scnfileitem{Отсутствует \textit{семантическая совместимость}(семантическая унификация, взаимопонимание) между \textit{интеллектуальными компьютерными системами}, основной причиной чего является отсутствие согласованной системы общих используемых \textit{понятий}.}
        \scnfileitem{Семантическая недружественность \textit{пользовательского интерфейса} и отсутствие встроенной справочной системы, позволяющей запрашивать информацию об элементах интерфейса и возможностях системы, приводят к низкой эффективности эксплуатации всех возможностей \textit{интеллектуальной компьютерной системы}.}
        \scnfileitem{Анализ проблем автоматизации всех \textit{видов человеческой деятельности} убеждает в том, что дальнейшая автоматизация \textit{человеческой деятельности} требует не только повышения уровня \textit{интеллекта} соответствующих \textit{интеллектуальных компьютерных систем}, но и реализации их способности:\\
            \begin{itemize}
                \item устанавливать свою \textit{семантическую совместимость} (взаимопонимание) как с другими \textit{компьютерными системами}, так и со своими пользователями;
                \item поддерживать эту \textit{семантическую совместимость} в процессе собственной эволюции, а также эволюции пользователей и других \textit{компьютерных систем};
                \item координировать свою деятельность с пользователями и другими \textit{компьютерными системами} при коллективно решении различных задач;
                \item участвовать в распределении работ (подзадач) при коллективном решении различных задач.
            \end{itemize}
            Важно подчеркнуть то, что реализация вышеперечисленных способностей создаст возможность для существенной и даже полной автоматизации \textit{системной интеграции} \textit{компьютерных систем} в комплексы взаимодействующих систем и автоматизации реинжиниринга таких комплексов. Такая автоматизация системной интеграции и её реинжиниринга:\\
            \begin{itemize}
                \item даст возможность комплексам кибернетических систем \uline{самостоятельно} адаптироваться к решению новых задач;
                \item существенно повысит эффективность эксплуатации таких комплексов компьютерных систем, так как реинжиниринг системной интеграции компьютерных систем, входящих в такой комплекс, часто востребован (например, при реконструкции предприятия);
                \item существенно сокращает число ошибок по сравнению с ручным (неавтоматизированным) выполнением \textit{системной интеграции} и её \textit{реинжиниринга}, которые, к тому же, требует высокой квалификации.
            \end{itemize}
            Таким образом следующий этап повышения уровня автоматизации \textit{человеческой деятельности} настоятельно требует создания таких \textit{интеллектуальных компьютерных систем}, которые могли бы легко сами (без системного интегратора) объединяться для совместного решения сложных задач.}
    \end{scnrelfromset}
    
    \scnheader{Образовательная деятельность в области искусственного интеллекта}
    \scntext{текущее состояние}{Целенаправленная подготовка специалистов в области Искусственного интеллекта имеет богатую историю и осуществляется во многих ведущих университетах (Stanford University, MIT, МГУ (Москва), НИУ МЭИ (Москва), РГГУ (Москва), СПбГУ (Санкт-Петербург), ДВФУ (Владивосток), НГТУ (Новосибирск), НТУУ КПИ (Киев), БГУИР (Минск), БГУ (Минск), БрГТУ (Брест) и других).}
    \begin{scnrelfromset}{проблемы текущего состояния}
        \scnfileitem{Поскольку деятельность в области \textit{Искусственного интеллекта} сочетает в себе и высокую степень наукоемкости и высокую степень сложности инженерных работ, подготовка специалистов в этой области требует одновременного формирования у них как научно-исследовательских навыков, культуры и стиля мышления, так и инженерно-практических навыков, культуры и стиля мышления. С точки зрения методики и психологии обучения сочетание фундаментальной научной и инженерно-практической подготовки специалистов является весьма сложный образовательной педагогической задачей.}
        \scnfileitem{Отсутствует \textit{семантическая совместимость} между различными учебными дисциплинами, что приводит к мозаичности восприятия информации}
        \scnfileitem{Отсутствует системный подход к подготовке молодых специалистов в области \textit{Искусственного интеллекта}}
        \scnfileitem{Нет персонификации обучения, а также установки на выявление, раскрытие и развитие индивидуальных способностей}
        \scnfileitem{Отсутствует целенаправленное формирование мотивации к творчеству}
        \scnfileitem{Нет формирования навыков работы в реальных коллективах разработчиков}
        \scnfileitem{Отсутствует адаптация к реальной практической деятельности}
        \scnfileitem{Любая современная технология (в том числе и Технология OSTIS) должна иметь высокие темпы своего развития, поскольку без этого невозможно поддерживать высокий уровень её конкурентоспособности. Но для быстро развиваемой технологии требуется:
            \begin{itemize}
                \item не просто высокая квалификация кадров, использующих и развивающих технологию,
                \item но и высокие \uline{темпы} повышения уровня этой квалификации, так как без этого невозможно эффективно использовать и развивать \uline{быстро меняющуюся} технологию.
            \end{itemize}
            Из этого следует, что образовательная деятельность в области \textit{Искусственного интеллекта} и соответствующая ей технология должна быть не просто важной частью деятельности в области \textit{Искусственного интеллекта}, а частью, глубоко интегрированной во все остальные виды деятельности в области \textit{Искусственного интеллекта}. Так, например, каждая \textit{интеллектуальная компьютерная система} должная быть ориентирована не только на обслуживание своих конечных пользователей, не только на организацию целенаправленного взаимодействия со своими разработчиками, которые постоянно совершенствуют эту систему, и не только на обеспечение минимального порога вхождения для новых конечных пользователей и разработчиков, но и на организацию постоянного и персонифицированного повышения квалификации каждого своего конечного пользователя и разработчика в условиях постоянных изменений, вносимых в указанную \textit{интеллектуальную компьютерную систему}. Для этого эксплуатируемая \textit{интеллектуальная компьютерная система} должна знать, что в ней изменилось, на что она способна и как эти способности инициировать (содержание и форма, соответствующих пользовательских команд).}
    \end{scnrelfromset}
    \scnhaselement{Подготовка молодых специалистов в области Искусственного интеллекта}
    \scntext{примечание}{Когда мы говорим о \textit{конвергенции} и \textit{интеграции} в области \textit{Искусственного интеллекта}, речь идет не только о конвергенции между \textit{интеллектуальными компьютерными системами}, но также и между различными \uline{видами} и областями \textit{человеческой деятельности}. Таким образом, \textit{учебная деятельность}, направленная на подготовку специалистов в области \textit{Искусственного интеллекта}, органически входит в состав \textit{деятельности в области} \textit{Искусственного интеллекта}, а важнейшим направлением повышения эффективности этой деятельности является ее \textit{конвергенция} и \textit{интеграция} с другими видами \textit{деятельности в области Искусственного интеллекта}.}
    
    \scnheader{Подготовка молодых специалистов в области Искусственного интеллекта}
    \scntext{проблемы}{Сложность \textit{Подготовки молодых специалистов в области Искусственного интеллекта} заключается не только в высокой степени наукоемкости этой области, но и в том, что формирование у них соответствующих знаний и навыков осуществляется в условиях быстрого морального старения текущего состояния технологий \textit{Искусственного интеллекта}, существенные изменения в которых происходят за время обучения студентов и магистрантов. Поэтому надо учить не текущему уровню развития \textit{Искусственного интеллекта}, а тому уровню развития, который будет достигнут через пять и более лет.}
    \scntext{примечание}{При подготовке молодых специалистов в области \textit{Искусственного интеллекта} необходимо формировать у них:
    	\begin{itemize}
    		\item культуру формализации (математическую культуру);
    		\item системную культуру (в частности, умение осуществлять качественную стратификацию сложных динамических систем);
    		\item технологическую культуру (в частности, умение отличать то, что следует унифицировать и то, унификация чего ограничивает направление эволюции заданного класса сложных систем);
    		\item технологическую дисциплину;
    		\item культуру коллективного творчества (в частности, первоначальную \textit{интероперабельность});
    		\item высокую \textit{познавательную активность} и мотивацию;
    		\item умение сочетать индивидуальную творческую свободу и самостоятельность с обеспечением совместимости своих результатов с результатами коллег, то есть сочетать свободу в создании (порождении) новых смыслов при согласованности (совместимости) форм их представления --- о понятиях, терминах и синтаксисе не спорят, а договариваются.
    	\end{itemize}}
    
    \scnheader{Бизнес-деятельность в области Искусственного интеллекта}
    \scnidtf{Организационная деятельности в области Искусственного интеллекта}
    \scntext{текущее состояние}{Острая потребность в существенном повышении уровня автоматизации в самых различных областях человеческой деятельности (в промышленности, медицине, транспорте, образовании, строительстве и во многих других), а также современные результаты в развитии \textit{технологий Искусственного интеллекта} привели к существенному расширению работ по созданию \textit{прикладных интеллектуальных компьютерных систем} и к появлению большого количества коммерческих организаций, ориентированных на разработку таких приложений.}
    \begin{scnrelfromset}{проблемы текущего состояния}
        \scnfileitem{Не так просто обеспечить баланс тактических и стратегических направлений развития всех форм деятельности в области \textit{Искусственного интеллекта} (научно-исследовательской деятельности, разработки технологии проектирования и производства интеллектуальных компьютерных систем, разработки прикладных систем, образовательной деятельности), а также баланс между всеми перечисленными формами деятельности.}
        \scnfileitem{В настоящее время отсутствует глубокая конвергенция различных форм деятельности в области \textit{Искусственного интеллекта} (в первую очередь, конвергенция развития технологий \textit{Искусственного интеллекта} и разработки различных прикладных интеллектуальных компьютерных систем), что существенно затрудняет развитие каждой из этих форм.}
        \scnfileitem{Высокий уровень наукоемкости работ в области \textit{Искусственного интеллекта} предъявляет особые требования к квалификации сотрудников и к их способности работать в составе творческих коллективов.}
        \scnfileitem{Для повышения квалификации своих сотрудников и для обеспечения высокого уровня своих разработок необходимо активное сотрудничество с различными научными школами, с кафедрами, осуществляющими подготовку молодых специалистов в области \textbf{\textit{Искусственного интеллекта}}, активное участие в подготовке и проведении соответствующих конференций, семинаров, выставок.}
    \end{scnrelfromset}
    
    \scnheader{Искусственный интеллект}
    \begin{scnrelfromset}{\scnkeyword{сверхзадачи текущего состояния}}
        \scnfileitem{Построение и перманентное развитие \textit{общей формальной теории интеллектуальных систем}}
	        \begin{scnindent}
	        	\scntext{уточнение}{Построение \textbf{\textit{Общей формальной теории интеллектуальных компьютерных систем}}, в рамках которой была бы обеспечена совместимость всех направлений \textit{Искусственного интеллекта}, всех моделей представления знаний, всех моделей решения задач, всех компонентов \textit{интеллектуальных компьютерных систем}.}
		        \begin{scnrelfromset}{подзадачи}
		            \scnfileitem{Уточнение требований, предъявляемых к интеллектуальным компьютерным системам --- уточнение свойств интеллектуальных компьютерных систем, определяющих высокий уровень их интеллекта.}
		            \scnfileitem{Конвергенция и интеграция всевозможных видов знаний и всевозможных моделей решения задач в рамках каждой интеллектуальной компьютерной системы.}
		            \scnfileitem{Ориентация на последующую разработку унифицированных семантически совместимых формальных моделей интеллектуальных систем.}
		            \scnfileitem{Ориентация на разработку различного вида универсальных интерпретаторов формальных моделей интеллектуальных систем (и в том числе компьютеров нового поколения ) и обеспечение четкой стратификации между формальными моделями интеллектуальных систем и различными вариантами построения их интерпретаторов, обеспечивающей высокую степень независимости эволюции формальных моделей интеллектуальных систем и эволюции их интерпретаторов. Это требует особой детализации формальных моделей интеллектуальных систем.}
		            \scnfileitem{Обеспечение коммуникационной (социальной) совместимости (договороспособности) интеллектуальных компьютерных систем, позволяющей им самостоятельно формировать коллективы интеллектуальных компьютерных систем и их пользователей, а также самостоятельно согласовывать (координировать) деятельность в рамках этих коллективов при решении сложных задач в непредсказуемых условиях. Без этого невозможна реализация таких проектов, как умный дом, умный город, умное предприятие, умная больница и т.д.}
		        \end{scnrelfromset}
		    \end{scnindent}
		\scnfileitem{Создание \textit{инфраструктуры}, обеспечивающей интенсивное перманентное развитие \textit{Общей формальной теории интеллектуальных компьютерных систем} в самых различных направлениях, гарантирующее сохранение логико-семантической целостности этой \textit{теории} и совместимости всех направлений ее развития.}
		\scnfileitem{Создание \textit{инфраструктуры}, обеспечивающей интенсивное перманентное развитие \textit{Комплексной технологии разработки и эксплуатации интеллектуальных компьютерных систем нового поколения} в самых различных направлениях, гарантирующее сохранение целостности этой \textit{технологии} и совместимости всех направлений ее развития.}
		\scnfileitem{На основе \textit{Общей формальной теории интеллектуальных компьютерных систем} построение \textit{Технологии комплексной поддержки жизненного цикла интеллектуальных компьютерных систем нового поколения}, обладающих высоким уровнем \textit{интероперабельности} и совместимости.}
        \scnfileitem{Создание и перманентное развитие \textit{общей комплексной технологии} проектирования и производства \textit{семантически совместимых} \textit{интеллектуальных компьютерных систем}, способных координировать свою деятельность с себе подобными.}
        	\begin{scnindent}
		        \begin{scnrelfromset}{подзадачи}
		            \scnfileitem{Четкое описание стандарта интеллектуальных компьютерных систем, обеспечивающего семантическую совместимость разрабатываемых систем.}
		            \scnfileitem{Разработка мощных библиотек семантически совместимых и многократно (повторно) используемых компонентов разрабатываемых интеллектуальных компьютерных систем.}
		            \scnfileitem{Обеспечение низкого порога вхождения в технологию проектирования интеллектуальных компьютерных систем как для пользователей технологии (т.е. разработчиков прикладных или специализированных интеллектуальных компьютерных систем), так и для разработчиков самой технологии.}
		            \scnfileitem{Обеспечение высоких темпов развития технологии за счет учета опыта разработки различных приложений путем активного привлечения авторов приложений к участию в развитии (совершенствовании) технологии.}
		        \end{scnrelfromset}
		 	\end{scnindent}
        \scnfileitem{Разработка компьютеров нового поколения, ориентированных на производство высокопроизводительных \textit{интеллектуальных компьютерных систем} самого различного назначения и высокого качества.}
        \scnfileitem{Создание глобальной \textit{экосистемы} взаимодействующих между собой \textit{интеллектуальных компьютерных систем}, обеспечивающих комплексную автоматизацию всех \textit{видов человеческой деятельности}.}
	        \begin{scnindent}
	        	\scntext{подзадача}{Построение формальной модели человеческой деятельности в контексте теории smart-общества.}
	        \end{scnindent}
        \scnfileitem{Создание и перманентное развитие глобальной \textit{социотехнической экосистемы}, которая состоит из \textit{интеллектуальных компьютерных систем}, а также всех пользователей этих систем, которая обеспечивает комплексную автоматизацию всех \textit{видов человеческой деятельности}.}
        \scnfileitem{Необходим переход от эклектичного построения сложных \textit{интеллектуальных компьютерных систем}, использующих различные виды \textit{знаний} и различные виды \textit{моделей решения задач}, к их глубокой \textit{\textbf{интеграции}} и унификации, когда одинаковые модели представления и модели обработки знаний реализуется в разных системах и подсистемах одинаково.}
        \scnfileitem{Необходимо сократить дистанцию между современным уровнем \textbf{\textit{теории интеллектуальных компьютерных систем}} и практики их разработки.}
    \end{scnrelfromset}
    \scnidtf{Деятельность в области Искусственного интеллекта (как совокупность всех форм и направлений этой деятельности)}
    \scntext{проблема текущего состояния}{Эпицентром современных проблем развития деятельности в области \textit{Искусственного интеллекта} является \textit{конвергенция} и \textit{глубокая интеграция} всех форм, направлений и результатов этой деятельности. Уровень взаимосвязи, взаимодействия и \textit{конвергенции} между различными формами и направлениями деятельности в области \textit{Искусственного интеллекта} явно недостаточен. Это приводит к тому, что каждая из них развивается обособленно, независимо от других.  Речь идет о \textit{конвергенции} между такими направлениями \textit{Искусственного интеллекта}, как представление знаний, решение интеллектуальных задач, интеллектуальное поведение, понимание и др., а также между такими формами \textit{человеческой деятельности в области Искусственного интеллекта}, как научные исследования, разработка технологий, разработка приложений, образование, бизнес. Почему на фоне уже достаточно длительного интенсивного развития научных исследований в области \textit{Искусственного интеллекта} до сих пор не создан рынок интеллектуальных компьютерных систем и комплексная технология \textit{Искусственного интеллекта}, обеспечивающая разработку широкого спектра \textit{интеллектуальных компьютерных систем} самого различного назначения и доступной широкому контингенту инженеров. Потому что сочетание высокого уровня наукоемкости и прагматизма этой проблемы требует для ее решения принципиально нового подхода к организации взаимодействия \textit{\uline{ученых}}, работающих в области \textit{Искусственного интеллекта}, \textit{\uline{разработчиков}} средств автоматизации проектирования \textit{интеллектуальных компьютерных систем}, \uline{\textit{разработчиков}} средств реализации интеллектуальных компьютерных систем, включая средства аппаратной поддержки интеллектуальных компьютерных систем, \uline{\textit{разработчиков}} прикладных интеллектуальных компьютерных систем. Такое \uline{целенаправленное} взаимодействие должно осуществляться как в рамках каждой из этих форм деятельности в области \textit{Искусственного интеллекта}, так и между ними. Таким образом, основной тенденцией дальнейшего развития теоретических и практических работ в области \textit{Искусственного интеллекта} является конвергенция как самых разных видов (форм и направлений) человеческой деятельности в области \textit{Искусственного интеллекта}, так и самых разных продуктов (результатов) этой деятельности. Необходимо ликвидировать барьеры между различными видами и продуктами деятельности в области \textit{Искусственного интеллекта} в целях обеспечения их совместимости и интегрируемости.}
    \scntext{проблема текущего состояния}{Проблема создания быстро развивающегося рынка семантически совместимых интеллектуальных систем  это вызов, адресованный специалистам в области \textit{Искусственного интеллекта}, требующий преодоления вавилонского столпотворения во всех его проявлениях, формирование высокой культуры договороспособности и унифицированной, согласованной формы представления коллективно накапливаемых, совершенствуемых и используемых знаний. Ученые, работающие в области \textit{Искусственного интеллекта}, должны обеспечить конвергенцию результатов различных направлений \textit{Искусственного интеллекта} и построить:
        \begin{itemize}
            \item общую теорию интеллектуальных компьютерных систем;
            \item общую технологию проектирования семантически совместимых интеллектуальных компьютерных систем, включающую соответствующие стандарты интеллектуальных компьютерных систем и их компонентов. Инженеры, разрабатывающие интеллектуальные компьютерные системы, должны сотрудничать с учеными и участвовать в развитии технологии проектирования интеллектуальных компьютерных систем.
        \end{itemize}}
    
    \scnheader{конвергенция в области Искусственного интеллекта}
    \scnrelfrom{разбиение}{Направления конвергенции в области Искусственного интеллекта}
    \begin{scnindent}
	    \scnhaselement{конвергенция Искусственного интеллекта со смежными научными дисциплинами}
	    \begin{scnindent}
            \begin{scnrelfromset}{примечание}
                \scnitem{Искусственный интеллект}
                \begin{scnindent}
                    \begin{scnrelbothlist}{смежная дисциплина}
                    \scnitem{Логика}
                    \scnitem{Психология человека}
                    \scnitem{Зоопсихология}
                    \scnitem{Нейропсихология}
                    \scnitem{Этология}
                    \scnitem{Кибернетика}
                    \scnitem{Общая теория систем}
                    \scnitem{Семиотика}
                    \scnitem{Лингвистика}
                    \end{scnrelbothlist}
                \end{scnindent}
            \end{scnrelfromset}
        \end{scnindent}
	    \scnhaselement{конвергенция различных направлений Искусственного интеллекта}
	    \begin{scnindent}
		    \scnidtf{Конвергенция различных направлений исследований в области Искусственного интеллекта, результатом которой должна быть формализованная практически ориентированная общая теория интеллектуальных систем и, в частности, интеллектуальных компьютерных систем}
		    \scnidtf{Конвергенция между различными направлениями и продуктами научных исследований в области искусственного интеллекта, результатом (целевым продуктом) которой должна стать общая формальная теория интеллектуальных компьютерных систем}
		    \scntext{примечание}{Разобщенность различных направлений исследований в области искусственного интеллекта является главным препятствием создания общей комплексной технологии проектирования интеллектуальных компьютерных систем}
	    \end{scnindent}
	    \scnhaselement{конвергенция различного вида знаний в памяти интеллектуальной компьютерной системы}
	    \begin{scnindent}
	    	\scnidtf{Конвергенция и интеграция внутреннего представления в памяти интеллектуальной компьютерной системы различного вида знаний}
	    \end{scnindent}
	    \scnhaselement{конвергенция различных моделей решения задач в памяти интеллектуальной компьютерной системы}
	    \begin{scnindent}
	    	\scnidtf{Конвергенция и интеграция различных моделей решения задач, которая включает логико-семантическую типологию задач и типологию моделей решения задач и требует уточнения семантики таких понятий как задача, класс задач, метод, класс методов, модель решения задач (иерархический метод интерпретации класса методов)}
	    \end{scnindent}
	    \scnhaselement{конвергенция интеллектуальных компьютерных систем}
	    \begin{scnindent}
		    \scnidtf{Обеспечение семантической совместимости (взаимопонимания) интеллектуальных систем, согласование используемых онтологий}
		    \scnidtf{Конвергенция между различными прикладными компьютерными системами, результатом (целевым продуктом) которой должна стать экосистема, состоящая из перманентно эволюционирующих, семантически совместимых и взаимодействующих интеллектуальных компьютерных систем, а также их пользователей}
		    \scntext{пояснение}{Конвергенция (семантическая совместимость) всех разрабатываемых интеллектуальных компьютерных систем (в том числе прикладных), преобразующая набор индивидуальных (самостоятельных) интеллектуальных компьютерных систем различного назначения в коллектив активно взаимодействущих интеллектуальных компьютерных систем для совместного (коллективного) решения сложных (комплексных) задач и для перманентной поддержки семантической совместимости в ходе индивидуальной эволюции каждой интеллектуальной компьютерной системы.}
	    \end{scnindent}
	    \scnhaselement{конвергенция средств автоматизации проектирования различного вида компонентов интеллектуальных компьютерных систем}
	    \begin{scnindent}
		    \scnidtf{Конвергенция (семантическая совместимость) средств автоматизации проектирования различного вида компонентов интеллектуальных компьютерных систем, результатом которой должен быть общий комплекс средств автоматизации проектирования всех компонентов интеллектуальных компьютерных систем}
		    \scnidtf{Конвергенция между инструментальными средствами, обеспечивающими автоматизацию проектирования различных компонентов или различных классов интеллектуальных компьютерных систем, результатом (целевым продуктом) которой должен стать единый комплекс методологических и инструментальных средств, ориентированный на поддержку комплексного проектирования любых интеллектуальных компьютерных систем}
	    \end{scnindent}
	    \scnhaselement{конвергенция логико-семантических моделей интеллектуальных компьютерных систем}
	    \begin{scnindent}
	    	\scntext{примечание}{\textit{логико-семантические модели интеллектуальных компьютерных систем} являются результатом (сухим остатком) \textit{проектирования} этих систем и представляют собой формальное представления исходного (начального) состояния \textit{баз знаний} разрабатываемых \textit{интеллектуальных компьютерных систем}}
	    \end{scnindent}
	    \scnhaselement{конвергенция средств интерпретации логико-семантических моделей разрабатываемых интеллектуальных компьютерных систем}
	    \begin{scnindent}
	    	\scntext{пояснение}{Конвергенция (совместимость) средств реализации (производства) интеллектуальных компьютерных систем на основе спроектированных формальных моделей создаваемых интеллектуальных компьютерных систем (средств интерпретации спроектированных моделей интеллектуальных компьютерных систем). Такая интерпретация может осуществляться либо программным путем на современных компьютерах, либо путем создания принципиально новых компьютеров, специально ориентированных на интерпретацию формальных моделей интеллектуальных компьютерных систем, помещаемых в память указанных компьютеров}
	    \end{scnindent}
	    \scnhaselement{конвергенция между информационно-программным и аппаратным обеспечением интеллектуальных компьютерных систем}
	    \begin{scnindent}
	    	\scnidtf{Конвергенция между Software и Hardware интеллектуальных компьютерных систем}
	    \end{scnindent}
	    \scnhaselement{Конвергенция различных форм деятельности в области Искусственного интеллекта}
	    \begin{scnindent}
	    \scnidtftext{пояснение}{Конвергенция между:
	        \begin{itemize}
	            \item научными исследованиями по созданию общей теории интеллектуальных компьютерных систем;
	            \item разработкой средств автоматизации проектирования интеллектуальных компьютерных систем;
	            \item разработкой средств интерпретации спроектированных формальных моделей интеллектуальных компьютерных систем;
	            \item разработкой прикладных интеллектуальных компьютерных систем различного назначения;
	            \item подготовкой и перманентным повышением квалификации кадров, способных эффективно участвовать во всех перечисленных направлениях деятельности.
	        \end{itemize}}
	    \scnidtf{Конвергенция между:
	        \begin{itemize}
	            \item научно-исследовательской деятельностью в области искусственного интеллекта;
	            \item инженерно-технологической деятельностью, которая направлена на разработку комплексной технологии проектирования интеллектуальных компьютерных систем и которая имеет высокий уровень наукоемкости;
	            \item инженерно-прикладной деятельностью, которая направлена на разработку прикладных интеллектуальных систем и которая также имеет высокий уровень наукоемкости, обусловленной необходимостью качественной формализации соответствующих предметных областей и, в частности, методов решения задач в этих областях;
	            \item образованием (образовательной деятельностью) в области искусственного интеллекта, повышение эффективности которого настоятельно требует раннего и поэтапного вовлечения студентов в реальные, а не учебные проекты --- сначала в инженерно-прикладные, потом в инженерно- исследовательские проекты;
	            \item деятельностью, направленной на создание инфраструктуры, обеспечивающей поддержку открытого массового активного международного сотрудничества по консолидации усилий, направленных на решение современных проблем в области искусственного интеллекта;
	            \item бизнесом в области искусственного интеллекта, который не просто должен обеспечить финансовую поддержку перечисленных видов деятельности, но и обеспечить грамотный баланс между ними, грамотное сочетание тактических и стратегических целей
	        \end{itemize}}
	    \scntext{примечание}{Глубокая конвергенция между всеми этими формами деятельности возможна только тогда, когда \uline{каждый} участник создания комплексной технологии искусственного интеллекта является участником \uline{каждой} из перечисленных форм деятельности.}
	    \end{scnindent}
    \end{scnindent}
    	
    \scnheader{Искусственный интеллект}
    \begin{scnrelfromset}{\scnkeyword{методологические проблемы текущего состояния}}
    	\scnfileitem{Отсутствие массового осознания того, что создание рынка \textit{интеллектуальных компьютерных систем нового поколения}, обладающих \textit{семантической совместимостью} и высоким уровнем \textit{интероперабельности}, а также создание комплексов (экосистем), состоящих из таких \textit{интеллектуальных компьютерных систем} и обеспечивающих автоматизацию различных \textit{видов человеческой деятельности}, \uline{невозможно}, если коллективы разработчиков таких систем и комплексов существенно не повысят уровень \textit{социализации} \textbf{\uline{всех}} своих сотрудников. Уровень качества коллектива разработчиков, то есть уровень квалификации сотрудников и уровень согласованности их деятельности, должен превышать уровень качества систем, разрабатываемых этим коллективом. Особое значение рассматриваемая проблема согласованности деятельности специалистов в области \textit{Искусственного интеллекта} имеет для построения \textit{Общей формальной теории интеллектуальных компьютерных систем нового поколения}, а также \textit{Комплексной технологии разработки и эксплуатации интеллектуальных компьютерных систем нового поколения}.}
        \scnfileitem{Далеко не всеми учеными, работающими в области искусственного интеллекта принимается прагматичность практической направленности этой науки}
        \scnfileitem{Не всеми принимается необходимость конвергенции различных направлений искусственного интеллекта и необходимость их интеграции в целях построения общей теории интеллектуальных систем}
        \scnfileitem{Не всеми принимается необходимость \textbf{\textit{конвергенции}} различных видов деятельности в области \textit{Искусственного интеллекта}}
        \scnfileitem{Важным препятствием для \textbf{\textit{конвергенции}} результатов научно-технической деятельности является сформировавшийся в науке и технике акцент на выявлении не сходств, а отличий. Чтобы убедиться в этом достаточно обратить внимание на то, что уровень научных результатов оценивается научной \uline{новизной}, которая может имитироваться новизной не по существу, а по форме представления (например, с помощью новых понятий или даже новых терминов). Результаты в технике, например, в патентах также оцениваются \uline{отличиями} от предшествующих технических решений. Но для \textbf{\textit{конвергенции}} нужны другие акценты --- не поиск отличий, а выявление неочевидных сходств и превращение их в очевидные сходства, представленные в одинаковой \uline{форме}}
        \scnfileitem{Нет движения к построению \textit{Комплексной технологии проектирования, реализации, сопровождения, реинжиниринга и эксплуатации интеллектуальных компьютерных систем}. Речь идет о комплексном подходе к технологическому обеспечению \uline{всех этапов} \uline{жизненного цикла} \textit{интеллектуальных компьютерных систем}.}
        \scnfileitem{Нет движения к построению общей компьютерной технологии интеллектуальных компьютерных систем.}
        \scnfileitem{Нет движения к построению экосистем интеллектуальных компьютерных систем.}
        \scnfileitem{Не всеми принимается необходимость конвергенции различных форм деятельности в области Искусственного интеллекта.}
        \scnfileitem{Нет активного развития работ по созданию \textit{Глобальной} \textit{экосистемы интеллектуальных компьютерных систем нового поколения}.}
        \scnfileitem{В основе современной организации и автоматизации \textit{человеческой деятельности} лежит \scnqqi{Вавилонское столпотворение} постоянно расширяемого многообразия \textit{языков.} Имеются в виду не только \textit{естественные} \textit{языки}, но и \textit{формальные} \textit{язык}и, направленные на точное представление \textit{знаний} различного вида. Многообразие различных \textit{специализированных} \textit{языков} пронизывает всю \textit{человеческую деятельность} --- во многих областях \textit{человеческой деятельности} для решения различных видов \textit{задач}, для разработки различных \textit{моделей решения задач} создаются \textit{специализированные языки}. Примером этого является многообразие \textit{языков программирования}. \textit{Специализированные языки} могут и должны появляться, но только как \textit{\textit{подъязыки}} более общих \textit{языков}, синтаксис каждого из которых совпадает с \textit{синтаксисом} всех соответствующих ему \textit{подъязыков}. При этом в рамках \textit{Общей формальной теории интеллектуальных компьютерных систем} должен быть выделен один \textit{универсальный формальный} \textit{язык} --- язык-ядро, по отношению к которому все остальные используемые \textit{формальные языки} являются \textit{подъязыками}. \textit{денотационная семантика} указанного \textit{универсального формального языка} должна задаваться соответствующей \textit{формальной онтологией} максимально высокого уровня. Иначе о какой \textbf{\textit{конвергенции}} и \textit{интеграции} \textit{знаний}, о какой \textit{семантической совместимости} компьютерных систем можно вести речь.}
    \end{scnrelfromset}
    \scntext{примечание}{Современная трактовка целей и задач \textit{Искусственного интеллекта} как научно-технической дисциплины требует переосмысления, так как, к сожалению, носит несогласованный, а часто и значительно более узкий характер, чем этого требует текущее положение.}
    \scntext{ключевой фактор решения}{Различные направления \textit{конвергенции} и \textit{интеграции}, обеспечивающие переход к \textit{интеллектуальным компьютерным системам нового поколения}, к соответствующей технологии комплексной поддержки их жизненного цикла и к существенному повышению уровня автоматизации всего комплекса человеческой деятельности.}
    \begin{scnindent}
        \begin{scnrelfromset}{включение}
            \scnfileitem{\textit{конвергенция} и \textit{интеграция} различных моделей представления и обработки \textit{информации} в \textit{интеллектуальных компьютерных системах нового поколения}.}
            \begin{scnindent}
            	\begin{scnrelfromset}{включение}
	                \scnfileitem{\textit{конвергенция} и \textit{интеграция} различных \textit{видов} \textit{знаний} в \textit{базах знаний} \textit{интеллектуальных компьютерных систем нового поколения}.}
	                \scnfileitem{\textit{конвергенция} и \textit{интеграция} различных \textit{моделей решения задач}.}
	                \scnfileitem{\textit{конвергенция} и \textit{интеграция} различных \textit{видов интерфейсов} \textit{интеллектуальных компьютерных систем нового поколения}.}
                \end{scnrelfromset}
            \end{scnindent}
            \scnfileitem{\textit{конвергенция} и \textit{интеграция} различных направлений \textit{Искусственного интеллекта} в целях построения \textit{Общей формальной теории интеллектуальных компьютерных систем нового поколения}}
            \scnfileitem{\textit{конвергенция} и \textit{интеграция} технологий \textit{проектирования} различных \textit{компонентов интеллектуальных компьютерных систем нового поколения} в целях построения комплексной \textit{Технологии проектирования интеллектуальных компьютерных систем нового поколения}.}
            \scnfileitem{\textit{конвергенция} и \textit{интеграция} технологий поддержки различных \textit{этапов жизненного цикла} \textit{интеллектуальных компьютерных систем нового поколения} в целях построения \textit{Технологии комплексной поддержки всех этапов жизненного цикла интеллектуальных компьютерных систем нового поколения}.}
            \scnfileitem{\textit{конвергенция} и \textit{интеграция} различных \textit{видов человеческой деятельности в области Искусственного интеллекта} (\textit{научно-исследовательской деятельности}, \textit{развития технологического комплекса}, \textit{прикладной инженерии}, \textit{образовательной деятельности}) для повышения уровня согласованности и координации этих \textit{видов деятельности}, а также для повышения уровня их комплексной автоматизации с помощью\textit{семантически совместимых} \textit{интеллектуальных компьютерных систем нового поколения}.}
            \scnfileitem{\textit{конвергенция} и \textit{интеграция} самых различных \textit{видов и областей человеческой деятельности}, а также средств комплексной автоматизации этой деятельности с помощью \textit{интеллектуальных компьютерных систем нового поколения}.}
        \end{scnrelfromset}
    \end{scnindent}

    \scnheader{Человеческая деятельность в области Искусственного интеллекта}
    \begin{scnrelfromset}{принципы, лежащие в основе}
    	\scnfileitem{\textit{комплексная конвергенция} --- как \scnqq{вертикальная} \textit{конвергенция} между различными \textit{видами деятельности} в области \textit{Искусственного интеллекта}, так и \scnqqi{горизонтальная} \textit{конвергенция} в рамках каждого из этих \textit{видов деятельности}, соответствующая различным компонентам или различным классам \textit{интеллектуальных компьютерных систем} --- базам знаний, решателям задач, различным моделям решения задач, различным видам интерфейсов (зрительным, аудио, естественно-языковым), робототехническим интеллектуальным компьютерным системам, интеллектуальным обучающим системам, интеллектуальным автоматизированным системам управления, интеллектуальным системам автоматизации проектирования и так далее.}
    	\scnfileitem{\textit{\scnqq{горизонтальная} конвергенция} в рамках каждого вида \textit{человеческой деятельности} в области \textit{Искусственного интеллекта}.}
    	\begin{scnindent}
            \begin{scnrelfromset}{включение}
                \scnfileitem{\textit{конвергенция} в рамках \textit{научно-исследовательской деятельности в области Искусственного интеллекта}, означающую переход от независимого развития различных направлений \textit{Искусственного интеллекта} к общей теории \textit{интеллектуальных компьютерных систем}.}
                \scnfileitem{\textit{конвергенция} в рамках развития \textit{технологий Искусственного интеллекта}, означающую переход от независимого развития частных технологий к созданию единого комплекса семантически совместимых частных технологий.}
                \scnfileitem{\textit{конвергенция} в рамках \textit{инженерной деятельности в области Искусственного интеллекта}, означающую переход от практики независимой разработки различных прикладных \textit{интеллектуальных компьютерных систем} к разработке комплекса (экосистемы) интероперабельных \textit{интеллектуальных компьютерных систем}.}
                \scnfileitem{\textit{конвергенция} в рамках \textit{учебной деятельности в области Искусственного интеллекта}, обозначающую переход от изучения отдельных учебных дисциплин к формированию у молодых специалистов целостной картины текущего состояния \textit{Искусственного интеллекта} и проблемных направлений дальнейшего развития.}
                \scnfileitem{\textit{конвергенция} в рамках \textit{общей организационной деятельности в области Искусственного интеллекта}, переход от отдельных вышеперечисленных видов деятельности в области \textit{Искусственного интеллекта} к единому комплексу всех этих видов деятельности и обеспечивающую конвергенцию и интеграцию указанных видов деятельности в области \textit{Искусственного интеллекта}, что существенно повысит их качество, поскольку каждый из этих видов деятельности находится в сильной зависимости от всех остальных.}
            \end{scnrelfromset}
        \end{scnindent}
    	\scnfileitem{Организация разработки и перманентного развития предлагаемой \textit{технологии} в виде \textbf{\textit{открытого международного проекта}}.}
    	\begin{scnindent}
            \begin{scnrelfromset}{включение}
                \scnfileitem{Свободный доступ к использованию текущей версии разрабатываемой \textit{технологии}.}
                \scnfileitem{Возможность каждому желающему войти в состав коллектива разработчиков этой \textit{технологии}.}
            \end{scnrelfromset}
    	\end{scnindent}
    	\scnfileitem{\textit{Поэтапность} процесса формирования рынка \textit{семантически совместимых} и \textit{активно взаимодействующих} между собой \textit{интеллектуальных компьютерных систем нового} \textit{поколения}.}
    	\begin{scnindent}
            \begin{scnrelfromvector}{включение}
                \scnfileitem{Разработка \textit{логико-семантических моделей} (баз знаний) нескольких \textit{прикладных интеллектуальных компьютерных систем нового поколения}.}
                \scnfileitem{Программная реализация на современных \textit{ostis-платформах}.}
                \scnfileitem{Установка каждой разработанной \textit{логико-семантической модели прикладной интеллектуальной компьютерной системы} на \textit{ostis-платформу} с последующим \textit{тестированием} и \textit{реинжинирингом} каждой такой модели.}
                \scnfileitem{Разработка и перманентное совершенствование логико-семантической модели (базы знаний) \textit{интеллектуальной компьютерной метасистемы}, которая содержит (1) описание \textit{стандарта интеллектуальных компьютерных систем нового поколения}, (2) \textit{библиотеку} многократно используемых (в различных \textit{интеллектуальных компьютерных системах}) знаний различного вида и, в частности, различных \textit{методов решения задач}, (3) \textit{методы проектирования} и \textit{средства поддержки проектирования} \textit{различных видов компонентов интеллектуальных компьютерных систем} (компонентов \textit{баз знаний, решателей задач, интерфейсов}).}
                \scnfileitem{Разработка \textit{ассоциативного семантического компьютера} в качестве аппаратной реализации \textit{платформы интерпретации логико-семантических моделей интеллектуальных компьютерных систем нового поколения}.}
                \scnfileitem{Перенос разработанных \textit{логико-семантических моделей интеллектуальных компьютерных систем нового поколения} на новые, более эффективные варианты реализации платформы интерпретации этих моделей.}
                \scnfileitem{Развитие \textit{рынка интеллектуальных компьютерных систем нового поколения} в виде Глобальной экосистемы, состоящей из активно взаимодействующих таких систем и ориентированной на комплексную автоматизацию всех \textit{видов} \textit{человеческой деятельности}.}
                \scnfileitem{Создание \textbf{\textit{рынка знаний}} на основе \textit{Экосистемы OSTIS}.}
                \scnfileitem{Автоматизация \textit{реинжиниринга} эксплуатируемых \textit{интеллектуальных компьютерных систем нового поколения} в направлении приведения их в соответствие с новыми версиями \textit{стандарта интеллектуальных компьютерных} \textit{систем} путем автоматической замены устаревших \textit{компонентов} в этих системах на текущие версии этих компонентов.}
            \end{scnrelfromvector}
        \end{scnindent}
    \end{scnrelfromset}
    
    \scnheader{следует отличать*}
    \begin{scnhaselementset}
        \scnitem{конвергенция}
	        \begin{scnindent}
	        	\scnidtf{Процесс сближения структурных и/или функциональных характеристик нескольких (как минимум двух) заданных сущностей}
	            \scnidtf{Процесс конвергенции заданных сущностей в ходе их изменения, совершенствование, эволюции}
	            \scnsubset{процесс}
	         \end{scnindent}
        \scnitem{конвергенция\scnsupergroupsign}
        	\begin{scnindent}
	            \scnidtf{Степень близости (сходство) заданных сущностей}
	            \scniselement{свойство}
        	\end{scnindent}
    \end{scnhaselementset}
    
    \scnheader{конвергенция}
    \scntext{примечание}{\textit{Конвергенция} пар конкретных искусственных сущностей (например, технических систем) есть стремление их унификацию (в частности, к стандартизации), т.е. стремление к минимизации многообразия форм решения аналогичных практических задач --- стремление к тому, чтобы все, что можно сделать одинаково, сделалось одинаково, но без ущерба требуемого качества. Последнее очень важно, так как безграмотная стандартизация может привести к существенному торможению прогресса. Ограничение многообразия форм не должно приводить к ограничению содержания, возможностей. Образно говоря, \scnqqi{словам должно быть тесно, а мыслям --- свободно}.}
    \scntext{примечание}{Методологически конвергенция искусственно создаваемых сущностей (артефактов) сводится (1) к выявлению (обнаружению) принципиальных сходств между этими сущностями, которые часто весьма закамуфлированы и их трудно увидеть, и (2) к реализации обнаруженных сходств одинаковым образом (в одинаковой форме, в одинаковом синтаксисе). Образно говоря, от семантической (смысловой) эквивалентности требуется перейти и к синтаксической эквивалентности. Кстати, в этом как раз и заключается суть (идея) смыслового представления информации (знаний), целью которого является создание такой языковой среды (\textit{смыслового пространства}), в рамках которого (1) семантически эквивалентные информационные конструкции полностью совпадали, а (2) конвергенция информационных конструкций сводилась бы к выявлению изоморфных фрагментов этих конструкций.}
    \scntext{примечание}{Очень важно уточнить, формализовать понятие конвергенции (конвергенции знаний, методов, модели решения задач, конвергенции интеллектуальных компьютерных систем в целом)}
    \scnsuperset{конвергенция информационных конструкций}
	    \begin{scnindent}
	    	\scnidtf{конвергенция синтаксических и семантических свойств информационных конструкций}
	    \end{scnindent}
    \scnsuperset{конвергенция языков}
    \scnsuperset{конвергенция научных дисциплин}
    	\begin{scnindent}
    		\scnidtf{конвергенция различных научных дисциплин или различных направлений одной и той же и дисциплины}
    	\end{scnindent}
    \scnsuperset{конвергенция баз знаний}
    \scnsuperset{конвергенция моделей решения задач}
    \scnsuperset{конвергенция гибридных решателей задач}
    \scnsuperset{конвергенция кибернетических систем}
    \scnsuperset{конвергенция интеллектуальных систем}
    	\begin{scnindent}
    		\scnsuperset{конвергенция интеллектуальных систем, направленная на обеспечение их \uline{семантической совместимости}}
    	\end{scnindent}
    
    \scnheader{конвергенция результатов научно-технической деятельности}
    \scntext{примечание}{Важным препятствием для конвергенции результатов научно-технической деятельности является сформировавшийся в науке и технике акцент на выявлении не сходств, а отличий. Чтобы убедиться в этом достаточно обратить внимание на то, что уровень научных результатов оценивается научной \uline{новизной}, которая может имитироваться новизной не по существу, а по форме представления (например, с помощью новых понятий или даже новых терминов). Результаты в технике, например, в патентах также оцениваются \uline{отличиями} от предшествующих технических решений. Но для конвергенции нужны другие акценты --- ни поиск отличий, а выявление неочевидных сходств и превращения их в очевидные сходства, представленные в одинаковой \uline{форме}.}
    
    \scnheader{совместимость\scnsupergroupsign}
    \scnidtf{совместимость заданных двух или более сущностей\scnsupergroupsign}
    \scnidtf{простота интеграции заданной группы сущностей\scnsupergroupsign}
    \scnidtf{интегрируемость\scnsupergroupsign}
    \scntext{примечание}{Степень (уровень) совместимости заданных сущностей может рассматриваться как оценка результата их конвергенции. Чем качественнее (основательнее, глубже) проведена конвергенция заданных сущностей, тем выше уровень их совместимости и, собственно, тем легче их интегрировать.}
    \scnsuperset{cовместимость информационных конструкций\scnsupergroupsign}
    	\begin{scnindent}
    		\scnsuperset{семантическая совместимость информационных конструкций\scnsupergroupsign}
    	\end{scnindent}
    \scnsuperset{совместимость языков\scnsupergroupsign}
    	\begin{scnindent}
    		\scnsuperset{семантическая совместимость языков\scnsupergroupsign}
    	\end{scnindent}
    \scnsuperset{семантическая совместимость научных дисциплин\scnsupergroupsign}
    \scnsuperset{совместимость баз знаний\scnsupergroupsign}
    \scnsuperset{совместимость моделей решения задач\scnsupergroupsign}
    \scnsuperset{совместимость кибернетических систем\scnsupergroupsign}
    	\begin{scnindent}
    		\scnsuperset{семантическая совместимость кибернетических систем\scnsupergroupsign}
    	\end{scnindent}
    \scnsuperset{семантическая совместимость\scnsupergroupsign}
    
    \scnheader{интеграция*}
    \scnidtf{объединение нескольких разных сущностей, в результате чего возникает некоторая объединённая целостная сущность*}
    \scnsuperset{эклектичная интеграция*}
    	\begin{scnindent}
    		\scnidtf{Интеграция разнородных (гетерогенных) сущностей, которой не предшествует конвергенция (сближение) этих сущностей*}
    	\end{scnindent}
    \scnsuperset{глубокая интеграция*}
    \scntext{примечание}{Понятие \textit{интеграции*} и особенно понятие \textit{глубокой интеграции*} имеет тесную связь с понятием \textit{конвергенции\scnsupergroupsign}. Чем выше степень конвергенции (степень сближения) интегрируемых объектов, тем выше качество результата интеграции. Особенно, если речь идёт о глубокой интеграции.}
    
    \scnheader{глубокая интеграция*}
    \scnidtf{бесшовная интеграция*}
    %TODO ссылка на Грибову
    \scnidtf{интеграция однородных сущностей, предполагающая глубокую взаимную диффузию (сращивание) соединяемых сущностей, которая не обязательно должна осуществляться физически}
    \scntext{примечание}{Примером виртуальной глубокой интеграции является формирование коллектива \uline{семантический совместимых} индивидуальный кибернетических систем}
    \scnidtf{бесшовная интеграция*}
    \scnidtf{гибридизация*}
    \scnidtf{интеграция, результатом которой являются гибридные объекты*}
    \scnidtf{интеграция, которой предшествует высокий уровень конвергенции интегрируемых объектов*}
    \scnidtf{(конвергенция + интеграция)*}
    \scnidtf{бесшовная интеграция}
    \scnidtf{интеграция, в результате которой возникает гибридная система*}
    \scnidtf{интеграция, которой предшествует конвергенция (в частности, унификация) интегрируемых систем, приведение этих систем к максимально похожему виду (общему знаменателю)*}
    %TODO сложно при чтении воспринимать, конвергенция и приведение как-то сливаются, становится не совсем понятно, к чему относится приведение к конвергенции или к интеграции, может как-то более явно указать, что конвергенция это то приведение?
    \scnidtf{интеграция с диффузией , взаимопроникновением на основе унификации того, что можно сделать одинаковым*}
    
    \scnheader{интеграция*}
    \scnsuperset{интеграция информационных конструкций}
    \scnsuperset{интеграция языков}
    \scnsuperset{интеграция научных дисциплин}
    \scnsuperset{интеграция баз знаний}
    \scnsuperset{интеграция моделей решения задач}
    \scnsuperset{интеграции индивидуальных кибернетических систем}
    \begin{scnindent}
    	\scnsuperset{слияние индивидуальных кибернетических систем}
    		\begin{scnindent}
    			\scnidtf{преобразование нескольких \uline{искусственных} индивидуальных кибернетических систем в интегрированную индивидуальную кибернетическую систему, которая способна решать все задачи, каждая из которых могла бы быть решена в рамках какой-либо из интегрируемых систем}
    		\end{scnindent}
    	\scnsuperset{формирование коллектива индивидуальных кибернетических систем}
    		\begin{scnindent}
    			\scnidtf{формирования многоагентной системы, состоящей из индивидуальных кибернетических систем}
    		\end{scnindent}
    	\scntext{примечание}{Эффективность интеграции индивидуальных кибернетических систем определяется тем, насколько объем задач, решаемых коллективом индивидуальных кибернетических систем, превысит объединение объёмов задач, решаемых членами коллектива в отдельности.}	
    \end{scnindent}
    
\bigskip
\end{scnsubstruct}
\scnendcurrentsectioncomment

        \scnsegmentheader{Уточнение Понятия Технологии OSTIS}
\begin{scnsubstruct}
    \scnheader{Технология OSTIS}
    \scnidtf{Комплекс (семейство) технологий, обеспечивающих проектирование, производство, эксплуатацию и реинжиниринг интеллектуальных \textit{компьютерных систем} (\textit{ostis-систем}), предназначенных для автоматизации самых различных видов человеческой деятельности и в основе которых лежит смысловое представление и онтологическая систематизация знаний, а также агентно-ориентированная обработка знаний}
    \scnidtf{Open Semantic Technology for Intelligent Systems}
	    \begin{scnindent}
	    	\scntext{сокращение}{OSTIS}
	    \end{scnindent}
    \scnidtf{Семейство (комплекс) \textit{ostis-технологий}}
    \scnidtf{Комплексная открытая семантическая технология проектирования, производства, эксплуатации и реинжиниринга гибридных, семантически совместимых, активных и договороспособных \textit{интеллектуальных компьютерных систем}}
    \begin{scnrelfromset}{принципы, лежащие в основе}
        \scnfileitem{Ориентация на разработку \textit{интеллектуальных компьютерных систем}, имеющих высокий уровень \textit{интеллекта} и, в частности, высокий уровень \textit{социализации}. Указанные системы, разработанные по \textit{Технологии OSTIS}, будем называть \textbf{\textit{ostis-системами}}}
        \scnfileitem{Ориентация на \uline{комплексную} автоматизацию всех видов и областей \textit{человеческой деятельности} путем создания сети взаимодействующих и координирующих свою деятельность \textit{ostis-систем}. Указанную сеть \textit{ostis-систем} вместе с их пользователями будем называть \textbf{\textit{Экосистемой OSTIS}}}
        \scnfileitem{Поддержка перманентной эволюции \textit{ostis-систем} в ходе их эксплуатации.}
        \scnfileitem{\textit{Технология OSTIS} реализуется в виде сети \textit{ostis-систем}, которая является частью \textit{Экосистемы OSTIS}.Ключевой \textit{ostis-системой} указанной сети является \textbf{\textit{Метасистема OSTIS}} (Intelligent MetaSystem for ostis-systems), реализующая \textbf{\textit{Ядро Технологии OSTIS}}, которое включает в себя базовые (предметно независимые) методы и средства проектирования и производства \textit{ostis-систем} с интеграцией в их состав типовых встроенных подсистем поддержки эксплуатации и реинжиниринга \textit{ostis-систем}. Остальные \textit{ostis-системы}, входящие в состав рассматриваемой сети, реализуют различные специализированные \textit{ostis-технологии} проектирования различных классов \textit{ostis-систем}, обеспечивающих автоматизацию любых областей и \textit{видов человеческой деятельности}, кроме \textit{проектирования ostis-систем}.}
        \scnfileitem{Конвергенция и интеграция на основе \textit{смыслового представления знаний} всевозможных научных направлений \textit{Искусственного интеллекта} (в частности, всевозможных базовых знаний и навыков решения \textit{интеллектуальных задач}) в рамках \textit{Общей формальной семантической теории \mbox{ostis-систем}}.}
        \scnfileitem{Ориентация на разработку компьютеров нового поколения, обеспечивающих эффективную (в том числе производительную) интерпретацию логико-семантических моделей \textit{ostis-систем}, которые представлены \textit{базами знаний} этих систем, имеющими \textit{смысловое представление}.}
    \end{scnrelfromset}
\end{scnsubstruct}

\scnheader{Понятие ostis-системы}
\begin{scnsubstruct}
    \scnheader{ostis-система}
    \scnidtf{\textit{интеллектуальная компьютерная система}, спроектированная и реализованная по требованиям и стандартам \textit{Технологии OSTIS}, которые задокументированы в \textit{Общей теории ostis-систем}}
    \scnidtf{Множество \textit{ostis-систем} различного назначения}
    \begin{scnindent}
        \scniselement{имя собственное}	
    \end{scnindent}
    \scnidtf{Множество всевозможных \textit{интеллектуальных компьютерных систем}, построенных по \textit{Технологии OSTIS}}
    \scnsubset{интеллектуальная компьютерная система}
    \scnidtf{\textit{интеллектуальная компьютерная система}, которая построена в соответствии с требованиями и стандартами \textit{Технологии OSTIS}, что обеспечивает существенное развитие целого ряда \textit{свойств} (способностей) этой \textit{компьютерной системы}, позволяющих значительно повысить \textit{уровень интеллекта} этой системы (и, прежде всего, ее \textit{уровень обучаемости} и \textit{уровень социализации})}
    \begin{scnsubdividing}
        \scnitem{индивидуальная ostis-система}
        \scnitem{коллективная ostis-система}
        	\begin{scnindent}
	            \begin{scnsubdividing}
	                \scnitem{простой коллектив ostis-систем}
	                \scnitem{иерархический коллектив ostis-систем}
	            \end{scnsubdividing}
            \end{scnindent}
    \end{scnsubdividing}
    \scntext{примечание}{Когда речь идет о таком компоненте \textit{Технологии OSTIS}, как \textit{Общая теория ostis-систем}, имеется в виду строгое формальное уточнение того, как устроена \textit{ostis-система}, какова ее архитектура, принципы организации памяти, принципы организации представления и обработки информации, принципы организации интерфейса с внешней средой (в том числе, с пользователями)}
    \begin{scnrelfromset}{принципы, лежащие в основе}
        \scnfileitem{Хранение информации в памяти \textit{ostis-системы} ориентируется на \textit{\uline{смысловое} представление информации} --- без синонимии и омонимии знаков, без семантической эквивалентности информационных конструкций, т.е. без дублирования информации.}
        \begin{scnindent}
	        \scnrelfrom{ключевой знак}{\scnkeyword{смысловое представление информации}}
            \begin{scnindent}
                \scnidtf{смысл представленной информационной конструкции}
                \begin{scnrelfromvector}{принципы, лежащие в основе}
                    \scnfileitem{В рамках смыслового представления информационной конструкции все \textit{знаки}, входящие в эту \textit{информационную конструкцию} уникальны, т.е. обозначают \uline{разные} описываемые \textit{сущности}. Другими словами, в рамках \textit{смыслового представления информационной конструкции} запрещено присутствие \textit{синонимичных знаков}.}
                    \scnfileitem{В рамках \textit{смыслового представления информационной конструкции} \uline{все} \textit{сущности}, описываемые этой \textit{информационной конструкцией}, должны быть \uline{явно} представлены своим \textit{знаком}.}
                    \scnfileitem{Каждый \textit{знак}, входящий в \textit{смысловое представление информационной конструкции} является \textit{синтаксически элементарным} (атомарным) \textit{фрагментом} этой конструкции, внутреняя структура которого несущественна (существенен только алфавит таких фрагментов).}
                    \scnfileitem{Поскольку любая описываемая \textit{сущность} может быть связана неограниченным числом \textit{связей} с другими \textit{сущностями} (при этом указанные связи также считаются описываемыми сущностями), \textit{смысловое представление инофрмационной конструкции} является \textit{графоподобной конструкцией}}
                    \scnfileitem{Интеграция (объединение, соединение) \textit{информационной конструкции}, представленных в смысловой форме сводится к \textit{склеиванию} (отождествлению) \textit{синонимичных знаков}.}
                    \scnfileitem{Смысл представленной информации содержится не в самих \textit{знаках}, а в конфигурации \textit{связей} между ними, которая отражает (является информационной моделью) описываемой конфигурации связей между описываемыми \textit{сущностями}. Суть смыслового представления информационной конструкции заключается в том, что конфигурация \textit{связей} между \textit{знаками}, входящими в эту \textit{информационную конструкцию}, становится \uline{\textit{изоморфной}} конфигурации \textit{связей} между описываемыми \textit{сущностями}, которые обозначаются этими \textit{знаками}.}
                    \scnfileitem{Способ (язык) \textit{смыслового представления информации}, должен быть универсальным, т.е. должно быть обеспечено описание (и, прежде всего, обозначение) \uline{любых} \textit{связей} между описываемыми сущностями. При этом, если описываемые \textit{связи} считать одним из видов описываемых \textit{сущностей}, то можно описывать \textit{связи} между \textit{связями}, \textit{связи}, связывающие \textit{связи} с описываемыми \textit{сущностями} иных видов.}
                \end{scnrelfromvector}
            \end{scnindent}
        \end{scnindent}
        \scnfileitem{Абстрактная память \textit{ostis-системы} является графодинамической (т.е. нелинейной (графовой) и структурно перестраиваемой). Переработка информации в памяти \textit{ostis-системы} сводится не столько к изменению состояния элементов памяти (это происходит только при изменении синтаксического типа элементов и при изменении содержимого тех элементов, которые обозначают файлы), сколько к изменению \uline{конфигурации связей} между ними.}
        \scnfileitem{Ориентация на компьютеры нового поколения.}
        \scnfileitem{В основе организации решения задач в памяти \textit{ostis-системы} лежит \textit{агентно-ориентированная модель обработки информации}, управляемая ситуациями и событиями, возникающими в обрабатываемой информации (точнее, в обрабатываемой \textit{базе знаний}). С точки зрения архитектуры \textit{ostis-система} представляет собой \uline{иерархическую} многоагентную систему с общедоступной памятью (т.е. с памятью, общедоступной \uline{всем} агентам \textit{ostis-системы}).
            \\Заметим при этом, что общая память большинства исследуемых в настоящее время \textit{многоагентных систем} является не общедоступной, а распределенной, т.е. представляет собой абстрактное (виртуальное) объединение, в состав которого входит память каждого агента многоагентной системы. Координация деятельности агентов \textit{ostis-системы} при выполнении сложных \textit{действий в памяти} \textit{ostis-системы} реализуется также через \textit{память ostis-системы} с помощью хранимых в памяти \textit{методов} решения различных \textit{классов задач}, а также с помощью хранимых в памяти \textit{планов} решения конкретных задач.
            \\На основании этого можно строить неограниченную иерархическую систему \textit{агентов ostis-системы} --- от элементарных агентов, обеспечивающих выполнение базовых действий в памяти \textit{ostis-системы}, до неэлементарных агентов, представляющих собой коллективы (группы) элементарных и/или неэлементарных агентов, обеспечивающих решение различных типовых задач с помощью соответствующих методов и планов.}
        \scnfileitem{Реализация децентрализованного ситуационного управления деятельностью \textit{ostis-систем} не только на уровне внутренних информационных процессов, но также на уровне организации индивидуальной деятельности во внешней среде и даже на уровне участия в коллективной деятельности в рамках различных коллективов \textit{ostis-систем}. Организация выполнения \textit{ostis-системой действий во внешней среде} осуществляется следующим образом:\\
            \begin{itemize}
                \item Выделяются классы \textit{элементарных действий во внешней среде}, для реализации каждого из которых вводятся \textit{эффекторные агенты} \textit{ostis-системы}.
                \item Координация деятельности \textit{эффекторных агентов} \textit{ostis-системы} при выполнении \textit{сложных действий во внешней среде} осуществляется через \textit{память ostis-системы} с помощью хранимых в памяти \textit{методов} и \textit{планов} решения различных задач во \textit{внешней среде}, а также с помощью \textit{рецепторных агентов} \textit{ostis-системы}, обеспечивающих обратную связь и, соответственно, сенсомоторную координацию.
            \end{itemize}}
        \scnfileitem{Унификация базового набора (базовой системы) используемых \textit{понятий}, что является основой обеспечения \textit{семантической совместимости} всех \textit{ostis-систем}.}
        \scnfileitem{В основе структуризации информации (\textit{базы знаний}), хранимой в памяти \textit{ostis-системы}, лежит иерархическая система \textit{предметных областей} и соответствующих им \textit{формальных онтологий}.}
        \scnfileitem{Переход от исследования обработки данных (data science) к исследованию обработки знаний (knowledge science), что предполагает при разработке различных классов задач \uline{учет семантики обрабатываемой информации}. В этом смысле традиционное программирование хромает на одну ногу}
        \scnfileitem{Способность к пониманию (к семантическому погружению, к семантической интеграции) новых приобретаемых знаний (и, в том числе, новых навыков) в состав текущего состояния \textit{базы знаний}.}
        \scnfileitem{Способность к \textit{семантической конвергенции} (к обнаружению сходств) новых приобретаемых знаний (и, в частности, навыков) со знаниями, входящими в состав текущего состояния \textit{базы знаний} \textit{ostis-системы}.}
        \scnfileitem{Способность к интеграции различных видов \textit{знаний}.}
        \scnfileitem{Способность к интеграции различных \textit{моделей решения задач}.}
        \scnfileitem{Способность \textit{ostis-систем} понимать друг друга, а также любого своего пользователя путем согласования системы используемых понятий (по терминам и по денотационной семантике). Способность \textit{ostis-системы} обеспечивать и поддерживать высокий уровень своей \textit{семантической совместимости} (высокий уровень взаимопонимания) с другими \textit{ostis-системами} в процессе собственной эволюции, а также эволюции других ostis-систем, которая приводит к расширению и/или корректировке системы используемых \textit{понятий}.}
        \scnfileitem{Способность \textit{ostis-системы} согласовывать, координировать свою деятельность с другими системами при решении задач, которые усилиями одной (индивидуальной) интеллектуальной компьютерной системы не могут быть решены либо принципиально, либо за разумное время.}
        \scnfileitem{Высокая степень индивидуальной обучаемости \textit{ostis-систем}, обеспечиваемая высокой степенью их гибкости, стратифицированности, рефлексивности, а также универсальностью средств представления и образования \textit{знаний}.}
        \scnfileitem{Высокая степень коллективной обучаемости \textit{ostis-систем}, обеспечиваемая высокой степенью их \textit{семантической совместимости}.}
    \end{scnrelfromset}
	\begin{scnindent}
	   	\scntext{следовательно}{Перечисленные свойства \textit{ostis-систем} свидетельствуют о том, что они имеют существенно более высокий \textit{уровень интеллекта} и, в частности, более высокий \textit{уровень социализации} по сравнению с современными \textit{интеллектуальными компьютерными системами}.}
	\end{scnindent}

    \scnheaderlocal{\scnnonamednode}
	\begin{scneqtoset}
		\scnitem{память*}
		\scnitem{ostis-система}
	\end{scneqtoset}
    \scnrelfrom{сужение второго домена заданного отношения для заданного первого домена}{память ostis-системы}
	\begin{scnindent}
    	\scnsubset{смысловая память}
	\end{scnindent}
   
    \scnheaderlocal{\scnnonamednode}
	\begin{scneqtoset}
		\scnitem{информация, хранимая в памяти кибернетической системы*}
		\scnitem{ostis-система}
	\end{scneqtoset}
    \scnrelfrom{сужение второго домена заданного отношения для заданного первого домена}{база знаний ostis-системы}
   	\begin{scnindent}
   		\scnsubset{смысловое представление информации}
   	\end{scnindent}
    
    \scnheader{решатель задач ostis-системы}
    \scnsubset{агентно-ориентированная модель обработки информации в памяти}

    \scnheader{смысловое представление информации}
    \begin{scnrelfromset}{принципы, лежащие в основе}
        \scnfileitem{Каждый синтаксически элементарный (атомарный) фрагмент представленной информации является обозначением некоторой сущности, которая может быть реальной или абстрактной, конкретной (фиксированной, константной) или произвольной (переменной), постоянной или временной, четкой (достоверной) или нечеткой (недостоверной с возможным дополнительным уточнением степени правдоподобности).}
	        \begin{scnindent}
	        	\scntext{следовательно}{В состав смыслового представления информации не могут входить буквы (не являются обозначениями сущностей), слова, словосочетания (не являются элементарными фрагментами), разделители, ограничители (не являются обозначениями сущностей)}
	        \end{scnindent}
        \scnfileitem{В рамках смыслового представления информации отсутствует синонимия (пары синонимичных знаков), омонимия (омонимичные знаки), семантическая эквивалентность (пары семантически эквивалентных информационных конструкций), т.е. отсутствует любая форма дублирования информации, а также отсутствует неоднозначность соотношения между знаками и их денотатами.}
    \end{scnrelfromset}
    \begin{scnindent}
        \scntext{следовательно}{Смысловое представление информации не может выглядеть как цепочка (строка, последовательность) синтаксически элементарных фрагментов, поскольку каждая описываемая сущность и взаимно однозначно соответствующий ей ее знак может быть связана не с двумя, а с любым количеством описываемых сущностей. Другими словами, смысловое представление информации является нелинейной (графовой) информационной конструкцией.}
        \begin{scnindent}
            \scntext{следовательно}{Если внутреннее представление информации в памяти компьютерной системы является смысловым представлением, то обработка информации в такой памяти носит графодинамический характер и сводится не к изменению состояния элементов памяти, а к изменению конфигурации связей между ними.}
        \end{scnindent}
    \end{scnindent}
    \scntext{примечание}{Ключевая проблема современного этапа развития общей теории интеллектуальных компьютерных систем и технологии их разработки это проблема обеспечения \textbf{\textit{семантической совместимости}}
        \begin{scnitemize}
            \item различных видов знаний, входящих в состав баз знаний интеллектуальных компьютерных систем;
            \item различных видов моделей решателей задач;
            \item различных интеллектуальных компьютерных систем в целом;
        \end{scnitemize}
        Для решения этой проблемы очевидно необходима унификация (стандартизация) формы представления знаний в памяти интеллектуальных компьютерных систем. Предлагаемым нами подходом для такой унификации и является ориентация на \textbf{\textit{смысловое представление информации}} (знаний) в памяти интеллектуальных компьютерных систем. Основой предполагаемого нами подхода к обеспечению высокого уровня обучаемости и семантической совместимости интеллектуальных компьютерных систем, а также к разработке стандарта интеллектуальных компьютерных систем является унификация \textbf{\textit{смыслового представления информации}} (знаний) в памяти интеллектуальных компьютерных систем и построение глобального \textbf{\textit{смыслового пространства}} знаний.}
    \begin{scnindent}
        \scntext{примечание}{Информация в знаковой конструкции в основном содержится не в самих знаках (в их структуре), а в связях между знаками. При этом существенно, чтобы эти связи (синтаксические связи) имели четкую смысловую (семантическую) интерпретацию. Если структура знаков содержит информацию об обозначаемой сущности всегда можно заменить на бесструктурные знаки, которые имеют семантическую окрестность}
    \end{scnindent}
    
    \scnheader{семантическая сеть}
    \scnsubset{смысловое представление информации}
    \scntext{пояснение}{Семантическая сеть нами рассматривается не как красивая метафора сложноструктурированных знаковых конструкций, а как формальное уточнение понятия смыслового представления информации, как принцип представления информации, лежащей в основе нового поколения компьютерных языков и самих компьютерных систем --- графовых языков и графовых компьютеров.}
    \scnsubset{знаковая конструкция}
    \scntext{пояснение}{Семантическая сеть --- это знаковая конструкция, обладающая следующими свойствами:
        \begin{itemize}
            \item внутренюю структуру (строение) знаков, входящих в семантическую сеть не требуется учитывать при ее семантическом анализе (понимании)
            \item Смысл семантической сети определяется денотационной семантикой всех входящих в нее знаков и конфигурацией связей инцидентности этих знаков
            \item Из двух инцидентных знаков, входящих в семантическую сеть, один является знаком связи
            \item Отсутствие синонимии, омонимии
        \end{itemize}}
    \scnrelfrom{предлагаемый подход}{\scnkeyword{SC-код}}
    \begin{scnindent}
        \scnidtf{Предлагаемое в рамках \textit{Технологии OSTIS} уточнение понятия \textit{семантической сети}}
        \scnsubset{семантическая сеть}
        \scnidtf{Semantic Computer Code}
        \scnrelfrom{смотрите}{Предметная область и онтология внутреннего языка ostis-систем}
    \end{scnindent}
    
    \scnheader{многоагентная система}
    \scnsubset{кибернетическая система}
    \scntext{пояснение}{Кибернетическая система, представляющая собой множество кибернетических систем, способных коммуницировать, т.е. обмениваться информацией друг с другом (причем не обязательно каждый с каждым)}
    
    \scnheader{агент*}
    \scnidtf{агент многоагентной системы*}
    
    \scnheader{внешняя среда*}
    \scnidtf{внешняя среда кибернетической системы}
    
    \scnheader{память*}
    \scnidtf{внутренняя (информационная) среда кибернетической системы}
    \scntext{примечание}{Не каждая кибернетическая система (в том числе многоагентная система) имеет явно выделенную память, являющуюся хранилищем накапливаемой информации, накапливаемого опыта.}
    
    \scnheader{многоагентная система}
    \begin{scnsubdividing}
        \scnitem{многоагентная система без общей памяти}
        \scnitem{многоагентная система с общей памятью}
    \end{scnsubdividing}
    \begin{scnsubdividing}
        \scnitem{многоагентная система, в которой управление агентами осуществляется только путем обмена сообщениями между ними}
        \scnitem{многоагентная система, в которой управление агентами осуществляется через общую для них память}
    \end{scnsubdividing}
    \begin{scnsubdividing}
        \scnitem{многоагентная система с централизованным управлением агентами}
        \scnitem{многоагентная система с децентрализованным управлением агентами}
    \end{scnsubdividing}
    \begin{scnsubdividing}
        \scnitem{многоагентная система, в которой областью деятельности всех ее агентов является только внешняя среда этой системы}
        \scnitem{многоагентная система, в которой областью деятельности ее агентов является как внешняя среда, так и память этой системы}
        \begin{scnindent}
            \scntext{примечание}{некоторые агенты такой системы могут работать только в памяти}
        \end{scnindent}
    \end{scnsubdividing}
    
    \scnheader{агентно-ориентированная модель обработки информации в памяти}
    \scnidtf{агентно-ориентированная модель решения задач}
    \scnidtf{агентно-ориентированная архитектура решателя задач, представляющая собой многоагентную систему, в которой управление ее агентами осуществляется общей для них памятью и областью деятельности агентов является та же самая общая для них память}
	    \begin{scnindent}
	    	\scntext{следовательно}{условием инициирования каждого указанного агента является возникновение в указанной памяти соответствующего вида ситуации или события}
	    \end{scnindent}
    \begin{scnreltoset}{пересечение}
        \scnitem{многоагентная система, в которой управление агентами осуществляется через общую для них память}
        \scnitem{многоагентная система с децентрализованным управлением агентами}
        \scnitem{многоагентная система, в которой областью деятельности ее агентов является как внешняя среда, так и память этой системы}
    \end{scnreltoset}
    
    \scnheader{агентно-ориентированная модель обработки информации в памяти}
    \begin{scnrelfromset}{принципы, лежащие в основе}
        \scnfileitem{Распределение целенаправленной деятельности между агентами, выполняющими различные действия в памяти, осуществляется на основе генерируемой в \textit{базе знаний} иерархической системы, описывающей связь (сведение) инициированных целей (задач) с подцелями (подзадачами).}
        \scnfileitem{Условием инициирования агента является появление в базе знаний формулировки той цели (задачи), которая, во-первых, инициирована, а, во-вторых, либо может быть полностью достигнута (решена) этим агентом, либо может быть этим агентом достигнута (решена) частично.}
        \scnfileitem{В результате частичного достижения (решения) некоторой цели (задачи) агент может сгенерировать новые подцели (подзадачи).}
        \scnfileitem{Таким образом, условием инициирования агента обработки информации (базы знаний) является появление соответствующей этому агенту ситуации или соответствия.}
    \end{scnrelfromset}
    \scnrelfrom{предлагаемый подход}{\scnkeyword{абстрактная sc-машина}}
    \begin{scnindent}
        \scnidtf{Предлагаемое в рамках \textit{Технологии OSTIS} уточнение понятия агентно-ориентированной модели обработки информации в памяти}
    \end{scnindent}
    \scnsuperset{абстрактная sc-машина}
    \scntext{примечание}{Децентрализованное (агентно-ориентированное) управление процессом решения задач в ostis-системах реализуется как на внутреннем уровне (на уровне решателя задач ostis-системы), так и на внешнем уровне (на уровне взаимодействия между ostis-системами)}
    
    \scnheader{стандартизация ostis-систем}
    \scnidtf{унификация \textit{ostis-систем}}
    \scntext{пояснение}{Стандартизация \textit{ostis-систем} включает в себя:
        \begin{itemize}
            \item cтандартизацию языка внутреннего представления информации в памяти \textit{ostis-систем}
            \item cтандартизацию принципов децентрализованного управления обработкой информации в памяти \mbox{\textit{ostis-систем}}
            \item cтандартизацию языка описания ситуаций и событий (в памяти \textit{ostis-систем}), которые являются условиями инициирования различных информационных процессов в памяти \textit{ostis-систем}
            \item стандартизацию базового языка спецификации (описания, программирования) агентов, выполняющих соответствующие информационные процессы в памяти \textit{ostis-систем}
            \item стандартизацию базовых языков ввода/вывода информации в/из памяти \textit{ostis-систем}.
        \end{itemize}}
    
    \scnheader{SC-код}
    \scnidtf{Стандарт \textit{смыслового представления информации} в памяти \textit{ostis-системы}, а, точнее, \textit{стандарт семантических сетей}}
    
    \scnheader{абстрактная sc-машина}
    \scnidtf{Стандарт \textit{агентно-ориентированной модели обработки информации в памяти ostis-системы}}
    
    \scnheader{стандартизация}
    \scnidtf{унификация}
    \begin{scnrelfromset}{проблемы текущего состояния}
        \scnfileitem{Разработка и совершенствование стандартов происходит очень медленно}
        \scnfileitem{В разработке и совершенствовании стандартов принимает участие явно недостаточное число профессионалов --- не учитываются все мнения}
        \scnfileitem{В разработке и совершенствовании стандарта отсутствует четкая методика формирования консенсуса}
        \scnfileitem{При введении новой версии стандарта отсутствует четкая методика перевода на новую версию стандарта всех систем, разработанных по предыдущей версии}
    \end{scnrelfromset}
    \scntext{предлагаемый подход}{Стандарт --- это перманентно совершенствуемая \textit{база знаний}, поддержку эволюции которой осуществляет соответствующий портал}
    
    \scnheader{конвергенция знаний в памяти ostis-системы}
    \begin{scnrelfromset}{принципы, лежащие в основе}
        \scnfileitem{Вводится \uline{универсальный} базовый язык внутреннего \uline{смыслового} представления знаний в памяти \mbox{ostis-систем} (\mbox{\textit{SC-код}}), по строению к которому все внутренние языки, ориентированные на представление знаний различного вида (логические языки, языки представления методов решения задач (в частности, программ), язык формулировки задач, онтологические языки и многие другие) являются подъязыками \mbox{\textit{SC-кода}}, синтаксис которых полностью совпадает с синтаксисом \mbox{\textit{SC-кода}}.}
        \scnfileitem{Конвергенция различных знаний сводится к согласованию систем понятий, используемых для представления знаний различного вида. Такое согласование направлено на увеличение числа общих понятий, используемых при представлении различных знаний.}
    \end{scnrelfromset}
    
    \scnheader{конвергенция моделей решения задач в \mbox{ostis-системе}}
    \begin{scnrelfromset}{принципы, лежащие в основе}
        \scnfileitem{Синтаксис языка представления соответствующего класса методов решения задач в памяти --- синтаксис \mbox{SC-кода}}
        \scnfileitem{Денотационная семантика описывается в виде соответствующей онтологии и представляется в виде текста \mbox{SC-кода}}
        \scnfileitem{Операционная семантика каждой модели решения задач --- коллектив \uline{агентов}. Он может быть иерархическим на основе различных моделей решателей, но есть базовая модель интерпретации \uline{любых} методов:\\
            \begin{scnitemize}
                \item Язык SCP
                \begin{scnitemizeii}
                    \item cинтаксис совпадает с синтаксисом SC-кода
                    \item денотационная семантика --- процедурный язык программирования в графодинамической памяти
                    \item операционная семантика реализуется на уровне программной или аппаратной платформы
                \end{scnitemizeii}
                \item sc-агенты работают в общей среде --- (sc-памяти) параллельно, асинхронно на основе ряда правил, позволяющих им не мешать друг другу
            \end{scnitemize}}
    \end{scnrelfromset}
    
    \scnheader{интеграция знаний в памяти ostis-системы*}
    \scntext{пояснение}{Интеграция знаний в памяти \textit{ostis-систем} сводится к склеиванию (отождествлению) синонимичных знаков}
    
    \scnheader{интеграция моделей решения задач в ostis-системе*}
    \scntext{пояснение}{Поскольку модель решения задач, используемая ostis-системой, представлена в памяти ostis-системы как соответствующий вид знаний, интеграция различных моделей решения задач может происходить в ostis-системе точно так же, как и интеграция любых других видов знаний. Кроме того, когда речь идет об интеграции различных моделей решения задач, имеется в виду возможность одновременного использования различных моделей решения задач при обработке одних и тех же знаний и, в частности, при решении одной и той же задачи. Такая возможность в ostis-системе обеспечивается \textit{агентно-ориентированной моделью обработки информации} в памяти ostis-системы. Таким образом, такого рода интеграция различных моделей решения задач для ostis-систем является тривиальной.}
    
    \scnheader{ostis-система}
    \begin{scnrelfromset}{достоинства}
        \scnfileitem{Высокий уровень способности \textit{ostis-системы} осуществлять семантическую интеграцию знаний в своей памяти (в частности, при погружении новых знаний в текущее состояние базы знаний) \uline{обеспечивается} смысловым характером внутреннего кодирования информации,  хранимой в памяти ostis-системы и, в частности, тем, что во внутреннем коде базы знаний \textit{ostis-системы} запрещены омонимичные знаки и пары синонимичных знаков.}
        \scnfileitem{Высокий уровень способности интегрировать различные виды знаний в \textit{ostis-системах} обеспечивается тем, что каждый язык, ориентированный на представление знаний соответствующего вида является \uline{подъязыком} одного и того же базового языка \textit{SC-кода}.}
        \begin{scnindent}
            \scntext{примечание}{Кроме того можно говорить об иерархии sc-языков}
        \end{scnindent}
        \scnfileitem{Высокий уровень способности интегрировать различные модели решения задач в \textit{ostis-системах} \uline{обеспечивается}:\\
            \begin{itemize}
                \item тем, что все эти модели ориентированы на обработку информации, представленной в \textit{SC-коде};
                \item один и тот же фрагмент базы знаний ostis-системы (т.е. одна и та же конструкция SC-кода) может одновременно обрабатываться несколькими \uline{разными} моделями решения задач;
                \item все модели решения задач в ostis-системах интегрируются с помощью одной и той же базовой модели решения задач --- \textit{scp-модели решения задач}.
            \end{itemize}}
        \scnfileitem{Высокий уровень обучаемости \textit{ostis-систем} \uline{обеспечивается}:\\
            \begin{itemize}
                \item высоким уровнем семантической гибкости информации, хранимой в памяти ostis-системы, поскольку каждое удаление или добавление синтаксически элементарного фрагмента хранимой информации, а также удаление или добавление каждой связи инцидентности между такими элементами имеет четкую семантическую интерпретацию;
                \item высоким уровнем стратифицированности хранимой информации, что обеспечивается онтологически ориентированной структуризацией базы знаний ostis-системы;
                \item высоким уровнем рефлексии ostis-системы, что обеспечивается мощными метаязыковыми возможностями языка внутреннего представления информации (знаний) в памяти \textit{ostis-систем}.
            \end{itemize}}
        \scnfileitem{Каждая \textit{ostis-система} имеет высокий \textit{уровень обучаемости} (способности к быстрому расширению своих \textit{знаний} и \textit{навыков}) и высокий \textit{уровень социализации} (способности к эффективному участию в деятельности различных коллективов  коллективов, состоящих из \textit{ostis-систем}, и сообществ, состоящих из \textit{ostis-систем} и людей.}
        \begin{scnindent}
	        \begin{scnrelfromset}{детализация достоинства}
	            \scnfileitem{Существуют четкие формальные критерии, определяющие \textit{уровень семантической совместимости} (уровень семантической конвергенции) различных знаний, навыков, целых \textit{ostis-систем} (точнее, баз знаний этих систем). Очевидно, что \textit{уровень семантической совместимости} прежде всего определяется количеством точек соприкосновения в сравниваемых \textit{знаниях}, \textit{навыках} и \textit{базах знаний}  это \textit{знаки}, присутствующие \uline{в разных} сравниваемых объектах, но имеющие одинаковые денотаты (т.е. обозначающие одинаковые сущности). При этом среди таких знаков, обозначающих одинаковые сущности и присутствующих в разных сравниваемых объектах особенно важны знаки, обозначающие \textit{понятия}.Количество таких общих понятий в сравниваемых знаниях, навыках, базах знаний определяет уровень семантической совместимости (уровень согласованности) систем используемых понятий в сравниваемых указанных объектах. Увеличение количества знаков, обозначающих одинаковые сущности и присутствующих в разных сравниваемых объектах, может привести к тому, что в разных указанных сравниваемых объектах будут присутствовать не только семантически эквивалентные знаки, но и семантически эквивалентные целые фрагменты (целые информационные конструкции).Существенно при этом подчеркнуть, что семантически эквивалентные знаковые конструкции, представленные на внутреннем языке ostis-систем (в SC-коде), в памяти разных ostis-систем всегда являются синтаксически изоморфными графовыми конструкциями, в которых соответствие изоморфизма связывает знаки, хранимые в памяти разных ostis-систем, но обозначающие одинаковые сущности (точнее, одну и ту же сущность). Заметим также, что в рамках памяти каждой индивидуальной \textit{ostis-системы} синонимия знаков и, соответственно, семантическая эквивалентность знаковых конструкций запрещены.}
	            \scnfileitem{Благодаря постоянно развиваемым семантическим стандартам \textit{Технологии OSTIS} , которые представлены системой формальных онтологий для самых различных предметных областей, разрабатываемые \textit{ostis-системы} \uline{изначально} имеют достаточно высокий \textit{уровень семантической совместимости} со всеми остальными \textit{ostis-системами}. Более того, в \textit{Технологии OSTIS} выделяется целое ядро всех ostis-систем, содержащее фундаментальные базовые знания и базовые навыки, одинаковые для всех ostis-систем и позволяющее каждой копии этого ядра развиваться (общаться, специализироваться) в любом направлении.}
	            \scnfileitem{Каждая ostis-система, взаимодействуя с людьми (пользователями) или с другими \mbox{ostis-системами}, обладает способностью повышать уровень семантической совместимости (взаимопонимания) с ними, а также поддерживать (сохранять) высокий уровень такой совместимости в условиях (1) собственной эволюции, (2) эволюции других ostis-систем и пользователей, (3) эволюции семантических стандартов Технологии OSTIS. Указанное взаимодействие, в основном, направлено на согласование изменений в системе используемых понятий, т.е. корректировки соответствующих фрагментов онтологий.}
	            \scnfileitem{Благодаря высокому уровню семантической совместимости ostis-систем и смысловому представлению знаний в памяти ostis-систем существенно снижается сложность и повышается качество семантического анализа и понимания информации, поступающей (сообщаемой, передаваемой) ostis-системе от других ostis-систем или пользователей.}
	            \scnfileitem{Каждая ostis-система способна:
	                \begin{itemize}
	                    \item самостоятельно или по приглашению войти в состав ostis-коллектива (коллектива ostis-систем) или в состав ostis-сообщества, состоящего из ostis-систем и людей. Такие коллективы и сообщества создаются на временной (разовой) или постоянной основе для коллективного решения сложных задач
	                    \item участвовать в распределении (в т.ч. в согласовании распределения) задач --- как разовых задач, так и долгосрочных задач (обязанностей)
	                    \item мониторить состояние всего процесса коллективной деятельности и координировать свою деятельность с деятельностью других членов коллектива при возможных непредсказуемых изменениях условий (состояния) соответствующей среды.
	                \end{itemize}}
	        \end{scnrelfromset}
        \end{scnindent}
        \scnfileitem{Высокий уровень интеллекта ostis-систем и, соответственно, высокий уровень их самостоятельности и целенаправленности позволяет ostis-системам быть полноправными членами самых различных сообществ, в рамках которых ostis-системы получают права самостоятельно инициировать (на основе детального анализа текущего положения дел и, в том числе, текущего состояния плана действий сообщества) широкий спектр действий (задач), выполняемых другими членами сообщества, и тем самым участвовать в согласовании и координации деятельности членов сообщества.}
        \scnfileitem{Способность ostis-системы согласовывать свою деятельность с другими ostis-системами, а также корректировать деятельность всего коллектива ostis-систем, адаптируясь к различного вида изменениям среды (условий), в которой эта деятельность осуществляется, позволяет существенно автоматизировать деятельность системного интегратора как на этапе сборки коллектива ostis-систем, так и на этапе его обновления (реинжиниринга).}
    \end{scnrelfromset}
    \scntext{примечание}{Достоинства \textit{ostis-систем} обеспечиваются:
        \begin{itemize}
            \item достоинствами \textit{SC-кода} --- языка внутреннего кодирования информации, хранимой в памяти \textit{ostis-систем}
            \item достоинствами организации \textit{sc-памяти} --- памяти \textit{ostis-систем}
            \item достоинствами \textit{sc-моделей баз знаний} ostis\textit{}систем средствами структуризации таких \textit{баз знаний}
            \item достоинствами \textit{sc-моделей решения задач} --- агентно-ориентированных моделей решения задач, используемых в \textit{ostis-системах}.
        \end{itemize}}
    
\end{scnsubstruct}
\scnsourcecommentinline{Завершили рассмотрение понятия ostis-системы}

\scnheader{Понятие ostis-сообщества}
\begin{scnsubstruct}
	
    \scnheader{ostis-сообщество}
    \scnidtf{Человеко-машинный симбиоз, представляющий собой коллектив, состоящий из людей и ostis-систем и обеспечивающий высокий уровень автоматизации определённого (соответствующего) вида человеческой деятельности.}
    \scntext{примечание}{В состав каждого ostis-сообщества входит корпоративная ostis-система, которая в рамках этого \mbox{ostis-сообщества} выполняет:
        \begin{scnitemize}
            \item роль координатора деятельности членов данного ostis-сообщества;
            \item роль памяти ostis-сообщества, т.е. хранителя общих (обобществляемых, общедоступных) знаний для всех членов данного ostis-сообщества, которое несет ответственность за совершенствование этих знаний, а также для всех членов всех тех ostis-сообществ, в состав которых данное ostis-сообщество входит (указанные субъекты являются пользователями рассматриваемых общих знаний). Таким образом, корпоративная ostis-система некоторого ostis-сообщества является официальным представителем этого ostis-сообщества во всех ostis-сообществах, в состав которых входит, и, следовательно, является координатором деятельности даного ostis-сообщества (как единого целого) в рамках всех ostis-сообществ, в состав которых оно входит;
        \end{scnitemize}}
    
    \scnheader{есть сходства*}
    \begin{scnhaselementset}
        \scnitem{ostis-сообщество}
        \scnitem{решатель задач ostis-системы}
    \end{scnhaselementset}
    \begin{scnindent}
    	\begin{scnrelfromset}{пояснение}
   			\scnitem{ostis-сообщество}
		    \begin{scnindent}
	   			\scnsuperset{многоагентная система, в которой управление агентами осуществляется через общую для них память}
	   			\scnsuperset{многоагентная система, с децентрализованным управлением агентами}
	   			\scnsuperset{многоагентная система, в которой областью деятельности её агентов является как внешняя среда, так и память этой системы}
  			 \end{scnindent}
   			\scnitem{решатель задач ostis-системы}
   			\begin{scnindent}
	   			\scnsuperset{многоагентная система, в которой управление агентами осуществляется через общую для них память}
	   			\scnsuperset{многоагентная система, с децентрализованным управлением агентами}
	   			\scnsuperset{многоагентная система, в которой областью деятельности её агентов является как внешняя среда, так и память этой системы}
  			 \end{scnindent}
   			\scnitem{агентно-ориентированная модель обработки информации в памяти}
    	\end{scnrelfromset}
    \end{scnindent}
    
    \scnheader{многоагентная система с децентрализованным управлением агентами}
    \begin{scnrelfromlist}{включение;пример}
        \scnitem{оркестр, играющий без дирижера или даже без композитора}
        \begin{scnindent}
            \scntext{необходимое требование}{каждый участник оркестра должен иметь квалификацию дирижера или композитора}
        \end{scnindent}
       \scnitem{комплексная строительная бригада, работающая без прораба}
        \begin{scnindent}
            \scntext{необходимое требование}{каждый участник строительной бригады должен иметь квалификацию прораба}
        \end{scnindent}
        \scnitem{научно-исследовательская лаборатория, работающая без заведующего и научного руководителя}
        \begin{scnindent}
            \scntext{необходимое требование}{каждый участник научно-исследовательской лаборатории должен иметь квалификацию заведующего или научного руководителя}
        \end{scnindent}
        \scnitem{кафедра, работающая без заведующего и ученого секретаря}
        \begin{scnindent}
            \scntext{необходимое требование}{каждый участник кафедры должен иметь квалификацию заведующего и ученого секретаря}
        \end{scnindent}
    \end{scnrelfromlist}
\end{scnsubstruct}
\scnsourcecommentinline{Завершили рассмотрение понятия ostis-сообщества}
\bigskip

\scnstructheader{Понятие ostis-технологии}
\begin{scnsubstruct}
	
    \scnheader{ostis-технология}
    \begin{scnreltoset}{объединение}
        \scnitem{ostis-технология проектирования}
        \begin{scnindent}
            \begin{scnsubdividing}
                \scnitem{ostis-технология проектирования ostis-систем соответствующего класса}
                \begin{scnindent}
                    \scnhaselement{Базовая ostis-технология проектирования ostis-систем}
                \end{scnindent}
                \scnitem{ostis-технология проектирования соответствующего класса компонентов ostis-систем}
                \begin{scnindent}
                    \scnhaselement{Базовая ostis-технология проектирования баз знаний ostis-систем}
                    \scnhaselement{Базовая ostis-технология проектирования решателей задач ostis-систем}
                    \scnhaselement{Базовая ostis-технология проектирования интерфейсов ostis-систем}
                \end{scnindent}
                \scnitem{ostis-технология проектирования объектов заданного класса, не являющихся ostis-системами}
            \end{scnsubdividing}
        \end{scnindent}
        \scnitem{ostis-технология производства}
        \begin{scnindent}
            \scnsuperset{технология производства спроектированных ostis-систем}
            \scnsuperset{ostis-технология управления производством спроектированных продуктов заданного класса, не являющихся ostis-системами}
        \end{scnindent}
        \scnitem{технология эксплуатации ostis-систем}
        \begin{scnindent}
            \scnhaselement{Базовая технология эксплуатации ostis-систем}
	        \scnsuperset{технология эксплуатации ostis-систем соответствующего класса}
	            \begin{scnindent}
		            \scnsuperset{ostis-технология управления производством спроектированных продуктов заданного класса, не являющихся ostis-системами}
		            \begin{scnindent}
		            	\scnidtf{технология эксплуатации ostis-систем управления производством спроектированных продуктов заданного класса, не являющихся ostis-системами}
		            \end{scnindent}
	            \end{scnindent}
        \end{scnindent}
        \scnitem{технология реинжиниринга ostis-систем}
        \begin{scnindent}
            \scnhaselement{Базовая технология реинжиниринга ostis-систем}
            \scnsuperset{технология реинжиниринга ostis-систем соответствующего класса}
        \end{scnindent}
    \end{scnreltoset}
    
    \scnheader{ostis-технология}
    \scnidtf{компонент Технологии OSTIS}
    \scnhaselement{Ядро Технологии OSTIS}
	    \begin{scnindent}
	    	\scnidtf{Базовая ostis-технология}
	    \end{scnindent}
    \scnsuperset{частная ostis-технология}
    \begin{scnindent}
	    \scnsuperset{ostis-технология проектирования соответствующего класса компонентов ostis-систем}
	    \begin{scnindent}
		    \scnhaselement{Технология проектирования баз знаний ostis-систем}
		    \scnhaselement{Технология проектирования решателей задач ostis-систем}
		    \scnhaselement{Технология проектирования невербальных интерфейсов ostis-систем с внешней средой}
		    \scnhaselement{Технология проектирования интерфейсов ostis-систем с другими техническими системами}
		    \scnhaselement{Технология проектирования пользовательских интерфейсов ostis-систем}
		 \end{scnindent}
	\end{scnindent}
    \scnsuperset{специализированная ostis-технология проектирования ostis-систем соответствующего класса}
    \begin{scnindent}
	    \scnhaselement{Технология проектирования ostis-систем управления предприятиями рецептурного производства}
	    \scnhaselement{Технология проектирования ostis-систем управления предприятиями производства молочной продукции}
	    \scnhaselement{Технология проектирования интеллектуальных обучающих ostis-систем}
	    \scnhaselement{Технология проектирования интеллектуальных обучающих ostis-систем для школьников}
	    \scnhaselement{Технология проектирования интеллектуальных обучающих ostis-систем для подготовки специалистов в области Математики}
	    \scnhaselement{Технология проектирования интеллектуальных обучающих ostis-систем для подготовки специалистов в области Искуственного интеллекта}
	\end{scnindent}
    
    \scnheader{ostis-технология проектирования}
    \scntext{примечание}{Каждой ostis-технологии проектирования соответсвует своя ostis-система автоматизации проектирования соответствующего класса объектов}
    \scnrelfrom{соответствующее семейство средств автоматизации}{ostis-система автоматизации проектирования}
    \scnrelfrom{соответствующее семейство классов проектируемых объектов}{(ostis-система автоматизации проектирования ostis-систем $\cup$ ostis-система автоматизации проектирования объектов, не являющихся ostis-системами)}
    \scnsuperset{ostis-технология проектирования ostis-систем соответствующего класса}
    
    \scnheader{ostis-технология проектирования ostis-систем соответствующего класса}
    \scnidtf{технология проектирования \textit{ostis-систем} соответствующего (заданного) класса, который, в свою очередь, соответствует определенному \textit{виду человеческой деятельности}, подвиды которого автоматизируются с помощью указанных выше проектируемых \textit{ostis-систем}}
    
    \scnheader{ostis-технология}
    \begin{scnrelfromlist}{отношение, заданное на данном множестве}
        \scnitem{частная технология*}
        \scnitem{специализированная технология*}
        \scnitem{комплекс специализированных технологий*}
    \end{scnrelfromlist}
    \scntext{пояснение}{Базовая частная или специализированная технология, входящая в состав комплексной \textit{Технологии OSTIS}, которая:
        \begin{scnitemize}
            \item направлена на автоматизацию конкретного вида человеческой деятельности;
            \item ориентирована на использование ostis-систем (как индивидуальных, так и коллективных) в качестве самостоятельных субъектов или активных интеллектуальных инструментов, либо на использование человеко-машинных ostis-сообществ при решении:
            \begin{scnitemizeii}
                \item как задач, выполняемых в памяти ostis-систем (в т.ч. в памяти коллективов ostis-систем);
                \item так и задач, выполняемых во внешней среде ostis-систем, в процессе решения которых субъектами соответствующих действий либо ostis-системы (индивидуальные или коллективные), либо конкретные персоны, либо ostis-сообщества.
            \end{scnitemizeii}
        \end{scnitemize}}
    \scnidtf{Множество всевозможных технологий, соответствующих стандартам технологии OSTIS и направленных на автоматизацию различных конкретных видов человеческой деятельности}
    \scnrelboth{следует отличать}{Технология OSTIS}
    \begin{scnindent}
    	\scntext{примечание}{\textit{Технология OSTIS} в отличие от понятия \textit{ostis-технологии} представляет собой не множество технологий, а комплекс взаимосвязанных между собой самых различных технологий, превращающий указанное множество технологий в единую объединенную технологию, в сумму взаимосвязанных глубоко интегрированных технологий. В этом смысле Технология OSTIS является максимальной ostis-технологией, в состав которой входят все ostis-технологии.}
    \end{scnindent}
    \begin{scnrelfromlist}{включение;пример}
        \scnitem{ostis-технология проектирования и перепроектирования}
        \scnitem{ostis-технология производства}
        \scnitem{ostis-технология образования}
    \end{scnrelfromlist}
    \begin{scnhaselementrolelist}{пример}
        \scnitem{Технология OSTIS}
        \scnitem{OSTIS-технология публикации и согласования результатов научно-технической деятельности (в широком смысле)}
        \scnitem{OSTIS-технология проектирования, реализации и реинжиниринга ostis-систем}
        \scnitem{OSTIS-технология разработки стандартов Технологии OSTIS}
    \end{scnhaselementrolelist}
    
    \scnheader{ostis-технология коллективной разработки информационных ресурсов}
    \scnsuperset{ostis-технология коллективного проектирования}
    \scnsuperset{ostis-технология коллективной разработки планов}
    \scnsuperset{ostis-технология публикации и согласования результатов научно-технической деятельности}
    \scnsubset{ostis-технология}
    
    \scnheader{ostis-технология эксплуатации ostis-систем}
    \scnidtf{Общие методы и средства (языковые и интерфейсные) организации взаимодействия ostis-систем со своими конечными пользователями}
    \scnsubset{ostis-технология}
    \scntext{примечание}{Поскольку в рамках Экосистемы OSTIS каждому человеку придется взаимодействовать с больщим числом ostis-систем разного назначения, принципы организации взаимодействия всех ostis-систем со своими пользователями должны быть абсолютно одинаковыми. Удобство (usability) пользовательских интерфейсов должно быть направлено не только на синтаксическую красоту, но и на простую семантическую интерпретацию (понятность).}
    
    \scnheader{ostis-технология проектирования ostis-систем}
    \scnidtf{Технология построения (разработки) логико-семантических моделей (sc-моделей) ostis-систем}
    \scniselement{ostis-технология}
    \scntext{примечание}{Продуктом каждого завершенного (целостного) коллективного проекта, реализованного в рамках этой технологии, является полная \textit{логико-семантическая модель ostis-системы}.}
    \scnrelfrom{класс продуктов}{логико-семантическая модель ostis-системы}
    \scnrelfrom{средство}{Метасистема OSTIS}
    \scnrelfrom{класс субъектов}{коллектив разработчиков ostis-системы}
    \scnrelfrom{класс исходных данных}{исходная спецификация ostis-системы}
    
    \scnheader{ostis-технология производства ostis-систем}
    \scnidtf{Технология сборки и установки ostis-систем}
    \scniselement{ostis-технология}
    \scnrelfrom{исходная информация}{логико-семантическая модель ostis-системы}
    \scnrelfrom{комплектация}{универсальный интерпретатор логико-семантических моделей ostis-систем}
    \begin{scnindent}
    	\scntext{примечание}{Это, своего рода, мотор, движок ostis-систем}
    \end{scnindent}
    \scnrelfrom{методы}{Методика производства ostis-систем}
    \scnrelfrom{активный инструмент}{Метасистема OSTIS}
    \scnrelfrom{продукты}{ostis-система}
    
    \scnheader{ostis-технология реинжиниринга ostis-систем}
    \scnidtf{Технология обновления (перепроектирования) ostis-систем в ходе их эксплуатации}
    \scniselement{ostis-технология}
    
    \scnheader{следует отличать*}
    \begin{scnhaselementset}
        \scnitem{Технология реинжиниринга ostis-систем}
        \scnitem{Технология проектирования ostis-систем}
    \end{scnhaselementset}
    \begin{scnindent}
    	\scntext{примечание}{Эти технологии сходны. Их методы и средства совпадают. Не совпадают только исходные данные и результаты, которыми в \textit{Технологии обновления ostis-систем} являются предшествующие и последующие состояния ostis-систем. В \textit{Технологии проектирования ostis-систем} исходными данными являются исходные спецификации (замыслы) проектируемых ostis-систем, и результатами --- полные логико-семантические модели этих систем}
    \end{scnindent}
    	
    \scnheader{Технология OSTIS}
    \scnidtf{Совокупность (интеграция, объединение) всех \textit{ostis-технологий}}
    \scnrelto{интеграция}{ostis-технология}
    \scnidtf{Комплекс (множество) семантически совместимых \textit{технологий}, в состав которого входит \textit{Ядро Технологии OSTIS} и иерархическая система \textit{ostis-технологий}, каждая из которых ориентирована на \textit{проектирование}, \textit{производство}, \textit{эксплуатацию} или \textit{реинжиниринг} соответствующего \textit{класса ostis-систем}, обеспечивающих автоматизацию соответствующего \textit{вида человеческой деятельности}. При этом каждая такая проектируемая \textit{ostis-система} автоматизирует либо область, либо \textit{вид человеческой деятельности}, которая (который) является соответственно либо экземпляром (элементом), либо подвидом (подклассом) указанного выше \textit{вида человеческой деятельности}, соответствующего используемой \textit{специализированной \textit{ostis-технологии}}.}
    
    \scnheader{Ядро Технологии OSTIS}
    \scnidtf{Универсальная базовая \textit{ostis-технология}}
    \scnidtf{Универсальный компонент Технологии OSTIS}
    \scniselementrole{ключевой элемент}{ostis-технология}
    \scnrelto{ядро}{Технология OSTIS}
    \scnhaselement{технология}
    \scnrelfrom{вид деятельности, выполняемой с помощью технологии}{проектирование, производство, эксплуатация и реинжиниринг ostis-системы}
    \begin{scnindent}
	    \begin{scnreltoset}{объединение}
	        \scnitem{проектирование ostis-системы}
            \begin{scnindent}
                \scnidtf{построение логико-семантической модели \textit{ostis-системы}}
            \end{scnindent}
	        \scnitem{производство ostis-системы}
            \begin{scnindent}
                \scnidtf{сборка логико-семантической модели ostis-системы и загрузка этой модели в память универсального интерпретатора таких моделей}
            \end{scnindent}
	        \scnitem{эксплуатация ostis-системы}
            \begin{scnindent}
                \scnidtf{базовый (предметно-независимый) уровень организации деятельности конечного пользователя ostis-системы с помощью соответствующих методови средств}
            \end{scnindent}
	        \scnitem{реинжиниринг ostis-системы}
                \begin{scnindent}
                    \scnidtf{совершенствование \textit{ostis-системы} в процессе её эксплуатации}
                \end{scnindent}
	    \end{scnreltoset}
	    \scnrelfrom{создаваемые продукты}{ostis-система}
	    \begin{scnindent}
	        \scnidtf{\textit{интеллектуальная компьютерная система}, построенная в соответствии со стандартом \textit{Технологии OSTIS}, предъявляемым к продуктам, создаваемым с помощью этой технологии}
	        \begin{scnindent}
	        	\scntext{примечание}{Указанный стандарт продуктов, создаваемых с помощью технологии OSTIS есть не что иное, как \textit{общая формальная семантическая теория интеллектуальных компьютерных систем}}
	        \end{scnindent}
	    \end{scnindent}
	\end{scnindent}    
   
    \begin{scnrelfromlist}{частная технология}
        \scnitem{Базовая Технология Проектирования ostis-систем}
        \begin{scnindent}
	        \begin{scnrelfromlist}{частная технология}
	            \scnitem{Технология проектирования баз знаний ostis-систем}
	            \scnitem{Технология проектирования решателей задач ostis-систем}
	            \scnitem{Технология проектирования интерфейсов ostis-систем}
                \begin{scnindent}
                    \begin{scnrelfromlist}{частная технология}
                        \scnitem{Технология проектирования невербальных интерфейсов ostis-систем с внешней средой}
                        \scnitem{Технология проектирования интерфейсов ostis-систем с другими техническими системами}
                        \scnitem{Технология проектирования пользовательских интерфейсов ostis-систем}
                    \end{scnrelfromlist}
                \end{scnindent}
	        \end{scnrelfromlist}
	        \scnrelfrom{реализация}{Метасистема OSTIS}
	        \begin{scnindent}
		        \scnidtf{Intelligent MetaSystem for ostis-systems design}
		        \scnidtf{OSTIS-система автоматизации проектирования ostis-систем}
	        \end{scnindent}
	  \end{scnindent}
        \scnitem{Технология производства ostis-систем}
        \begin{scnindent}
            \scntext{пояснение}{Основным компонентом, точнее, инструментальным средством \textit{технологии производства \mbox{ostis-систем}} является \textit{универсальный интерпретатор логико-семантических моделей \mbox{ostis-систем}}. Указанные \textit{логико-семантические модели ostis-систем} являются результатом \textit{проектирования ostis-систем} и представляют собой начальные (исходные) состояния \textit{баз знаний} разрабатываемых \textit{ostis-систем}. В отличие от \textit{инструмента производства ostis-систем}, методика их производства весьма проста и сводится к сборке разработанных логико-семантических моделей (начального состояния \textit{баз знаний}) разрабатываемых \textit{ostis-систем} и загрузке этих моделей в память \textit{универсального интерпретатора логико-семантических моделей \mbox{ostis-систем}}.}
            \scnrelfrom{реализация}{универсальный интерпретатор логико-семантических моделей ostis-систем}
            \begin{scnindent}
	            \scntext{пояснение}{Такой интерпретатор логико-семантических моделей ostis-систем может быть реализован либо программно на \textit{современных компьютерах}, либо аппаратно в виде компьютеров нового поколения, ориентированных на реализацию интеллектуальных компьютерных систем.}
	            \scntext{пояснение}{С формальной точки зрения универсальный интерпретатор логико-семантических моделей ostis-систем является пустой ostis-системой, которая способна приобретать и записывать формализованную информацию в свою память.}
            \end{scnindent}
        \end{scnindent}
        \scnitem{Базовая технология эксплуатации ostis-систем}
        \begin{scnindent}
            \scnidtf{Общая технология эксплуатации ostis-систем, включающая в себя общие методы и средства, используемые в процессе эксплуатации любых ostis-систем}
	         \scnrelfrom{реализация}{встраиваемая ostis-система поддержки эксплуатации ostis-систем}
	         \begin{scnindent}
	            \scntext{пояснение}{Данная ostis-система входит (интегрирована) в состав каждой ostis-системы.}
	         \end{scnindent}
        \end{scnindent}
        \scnitem{Базовая технология реинжиниринга ostis-систем}
        \begin{scnindent}
            \scnrelfrom{реализация}{встраиваемая ostis-система поддержки реинжиниринга ostis-систем}
            \begin{scnindent}
            	\scntext{пояснение}{Данная ostis-система входит (интегрирована) в состав каждой ostis-системы и обеспечивает внесение изменений руками инженеров, сопровождающих эксплуатацию ostis-системы, или авторов базы знаний этой ostis-системы в текущее состояние базы знаний ostis-системы в ходе её экспуатации}
            \end{scnindent}
       \end{scnindent}
    \end{scnrelfromlist}
    \begin{scnrelfromlist}{специализированная технология}
        \scnitem{Общая технология проектирования ostis-систем автоматизации проектирования}
        \begin{scnindent}
            \begin{scnrelfromlist}{специализированная технология}
                \scnitem{Технология проектирования ostis-систем автоматизации проектирования строительных объектов}
                \scnitem{Технология проектирования ostis-систем автоматизации проектирования автомобилей}
                \scnitem{Технология проектирования ostis-систем автоматизации проектирования интегральных микросхем}
            \end{scnrelfromlist}
        \end{scnindent}
        \scnitem{Технология проектирования ostis-систем управления производством}
        \begin{scnindent}
            \begin{scnrelfromlist}{специализированная технология}
                \scnitem{Технология проектирования ostis-систем управления строительством различных объектов}
                \scnitem{Технология проектирования ostis-систем управления производством автомобилей}
                \scnitem{Технология проектирования ostis-систем управления производством микросхем}
                \scnitem{Технология проектирования ostis-систем управления предприятиями рецептурного производства}
                \begin{scnindent}
                    \scnrelfrom{специализированная технология}{Технология проектирования ostis-систем управления предприятиями производства молочной продукции}
                \end{scnindent}
            \end{scnrelfromlist}
        \end{scnindent}
        \scnitem{Технология проектирования интеллектуальных обучающих ostis-систем}
        \begin{scnindent}
            \begin{scnrelfromset}{комплекс специализированных технологий}
                \scnitem{Технология проектирования интеллектуальных обучающих ostis-систем для школьников}
                \scnitem{Технология проектирования интеллектуальных обучающих ostis-систем для студентов по общеобразовательным дисциплинам}
                \scnitem{Технология проектирования интеллектуальных обучающих ostis-систем для студентов по профильным дисциплинам}
                \scnitem{Технология проектирования интеллектуальных обучающих ostis-систем для магистрантов}
            \end{scnrelfromset}
            \begin{scnrelfromset}{комплекс специализированных технологий}
                \scnitem{Технология проектирования интеллектуальных обучающих ostis-систем по Математике}
                \scnitem{Технология проектирования интеллектуальных обучающих ostis-систем по Искусственному интеллекту}
            \end{scnrelfromset}
		\end{scnindent}
    \end{scnrelfromlist}
    
    \scnheader{специализированная ostis-технология}
    \scntext{примечание}{Приведённый нами перечень \textit{специализированных ostis-технологий} охватывает только некоторые области (фрагменты) \textit{человеческой деятельности}, подлежащие автоматизации с помощью \textit{ostis-технологий} в рамках \textit{Экосистемы OSTIS}.}
    
    \scnheader{Ядро Технологии OSTIS}
    \scntext{примечание}{Форма реализации \textit{Ядра Технологии OSTIS} (в виде ostis-системы \textit{Метасистемы OSTIS}) позволяет:
        \begin{itemize}
            \item использовать достоинства \textit{Технологии OSTIS} для повышения уровня автоматизации развития самой \textit{Технологии OSTIS} и для существенного повышения темпов такого развития;
            \item приобрести очень важный опыт применения \textit{Технологии OSTIS}
            \item создать центрально ядро \textit{Экосистемы OSTIS}, обеспечивающее поддержку семантической совместимости всех \textit{ostis-систем} и \textit{ostis-сообществ}, входящих в состав \textit{Экосистемы OSTIS}.
        \end{itemize}}

    \scnheader{Технология OSTIS}
    \scntext{пояснение}{\textit{Технология OSTIS} рассматривается нами как один из вариантов комплексного решения всех перечисленных выше сверхзадач, которые направлены на развитие деятельности в области искусственного интеллекта и которые, очевидно, сильно связаны друг с другом. Таким образом, \textit{Технология OSTIS} включает в себя:
        \begin{itemize}
            \item и постоянно развивающуюся общую формальную теорию интеллектуальных компьютерных систем, представленную в виде базы знаний соответствующего портала научно-технических знаний;
            \item и постоянно развивающийся комплекс моделей, методов и средств, используемых при проектировании интеллектуальных компьютерных систем и оформленных в виде интеллектуальной системы информационной и инструментальной поддержки (автоматизации) проектирования семантически совместимых интеллектуальных компьютерных систем;
            \item и постоянно развивающуюся глобальную экосистему, состоящую из семантически совместимых взаимодействующих интеллектуальных компьютерных систем, ориентированных на комплексную автоматизацию всевозможных видов человеческой деятельности;
        \end{itemize}}
    \scntext{пояснение}{Целью создания \textit{Технологии OSTIS} является не только построение методики, обеспечивающей четкую организацию коллективной \textit{человеческой деятельности} по проектированию, производству, эксплуатации и реинжинирингу \textit{интеллектуальных компьютерных систем}, но и построение мощных средств автоматизации (компьютерной поддержки) этой деятельности. Здесь важно подчеркнуть то, что \textit{интеллектуальные компьютерные системы}, разрабатываемые с помощью \textit{Технологии OSTIS (ostis-системы)} могут быть использованы для автоматизации \uline{любых} видов \textit{человеческой деятельности} и, в том числе, для автоматизации коллективной \textit{человеческой деятельности} по проектированию, производству, эксплуатации и реинжинирингу \textit{ostis-систем}. В рамках \textit{Технологии OSTIS} так и происходит --- автоматизация проектирования, производства, эксплуатации и реинжиниринга \textit{ostis-систем} осуществляется с помощью специально предназначенных для этого \textit{ostis-систем}, некоторые из которых (например, для поддержки эксплуатации и реинжиниринга \textit{ostis-систем}) являются \textit{ostis-системами}, встроенными (интегрированными) в те \textit{ostis-системы}, поддержку эксплуатации и реинжиниринга которых они осуществляют.}
    \scnidtf{Комплексная технология, обеспечивающая автоматизацию самых различных действий (в том числе, и всевозможных видов человеческой деятельности) на основе семантически совместимых интеллектуальных компьютерных систем, способных координировать (согласовывать) свои действия как с себе подобными, так и с людьми}
    \scnidtf{Сумма (интеграция) всевозможных \textit{ostis-технологий}}
    \begin{scnrelfromlist}{достоинство}
        \scnfileitem{\textit{Технология OSTIS} представляет собой принципиально новый уровень развития \textit{информационных технологий}, в основе которого лежит переход от (from) data science к (to) knowledge science}
        \scnfileitem{Открытый характер \textit{Технологии OSTIS} как для тех, кто желает участвовать в её развитии, так и для пользователей \textit{Технологии OSTIS} --- для разработчиков прикладных \textit{ostis-систем}}
        \scnfileitem{Низкий порог вхождения для желающих развивать и желающих использовать имеющиеся в текущий момент методы и средства \textit{Технологии OSTIS}, что обеспечивается поддержкой качественного состояния документации по текущей версии \textit{Технологии OSTIS} с дополнительным описанием эволюции (развития) Технологии OSTIS, а также плана дальнейшего её развития}
        \scnfileitem{Децентрализованный характер управления проектами разработки \textit{ostis-систем}, основанный на четком согласовании коллективом разработчиков проектных задач}
        \scnfileitem{Ориентация на новое поколение компьютеров, без появления которых дальнейшее развитие \textit{технологий искусственного интеллекта} невозможно. При этом \textit{Технология OSTIS} позволяет достаточно конструктивно сформулировать требования, предъявляемые к таким компьютерам}
        \scnfileitem{\textit{Технология OSTIS} не только обеспечивает автоматизацию широкого многообразия видов человеческой деятельности, но и существенно повышает уровень (качество) этой автоматизации, благодаря (1) широкому применению методов и средств \textit{искусственного интеллекта} и (2) создание условий для \textit{конвергенции}, семантической совместимости и \textit{глубокой интеграции} как автоматизируемых видов человеческой деятельности, так и продуктов этой деятельности. В частности, это касается и автоматизации человеческой деятельности в области \textit{искусственного интеллекта}. \textit{Технология OSTIS} рассматривается как предлагаемый подход к конвергенции и интеграции как различных видов деятельности в области искусственного интеллекта, так и результатов этой деятельности (частных теорий различных компонентов и различных видов интеллектуальных систем, частных методов и средств проектирования различных видов и различных компонентов интеллектуальных компьютерных систем)}
        \scnfileitem{Ориентация на разработку компьютерных систем и коллективов таких систем, имеющих высокий уровень \textit{интеллекта} Ориентация на разработку глобальной сети \textit{интеллектуальных компьютерных систем}, обеспечивающей комплексную автоматизацию всех видов и областей \textit{человеческой деятельности}}
        \scnfileitem{Создание условий для формирования \textit{рынка знаний} на основе иерархической системы семантически совместимых \textit{порталов знаний}, соответствующих самым различным областям и \textit{видам человеческой деятельности}}
        \scnfileitem{Создание условий для перехода от традиционной формы публикации статей, монографий, отчетов и прочих документов к их публикации как фрагментов \textit{баз знаний} соответствующих \textit{порталов знаний}, что полностью исключает дублирование информации в публикуемых документах и обеспечивает непосредственное использование этой информации в \textit{интеллектуальных компьютерных системах}}
    \end{scnrelfromlist}
    \scntext{ближайшая задача}{Обеспечить низкий порог входа в Технологию OSTIS:
        \begin{scnitemize}
            \item для желающих участвовать в развитии \textit{Технологии OSTIS}, т.е. в совершенствовании \textit{Метасистемы OSTIS} (системы информационной поддержки и автоматизации проектирования \textit{ostis-систем}), которая сама также является \textit{ostis-системой}
            \item для разработчиков \textit{интеллектуальных компьютерных систем}, желающих использовать для этого \textit{Технологию OSTIS} (эти разработчики являются конечными пользователями \textit{Метасистемы OSTIS});
            \item для конечных пользователей всевозможных иных \textit{ostis-систем}, т.е. компьютерных систем, разработанных по \textit{Технологии OSTIS} с непосредственным использованием в качестве инструмента \textit{Метасистемы OSTIS} (подчеркнем при этом, что базовы принципы организации взаимодействия \textit{Метасистемы OSTIS} с конечными пользователями полностью совпадают с базовыми принципами организации взаимодействия всех остальных \textit{ostis-систем}, разработанных с помощью \textit{Метасистемы OSTIS}, со своими конечными пользователями. Это обусловлено тем, что \textit{Метасистема OSTIS} сама также является \textit{ostis-системой} --- материнской \textit{ostis-системой}).
        \end{scnitemize}}
    \scntext{примечание}{Для решения указанной задачи необходимо создать инфраструктуру коллективного перманентного обновления (совершенствования) комплексной документации по \textit{Технологии OSTIS}, которая:
        \begin{scnitemize}
            \item обеспечила бы достаточную полноту и четкость фиксации текущего состояния \textit{Технологии OSTIS} и удовлетворяла бы как разработчиков \textit{Технологии OSTIS} (т.е. разработчиков \textit{Метасистемы OSTIS}), так и разработчиков \textit{ostis-систем}, не являющихся \textit{Метасистемой OSTIS} (т.е. конечных пользователей \textit{Метасистемы OSTIS}), и также конечных пользователей любых \textit{ostis-систем}
            \item обеспечила бы высокие темпы совершенствования данной документации на основании (1) четких правил согласования и утверждения различного рода предложений, (2) максимально возможной автоматизации процессов анализа, согласования и утверждения указанных предложений, (3) постоянного расширения числа авторов и (4) четких правил защиты авторских прав;
            \item обеспечила бы четкую фиксацию границ между текущим состоянием \textit{Технологии OSTIS} и разрабатываемыми, тестируемыми фрагментами её будущих версий с обоснованием таких нововведений и с планом их включения в соответствующую версию \textit{Технологии OSTIS}
            \item обеспечила бы четкую семантическую интеграцию документации той части \textit{Технологии OSTIS}, которая касается проектирования семантических моделей \textit{ostis-систем} и которая фактически сводится к проектированию \textit{баз знаний ostis-систем}, а также документации той части \textit{Технологии OSTIS}, которая описывает различные варианты программной или аппаратной реализации универсального интерпретатора логико-семантических моделей \textit{ostis-систем}. Подчеркнем при этом, что универсальность используемого в \textit{Технологии OSTIS} языка представления знаний дает возможность описывать на нем все, что угодно, в том числе и интерпретаторы семантических моделей \textit{ostis-систем}. Но делать это нужно с разумной степенью детализации.
        \end{scnitemize}}
    \scntext{ближайшая задача}{Осуществить конвергенцию и интеграцию всевозможных частных технологий проектирования и реализации различных видов компонентов интеллектуальных компонентов систем (в частности, баз знаний, различного вида логических моделей, искусственных нейронных сетей и т.п.)}
    \begin{scnrelfromlist}{класс создаваемых продуктов}
        \scnitem{ostis-система}
        \begin{scnindent}
            \scnidtf{индивидуальная ostis-система}
            \scntext{примечание}{Существенно подчеркнуть, что \textit{Технология OSTIS} порождает не просто множество \textit{ostis-систем}, а множество семантически совместимых и взаимодействующих \textit{ostis-систем}, образующих экосистему, которую будем называть \textit{Экосистемой OSTIS} (Экосистемой ostis-систем и их пользователей). Таким образом, можно считать что интегрированным продуктом \textit{Технологии OSTIS} является не множество ostis-систем, а  система (экосистема) \textit{ostis-систем}.}
        \end{scnindent}
        \scnitem{коллектив ostis-систем}
        \begin{scnindent}
            \scnsuperset{простой коллектив ostis-систем}
            \begin{scnindent}
            	\scnidtf{коллектив ostis-систем, членами которого являются только индивидуальные \mbox{ostis-системы}}
            \end{scnindent}
            \scnsuperset{иерархический коллектив ostis-систем}
            \begin{scnindent}
            	\scnidtf{коллектив ostis-систем, по крайней мере одним членом которого является коллектив ostis-систем}
           	\end{scnindent}
        \end{scnindent}
        \scnitem{ostis-сообщество}
        \begin{scnindent}
            \scnsuperset{простое ostis-сообщество}
            \scnsuperset{иерархическое ostis-сообщество}
            \begin{scnindent}
            	\scnhaselement{Экосистема OSTIS}
            \end{scnindent}
        \end{scnindent}
    \end{scnrelfromlist}
    \begin{scnrelfromlist}{основной продукт}
        \scnitem{\textit{Экосистема OSTIS}}
        \begin{scnindent}
            \scnidtf{Максимальное \textit{ostis-сообщество}, направленное на автоматизацию всех видов человеческой деятельности}
        \end{scnindent}
        \scnitem{\textit{Консорциум OSTIS}}
        \begin{scnindent}
            \scnidtf{\textit{ostis-сообщество}, направленное на развитие \textit{Технологии OSTIS}}
        \end{scnindent}
        \scnitem{\textit{Метасистема OSTIS}}
        \begin{scnindent}
            \scnidtf{Метасистема, являющаяся
                \begin{scnitemize}
                    \item \textit{корпоративной ostis-системой}, обеспечивающей организацию (координацию) деятельности \textit{Консорциума OSTIS}
                    \item формой представления реализации и фиксации текущего состояния \textit{Ядра Технологии OSTIS}
                    \item корпоративной \textit{ostis-системой}, взаимодействующей со всеми корпоративными \mbox{ostis-системами}, каждая из которых координирует развитие соответствующей \textit{специализированной ostis-технологии}.
                \end{scnitemize}}
       \end{scnindent}
    \end{scnrelfromlist}
    
    \scnheader{Экосистема OSTIS}
    \scntext{примечание}{Важной особенностью и достоинством \textit{Технологии OSTIS} является то, что все остальные продукты её использования (конкретные \textit{ostis-системы}) объединяются в сеть, т.е. становятся единым целостным продуктом использования \textit{Технологии OSTIS} --- \textit{Экосистема OSTIS}}
    \scntext{пояснение}{Социально-техническая сеть, состоящая из людей и \textit{ostis-систем}, которые являются
        \begin{scnitemize}
            \item семантически совместимыми;
            \item постоянно эволюционирующими индивидуально;
            \item постоянно поддерживающими свою совместимость с другими агентами в ходе своей индивидуальной эволюции;
            \item способными децентрализованно координировать свою деятельность.
        \end{scnitemize}}
    
    \scnheader{Технология OSTIS}
    \begin{scnrelfromset}{решаемая проблема}
        \scnitem{обеспечение семантической совместимости \uline{разрабатываемых} компьютерных систем}
        \begin{scnindent}
            \begin{scnrelfromset}{подход к решению}
                \scnitem{применение смыслового представления информации в памяти компьютерных систем}
                \scnitem{согласование и унификация системы используемых понятий и соответствующей иерархической системы формальных онтологий}
            \end{scnrelfromset}
        \end{scnindent}
        \scnitem{обеспечение \uline{поддержки} семантической совместимости компьютерных систем в ходе их эксплуатации и эволюции}
        \begin{scnindent}
            \scnrelfrom{подход к решению}{создание самоорганизованной экосистемы компьтерных систем}
        \end{scnindent}
    \end{scnrelfromset}

    \bigskip
   
\end{scnsubstruct}
\scnsourcecommentinline{Завершили рассмотрение \textit{понятия ostis-технологии}}

        \scnsegmentheader{Перспективы использования Технологии OSTIS для повышения качества человеческой деятельности в области Искусственного интеллекта}
\begin{scnsubstruct}
    \begin{scnrelfromlist}{рассматриваемые вопросы}
        \scnfileitem{Могут ли \uline{практические} результаты работ в области \textit{Искусственного интеллекта}, существенно повысить эффективность развития \textit{Искусственного интеллекта} как научно-технической дисциплины, включая \uline{все формы} деятельности в области \textit{Искусственного интеллекта}}
        \scnfileitem{Какова перспектива использования \textit{Технологии OSTIS} для автоматизации других областей и видов \textit{человеческой деятельности}}
    \end{scnrelfromlist}
    \bigskip
    
    \scnheader{Научно-исследовательская деятельность в области Искусственного интеллекта}
    \scnrelfrom{субъект деятельности}{OSTIS-сообщество научно-исследовательской деятельности в области Искусственного интеллекта}
    \begin{scnindent}
    	\scntext{примечание}{Имеются ввиду специалисты разных стран и разных направлений \textit{Искусственного интеллекта}}
    \end{scnindent}
    \begin{scnrelfromset}{направления деятельности}
        \scnitem{Конвергенция и интеграция различных направлений Искусственного интеллекта}
        \scnitem{Конвергенция Искусственного интеллекта как отдельной научно-технической дисциплины с другими смежными научными дисциплинами}
        \begin{scnindent}
            \scntext{примечание}{Конвергенция с математикой, кибернетикой, информатикой, общей теорией систем, психологией, семиотикой, лингвистикой, гносеологией, логикой, методологией и др.}
        \end{scnindent}
        \scnitem{Разработка Общей теории интеллектуальных систем}
        \begin{scnindent}
            \scntext{примечание}{Речь идет как о естественных, так и об искусственных \textit{интеллектуальных системах}.}
        \end{scnindent}
    \end{scnrelfromset}
    \scnrelfrom{средство автоматизации}{OSTIS-портал научных знаний в области Искусственного интеллекта}
    \scnrelfrom{технология}{OSTIS-технология организации коллективной научно-теоретической деятельности }
    \begin{scnindent}
	    \scnrelto{частная технология}{Технология \textbf{\textit{реинжиниринга}} ostis-систем}
	    \begin{scnindent}
		    \scnidtf{Технология коллективного \textbf{\textit{реинжиниринга}} \textit{баз знаний ostis-систем}, обеспечивающая конвергенцию, интеграцию и согласование различных точек зрения и реализуемая абсолютно одинаковыми \textit{ostis-системами}, которые встраиваются (интегрируются) в состав каждой \textit{ostis-системы}}
		    \scnrelfrom{реализация}{Встраиваемая ostis-система поддержки реижиниринга ostis-систем}
	    \end{scnindent}
    \end{scnindent}
    \scnrelfrom{продукт}{Общая формальная теория интеллектуальных систем}
    \begin{scnindent}
    	\scntext{примечание}{В рамках \textit{Технологии OSTIS} \textit{Общая формальная теория интеллектуальных систем} представляется в виде \textit{базы знаний} \textit{OSTIS-портала научных знаний в области Искусственного интеллекта}.}
    \end{scnindent}
    
    \scnheader{Общая формальная теория интеллектуальных систем}
    \scntext{примечание}{Зачем нужна \textit{Общая теория интеллектуальных систем}?
        \\Очевидно, что без этой теории невозможно построить набор методов и средств, обеспечивающий комплексную поддержку разработки \textit{интеллектуальных компьютерных систем} различного назначения и с различным набором навыков (способностей, возможностей), которыми могут обладать \textit{интеллектуальные компьютерные системы}, но необязательно каждая из них.При этом важно не просто построить \textit{Общую теорию интеллектуальных систем} и довести ее до строгого формального уровня, но также довести представление такой формальной теории до уровня базы знаний соответствующего \textit{портала научных знаний}.}
        
    \scnheader{OSTIS-технология организации коллективной научно-теоретической деятельности}
    \scntext{пояснение}{Подчеркнем, что \textit{Технология OSTIS} создает достаточно удобные (конструктивные) условия для решения таких проблем, как
        \begin{scnitemize}
            \item cогласование систем понятий разных научных дисциплин (в частности, разных направлений \textit{Искуственного интеллекта}) и, как следствие, возможность реализациидостаточно качественной семантической совместимости;
            \item конвергенция разных научных дисциплин, важным механизмом которой является увеличение числа общих понятий, используемых этими дисциплинами (в частности, этого можно добиться путем введения таких понятий, каждое из которых является обобщением, например, двух понятий, одно из которых относится к одной дисциплине, а другое --- к другой)
            \item интеграция научных дисциплин.
        \end{scnitemize}
        Удобство (конструктивность, формализованность) решения указанных проблем обусловлено тем, что каждая научная дисциплина представляется постоянно развивающейся базой знаний, которая в \textit{Технологии OSTIS} представляется в виде специальнойсемантической сети (в виде текста \textit{SC-кода}), которой соответствуют достаточно простые синтаксические и семантические правила.Важной проблемой организации научно-теоретической деятельностиявляется реализация эффективной процедуры согласования различных точек зрения и обеспечения их конвергенции и глубокой (бесшовной) интеграции. В рамках коллективного развития базы знаний портала научных знаний можно обеспечить:
        \begin{scnitemize}
            \item существенное сокращение времени, затрачиваемого на согласование используемых понятий;
            \item существенное повышение эффективности рецензирования самых различных предложений;
            \item существенное сокращение времени, затрачиваемого на публикацию научных результатов, так как меняется форма публикаций-публикации. Эти результаты оформляются в смысловом виде как фрагменты соответствующей базы знаний, что предполагает отсутствие дублированиянаучных текстов, т.е. отсутствие возможности представления одного и того же результата во многих формах в разных статьях и монографиях;
            \item автоматизацию анализа качества новых знаний, предлагаемых в состав совершенствуемой базы знаний;
            \item автоматизацию мониторинга общего качества всей базы знаний.
        \end{scnitemize}
        Очевидно, что качество накапливаемых человечеством \textit{научно-технических знаний} во многом определяет качество (уровень развития) всего человечества как коллектива интеллектуальных систем. Качество указанных знаний определяется
        \begin{scnitemize}
            \item трудоемкостью их накопления и систематизации
            \item уровнем конвергенции различных научно-технических дисциплин (уровнем целостности всего комплекса знаний)
            \item четкостью фиксации текущего (согласованного) состояния накопленных знаний
            \item четкостью фиксации истории эволюции накапливаемых знаний
            \item трудоемкостью согласования различных техн ..
            \item четкостью фиксации противоречий и разногласий
        \end{scnitemize}}
    
    \scnheader{язык научно-технической информации}
    \scntext{примечание}{Современные научно-технические тексты не являются естественно-языковыми --- это смесь формальных и естественно языковых текстов. Но именно в таком виде осуществляется накопление человеческих знаний в Internete. Необходим универсальный формальный язык, в который был бы в достаточной степени удобным (привычным) и для человека, и для интеллектуальных компьютерных систем. Перспективным подходом к решению этой проблемы является разработка универсального языка семантических сетей.\\
        Переход к \textit{интегрированному} семантическому пространству научно-технических знаний, в основе которой лежит интернациональная семантическая формализация знаний.\\
        Например, научно техническая статья должна быть \textit{завершенной} нек матем знаний, возволяющей, например, полностью автоматизировать анализ (рецензирование, верификацию) поступившей статьи (например верификацию локазательства теорем).\\
        А научно-технический журнал превращается в портал знаний по заданной научно-технической области.\\
        Перманентное развитие баз знаний такого портала становится открытым проектоом, в рамках которого
        \begin{scnitemize}
            \item каждый желающий может высказать свое предложение
            \item каждый может высказать \uline{замечание} по поводу какого-либо предложения (быть рецензентом)
            \item внести исправления в свое предложение (по замечаниям)
            \item признать исправления своих замечаний
            \item проголосовать за/против предложение или исправленное предложение
            \item признание каждого предложения осуществляется \uline{автоматически} на основании высказанных мнений с учетом квалификации каждого высказывшегося по отношению к соответствующему разделу базы знаний.
            \item текущая квалификация каждого специалиста постоянно уточняется (повышается) на основании вклада этого специалиста в развитие базы знаний портала
            \item вклад любого сепциалиста персонифицируется (защита интеллектуальной собственности) и его ценность (значимость) автоматически оценивается (на основе анализа того, как используется знания, автором или рецензентом --- соавтором которых специалист является, в коммерческих проектах!! При этом ссылка на использованные знания является обязательной)
        \end{scnitemize}}
    
    \scnheader{OSTIS-портал научных знаний в области Искусственного интеллекта}
    \scntext{в перспективе}{В перспективе каждый \textit{ostis-портал научных знаний} может преобразоватьсяв сеть семантически совместимых \textit{ostis-порталов научных знаний}, соответствующих различным направлениям заданной \textit{научной дисциплиной} (например, различным направлениям \textit{Искусственного интеллекта})}

    \scnheader{Коллектив специалистов в области Искусственного интеллекта}
    \scntext{в перспективе}{\textit{OSTIS-сообщество} субъектов \textit{научно-исследовательской деятельности} в области \textit{Искусственного интеллекта}}
    \begin{scnindent}
	    \scntext{уточнение}{указанными субъектами являются объединенные в сеть специалисты в области \textit{Искусственного интеллекта}, неформальные группы таких специалистов, организации или подразделения различных организаций, работающих в области \textit{Искусственного интеллекта} и \textit{ostis-порталы научных знаний} вобласти Искусственного интеллекта}
	    \scnrelto{часть}{Экосистема OSTIS}
    \end{scnindent}

    \scnheader{Разработка Базовой Комплексной технологии проектирования интеллектуальных компьютерных систем}
    \scnrelfrom{субъект деятельности}{Коллектив разработчиков Базовой Комплексной технологии проектирования интеллектуальных компьютерных систем}
    \begin{scnindent}
	    \scntext{примечание}{Речь идет об открытом проекте разработки указанной технологии и, соответственно, об открытом международном коллективе разработчиков, формируемом на добровольной основе}
	    \begin{scnrelfromset}{направления деятельности}
	        \scnitem{Разработка Общей теории интеллектуальных компьютерных систем}
	        \begin{scnindent}
	            \scntext{примечание}{Речь идет об \uline{искусственных} (компьютерных) интеллектуальных системах и о разработке \uline{стандарта} таких технологических систем.}
	        \end{scnindent}
	        \scnitem{Разработка Теории проектирования интеллектуальных компьютерных систем}
	        \begin{scnindent}
	            \scntext{примечание}{Имеются в виду интеллектуальные компьютерные системы, соответствующие стандарту, разработанному в виде общей теории таких систем, имеется в виду рассмотрение самого процесса проектирования таких систем, т.е. рассмотрение методов их проектирования и проектных библиотек.}
	        \end{scnindent}
	        \scnitem{Разработка комплекса средств автоматизации проектированияинтеллектуальных компьютерных систем}
	        \begin{scnindent}
	            \scntext{примечание}{Данные средства автоматизации проектирования (средства решения проектных задач) при их реализации с помощью \textit{Технологии OSTIS} входят в состав решателя задач \textit{Метасистемы OSTIS}.}
	        \end{scnindent}
	        \scnitem{Конвергенция и интеграция различного вида знаний, хранимых в памяти проектируемых интеллектуальных компьютерных систем}
	        \scnitem{Конвергенция и интеграция различных моделей решения задач,используемых проектируемыми интеллектуальными компьютернымисистемами}
	    \end{scnrelfromset}
    \end{scnindent}
    \scnrelfrom{предлагаемый подход}{\textbf{Проект Метасистемы OSTIS}}
    \begin{scnindent}
    \scnidtf{Разработка Базовой Комплексной оstis-технологии проектирования оstis-систем}
    \scnidtf{Разработка Базовой Комплексной технологии проектирования оstis-систем с помощью специально предназначенной для этого \textit{оstis-системы}, котораяназвана нами \textit{Метасистемой OSTIS}}
    \scnidtf{Проект разработки \textit{Метасистемы OSTIS}}
    \begin{scnrelfromset}{принципы, лежащие в основе}
        \scnfileitem{Речь идет о проектировании не просто интеллектуальных компьютерных систем, а \mbox{ostis-систем}, в виде которых можно построить любую интеллектуальную компьютерную систему. Соблюдение этого принципа является важнейшейцелью эволюции \textit{Технологии ОSTIS}}
        \scnfileitem{Система автоматизации проектирования ostis-систем реализуется также в виде ostis-системы --- \textit{Метасистемы OSTIS}}
        \scnfileitem{Эволюция технологии проектирования ostis-систем сводится к эволюции (реинжинирингу) базы знаний \textit{Метасистемы OSTIS}.}
    \end{scnrelfromset}
    \scntext{примечание}{Если речь вести о \textit{Технологии ОSTIS}, то следует говорить не только о самой данной технологии, но и о проекте, направленном на создание и перманентное совершенствование этой технологии, так как важнейшей особенностью и достоинством \textit{Технологии ОSTIS} являются высокие темпы ее эволюции. Указанное достоинство обеспечивается прежде всего тем, что Технология ОSTIS реализуется в виде ostis-системы (\textit{Метасистемы OSTIS}).}
    \scnrelfrom{средство автоматизации}{Метасистема OSTIS}
    \begin{scnindent}
	    \scnidtf{ОSTIS-система автоматизации комплексного проектирования ostis-систем}
	    \scntext{примечание}{При \textit{Разработке Базовой Комплексной технологии проектирования интеллектуальных компьютерных систем} (точнее ostis-систем) средством автоматизации этой деятельности является не вся \textit{Метасистема OSTIS}, а только ее часть --- входящая в состав \textit{Метасистемы OSTIS} типовая \textit{Встраиваемая ostis-система поддержки реижиниринга ostis-систем}, которая поддерживает деятельность разработчиков базы знаний Метасистемы OSTIS. Это обусловлено тем, что вся деятельность по \textit{Разработке Базовой Комплексной технологии проектирования интеллектуальных компьютерных систем} (ostis-систем) сводится к разработке (инженирингу) и обновлению (совершенствованию, реинжинирингу) \textit{Базы знаний Метасистемы OSTIS}).}
    \end{scnindent}
    \scnrelfrom{технология}{Технология реинжиниринга ostis-систем}
    \begin{scnindent}
    	\scnrelfrom{реализация}{Встраиваемая ostis-система поддержки реинжиниринга ostis-систем}
    \end{scnindent}
    \scnrelfrom{продукт}{Комплексная ostis-технология проектирования ostis-систем}
    \begin{scnindent}
    	\scnrelfrom{реализация}{Метасистема OSTIS}
    \end{scnindent}
    \scnidtf{Человеко-машинная деятельность, осуществляемая в рамках \textit{Экосистемы OSTIS} и направленная на разработку и перманентное совершенствование \textit{Метасистемы OSTIS}, которая является формой представления (отображения) (1) текущего состояния \textit{Технологии OSTIS}, как комплекса методов и средств автоматизации (поддержки) разработки\textit{ostis-систем} и (2) текущего состояния самого \textit{Проекта Метасистемы OSTIS}.}
    \scntext{примечание}{Принципы (правила) организации деятельности в рамках \textit{Проекта Метасистемы OSTIS} полностью совпадают с принципами (правилами) организации деятельности в рамках любого другого проекта, направленного на разработку и совершенствование любой другой ostis-системы.}
    \scnrelto{ключевой подпроект}{Проект Экосистемы OSTIS}
    \begin{scnindent}
    	\scnidtf{Совместная деятельность ученых, инженеров и ostis-систем, входящих в \textit{Экосистему OSTIS}, направленная на перманентное совершенствование \textit{Экосистемы OSTIS} --- на совершенствование (реинжиниринг) входящих в неё  \textit{ostis-систем} и на создание новых ostis-систем и их включение в состав \textit{Экосистемы OSTIS.}}
    \end{scnindent}
    \scntext{пояснение}{\textit{ostis-система}, являющаяся:
        \begin{scnitemize}
            \item ostis-порталом научно-технических знаний по \textit{Технологии OSTIS}, база знаний которого включает в себя:
            \begin{scnitemizeii}
                \item формальную теорию \textit{ostis-систем}
                \item формальную теорию (методику) проектирования  \textit{ostis-систем}
                \item формальную спецификацию средств автоматизации проектирования \textit{ostis-систем}
                \item библиотеку проектирования \textit{ostis-систем}
                \item формальную спецификацию средств производства спроектированных \textit{ostis-систем}
            \end{scnitemizeii}
            \item \textit{ostis-системой} автоматизации (поддержки) проектирования \textit{ostis-систем}
            \item \textit{ostis-системой} поддержки производства (сборки, синтеза, генерации) спроектированных \textit{ostis-систем}
            \item \textit{ostis-системой} поддержки реинжиниринга \textit{ostis-систем} в ходе их эксплуатации
        \end{scnitemize}}
    \end{scnindent}

    \scnheader{Метасистема OSTIS}
    \scnidtf{Универсальная базовая (предметно-независимая) ostis-система автоматизации проектирования ostis-систем (любых ostis-систем)}
    \scnrelboth{следует отличать}{специализированная ostis-система автоматизации проектирования ostis-систем}
    \scniselement{ostis-система}
    \scnrelto{корпоративная ostis-система}{Консорциум OSTIS}
    \scnidtf{Интеллектуальная метасистема, построенная по стандартам \textit{технологии OSTIS} и предназначенная (1) для инженеров \textit{ostis-систем} --- для поддержки проектирования. Реализации и обновления (реинжиниринга) \textit{ostis-систем} и для разработчиков \textit{Технологии OSTIS} --- для поддержки коллективной деятельности по развитию стандартов и библиотек \textit{Технологии OSTIS.}}
    \scnrelto{форма реализации}{Технология OSTIS}
    \scnrelto{продукт}{Проект Метасистемы OSTIS}
    \scnidtf{Интеллектуальная Метасистема, являющаяся формой (вариантом) реализации (представления, оформления) \textit{Технологии OSTIS} в виде \textit{ostis-системы}}
    \scntext{примечание}{Тот факт, что Технология OSTIS реализуется в виде ostis-системы, является весьма важным для эволюции Технологии OSTIS, поскольку методы и средства эволюции (перманентного совершенствования) Технологии OSTIS становятся фактически совпадающими с методами и средствами разработки любой (!) ostis-системы на всех этапах их жизненного цикла.
        \\Другими словами, эволюция Технологии OSTIS осуществляется методами и средствами самой этой технологии.}
    \scnidtf{Система комплексной автоматизации (информационной и инструментальной поддержки) проектирования и реализации ostis-систем, которая сама реализована также в виде ostis-системы.}
    \scnidtf{Портал знаний по Технологии OSTIS, интегрированный с САПРом ostis-систем и реализованный в виде ostis-системы.}
    \scniselement{портал научно-технических знаний}
    
    \begin{scnset}
        \scnitem{Метасистема OSTIS}
        \begin{scnindent}
            \scniselement{система автоматизации проектирования}
            \begin{scnindent}
                \scnidtf{CAD-система}
                \begin{scnindent}
                    \scnrelto{аббревиатура}{\scnfilelong{Computer Aided Design system}}
                \end{scnindent}
            \end{scnindent}
            \scniselement{интеллектуальная обучающая система}
        \end{scnindent}
    \end{scnset}
    \scnrelboth{семантическая эквивалентность}{\scnfilelong{Метасистема OSTIS является одновременно и системой автоматизации проектирования ostis-систем, и интеллектуальной системой, обучающей методам  и средствам проектирования ostis-систем.}}
    \begin{scnindent}
    	\scntext{следовательно}{этот факт существенно повышает качество проектирования прикладных ostis-систем, расширяет контингент разработчиков ostis-систем и интегрирует проектную (инженерную) деятельность в области искусственного интеллекта с образовательной деятельностью в этой области.}
    \end{scnindent}
    
    \scnheader{Разработка технологии производства спроектированных интеллектуальных компьютерных систем}
    \scnrelfrom{предлагаемый подход}{Проект разработки универсальных интерпретаторов логико-семантических моделей ostis-систем}
    \scnrelfrom{класс продуктов}{универсальный интерпретатор логико-семантических моделей ostis-систем}
    \begin{scnindent}
        \scnidtf{пустая ostis-система --- ostis-система, на базе которой можно построить любую ostis-систему, если логико-семантическую модель этой системы, загрузить в память указанной выше пустой ostis-системы}
        \scnidtf{Базовый интерпретатор логико-семантических моделей ostis-систем}
        \scnidtf{Интерпретатор Универсальной абстрактной sc-машины}
	\end{scnindent}
	
    \scnheader{Проект разработки универсальных интерпретаторов логико-семантических моделей ostis-систем.}
    \scnidtf{Проект реализации универсальной абстрактной sc-машины}
    \scnrelfrom{альтернативный подпроект}{Проект Программной реализации интерпретаторов Универсальной абстрактной sc-машины}
    \scnrelfrom{альтернативный подпроект}{Проект разработки универсальных sc-компьютеров}
    \scntext{применение}{Подчеркнём, что разные альтернативные варианты реализации универсального интерпретатора логико-семантических моделей ostis-систем (универсальной абстрактной sc-машины) никоим образом не влияет на процесс и результат проектирования ostis-систем, то есть на процесс и результат построения логико-семантических моделей разрабатываемых ostis-систем.\\
        
    Другими словами, принципы представления и структуризации логико-семантических моделей ostis-систем и архитектура универсального интерпретатора этих моделей чётко стратифицированы и, следовательно, могут эволюционировать в достаточной степени независимо друг от друга. Тем не менее некоторая зависимость всё же есть --- согласованная трактовка понятия универсальной sc-машины и согласованная форма (язык) передача логико-семантической модели разрабатываемой ostis-системы из \textit{Метасистемы OSTIS} в пустую ostis-систему.}
    
    
    \scnheader{Универсальная абстрактная sc-машина}
    \scntext{пояснение}{Абстрактная машина, которая задается:
        \begin{scnitemize}
            \item \textit{SC-кодом} --- внутренним языком представления знаний в памяти \textit{ostis-системы}
            \item абстрактной \textit{sc-памятью}, которая уточняет динамику обрабатываемых текстов \textit{SC-кода}
            \item универсальным набором (семейством) \textit{sc-агентов}, осуществляющих обработку текстов \textit{SC-кода}.
        \end{scnitemize}}
    \begin{scnindent}
    	\scntext{примечание}{В основе Универсальной абстрактной \textit{sc-машины} лежит интерпретатор \textit{Языка SCP} --- Базового языка программирования \textit{ostis-систем}.}
    \end{scnindent}
    
    \scnheader{Язык SCP}
    \scnidtf{Базовый язык программирования ostis-систем с его синтаксисом, денотационной семантикой и операционной семантикой.}
    \scnidtf{Язык программирования SCP (Semantic Code Programming)}
    \scnrelfrom{смотрите}{\nameref{sd_scp}}
    
    \scnheader{Проект программной реализации интерпретаторов Универсальной абстрактной sc-машины.}
    \scnidtf{Проект разработки программной реализации Универсальной абстрактной sc-машины на современных компьютерах}
    \scnrelfrom{класс продуктов}{программно реализованный интерпретатор Универсальной абстрактной sc-машины}
    \begin{scnindent}
    	\scnrelfrom{смотрите}{\nameref{sd_program_interp}}
    \end{scnindent}
    
    \scnheader{Проект разработки универсальных sc-компьютеров}
    \scnidtf{Проект разработки аппаратной реализации универсальной абстрактной sc-машины в виде компьютера нового поколения, ориентированного на использование в интеллектуальных компьютерных системах( в нашем случае --- в ostis-системах)}
    \scnrelfrom{класс продуктов}{универсальный sc-компьютер}
    \scnidtf{универсальный ostis-компьютер}
    \scnidtf{cемантический ассоциативный компьютер для ostis-систем}
    \scnidtf{аппаратно реализованный интерпретатор абстрактной sc-машины}
    \begin{scnindent}
    	\scnrelfrom{смотрите}{\nameref{sd_sem_comp}}
    \end{scnindent}
    \scntext{примечание}{Тот факт, что универсальный sc-компьютер, разрабатывается под конкретную технологию проектирование интеллектуальных компьютерных систем (Технологию OSTIS), которая развивается, накапливает опыт разработки и внедрения самых различных прикладных интеллектуальных систем независимо от наличия универсальных sc-компьютеров, имеет принципиальное значение. Опыт создания компьютеров, имеющих принципиально новую архитектуру, показывает, что разработка компьютеров нового поколения без серьезной подготовки технологий их применения, без подготовки соответствующий инфраструктуры приводит к неэффективному использованию результатов разработки и к их быстрому моральному старению.}
    \scntext{примечание}{Разработка универсального sc-компьютера является важнейшим следующим этапом развития технологии OSTIS, который обеспечит существенное повышение производительности (быстродействия) ostis-систем.\\
        Развитие технологий искусственного интеллекта неизбежно приведёт к необходимости создания компьютеров принципиально нового поколения, предназначенных для использования в интеллектуальных компьютерных системах. Поэтому изначально ориентация Технологии OSTIS на компьютеры нового поколения является принципиальной и весьма перспективной особенностью Технологии OSTIS, обеспечивающей её высокую конкурентоспособность.}
    
    \scnheader{Проект разработки универсальных интерпретаторов логико-семантических моделей ostis-систем}
    \scntext{примечание}{При построении любого интерпретатора любой информационной машины (в нашем случае --- абстрактной sc-машины) должны быть чётко полно, а самое главное на формальном языке (в нашем случае --- SC-коде) описано следующее:
        \begin{itemize}
            \item синтаксис, денотационная семантика и операционная семантика интерпретируемой машины (в нашем случае для абстрактной sc-машины это синтаксис и денотационная семантика SC-кода и языка SCP, а также операционная семантика языка SCP);
            \item синтаксис, денотационная семантика и операционная семантика интерпретирующей информационный машины;
            \item соотношение между указанными формальными моделями интерпретируемой информационной машины и интерпретирующей информационной машины, определяющее семантическую и операционную (функциональную) эквивалентность.
        \end{itemize}
        Подчеркнем, что без построения указанной строгой формальной модели соответствия (эквивалентности) интегрируемой и интерпретирующей информационной машины организовать качественную коллективную разработку интерпретаторов сложной информационной машины (например, абстрактной sc-машины) невозможно, так как будет совершаться большое количество поздно обнаруживаемых ошибок.}
    \scntext{примечание}{Разрабатываемые сейчас варианты реализации \textit{универсального интерпретатора логико-семантических моделей ostis-систем} (программный и аппаратный вариант) являются в известной мере привычными объектами проектирования для современных технологий проектирования программных систем и технологий проектирования интегрированных микросхем и их комплексов.\\
        Тем не менее, повышение уровня сложности указанных объектов проектирования и указанных характеристик проектирования требует существенного повышения уровня интеллекта у соответствующих систем автоматизации (поддержки) проектирования). \textit{Технология OSTIS} уже имеет достаточный опыт разработки \textit{ostis-систем автоматизации проектирования} (достаточно указать Метасистему OSTIS, обеспечивающую автоматизацию проектирования ostis-систем). Таким образом для повышения качества разработки \textit{Программной реализации универсальной абстрактной sc-машины} и разработки \textit{универсального sc-компьютера} целесообразно разработать, соответственно, \textit{OSTIS-систему поддержки проектирования сложных программных систем}, а также \textit{OSTIS-систему поддержки проектирования интегральных микросхем и их комплексов}. Здесь речь может идти об интеллектуальных надстройках над существующими средствами автоматизации проектирования и управления проектами.\\
        При проектировании \textit{Программной реализации универсальной абстрактной sc-машины}, а также \textit{универсального sc-компьютера} такая интеллектуальная надстройка абсолютно необходима, поскольку при проектировании указанных объектов необходимо четко отслеживать соответствия между компонентами этих объектов и компонентами интерпретируемой ими \textit{универсальной абстрактной sc-машины}. Актуальность указанной интеллектуальной надстройки обусловлена также тем, что \textit{универсальная абстрактная sc-машина} может корректироваться.\\
        Следует отметить возможную связь между процессом проектирования \textit{Программной реализации универсальной абстрактной sс-машины} и проектированием \textit{универсального sc-компьютера}. Дело в том, что \textit{Программную реализацию универсальной абстрактной sc-машины} можно и нужно рассматривать как программную модель не только интегрируемой \textit{универсальной абстрактной sc-машины}, но и проектируемого \textit{универсального sc-компьютера}. Таким образом, реализацию универсальной sc-машины можно развивать в двух направлениях:
            \begin{itemize}
                \item в направлении повышения её производительности;
                \item в направлении более детальной эмуляции универсального sc-компьютера на уровне взаимодействия всё более мелких компонентов этого компьютера.
            \end{itemize}}
    
    \scnheader{Специализированная инженерия в области Искусственного интеллекта}
    \scnrelfrom{предлагаемый подход}{Специализированная инженерия, осуществляемая на основе Технологии OSTIS}
	\begin{scnindent}
	    \begin{scnrelfromset}{декомпозиция}
	        \scnitem{Разработка ostis-систем автоматизации проектирования различных классов ostis-систем}
	        \begin{scnindent}
	            \scnidtf{Разработка специализированных ostis-технологий}
	            \begin{scnrelfromlist}{часть}
	                \scnitem{Разработка ostis-систем автоматизации проектирования ostis-систем автоматизации проектирования}
	                \begin{scnindent}
	                    \scnidtf{Разработка ostis-технологий проектирования}
	                \end{scnindent}
	                \scnitem{Разработка ostis-систем автоматизации проектирования ostis-систем автоматизации производства}
	                \begin{scnindent}
	                    \scnidtf{Разработка ostis-технологий управления производством}
	                \end{scnindent}
	                \scnitem{Разработка ostis-систем автоматизации проектирования ostis-систем управления транспортными системами}
	                \begin{scnindent}
	                    \scnidtf{Разработка ostis-технологий управления транспортными системами}
	                \end{scnindent}
	                \scnitem{Разработка ostis-систем автоматизации проектирования диагностических ostis-систем}
	                \begin{scnindent}
	                    \scnidtf{Разработка ostis-технологий диагностики (технической, медицинской)}
	                \end{scnindent}
	                \scnitem{Разработка ostis-систем автоматизации проектирования обучающих ostis-систем}
	                \begin{scnindent}
	                    \scnidtf{Разработка ostis-технологий обучения людей}
	                \end{scnindent}
	                \scnitem{Разработка ostis-систем автоматизации проектирования ostis-систем управления умными домами}
	                \begin{scnindent}
	                    \scnidtf{Разработка ostis-технологий управления умными домами}
	                \end{scnindent}
	                \scnitem{Разработка ostis-систем автоматизации проектирования ostis-систем управления умными больницами}
	                \begin{scnindent}
	                    \scnidtf{Разработка ostis-технологий управления умными больницами}
	                \end{scnindent}
	                \scnitem{Разработка ostis-систем автоматизации проектирования ostis-систем управления умными поликлиниками}
	                \begin{scnindent}
	                    \scnidtf{Разработка ostis-технологий управления умными поликлиниками}
	                \end{scnindent}
	                \scnitem{Разработка ostis-систем автоматизации проектирования ostis-систем управления умными городскими районами}
	                \begin{scnindent}
	                    \scnidtf{Разработка ostis-технологий управления умными городскими районами}
	                \end{scnindent}
	                \scnitem{Разработка ostis-систем автоматизации проектирования ostis-систем управления умными городами}
	                \begin{scnindent}
	                    \scnidtf{Разработка ostis-технологий управления умными городами}
	                \end{scnindent}
	            \end{scnrelfromlist}
	        \end{scnindent}
	        \scnitem{Разработка (на основе соответствующих ostis-технологий проектирования) ostis-систем автоматизации проектирования различных классов объектов, не являющихся ostis-системами}
	        \begin{scnindent}
	            \begin{scnrelfromlist}{часть}
	                \scnitem{Разработка семейства ostis-систем автоматизации проектирования различных видов интегральных микросхем}
	                \scnitem{Разработка семейства ostis-систем автоматизации проектирования различных видов автомобилей}
	                \scnitem{Разработка семейства ostis-систем автоматизации проектирования различных видов строительных объектов}
	            \end{scnrelfromlist}
	        \end{scnindent}
	        \scnitem{Разработка ostis-систем автоматизации производства}
	        \begin{scnindent}
	            \scnidtf{Разработка интеллектуальных систем управления производственными предприятиями}
	        \end{scnindent}
	        \scnitem{Разработка ostis-систем управления транспортными средствами}
	        \scnitem{Разработка диагностических ostis-систем}
	        \scnitem{Разработка обучающих ostis-систем}
	        \scnitem{Разработка ostis-систем управления умными домами}
	        \scnitem{Разработка ostis-систем управления умными больницами}
	        \scnitem{Разработка ostis-систем управления умными поликлиниками}
	        \scnitem{Разработка ostis-систем управления умными городскими районами}
	        \scnitem{Разработка ostis-систем управления умными городами}
	    \end{scnrelfromset}
	\end{scnindent}
 
    \scnheader{Образовательная деятельность в области Искусственного интеллекта}
    \scnrelfrom{предлагаемый подход}{Образовательная деятельность в области Искусственного интеллекта, осуществляемая на основе Технологии OSTIS}
	\begin{scnindent}
	    \scniselement{образовательная деятельность}
	    \scniselement{человеческая деятельность, осуществляемая на основе Технологии OSTIS}
	    \begin{scnindent}
	    	\scnidtf{человеческая деятельность, комплексная автоматизация которой осуществляется либо индивидуальной \textit{ostis-системой}, либо \textit{коллективом ostis-систем} (сетью ostis-систем)}
	    \end{scnindent}
	    \scnrelfrom{субъект}{OSTIS-сообщество Образовательной деятельности в области Искусственного интеллекта}
	    \begin{scnindent}
		    \scnidtf{глобальное (максимальное) OSTIS-сообщество, осуществляющее Образовательную деятельность в области Искусственного интеллекта и обеспечивающее активное и взаимовыгодное сотрудничество между всеми заинтересованными в этом субъектами и, в первую очередь, с соответствующими кафедрами различных вузов}
		    \scnrelto{часть}{Экосистема OSTIS}
		    \begin{scnindent}
			    \scnidtf{глобальная сеть ostis-систем вместе с их пользователями}
			    \scnidtf{глобальное ostis-сообщество}
			\end{scnindent}
		    \scniselement{ostis-сообщество}
		    \begin{scnindent}
		    	\scnidtf{локальная сеть \textit{ostis-систем} вместе с их пользователями}
		    \end{scnindent}
		    \scntext{пояснение}{Данное \textit{ostis-сообщество} включает в себя:
		        \begin{scnitemize}
		            \item все кафедры, которые готовят молодых специалистов в области \textit{Искусственного интеллекта} и которые могут входить в состав самых различных вузов;
		            \item все те организации, которые разрабатывают или эксплуатируют интеллектуальные компьютерные системы и которые готовы сотрудничать с вузами для повышения квалификации поступающих к ним молодых специалистов в области \textit{Искусственного интеллекта}
		            \item студентов, магистрантов и аспирантов, обучающихся в области \textit{Искусственного интеллекта} в разных вузах;
		            \item их преподавателей;
		            \item семейство интеллектуальных обучающих ostis-систем по различным дисциплинам (направлениям) Искусственного интеллекта, которые семантически совместимы и тесно связаны с \textit{OSTIS-порталом научных знаний по Искусственному интеллекту} и с \textit{Метасистемой OSTIS}
		            \item \textit{OSTIS-портал научных знаний по Искусственному интеллекту}, осуществляющий поддержку развития Общей теории интеллектуальных систем как естественного, так и искусственного происхождения;
		            \item \textit{Метасистема OSTIS}, осуществляющая поддержку развития \textit{Общей теории интеллектуальных компьютерных систем} (искусственных интеллектуальных систем) и поддержку развития \textit{Базовой универсальной комплексной технологии проектирования интеллектуальных компьютерных систем}
		            \item семейство персональных ostis-ассистентов студентов, магистрантов и аспирантов, обучающихся в области \textit{Искусственного интеллекта}
		            \item семейство персональных ostis-ассистентов преподавателей, осуществляющих подготовку молодых специалистов в области \textit{Искусственного интеллекта}
		            \item семейство кафедральных корпоративных \textit{ostis-систем}, осуществляющих управление учебным процессом на уровне кафедр, обеспечивающих подготовку молодых специалистов в области Искусственного интеллекта. В рамках таких корпоративных \textit{ostis-систем} осуществляется:
		            \begin{scnitemizeii}
		                \item составление кафедрального расписания занятий на следующий семестр и его согласование с расписанием других кафедр этого же вуза;
		                \item распределение учебной нагрузки на очередной семестр и учебный год;
		                \item мониторинг проведения различного вида занятий (лекций, консультаций, семинаров, практических занятий, зачетов/экзаменов);
		                \item мониторинг самостоятельной деятельности обучаемых (курсовых и дипломных проектов, рефератов, диссертаций, тестов и др.);
		                \item фиксация текущего соответствия между учебными дисциплинами и разделами \textit{Общей теории интеллектуальных систем} и \textit{Базовой универсальной комплексной технологии проектирования интеллектуальных компьютерных систем} (речь идет не только о дисциплинах, непосредственно относящихся к \textit{Искусственному интеллекту}, но и о различных общеобразовательных и смежных дисциплинах, таких, как теория познания, методология, иностранные языки, современные компьютерные системы и сети, компьютеры нового поколения, теория алгоритмов и программ, ориентированных на современные компьютеры, семантическая теория алгоритмов и программ, ориентированных на обработку баз знаний и др.). Принципиально важно сформировать у студентов, магистрантов и аспирантов целостную картину проблематики \textit{Искусственного интеллекта} и место \textit{Искусственного интеллекта} в общей Картине Мира. Барьеров между учебными дисциплинами быть не должно.
		            \end{scnitemizeii}
		            \item Корпоративная \textit{ostis-система} OSTIS-сообщества, являющегося субъектом \textit{Образовательной деятельности в области Искусственного интеллекта}. Через эту корпоративную \textit{ostis-систему} осуществляется взаимодействие между всеми членами указанного \textit{\mbox{ostis-сообщества}} и, прежде всего между кафедрами, осуществляющими подготовку молодых специалистов в области \textit{Искусственного интеллекта}.
		        \end{scnitemize}}
        \end{scnindent}
    \end{scnindent}
    \begin{scnrelfromvector}{принципы, лежащие в основе}
        \scnfileitem{Подготовка молодых специалистов в области \textit{Искусственного интеллекта} должна осуществляться путем поэтапного и непосредственного их подключения к реальным коллективным проектам:\\
            \begin{scnitemize}
                \item к развитию базы знаний по \textit{Общей теории интеллектуальных систем}, хранимой в памяти соответствующего интеллектуального портала знаний
                \item к развитию базы знаний по \textit{Общей теории интеллектуальных компьютерных систем}, хранимой в памяти соответствующего интеллектуального портала знаний (в памяти \textit{Метасистемы OSTIS})
                \item к развитию базы знаний по \textit{Базовой комплексной технологии проектирования интеллектуальных компьютерных систем}, хранимой в памяти интеллектуальной компьютерной системы автоматизации проектирования интеллектуальных компьютерных систем (в памяти \textit{Метасистемы OSTIS})
                \item к развитию различных методов и средств проектирования различных компонентов \textit{интеллектуальных компьютерных систем}
                \item к развитию различных специализированных технологий проектирования различных классов интеллектуальных компьютерных систем
                \item к разработке различных прикладных интеллектуальных компьютерных систем на основе развиваемой Базовой (универсальной) комплексной технологии проектирования интеллектуальных компьютерных систем.
            \end{scnitemize}}
        \scnfileitem{Каждый студент и магистрант в процессе обучения привлекается к нескольким разным формам деятельности в области \textit{Искусственного интеллекта} и, в частности, обязательно и к разработке приложений, и к развитию технологий. Специалист, занимающийся автоматизацией какой-либо деятельности должен на себе прочувствовать проблемы и трудности этой автоматизируемой деятельности}
        \scnfileitem{Все студенты, магистранты и преподаватели должны активно участвовать в анализе эффективности своей образовательной деятельности и активно способствовать повышению эффективности и повышению уровня автоматизации этой деятельности с помощью развиваемой технологии проектирования и производства интеллектуальных компьютерных систем. Данный принцип можно условно назвать устранением синдрома сапожника без сапог.}
        \scnfileitem{Результаты самостоятельной работы студентов и магистрантов (лабораторных работ, практических занятий, рефератов, курсовых работ и проектов, дипломных работ и проектов, магистерских диссертаций) должны быть востребованы в тех проектах, к которым они подключены и должны быть доведены до уровня внедрения в эти проекты, т. е. должны быть по соответствующей процедуре согласованы и одобрены. При этом приветствуется и соответствующим образом поощряется любая такого рода инициатива студентов и магистрантов. Указанная востребованность (полезность) результатов самостоятельной работы студентов и магистрантов предполагает то, что отчеты по этим результатам оформляются в формализованном виде --- в виде исходных текстов соответствующих фрагментов баз знаний. При этом указанные результаты могут требовать как весьма высокой квалификации, так и не очень высокой (например, квалификации первокурсника). К таким несложным, но весьма полезным работам относятся:\\
            \begin{scnitemize}
                \item введение в \textit{базы знаний} полезных библиографических ссылок и цитат
                \item сравнительный анализ различных положений, представленных в некоторой разрабатываемой базе знаний
                \item различные пояснения, примечания и комментарии, вводимые в \textit{базу знаний}
                \item спецификация выявленных в разрабатываемой базе знаний ошибок, противоречий, информационных дыр и информационного мусора
                \item примеры, иллюстрирующие различные понятия
                \item упражнения к различным разделам разрабатываемых \textit{баз знаний}, которые особенно актуальны для интеллектуальных компьютерных систем, используемых в учебном процессе (это не только интеллектуальные обучающие системы).
            \end{scnitemize}}
        \scnfileitem{Вклад каждого студента и магистранта в развитие всех проектов, в которых он принимает участие, фиксируется и при подведении итогов по каждому семестру соответствующим образом оценивается. Это своего рода предтеча будущего рынка знаний.}
        \scnfileitem{Учебным пособием по каждой учебной дисциплине должна быть база знаний или некоторый раздел базы знаний некоторой интеллектуальной компьютерной системы. Такой может быть либо интеллектуальная обучающая система, либо, например, \textit{Метасистема OSTIS}. Условием максимально эффективного проведения лекционного занятия является предварительное прочтение студентами или магистрантами материала предстоящей лекции (соответствующего раздела базы знаний). Тогда на лекции можно акцентировать внимание не на изложение материала, опубликованного в виде базы знаний, а на обсуждение непонятных фрагментов этого материала, на обсуждение проблем, касающихся содержания (принципиальных положений) этого материала. Все это формирует культуру взаимопонимания и согласования различных точек зрения, а также способствует повышению качества базы знаний, представляющей материал соответствующей учебной дисциплины.}
        \scnfileitem{Важнейшей задачей подготовки молодых специалистов является формирование у них:\\
            \begin{scnitemize}
                \item высокой математической культуры (культуры формализации)
                \item высокой системной культуры (понимания того, что количество далеко не всегда переходит в ожидаемое качество)
                \item высокого уровня технологической культуры, технологической дисциплины, четкого соблюдения текущих стандартов и способности участвовать в эволюции стандартов
                \item способности работать в наукоемких проектах в составе творческих коллективов с децентрализованным управлением
                \item способности к достижению семантической совместимости (взаимопонимания) со своими коллегами
                \item договороспособности (способности к согласованию различных точек зрения).
            \end{scnitemize}}
        \scnfileitem{Подготовку молодых специалистов в области \textit{Искусственного интеллекта} можно осуществлять с ориентацией на следующие условно выделенные уровни их квалификации:\\
            \begin{scnitemize}
                \item инженерия прикладных \textit{интеллектуальных компьютерных систем} по заданной технологии
                \item инженерия специализированных технологий проектирования различных классов прикладных интеллектуальных компьютерных систем (на основе базовой универсальной комплексной технологии проектирования интеллектуальных компьютерных систем)
                \item инженерия базовой универсальной комплексной технологии проектирования интеллектуальных компьютерных систем
                \item инженерия программных и аппаратных средст, интерпретации логико-семантических моделей интеллектуальных компьютерных систем
                \item инженерия комплексов интеллектуальных компьютерных систем
                \item научно-исследовательская деятельность по развитию \textit{Общей формальной теории интеллектуальных компьютерных систем}.
            \end{scnitemize}}
    \end{scnrelfromvector}

    \scnheader{Бизнес-деятельность в области Искусственного интеллекта}
    \scnrelfrom{предлагаемый подход}{Бизнес-деятельность в области Искусственного интеллекта, осуществляемая на основе \textit{Технологии OSTIS}}
	\begin{scnindent}
		\scnrelfrom{субъект}{OSTIS-сообщество Бизнес-деятельности в области Искусственного интеллекта, осуществляемой на основе \textit{Технологии OSTIS}}
		\begin{scnindent}
		    \scnidtf{Глобальное (максимальное) OSTIS-сообщество, осуществляющее Бизнес-деятельность в области Искусственного интеллекта}
		    \scnrelto{часть}{Экосистема OSTIS}
		    \scniselement{ostis-сообщество}
		    \scntext{пояснение}{Речь идет об ostis-сообществе, которое включает в себя все компании и лаборатории, работающие в области Искусственного интеллекта и желающие на взаимовыгодных условиях сотрудничать в направлении совместного, перманентного и интенсивного развития стандартов, методов и средств комплексного проектирования и производства семантически совместимых и договороспособных интеллектуальных компьютерных систем, способных самостоятельно и целенаправленно взаимодействовать друг с другом. Кроме указанных компаний и лабораторий в состав рассматриваемого ostis-сообщества входят:
		        \begin{scnitemize}
		            \item семейство корпоративных ostis-систем, которые представляют интересы указанных компаний и лабораторий в рамках рассматриваемого ostis-сообщества и которые обеспечивают автоматизацию внутренней деятельности (бизнес-процессов) этих компаний и лабораторий, включая делопроизводство, юридический мониторинг, бухгалтерскую деятельность, административно-хозяйственную деятельность, управление персоналом, управление выполняемыми проектами и т.д.;
		            \item Корпоративная ostis-система OSTIS-сообщества, являющегося субъектом Бизнес-де\-я\-тель\-ности в области Искусственного интеллекта. Через эту корпоративную ostis-систему осуществляется взаимодействие между членами рассматриваемого ostis-сообщества --- между компаниями и лабораториями, работающими в области искусственного интеллекта.
		        \end{scnitemize}}
        \end{scnindent}
    \end{scnindent}
    
    \scnheader{Консорциум OSTIS}
    \scniselement{ostis-сообщество}
    \scntext{пояснение}{Весь комплекс деятельности в области \textit{Искусственного интеллекта} мы декомпозировали на шесть форм (частей). Для каждой из этих форм деятельности создается свое \textit{ostis-сообщество}, каждому из которых, в свою очередь, соответствует своя \textit{корпоративная ostis-система}. \textit{Консорциум OSTIS} объединяет все указанные \textit{ostis-сообщества}, включая в свой состав (в состав \textit{Консорциума OSTIS}) прежде всего все \textit{корпоративные ostis-системы} указанных \textit{ostis-сообществ}. Кроме того, для координации деятельности членов самого \textit{Консорциума OSTIS} создается \textit{Корпоративная ostis-система Консорциума OSTIS}. Напомним, что для каждого \textit{ostis-сообщества}, создается соответствующая ему \textit{корпоративная ostis-система}, являющаяся ключевым членом этого \textit{ostis-сообщества} и осуществляющая координацию всех остальных его членов.}
    \begin{scnrelfromlist}{член ostis-сообщества}
        \scnitem{Корпоративная система Консорциума OSTIS}
    	\begin{scnindent}
            \scnrelto{корпоративная ostis-система}{Консорциум OSTIS}
        	\begin{scnindent}
            	\scnrelto{субъект}{Деятельность в области Искусственного интеллекта, осуществляемая на основе Технологии OSTIS}
            \end{scnindent}
        \end{scnindent}
        \scnitem{OSTIS-портал научных знаний в области Искусственного интеллекта}
        \begin{scnindent}
            \scnrelto{корпоративная ostis-система}{OSTIS-сообщество научно-исследовательской деятельности в области Искусственного интеллекта}
            \begin{scnindent}
            	\scnrelto{субъект}{научно-исследовательская деятельность в области Искусственного интеллекта, осуществляемая на основе Технологии OSTIS}
            \end{scnindent}
        \end{scnindent}
        \scnitem{Метасистема OSTIS}
        \begin{scnindent}
            \scnrelto{корпоративная ostis-система}{OSTIS-сообщество Проекта Метасистемы OSTIS}
            \begin{scnindent}
            	\scnrelto{субъект}{Проект Метасистемы OSTIS}
        	\end{scnindent}
        \end{scnindent}
        \scnitem{Корпоративная система OSTIS-сообщества Проекта разработки универсального интерпретатора логико-семантических моделей ostis-систем}
        \begin{scnindent}
            \scnrelto{корпоративная ostis-система}{Корпоративная ostis-система OSTIS-сообщество Проекта разработки универсального интерпретатора логико-семантических моделей ostis-систем}
            \begin{scnindent}
            	\scnrelto{субъект}{Проект разработки универсального интерпретатора логико-семантических моделей ostis-систем}
        	\end{scnindent}
        \end{scnindent}
        \scnitem{Корпоративная система OSTIS-сообщества специализированной инженерии в области Искусственного интеллекта, осуществляемой на основе Технологии OSTIS}
        \begin{scnindent}
            \scnrelto{корпоративная ostis-система}{OSTIS-сообщество Специализированной инженерии в области Искусственного интеллекта, осуществляемой на основе Технологии OSTIS}
            \begin{scnindent}
            	\scnrelto{субъект}{Специализированная инженерия в области Искусственного интеллекта, осуществляемая на основе Технологии OSTIS}
            \end{scnindent}
        \end{scnindent}
        \scnitem{Корпоративная система OSTIS-сообщества образовательной деятельности в области Искусственного интеллекта, осуществляемой на основе Технологии OSTIS}
        \begin{scnindent}
            \scnrelto{корпоративная ostis-система}{OSTIS-сообщество Образовательной деятельности в области Искусственного интеллекта, осуществляемой на основе Технологии OSTIS}
            \begin{scnindent}
            	\scnrelto{субъект}{Образовательная деятельность в области Искусственного интеллекта, осуществляемая на основе Технологии OSTIS}
            \end{scnindent}
        \end{scnindent}
        \scnitem{Корпоративная система OSTIS-сообщества Бизнес-деятельности в области Искусственного интеллекта, осуществляемой на основе Технологии OSTIS}
        \begin{scnindent}
            \scnrelto{корпоративная ostis-система}{OSTIS-сообщество Бизнес-деятельности в области Искусственного интеллекта, осуществляемой на основе Технологии OSTIS}
            \begin{scnindent}
            	\scnrelto{субъект}{Бизнес-деятельность в области Искусственного интеллекта, осуществляемая на основе Технологии OSTIS}
            \end{scnindent}
        \end{scnindent}
    \end{scnrelfromlist}
    \scntext{примечание}{Конвергенция и интеграция различных форм и направлений деятельности в области \textit{Искусственного интеллекта} должна проходить через каждого персонального члена \textit{Консорциума OSTIS} --- желательно, чтобы большинство из них были одновременно:
        \begin{scnitemize}
            \item и участниками научно-исследовательской деятельности в области \textit{Искусственного интеллекта} (аспирантами, докторантами и т.д.);
            \item и участниками совершенствования (развития) целостного комплекса методов и средств проектирования и реализации \textit{интеллектуальных компьютерных систем}
            \item и разработчиками различных прикладных \textit{интеллектуальных компьютерных систем}
            \item и преподавателями, участвующими в подготовке молодых специалистов в области \textit{Искусственного интеллекта}.
        \end{scnitemize}}
    \scnidtf{OSTIS-сообщество субъектов всех форм и направлений деятельности в области Искусственного интеллекта}
    \scnrelto{часть}{Экосистема OSTIS}
    \scnidtf{Научно-техническое и учебное объединение специалистов и организаций, работающих в области Искусственного интеллекта}
    \scntext{перспективы}{Создание Консорциума OSTIS на основе широкого применения Технологии OSTIS может и должно осуществляться с поэтапным расширением состава участников и поэтапным повышением уровня автоматизации деятельности Консорциума OSTIS. Ключевыми направлениями деятельности Консорциума OSTIS являются:
        \begin{scnitemize}
            \item Существенное повышение темпов эволюции Ядра Технологии OSTIS, темпов перехода на все более совершенные версии стандартов интеллектуальных компьютерных систем, проектных библиотек и средств автоматизации проектирования интеллектуальных компьютерных систем;
            \item Разработка компьютеров нового поколения, ориентированных на интерпретацию логико-семантических моделей интеллектуальных компьютерных систем;
            \item Разработка иерархического семейства семантически совместимых специализированных технологий проектирования различных классов интеллектуальных компьютерных систем;
            \item Создание условий для развития технологий искусственного интеллекта в направлении унификации интеллектуальных компьютерных систем для обеспечения их конвергенции и семантической совместимости.
        \end{scnitemize}}
    \bigskip
\end{scnsubstruct}

        \scnsegmentheader{Уточнение Понятия Экосистемы OSTIS}
\begin{scnsubstruct}
    \scnidtf{Использование \textit{Технологии OSTIS} для повышения качества и, в частности, уровня автоматизации всех \textit{областей человеческой деятельности}}
    \scnidtf{Понятие \textit{Экосистемы OSTIS} как формы реализации \textit{smart-общества}, представляющего собой сеть взаимодействующих людей, интеллектуальных компьютерных систем, умных домов, умных предприятий, умных больниц, умных учебных заведений, умных городов, умных транспортных систем и т.п.}
    \begin{scnrelfromset}{рассматриваемые вопросы}
        \scnfileitem{Какова архитектура \textit{Экосистемы OSTIS}}
        \scnfileitem{Какова архитектура \textit{ostis-сообщества}, входящего в состав \textit{Экосистемы OSTIS}}
        \scnfileitem{Как взаимодействуют между собой различные \textit{ostis-сообщества} в рамках \textit{Экосистемы OSTIS}}
        \scnfileitem{Как интегрируется \textit{деятельность} различных \textit{ostis-сообществ} и результаты этой \textit{деятельности}}
        \scnfileitem{Какова типология \textit{ostis-сообществ} и по каким признакам классификации можно эту типологию проводить}
        \scnfileitem{Можно ли опыт автоматизации деятельности в области \textit{Искусственного интеллекта} с помощью \textit{Технологии OSTIS} расширить на все многообразие областей и видов человеческой деятельности}
        \scnfileitem{Как выглядит систематизация областей и видов человеческой деятельности}
        \scnfileitem{Как осуществляется конвергенция и интеграция различных областей и видов человеческой деятельности}
        \scnfileitem{Как взаимодействуют \textit{ostis-системы}, осуществляющие автоматизацию различных областей видов человеческой деятельности}
        \scnfileitem{Как может выглядеть \uline{комплексная} автоматизация всех областей и видов \textit{человеческой деятельности} с помощью \textit{Технологии OSTIS}}
    \end{scnrelfromset}
    \begin{scnrelfromvector}{план изложения}
        \scnfileitem{Что такое \textit{Экосистема OSTIS}}
        \scnfileitem{Структура \textit{Экосистемы OSTIS}}
        \scnfileitem{Что такое \textit{ostis-система}, являющаяся агентом \textit{Экосистемы OSTIS}}
        \scnfileitem{Что такое \textit{ostis-сообщество} с точки зрения \textit{Экосистемы OSTIS}}
        \scnfileitem{Что такое Проект создания \textit{Экосистемы OSTIS}}
        \scnfileitem{Цель создания и основные свойства \textit{Экосистемы OSTIS}}
        \scnfileitem{Как структурируется \textit{человеческая деятельность}}
        \scnfileitem{Как выглядит \textit{рынок знаний}, реализуемый в рамках \textit{Экосистемы OSTIS}}
        \scnfileitem{Чем определяется качество \textit{человеческой деятельности}}
        \scnfileitem{Что такое эффективная автоматизация \textit{человеческой деятельности}}
        \scnfileitem{Почему повышение эффективности \textit{человеческой деятельности} невозможно без \textit{интеллектуальных компьютерных систем}}
        \scnfileitem{Какие достоинства имеет \textit{Экосистема OSTIS}}
    \end{scnrelfromvector}

    \scnheader{Экосистема OSTIS}
    \scntext{вопрос}{Какова структура Экосистемы OSTIS}
    \scntext{пояснение}{Популяция
        \begin{itemize}
            \item семантически совместимых
            \item эволюционируемых
            \item активно взаимодействующих
            \item способных координировать (согласовывать) свою деятельность с другими субъектами
        \end{itemize}
        интеллектуальных компьютерных систем (\textit{ostis-систем}). При этом указанная популяция \textit{ostis-систем} поддерживает децентрализованное управление собственной деятельностью, а также деятельностью людей (пользователей \textit{ostis-систем}) и человеко-машинных сообществ (\textit{ostis-сообществ}), обеспечивая тем самым автоматизацию системной интеграции любых новых субъектов (\textit{ostis-систем}, людей, \textit{ostis-сообществ}) в состав \textit{Экосистемы OSTIS}.}
    \begin{scnrelfromvector}{принципы, лежащие в основе}
        \scnfileitem{\textit{Экосистема OSTIS} представляет собой сеть \textit{ostis-сообществ}}
        \scnfileitem{Каждому \textit{ostis-сообществу} взаимно однозначно соответствует \textit{корпоративная ostis-система} этого \textit{ostis-сообщества}, которая:
            \begin{itemize}
                \item обеспечивает координацию деятельности членов соответствующего \textit{ostis-сообщества}
                \item является представителем этого \textit{ostis-сообщества} в других \textit{ostis-сообществах}, членом которых указанное \textit{ostis-сообщество} является.
            \end{itemize}}
        \scnfileitem{Каждое \textit{ostis-сообщество} может входить в состав любого другого \textit{ostis-сообщества} по своей инициативе. Формально это означает, что \textit{корпоративная ostis-система} первого \textit{ostis-сообщества} является членом другого \textit{ostis-сообщества}.}
        \scnfileitem{Каждому специалисту, входящему в состав \textit{Экосистемы OSTIS} ставится во взаимнооднозначное соответствие его \textit{персональный ostis-ассистент}, который трактуется как \textit{корпоративная \mbox{ostis-система}} вырожденного \textit{ostis-сообщества}, состоящего из одного человека.}
    \end{scnrelfromvector}
    
    \scnheader{следует отличать*}
    \begin{scnhaselementset}
        \scnitem{корпоративная ostis-система*}
        \begin{scnindent}
        	\scnidtf{корпоративная ostis-система данного ostis-сообщества*}
        \end{scnindent}
        \scnitem{корпоративная ostis-система}
        \begin{scnindent}
            \scnrelto{второй домен}{корпоративная ostis-система*}
        \end{scnindent}
        \scnitem{член ostis-сообщества*}
        \scnitem{персональный ostis-ассистент*}
        \begin{scnindent}
        	\scnidtf{персональный ostis-ассистент данного специалиста*}
            \scnsubset{корпоративная ostis-система*}
        \end{scnindent}
        \scnitem{персональный ostis-ассистент}
        \begin{scnindent}
            \scnrelto{второй домен}{персональный ostis-ассистент*}
            \scnsubset{корпоративная ostis-система}
        \end{scnindent}
    \end{scnhaselementset}

    \scnheader{есть сходства*}
    \begin{scnhaselementset}
        \scnitem{Экосистема OSTIS}
        \scnitem{ostis-сообщество}
        \begin{scnindent}
            \scnhaselement{Экосистема OSTIS}
        \end{scnindent}
    \end{scnhaselementset}
	\begin{scnindent}
    	\scntext{пояснение}{\textit{Экосистема OSTIS} является максимальным \textit{ostis-сообществом}, включающим в себя все существующие \textit{ostis-сообщества}}
	\end{scnindent}
	
    \scnheader{Экосистема OSTIS}
    \scnidtf{Максимальное \textit{иерархическое ostis-сообщество}, обеспечивающее комплексную автоматизацию \uline{всех} видов \textit{человеческой деятельности}}
    \scnidtf{Глобальное \textit{ostis-сообщество}, которое не может входить в состав какого-либо другого \textit{ostis-сообщества}}
    \scniselement{иерархическое ostis-сообщество}
    \begin{scnindent}
	    \scnidtf{такое \textit{ostis-сообщество}, по крайней мере одним из членов которого является некоторое другое \textit{ostis-сообщество}}
	    \scnsubset{ostis-сообщество}
	    \begin{scnindent}
	    	\scnrelboth{следует отличать}{коллектив ostis-систем}
	    \end{scnindent}
    \end{scnindent}
    \scntext{пояснение}{Понятие \textit{ostis-сообщества} представляет собой не только \textit{коллектив ostis-систем}, но также определенный \textit{коллектив людей} (пользователей и разработчиков соответствующих \textit{ostis-систем})}
    \scnsuperset{ostis-система автоматизации проектирования}
    \scnsuperset{ostis-система автоматизации производства}
    \begin{scnindent}
    	\scnidtf{ostis-система управления производством}
    \end{scnindent}
    \scnsuperset{ostis-система автоматизации образовательной деятельности}
    \scnsuperset{обучающаяся ostis-система}
    \scnsuperset{корпоративная ostis-система виртуальной кафедры}
    \begin{scnindent}
    	\scnidtf{корпоративная ostis-система, обеспечивающая интеграцию деятельности кафедр одинакового профиля и, возможно, различных вузов}
    \end{scnindent}
    \scnsuperset{ostis-система автоматизации бизнес-деятельности}
    \scnsuperset{ostis-система автоматизации управления}
    \scnsuperset{ostis-система управления проектами соответствующего вида}
    \scnsuperset{ostis-система сенсомоторной координации при выполнении определенного вида сложных действий во внешней среде}
    \scnsuperset{ostis-система управления самостоятельным перемещением робота по пересеченной местности}
    
    \scnheader{обучающая ostis-система}
    \scntext{примечание}{Поскольку качество эксплуатации каждой \textit{ostis-системы} зависит не только от нее, но и от квалификации пользователя (семантическая совместимость, знания о возможностях системы), каждая \textit{ostis-система} должна быть способна обучать пользователя знаниям и навыкам эффективного её использования.}

    \scnheader{ostis-сообщество}
    \scnidtf{устойчивый фрагмент \textit{Экосистемы OSTIS}, обеспечивающий комплексную автоматизацию определенной части коллективной человеческой деятельности и перманентное повышение ее эффективности (в т.ч. уровня автоматизации)}
    \scntext{примечание}{Наряду с \textit{Экосистемой OSTIS}, идея \textit{ostis-сообщества} представляет собой естественный этап перехода (эволюции) творчески ориентированных коллективов людей в принципиально новое существенно более интеллектуальное и, соответственно, более позитивное качество}
    \begin{scnreltolist}{перманентно-решаемая задача}
        \scnitem{перманентная поддержка семантической совместимости членов ostis-сообщества}
        \scnitem{перманентная поддержка высокого качества базы знаний, доступной всем членам ostis-сообщества}
        \begin{scnindent}
            \scntext{пояснение}{Имеется в виду поддержка непротиворечивости (корректности отсутствия синонимов, омонимов и противоречий), полноты (отсутствия информационных дыр) и чистоты (отсутствия информационного мусора)}
        \end{scnindent}
        \scnitem{перманентная поддержка мониторинга эффективности распределения работ между членами ostis-сообщества и контроля исполнительской дисциплины}
        \scnitem{перманентная поддержка мониторинга динамики роста квалификации каждого члена ostis-сообщества}
    \end{scnreltolist}
    
    \scnheader{коллектив людей}
    \scnidtf{человеческое общество}
    \scntext{примечание}{Низкий уровень интеллекта современных коллективов людей определяется (1) низким уровнем качества организации общей памяти каждого такого коллектива (общей памяти всех его членов) (2) низкий уровень качества знаний, хранимых в этой памяти (как минимум это наличие большого количества синонимов, омонимов и противоречий). Поэтому переход от современных коллективов людей к соответствующим ostis-сообществам существенно повышает уровень интеллекта этих коллективов}

    \scnheader{Экосистема OSTIS}
    \scniselement{ostis-сообщество}
    \scnrelto{субъект}{Объединенная человеческая деятельность, осуществляемая на основе Технологии OSTIS}
    \scnrelfrom{корпоративная ostis-система}{Корпоративная система Экосистемы OSTIS}
    \begin{scnindent}
    	\scntext{пояснение}{Основным назначением \textit{Корпоративной системы Экосистемы OSTIS} является организация общего взаимодействия при выполнении самых различных видов и \textit{областей человеческой деятельности}, которые могут быть либо полностью автоматизированными, либо частично автоматизированными, либо вообще неавтоматизированными. Из этого следует, что база знаний \textit{Корпоративной системы Экосистемы OSTIS} должна содержать \textit{Общую формальную теорию человеческой деятельности}, включающей в себя типологию видов и областей \textit{человеческой деятельности}, а также общую \textit{методологию} этой \textit{деятельности}.}
    \end{scnindent}
    	
    \scnheader{Деятельность в области Искусственного интеллекта, осуществляемая на основе Технологии OSTIS}
    \scnrelfrom{субъект}{Консорциум OSTIS}
    \begin{scnindent}
    	\scnrelfrom{корпоративная ostis-система}{Корпоративная ostis-система Консорциума OSTIS}
    \end{scnindent}
    \scnrelfrom{основной продукт}{Экосистема OSTIS}
    \begin{scnindent}
	    \scnrelfrom{частное ostis-сообщество}{Консорциум OSTIS}
	    \scnrelfrom{член ostis-сообщества}{Консорциум OSTIS}
	    \begin{scnindent}
            \begin{scnrelfromset}{примечание}
                \scnitem{член ostis-сообщества*}
                \begin{scnindent}
                    \scnsubset{частное ostis-сообщество*}
                    \begin{scnindent}
                        \scnidtf{\textit{ostis-сообщество}, входящее в состав заданного \textit{ostis-сообщества}}
                    \end{scnindent}
                \end{scnindent}
            \end{scnrelfromset}
	    \end{scnindent}
	\end{scnindent}
    \scnidtf{Проект, основной целью (продуктом) которого является создание \textit{Экосистемы OSTIS}}
    \scnrelfrom{часто используемый sc-идентификатор}{Проект Экосистемы OSTIS}
    \scnidtf{Деятельность, направленная на создание и перманентное развитие \textit{Экосистемы OSTIS}}
    \scnidtf{Проект, направленный на проектирование, производство и реинжиниринг \textit{ostis-систем}, входящих в сеть \textit{Экосистемы OSTIS}, а также на проектирование и реинжиниринг Экосистемы OSTIS в целом (как сети \textit{ostis-систем} и их пользователей)}
    \begin{scnrelfromlist}{подпроект}
        \scnitem{Проект Метасистемы OSTIS}
        \scnitem{Проект программной реализации абстрактной sc-машины}
        \scnitem{Проект разработки универсального sc-компьютера}
    \end{scnrelfromlist}
    \scntext{примечание}{В состав \textit{Проекта Экосистемы OSTIS} входит большое количество \textit{проектов} (\textit{подпроектов}*), направленных на \textit{проектирование} и \textit{производство ostis-систем} самого различного назначения}
    \scntext{пояснение}{Распространим предлагаемый нами подход к повышению эффективной и человеческой \textit{Деятельности в области Искусственного интеллекта} на всю \textit{Объединенную человеческую деятельность} в целом, т.е. рассмотрим структуру Глобального \textit{ostis-сообщества} (\textit{Экосистемы OSTIS})}
    \scntext{эпиграф}{От \textit{Консорциума OSTIS} к \textit{Экосистеме OSTIS}}
    
    \scnheader{Экосистема OSTIS}
    \scntext{примечание}{Подчеркнем, что \textit{Экосистема OSTIS} является:
        \begin{itemize}
            \item с одной стороны, \textit{основным продуктом*} человеческой \textit{Деятельности в области Искусственного интеллекта, осуществляемой на основе Технологии OSTIS} (эту Деятельность мы также будем называть Проектом Экосистемы OSTIS),
            \item а, с другой стороны, \textit{субъектом* Объединенной человеческой деятельности, осуществляемой на основе Технологии OSTIS}.
        \end{itemize}
        Особо подчеркнем то, что продуктом человеческой \textit{Деятельности в области Искусственного интеллекта, осуществляемой на основе Технологии OSTIS}, является не просто множество \textit{ostis-систем} различного назначения, а Экосистема, состоящая из \underline{взаимодействующих} \textit{ostis-систем} и их пользователей}
    \scntext{пояснение}{Принципиальным является то, что продуктом (результатом применения) \textit{Технологии OSTIS} является не просто множество \textit{ostis-систем}, а целая система, состоящая из \textit{ostis-систем} и их пользователей, взаимодействующих между собой и осуществляющих комплексную автоматизацию всех \textit{видов человеческой деятельности}, а также комплексное повышение уровня эффективности организации человеческой деятельности (и, в частности, повышение уровня автоматизации этой деятельности).}

    \scnheader{Экосистема OSTIS}
    \begin{scnrelfromlist}{\scnkeyword{вопрос}}
        \scnitem{Каковы основные свойства Экосистемы OSTIS}
        \scnitem{Какова основная цель создания Экосистемы OSTIS}
    \end{scnrelfromlist}
    \scnidtf{Экосистема ostis-систем и их пользователей}
    \scnrelto{общий создаваемый продукт}{Технология OSTIS}
    \scnidtf{Расширяемый коллектив эволюционируемых, семантически совместимых и взаимодействующих ostis-систем и их пользователей}
    \scniselement{многоагентная система}
    \scnidtf{Многоагентная система, агентами которой являются ostis-системы, а также их конечные пользователи и разработчики}
    
    \scnheader{Экосистема интеллектуальных компьютерных систем}
    \scnidtf{Smart-сообщество}
    \scnidtf{Smart-сообщество интеллектуальных компьютерных систем и людей}
    \scnidtf{Интеллектуальная многоагентная система, состоящая из интеллектуальных компьтерных систем и людей}
    \scntext{примечание}{Многоагентная система может состоять из кибернетических систем, не являющихся интеллектуальными.}
    \scntext{примечание}{Многоагентная система может состоять из интеллектуальных систем, но сама не быть интеллектуальной. Количество далеко не всегда переходит в нужное качество.}
    \scnidtf{Экосистема ostis-систем, а также их разработчиков и пользователей}
    \scnidtf{Эволюционирумая сеть ostis-систем, обеспечивающая конвергенцию и интеграцию всех видов человеческой деятельности}
    
    \scnheader{Экосистема OSTIS}
    \scnidtf{Глобальная компьютерная сеть ostis-систем, обеспечивающая комплексную автоматизацию всевозможных видов и областей человеческой деятельности и отражающая иерархию уровней этой деятельности}
    \scnidtf{Глобальная \textit{многоагентная система}, состоящая из людей и семантически совместимых \textit{интеллектуальных компьютерных систем}, построенных по \textit{Технологии OSTIS}, которые, взаимодействуя между собой и с людьми, обеспечивают существенное повышение уровня автоматизации всех видов \textit{человеческой деятельности} и существенное повышение эффективности человеческого взаимодействия}
    \scnidtf{Предлагаемых нами подход к реализации smart-общества}
    \scnidtf{Smart-общество, построенное на основе Технологии OSTIS}
    \scnidtf{следующий этап развития человеческого общества, обеспечивающий существенное повышение уровня общественного (коллективного) интеллекта путем преобразования человеческого общества в экосистему, состоящую из людей и семантически совместимых интеллектуальных систем}
    \scntext{обоснование}{Необходимо обеспечить не только повышение уровня автоматизации человеческой деятельности (как информационной (умственной), так и физической), но и существенное повышение уровня интеллекта человеческого общества как социальной кибернетической системы путем создания многоагентной кибернетической системы, сосотоящей из \textit{интеллектуальных компьютерных систем} и людей и имеющей \textit{высокий уровень интеллекта}. Человечества пока не умеет создавать инетеллектуальные сообщества (коллективы) людей и, тем более, интеллектуальные общества людей и интеллектуальных компьютерных системы.Уровень интеллекта каждого такого сообщества обычно определяется уровнем интеллекта его руководителя (лица, принимающего решение).А надо, чтобы уровень интеллекта сообщества был результатом интеграции интеллектуального потенциала всех его членов. При этом следует помнить, что интеллект определяется не только и не столько множеством решаемых задач, а \uline{скоростью} расширения этого множества.}
    \scntext{примечание}{Предметом инженерной деятельности в области \textit{искусственного интеллекта} следует считать не множество \textit{интеллектуальных компьютерных систем} (например, \textit{ostis-систем}), а весь комплекс взаимодействующих между собой \textit{интеллектуальных компьютерных систем}. Назовём такой комплекс \textit{Экосистемой интеллектуальных компьютерных систем} (в нашем случае это Экосистема OSTIS Экосистема взаимодействующих \textit{ostis-систем}) здесь важно построить архитектуру таковой экосистемы, в основе которой должна лежать комплексная формальная модель всевозможных видов человеческой деятельности, автоматизируемых с помощью интеллектуальных компьютерных систем (ostis-систем). Указання комплексная модель человеческой деятельности является необходимой основой создания smart-общества (общества 5.0)}
    \scnrelto{общий создаваемый продукт}{Технология OSTIS}
    \scnidtf{Общий (объединенный, интегрированный) продукт использования \textit{Технологии OSTIS}, представляющий собой глобальную сеть \textit{ostis-систем}, обеспечивающий комплексную автоматизацию и интеграцию всевозможных \textit{видов человеческой деятельности} и, в частности, включающий в себя (в виде соответствующего \mbox{\textit{ostis-сообщества}}) \textit{консорциум OSTIS}, т.е. инфраструктуру, направленную на перманентное развитие \textit{Технологии OSTIS} (как Ядра Технологии OSTIS, так и иерархического семейства \textit{специализированных ostis-технологий})}
    \scntext{следовательно}{\textit{Экосистема OSTIS} представляет собой саморазвивающуюся сеть ostis-систем}
    \scntext{пояснение}{Сверхзадачей \textit{Экосистемы OSTIS} является не просто комплексная автоматизация всех \textit{видов человеческой деятельности} (разумеется, только тех видов деятельности, автоматизация которых целесообразна), но и существенное повышение уровня интеллекта различных человеческих (точнее человеко-машинных) сообществ и всего человеческого общества в целом. Это потребует соблюдения ряда требований, предъявляемых не только к \textit{интеллектуальным компьютерным системам}, но и к людям, входящим в состав \textit{Экосистемы OSTIS}}
    \scntext{пояснение}{\textit{Экосистема OSTIS} представляет собой открытый коллектив взаимодействующих интеллектуальных систем, состав которого входят \textit{ostis-системы} и их пользователи (конечные пользователи и разработчики,участвующие в совершенствование этих \textit{ostis-систем}). Особое место среди \textit{ostis-систем}, входящих в состав \textit{экосистемы OSTIS}, занимают \textit{корпоративные ostis-системы}, через которое осуществляется координация и эволюция деятельности некоторых групп \textit{ostis-систем} и их пользователей. Основная цель корпоративных \textit{ostis-систем} --- локализовать базы знаний указанных групп ostis-систем, перевести их из статуса виртуальных в статус реальных и автоматизировать их эволюцию.}
    \scnidtf{Сообщество ostis-систем или людей, обеспечивающее принципиально новый уровень автоматизации человеческой деятельности и принципиально \uline{новый уровень интеллекта человеческого общества}}
    
    \scnheader{Экосистема OSTIS}
    \scntext{примечание}{Очень важно проектировать не только саму \textit{Экосистему OSTIS}, ну и процесс \uline{поэтапного перехода} от современной глобальной сети \textit{компьютерных систем} к глобальной сети \textit{ostis-систем} (т.е. к \textit{Экосистеме OSTIS}). В рамках такого переходного периода \textit{ostis-системы} могут выполнять роль системных интеграторов различных ресурсов и сервисов, реализованных современными \textit{компьютерными системами}, поскольку уровень интеллекта \textit{ostis-систем} позволяет им с любой степенью детализации специфицировать интегрируемые \textit{компьютерные системы} и, следовательно, достаточно адекватно понимать, что знает и/или умеет каждая из них, а также достаточно качественно координировать их деятельность и обеспечивать релевантный поиск нужного ресурса и сервиса. Кроме того системы могут выполнять роль интеллектуальных help-систем --- помощников и консультантов по вопросам эффективной эксплуатации различных \textit{компьютерных систем} со сложными функциональными возможностями, имеющими пользовательский интерфейс с нетривиальной семантикой и использующимися в сложных предметных областях. Такие интеллектуальные help-системы можно сделать интеллектуальными посредниками между соответствующими компьютерными системами их пользователями. При этом пользователь может работать одновременно и с help-системой и с соответствующей эксплуатируемой компьютерной системой, консультируясь с help-системой в затруднительных для него ситуациях. Основными недостатками такого варианта является то,что: (1) пользователь должен использовать два разных интерфейса и (2) help-система не может мониторить деятельность пользователя и, следовательно, пользователь сам должен сообщать системе о своих затруднительных ситуациях. Указанные недостатки можно устранить, если компьютерную систему, которая построена по современным технологиям и эксплуатация которой нуждается в качественной консультационной (help-овой) поддержке, интегрировать с соответствующей ей help-системой, построенной по стандартам технологии OSTIS, так, чтобы пользовательским интерфейсом такой интегрированной системы стал пользовательский интерфейс, соответствующий стандартам технологии OSTIS и важнейшим достоинством которого является чёткая формализация семантики всех элементов управления пользовательским интерфейсом. Благодаря этому взаимодействие пользователя с пользовательским интерфейсом ostis-систем становится, во-первых, осмысленным и, во-вторых, позволяющим легко переносить опыт интерфейсного взаимодействия с одной ostis-системы на другую ostis-систему.}

    \scnheader{Объединенная человеческая деятельность}
    \scnidtf{максимальная область человеческой деятельности}
    \scnidtf{вся человеческая деятельность}
    \scnidtf{человеческая деятельность в целом}
    \scnidtf{объединение всевозможных областей человеческой деятельности}
    \scniselement{человеческая деятельность}
    \begin{scnindent}
	    \scnidtf{область человеческой деятельности}
	    \scnidtf{система целенаправленных действий некоторого количества(возможно одного) людей над некоторыми объектами с помощью некоторых инструментов}
	    \scnsubset{деятельность}
	    \begin{scnindent}
		    \scnidtf{трудно выполнимое сложное действие}
		    \scnidtf{область деятельности}
		    \scnidtf{система целенаправленных действий некоторых (возможно одного) субъектов над некоторыми объектами с помощью некоторых инструментов}
	    \end{scnindent}
    \end{scnindent}
    
    \scnheader{следует отличать*}
    \begin{scnhaselementset}
        \scnitem{Объединенная человеческая деятельность}
        \begin{scnindent}
            \scnidtf{человеческая деятельность в целом}
            \scnidtf{максимальная область человеческой деятельности}
        \end{scnindent}
        \scnitem{область человеческой деятельности}
        \begin{scnindent}
            \scnidtf{фрагмент (часть, раздел) человеческой деятельности}
            \scnidtf{человеческая деятельность}
            \scnidtf{деятельность, осуществляемая либо одним человеком (индивидуальная человеческая деятельность), либо коллективом людей}
            \scnsubset{деятельность}
            \begin{scnindent}
            	\scnsubset{действие}
            \end{scnindent}
            \scnsuperset{индивидуальная человеческая деятельность}
            \scnsuperset{коллективная человеческая деятельность}
            \scnidtf{множество всевозможных областей человеческой деятельности}
        \end{scnindent}
        \scnitem{вид человеческой деятельности}
        \begin{scnindent}
            \scnidtf{класс однотипных областей человеческой деятельности, которому можно поставить в соответствие некоторую технологию}
            \scnsubset{вид деятельности}
            \begin{scnindent}
            	\scnsubset{класс действий}
            \end{scnindent}
        \end{scnindent}
    \end{scnhaselementset}
    
    \scnheader{следует отличать*}
    \begin{scnhaselementset}
        \scnitem{Объединенная человеческая деятельность}
        \begin{scnindent}
            \scnidtf{максимальный процесс человеческой деятельности, включающий в себя деятельность всех людей и всех сообществ}
        \end{scnindent}
        \scnitem{человеческая деятельность}
        \begin{scnindent}
            \scnidtf{множество всевозможных целостных, целенаправленных фрагментов \textit{Объединенной человеческой деятельности}}
            \scntext{примечание}{На данном множестве заданы такие отношения, как \textit{часть*}, \textit{декомпозиция*}. Т.е. конкретный экземпляр (элемент) данного множества может быть \textit{частью*} (входить в состав) другой конкретной человеческой деятельности. Более того, целесообразно рассматривать достаточно сложную иерархию процессов человеческой деятельности.}
            \scnidtf{конкретный процесс человеческой деятельности}
            \scnidtf{бизнес-процесс}
            \scnidtf{деятельность, основными субъектами которой являются люди и различные сообщества людей}
            \scnsubset{деятельность}
            \begin{scnindent}
            	\scnsubset{действие}
            \end{scnindent}
            \scntext{примечание}{Если для автоматизации человеческой деятельности используются интеллектуальные компьютерные системы, то эти системы также становятся достаточно самостоятельными полноценными субъектами этой деятельности, мнение которых обязательно принимается во внимание, но при этом интеллектуальные компьютерные системы не становятся основными субъектами человеческой деятельности.}
            \scnhaselement{объединенная человеческая деятельность}
            \begin{scnindent}
            	\scnidtf{максимальная человеческая деятельность, для которой не существует никакой другой конкретной человеческой деятельности, частью* которой указанная Максимальная человеческая деятельность является.}
            \end{scnindent}
            \scntext{примечание}{каждая конкретная человеческая деятельность (каждый бизнес-процесс) может быть:
                \begin{itemize}
                    \item либо полностью автоматизирована от человека требуется только корректно сформулировать соответствующую команду (цель инициируемого действия)
                    \item либо автоматизирована, но требующая от человека управления функционированием соответствующего одного инструментального средства
                    \item либо состоящая из фрагментов (подпроцессов, частных бизнес-процессов), некоторые из которых автоматизированы, а некоторые нет
                    \item либо полностью неавтоматизирована (т.е. выполняется вручную)
                \end{itemize}}
            \scntext{примечание}{Когда речь идет о спецификации конкретной человеческой деятельности (конкретного бизнес-процесса), важно провести четкую грань между теми действиями, которые выполняются автоматически (в том числе интеллектуальными компьютерными системами), и действиям, которые выполняются людьми вручную --- это как минимум действия по формулировке команд, которые адресуются соответствующим инструментальным средствам (язык и, соответственно, интерфейс формулировки таких команд для разных инструментальных средств может сильно отличаться).Отсутствие унификации языка взаимодействия (интерфейсы) между людьми и различными инструментальными средствами (автомобилями, станками, холодильниками, газовыми плитами, микроволновками, компьютерными системами различного назначения) существенно снижает комплексную эффективность автоматизации человеческой деятельности, т.к. вынуждает людей тратить много времени на усвоение не сути (смысла) автоматизации, а формы (синтаксиса) своей деятельности по организации использования различных средств автоматизации.}
        \end{scnindent}
        \scnitem{вид человеческой деятельности}
        \begin{scnindent}
            \scnidtf{класс (множество однотипных) процессов человеческой деятельности}
            \scnidtf{класс бизнес-процессов}
            \scnidtf{множество всевозможных классов бизнес-процессов}
            \scnsubset{вид деятельности}
            \scntext{примечание}{Каждый конкретный вид человеческой деятельности (т. е. каждый элемент множества \textit{вид человеческой деятельности}) является \textit{подмножеством*} множества человеческая деятельность.Каждому виду человеческой деятельности соответствует своя \textit{технология человеческой деятельности}, т.е. свой набор \textit{методов} и \textit{средств}, обеспечивающих выполнение каждой конкретной деятельности, принадлежащей этому виду.}
        \end{scnindent}
        \scnitem{область человеческой деятельности}
        \begin{scnindent}
            \scnsubset{человеческая деятельность}
            \scnidtf{достаточно крупный фрагмент человеческой деятельности}
            \scnidtf{раздел человеческой деятельности}
        \end{scnindent}
    \end{scnhaselementset}
    
    \scnheader{Экосистема OSTIS}
    \scntext{примечание}{Содержательную типологию \textit{ostis-систем}, входящих в состав \textit{Экосистемы OSTIS} следует проводить на основе глубокого анализа содержательной структуры человеческой деятельности, требующей взаимодействия человека с другими людьми и даже с организациями. Очевидно, что эффективность такого взаимодействия во многом определяется качеством организации информационного взаимодействия, уровнем взаимопонимания, уровнем квалификации участников, оперативностью получения качественной консультативной помощи по любому (!) вопросу.}
    
    \scnheader{вид человеческой деятельности, продуктом которой является информационная модель некоторого объекта или класса объектов}
    \scnidtf{вид человеческой деятельности, направленной на построение описания (спецификации) некоторого объекта исследования или класса таких объектов}
    \scnsubset{вид человеческой деятельности}
    \scnhaselement{научно-исследовательская деятельность}
    \begin{scnindent}
	    \scnhaselement{Научно-исследовательская деятельность в области Искусственного интеллекта}
	    \begin{scnindent}
	    	\scnidtf{разработка Общей теории интеллектуальных систем}
	    \end{scnindent}
	\end{scnindent}
    \scnhaselement{разработка теории искусственных объектов заданного класса}
    \begin{scnindent}
	    \scnhaselement{Разработка Общей теории интеллектуальных компьютерных систем}
		\begin{scnindent}
		    \scnidtf{разработка стандарта интеллектуальных компьютерных систем}
		\end{scnindent}
   	\end{scnindent}
    \scnhaselement{разработка теории проектирования искусственных объектов заданного класса}
    \begin{scnindent}
	    \scnidtf{разработка системы проектных действий для искусственных объектов (артефактов) заданного класса}
	    \scnhaselement{разработка теории проектирования интеллектуальных компьютерных систем}
	    \begin{scnindent}
	    	\scnidtf{разработка стандарта организации коллективных проектных действий для проектирования интеллектуальных компьютерных систем}
	    \end{scnindent}
    \end{scnindent}
    \scnhaselement{разработка теории производства спроектированных искусственных объектов заданного класса}
    \begin{scnindent}
	    \scnidtf{разработка стандарта системы производственных действий, методов и инструментов, обеспечивающих производство спроектированных артефактов заданного класса}
	    \scnhaselement{Разработка Теории производства спроектированных интеллектуальных компьютерных систем}
    \end{scnindent}
    \scnhaselement{проектирование искусственного объекта заданного класса}
    \begin{scnindent}
	    \scnidtf{проектная деятельность, направленная на построение такой информационной модели (спецификации) искусственно создаваемого объекта (артефакта) заданного класса, которой достаточно для производства этого объекта}
	    \scnsuperset{проектирование конкретной интеллектуальной компьютерной системы}
	    \begin{scnindent}
	    	\scnidtf{процесс проектирования некоторой компьютерной системы по заданной технологии проектирования}
	    \end{scnindent}
    \end{scnindent}
    \scntext{пояснение}{Данный вид человеческой деятельности характерен следующими особенностями:
        \begin{itemize}
            \item очень часто продукт этой деятельности (создаваемая информационная конструкция) имеет высокую степень сложности и, следовательно, указанная деятельность не может быть индивидуальной, а несет коллективный характер;
            \item основными факторами качественного коллективного построения сложной информационной конструкции являются семантическая совместимость (взаимопонимание) авторов, а также согласованность их действий;
            \item важнейшим направлением автоматизации коллективной деятельности, объект и продукт которой представляет собой сложную информационную конструкцию, является автоматизация редактирования коллективно создаваемого информационного объекта, а также автоматизация обеспечения семантической совместимости и согласованности продуктов индивидуальной деятельности всех соавторов;
            \item указанную автоматизацию легко реализовать с помощью корпоративной интеллектуальной компьютерной системы, объединяющей всех соавторов создаваемого информационного объекта и снабженной мощными средствами поддержки коллективного проектирования различных разделов базы знаний этой системы. Примерами таких систем являются интеллектуальные порталы различного вида знаний.
        \end{itemize}}
    
    \scnheader{вид человеческой деятельности, продуктом которой является информационная модель некоторого объекта или класса объектов}
    \scntext{пояснение}{Здесь речь идет о коллективной человеческой деятельности, которая принципиально не может быть полностью автоматизирована (исследовательская, проектная), то основной проблемой ее автоматизации являются
        \begin{itemize}
            \item недостаточный уровень семантической совместимости и взаимопонимания между людьми и отсутствие сознания серьезности этой проблемы;
            \item недостаточный уровень договоренности и отсутствия понимания серьезности этой проблемы;
            \item отсутствие четкой методики согласования точек зрения и отсутствие понимая серьезности этой проблемы.
        \end{itemize}
        Интеллектуальные компьютерные системы могут и должны создать корпоративную среду для решения этих проблем.По сути это не что иное, как поддержка коллективного проектирования соответствующих разделов баз знаний интеллектуальной компьютерной системы, реализуемая на \uline{семантическом уровне}, когда интеллектуальная компьютерная система становится самостоятельным полноправным участником деятельности, в обязанности которого входит:
        \begin{itemize}
            \item анализ семантической совместимости точек уровня различных участников,
            \begin{itemize}
                \item выявление противоречий и альтернатив
            \end{itemize}
            \item фиксация авторства
            \item отмена современной формы представления интеллектуального продукта (статьи, книги, документы)
        \end{itemize}
        Недостаточно высокий уровень семантической согласованности используемых понятий приводит к огромному количеству искусственно создаваемых противоречий.При этом следует отличать семантические противоречия (например, синонимию вводимых знаков) и, соответственно, методику их устранения или разногласия по поводу системы вводимых понятий от терминологических разногласий, методика устранения которых может и должна быть максимально простой и лишенной эмоциональной окраски. Излишнее увлечение терминологическими спорами существенно тормозит творческий процесс, но и несерьезное отношение к постоянному совершенствованию и соблюдению \uline{правил} построения терминов также недопустимо.}
        
    \scnheader{Рынок знаний, реализуемый в рамках Экосистемы OSTIS}
    \scntext{пояснение}{Важнейшим видом предметно-независимой человеческой деятельности, осуществляемой в рамках \textit{Экосистемы OSTIS} является перманентный реинжиниринг всех \textit{ostis-систем}, входящих в \textit{Экосистему OSTIS}. Указанная деятельность должна быть направлена на перманентную и быструю эволюцию всех ostis-систем и, самое важное, на эволюцию \textit{Экосистемы OSTIS} в целом. Особо следует подчеркнуть, что эволюция \textit{ostis-систем} и \textit{Экосистемы OSTIS} в целом представляет собой весьма сложный творческий, коллективный процесс, который принципиально может быть автоматизирован \uline{только частично}. При этом от людей, участвующих в этом процессе требуется высокая квалификация, высочайшая системная культура на уровне глубокого знания общей теории систем, высокая математическая культура --- культура формализации, высокая культура конвергенции (обнаружения сходств, доведение их до формальных аналогий), высокая культура глубокой интеграции, высокий уровень договороспособности.\\
        Кроме указанных требований необходим высочайший уровень мотивации к тому, чтобы эволюция отдельных компонентов \textit{Экосистемы OSTIS} (в частности, отдельных \textit{ostis-систем}) не осуществлялась в ущерб эволюции \textit{Экосистемы OSTIS} в целом, например, путём привнесения эклектичности, многообразия форм решения похожих проблем, путем ослабления фундаментального требования \uline{максимально возможной простоты} и логичности принципов, лежащих в основе Экосистемы OSTIS.\\
        Существенно подчеркнуть, что эволюция \textit{ostis-систем} и \textit{Экосистем OSTIS} в целом сводится к коллективному реинжинирингу \textit{баз знаний ostis-систем}, что, в свою очередь сводится к:
        \begin{itemize}
            \item ручной генерации предлагаемых дополнительных (новых) знаний в базу знаний указываемой \mbox{ostis-системы}
            \item ручной генерации предлагаемых изменений текущего состояниябазы знаний указываемой ostis-системы;
            \item автоматическому назначению компетентных и заинтересованныхрецензентов;
            \item ручному рецензированию каждого поступившего предложения,результатом чего является:
            \begin{scnitemizeii}
                \item либо полное одобрение;
                \item либо полное неодобрения с предлагаемой аргументацией;
                \item либо детальная рекомендация доработки, предположения;
            \end{scnitemizeii}
            \item автоматическому назначению достаточно широкого круга компетентных и заинтересованных специалистов для утвержденияпоступившего предложения (после получения одобрения от всех назначенных экспертов);
            \item автоматическому принятию решения по одобрению поступившегопредложения на основании мнения всех привлечённых экспертов испециалистов.
        \end{itemize}
        Таким образом в \textit{базе знаний} каждой \textit{ostis-системы} можно (и нужно!)фиксировать весь процесс обсуждения каждого поступившего предложения суказанием (1) моментов времени всех привлечённых событий; (2) участников каждого события (авторов предложений, авторов рецензий участников голосования).\\
        Кроме того, каждая \textit{ostis-система}, анализируя процесс использованияхранимых ею знаний в процессе эксплуатации, может оценивать частотунепосредственного и опосредованного использования этих знаний, т.е. может оценить степень востребованности этих знаний.\\
        Следовательно, в перспективе \textit{Экосистема OSTIS} может с достаточновысокой степенью \uline{объективности} может оценивать объем и значимость вклада каждого специалиста в развитие распределенной базы знаний \textit{Экосистемы OSTIS}. Это является фундаментальной основой дляформирования достаточно объективного (честного) \textit{рынка знаний}.}
        
    \scnheader{Рынок знаний, реализуемый в рамках Экосистемы OSTIS}
    \scntext{правило для авторов в рамках Экосистемы OSTIS}{знания, \uline{предлагаемые} для рецензирования, согласования,утверждения и публикации в базе знаний соответствующей ostis-системы должны быть специфицированы (указана ostis-система, атомарный раздел базы знаний, дата и время, автор,новый вид публикации, рынок знаний,защита авторского права не на уровне документов, а на уровне смысла.}
    \scntext{коллективное совершенствование базы знаний}{Абсолютно идеальных решений (в том числе проектных) не бывает. Поэтому (1) не надо бояться ошибок и (2) надо минимизировать степень ошибочности за счёт (2.1) \uline{оперативности} исправления ошибок и (2.2) повышения качества (уровня) анализа при принятии решения путем (2.2.1) \uline{коллективного} характера экспертизы,(2.2.2) достаточного количества привлекаемыхэкспертов и (2.2.3) учёта уровня осведомленности(квалифицированности и  погруженности всоответствующую предметную область ионтологию). Для каждого эксперта, привлекаемого к принятию решения нужен постоянно уточняемый, по объективным критериям коэффициент осведомленности-авторитетности каждого эксперта к каждой конкретной предметной области.}
    \scntext{правила редактирования Общей базы знаний коллектива интеллектуальной системы}{
        \begin{itemize}
            \item Если Вы в рамках базы знаний разрабатываемой Вами ostis-системы хотите ввести знак новой ранее не описываемой сущности, то Вы должны проверить, что эта сущностьдействительно не описывалась в рамках виртуальной базы знаний всей Экосистемы OSTIS
            \begin{scnitemizeii}
                \item Если в результате такой проверки выяснилось, что указанная сущность уже рассматривалась, то Вы должны использоватьвведенный ранее основной внешний идентификатор этой сущности (Если он Вам не нравится, можете предложить, но пока не использовать, свой)
                \item Если сущность не рассматриваласть, нужно специфицировать, связать с семантическиблизким (особенно для понятий)
            \end{scnitemizeii}
        \end{itemize}
        От толковых словарей и энциклопедий --- к стройнойсемантической сети таких спецификации \uline{всех} описываемых сущностей, которые позволяют установить (желательно автоматически) наличие или отсутствие в рамках технического состояния базызнаний синонимичного знака для любого нового знака, вводимого в базу знаний.}
    \scntext{cтруктура качественной спецификации}{Нужно стремиться:
        \begin{itemize}
            \item к однозначности такой спецификации;
            \item координаты в пространстве декомпозиций
            \item к семантической близости;
            \item сходства, отличия;
        \end{itemize}}
    
    \scnheader{качество человеческой деятельности}
    \scnidtf{качество деятельности человеческого общества}
    \scntext{пояснение}{Поскольку человеческого общество в целом является кибернетической системой, (которая принадлежит классу иерархических многоагентных систем, качество деятельности человеческого общества можно оценивать по критериям качества кибернетических систем.
        \\На основании этих критериев можно оценивать:
        \begin{itemize}
            \item качество информационной среды, формируемой человеческим обществом, т.е. качество накапливаемой и общедоступной информации;
            \item качество текущего состояния общечеловеческих знаний;
            \item качество методов и технологий, используемых для решения задач как в рамках накопленных человечеством знаний, так и врамках внешней среды человеческого общества;
            \item качество организаций человеческой деятельности в целом;
            \item обучаемость (темпы эволюции) человеческого общества в целом.
        \end{itemize}
        Современный этап развития науки и техники характерен тем, что при оценке качества научно-технических результатов акцентируется внимание на новизне результатов, на их \uline{отличиях} от текущего положения дел. Это создает почву и дляимитации этой новизны и для увеличения барьеров между различными дисциплинами, что существенно препятствует конвергенции и интеграции различных дисциплин. Указанная конвергенция и инеграция, в частности, необходима для \uline{комплексной} автоматизации \uline{всех} видов человеческойдеятельности в рамках smart-общества. Очевидно, что основной такой комплексной автоматизации должна быть \textbf{\textit{Общая формальная теория человеческой деятельности}}.}
        
    \scnheader{уровень конвергенции и интерации человеческой деятельности и её результатов}
    \scnrelto{свойство-предпосылка}{Качество человеческой деятельности}
    \scntext{пояснение}{Повышение уровня конвергенции и интеграции различных видов человеческой деятельности и, соответственно, результатовэтой деятельности является важнейшим фактором (важнейшим направлением) повышения качества(эффективности) человеческой деятельности, а, следовательно, и качества самого человеческого общества как сложной распределенной социотехнической кибернетической системы.}
    \scntext{вопрос}{Что является главным препятствием существенному повышению уровня конвергенции и интеграции человеческой деятельности.}
    \begin{scnindent}
    	\scntext{ответ}{Главным препятствием повышению уровня конвергенции и интеграции человеческой деятельности является то, что на текущем этапе эволюции человеческого общества основным механизмом эволюции является конкуренция. Конкуренция предполагает противопоставление результатов своей деятельности результатам конкурентов. т.е. акцентирует внимание на отличиях, новизне, преимуществах своих результатов по отношению к результатам своих конкурентов. При этом мысль о целесообразности объединения усилий со своими конкурентами чаще обусловлена стремлением повысить уровень конкурентоспособности и прибыли по отношению к другим более сильным конкурентам и значительно реже обусловлена искренним стремлением получить более качественный результат.
            \\Таким образом, повышение уровня конвергенции и интеграции всех видов человеческой деятельности требует весьма сложного перехода от использования механизма конкуренции в её современном виде к созданию мощной технологической основы, обеспечивающей широкое взаимовыгодное сотрудничество и гарантированные возможности самореализации каждого человека и каждого коллектива. Фундаментом указанной технологической основы может и должен стать общечеловеческий рынок знаний, который построен на базе сети интеллектуальных компьютерных систем и в рамках которого фиксируется и объективно оценивается значимость вклада каждого человека и каждого коллектива.}
    \end{scnindent}
    
    \scnheader{автоматизация человеческой деятельности}
    \begin{scnrelfromlist}{вопрос}
        \scnfileitem{В чем заключаются проблемы комплексной автоматизации человеческой деятельности}
        \scnfileitem{Как автоматизировать участие человеческая одновременно в нескольких разных действиях (разных областях деятельности), принадлежащих в общем случае разным видам деятельности}
    \end{scnrelfromlist}

    \scnheader{Экосистема OSTIS}
    \scntext{примечание}{Во многом разработка принципов организации взаимодействия интеллектуальных компьютерных систем (ostis-систем) и людей, входящих в состав \textit{Экосистемы OSTIS} должна опираться на анализ того, как взаимодействие осуществляется между людьми, когда основные проблемы возникают из-за:
        \begin{itemize}
            \item отсутствия взаимопонимания (семантической совместимости),
            \item противоречий между целями различных субъектов,
            \item имитации целенаправленных действий,
            \item нарушений каких-либо соглашений, договоренностей и даже законов (правил поведения и обязанностей).
        \end{itemize}
        Общая (общедоступная) \textit{база знаний} всей \textit{Экосистемы OSTIS}, а также корпоративная \textit{база знаний} каждого \textit{ostis-сообщества}, входящего в состав \textit{Экосистемы OSTIS}, является распределенной, но при этом обязательно целостной. Она поддерживается группой специальных \textit{ostis-систем}, являющихся \textit{порталами знаний} по самым различным областям. Для \textit{Технологии OSTIS} роль такого \textit{портала знаний} выполняет \textit{Метасистема OSTIS}. \textit{Экосистема OSTIS} представляет собой многоагентную социотехническую систему, в которой каждая \textit{индивидуальная ostis-система}, входящая в состав \textit{Экосистемы OSTIS}, каждый пользователь указанных \textit{ostis-систем}, а также каждое \textit{ostis-сообщество}, входящее в Экосистему, является её самостоятельным \textit{субъектом*} (когнитивным агентом). При этом каждый субъект \textit{Экосистемы OSTIS} должен соблюдать определенные правила, обеспечивающие качественную (эффективную) эксплуатацию и эволюцию \textit{Экосистемы OSTIS}.}
    \scntext{резюме}{Сама идея комплексной автоматизации всех видов человеческой деятельности предполагает необходимость:
        \begin{itemize}
            \item разработки достаточно детальных формальных теорий всех видов человеческой деятельности, причем, теорий, доведенных до уровня разделов баз знаний соответствующих корпоративных компьютерных систем --- это, фактически, строгое описание стандартов различных видов человеческой деятельности, доведенное до такого уровня, чтобы соответствующая корпоративная система \underline{понимала} , в какой деятельности она участвует, и могла быть активным и полноценным субъектом (участником) этой деятельности;
            \item серьезного отношения и научного подхода к формализации различных видов человеческой деятельности, к разработке самых различных стандартов;
            \item глубокой конвергенции различных областей (разделов) человеческой деятельности и, соответственно, различных видов человеческой деятельности, осуществляемой в условиях достигнутого уровня автоматизации этой деятельности. Это предполагает необходимость рассмотрения каждого вида человеческой деятельности в контексте \textbf{\textit{Общей теории человеческой деятельности}} в условиях \underline{текущего} состояния уровня автоматизации этой деятельности;
            \item обеспечения высоких темпов эволюции и, следовательно, высокого уровня \underline{гибкости} \textit{Общей теории человеческой деятельности} и теорий (стандартов) каждого \textit{вида деятельности} в силу их большой зависимости от текущего уровня автоматизации;
            \item автоматизации взаимодействия субъектов не только внутри каждой области (раздела) человеческой деятельности, но и между этими областями (разделами), что предполагает автоматизацию \underline{представительства} каждой области (раздела) человеческой деятельности во множестве всех таких областей;
            \item понимания того, что эффективность человеческой деятельности во многом определяется скоординированностью, адекватностью, грамотностью поведения каждого субъекта. Поэтому автоматизация человеческой деятельности должна быть направлена на более глубокую координацию этой деятельности на основе учета смысла и целей этой деятельности. А это превращает средства автоматизации в полноценных субъектов коллективной деятельности
        \end{itemize}}
    
    \scnheader{автоматизация человеческой деятельности}
    \scntext{примечание}{Рассмотрение комплексной автоматизации человеческой \textit{деятельности в области Искусственного интеллекта} естественным образом можно расширить (обобщить) до рассмотрения комплексной автоматизации человеческой деятельности в целом.}
    \scntext{примечание}{Для того, чтобы обеспечить качественную автоматизацию любой \textit{области человеческой деятельности} с помощью \textit{интеллектуальных компьютерных систем}, необходимо построить \textit{формальную модель} этой области деятельности и довести эту модель до такого уровня формализации, чтобы она могла стать частью \textit{базы знаний интеллектуальной компьютерной системы}, используемой для автоматизации указанной \textit{области человеческой деятельности}. Очевидно, что, чем субъекты, участвующие в какой-либо коллективной деятельности, (люди и интеллектуальные компьютерные системы) лучше понимают суть, цели, критерии качества указанной коллективной деятельности, тем выше качество выполнения этой деятельности.}
    
    \scnheader{формальная модели автоматизируемой области человеческой деятельности}
    \scnhaselement{объект деятельности}
    \scnhaselement{среда деятельности}
    \scnhaselement{инструменты (инструментальные средства) деятельности}
    \scnhaselement{субъект деятельности}
    \scnhaselement{текущее состояния деятельности (как процесса) --- в том числе, спецификация действий (целей, задач), выполняемых в текущий момент}
    \scnhaselement{формулировка закономерностей --- в том числе, правил поведения субъектов деятельности}
    \scnhaselement{спецификация всех используемых субъектами деятельности методов выполнения сложных действий (решения задач)}
    
   \scnheader{область человеческой деятельности}  
   \scnhaselement{процесс взаимодействия умного дома с его жильцами и посетителями}
   \scnhaselement{процесс взаимодействия умного предприятия, выпускающего определенного вида продукцию, с его сотрудниками}
   \scnhaselement{процесс взаимодействия студентов и преподавателей в рамках умной кафедры, осуществляющей подготовку молодых специалистов по какой-либо инженерной специальности}
   \scnhaselement{процесс взаимодействия постояльцев, посетителей и сотрудников умного отеля}
   \scnhaselement{процесс взаимодействия посетителей и сотрудников умного музея}
   \scnhaselement{процесс взаимодействия пациентов и медицинского персонала умной поликлиники, умной больницы}
   \scnhaselement{процесс взаимодействия граждан и чиновников в рамках умной администрации некоторого региона}
   \scnhaselement{процесс взаимодействия жителей и гостей в рамках умного города}
   \scntext{примечание}{Для комплексной автоматизации человеческой деятельности в целом (Объединенной человеческой деятельности) автоматизации отдельных областей человеческой деятельности явно не достаточно, поскольку тесные связи между различными областями человеческой деятельности требуют автоматизации не только деятельности внутри каждой из этих областей, но и внешней деятельности, обусловленной необходимостью взаимодействия между различными областями деятельности, например, в рамках более крупных областей деятельности. Так, например, каждое предприятие взаимодействует со своими поставщиками и потребителями. Очевидно, что автоматизация такой внешней деятельности и, тем более, автоматизация с использованием интеллектуальных компьютерных систем существенно упрощается, если будут совпадать (будут унифицированы) принципы, лежащие в основе автоматизации каждой области деятельности, а также принципы автоматизации крупных областей деятельности, в состав которых входит некоторое количество более мелких областей человеческой деятельности.
        \\Таким образом, для комплексной автоматизации человеческой деятельности в целом с применением интеллектуальных компьютерных систем и для обеспечения эффективной интеграции различных областей человеческой деятельности необходима разработка \textbf{\textit{Общей формальной теории человеческой деятельности}}, которая объединила бы формальные модели всевозможных областей человеческой деятельности.}
        
    \scnheader{Общая формальная теория человеческой деятельности}
    \scnhaselement{формальная теория видов человеческой деятельности}
     \begin{scnindent}
     	\scntext{примечание}{Поскольку каждый \textbf{\textit{вид человеческой деятельности}} --- это класс однотипных \textit{областей человеческой деятельности}, формальная теория каждого вида человеческой деятельности --- это формальное представление \underline{стандарта} соответствующего класса областей человеческой деятельности. Так, например, можно говорить о формальной модели конкретного предприятия рецептурного производства (например, предприятие Савушкин продукт , выпускающего молочную продукцию), но можно говорить и о формальной теории всего класса предприятий рецептурного производства --- о формальном представлении стандарта ISA-88. Формальная теория каждого вида человеческой деятельности включает в себя формальное описание технологии, обеспечивающей осуществление каждой области (фрагмента) человеческой деятельности, принадлежащей указанному виду деятельности. В описание технологии входит описание используемых методов, средств и основных объектов и субъектов деятельности.}
      \end{scnindent}
     \scnhaselement{иерархическая декомпозицию Объединенной человеческой деятельности по нескольким признакам}
        \begin{scnindent}
     	\scntext{примечание}{Основными признаками такой декомпозиции являются региональный признак и целевая направленность деятельности. По региональному признаку на высшем уровне иерархии выделяются такие области человеческой деятельности, как Деятельность Франции, Деятельность Германии и далее деятельность всех стран. По признаку целевой направленности на высшем уровне иерархии выделяются: Научно-исследовательская деятельность человечества, Проектная деятельность человечества, Производственная деятельность человечества, Образовательная деятельность человечества, Здравоохранительная деятельность человечества, Природоохранная деятельность человечества, Административная деятельность человечества и др. Дальнейшая декомпозиция областей человеческой деятельности по признаку целевой направленности выделяет такие области деятельности, как
            \begin{scnitemizeii}
                \item \textit{Научно-исследовательская деятельность человечества в области Математики}
                \item \textit{Научно-исследовательская деятельность человечества в области Лингвистики}
                \item и др.
            \end{scnitemizeii}
            Заметим при этом, что, в отличие от чисто научных дисциплин, дисциплины научно-технического типа (например, дисциплина \textit{Искусственный Интеллект}) представляют собой симбиоз фрагментов (областей) деятельности, принадлежащих разным видам деятельности:
            \begin{scnitemizeii}
                \item научно-исследовательской деятельности;
                \item деятельности по разработке технологии проектирования (CAD);
                \item деятельности по разработке технологии производства (CAM);
                \item проектная деятельность;
                \item производство спроектированного объекта;
                \item образовательной деятельности;
                \item бизнес-деятельности.
            \end{scnitemizeii}}
        	\begin{scnindent}
        		\scntext{уточнение}{Кроме указанных областей человеческой деятельности выделяются области, соответствующие различным сочетаниям значений указанных признаков декомпозиции областей человеческой деятельности. Примерами таких областей являются: Научно-исследовательская деятельность Франции, Образовательная деятельность Германии. Подчеркнем то, что количество областей человеческой деятельности, выделенных в результате указанной иерархической декомпозиции \textit{Объединенной человеческой деятельности}, является, хоть и не очень большим, но конечным в каждый момент времени.}
        	\end{scnindent}
         \end{scnindent}
         \begin{scnrelfromlist}{задача}
            \scnfileitem{унификация формального описания самых различных технологий для самых различных областей человеческой деятельности}
            \scnfileitem{унификация формального описания всевозможных видов человеческой деятельности}
            \scnfileitem{унификация формального описания связей между различными областями и видами человеческой деятельности, различными субъектами деятельности, объектами, средствами (инструментами)}
            \scnfileitem{глубокая конвергенция всех видов человеческой деятельности, областей человеческой деятельности, используемых методов}
     	\end{scnrelfromlist}
    \begin{scnrelfromvector}{что делать}
        \scnfileitem{Необходим переход от локальной автоматизации различных областей и видов человеческой деятельности путем независимой друг от друга разработки систем автоматизации бизнес-процессов даже близких по виду деятельности предприятий к комплексной автоматизации человеческой деятельности в целом прежде всего для обеспечения совместимости различных областей деятельности и исключения ужасающего и никому не нужного дублирования (многообразия форм) автоматизации аналогичных бизнес-процессов}
        \scnfileitem{Все многообразие человеческой деятельности необходимо четко стратифицировать, доведя эту стратификацию до строгого формального представления}
        \scnfileitem{Необходимо
            \begin{itemize}
                \item четко выделить все виды человеческой деятельности, соответствующие текущему уровню развития человеческого общества
                \item построить четкую иерархию этих видов на основании отношения, связывающего виды человеческой деятельности с их подвидами
                \item унифицировать человеческую деятельность в рамках каждого выделенного вида, разработав соответствующие стандарты, для каждого из которых построить четкую систему используемых понятий
                \item довести указанные стандарты до такого уровня формализации, чтобы они стали частью базы знаний интеллектуальной системы автоматизации соответствующего вида человеческой деятельности.
            \end{itemize}}
        \scnfileitem{Необходимо обеспечить конвергенцию, семантическую совместимость и глубокую интеграцию различных видов и областей человеческой деятельности путем:\\
            \begin{itemize}
                \item согласования систем понятий, соответствующих стандартам разных видов человеческой деятельности, и особенно согласования систем понятий между стандартами видов и подвидов человеческой деятельности
                \item представления стандарта каждого вида человеческой деятельности в виде формальной онтологии
                \item построения такой иерархической системы формальных онтологий, соответствующих всевозможным видам человеческой деятельности, в которой обеспечивалась бы конвергенция и \underline{семантическая совместимость} онтологий, входящих в эту систему, а также \underline{наследование свойств} от онтологии каждого вида человеческой деятельности к онтологии каждого подвида этого вида человеческой деятельности.
            \end{itemize}}
    \end{scnrelfromvector}
    \begin{scnindent}
    	\scntext{следовательно}{Таким образом, в целях повышения эффективности автоматизации человеческой деятельности и, в первую очередь, в целях существенного снижения трудозатрат на такую автоматизацию необходимо с точки зрения общей теории систем фундаментально переосмыслить современную организацию человеческой деятельности, поскольку автоматизация беспорядка приводит к ещё большему беспорядку. На этом пути имеется только одно препятствие --- противодействие лени с высоким уровнем эгоизма, которым современный беспорядок организации человеческой деятельности выгоден.}
    \end{scnindent}
    
\scnheader{Комплексная автоматизация человеческой деятельности в области Искусственного интеллекта с помощью интеллектуальных компьютерных систем нового поколения}
\begin{scnsubstruct}
	\scnheader{Поддержка жизненного цикла интеллектуальных компьютерных систем нового поколения}
    \scntext{примечание}{В рамках \textit{Технологии OSTIS} поддержка жизненного цикла интеллектуальных компьютерных систем нового поколения (\textit{ostis-систем}) осуществляется на основе \textit{Метасистемы OSTIS}, которая относится к классу \textit{ostis-систем} и фактически является формой реализации указанной Технологии.}
    \scntext{принципы, лежащие в основе}{Автоматизация поддержки жизненного цикла \textit{ostis-систем} осуществляется как в форме инструментального обслуживания инженерной деятельности (в частности, Метасистема OSTIS является системой автоматизации проектирования ostis-систем), так и в форме информационного обслуживания указанной деятельности. Для этого база знаний \textit{Метасистемы OSTIS} содержит: 
	\begin{itemize}
		\item текущее состояние полного текста \textit{Стандарта ostis-систем};
		\item Текущее состояние Библиотеки многократно используемых компонентов ostis-систем;
		\item Используемые и реализуемые инженерами методики поддержки жизненного цикла ostis-систем;
		\item Документацию инструментальных средств, инженерами для поддержки жизненного цикла  ostis-систем.
	\end{itemize}}
	
	\scnheader{Метасистема OSTIS}
	\begin{scnrelfromlist}{задача}
		\scnfileitem{Обеспечить автоматизацию \textit{Поддержки жизненного цикла Стандарта ostis-систем}, то есть обеспечивает организацию взаимодействия между авторами этого Стандарта, направленного на перманентное его развитие.}
		\scnfileitem{Обеспечить автоматизацию \textit{Поддержки жизненного цикла Технологии OSTIS}, которая сводится к поддержке жизненного цикла основной части базы знаний \textit{Метасистемы OSTIS}, которая является полной документацией текущего состояния \textit{Технологии OSTIS}.}
	\end{scnrelfromlist}
	
	\scnheader{Человеческая деятельность в области Искусственного интеллекта}
	\scntext{примечание}{Автоматизация направлений \textit{Человеческой деятельности в области Искусственного интеллекта} также может осуществляться с помощью \textit{ostis-систем}, семантически совместимых и взаимодействующих с \textit{Метасистемой OSTIS} в рамках \textit{Экосистемы OSTIS}.}
\end{scnsubstruct}

    \scnheader{следует отличать}
    \begin{scnhaselementset}
        \scnitem{вид человеческой деятельности}
        \begin{scnindent}
            \begin{scnhaselementrolelist}{пример}
                \scnitem{научно-исследовательская деятельность}
                \scnitem{проектирование}
                \begin{scnindent}
                    \scnidtf{проектная деятельность}
                    \scnsuperset{проектирование интеллектуальной компьютерной системы}
                    \begin{scnindent}
                    	\scnsuperset{проектирование ostis-системы}
               		\end{scnindent}
                 \end{scnindent}
            \end{scnhaselementrolelist}
        \end{scnindent}
        \scnitem{область человеческой деятельности}
        \begin{scnindent}
            \begin{scnhaselementrolelist}{пример}
                \scnitem{Научно-исследовательская деятельность в области Искусственного интеллекта}
                \begin{scnindent}
                    \scniselement{научно-исследовательская деятельность}
                    \scnrelfrom{часть}{Научно-исследовательская деятельность РАИИ}
                \end{scnindent}
                \scnitem{Проектирование Метасистемы OSTIS}
                \begin{scnindent}
                    \scniselement{проектирование ostis-системы}
                \end{scnindent}
            \end{scnhaselementrolelist}
        \end{scnindent}
        \scnitem{подвид человеческой деятельности*}
        \begin{scnindent}
            \scnsubset{включение*}
            \begin{scnindent}
            	\scnidtf{подмножество*}
            \end{scnindent}
        \end{scnindent}
        \scnitem{подобласть человеческой деятельности*}
        \begin{scnindent}
            \scnsubset{часть*}
        \end{scnindent}
    \end{scnhaselementset}

    \scnheader{информационная технология}
    \scntext{пояснение}{Множество технологий, связанных с проектированием и производством компьютерных систем и их компонентов, с эксплуатацией компьютерных систем, а также с их использованием в качестве инструмента в составе самых различных технологий. В рамках различных информационных технологий компьютерные системы рассматриваются как инструментальные средства, как вспомогательные субъекты, обеспечивающие автоматизацию соответствующих видов деятельности. Но в некоторых информационных технологиях компьютерные системы являются также и \underline{объектами} автоматизируемых видов деятельности. Примерами таких технологий являются:
        \begin{itemize}
            \item технология проектирования компьютерных систем;
            \item технология реализации (сборки) компьютерных систем;
            \item технология обновления компьютерных систем.
        \end{itemize}}
    \scnhaselement{Комплекс современных информационных технологий}
    \scnhaselement{Комплекс современных технологий искусственного интеллекта}
    \scnhaselement{Технология OSTIS}

    \scnheader{автоматизация человеческой деятельности}
    \scnidtf{человеческая деятельность, направленная на повышения уровня автоматизации человеческой деятельности, а также на повышение качества (в том числе, уровня интеллекта) человеческого общества как многоагентной кибернетической системы}
    \scnsubset{вид человеческой деятельности}
    \scntext{примечание}{Важнейшим этапом автоматизации человеческой деятельности в перспективе должен стать переход к существенно более высокому уровню \textit{интеллекта человеческого общества} как целостной кибернетической системы путем преобразования современного человеческого общества в сообщество взаимодействующих между собой людей и интеллектуальных компьютерных систем. Такое сообщество иногда называют smart-обществом, обществом 5.0.\\
        Особо подчеркнем то, что переход к такому интеллектуальному обществу требует существенного переосмысления современной организации различных видов человеческой деятельности. Прежде всего, следует подчеркнуть, что эффективность (коэффициент полезного действия) современной организации человеческой деятельности в целом ужасающе низка, а, как известно, автоматизация беспорядка (даже с помощью интеллектуальных компьютерных систем) приводит к ещё большему беспорядку.}
    
    \scnheader{вид человеческой деятельности}
    \scnidtf{Множество всевозможных видов человеческой деятельности}
    \scnrelfrom{разбиение}{Разбиение Множества видов человеческой деятельности по степени их автоматизируемости}
    \begin{scnindent}
	    \begin{scneqtoset}
	        \scnitem{вид человеческой деятельности, который принципиально может быть автоматизирован полностью}
	        \begin{scnindent}
	            \scnidtf{Множество полностью автоматизируемых видов человеческой деятельности}
	        \end{scnindent}
	        \scnitem{вид человеческой деятельности, который может быть автоматизирован только частично}
	        \begin{scnindent}
	            \scnidtf{Множество частично автоматизируемых видов человеческой деятельности}
	        \end{scnindent}
	        \scnitem{вид человеческой деятельности, который принципиально никак не может быть автоматизирован}
	        \begin{scnindent}
	            \scnidtf{Множество неавтоматизируемых видов человеческой деятельности, которые могут быть выполнены только вручную\ (точнее самими людьми с возможным использованием каких-либо пассивных\ инструментов --- топора, лопаты и т. п.)}
	        \end{scnindent}
	    \end{scneqtoset}
    \end{scnindent}
    
    \scnheader{вид человеческой деятельности, который принципиально может быть автоматизирован полностью}
    \scntext{примечание}{Есть виды человеческой деятельности, которые принципиально могут быть автоматизированы \underline{полностью}, но в текущий момент эта автоматизация не полна. Это ,например, частично автоматизированная деятельность по производству спроектированных искусственных объектов. Здесь важна \underline{четкость} распределения обязанностей между различными средствами автоматизации и поэтапное исключение неавтоматизированных действий, вручную выполняемых людьми (например, сотрудниками производственных предприятий), т. е. поэтапная автоматизация этих действий.}

    \scnheader{автоматизация человеческой деятельности}
    \scntext{вопрос}{Почему для комплексной автоматизации человеческой деятельности целесообразно использовать семантически совместимые, договороспособные, самостоятельные интеллектуальные компьютерные системы, которым можно делегировать права на принятие некоторых решений.}
    \scntext{вопрос}{Почему для повышения уровня комплексной \textit{автоматизации человеческой деятельности} необходим переход от современных (традиционных) \textit{компьютерных систем} и соответствующих им информационных технологий, а также от \textit{современных интеллектуальных компьютерных систем} и соответствующих им современных технологий искусственного интеллекта к \textit{интеллектуальным компьютерным системам} \uline{нового поколения} и к соответствующей им Комплексной технологии проектирования таких систем.}
    \begin{scnindent}
    \scntext{ответ}{В силу отсутствия унификации представления обрабатываемой информации в традиционных компьютерных системах и, как следствие, отсутствия совместимости этих систем как на синтаксическом уровне, так и на семантическом уровне, принциально не существует универсального метода системной интеграции традиционных компьютерных систем и, следовательно, невозможна полная автоматизация решения этой задачи. Системная интеграция традиционных компьютерных систем практически всегда осуществляется вручную с учетом индивидуальной специфики каждой интегрируемой системы и, следовательно, является весьма трудоемкой и требующей высокой квалификации разработчиков.
        \\Но на данном этапе эволюции компьютерных систем крайне актуальной является полная автоматизация их интеграции без какого бы то ни было участия разработчиков и тем более конечных пользователей. Если компьютерные системы не приобретут способность \uline{самостоятельно} взаимодействовать между собой в целях решения сложных комплексных задач, то эффективность использования человечеством интенсивно расширяемого многообразия весьма полезных и качественно реализованных информационных ресурсов и сервисов будет весьма низкой. Традиционно компьютерные технологии позволяют реализовать \uline{любую} модель обработки информации (в том числе и \uline{любую} интеллектуальную модель --- нейросетевую, логическую и т.д.). Однако актуальным является не реализация самих этих моделей, а их интеграция, что требует обоспечения синтаксически и семантически совместимых компьютерных систем и полной автоматизации их системной интеграции. Следовательно необходим переход на принципиально новое поколение компьютерных технологий и, в частности, на принципиально новое поколение самих компьютеров, ориентированных на решение проблем совместимости компьютерных систем и полной автоматизации их системной интеграции.
        \\Таким образом, дальнейшее повышение уровня автоматизации различных видов человеческой деятельности потребует перехода на принципиально новый уровень информационных технологий --- от современных (традиционных) \textit{компьютерных систем} к компьютерным системам, имеющим существенно более \textit{высокий уровень интеллекта} и способным не только индивидуально решать достаточно сложные (в том числе, интеллектуальные) задачи, но и эфффективно \uline{самостоятельно} взаимодействовать между собой, координируя свою деятельность при решении задач, принадлежащих априори неизвестным (заранее не предусмотренным) классам задач и требующих коллективного (корпоративного) решения.
        \\Основные проблемы автоматизации \textit{человеческой деятельности} в настоящее время лежат не в области разработки средств автоматизации решения различных конкретных классов задач (в том числе и весьма сложных, интеллектуальных, труднорешаемых задач), а в области системной интеграции этих средств в комплексы, компоненты которых способны самостоятельно кооперироваться для совместного (коллективного) решения сложных задач. Но для этого указанные компоненты должны уметь согласовывать, координировать свои действия, должны понимать друг друга, должны быть семантически совместимы.}
    \end{scnindent}
    
    \scnheader{интеллектуальная компьютерная система}
    \scntext{примечание}{Различные интеллектуальные компьютерные системы могут быть эффективно использованы в качестве средств автоматизации самых различных видов человеческой деятельности. Но, поскольку все виды человеческой деятельности взаимосвязаны (как минимум потому, что каждый человек может одновременно участвовать сразу в нескольких видах деятельности, причем в разные моменты времени этот набор видов деятельности для каждого человека может быть различным), интеллектуальная компьютерная система автоматизации каждого вида человеческой деятельности должна эффективно взаимодействовать с другими интеллектуальными компьютерными системами, осуществляющими автоматизацию других видов человеческой деятельности. Другими словами, необходимо переходить от автоматизации отдельных видов человеческой деятельности к автоматизации комплекса всех видов человеческой деятельности. Для этого необходимо:
        \begin{itemize}
            \item не просто достаточно детально разработать \uline{теорию каждого вида деятельности}, выделив (1) все классы автоматизируемых действий, (2) все классы неавтоматизируемых действий, (3) соответствующие этим классам методы выполнения действий (в частности, это могут быть обобщенные бизнес-процессы), которые по сравнению с используемыми в настоящий момент могут потребовать существенного реинжиниринга бизнес-процессов;
            \item но и представить эти теории в формализованном и унифицированном виде в качестве фрагментов баз знаний соответствующих интеллектуальных компьютерных систем, обеспечив при этом высокую степень конвергенции этих теорий.
        \end{itemize}}
    \begin{scnrelfromlist}{возможное амплуа}
        \scnitem{средство автоматизации проектирования}
        \begin{scnindent}
            \scnsuperset{средство автоматизации проектирования интеллектуальных компьютерных систем}
            \begin{scnindent}
            	\scntext{примечание}{В силу большой сложности процесса проектирования интеллектуальных компьютерных систем для автоматизации этого процесса необходимо использовать именно интеллектуальные компьютерные системы.}
            \end{scnindent}
        \end{scnindent}
        \scnitem{средство автоматизации производства}
        \scnitem{средство повышения качества эксплуатации сложного объекта}
        \begin{scnindent}
            \scnsuperset{средство help-поддержки конечных пользователей}
            \scnsuperset{средство управления процессом повышения качества деятельности конечных пользователей}
            \scnsuperset{средство поддержки оптимальных эксплуатационных свойств эксплуатируемого объекта}
            \begin{scnindent}
            	\scntext{пояснение}{Здесь имеется в виду мониторинг состояния эксплуатируемого объекта, контроль условий эксплуатации, своевременная профилактика и ремонт.}
            \end{scnindent}
            \scnsuperset{средство поддержки совершенствования эксплуатируемого объекта в ходе его эксплуатации}
            \scntext{примечание}{Для интеллектуальных компьютерных систем все средства повышения качества их эксплуатации целесообразно встраивать в эти системы. Имеется в виду слияние нескольких интеллектуальных компьютерных систем в одну интегрированную.}
        \end{scnindent}
        \scnitem{средство автоматизации научно-исследовательской деятельности в рамках заданной научной дисциплины}
        \scnitem{средство автоматизации образовательной деятельности}
        \begin{scnindent}
            \scntext{примечание}{Автоматизация образовательной деятельности может осуществляться в рамках:
                \begin{itemize}
                    \item заданной учебной дисциплины
                    \item заданной учебной специальности
                    \item заданного учебного заведения
                    \item заданного государства.
                \end{itemize}}
        \end{scnindent}
        \scnitem{средство автоматизации бизнес-деятельности в заданной научно-технической области}
        \begin{scnindent}
        	\scntext{примечание}{Здесь важна автоматизация контроля за реализацией всех направлений организационной деятельности с учетом разработанных и постоянно уточняемых и корректируемых планов, а также с учетом согласованных приоритетов.}
        \end{scnindent}
        \scnitem{средство автоматизации деятельности в области здравоохранения}
        \scnitem{средство автоматизации административной деятельности}
        \scnitem{средство автоматизации деятельности жилищно-коммунального хозяйства}
        \scnitem{юриспруденция}
        \scnitem{правоохранительная деятельность}
        \scnitem{транспорт}
    \end{scnrelfromlist}
    
    \scnheader{Экосистема OSTIS}
    \scntext{примечание}{\textit{Экосистема OSTIS} является основой для перевода уровня информатизации различных областей человеческой деятельности на принципиально новый уровень, а также для интеграции соответствующих проектов --- Общество 5.0, Industry 4.0, University 3.0, Умный дом, Умный город и других (без интеллектуальных компьютерных систем все эти проекты невозможны).
        \\Все эти проекты должны быть приведены в единую стройную иерархическую систему взаимосвязанных проектов, охватывающих весь объем и многообразие человеческой деятельности.}
  
        \scnheader{Экосистема OSTIS}
        \scntext{вопрос}{Какие достоинства имеет Экосистема OSTIS}
        \scntext{достоинство}{Важнейшей особенностью Экосистемы OSTIS является то, что входящие в нее \textit{ostis-системы} благодаря высокому уровню их интеллекта и, в частности, высокому уровню их социализации, становятся самостоятельными, активными и полноправными субъектами, участвующими в реализации самых различных видов человеческой деятельности, что существенно повышает уровень её автоматизации.}
        \begin{scnrelfromset}{Что такое интеллектуальная система}
            \scnitem{\uline{система}(!)свойств}
            \scnitem{В.К. Финн}
            \scnitem{требования, предъявленные к интеллектуальным компьютерным системам}
        \end{scnrelfromset}
        \begin{scnrelfromset}{достоинства}
            \scnfileitem{семантическая совместимость
                \begin{itemize}
                    \item интеллектуальных компьютерных систем между собой
                    \item интеллектуальных компьютерных систем с их пользователями и разработчиками
                    \item семантическая совместимость = взаимопонимание
                \end{itemize}}
            \scnfileitem{Перманентная поддержка семантической совместимости}
            \scnfileitem{Способность координировать свои действия (договороспособность, координация(!)) при коллективном решении задач автоматизации \textbf{системной интеграции} интеллектуальных компьютерных систем, ручная реализация системной интеграции --- главный тормоз комплексной автоматизации. Многоагентная система из интеллектуальных компьютерных систем + \textbf{людей}.
                \begin{itemize}
                    \item Каждая ostis-система является, кроме всего прочего, способной обучать(повышать квалификацию) своих пользователей т.е. повышать эффективность своей эксплуатации
                    \\ostis-система
                    \\\scnsubset{интеллектуальная обучающая система}
                \end{itemize}}
            \scnfileitem{Экосистема интеллектуальных компьютерных систем
                \\\scneq{комплексная автоматизация человеческой деятельности}
                Достоинства Экосистемы(преимущества) и перспективы создания и развития Технологии OSTIS
                \begin{itemize}
                    \item Технология OSTIS как основа эволюции человеческого общества => переход к smart-обществу, к более интеллектуальному обществу
                    \item Экосистема OSTIS как продукт Технологии, т.е продукт технологии --- не отдельные интеллектуальные компьютерные системы, а целая Экосистема
                \end{itemize}
                это вариант smart-общества
                \\smart-предприятие, smart-город
                \\Требования к технологии разработки интеллектуальных компьютерных систем
                \begin{itemize}
                    \item smart-сообщество разработчиков интеллектуальных компьютерных систем и разработчиков самой технологии
                    \item консорциум!!
                    \item стандарты интеллектуальных компьютерных систем
                    \item стандарты процесса разработки различных интеллектуальных компьютерных систем
                    \item стандарты процесса совершенствования самой технологии
                    \item ориентация на новые компьютеры
                \end{itemize}
                Экосистема OSTIS -> цель
                \\\scneq{smart-общество = общество 5.0 как интеграция всевозможных специализированных smart-сообществ(...)}}
            \scnfileitem{конвергенция в области Искусственного интеллекта и не только(!!)(это необходимо для Экосистемы интеллектуальных компьютерны систем)}
            \scnfileitem{глубокая интеграция(совместимость)}
            \scnfileitem{новое поколение компьютеров}
            \scnfileitem{Консорциум OSTIS по разработке глобального комплекса семантически совместимых технологий, обеспечивающих комплексную автоматизацию всевозможных видов человеческой деятельности(для Экосистемы интеллектуальных компьютерных систем).Достоинства эволюции интеллектуальных компьютерных систем автоматизации проектирования интеллектуальных компьютерных систем распространяется на все технические дисциплины и соответственно сообщества.}
            \scnfileitem{система Проектов OSTIS, реализуемых консорциумом OSTIS --- как прообраз project-management нового типа, ориентированного на реализацию \uline{наукоемких} проектов с децентрализованным управлением и с перманентным коллективным уточнением и детализацией целей}
            \scnfileitem{\textbf{Рынок знаний и его реализация}
                \begin{itemize}
                    \item Смысловые представления знаний и глобальный характер минимизирует субъективизм, предвзятость, более эффективно защищает авторские права
                    \item Тормозит научно-техническое развитие(прогресс) становиться труднее
                    \item Выигрывает тот, кто действительно способствует прогрессу, а не тормозит его
                \end{itemize}
                Нет центральных и периферийных публикаций --- есть общая база знаний, в которой нет семантической эквивалентности(и, следовательно, нет плагиата). Монография OSTIS + \textbf{Метасистема OSTIS} как новый уровень автоматизации создания и эволюции научно-технического каталога статей и монографий к семантически совместимым базам и порталам научно-технических знаний! Достоинства эволюции портала научных знаний по Искусственному интеллекту и соответственно сообщества ученых распространяется на все научные дисциплины}
        \end{scnrelfromset}
    
\scnheader{Комплексная автоматизация всевозможных видов и областей человеческой деятельности с помощью и.к.с.}
\begin{scnsubstruct}
\scntext{примечание}{В предметной области рассмотрено то, как осуществляется и автоматизируется с помощью интеллектуальных компьютерных систем нового поколения \uline{весь комплекс} \textit{Человеческой деятельности в области Искусственного интеллекта}. Сейчас обобщим это и рассмотрим принципы организации и комплексной автоматизации \textit{человеческой деятельности} в целом, то есть автоматизации самых различных видов и областей человеческой деятельности.}

\scnheader{Общие принципы систематизации человеческой деятельности и ее комплексной автоматизации с помощью интеллектуальных компьютерных систем нового поколения}
\begin{scnsubstruct}
\scnheader{Почему можно обобщить опыт комплексной организации, структуризации и автоматизации \textit{человеческой} деятельности в области \textit{Искусственного интеллекта}?}
\scnidtf{Почему можно обобщить опыт комплексной организации, структуризации и автоматизации \textit{человеческой} деятельности в области создания и сопровождения интеллектуальных компьютерных систем?}
\scniselement{вопрос}
\begin{scnrelfromvector}{ответ}
	\scnfileitem{Потому, что человеческая деятельность, направленная на поддержку всего \textit{жизненного цикла интеллектуальных компьютерных систем} нового поколения, является \uline{частной} \uline{областью деятельности} по отношению к виду человеческой деятельности, направленному на (обеспечивающему) поддержку всего жизненного цикла \uline{любой искусственной} (искусственно создаваемой) сущности (любого артефакта). В зависимости от сложности искусственно создаваемой сущности, уровень сложности человеческой деятельности, направленной на поддержку жизненного цикла этой сущности, может быть самым различным, но общая структура этой деятельности, соответствующая различным этапам жизненного цикла искусственно создаваемых сущностей, а также необходимым направлениям \uline{обеспечения} этой инженерной деятельности является одинаковой для искусственных сущностей различных классов.}
	\begin{scnindent}
        \begin{scnrelfromvector}{направления обеспечения поддержки жизненного цикла искусственных сущностей}
            \scnfileitem{Научно-исследовательская деятельность, направленная на изучение искусственных сущностей соответствующего класса.}
            \scnfileitem{Разработка стандарта искусственных сущностей указанного класса.}
            \scnfileitem{Разработка технологии поддержки искусственных сущностей указанного класса.}
            \scnfileitem{Подготовка кадров, способных осуществлять поддержку жизненного цикла искусственных сущностей указанного класса, то есть способных эффективно использовать указанную выше технологию.}
            \scnfileitem{ПодготоВо-первых, пвка кадров, способных участвовать в указанной выше научно-исследовательской деятельности.}
            \scnfileitem{Подготовка кадров, способных участвовать в разработке стандарта искусственных сущностей заданного класса.}
            \scnfileitem{Подготовка кадров, способных участвовать в разработке и развитии указанной выше технологии.}
            \scnfileitem{Организационное обеспечение всего комплекса работ по развитию и использованию указанной технологии.}
        \end{scnrelfromvector}
    \end{scnindent}
	\scnfileitem{Потому, что многие сложные технические системы фактически становятся \textit{интеллектуальными компьютерными системами} (в том числе распределенными) с различными наборами сенсорных и эффекторных подсистем --- интеллектуальными автомобилями с автопилотом и автоштурманом, интеллектуальными заводами-автоматами, умными домами, умными городами и тому подобными.}
	\scnfileitem{Потому, что характер деятельности \textit{интеллектуальных компьютерных систем нового поколения} и характер деятельности каждого \textit{человека} и каждой организации по сути мало чем отличаются, поскольку \textit{интеллектуальные компьютерные системы нового поколения} становятся \uline{равноправными} партнерами (субъектами) \textit{человеческой деятельности}, так как уровень их самостоятельности, ответственности, интероперабельности и интеллектуальности приближается к соответствующим качествам \textit{естественных} субъектов человеческой деятельности (физических лиц, юридических лиц, подразделений крупных организаций, неформальных организаций).}
\end{scnrelfromvector}

\scnheader{Человеческая деятельность в области Искусстенного интеллекта}
\scntext{примечание}{Итак, структуризацию \textit{человеческой деятельности} в области \textit{Искусственного интеллекта} на основе понятий \textit{вида деятельности}, \textit{области деятельности}, \textit{продукта деятельности} (объекта деятельности) можно легко обобщить для всех \textit{научно-технических дисциплин}, что дает возможность рассматривать автоматизацию деятельности в рамках всех \textit{научно-технических дисциплин} с общих позиций, так как автоматизация различных \textit{видов деятельности} в рамках различных \textit{научно-технических дисциплин} может выглядеть аналогичным образом, а иногда может быть реализована с помощью одной и той же \textit{интеллектуальной компьютерной системы}. Так например, любая \textit{интеллектуальная компьютерная система автоматизации проектирования} технических систем заданного вида может быть построена на основе \textit{интеллектуальной компьютерной системы автоматизации проектирования и реинжиниринга баз знаний}, поскольку результатом проектирования любой \textit{технической системы} является формальная модель (описание, спецификация, документация) этой \textit{технической системы}, обладающая достаточно полнотой для воспроизводства (реализации) этой системы.}

\scnheader{Искусственного интеллекта}
\scntext{задача}{На текущем этапе развития \textit{Искусственного интеллекта} необходимо переходить от автоматизации отдельных \textit{видов человеческой деятельности} к интегрированной автоматизации всего комплекса \textit{человеческой деятельности}, к созданию и постоянной эволюции всей \textbf{\textit{Глобальной экосистемы интеллектуальных компьютерных систем}}, самостоятельно взаимодействующих как между собой, так и с людьми, автоматизацию деятельности которых они осуществляют, а также с современными компьютерными системами, не являющимися интеллектуальными системами. При этом надо помнить, что основные \scnqqi{накладные} расходы, основные проблемы, возникают на \scnqqi{стыках} при интеграции различных технических решений. Разработчик каждой подсистемы должен гарантировать отсутствие указанных \scnqqi{накладных} расходов. При этом необходимо подчеркнуть, что следует ориентироваться не столько на создание эффективной \textit{Глобальной экосистемы интеллектуальных компьютерных систем}, сколько на создание эффективных методик и средств, направленных на \textit{перманентную эволюцию} такой \textit{экосистемы}.}

\scnheader{Методика автоматизации человеческой деятельности в области Искусстенного интеллекта}
\begin{scnrelfromvector}{этапы}
	\scnfileitem{Построение общей \textbf{\textit{структуры человеческой деятельности}}, в основе которой лежит иерархия \textit{человеческой деятельности} по видам деятельности и продуктам деятельности с четкой фиксацией различного вида связей между различными компонентами этой структуры.}
	\scnfileitem{Формализация различных \textit{видов человеческой деятельности}.}
	\scnfileitem{Разработка \textbf{\textit{технологии}}, обеспечивающей максимально возможную автоматизацию этой деятельности с помощью \textit{интеллектуальных компьютерных систем нового поколения}.}
	\scnfileitem{Обеспечение максимально возможной \textbf{\textit{конвергенции}} различных \textit{видов деятельности}, что позволит сократить многообразие средств автоматизации (то есть соответствующих \textit{интеллектуальных компьютерных систем нового поколения}).}
	\scnfileitem{Обеспечение максимально возможной \textbf{\textit{конвергенции технологий}} выполнения одного и того же \textit{вида деятельности} для разных объектов деятельности (конвергенции технологий проектирования объектов различных классов, конвергенции технологий мониторинга, профилактики и диагностики для агентов различных классов и так далее) и, тем самым, обеспечить \textbf{\textit{конвергенцию}} соответствующих средств автоматизации, построенных на основе \textit{интеллектуальных компьютерных систем нового поколения}.}
\end{scnrelfromvector}

\end{scnsubstruct}

\scnheader{Многообразие видов человеческой деятельности и связей между ними}
\begin{scnsubstruct}

    \scnheader{вид человеческой деятельности}
    \scniselement{вид деятельности}
    \scnhaselementrole{класс объектов}{класс всевозможных социально значимых объектов, на которые имеет смысл воздействовать}
    \begin{scnindent}
    	\scnidtf{класс всевозможных социально значимых объектов, поддержку жизненного цикла которых целесообразно осуществлять}
    \end{scnindent}

	\scnheader{поддержка жизненного цикла}
	\scnidtf{поддержка жизненного цикла социально значимых сущностей}
    \scntext{примечание}{Базовым видом человеческой деятельности можно считать \textbf{\textit{поддержку жизненного цикла}} различных сущностей.}
	\scniselement{вид деятельности}
	\begin{scnrelfromlist}{частный вид деятельности, выполняемой на некотором этапе}
		\scnitem{проектирование}
		\scnitem{производство}    
		\scnitem{начальное обучение}  
		\begin{scnindent}
			\scnidtf{настройка}
		\end{scnindent}
		\scnitem{мониторинг качества}
		\begin{scnindent}
			\scnidtf{плановое обследование и диагностика}
		\end{scnindent}
		\scnitem{восстановление требуемого уровня качества}
		\begin{scnindent}
			\scnidtf{ремонт, лечение}
		\end{scnindent}
		\scnitem{реинжиниринг}
		\begin{scnindent}
			\scnidtf{обновление, совершенствование}
		\end{scnindent}
		\scnitem{обеспечение безопасности}
		\scnitem{использование}
		\begin{scnindent}
			\scnidtf{эксплуатация, употребление}
		\end{scnindent}
	\end{scnrelfromlist}
	\begin{scnrelfromlist}{частный вид деятельности над подклассом объектов деятельности}
		\scnitem{научно-исследовательская деятельность}
		\begin{scnindent}
			\scnidtf{поддержка жизненного цикла научных теорий}
			\scnrelfrom{класс объектов деятельности}{научная теория}
		\end{scnindent}
		\scnitem{стандартизация}
		\begin{scnindent}
			\scnidtf{поддержка жизненного цикла стандартов}
			\scnrelfrom{класс объектов деятельности}{стандарт}
		\end{scnindent}
		\scnitem{поддержка жизненного цикла технологий}
		\begin{scnindent}
			\scnrelfrom{класс объектов деятельности}{технология}
		\end{scnindent}
		\scnitem{образовательная деятельность}
		\begin{scnindent}
			\scnidtf{учебная деятельность}
			\scnidtf{поддержка жизненного цикла кадровых ресурсов}
			\scnrelfrom{класс объектов деятельности}{кадровый ресурс}
		\end{scnindent}
		\scnitem{поддержка жизненного цикла метасистем комплексного управления поддержкой и обеспечение поддержки жизненного цикла сущностей соответствующих классов}
		\begin{scnindent}
			\scnrelfrom{класс объектов деятельности}{метасистема комплексного управления поддержкой и обеспечением поддержки жизненного цикла сущностей соответствующих классов}
		\end{scnindent}
	\end{scnrelfromlist}

\scnheader{Человеческая деятельность в области Искусственного интеллекта}
\scntext{примечание}{Общая структура \textit{человеческой деятельности} рассматривается путём обобщения структуры \textit{Человеческой деятельности в области Искусственного интеллекта}}

\scnheader{поддержка жизненного цикла}
\scntext{примечание}{\textit{поддержка жизненного цикла} различных социально значимых объектов является особым видом \textit{человеческой деятельности}. 
	\\Во-первых, эффективность \textit{Человеческой деятельности} в целом зависит (1) от длительности социально полезной (активной) фазы жизненного цикла используемых объектов и (2) от объема затрат общества на поддержание необходимых социально полезных свойств используемых объектов. 
	\\Во-вторых, характер и \textit{технология} поддержки жизненного цикла разных видов социально значимых объектов могут существенно отличаться друг от друга. Так, например, существенно отличается организация поддержки жизненного цикла автомобилей, традиционных компьютерных систем различного назначения, современных интеллектуальных компьютерных систем, интероперабельных интеллектуальных компьютерных систем, людей, предприятий, домов, различных юридических лиц, населенных пунктов и других. При этом типология социально значимых объектов, жизненный цикл которых должен поддерживаться, включает в себя самые разнообразные классы объектов - искусственно создаваемые материальные информационные продукты человеческой деятельности, всех людей, всевозможные социальные сообщества и предприятия. Многообразие типов социально значимых объектов порождает многообразие соответствующих им технологий, что усложняет комплексную автоматизацию человеческой деятельности в целом.}
\scntext{примечание}{Заметим, что \textit{видов человеческой деятельности} значительно меньше, чем \textit{областей человеческой деятельности}. Это в определенной степени обусловлено тем, что видов связей между сущностями (относительных понятий) значительно меньше, чем классов различных сущностей. Данное обстоятельство указывает на то, что в основе движения в направление глобальной автоматизации деятельности \textit{общества} должна лежать ориентация на грамотную систематизацию \textit{видов человеческой деятельности}, и на их максимально глубокую \textit{конвергенцию} (как внутри каждого вида деятельности, так и между различными видами). Благодаря этому искусственно привносимое многообразие средств автоматизации \textit{человеческой деятельности} может быть сведено к минимуму.}

\scnheader{следует отличать}
\begin{scnhaselementset}
	\scnitem{научно-исследовательская деятельность}
	\begin{scnindent}
		\scnidtf{поддержка жизненного цикла научных теорий}
	\end{scnindent}
	\scnitem{стандартизация}
	\begin{scnindent}
		\scnidtf{разработка и развитие стандартов}
		\scnidtf{поддержка жизненного цикла стандартов}
	\end{scnindent}
	\scnitem{поддержка жизненного цикла технологий}
\end{scnhaselementset}

\scnheader{научно-исследовательская деятельность}
\scnrelfrom{результат}{Общая теория сущностей заданного класса}
\scntext{определение}{\textit{Научно-исследовательская деятельность} направлена на \textbf{\textit{изучение сущностей заданного класса}}, на изучение принципов, лежащих в основе их структуры и функционирования. В рамках этого вида деятельности важна новизна и конкуренция идей и подходов, важно соотношение между структурой (архитектурой), организацией функционирования исследуемых \textit{сущностей} и общими характеристиками (параметрами) качества этих сущностей, общими предъявляемыми к ним требованиями.
	\\Продуктом рассматриваемой деятельности является \textit{Общая теория сущностей заданного класса}, которая отражает множественность и даже \uline{противоречивость} разных точек зрения и важнейшим направлением развития (эволюции) которой является сближение (\textit{конвергенция}) различных точек зрения и обеспечение совместимости и непротиворечивости между ними.
	\\В основе \textit{научно-исследовательской деятельности} лежит конкуренция точек зрения, принципиальная новизна идей и верифицированных результатов, направленных на выявление и обоснование неочевидных свойств и закономерностей соответствующей \textit{предметной области}, на разработку методов решения различных \textit{классов задач}, решаемых в рамках этой \textit{предметной области}. Цель \textit{научно-исследовательской деятельности} --- требуемая детализация вырабатываемых знаний об объектах исследований соответствующих \textit{предметных областей}.}

\scnheader{стандартизация}
\scntext{определение}{В отличие от \textit{научно-исследовательской деятельности} в основе разработки \textit{стандарта} создаваемых сущностей и разработки соответствующей \textit{технологии} поддержки их жизненного цикла лежит \uline{согласование} различных точек зрения (поиск консенсуса) и максимально возможное их \uline{упрощение} (соблюдение \textit{Принципа Бритвы Оккама}). Необходимость такой методологической установки обусловлена массовым характером \textit{человеческой деятельности} по созданию и \textit{поддержке жизненного цикла} соответствующего класса сущностей и необходимостью вовлечения в эту деятельность людей с \textit{разной} (в том числе и достаточно низкой) квалификацией.
	\\В процессе \textit{разработки стандарта сущностей заданного класса} важна не конкуренция различных точек зрения, а их \textit{конвергенция}, \textit{семантическая совместимость} и глубокая интеграция. Каждый \textit{стандарт} \textit{искусственных сущностей заданного класса} --- это согласованная \uline{на текущий момент} точка зрения (консенсус) о структуре, функционировании, свойствах и закономерностях искусственных сущностей заданного класса, согласованная (общепризнанная) часть \textit{Общей теория искусственных сущностей заданного класса}, доступная для понимания широкому контингенту практиков (инженеров), которые проектируют, производят и поддерживают весь жизненный цикл конкретных \textit{искусственных сущностей заданного класса}.}

\scnheader{жизненный цикл технологий}
\scntext{примечание}{При создании и \textit{поддержке жизненного цикла технологий} должны учитываться ряд требований, предъявляемых к \uline{любым} \textit{технологиям}.}
\begin{scnindent}
	\begin{scnrelfromlist}{требование}
		\scnitem{комплексность}
		\begin{scnindent}
			\scntext{пояснение}{максимально возможное покрытие всех задач, которые должны решаться с помощью \textit{технологии} (как минимум всех этапов жизненного цикла)}
		\end{scnindent}
		\scnitem{простота}
		\begin{scnindent}
			\scntext{пояснение}{максимально возможная простота в использовании \textit{технологии} (необходимая полнота документации, интеллектуальная help-поддержка, отсутствие лишней информации, которая не является необходимой для использования \textit{технологии}, наличие богатой и систематизированной библиотеки типовых многократно используемых решений)}
		\end{scnindent}
	\end{scnrelfromlist}
\end{scnindent}

\scnheader{общество}
\scntext{определение}{общество --- это иерархическая система взаимодействующих индивидуальных и коллективных субъектов, каждый из которых:
\begin{itemize}
	\item Производит либо часть социально значимой продукции, производимой коллективным субъектом, в состав которого входит данный субъект, либо целостный социально-значимый продукт (производимый товар), потребляемый другими внешними субъектами или оказывает некоторую услугу другому субъекту, направленную на обеспечение жизнедеятельности и совершенствование этого другого субъекта.
	\item Потребляет продукцию, произведенную другими субъектами, необходимую для производства собственной продукции (сырье и оборудование), а также необходимую для обеспечения своей жизнедеятельности.
	\item Потребляет услуги, оказываемые другими субъектами необходимые для производства собственной продукции или услуг, а также необходимые для совершенствования своей деятельности.
\end{itemize}}

\scnheader{автоматизация человеческой деятельности}
\begin{scnrelfromlist}{направление}
	\scnfileitem{автоматизация социально полезной профессиональной деятельности всех субъектов деятельности (как индивидуальных субъектов --- всех физических лиц, так и всевозможных коллективных --- корпоративных субъектов, в том числе юридических лиц).}
	\scnfileitem{автоматизация обеспечения (создания) комфортных условий для всех субъектов деятельности общества на основе мониторинга деятельности и конкретного (адаптированного) содействия эволюции каждого субъекта с учетом его непосредственных потребностей и проблем.}
\end{scnrelfromlist}
\scntext{текущее состояние}{Организация взаимодействий каждого субъекта с внешней средой должна осуществляться как со стороны этого субъекта, так и со стороны указанной внешней среды (то есть со стороны общества). \textit{общество} должно повернуться \scnqqi{лицом} к каждому субъекту и не бросать его на произвол судьбы. В настоящее время создание (обеспечение) условий субъектов деятельности общества отдано на откуп каждого такого субъекта. Общество в лице специально предназначенных для этого других субъектов оказывает услуги и снабжает товарами \uline{по заказу} (по инициативе) нуждающегося в этом субъекта. Таким образом, ответственность за развитие каждого субъекта деятельности ложится исключительно на \scnqqi{плечи} этого субъекта. Поддержка общества носит общий характер и никак не учитывает особенности текущего положения каждого субъекта.
    \\Важнейшей причиной, препятствующей дальнейшему повышению общего уровня автоматизации человеческой деятельности является то, что автоматизация различных областей человеческой деятельности осуществляется \uline{локально}.
    \\На современном этапе применения интеллектуальных компьютерных систем основной проблемой является не автоматизация локальных видов и областей человеческой деятельности, а автоматизация комплексных процессов человеческой деятельности, требующая \textit{интеграции} в априори \uline{непредсказуемых} комбинациях самых различных информационных ресурсов и самых различных автоматизированных сервисов, реализуемых в виде специализированных интеллектуальных компьютерных систем.
    \\Локальность автоматизации человеческой деятельности приводит к тому, что вся человеческая деятельность приобретает облик \scnqqi{архипелага}, состоящего из хорошо автоматизированных \scnqqi{островов}, но соединяемых между собой \scnqqi{вручную}. Это \scnqqi{ручное} не автоматизируемое соединение указанных \scnqqi{островов} полностью зависит от человеческого фактора и квалификации соответствующих исполнителей.
    \\Указанное \scnqqi{ручное} соединение некоторого множества семантически близких автоматизированных областей человеческой деятельности можно автоматизировать, но делать это надо очень грамотно на высоком уровне системной культуры и на фундаментальной основе общей теории человеческой деятельности.}
    \scntext{проблема}{Важная причина, препятствующая дальнейшему повышению общего уровня автоматизации общества заключается в том, что автоматизация различных областей человеческой деятельности осуществляется без выявления и глубокого анализа сходства некоторых видов деятельности в разных областях и соответственно без сближения, \textbf{\textit{конвергенции}} и \textbf{\textit{унификации}} этих \textit{видов деятельности}.}
    \scntext{примечание}{Важнейшим направлением повышения уровня автоматизации человеческой деятельности является переход к автоматизации все более и более \uline{комплексных} (крупных) видов и областей человеческой деятельности например, от автоматизации деятельности различных предприятий, организаций, хозяйственных служб к автоматизации деятельности города в целом).
    \\Автоматизации комплексных видов человеческой деятельности требует создания комплекса активно взаимодействующих компьютерных систем, каждая из которых обеспечивает автоматизацию соответствующего частного вида человеческой деятельности, входящего в состав автоматизируемого комплексного вида деятельности. При этом число уровней иерархии автоматизируемых видов человеческой деятельности ничем не ограничивается. Очевидно, что уровень автоматизации комплексных видов человеческой деятельности определяется:
    \begin{itemize}
        \item уровнем конвергенции (сближения, совместимости) соответствующих частных видов деятельности;
        \item качеством интеграции этих частных видов деятельности;
        \item уровнем конвергенции компьютерных систем, обеспечивающих автоматизацию указанных частных видов деятельности;
        \item качеством взаимодействия этих компьютерных систем то есть уровнем интероперабельности этих систем).
    \end{itemize}}
\scntext{примечание}{Уровень эволюции общества во многом зависит от уровня автоматизации человеческой деятельности, от уровня развития соответствующих технологий такой автоматизации. Но эта зависимость выглядит значительно сложнее чем, кажется на первый взгляд, особенно, если речь идет об автоматизации не физической, интеллектуальной человеческой деятельности (как индивидуальной, так и коллективной). Безграмотная, а тем более социально безответственная или злонамеренная автоматизация информационной деятельности общества способны нанести огромный ущерб его развитию. Такая безграмотность и безответственность, например, приводит к таким побочным факторам, как компьютерная зависимость, виртуализация окружающей среды, поверхностный характер мышления, снижение познавательной мотивации и активности и многое другое.}
\begin{scnindent}
	\begin{scnrelfromlist}{следовательно}
		\scnfileitem{Необходимо существенно повысить уровень социальной ответственности у разработчиков компьютерных систем и соответствующих технологий.}
		\scnfileitem{Опасность от безграмотного, социально безответственного и тем более злонамеренного внедрения интеллектуальных компьютерных систем нового поколения может иметь для человечества летальный характер.}
	\end{scnrelfromlist}
\end{scnindent}
	
\scnheader{человеческое общество}
\scntext{направления развития}{Если рассматривать \textit{общество} как \textit{многоагентную систему}, состоящую из самостоятельных интеллектуальных агентов, то, очевидно, что важнейшими факторами, определяющими повышение качества (уровня развития) \textit{общества} являются:
    \begin{scnitemize}
        \item повышение эффективности использования опыта, накопленного \textit{обществом}, эффективности использования человечеством \textit{знаний} и \textit{навыков}
        \item повышение темпов приобретения, накопления и систематизации эффективно используемым человечеством \textit{знаний} и \textit{навыков}.
    \end{scnitemize}
    Решение указанных проблем становится вполне возможным, если для этого использовать \textit{интеллектуальные компьютерные системы нового поколения}, с помощью которых накапливаемые человечеством \textit{знания} и \textit{навыки} будут организованы как систематизированная распределенная библиотека многократно используемых информационных ресурсов (\textit{знаний} и \textit{навыков}).}
\begin{scnindent}
    \scntext{следовательно}{Систематизация и автоматизация многократного использования накапливаемых человечеством информационных ресурсов требует их конвергенции, глубокой интеграции и формализации. Особое место в этом процессе занимает математика, как основа систематизации и формализации знаний и навыков на уровне формальных \textit{онтологий верхнего уровня}.}
\end{scnindent}
	
\end{scnsubstruct}

\end{scnsubstruct}

\end{scnsubstruct}

    \end{scnsubstruct}
    \begin{scnrelfromvector}{заключение}
        \scnfileitem{Благодаря тому, что \textit{интеллектуальные компьютерные системы нового поколения} становятся самостоятельными и активными субъектами \textit{человеческой деятельности} в достаточной степени равноправными людям (естественным индивидуальным субъектам человеческой деятельности), характер и, соответственно, уровень автоматизации \textit{человеческой деятельности} существенно меняется --- снимается необходимость \uline{управлять} средствами автоматизации, поскольку такое \scnqq{ручное} управление заменяется распределением обязанностей и ответственности между людьми и \textit{интеллектуальными компьютерными системами нового поколения}.}
        \scnfileitem{Если автоматизация \uline{любого} вида в любой области \textit{человеческой деятельности} будет осуществляться с помощью \textit{интеллектуальных компьютерных систем нового поколения} и если \textit{интеллектуальные компьютерные системы нового поколения}, обеспечивающие автоматизацию \uline{разных} видов и областей \textit{человеческой деятельности}, будут содержательно взаимодействовать между собой, то общий уровень автоматизации \textit{человеческой деятельности} существенно возрастет благодаря тому, что отпадет необходимость \uline{вручную} координировать использование различных средств автоматизации.}
        \scnfileitem{Эффективность и трудоемкость автоматизации различных видов и областей \textit{человеческой деятельности} будет существенно определяться степенью \textbf{\textit{конвергенции}} между различными видами и областями \textit{человеческой деятельности}. Необходимо построить иерархическую модель \textit{человеческой деятельности,} в рамках которой должна быть проведена грамотная систематизация и стратификация всех видов и областей \textit{человеческой деятельности}, направленная против излишнего эклектического многообразия. 
        	\\Таким образом, прежде, чем осуществлять комплексную автоматизации \textit{человеческой деятельности} с помощью \textit{интеллектуальных компьютерных систем} \textit{нового поколения}, необходимо с позиций общей теории систем переосмыслить организацию этой деятельности. В противном случае автоматизация беспорядка приведет к еще большему беспорядку.}
        \scnfileitem{Особо подчеркнем то, что многие из рассмотренных нами проблем текущего состояния и направлений дальнейшего развития \textit{Человеческой деятельности в области Искусственного интеллекта} аналогичны проблемам и тенденциям развития многих других научно-технических дисциплин. Следовательно, подходы к решению этих проблем могут носить междисциплинарный характер.}
        \scnfileitem{Время каждого человека является главным невосполнимым ресурсом общества и тратить его надо не на рутинную поддержку жизненного цикла всевозможных социально значимых объектов, а на комплексное развитие соответствующих \textit{технологий}. Автоматизация человеческой деятельности с помощью глобальной системы интероперабельных семантически совместимых и активно взаимодействующих \textit{интеллектуальных компьютерных систем} в самых разных областях \textit{человеческой деятельности} позволит существенно сократить время каждого человека на выполнение рутинной, легко автоматизируемой деятельности. Человеческая деятельность должна стать ориентированной на максимально возможную самореализацию, раскрытие \uline{творческого} потенциала каждого человека, направленного на ускорение темпов повышения уровня интеллекта всего общества.}
        \scnfileitem{Необходимо создать \textit{Глобальную экосистему интеллектуальных компьютерных систем нового поколения}.}
        \begin{scnindent}
            \begin{scnrelfromlist}{требование}
                \scnfileitem{Построение формальной модели \textit{человеческой деятельности}.}
                \scnfileitem{Переход от эклектичного построения сложных \textit{интеллектуальных компьютерных систем}, использующих различные виды \textit{знаний} и различные виды \textit{моделей решения задач}, к их глубокой \textit{интеграции} и \textit{унификации}, когда одинаковые модели представления и модели обработки знаний реализуется в разных системах и подсистемах одинаково.}
                \scnfileitem{Сокращение дистанции между современным уровнем \textit{теории интеллектуальных компьютерных систем} и практики их разработки.}
                \scnfileitem{Разработку грамотной тактики и стратегии переходного периода, в рамках которого современные \textit{интеллектуальные компьютерные системы} должны постепенно заменяться на \textit{интеллектуальные компьютерные системы нового поколения}, которые должны эффективно взаимодействовать не только между собой, но и с хорошо зарекомендовавшими себя современными информационными ресурсами и сервисами.}
            \end{scnrelfromlist}
        \end{scnindent}
\end{scnrelfromvector}
\bigskip
\scnendcurrentsectioncomment
\end{SCn}
