\scnsegmentheader{Анализ структуры Деятельности в области Искусственного интеллекта}
\begin{scnsubstruct}
    \scntext{аннотация}{Для того, чтобы рассмотреть проблемы дальнейшего развития \textit{деятельности} в области \textit{Искусственного интеллекта} как \textit{научно-технической дисциплины} и, в частности, проблемы комплексной автоматизации этой \textit{деятельности}, необходимо уточнить структуру указанной \textit{деятельности}.}
    
	\scnheader{Человеческая деятельность в области Искусственного интеллекта}
	\scntext{примечание}{Человеческая деятельность в области \textit{Искусственного интеллекта} направлена на исследование и создание \textit{интеллектуальных компьютерных систем} различного вида и различного назначения.}
	\begin{scnhaselementrolelist}{объект исследования}
		\scnitem{индивидуальные интеллектуальные компьютерные системы}
		\begin{scnindent}
			\scnhaselement{когнитивные агенты}
		\end{scnindent}
		\scnitem{многоагентные интеллектуальные компьютерные системы}
		\begin{scnindent}
			\scnhaselement{сообщества, состоящие из индивидуальных интеллектуальных компьютерных систем}
		\end{scnindent}
		\scnitem{человеко-машинные сообщества, состоящие из интеллектуальных компьютерных систем и их пользователей}
	\end{scnhaselementrolelist}

	\scnheader{Искусственный интеллект}
    \scniselement{область человеческой деятельности}
    \scniselement{научно-техническая дисциплина}
    \begin{scnindent}
    	\scniselement{вид человеческой деятельности}
    \end{scnindent}
    \begin{scnrelfromlist}{цель}
		\scnfileitem{построение теории интеллектуальных систем}
		\scnfileitem{создание технологии разработки интеллектуальных компьютерных систем (искусственных интеллектуальных систем)}
		\scnfileitem{переход на принципиально новый уровень комплексной автоматизации всех \textit{видов человеческой деятельности}, который основан на массовом применение \textit{интеллектуальных компьютерных систем}}
		\begin{scnindent}
			\begin{scnrelfromlist}{детализация}
				\scnfileitem{наличие \textit{интеллектуальных компьютерных систем}, способных понимать друг друга и согласовывать свою деятельность}
				\scnfileitem{построение \textit{Общей теории человеческой деятельности}, осуществляемый в условиях нового уровня её автоматизации, (теории деятельности \textit{smart-общества}), которая должна быть понятна \textit{используемым интеллектуальным компьютерным системам} и которая потребует существенного переосмысления современной организации \textit{человеческой деятельности}}
			\end{scnrelfromlist}
		\end{scnindent}
	\end{scnrelfromlist}
	\begin{scnrelfromlist}{определение}
		\scnfileitem{Научно-техническая деятельность, направленная на построение теории интеллектуальных систем, а также на создание технологии проектирования и производства искусственных интеллектуальных систем (\textit{интеллектуальных компьютерных систем}).}
		\scnfileitem{Научно-техническая деятельность в области \textit{Искусственного интеллекта}.}
		\scnfileitem{Деятельность в области \textit{Искусственного интеллекта}}
		\scnfileitem{Научно-техническая деятельность, направленная на исследование феномена \textit{интеллекта}, а также на создание искусственных интеллектуальных систем (\textit{интеллектуальных компьютерных систем}) и включающая в себя соответствующую научно-исследовательскую деятельность, инженерно-технологическую, инженерно-прикладную, образовательную и организационную деятельность.}
		\scnfileitem{Междисциплинарная (трансдисциплинарная) область \textit{научно-технической деятельности}, направленная на разработку и эксплуатацию \textit{интеллектуальных компьютерных систем}, обеспечивающих автоматизацию различных сфер \textit{человеческой деятельности}.}
		\scnfileitem{Научно-техническая дисциплина направленная на разработку теории индивидуальных \textit{интеллектуальных компьютерных систем} и \textit{интеллектуальных сообществ} (коллективов) таких систем, а также средств поддержки их проектирования и реализации).}
		\scnfileitem{\textit{Научно-техническая дисциплина}, являющаяся частью \textit{кибернетики} (теория кибернетических систем), объектом исследования которой являются \textit{интеллектуальные компьютерные системы} (искусственные \textit{интеллектуальные системы}) и целями которой являются (1) разработка \textit{теории интеллектуальных компьютерных систем}, (2) разработка \textit{технологии(методов и средств)} \textit{проектирования и производства компьютерных систем}, а также, (3) разработка конкретных интеллектуальных компьютерных систем различного назначения.}
    \end{scnrelfromlist}
	\scnrelfrom{декомпозиция}{Декомпозиция Искусственного интеллекта по формам деятельности}
    \begin{scnindent}
		\begin{scneqtoset}
			\scnitem{Научно-исследовательская деятельность в области Искусственного интеллекта}
			\begin{scnindent}
				\scnrelfrom{продукт}{Общая теория интеллектуальных систем}
				\begin{scnindent}
					\scnidtf{Теория, уточняющая структуру и принципы функционирования \textit{интеллектуальных систем}, а также акцентирующая внимание на причинах (предпосылках) возникновение свойства интеллектуальности (феномены \textit{интеллекта})}
				\end{scnindent}
				\scniselement{коллективная научно-техническая деятельность}
				\begin{scnindent}
					\scniselement{вид человеческой деятельности}
					\scnidtf{научно-исследовательская дисциплина или направление}
					\scnrelboth{следует отличать}{продукт научно-исследовательской деятельности}
				\end{scnindent}
				\scntext{примечание}{В процессе \textit{Научно-исследовательской деятельности в области Искусственного интеллекта} осуществляется конкуренция различных точек зрения и подходов к построению формальных моделей различных компонентов \textit{интеллектуальных компьютерных систем}. Конечной целью такой деятельности является постоянно развиваемая \textit{Общая теория} \textit{интеллектуальных} \textit{компьютерных систем}, объектами исследования которой являются \textit{интеллектуальные компьютерные системы} и их формальные \textit{логико-семантические модели}, включающие в себя формальные модели различного вида \textit{знаний}, входящих в состав \textit{баз знаний} интеллектуальных компьютерных систем, а также различные \textit{модели решения задач} (логические модели различного вида, нейросетевые, генетические, продукционные, функциональные и другие).}
			\end{scnindent}
			\scnitem{Разработка Стандарта интеллектуальных компьютерных систем}
			\begin{scnindent}
				\scntext{примечание}{\textit{Разработка Стандарта интеллектуальных компьютерных систем} включает в себя перманентную эволюцию этого стандарта и поддержку целостности каждой его версии. Текущая версия \textit{Стандарта интеллектуальных компьютерных систем} --- это \uline{согласованная} (общепризнанная) \uline{на текущий момент} часть \textit{Общей теории интеллектуальных компьютерных систем.}}
			\end{scnindent}
			\scnitem{Разработка Базовой комплексной технологии проектирования интеллектуальных компьютерных систем}
			\begin{scnindent}
				\scnrelfrom{продукт}{Базовая комплексная технология проектирования интеллектуальных компьютерных систем}
				\begin{scnindent}
					\scntext{примечание}{Комплексность данной \textit{технологии} заключается в том, что она ориентирована (1) на проектирование \textit{интеллектуальных компьютерных систем} в целом, а не только отдельных их компонентов и (2) на создание объединённой \textit{технологии}, объединяющей самые различные технологические подходы на основе их \textit{конвергенции} и глубокой \textit{интеграции}}
				\end{scnindent}
				\scniselement{разработка технологии проектирования искусственных объектов заданного класса}
				\begin{scnindent}
					\scniselement{вид человеческой деятельности}
				\end{scnindent}
				\scntext{примечание}{\textit{Разработка технологии проектирования интеллектуальных компьютерных систем} включает в себя семейство методик проектирования, а также методов и средств автоматизации \textit{проектирования} различных \textit{компонентов} \textit{интеллектуальных компьютерных систем} и \textit{интеллектуальных компьютерных систем} в целом. Результатом проектирования \textit{интеллектуальных компьютерных систем} является полная формальная логико-семантическая модель этой системы.}
			\end{scnindent}
			\scnitem{Разработка Технологии производства спроектированных интеллектуальных компьютерных систем}
			\begin{scnindent}
				\scnhaselement{Разработка технологии реализации спроектированных интеллектуальных компьютерных систем}
				\scnhaselement{Разработка технологий эксплуатации и сопровождения интеллектуальных компьютерных систем}
				\scnrelfrom{продукт}{Технология производства спроектированных интеллектуальных компьютерных систем}
				\begin{scnindent}
					\scnidtf{технология реализации (сборки и установки) спроектированных интеллектуальных компьютерных систем}
					\scntext{примечание}{Очевидно, что данная \textit{технология} должна быть самым тесным образом связана с \textit{Базовой комплексной технологией проектирования интеллектуальных компьютерных систем} (по крайней мере данная \textit{технология} должна знать в какой форме ей на вход передаётся результат проектирования). Поэтому имеет смысл говорить об объединённой технологии проектирования и производства \textit{интеллектуальных компьютерных систем}}
				\end{scnindent}
				\scniselement{производства спроектированных искусственных объектов заданного класса}
				\begin{scnindent}
					\scniselement{вид человеческой деятельности}
				\end{scnindent}
				\scntext{примечание}{В основе технологии реализации (производства) спроектированных \textit{интеллектуальных компьютерных систем} лежит \textit{универсальный интерпретатор формальных логико-семантических моделей интеллектуальных компьютерных систем}, являющихся результатом проектирования указанных систем. Указанный универсальный интерпретатор может быть реализован либо в виде \textit{программной системы} на современных компьютерах, либо в виде \textit{универсального компьютера нового поколения}, ориентированного на интерпретацию формальных \textit{логико-семантических моделей интеллектуальных компьютерных систем}.}
			\end{scnindent}
			\scnitem{Специализированная инженерия в области Искусственного интеллекта}
			\begin{scnindent}
				\scnidtf{Прикладная инженерная деятельность в области Искусственного интеллекта}
				\scnidtf{Множество Процессов разработки (проектирования и производства) \textit{интеллектуальных компьютерных систем} различного назначения, кроме \textit{интеллектуальных компьютерных систем автоматизации проектирования} и автоматизации производства \textit{интеллектуальных компьютерных систем}}
				\scnrelfrom{продукт}{множество специализированных интеллектуальных компьютерных систем}
				\begin{scnindent}
					\scnrelfrom{основной sc-идентификатор}{прикладная интеллектуальная компьютерная система}
				\end{scnindent}
				\scnsubset{проектирование и производство искусственного объекта заданного класса на основе заданной технологии}
				\begin{scnindent}
					\scniselement{вид человеческой деятельности}
				\end{scnindent}
				\scntext{примечание}{Прикладная инженерная деятельность в области Искусственного интеллекта, то есть непосредственное проектирование, реализация и сопровождение включает в себя обновление (реинжиниринг), осуществляемое в ходе эксплуатации, конкретных \textit{интеллектуальных компьютерных систем.}}
			\end{scnindent}
			\scnitem{Образовательная деятельность в области Искусственного интеллекта}
			\begin{scnindent}
				\scnidtf{Учебная деятельность в области Искусственного интеллекта}
				\scnidtf{Деятельность, направленная на подготовку молодых специалистов области \textit{Искусственного интеллекта} на перманентное повышение квалификации действующих специалистов в этой области}
				\scntext{примечание}{Сложность и высокая степень наукоемкости задач, больших своего решения на текущем этапе развития \textit{Искусственного интеллекта}, добавляют к специалистам, работающим в этой области высокие требования к уровню их:
					\begin{itemize}
						\item математической культуры (культуры формализации),
						\item системной культуры,
						\item технологической культуры,
						\item инженерная культура,
						\item умения работать в коллективных наукоемких проектах.
					\end{itemize}}
				\scnsubset{образовательная деятельность}
				\begin{scnindent}
					\scniselement{вид человеческой деятельности}
				\end{scnindent}
				\scntext{примечание}{\textit{Учебная деятельность в области Искусственного интеллекта} направлена на подготовку специалистов области \textit{Искусственного интеллекта} и на перманентное повышение квалификации действующих специалистов в этой области. Без эффективной организации учебной деятельности в области \textit{Искусственного интеллекта} быстрый прогресс в этой области невозможен. Непосредственное включение учебной деятельности в общую структуру человеческой деятельности в области \textit{Искусственного интеллекта} обусловлено следующими обстоятельствами:
					\begin{itemize}
						\item необходимостью глубокой \textit{конвергенции} между различными направлениями и видами деятельности в области \textit{Искусственного интеллекта} и соответствующей спецификой требований, предъявляемых к специалистам в этой области --- каждый такой специалист должен быть достаточно компетентен и в научно-исследовательской деятельности в области \textit{Искусственного интеллекта}, и в разработке технологий (методик и средств) \textit{проектирования интеллектуальных компьютерных систем}, и в разработке технологий \textit{воспроизводства} (реализации) спроектированных \textit{интеллектуальных компьютерных систем}, а также технологий их \textit{эксплуатации} и \textit{сопровождения}, и в прикладной \textit{инженерной деятельности в области} \textit{Искусственного интеллекта};
						\item высокими темпами эволюции результатов в области \textit{Искусственного интеллекта}, что делает необходимой организацию обучения соответствующих специалистов путем их непосредственного подключения не к учебным (упрощенным) проектам, а к реальным проектам, реализуемым в текущий момент. Иначе подготовленные специалисты будут иметь квалификацию \scnqq{вчерашнего дня};
						\item существенным расширением объемов работ в области \textit{Искусственного интеллекта} и острой необходимостью массовой подготовки соответствующих специалистов.
					\end{itemize}}
			\end{scnindent}
			\scnitem{Бизнес-деятельность в области искусственного интеллекта}
			\begin{scnindent}
				\scntext{пояснение}{Речь идет о бизнес-деятельности в широком смысле как деятельности, направленный на создание инфраструктурных условий для качественного выполнения всех \textit{видов деятельности} в области \textit{Искусственного интеллекта}}
				\begin{scnindent}
					\begin{scnrelfromlist}{пример}
						\scnfileitem{разработка и реализация грамотной научно-технической политики, связывающие как тактические, так и стратегические цели}
						\scnfileitem{глубокая \textit{конвергенцию} всех форм и \textit{видов деятельности} в области \textit{Искусственного интеллекта}}
						\scnfileitem{организация взаимовыгодного сотрудничество различных школ и коллективов, работающих в области \textit{Искусственного интеллекта}}
						\scnfileitem{финансовое обеспечение}
						\scnfileitem{кадровое обеспечение}
						\scnfileitem{материально-техническое обеспечение}
						\scnfileitem{организация проведения различных мероприятий (конференций, выставок, семинаров)}
						\scnfileitem{публикационная деятельность и защита интеллектуальной собственности}
						\scnfileitem{материально-техническое обеспечение}
					\end{scnrelfromlist}
				\end{scnindent}
				\scnsubset{бизнес-деятельность научно-технической области}
				\begin{scnindent}
					\scniselement{вид человеческой деятельности}
				\end{scnindent}
			\end{scnindent}
			\scnitem{Организационная деятельность в области Искусственного интеллекта}
			\begin{scnindent}
				\scntext{примечание}{\textit{Организационная деятельность в области Искусственного интеллекта} направлена на создание инфраструктуры для качественного выполнения всех остальных видов деятельности в области \textit{Искусственного интеллекта}, а именно:
				\begin{itemize}
					\item для обеспечения глубокой \textit{конвергенции} между различными направлениями и видами деятельности в области \textit{Искусственного интеллекта} и, в частности, между теорией, технологиями и инженерной практикой в этой области;
					\item для обеспечения баланса между тактикой и стратегией в развитии деятельности в области \textit{Искусственного интеллекта} как ключевой основы существенного повышения уровня автоматизации всевозможных видов \textit{человеческой деятельности} и перехода к \textit{smart-обществу}.
				\end{itemize}}
			\end{scnindent}
		\end{scneqtoset}
    \end{scnindent}
    \scntext{оценка}{Декомпозиция \textit{человеческой} \textit{деятельности} в области \textit{Искусственного интеллекта} по \textit{видам} \textit{деятельности} не является традиционным признаком декомпозиции \textit{научно-технических дисциплин}. Обычно декомпозиция \textit{научно-технических дисциплин} осуществляется по содержательным направлениям, которые соответствуют декомпозиции \textit{технических систем}, исследуемых и разрабатываемых в рамках этих \textit{научно-технических дисциплин}, то есть соответствуют выделению в этих \textit{технических системах} различного вида компонентов.}
    
    \scnheader{Искусственный интеллект}
    \scnrelfrom{разбиение}{Декомпозиция Искусственного интеллекта по видам деятельности}
    \begin{scnindent}
	    \begin{scneqtoset} 
	    	\scnfileitem{исследование и разработка формальных моделей и языков представления знаний}
	    	\scnfileitem{исследование и разработка баз знаний}
	    	\scnfileitem{исследование и разработка логических моделей обработки знаний}
	   		\scnfileitem{исследование и разработка искусственных нейронных сетей}
	   		\scnfileitem{исследование и разработка подсистем компьютерного зрения}
	   		\scnfileitem{исследование и разработка подсистем обработки естественно-языковых текстов (синтаксический анализ, понимание, синтез)}
	    \end{scneqtoset}
    \end{scnindent}
    \scntext{примечание}{Важность декомпозиции \textit{Искусственного интеллекта} по \textit{видам} \textit{деятельности} определяется тем, что выделение различных \textit{видов деятельности} позволяет четко ставить задачу на разработку средств автоматизации этих \textit{видов деятельности}.}
    
   \scnheader{Cтруктура Человеческой деятельности в области Искусственного интеллекта}
   \begin{scnstruct}
    	\scnheader{Человеческая деятельность в области Искусственного интеллекта}
    	\scnidtf{Искусственный интеллект (как научно-техническая дисциплина)}
    	\scniselement{научно-техническая дисциплина}
    	\scnidtf{Искусственный интеллект (как научно-техническая дисциплина)}
    	\scnidtf{Человеческая деятельность в Предметной области интеллектуальных компьютерных систем}
    	\scniselement{деятельность}
    	\begin{scnrelfromset}{декомпозиция}
    		\scnitem{Интегральная деятельность по поддержке жизненного цикла всевозможных интеллектуальных компьютерных систем}
    		\begin{scnindent}
    			\scnrelfrom{декомпозиция}{поддержка жизненного цикла интеллектуальных компьютерных систем}
    			\begin{scnindent}
    				\scniselement{вид деятельности}
    				\scnsuperset{поддержка жизненного цикла ostis-систем}
    				\begin{scnrelfromset}{обобщенная декомпозиция}
    					\scnitem{проектирование интеллектуальных компьютерных систем}
    					\scnitem{производство интеллектуальных компьютерных систем}
    					\scnitem{начальное обучение интеллектуальных компьютерных систем}
    					\scnitem{мониторинг качества интеллектуальных компьютерных систем}
    					\scnitem{восстановление требуемого уровня качества интеллектуальных компьютерных систем}
    					\scnitem{реинжиниринг интеллектуальных компьютерных систем}
    					\scnitem{обеспечение безопасности интеллектуальных компьютерных систем}
    					\scnitem{эксплуатация интеллектуальных компьютерных систем конечными пользователями}
    				\end{scnrelfromset}
    			\end{scnindent}
    		\end{scnindent}
    		\scnitem{Поддержка жизненного цикла Общей теории интеллектуальных компьютерных систем}
    		\begin{scnindent}
    			\scniselement{научно-исследовательская деятельность}
    		\end{scnindent}        
    		\scnitem{Поддержка жизненного цикла Стандарта интеллектуальных компьютерных систем}
    		\begin{scnindent}
    			\scniselement{стандартизация}
    			\scnrelfrom{часть}{Поддержка жизненного цикла Стандарта ostis-систем}
    		\end{scnindent}        
    		\scnitem{Поддержка жизненного цикла Технологии комплексной поддержки жизненного цикла интеллектуальных компьютерных систем}
    		\begin{scnindent}
    			\scniselement{поддержка жизненного цикла технологий}
    			\begin{scnindent}
    				\scnidtf{создание и сопровождение технологий}
    			\end{scnindent}
    			\scnrelfrom{часть}{Поддержка жизненного цикла Технологии OSTIS}
    		\end{scnindent}
    		\scnitem{Поддержка жизненного цикла кадровых ресурсов для Человеческой деятельности в области Искусственного интеллекта}
    		\scnitem{Поддержка жизненного цикла системы комплексной организации взаимодействия между всеми направлениями Человеческой деятельности в области Искусственного интеллекта}
    		\begin{scnindent}
    			\scniselement{поддержка жизненного цикла метасистем комплексного управления поддержкой и обеспечением поддержки жизненного цикла сущностей соответствующего класса}
    		\end{scnindent}        
    	\end{scnrelfromset}
    \end{scnstruct}
    
    \scnheader{Человеческая деятельность в области Искусственного интеллекта}
    \begin{scnrelfromset}{практический результат}
    	\scnfileitem{Реорганизация и комплексная автоматизация \textit{человеческой деятельности в области Искусственного интеллекта} с помощью \textit{интеллектуальных компьютерных систем нового поколения}.}
    	\scnfileitem{\uline{Поэтапное} создание глобальной сети эффективно взаимодействующих \textbf{\textit{интеллектуальных компьютерных систем нового поколения}}, обеспечивающих \uline{комплексную} автоматизацию всевозможных видов и областей \textit{человеческой деятельности}.}
    \end{scnrelfromset}
    	
    \scnheader{Технология поддержки жизненного цикла интеллектуальных компьютерных систем}
    \scnrelfrom{вид деятельности}{поддержка жизненного цикла интеллектуальных компьютерных систем}
    \begin{scnrelfromset}{декомпозиция}
    	\scnitem{Технология проектирования интеллектуальных компьютерных систем}
    	\begin{scnindent}
    		\scnrelfrom{вид деятельности}{проектирование интеллектуальных компьютерных систем}
    	\end{scnindent}
    	\scnitem{Технология производства интеллектуальных компьютерных систем}
    	\begin{scnindent}
    		\scnrelfrom{вид деятельности}{производство интеллектуальных компьютерных систем}
    	\end{scnindent}
    	\scnitem{Технология начального обучения интеллектуальных компьютерных систем (адаптации к конкретной деятельности)}
    	\begin{scnindent}
    		\scnrelfrom{вид деятельности}{начальное обучение интеллектуальных компьютерных систем}
    	\end{scnindent}
    	\scnitem{Технология мониторинга качества интеллектуальных компьютерных систем}
    	\begin{scnindent}
    		\scnrelfrom{вид деятельности}{мониторинг качества интеллектуальных компьютерных систем}
    		\begin{scnindent}
    			\scnidtf{плановое тестирование и диагностика интеллектуальных компьютерных систем}
    		\end{scnindent}
    	\end{scnindent}
    	\scnitem{Технология восстановления требуемого уровня качества интеллектуальных компьютерных систем в ходе их эксплуатации}
    	\begin{scnindent}
    		\scnidtf{Технология выявления и исправления потенциально опасных ситуаций и событий в деятельности интеллектуальных компьютерных систем (ошибок, противоречий, и так далее)}
    		\scnrelfrom{вид деятельности}{восстановление требуемого уровня качества интеллектуальных компьютерных систем}
    	\end{scnindent}
    	\scnitem{Технология реинжиниринга  интеллектуальных компьютерных систем}
    	\begin{scnindent}
    		\scnidtf{Технология совершенствования, модернизации, обновления интеллектуальных компьютерных систем}
    		\scnrelfrom{вид деятельности}{реинжиниринг интеллектуальных компьютерных систем}
    	\end{scnindent}
    	\scnitem{Технология обеспечения безопасности интеллектуальных компьютерных систем}
    	\begin{scnindent}
    		\scnrelfrom{вид деятельности}{обеспечение безопасности интеллектуальных компьютерных систем}
    	\end{scnindent}
    	\scnitem{Технология эксплуатации интеллектуальных компьютерных систем конечными пользователями}
    	\begin{scnindent}
    		\scnrelfrom{вид деятельности}{эксплуатация интеллектуальных компьютерных систем конечными пользователями}
    	\end{scnindent}
    \end{scnrelfromset}
    
    \scnheader{Разработка базовой комплексной технологии проектирования интеллектуальных компьютерных систем}
    \begin{scnrelfromset}{декомпозиция}
        \scnitem{Разработка общей теории интеллектуальных компьютерных систем}
		\begin{scnindent}
			\scnrelfrom{продукт}{Общая теория интеллектуальных компьютерных систем}
			\scniselement{разработка теории искусственных объектов заданного класса}
			\begin{scnindent}
				\scniselement{вид человеческой деятельности}
			\end{scnindent}
		\end{scnindent}
        \scnitem{Разработка общей теории интеллектуальных компьютерных систем}
		\begin{scnindent}
			\scnrelfrom{продукт}{Теория проектирования интеллектуальных компьютерных систем}
			\begin{scnindent}
				\scntext{примечание}{В состав этой теории входят методы проектирования, библиотеки проектирования и спецификация используемых индустриальных средств.}
			\end{scnindent}
			\scnidtf{Разработка \textit{Теории проектной деятельности} по построению формальных моделей \textit{интеллектуальных компьютерных систем}}
			\scniselement{теории проектирования интеллектуальных объектов заданного класса}
			\begin{scnindent}
				\scniselement{вид человеческой деятельности}
			\end{scnindent}
		\end{scnindent}
        \scnitem{Разработка комплекса средств автоматизации проектирования интеллектуальных компьютерных систем}
		\begin{scnindent}
			\scniselement{разработка комплекса средств автоматизации проектирования искусственных объектов заданного класса}
			\begin{scnindent}
				\scniselement{вид человеческой деятельности}
			\end{scnindent}
		\end{scnindent}
    \end{scnrelfromset}
    
    \scnheader{Разработка Технологии производства спроектированных интеллектуальных компьютерных систем}
    \begin{scnrelfromset}{декомпозиция}
        \scnitem{Разработка Теории производства спроектированных интеллектуальных компьютерных систем}
		\begin{scnindent}
			\scnidtf{Разработка \textit{Теории производственной деятельности} по реализации (сборке и установке) \textit{интеллектуальных компьютерных систем}}
			\scnrelfrom{продукт}{Теория производства спроектированных интеллектуальных компьютерных систем}
			\begin{scnindent}
				\scntext{примечание}{В состав этой \textit{теории} входят \textit{методы производства} (сборки и установки) \textit{интеллектуальных компьютерных систем}, а также спецификация \textit{используемых инструментальных средств}.}
			\end{scnindent}
		\end{scnindent}
        \scnitem{Разработка комплекса средств автоматизации производства интеллектуальных компьютерных систем}
		\begin{scnindent}
			\scniselement{разработка комплекса средств автоматизации производства искусственных объектов заданного класса}
			\begin{scnindent}
				\scniselement{вид человеческой деятельности}
			\end{scnindent}
		\end{scnindent}
    \end{scnrelfromset}
    
    \scnheader{Специализированная инженерия в области Искусственного интеллекта}
    \scnidtf{проектирование и производство конкретной \textit{интеллектуальной компьютерной системы} по заданной технологии}
    \begin{scnrelfromset}{обобщенная декомпозиция}
        \scnitem{проектирование конкретных интеллектуальных компьютерные системы}
		\begin{scnindent}
			\begin{scnrelfromset}{обобщенное разбиение}
				\scnitem{проектирование интеллектуальной компьютерной системы автоматизации проектирования соответствующего класса интеллектуальных компьютерных систем}
				\scnitem{проектирование интеллектуальной компьютерной системы автоматизации проектирования соответствующего класса объектов, не являющихся интеллектуальными компьютерными системами}
				\scnitem{проектирование интеллектуальной компьютерной системы,не являющейся системой автоматизации проектирования}
			\end{scnrelfromset}
		\end{scnindent}
        \scnitem{производство конкретной спроектированной интеллектуальной компьютерной системы}
    \end{scnrelfromset}
    
    \scnheader{производство конкретной спроектированной интеллектуальной компьютерной системы}
    \scntext{примечание}{Производственная деятельность, направленная на \textit{производство} (реализацию) спроектированной \textit{интеллектуальной компьютерной системы} значительно уступает по уровню сложности деятельности по проектированию этой \textit{интеллектуальной компьютерной системы}, так как это производство сводится к сборке результата \textit{проектирования} (формальной логико-семантической модели разрабатываемой \textit{интеллектуальной компьютерной системы}) и загрузки этой модели в память компьютера или программного \textit{универсального интерпретатора логико-семантических моделей интеллектуальных компьютерных систем}, в качестве которого может быть использован:
        \begin{itemize}
            \item либо специально разработанный для этого \textit{компьютер}, ориентированные на обработку \textit{баз знаний} и интерпретацию различных \textit{интеллектуальных моделей решения задач},
            \item либо программная эмуляция такого \textit{компьютера}, реализованная на современных \textit{компьютерах} фон-неймановской архитектуры.
        \end{itemize}
        Простота производства спроектированных систем характерна для производства не только интеллектуальных, но и любых других \textit{компьютерных систем}.
        \\Мы выделяем производственный этап реализации \textit{интеллектуальных компьютерных систем} для того, чтобы по аналогии рассматривать этап производства (массового, мелкосерийного, разового производства) спроектированных искусственных объектов любого другого вида(микросхем, автомобилей, зданий, компьютеров).
        \\Очевидно, что массовое производство некоторых видов продукции может иметь весьма большой уровень сложности, но при этом суть \textit{производственной деятельности} как процесса перехода от проекта (спецификации) некоторого объекта к его реализации остаётся одной и той же независимо от уровня сложности реализуемого объекта (производимой продукции).}
        
    \scnheader{следует отличать*}
    \begin{scnhaselementset}
        \scnitem{специализированная интеллектуальная компьютерная система}
        \scnitem{интеллектуальная компьютерная система автоматизации проектирования интеллектуальных компьютерных систем}
        \scnitem{интеллектуальная компьютерная система автоматизации производства спроектированных интеллектуальных компьютерных систем}
    \end{scnhaselementset}
    \begin{scnhaselementset}
        \scnitem{человеческая деятельность}
		\begin{scnindent}
			\scnsuperset{научно-исследовательская деятельность}
			\scnsuperset{научно-техническая деятельность}
		\end{scnindent}
        \scnitem{продукт человеческой деятельности}
		\begin{scnindent}
			\scnsuperset{продукт научно-исследовательской деятельности}
			\scnsuperset{продукты научно-технической деятельности стакан}
		\end{scnindent}
    \end{scnhaselementset}
    
    \scnheader{Искусственный интеллект}
    \scnrelfrom{декомпозиция}{Декомпозиция Искусственного интеллекта по направлениям}
    \begin{scnindent}
	    \begin{scneqtoset}
	        \scnitem{Разработка теории представления знаний и технологии проектирования баз знаний актуальных компьютерных систем}
	        \scnitem{Разработка теории решения задач и технологии проектирования решателей задач интеллектуальных компьютерных систем}
			\begin{scnindent}
				\begin{scnrelfromset}{декомпозиция}
					\scnitem{Разработка теории решения интерфейсных задач и технологии проектирования соответствующих решателей}
					\begin{scnindent}
						\scnrelfrom{часть}{Разработка теории естественно языковых интерфейсов интеллектуальных компьютерных систем и технологии их проектирования}
					\end{scnindent}
					\scnitem{Разработка теории решения информационных задач в базах знаний интеллектуальных компьютерных систем и технологии проектирования соответствующих решателей}
					\scnitem{Разработка теории решения поведенческих задач во внешней среде интеллектуальных компьютерных систем и технологии проектирования соответствующих решателей}
				\end{scnrelfromset}
				\begin{scnrelfromset}{декомпозиция}
					\scnitem{Разработка логических моделей решения задач и технологии проектирования соответствующих решателей}
					\scnitem{Разработка нейросетевых моделей решение задач и технологии проектирования соответствующих решателей}
				\end{scnrelfromset}
			\end{scnindent}
	        \scnitem{Разработка универсальных интерпретаторов базовых моделей обработки баз знаний интеллектуальных компьютерных систем}
	        \scnitem{Разработка общей теории человеческой деятельности, автоматизируемой с помощью комплекса взаимодействующих интеллектуальных компьютерных систем}
	    \end{scneqtoset}
    \end{scnindent}
    \scntext{примечание}{Переход от современных интеллектуальных компьютерных систем к \textit{интеллектуальным компьютерным системам нового поколения} и к соответствующей комплексной технологии не требует от специалистов в области Искусственного интеллекта изменения сферы их научных интересов. От них требуется только преодолеть синдром \scnqqi{Вавилонского столпотворения}, оформляя свои научные результаты как часть общего коллективного продукта.}
    \scntext{примечание}{Проблемы текущего этапа развития \textit{Искусственного интеллекта}, направленного на создание Общей теории и технологии \textit{интеллектуальных компьютерных систем нового поколения}, требуют \uline{фундаментального} комплексного междисциплинарного подхода и принципиально новой организации научно-технической деятельности.}  

\bigskip
\end{scnsubstruct}
\scnendcurrentsectioncomment
